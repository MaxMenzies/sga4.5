% !TEX root = sga4.5.tex

\chapter{Dualité}\label{V}




















On trouvera dans cet expos\'e quelques th\'eor\`emes et compatibilit\'es, tous 
relatifs \`a la dualit\'e de Poincar\'e. Au paragraphe 1, le th\'eor\`eme de 
bidualit\'e locale en dimension $1$ \cite[I.5.1]{sga5} et quelques calculus de 
deaux. Au paragraphe 2, une d\'emonstration tr\`es \'economique de la dualit\'e 
de Poincar\'e sur les courbes, que m'a apprise M.\ Artin. Au paragraphe 3, une 
compatibilit\'e qui fait le lien entre deux d\'efinitions de l'accouplement qui 
donne lieu \`a la dualit\'e de Poincar\'e pour les courbes: par cup-produit, ou 
par autodualit\'e de la jacobienne. Au paragraphe 4, enfin, la preuve de la 
compatibilit\'e du titre. 

Dans tout l'expos\'e, les sch\'emas seront noeth\'eriens et s\'epar\'es, et $n$ 
est un entier inversible sur tous les sch\'emas consid\'er\'es. 










\section{Bidualit\'e locale, en dimension \texorpdfstring{$1$}{1}}\label{V:1}





\subsection{}\label{V:1-1}

Soit $S$ un sch\'ema r\'egulier purement de dimension $1$. Nous nous proposons 
de montrer que le complexe r\'eduit \`a $\dZ/n$ en degr\'e $0$ est 
dualisant, i.e. que pour $\cK\in \ob\D_c^b(S,\dZ/n)$ ($_c$ pour constructible), 
si on pose $D\cK=\R\underline\hom(\cK,\dZ/n)$, alors $D\cK$ est encore 
constructible \`a cohomologie born\'ee, et que le morphisme canonique $\alpha$ 
de $\cK$ dans $D D\cK$ est un isomorphisme. 

Pour $\cK$ dans $\D^-$, et $\cL$ quelconque, on a 
\[
  \hom(\cK\lotimes \cL,\dZ/n) = \hom(\cL,\underline\hom(\cK,\dZ/n)) \text{.}
\]
Le morphisme de $\cK$ dans $DD\cK$ est d\'efini en supposant $D\cK$ dans 
$\D^-$; si $\beta:D\cK\lotimes\cK\to \dZ/n$ est l'accouplement canonique, il 
est d\'efini par l'accouplement 
$\cK\lotimes D\cK=D\cK\lotimes \cK \xrightarrow\beta \dZ/n$. 





\subsection{}\label{V:1-2}

Soit $f:X\to S$ un morphisme s\'epar\'e de type fini. Posons 
$\cK_X=\R f^!\dZ/n$. Pour $\cK\in \ob\D^-(X,\dZ/n)$, on pose 
$D\cK = \R\underline\hom(\cK,\cK_X)$. L'adjonction entre $\R f_!$ et $\R f^!$ 
assure que 
\[\xymatrix{
  \R f_\ast \cK\lotimes \R f_\ast D\cK \ar[r] 
    & \R f_! (\cK\lotimes D\cK) \ar[r] 
    & \R f_! \R f^! \dZ/n \ar[r] 
    & \dZ/n \text{.}
}\]
La sym\'etrie de cette description montre que, pour $f$ propre, le diagramme 
\begin{equation}\label{V:eq:1-2-1}
\xymatrix{
  \R f_\ast \cK \ar[r] \ar[d] 
    & D D \R f_\ast \cK \ar[d]^-\sim \\
  \R f_\ast D D \cK \ar[r]^\sim 
    & D\R f_\ast D\cK
}
\end{equation}
est commutatif. 


\section{La dualité de Poincaré pour les courbes}\label{V:2}
