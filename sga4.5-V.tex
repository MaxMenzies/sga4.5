% !TEX root = sga4.5.tex

\chapter{Dualité}\label{V}




















On trouvera dans cet expos\'e quelques th\'eor\`emes et compatibilit\'es, tous 
relatifs \`a la dualit\'e de Poincar\'e. Au paragraphe 1, le th\'eor\`eme de 
bidualit\'e locale en dimension $1$ \cite[I.5.1]{sga5} et quelques calculus de 
deaux. Au paragraphe 2, une d\'emonstration tr\`es \'economique de la dualit\'e 
de Poincar\'e sur les courbes, que m'a apprise M.\ Artin. Au paragraphe 3, une 
compatibilit\'e qui fait le lien entre deux d\'efinitions de l'accouplement qui 
donne lieu \`a la dualit\'e de Poincar\'e pour les courbes: par cup-produit, ou 
par autodualit\'e de la jacobienne. Au paragraphe 4, enfin, la preuve de la 
compatibilit\'e du titre. 

Dans tout l'expos\'e, les sch\'emas seront noeth\'eriens et s\'epar\'es, et $n$ 
est un entier inversible sur tous les sch\'emas consid\'er\'es. 










\section{Bidualit\'e locale, en dimension \texorpdfstring{$1$}{1}}\label{V:1}





\subsection{}\label{V:1-1}

Soit $S$ un sch\'ema r\'egulier purement de dimension $1$. Nous nous proposons 
de montrer que le complexe r\'eduit \`a $\dZ/n$ en degr\'e $0$ est 
dualisant, i.e. que pour $\sK\in \ob\D_c^b(S,\dZ/n)$ ($(-)_c$ pour 
constructible), si on pose $D\sK=\rHom(\sK,\dZ/n)$, alors $D\sK$ est encore 
constructible \`a cohomologie born\'ee, et que le morphisme canonique $\alpha$ 
de $\sK$ dans $D D\sK$ est un isomorphisme. 

Pour $\sK$ dans $\D^-$, et $\sL$ quelconque, on a 
\[
  \hom(\sK\lotimes \sL,\dZ/n) = \hom(\sL,\Hom(\sK,\dZ/n)) \text{.}
\]
Le morphisme de $\sK$ dans $DD\sK$ est d\'efini en supposant $D\sK$ dans 
$\D^-$; si $\beta:D\sK\lotimes\sK\to \dZ/n$ est l'accouplement canonique, il 
est d\'efini par l'accouplement 
$\sK\lotimes D\sK=D\sK\lotimes \sK \xrightarrow\beta \dZ/n$. 





\subsection{}\label{V:1-2}

Soit $f:X\to S$ un morphisme s\'epar\'e de type fini. Posons 
$\sK_X=\R f^!\dZ/n$. Pour $\sK\in \ob\D^-(X,\dZ/n)$, on pose 
$D\sK = \rHom(\sK,\sK_X)$. L'adjonction entre $\R f_!$ et $\R f^!$ 
assure que 
\[\xymatrix{
  \R f_\ast \sK\lotimes \R f_\ast D\sK \ar[r] 
    & \R f_! (\sK\lotimes D\sK) \ar[r] 
    & \R f_! \R f^! \dZ/n \ar[r] 
    & \dZ/n \text{.}
}\]
La sym\'etrie de cette description montre que, pour $f$ propre, le diagramme 
\begin{equation*}\tag{1.2.1}\label{V:eq:1-2-1}
\xymatrix{
  \R f_\ast \sK \ar[r] \ar[d] 
    & D D \R f_\ast \sK \ar[d]^-\sim \\
  \R f_\ast D D \sK \ar[r]^\sim 
    & D\R f_\ast D\sK
}
\end{equation*}
est commutatif. 

Si $f$ est l'inclusion d'un point ferm\'e, on sait que 
$\R f^!\dZ/n=\dZ/n(-1)[-2]$ -- soit, \`a torsion et d\'ecalage pr\`es, $\dZ/n$. 
Sur le spectre d'un coprs, ce complexe est dualisant (dualit\'e de Pontrjagin 
pour les $\dZ/n$-modules). D'apr\`es \eqref{V:eq:1-2-1}, on a donc 
$\sK\iso D D\sK$ pour $\sK$ de la forme $\R f_!\sL$ -- et donc lorsque le 
support de $\underline\h^\bullet(\sK)$ est fini. 





\begin{theorem_}\label{V:1-3}
Soient $j:U\hookrightarrow S$ un ouvert dense de $S$ et $\sF$ un faisceau 
localement constant constructible de $\dZ/n$-modules sur $U$. On a 
$\D j_\ast \sF = j_\ast D \sF$, i.e. $\Hom(j_\ast \sF,\dZ/n) = j_\ast\Hom(\sF,\dZ/n)$ et $\Ext^i(j_\ast\sF,\dZ/n) = 0$ pour $i>0$.
\end{theorem_}

Sur $U$, $\Ext^i(j_\ast\sF,\dZ/n)=0$ pour $i>0$, car $\sF$ est localement 
constant (et $\dZ/n$ est un $\dZ/n$-module injectif), tandis que pour $i=0$ 
c'est le dual $\sF^\vee$ de $\sF$. On v\'erifie que 
$\hom(j_\ast\sF,\dZ/n)=j_\ast\sF^\vee$, et il reste \`a v\'erifier la nullit\'e 
des $\Ext^i$ ($i>0$) en les points de $S\setminus U$. Le probl\`eme est local 
en ces points. Ceci nous ram\`ene \`a supposer que $S$ est un trait strictement 
local et, que $U$ est r\'eduit \`a son point g\'en\'erique $\eta$. Soit 
$I=\gal(\bar\eta/\eta)$. Le faisceau $\sF$ s'identifie au module galoisien 
$\sF_{\bar\eta}$, et la fibre sp\'eciale de $j_\ast \sF$ \`a 
$\sF_{\bar\eta}^I$. 

Soit $i$ l'inclusion du point ferm\'e $s$, et appliquons $D$ aux suites 
exactes 
\[\xymatrix{
  0 \ar[r] 
    & j_! \sF \ar[r] 
    & j_\ast \sF \ar[r] 
    & i_\ast \sF_{\bar\eta}^I \ar[r] 
    & 0 \\
  0 \ar[r] 
    & j_! \sF_{\bar\eta}^I \ar[r] \ar[u] 
    & \sF_{\bar\eta}^I \ar[r] \ar[u] 
    & i_\ast \sF_{\bar\eta}^I \ar[r] \ar[u]
    & 0 \text{.}
}\]

On obtient un morphisme de triangles 
\[\xymatrix{
  i_\ast\left(\sF_{\bar\eta}^I\right)^\vee (-1)[-2] \ar[r] \ar@{=}[d] 
    & D j_\ast\sF \ar[r] \ar[d] 
    & \R j_\ast \sF^\vee \ar[d] \\
  i_\ast\left(\sF_{\bar\eta}^I\right)^\vee(-1)[-2] \ar[r] 
    & \left(\sF_{\bar\eta}^I\right)^\vee \ar[r] 
    & \R j_\ast \sF_{\bar\eta}^I \text{.}
}\]
La suite exacte longue d\'eduite de la premi\`er ligne fournit la nullit\'e des 
$\Ext^i$ ($i>2$) et, prenant la fibre en $s$, on trouve 
\[\xymatrix{
  0 \ar[r] 
    & \left(\underline\h^1(D j_\ast\sF\right)_s \ar[r] 
    & \h^1\left(I,\sF_{\bar\eta}^\vee\right) \ar[r]^-\partial \ar[d] 
    & \left(\sF_{\bar\eta}^I\right)^\vee(-1) \ar[r] \ar@{=}[d] 
    & \left(\underline\h^2 D j_\ast\sF\right)_s \ar[r] 
    & 0 \\
  & 0 \ar[r] 
    & \h^1\left(I,(\sF_{\bar\eta}^I)^\vee\right) \ar[r]^-\partial 
    & \left(\sF_{\bar\eta}^I\right)^\vee(-1) \ar[r] 
    & 0 \text{.}
}\]
Puisque 
$\left(\sF_{\bar\eta}^I\right)^\vee = \left(\sF_{\bar\eta}^\vee\right)_I$, 
posant $M=\sF_{\bar\eta}^\vee$, il faut finalement v\'erifier que 
\[\xymatrix{
  \h^1(I,M) \ar[r]^-\sim 
    & \h^1(I,M_I) \text{.}
}\]
Si $p$ est l'exposant caract\'eristique r\'esiduel, $I$ est extension d'un 
groupe isomorphe \`a $\widehat\dZ_{p'} = \varprojlim_{(m,p)=1} \dZ/m$ par un 
$p$-groupe $P$. Puisque $P$ est premier \`a l'ordre de $M$, on a 
$\h^1(P,M)=0$ ($i>0$) et $(M_I)^P=(M^P)_I$. Ceci permet de remplacer $M$ par 
$M^P$ et $I$ par $I/P$. On a enfin un isomorphisme functoriel 
$\h^1(\widehat\dZ_{p'},M)\sim ($coinvariants de $\widehat\dZ_{p'}$, dans $M)$, 
d'o\`u le th\'eor\`eme. 





\begin{theorem_}\label{V:1-4}
Pour $\sK\in \ob\D_c^b(S,\dZ/n)$, on a $\sK\iso D D \sK$.
\end{theorem_}

Par d\'evissage, on se ram\`ene \`a supposer que $\sK$ est r\'eduit \`a un 
faisceau constructible $\sF$ en degr\'e $0$, et que $\sF$ est soit \`a support 
fini (\ref{V:1-2}), soit de la forme $j_\ast\sF_1$ comme en \ref{V:1-3}. Dans 
ce second cas, \ref{V:1-3} nous ram\`ene \`a la bidualit\'e locale pour $\sF$ 
localement constant sur $U$. 










\section{La dualité de Poincaré pour les courbes, d'apr\`es M.\ Artin}\label{V:2}





\subsection{}\label{V:2-1}

Soit $X$ une courbe projective et lisse sur $k$ alg\'ebriquement clos. On pose 
$\sK_X=\dZ/n(1)[2]$ et, pour $\sK\in\ob \D_c^b(X,\dZ/n)$, 
$D\sK=\rHom(\sK,\sK_X)$. Pour $M$ un $\dZ/n$-module, on pose aussi 
$D M = \hom(M,\dZ/n)$; de m\^eme pour les complexes de modules. Le morphisme 
trace $\h^0(X,\sK_X) \to \dZ/n$, ou $\R\Gamma(X,\sK_X)\to \dZ/n$, d\'efinit un 
accouplement $\R\Gamma(X,\sK)\lotimes\R\Gamma(X,D\sK) \to \dZ/n$. La dualit\'e 
de Poincar\'e entre cohomologie et cohomologie \`a supports propres d'un ouvert 
$j:U\hookrightarrow X$ de $X$ dit que, pour $\sK=j_!\dZ/n$, cet accouplement 
identifie chaque facteur au dual de l'autre. 

J'expose ci-dessous une d\'emonstration, qui m'a \'et\'e communiqu\'ee par 
M.\ Artin, de ce que pour $\sK\in\ob\D_c^b(X,\dZ/n)$, cet accouplement est 
toujours parfait, i.e. d\'efinit un isomorphisme
\begin{equation*}\tag{2.1.1}\label{V:eq:2-1-1}
\xymatrix{
  \R\Gamma(X,\sK) \ar[r]^-\sim 
    & D \R\Gamma(X,D\sK) \text{.}
}
\end{equation*}

Pour tout faisceau constructible $\sF$, posons 
\[
  '\h^i(X,\sF) = \text{$\dZ/n$-dual de $\h^{-i}(X,D\sF)$.}
\]
Que \eqref{V:eq:2-1-1} soit un isomorphisme \'equivaut au 





\begin{theorem_}\label{V:2-2}
Pour $\sF$ un faisceau constructible de $\dZ/n$-modules, on a 
\begin{equation*}\tag{2.2.1}\label{V:eq:2-2-1}
\xymatrix{
  \h^i(X,\sF) \ar[r]^-\sim 
    & '\h^i(X,\sF) \text{.}
}
\end{equation*}
\end{theorem_}





\begin{lemma_}\label{V:2-3}
Soit $f:X\to Y$ un morphisme g\'en\'eriquement \'etale entre courbes lisses sur 
$k$. On a $\sK_X=f^\ast\sK_Y=\R f^!\sK_Y$, avec $\tr_f$ pour fl\`eche 
d'adjonction $\R f_!\R f^!\sK_Y\to \sK_Y$.
\end{lemma_}

% NOTE: \rhom or \rHom ? originally \rhom
On se ram\`ene \`a v\'erifier que pour $f$ fini, on a 
$f_\ast \R f^!\sK_Y = \rHom(f_\ast\dZ/n,\sK_Y)$ est $f_\ast\dZ/n$, avec $\tr_f$ 
pour fl\`eche d'adjonction. La nullit\'e des $\Ext^i$ ($i>0$) r\'esulte de 
\ref{V:1-3}, et l'assertion en r\'esulte. 

On peut dire, plus explicitement, que $\R f^!$ est le foncteur d\'eriv\'e du 
foncteur $f^!:f^!\sF(U)=\hom(f_!\dZ_U,\sF)$ et \ref{V:1-3} implique la 
nullit\'e des $\R^i f^!\dZ/n$ pour $i>0$.





\begin{corollary_}\label{V:2-4}
Soit $f:X'\to X$ un morphisme g\'en\'eneriquement \'etale de courbes 
projectives et lisses sur $k$. On a $'\h^i(X,f_\ast \sF)='\h^i(X',\sF)$; cet 
isomorphisme est fonctoriel et le diagramme 
\[\xymatrix{
  \h^i(X,f_\ast\sF) \ar[r] \ar@{=}[d] 
    & '\h^i(X,f_\ast\sF) \ar@{=}[d] \\
  \h^i(X',\sF) \ar[r] 
    & '\h^i(X',\sF) 
}\]
est commutatif.
\end{corollary_}

L'isomorphisme est donn\'e par \ref{V:1-2}: $D f_\ast\sF=f_\ast D\sF$. 





\subsection{}\label{V:2-5}

Prouvons \ref{V:2-2}. Si $\sF$ est r\'eduit \`a $\dZ/n$ en un point, prolong\'e 
par $0$, on a $\h^\bullet(D\sF) = \ext^\bullet(\sF,\sK_X)=\dZ/n$ en degr\'e 
$0$, et dans l'accouplement $\h^\bullet(\sF)\otimes\h^\bullet(D\sF) \to \dZ/n$, 
on a $1\otimes 1\mapsto 1$ (\hyperref[IV]{Cycle}, \ref{IV:2-1-5}): la dualit\'e 
est parfaite. Le cas o\`u $\sF$ est \`a support fini se traite de m\^eme. 

Pour $\sF$ constructible quelconque, $\sA$ le sous-faisceau de ses sections \`a 
support fini, et $\sG=\sF/\sA$, la suite exacte 
\begin{equation*}\label{V:eq:2-5-1}\tag{2.5.1}
\xymatrix{
  0 \ar[r] 
    & \sA \ar[r] 
    & \sF \ar[r] 
    & \sG \ar[r] 
    & 0 
}
\end{equation*}
fournit un diagramme 
\begin{equation*}\tag{2.5.2}\label{V:eq:2-5-2}
\xymatrix@=.5cm{
  & & 0 \ar[r] 
    & \h^0(X,\sA) \ar[r] \ar[d]^-\sim 
    & \h^0(X,\sF) \ar[r] \ar[d] 
    & \h^0(X,\sG) \ar[r] \ar[d] 
    & 0 \\
  0 \ar[r] 
  & '\h^{-1}(X,\sF) \ar[r] 
  & '\h^{-1}(X,\sF) \ar[r] 
  & '\h^0(X,\sA) \ar[r] 
  & '\h^0(X,\sF) \ar[r] 
  & '\h^0(X,\sF) \ar[r] 
  & 0 
}
\end{equation*}
et des isomorphismes compatibles $\h^i(X,\sF)\iso \h^i(X,\sG)$, 
$'\h^i(X,\sF) \iso '\h^i(X,\sG)$ ($i\ne 0,-1$). Si $j:U\to X$ est un ouvert sur 
lequel $\sF$ est localement constant, on a une suite exacte 
\begin{equation*}\tag{2.5.3}\label{V:eq:2-5-3}
\xymatrix{
  0 \ar[r] 
    & \sG \ar[r] 
    & j_\ast j^\ast \sF \ar[r] 
    & \sB \ar[r] 
    & 0
}
\end{equation*}
avec $\sB$ \`a support fini. Puisque 
$'\h^i(X,j_\ast j^\ast\sF) = D \h^{2-i} (X,j_\ast(j^\ast\sF)^\vee(1))$ 
d'apr\`es \ref{V:1-3}, cet $'\h^i=0$ pour $i<0$ et \eqref{V:eq:2-5-2} et la 
suite exacte longue d\'efinie par \eqref{V:eq:2-5-3} montrent que $'\h^i$ est 
toujours nul pour $i<0$. 

Si $\sF=\dZ/n$, le th\'eor\`eme est vrai pour $i=0$ et $2$. Ceci exprime que, 
pour $X$ connexe, le morphisme trace: $\h^2(X,\dmu_n) \iso \dZ/n$ est un 
isomorphisme. D'apr\`es \ref{V:2-4}, le th\'eor\`eme reste vrai pour $i=0$ et 
$\sF=f_\ast\dZ/n$. 

Pour $\sF$ \`a nouveau quelconque et $\sG$ comme plus haut, il existe 
$f:X'\to X$ comme en \ref{V:2-4} tel que $\sG$ se plonge dans $f_\ast\dZ/n$:
\[\xymatrix{
  0 \ar[r] 
    & \sG \ar[r] 
    & f_\ast\dZ/n \ar[r] 
    & \sQ \ar[r] 
    & 0 \text{.}
}\]
Changeant $X'$ en un $X''/X'$, on peut supposer que 
$\h^i(X,\sF) \to \h^i(X,f_\ast\dZ/n) = \h^i(X',\dZ/n)$ est nul pour $i>0$. 
Consid\'erons alors 
\[\xymatrix@=.3cm{
  0 \ar[r] 
    & \h^0(X,\sG) \ar[r] \ar[d] 
    & \h^0(S',\dZ/n) \ar[r] \ar[d]^-\sim 
    & \h^0(X,\sQ) \ar[r] \ar[d] 
    & \h^1(X,\sG) \ar[r]^-0 \ar[d] 
    & \h^1(X',\dZ/n) \ar[r] \ar[d] 
    & \h^1(X,\sQ) \ar[r] \ar[d] 
    & \cdots \\
  0 \ar[r] 
    & '\h^0(X,\sG) \ar[r] 
    & '\h^0(X',\dZ/n) \ar[r] 
    & '\h^0(X,\sQ) \ar[r] 
    & '\h^1(X,\sG) \ar[r] 
    & '\h^1(X',\dZ/n) \ar[r] 
    & '\h^1(X,\sQ) \ar[r] 
    & \cdots
}\]

L'isomorphisme en $\h^0(X')$ fournit $\h^0(X,\sG)\hookrightarrow '\h^0(X,\sG)$, 
et la m\^eme injectivit\'e pour $\sF$ \eqref{V:eq:2-5-2}. Appliqu\'e \`a $\sQ$, 
cela donne $\h^1(X,\sG)\hookrightarrow '\h^1(X,\sG)$. On applique cela \`a 
$\h^1(X',\dZ/n)$. Ce groupe ayant m\^eme ordre que 
$'\h^1(X',\dZ/n)=D(\h^1(X',\dZ/n)(1))$, on trouve un isomorphisme, et et 
$\h^1(X,\sG) \iso '\h^1(X,\sG)$. De m\^eme pour $\sF$. Pour $i=2$, on sait 
d\'ej\`a que $\h^2(X',\dZ/n) \iso '\h^2(X',\dZ/n)$, et on proc\`ede de m\^eme, 
puis s'arr\^ete, faute de combattants. 










\section{Dualit\'e de Poincar\'e pour les courbes}\label{V:3}

Dans ce paragraphe, nous prouvons la compatibilit\'e (\hyperref[I]{Arcata}  
\ref{I:eq:6-2-3-3}). 





\subsection{}\label{V:3-1}

Les notations sont celles de (\hyperref[I]{Arcata} \ref{I:6-2-3}): $\bar X$ est 
une courbe projective, lisse et connexe sur un corps alg\'ebriquement clos $k$, 
$D$ est un diviseur r\'eduit sur $\bar X$, $0$ un point de 
$X=\bar X\setminus D$ 
\[\xymatrix{
  X \ar@{^{(}->}[r]^-j 
    & \bar X 
    & \ar[l]_-i D \text{,}
}\]
et $\pic_D(\bar X)=\h^1(\bar X,_D\dG_m)$ o\`u $_D\dG_m$ est le faisceau des 
sections de $\dG_m$ congrues \`a $1\mod D$. On rappelle que 
$\h^1(X,\dmu_n) = \pic_D^0(\bar X)_n$, et que $x\to \sO(x)$ d\'efinit une 
application canonique $f:X\to \pic_D(\bar X)$. On pose 
$f_0(x)=f(x)-f(0):X\to \pic_D^0(\bar X)$.

Notons encore $j$ l'inclusion de $X\times X$ dans $X\times \bar X$. La 
diagonale $\Delta$ de $X\times X$ est ferm\'ee dans $X\times \bar X$; elle 
d\'efinit une classe dans 
$\h_\Delta^2(X\times X,\dmu_n)=\h_\Delta^2(X\times \bar X,j_!\dmu_n)$. On note 
$c$ son image dans $\h^2(X\times \bar X,j_!\dmu_n)$. 

La formule de K\"unneth assure que 
\[
  \h^\bullet(X\times \bar X,j_!\dmu_n) = \h^\bullet(X,\dZ/n)\otimes \h^\bullet(\bar X,j_!\dmu_n) = \h^\bullet(X,\dZ/n)\otimes \h_c^\bullet(X,\dmu_n) \text{.}
\]
Nous nous proposons de calculer la $(1,1)$-composante de $c$, 
\[
  c^{1,1} \in \h^1(X,\dZ/n)\otimes \h_c^1(X,\dmu_n) = \h^1(X,\h_c^1(X,\dmu_n)) = \h^1\left(X,\pic_D^0(\bar X)_n\right) \text{.}
\]

La suite exacte 
\[\xymatrix{
  0 \ar[r] 
    & \pic_D^0(\bar X)_n \ar[r] 
    & \pic_D^0(\bar X) \ar[r]^n 
    & \pic_D^0(\bar X) \ar[r] 
    & 0 
}\]
fait de $\pic_D^0(\bar X)$ un $\pic_D^0(\bar X)_n$-torseur sur 
$\pic_D^0(\bar X)$. Notons, $u$ la classe, dans 
$\h^1(X,\h_c^1(X,\dmu_n))=\h^1(X,\pic_D^0(\bar X)_n)$, de son image 
r\'eciproque par $f_0$. 





\begin{proposition_}\label{V:3-2}
Avec les notations pr\'ec\'edentes, $c^{1,1}=-u$.
\end{proposition_}

Soit $e$ la diff\'erence des images dans $\h^2(X\times \bar X,j_!\dmu_n)$ des 
classes de cohomologie de $\Delta$ et $X\times\{0\}$ \'etant de type de 
K\"unneth $(0,2)$, on a $c^{1,1}=e^{1,1}$. Soit la suite spectrale de Leray 
pour $\operatorname{pr}_1:X\times \bar X\to X$ et le faisceau $j_!\dmu_n$: 
\[
  E_2^{pq} = \h^p(X,\R^q {\operatorname{pr}_1}_\ast j_!\dmu_n) = \h^p(X,\h_c^q(X,\dmu_n)) \Rightarrow \h^{p+q}(X\times \bar X,j_!\dmu_n) \text{.}
\]
Elle d\'eg\'en\`ere: c'est le produit tensoriel de $\h^\bullet(X)$ par la 
suite spectrale de Leray (triviale) pour $\bar X\to \spec(k)$ et le faisceau 
$j_!\dmu_n$. Le diviseur $\Delta-X\times\{0\}$ est fibre par fibre de degr\'e 
$0$, de sorte que l'image de $e$ dans $E^{02}$ est nulle. On peut donc parler 
de son image $\bar e$ dans $E^{1,1}$, et il nous faut montrer que 
$\bar e = -u$. 

Soit $\sG$ sur $X\times\bar X$ le sous-faisceau de $\dG_m$ form\'e des sections 
locales dont la restriction au sous-sch\'ema $X\times D$ est $1$. On a encore 
une suite exacte 
\begin{equation*}\tag{3.2.1}\label{V:eq:3-2-1}
\xymatrix{
  0 \ar[r] 
    & j_!\dmu_n \ar[r] 
    & \sG \ar[r] 
    & \sG \ar[r] 
    & 0 
}
\end{equation*}
et $e$ est l'image par $\partial$ de la classe $e_1\in \h^1(X\times\bar X,\sG)$ 
du faisceau inversible $\sO(\Delta-X\times\{0\})$ trivialis\'e par $1$ sur 
$X\times D$. 

L'image directe par $\R\operatorname{pr}_{1\ast}$ du triangle distingu\'e par 
\eqref{V:eq:2-2-1} est un triangle distingu\'e 
\begin{equation*}\tag{3.2.2}\label{V:eq:3-2-2}
\xymatrix{
  & \ar[r]^-\partial 
    & \R \operatorname{pr}_{1\ast} j_!\dmu_n \ar[r] 
    & \R \operatorname{pr}_{2\ast}\sG \ar[r] 
    & \R \operatorname{pr}_{2\ast}\sG \ar[r]^-\partial 
    & 
}
\end{equation*}
et le diagramme 
\[\xymatrix{
  \h^1(X\times \bar X,\sG) \ar[r]^-\partial \ar@{=}[d]^-{(1)} 
    & \h^2(X\times \bar X,j_!\dmu_n) \ar@{=}[d] \\
  \h^1(X,\R\operatorname{pr}_{1\ast} \sG) \ar[r]^-\partial 
    & \h^2(X,\R\operatorname{pr}_{1\ast} j_!\dmu_n) 
}\]
est commutatif: notre probl\`eme devient celui d'identifier l'image dans 
$E^{11}=\h^1(X,\h_c^1(X,\dmu_n))$ (l'image dans $E^{02}$ \'etant nulle) de 
l'image par 
$\partial_{\text{\eqref{V:eq:3-2-2}}}:\h^1(X,\R\operatorname{pr}_{1\ast}\sG) \to \h^2(X,\R\operatorname{pr}_{1\ast} j_!\dmu_n)$ 
de la classe, encore not\'ee $e_1$, qui correspond \`a $e_1$ par 
(1). 

Le faisceau $\R^1\operatorname{pr}_{1\ast}\sG$ est d\'efini par le sch\'ema en 
groupe sur $X$ image r\'eciproque du groupe alg\'ebrique $\pic_D(\bar X)$ sur 
$\spec(k)$, et l'image de $e_1$ dans 
$\h^0(X,\R^1\operatorname{pr}_\ast\sG) = \hom(X,\pic_D(\bar X))$ est $f_0$. 

Repr\'esentons le triangle \eqref{V:eq:3-2-2} comme d\'efini par une suite 
exacte courte de complexes de faisceaux.
\[\xymatrix{
  0 \ar[r] 
    & \sA \ar[r] 
    & \sB \ar[r] 
    & \sC \ar[r] 
    & 0 \text{.}
}\]
Soit $\sB'$ le sous-faisceau de 
\[
  \tau_{\leqslant 1} \sB : \cdots \sB^0 \to \ker(d) \to 0 \cdots
\]
obtenu en remplaçant ${\sB'}^1=\ker(d)$ par l'image r\'eciproque, par 
$\ker(d)\to \underline\h^1(\sB) = \pic_D(\bar X)$ sur $X$, de 
$\pic_D^0(\bar X)$ sur $X$. Soient $\sA'=\sA\cap \sB'$ et $\sC'$ l'image de 
$\sB'$. On a $\sA'=\tau_{\leqslant 1}(\sA)$, et $\sC'$ se d\'eduit de $\sC$ 
comme $\sB'$ de $\sB$, d'o\`u un diagramme commutatif de suites exactes courtes 
\begin{equation*}\tag{3.2.2}\label{V:eq:3-2-2}
\xymatrix{
  0 \ar[r] 
    & \sA \ar[r] 
    & \sB \ar[r] 
    & \sC \ar[r] 
    & 0 \\
  0 \ar[r] 
    & \tau_{\leqslant 1} \sA \ar[r] \ar[u] \ar[d] 
    & \sB' \ar[r] \ar[u] \ar[d] 
    & \sC' \ar[r] \ar[u] \ar[d] 
    & 0 \\
  0 \ar[r] 
    & \h_c^1(X,\dmu_n)[1] \ar[r] 
    & \pic_D^0(\bar X)[1] \ar[r] 
    & \pic_D^0(\bar X)[1] \ar[r] 
    & 0
}
\end{equation*}
o\`u la derni\`ere ligne doit \^etre vue comme une suite exactes de complexes 
de faisceaux r\'eduits au degr\'e $1$ sur $X$. 
\begin{enumerate}[a)]
  \item $e\in \hh^2(X,\sA)$ provient de 
    $\widetilde e\in \hh^2(X,\tau_{\leqslant 1}\sA)$ (car son image dans 
    $E^{02}$ est nulle). On cherche l'image $\bar e$ de $\widetilde e$ dans 
    $\hh^2(X,\h_c^1(X,\dmu_n)[-1]) = \h^1(X,\h_c^1(X,\dmu_n))$.
  \item $e_1\in \hh^1(X,\sC)$ provient de $\widetilde e_1\in \hh^1(X,\sC')$, 
    car son image $f_0$ dans $\h^0(X,\pic_D(\bar X))$ est dans 
    $\h^0(X,\pic_D^0(\bar X))$. On peut prendre 
    $\widetilde e=\partial \widetilde e_1$. 
  \item On a donc $\bar e = \partial f_0$, o\`u $\partial $ est d\'efini par 
    \eqref{V:eq:3-2-2}, et o\`u on utilise les isomorphismes usuels 
    $\hh^2(X,\sF[-1]) = \h^1(X,\sF)$. Ces isomorphismes anticommutent \`a 
    $\partial$, de sorte que $\bar e=-\partial f_0$, pour $\partial$ d\'efini 
    par le suite exacte 
    \[\xymatrix{
      0 \ar[r] 
        & \h_c^1(X,\dmu_n) \ar[r] 
        & \pic_D^0(\bar X) \ar[r] 
        & \pic_D^0(\bar X) \ar[r] 
        & 0 
    }\]
    On conclut par (\hyperref[IV]{Cycles}, \ref{IV:1-1-1}). 
\end{enumerate}





\subsection{}\label{V:3-3}

Pour tout homomorphisme $\varphi:\h_c^1(X,\dmu_n)\to \dZ/n$, soit 
$\varphi(u)\in \h^1(X,\dZ/n)$ l'image de $u$ par 
$\varphi:\h^1(X,\h_c^1(X,\dmu_n)) \to \h^1(X,\dZ/n)$. Si 
$x\in \h_c^1(X,\dmu_n)$, le cup-produit $\varphi(u)\smallsmile x$ est dans 
$\h_c^2(X,\dmu_n)$. On se propose de prouver que 





\begin{proposition_}\label{V:3-4}
$\tr(\varphi(u)\smallsmile x) = \varphi(x)$.
\end{proposition_}

Cette identit\'e se d\'eduit, par application de $\varphi$, de 
\begin{equation*}\tag{3.4.1}\label{V:eq:3-4-1}
  \tr(u\smallsmile x) = x 
\end{equation*}
o\`u $\tr$ est cette fois l'application 
\[\xymatrix{
  \tr:\h_c^2(X,\h_c^1(X,\dmu_n)\otimes\dmu_n) \ar[r] 
    & \h_c^1(X,\dmu_n) \text{,}
}\]
de sorte que $x\mapsto \tr(u\smallsmile x)$ est l'application compos\'ee 
\begin{equation*}\tag{3.4.2}\label{V:eq:3-4-2}
\xymatrix{
  \h_c^1(X,\dmu_n) \ar[r]^-{u\otimes x} 
    & \h^1(X,\h_c^1(X,\dmu_n)) \otimes \h_c^1(X,\dmu_n) 
    & \ar[l]_-\sim \h^1(X,\dZ/n)\otimes \h_c^1(X,\dmu_n) \otimes \h_c^1(X,\dmu_n) \\
    \ar[r]^-{(1)} 
      & \h_c^2(X,\dmu_n) \otimes \h_c^1(X,\dmu_n) \ar[r]^-{\tr\otimes\operatorname{id}} 
      & \h_c^1(X,\dmu_n)
}
\end{equation*}
o\`u (1) est donn\'e par 
$a\otimes b\otimes c\mapsto (a\smallsmile c)\otimes b$. 

Exprimons ce morphisme en terme du morphisme trace 
$\tr:\h_c^3(X\times X,\dmu_n^{\otimes 2}) \to \h_c^2(X,\dmu_n)$ relatif \`a la 
seconde projection. Pour $x\in \h_c^3$, $\tr(x)$ se calcule ainsi: de ses 
composantes de K\"unneth, on ne garde que celle de type $(2,1)$, et on applique 
$\tr:\h_c62(X,\dmu_n) \to \dZ/n$. 

Notons $u'$ l'image de $u$ par l'application 
\[\xymatrix{
  \h^1(X,\h_c^1(X,\dmu_n)) 
    & \ar[l]_-\sim \h^1(X,\dZ/n) \otimes \h_c^1(X,\dmu_n) = \h^1(X,\dZ/n)\otimes \h^1(\bar X,j_!\dmu_n) \\
    \ar[r] 
      & \h^2(X\times \bar X,\dZ/n\boxtimes j_!\dmu_n) \text{.}
}\]
Vu la relation entre K\"unneth et cup-produit, on a 
\[
  \tr(u\smallsmile x) = -\tr(u'\smallsmile \operatorname{pr}_1^\ast x) \text{.}
\]
Le second cup-produit est celui de (\hyperref[IV]{Cycle} \ref{IV:1-2-4}):
\[\xymatrix{
  \h^2(X\times \bar X,\dZ/n\boxtimes j_!\dmu_n) \otimes \h^1(\bar X\times \bar X,j_!\dmu_n\boxtimes \dZ/n) \ar[r] 
    & \h^3(\bar X\times \bar X,j_!\dmu_n \boxtimes j_!\dmu_n) \text{.}
}\]
Le signe -- provient de la permutation de deux symboles de degr\'e $1$ dans 
(1). Puisque $c\smallsmile\operatorname{pr}_1^\ast x$ et 
$c^{1,1}\smallsmile \operatorname{pr}_1^\ast$ ont m\^eme $(2,1)$-composante de 
K\"unneth, et que $u'=c^{1,1}$, on a 
\[
  \tr(u\smallsmile x) = \tr(c\smallsmile \operatorname{pr}_1^\ast x) \text{.}
\]
Le cup-produit \`a droite peut cette fois s'interpr\'eter comme le 
cup-produit (\hyperref[IV]{Cycle} \ref{IV:1-2-5}) de 
$\cl(\Delta)\in \h_\Delta^2(X\times X,\dmu_n)$ par 
$\operatorname{pr}_1^\ast x\in \h_c^1(\Delta,\dmu_n)$. Sur $\Delta$, on a 
$\operatorname{pr}_1=\operatorname{pr}_2$, d'\`u 
\[
  \tr(c\smallsmile \operatorname{pr}_1^\ast x) = \tr(\cl(\Delta)\smallsmile \operatorname{pr}_1^\ast x) = \tr(\cl(\Delta)\smallsmile \operatorname{pr}_2^\ast x) = \tr(\cl(\Delta))\smallsmile x = x \text{.}
\]










\section{Compatibilit\'e \texorpdfstring{\cite[XVIII 3.1.10.3]{sga4}}{[SGA 4, XVIII 3.1.10.3]}}

Dans \cite[XVIII]{sga4}, en proie au d\'ecouragement, je n'ai pas v\'erifi\'e 
la compatibilit\'e du titre. Bien qu'elle s'av\`ere n'\^etre que fum\'ee, je me 
crois tenu du r\'eparer ici cette lacune. Pour $f:X\to S$ s\'epar\'e de type 
fini, il s'agissait de v\'erifier que le morphisme d'adjonction 
$\R f_!\R f^!\sL\to \sL$ est compatible \`a toute localisation \'etale 
$k:V\to S$. D\'esignant par $k$ et $f$ plusieurs fl\`eches, comme ci-dessous, 
il s'agit de v\'erifier une commutativit\'e:
\[\xymatrix{
  X_V \ar[r]^-k \ar[d]^-f 
    & X \ar[d]^-f \\
  V \ar[r]^-k 
    & S 
} \qquad 
\xymatrix{
  k^\ast \R f_! \R f^!\sL \ar[r]^-{k^\ast(\text{adj})} \ar[d]^-\sim 
    & k^\ast \sL \ar@{=}[d] \\
  \R f_! \R f^! k^\ast \sL \ar[r]^-{\text{adj}} 
    & k^\ast \sL \text{.}
}\]
Au niveau des complexes, la commutativit\'e voulue s'\'ecrit (loc. cit.) 
\[\xymatrix{
  k^\ast f_!^\bullet f^!_\bullet \sL \ar[r] \ar[d]^-{(1)} 
    & k^\ast \sL \ar@{=}[d] \\
  f_!^\bullet f^!_\bullet k^\ast \sL \ar[r] 
    & k^\ast \sL \text{.}
}\]
On la v\'erifie composante par composante, ce qui ram\`ene \`a une 
compatabilit\'e 




