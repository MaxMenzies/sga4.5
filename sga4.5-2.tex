% !TEX root = sga4.5.tex

\chapter{Rapport sur la formule des traces}\label{II}

Dans ce texte, j'ai tenté d'exposer de façon aussi directe que possible la 
théorie cohomologique de Grothendieck des fonctions $L$. Je suis de très près 
certains des exposés donnés par Grothendieck à l'IHES au printemps 1966. Dans 
l'esprit de ce volume, il ne sera pas fait appel à SGA 5 -- saui deux 
références à des passages de l'exposé XI, indépendant au reste de ce 
séminaire. 

Dans le \S 10 (p.81-90) de \cite{de73}, le lecteur trouvera un résume de la 
théorie, dans le cas crucial des courbes. 










\section{Quelques rappels}\label{II:1}





\subsection{Notations}\label{II:1-1}

$p,q,\dF,\dF_q$: $p$ est un nombre premier, $q=p^f$ une puissance de $p$ et 
$\dF$ un clôture algébrique du corps premier $\dF_p$. Pour toute puissance 
$p^n$ de $p$, on note $\dF_{p^n}$ le sous-corps à $p^n$ éléments de $\dF$. 

$X_0,X$: On désignera souvent par un symbole affecte d'un indice $0$ un objet 
sur $\dF_q$. Le même symbole, sans $0$, désigne alors l'objet qui s'en déduit 
par extension des scalaires à $\dF$. Par exemple, si $\sF_0$ est un faisceau 
sur un schéma $X_0$ sur $\dF_q$, on note $\sF$ son image réciproque sur 
$X=X_0\otimes_{\dF_q} \dF$. 

$F$: Si $X_0$ est un schéma sur $\dF_q$, on appelle \emph{morphisme de 
Frobenius} et on note $F$ l'endomorphisme de $X_0$ qui ``envoie le point de 
coordonnées $x$ dans celui de coordonnées $x^q$'': c'est l'identifie sur 
l'espace topologique sous-jacent et, pour $x$ une section locale du faisceau 
structurel, $F^* x = x^q$. 

On désigne encore par $F$ l'endomorphisme de $X$ qui s'en déduit par 
extension des scalaires. Sur l'ensemble $X(\dF) = X_0(\dF)$ des points 
rationnels de $X$, $F$ agit comme la substitution de Frobenius 
$\varphi\in\gal(\dF/\dF_q)$ définie par $\varphi(x)=x^q$. En particulier, 
$X^F=X_0(\dF_q)$. 

En cas d'ambiguïté, on écrira $F_{(q)}$ plutôt que $F$. 










\subsection{Correspondance de Frobenius}\label{II:1-2}

Soient $X_0$ un schéma sur $\dF_q$, et $\sF_0$ un faisceau sur $X_0$. J'aime 
voir comme suit la \emph{correspondance de Frobenius} (un $F$-endomorphisme de 
$\sF_0$): 
\begin{equation}\label{II:eq:1-2-1}
  F^* : F^*\sF_0 \iso \sF_0
\end{equation}

On regarde $\sF_0$ comme le faisceau des sections locales d'un espace 
$[\sF_0]$ étale sur $X_0$ ($[\sF_0]$ est un espace algébrique au sens de M. 
Artin). La fonctorialité de $F$ fournit un diagramme commutatif 
\[\xymatrix{
  [\sF_0] \ar[r]^-F \ar[d]
    & [\sF_0] \ar[d] \\
  X_0 \ar[r]^F 
    & X_0
}\]
Parce que $[\sF_0]$ est étale sur $X_0$, ce diagramme est cartésien; il fournit 
un isomorphisme $[\sF_0]\iso F^*[\sF_0]$ d'inverse \eqref{II:eq:1-2-1}. On note 
encore $F^*$ les correspondances ou morphismes déduits de $F$ par extension 
des scalaires ou par fonctorialité, tels 
$F^*:\h_c^\bullet(X,\sF)\to \h_c^\bullet(X,\sF)$. 

Pour une définition en forme, je renvoie à \cite[XI.1,2]{sga5}. On n'y c
considère que le cas où $q=p$, mais le cas général se traite de même. 





\subsection{}\label{II:1-3}

La correspondance de Frobenius est fonctorielle en $(X_0,\sF_0)$. Si 
$f_0:X_0\to Y_0$ est un morphisme de $\dF_q$-schémas, $\sF_0$ un faisceau sur 
$X_0$, $\sG_0$ un faisceau sur $Y_0$ et $u\in\hom(f_0^*\sG_0,\sF_0)$ un 
$f_0$-morphisme de $\sG_0$ dans $\sF_0$, on a un diagramme commutatif d'espaces 
$f_0 F = F f_0$, et au-dessus un diagramme commutatif de faisceaux 
\[\xymatrix{
  X_0 \ar[r]^-{f_0} \ar[d]^-F
    & Y_0 \ar[d]^-F \\
  X_0 \ar[r]^-{f_0} 
    & Y_0
}\qquad\qquad
\xymatrix{
  \sF_0 
    & \ar[l]_-u \sG_0 \\
  \sF_0 \ar[u]_-{F^*} 
    & \ar[l] \sG_0 \ar[u]^-{F^*}
}\]
En particulier, la correspondance sur $f_{0*}\sF_0$ déduite la correspondance 
de Frobenius de $(X_0,\sF_0)$ est la correspondance de Frobenius de 
$(U_0,f_{0*}\sF_0)$. 





\subsection{}\label{II:1-4}

Le morphisme naturel $Y\to Y_0$ est limite de morphismes étales. Il en 
résulte que pour tout faisceau abélien $\sF_0$ sur $X_0$, les images 
réciproques sur $Y$ des $\R^if_0\sF_0$ sont les $\R^if_*\sF$. 

Sur $\R^if_*\sF$, on dispose de 
\begin{enumerate}[\indent a)]
  \item la correspondance 
    $F^*\R^i f_*\sF \iso \R^i f_*\sF \xrightarrow{F^*} \R^if_*\sF$ déduite 
    par fonctorialité de la correspondance Frobenius; 
  \item la correspondance $F^*\R^if_*\sF\to\R^if_*\sF$ image réciproque sur 
    $Y$ de la correspondance de Frobenius sur $\R^ii f_{0*}\sF_0$. 
\end{enumerate}

On déduit de (\ref{II:1-3}) que ces correspondances coïncident. Elles 
induisent la terme initial d'une action de Frobenius sur toute la suite 
spectrale 
\[
  \h^p(Y,\R^q f_*\sF) \Rightarrow \h^{p+q}(X,\sF) \text{.}
\]
De même en cohomologie à support propre, et pour la suite spectrale 
\[
  \h_c^p(Y,\R^q f_* \sF) \Rightarrow \h_c^{p+q}(X,\sF)
\]





\subsection{Changement de corps fini}\label{II:1-5}

Soient $(X_0,\sF_0)$ sur $\dF_q$, $n$ un entier, et $(X_1,\sF_1)$ sur 
$\dF_{q^n}$ déduit de $(X_0,\sF_0)$ par extension des scalaires. On dispose 
d'une part de la correspondance de Frobenius $F_{(q)}:X_0\to X_0)$, 
$F_{(q)}^*:F_{(q)}^*\sF_0\to \sF_0$, d'autre part de 
$F_{(q^n)}:X_1\to X_1$, $F_{(q^n)}^*:F_{(q^n)}^*\sF_1\to\sF_1$. On vérifie 
facilement que $F_{(q^n)}$ se déduit de la puissance $n$-ième de $F_{(q)}$ 
par extension des scalaires de $\dF_q$ à $\dF_{q^n}$; après extension des 
scalaires à $\dF$, on a encore la même identité $F_{(q^n)}=F_{(q)}^n$ 
entre endomorphismes de $X$ et correspondances sur $X$. 

Si $x$ est un point fermé de $X_0$, avec $[k(x),\dF_q]=n$, et que 
$\bar x\in X_0(\dF) = X(\dF)$ est localisé en $x$, $\bar x$ est un point 
fixe de $F_{(q^n)}=F_{(q)}^n$. On note $F_x^*$ l'endomorphisme de 
$\sF_{\bar x}$ induit par $F_{(q^n)}^*$. A isomorphisme près, 
$(F_{\bar x}^*,\sF_{\bar x})$ ne dépend pas du choix de $\bar x$. La trace, 
etc. de $F_x^*$ se désigne par une notation telle que 
$\tr(F_x^*,\sF)$. 





\subsection{Fonction \texorpdfstring{$L$}{L}}\label{II:1-6}

Soit $X_0$ un schéma de type fini sur $\dF_q$. On note $|X_0|$ l'ensemble de 
ses points fermés et, pour $x\in |X_0|$, on pose $\deg(x)=[k(x),\dF_p]$. Si 
$\sF_0$ est un $\dQ_\ell$-faisceau sur $X_0$ (pour cette notion, voir 
ci-dessous) -- ou un faisceau constructible et plat de modules sur un 
anneau noethérien commutatif $A$, on note $L(X_0,\sF_0)$ la série formelle 
de terme constant $1$, dans $\dQ_\ell\llbracket t\rrbracket$ ou dans 
$A\llbracket t\rrbracket$, 
\[
  L(X_0,\sF_0) = \prod_{x\in|X_0|} \det\left(1-F_x^* t^{\deg(x)},\sF\right)^{-1} \text{.}
\]





\subsection{Remarque}\label{II:1-7}

Dans la définition (\ref{II:1-6}), on a $\deg(x)=[k(x),\dF_p]$ (degré 
absolu) et non $\deg(x)=[k(x),\dF_q]$ (degré relatif à $\dF_q$). Ceci à 
pour effet que $L(X_0,\sF_0)$ ne dépend que du schéma $X_0$ et du faisceau 
$\sF_0$, non de $q$ et du morphisme structurel $X_0\to\spec(\dF_q)$. 
L'indéterminée $t$ est un avatar de $p^{-s}$. 





\subsection{Le point de vue galoisien}\label{II:1-8}

Ce n$^\circ$ ne serra pas utilisé dans la suite du rapport. Pour $\sF_0$ un 
faisceau abélien sur $X_0$, le groupe de Galois $\gal(\dF/\dF_q)$ agit sur 
$\dF$, et par transport de structure sur $\h_c^i(X,\sF)$. En particulier, la 
substitution de Frobenius $\varphi$ (\ref{II:1-1}) induit un automorphisme, 
encore noté $\varphi$, de $\h_c^i(X,\sF)$. La correspondance de Frobenius 
définit de son côté un endomorphisme $F^*$ de $\h_c^i(X,\sF)$. On a 
\begin{equation}\label{II:eq:1-8-1}
  {F^*}^{-1} = \varphi \qquad \text{(sur $\h_c^i(X,\sF)$).}
\end{equation}

Pour le voir, appliquons (\ref{II:1-4}) au cas où $Y_0$ est un point: 
$Y_0=\spec(\dF_q)$. On trouve que $\h_c^i(X,\sF)$ est la fibre, au point 
géométrique $\spec(\dF)$ de $Y_0$, de $\R^i f_! \sF_0$, et que $F^*$ est 
la fibre de la correspondance de Frobenius de $\R^i f_!\sF_0$. Soit 
$[\R^i f_! \sF_0]$ comme en (\ref{II:1-2}). On a 
$\h_c^i(X,\sF) = [\R^i f_!\sF_0](\dF)$ et 
\begin{enumerate}[\indent a)]
  \item $\varphi$ agit par transport de structure, via son action sur $\dF$, 
  \item $F^*$ est l'inverse de 
    $F:[\R^i f_!\sF_0](\dF) \to [\R^i f_!\sF_0](\dF)$. 
\end{enumerate}

On conclut par l'identité $F=\varphi$ de (\ref{II:1-1}). 

Un des intérêts du point de vue galoisien est qu'il permet de raisonner par 
transport de structure. 










\section{\texorpdfstring{$\dQ_\ell$}{Ql}-faisceaux}\label{II:2}

La notion de $\dQ_\ell$-faisceaux est développée en détail dans 
\cite[V,VI]{sga5}. Nous n'utiliserons que les définitions et résultats 
ci-dessous. 





\begin{definition_}\label{II:2-1}
Soit $X$ un schéma noethérien. Un $\dZ_\ell$-faisceau $\sF$ sur $X$ est un 
système projectif de faisceaux $\sF_n$ ($n\geqslant 0$), avec $\sF_n$ un 
faisceau de $\dZ/\ell^{n+1}$-modules constructibles \cite[IX.2]{sga4}, et tel que 
le morphisme de transition $\sF_n\to\sF_{n-1}$ se factorise par un isomorphe 
\[
  \sF_n\otimes_{\dZ/\ell^{n+1}} \dZ/\ell^n \iso \sF_{n-1}\text{.}
\]
On dit que $\sF$ est \emph{lisse} si les $\sF_n$ sont localement constants. 
\end{definition_}

La terminologie de \cite{sga5} est différents, on y dit ``$\dZ_\ell$-faisceau 
constructible'' pour ``$\dZ_\ell$-faisceau,'' et ``constant tordu 
constructible'' pour ``lisse.'' 

De même, ci-dessous, pour les $\dQ_\ell$-faisceaux. 





\subsection{}\label{II:2-2}

Si $k$ est un corps séparablement clos, un faisceau de $\dZ/\ell^n$-modules 
sur $\spec(k)$ s'identifie à un $\dZ/\ell^n$-module, et le foncteur 
$\sF\mapsto \varprojlim \sF_n$ identifie les $\dZ_\ell$-faisceaux sur 
$\spec(k)$ sux $\dZ_\ell$-modules de type fini. Ceci justifie le définition 
suivante: la \emph{fibre} du $\dZ_\ell$-faisceau $\sF$ en le point 
géométrique $x$ de $X$ est le $\dZ_\ell$-module $\varprojlim (\sF_n)_x$. 





\subsection{}\label{II:2-3}

Supposons $X$ connexe, et soit $x$ un point géométrique de $X$. On sait que 
le foncteur ``fibre en $x$'' est une équivalence de la catégorie des 
faisceaux localement constants constructibles de $\dZ/\ell^n$-modules avec la 
catégorie des $\dZ/\ell^n$-modules de type fini munis d'une action continue 
de $\pi_1(X,x)$. Par passage à la limite, on en déduit la proposition 
suivante. 





\begin{proposition_}\label{II:2-4}
Sous les hypothèses ci-dessus, le foncteur ``fibre en $x$'' est une 
équivalence de la catégorie des $\dZ_\ell$-faisceaux lisse sur $X$ avec 
celle des $\dZ_\ell$-modules de type fini munis d'une action continue de 
$\pi_1(X,x)$. 
\end{proposition_}





\begin{proposition_}\label{II:2-5}
Soit $\sF$ un $\dZ_\ell$-faisceau sur un schéma noethérien $X$. Il existe 
une partition finie de $X$ en parties localement fermées $X_i$ telles que 
les $\sF|X_i$ soient lisses.
\end{proposition_}

Posons $\gr_\ell^n\sF=\ell^n\sF_m/\ell^{m+1}\sF_m$ pour $m\geqslant n$, et 
$\gr_\ell\sF=\bigoplus \gr_\ell^n\sF$. C'est un faisceau de modules sur 
l'anneau noethérien $\gr_\ell\dZ=\dZ/\ell[t]$, gradué de $\dZ$ pour la 
filtration $\ell$-adique. C'est un quotient de 
$\gr_\ell^0\sF \otimes_{\dZ/\ell} \dZ/\ell[t]$, donc un faisceau de 
$\gr_\ell\dZ$-modules constructible \cite[IX.2]{sga4}. et il existe une partition 
finie de $X$ en parties localement fermées $X_i$, telle que 
$\gr_\ell \sF|X_i$ sont localement constant. Chacun des faisceaux 
$\gr_\ell^n\sF|X_i$ est alors localement constant ainsi que les $\sF_m|X_i$, 
qui en sont des extensions successives. 





\subsection{}\label{II:2-6}

Si $\sF$ et $\sG$ sont deux $\dZ_\ell$-faisceaux sur $X$, on pose 
$\hom(\sF,\sG) = \varprojlim(\sF_n,\sG_n)$; c'est un $\dZ_\ell$-module. 
Puisque $\hom(\sF_n,\sG_n) = \hom(\sF_m,\sG_n)$ pour $m\geqslant n$, on a 
encore 
\[
  \hom(\sF,\sG) = \varprojlim_n \varinjlim_m \hom(\sF_m,\sG_n) \text{,}
\]
de sorte que le foncteur $\sF\mapsto \text{``}\varprojlim\text{''}\sF_n$ est 
pleinement fidèle, de la catégorie des $\dZ_\ell$-faisceaux dans celle 
pro-faisceaux sur $X$ ($=$ pro-objets de la catégorie des faisceaux sur $X$). 
On l'utilisera pour identifier les $\dZ_\ell$-faisceaux à certains 
pro-faisceaux (ceux tels que chaque $\sF_n=\sF/\ell^{n+1}\sF$ soit un faisceau, 
et que $\sF\iso \text{``}\varprojlim\text{''}\sF_n$). On vérifie aisément 
qui si $\sF+\text{``}\varprojlim\text{''}\sF_n$, avec $\sF_n$ un faisceau 
constructible de $\dZ/\ell^n$-modules, pour que $\sF$ soit un 
$\dZ_\ell$-faisceau, il faut et il suffit que les systèmes projectifs (en $n$) 
$\sF_n/\ell^{k+1}\sF_n$ soient essentiellement constants: le système 
projectif des $\sF_k'=\varprojlim_n\sF_n/\ell^{k+1}\sF_n$ vérifie alors les 
conditions de (\ref{II:2-1}), et $\sF=\text{``}\varprojlim\text{''}\sF_k'$. 

Dans ce qui suit, nous abandonnons la notation $\sF_n$ de (\ref{II:2-1}); si 
$\sF$ est un $\dZ_\ell$-faisceau, le faisceau $\sF_n$ de (\ref{II:2-1}) sera 
noté $\sF\otimes\dZ/\ell^{n+1}$. 





\subsection{}\label{II:2-7}

Dans la catégorie abélienne des pro-faisceaux, la sous-catégorie des 
$\dZ_\ell$-faisceaux est stable par noyau, conoyau et extension. C'est claire 
pour les conoyaux. Pour les noyaux, si $f:\sF\to\sG$ est un morphisme de 
$\dZ_\ell$-faisceaux, on déduit aisément de (\ref{II:2-4}, \ref{II:2-5}) 
et de Artin-Rees que le système projectif des 
$\ker(f:\sF\otimes\dZ/\ell^n\to\sG\otimes\dZ/\ell^n)$ vérifie le critère 
donné et (\ref{II:2-6}) ci-dessus. Il existe même un entier $r$ tel que pour 
tout $n$ 
\[
  \ker(f)\otimes\dZ/\ell^n = \ker(f:\sF\otimes\dZ/\ell^{n+r}\to\sG\otimes\dZ/\ell^{n+r})\otimes\dZ/\ell^n \text{.}
\]
Le cas des extensions est laissé au lecteur. 





\subsection{}\label{II:2-8}

Un $\dZ_\ell$-faisceau $\sF$ est \emph{sans torsion} si 
$\ker(\ell:\sF\to\sF) = 0$. Chaque $\sF\otimes\dZ/\ell^n$ est alors plat sur 
$\dZ/\ell^n$ (i.e. à fibres des $\dZ/\ell^n$-modules libres de type fini). On 
déduit de (\ref{II:2-4}, \ref{II:2-5}) que pour tout $\dZ_\ell$-faisceau 
$\sF$ il existe un entier $n$ tel que $\sF/\ker(\ell^n)$ soit sans torsion. 





\subsection{}\label{II:2-9}

La catégorie abélienne des \emph{$\dQ_\ell$-faisceaux} sur $X$ est le 
quotient de celle des $\dZ_\ell$-faisceaux par la sous-catégorie de 
$\dZ_\ell$-faisceaux de torsion. En termes équivalents, ses objets sont les 
$\dZ_\ell$-faisceaux et, notant $\sF\otimes\dZ_\ell$ le $\dZ_\ell$-faisceaux 
$\sF$, vu comme objet de le catégorie des $\dQ_\ell$-faisceaux, on a 
\[
  \hom(\sF\otimes\dQ_\ell,\sG\otimes\dQ_\ell) = \hom(\sF,\sG)\otimes_{\dZ_\ell}\dQ_\ell \text{.}
\]

La \emph{fibre} d'un $\dQ_\ell$-faisceau $\sF\otimes\dQ_\ell$ en un point 
géométrique $x$ de $X$ est la $\dQ_\ell$-espace vectoriel de dimension 
finie $\sF_x\otimes_{\dZ_\ell}\dQ_\ell$. 





\subsection{}\label{II:2-10}

Soient $X$ un schéma séparé de type fini sur un corps algébriquement 
clos $k$ et $\sF$ un $\dQ_\ell$-faisceau constructible sur $X$. Soit $\sF'$ un 
$\dZ_\ell$-faisceau constructible tel que 
$\sF\sim \sF'\otimes_{\dZ_\ell}\dQ_\ell$. Nous verrons en (\ref{II:4-11}) que 
\[
  \left(\varprojlim \h_c^q(X,\sF'\otimes \dZ/\ell^n)\right)\otimes_{\dZ_\ell}\dQ_\ell
\]
est un $\dQ_\ell$-espace vectoriel de dimension finie. Il ne dépend que de 
$\sF'$, et se note $\h_c^q(X,\sF)$. La preuve n'utilise pas que $\ell$ soit 
premier à la caractéristique, mais seul ce cas nous intéresse. 





\subsection{}\label{II:2-11}

Dans cet exposé, tous les calculs importants se feront au niveau des 
$\dZ/\ell^n$-faisceaux, le passage aux $\dQ_\ell$-faisceaux ne se faisant qu'en 
tout dernier lieu. En pratique, il est toutefois souvent commode de travailler 
systématiquement avec des $\dQ_\ell$-faisceaux. Dans les exposés 
V et VI de \cite{sga5}, Jouenolou donne le formalisme qui rend cela possible. 
Un des résultats essentiels (VI 2.2) est que si $f:X\to Y$ est un morphisme 
séparé de type fini de schémas noethériens, et $\sF$ un 
$\dZ_\ell$-faisceau sur $Y$, alors, pour chaque $q$, le pro-faisceau 
$\text{``}\varprojlim\text{''}\R^q f_! (\sF\otimes\dZ/\ell^n)$ est un 
$\dZ_\ell$-faisceau sur $Y$. Il démontrer même une propriété de 
régularité plus forte (à la Artin-Rees) pour le système projectif des 
$\R^q f_!(\sF\otimes\dZ/\ell^n)$. 

Jouanolou se place même dans un cadre plus général, englobant le cas 
très utile des $E_\lambda$-faisceau, pour $E_\lambda$ une extension finie de 
$\dQ_\ell$. Ils sont définis comme l'ont été les $\dQ_\ell$-faisceaux, 
avec $\dQ_\ell$ remplacé par $E_\lambda$, $\dZ_\ell$ par l'anneau 
$\sO_\lambda$ de la valuation $\lambda$ et $\dZ/\ell^n$ par 
$\sO_\lambda/\pi^n$, pour $\pi$ une uniformisante. Il reviendrait au même de 
définir un $E_\lambda$-faisceau comme un $\dQ_\ell$-faisceau $\sF$, muni 
d'une homomorphisme $E_\lambda\to\operatorname{End}(\sF)$. 










\section{Formule des traces et fonctions \texorpdfstring{$L$}{L}}\label{II:3}

\emph{On utilise les notations du \S\ref{II:1}, et $\ell$ est un nombre premier 
$\ne p$.}

Soient $X_0$ un schéma séparé de type fini sur $\dF_q$ ($q=p^f$) et 
$\sF_0$ un $\dQ_\ell$-faisceau constructible sur $X_0$. L'interprétation 
cohomologique de Grothendieck des fonctions $L$ est le théorème suivant: 





\begin{theorem_}\label{II:3-1}
$L(X_0,\sF_0) = \prod_i \det\left(1-F^* t^f, \h_c^i(X,\sF)\right)^{(-1)^{i+1}}$. 
\end{theorem_}

Il est obtenu comme conséquence de la formule des traces:





\begin{theorem_}\label{II:3-2}
Pour tout entier $n$, 
$\sum_{x\in X^{F^n}} \tr\left({F^n}^*,\sF_x\right) = \sum_i (-1)^i \tr\left({F^n}^*,\h_c^i(X,\sF)\right)$. 
\end{theorem_}





Rappelons brièvement la méthode classique pour déduire (\ref{II:3-1}) 
de (\ref{II:3-2}). Posons $T=t^f$; les deux membres sont des séries formelles 
en $T$, de terme constant $1$. Puisque $\dQ_\ell$ est de caractéristique 
$0$, il suffit de prouver que les dérivées logarithmiques 
$T\frac{d}{dT}\log$ des deux membres sont égales. 

Pour calculer ces dérivées logarithmiques, nous utiliserons la formule 
\[
  T \frac{d}{dT}\log\det(1-f T)^{-1} = \sum_{n>0} \tr(f^n) T^n \text{,}
\]
% NOTE: the following lines are not typesetting correctly
dont la démonstration (dans un cadre un peu plus général) est rappelée 
ci-dessous. Au premier nombre, on trouva 
\begin{align*}
  T\frac{d}{dT}\log L(X_0,\sF_0) 
    &= \sum_{x\in |X_0|} \sum_n [k(x):\dF_q] \tr\left(F_x^n,\sF\right) T^{n\cdot[k(x):\dF_q]} \\
    &= \sum_{n>0} T^n \sum_{x\in X^{F^n}} \tr\left({F^n}^*,\sF_x\right)
\end{align*}
et su second membre 
\[
  \sum_{n>0} T^n \sum (-1)^i \tr\left({F^n}^*,\h_c^i(X,\sF)\right) \text{ ; }
\]
on compare terme à terme. 





\begin{proposition_}\label{II:3-3}
Soient $A$ un anneau commutatif, $M$ un $A$-module projectif de type fini et 
$f$ un endomorphisme de $M$. Pour tout série formelle inversible $Q$, on pose 
$\frac{d}{dT}\log Q = Q^{-1}\frac{d}{dT}Q$. On a 
\begin{equation}\label{II:eq:3-3-1}
  T\frac{d}{dT}\log\det(1-f T)^{-1} = \sum_{n>0} \tr(f^n) T^n \text{.}
\end{equation}
\end{proposition_}

Soit $M'$ un $A$-module tel que $M\oplus M'$ soit libre de rang fini. 
Remplacer $(M,f)$ par $(M\oplus M',f\oplus 0)$ ne change pas les deux membres 
de (\ref{II:3-3}), et nous ramène au cas où $M$ est libre. Pour $M=A^d$, 
\eqref{II:eq:3-3-1} est une identité algébrique portant sur les coefficients 
$f_i^j$ de $f$. Par le principe de prolongement des identités algébriques, 
il suffit de la vérifier pour $A$ un corps algébriquement clos de 
caractéristique $0$. Les deux membres étant additifs en $M$ dans une 
extension $0\to M'\to M\to M''\to 0$, il suffit même de ne traiter que le cas 
où $M$ est de rang $1$. Si $f$ est la multiplication par $a$, 
\eqref{II:eq:3-3-1} se réduit alors à la formule 
\[
  T\frac{d}{dT}\log\left((1-a T)^{-1}\right) = \sum_{n>0} a^n T^n \text{.}
\]





\begin{corollary_}\label{II:3-4}
Soient $n$ un entier, et $f^{(n)}$ l'endomorphisme 
$(x_1,\dotsc,x_n)\mapsto \left(f(x_n),x_1,\dotsc,x_{n-1}\right)$ de $M^n$. 
On a 
\begin{equation}\label{II:eq:3-4-1}
  \det(1-f T^n) = \det(1-f^{(n)} T) \text{.}
\end{equation}
\end{corollary_}

Procédant comme en (\ref{II:3-3}), on se ramène à supposer que $A$ est de 
caractéristique $0$. Il suffit alors de vérifier que les dérivées 
logarithmiques $-T\frac{d}{dT}\log$ des deux membres sont égales. 
\begin{align*}
  -T\frac{d}{dT}\log\det(1-f T^n) &= - n T^n \frac{d}{dT^n} \log\det(1-f T^n) = n \sum_{m>0} \tr(f^m) T^{n m} \text{,} \\
  -T \frac{d}{dT}\log\det\left(1-f^{(n)} T\right) &= \sum_{m>0} \tr\left(f^{(n) m}\right) T^m \text{,}
\end{align*}
et on observe que 
\begin{align*}
  \tr(f^{(n) m}) &= 0 \qquad \text{si $n\nmid m$} \\
  \tr(f^{(n) n m}) &= n \tr(f^m) \text{.}
\end{align*}





\subsection{Remarque}\label{II:3-5}

Pour $A=\dQ_\ell$, \eqref{II:eq:3-4-1} est le cas particulier 
$X_0=\spec(\dF_{q^n})$ de (\ref{II:3-1}). Utilisant cette identité, il est 
facile de vérifier directement que le second membre de (\ref{II:3-1}) est 
indépendant de $q$, et du morphisme structural $X_0\to\spec(\dF_q)$. 





\subsection{Remarque}\label{II:3-6}

Si $j:X_0\hookrightarrow \bar X_0$ est une compactification de $X_0$, on a 
$L(X_0,\sF_0) = L(\bar X_0, j_!\sF_0)$ et 
$\h_c^i(X,\sF) = \h^i(\bar X,j_!\sF)$. Ceci ramène (\ref{II:3-1}) au cas 
propre, et explique l'usage de la cohomologie à supports propres. Pour $X$ 
propre, (\ref{II:3-2}) a la forme d'une formule des traces de Lefschetz. On 
notera que les termes locaux $\tr(F^{n *},\sF_x)$ ($x\in X^{F^n}$) ne 
dépendent que de la fibre de $\sF$ en $x$, et ne sont pas affectés de 
multiplicité. Pour $X$ et $\sF$ lisses, cela correspond au fait que le graphe 
de $F$ est transverse à la diagonale (puisque $d F = 0$). 





\subsection{Remarque}\label{II:3-7}

Si on admet le formalisme de $\dQ_\ell$-faisceaux (cf. \ref{II:2-11}; il faut 
disposer de la suite spectrale de Leray et du théorème de changement de 
base pour les images directes supérieures $\R^q f_!$), il est facile de 
ramener la preuve de (\ref{II:3-1}, \ref{II:3-2}), au cas où $X_0$ est 
une courbe lisse et où $\sF_0$ est lisse. Ceci est clairement expliqué 
dans \cite[\S 5]{gr95} (pour \ref{II:3-1}; \ref{II:3-2} se traite de même). 





\subsection{}\label{II:3-8}

On dispose de deux méthodes pour prouver (\ref{II:3-2}). 


\paragraph{Lefschetz-Verdier}
Si $X_0$ est propre (cf. \ref{II:3-6}), la formule des traces générale de 
Lefschetz-Verdier permet d'exprimer le second membre de (\ref{II:3-2}) comme 
une somme de termes locaux, un pour chaque point de $X^{F^n}$. Dans la version 
originale de \cite{sga5}, cette formule n'était prouvée que modulo la 
résolution des singularités. Le lecteur trouvera une preuve inconditionnelle 
dans la version définitive. Dans le cas des courbes, cas auquel on peut se 
réduire (\ref{II:3-7}), les ingrédients de la démonstration étaient 
d'ailleurs tous disponibles. 

Pour déduire (\ref{II:3-2}) de la formule de Lefschetz-Verdier, il faut 
pouvoir en calculer les termes locaux. Pour une courbe et l'endomorphisme de 
Frobenius, cela avait été fait par Artin et Verdier \cite{ve67}. 


\paragraph{Nielson-Wecken}
Un méthode inspirée dex travaux de Nielsen et Wecken permet de ramener 
(\ref{II:3-2}), à un cas particulier prouvé par Weil; c'est ce qui sera 
expliqué dans les paragraphes suivants. 










\section{Réduction à des théorèmes en \texorpdfstring{$\dZ/\ell^n$}{Z/l n}-cohomologie}\label{II:4}

Nous prouverons (\ref{II:3-2}) par passage à la limite à partir d'une 
théorème analogue en $\dZ/\ell^n$-cohomologie. L'énoncer présente la 
difficulté suivante. Soient $X_0$ un schéma séparé de type fini sur 
$\dF_q$, et $\sF_0$ un faisceau plat et constructible de $\dZ/\ell^n$-modules. 
Ses fibres sont des $\dZ/\ell^n$-modules libres de type fini, et le membre de 
gauche de (\ref{II:3-2}) a donc un sens (c'est un élément de $\dZ/\ell^n$). 
Par contre, les $\h_c^i(X,\sF)$ ne seront en général pas libres, de sorte 
que le membre de droite n'est pas défini. Obvier à cette difficulté est 
l'un des usages essentiels faits par Grothendieck de la théorie des 
catégories dérivées.





\subsection{}\label{II:4-1}

Pour la définition de la catégorie dérivée $\D(\cA)$ d'une catégorie 
abélienne $\cA$, et celle des sous-catégorie $\D^+(\cA)$, $\D^-(\cA)$, 
$\D^b(\cA)$, je renvoie au texte de Verdier dans ce volume. Pour $\Lambda$ 
un anneau, on écrit $\D(\Lambda)$ au lieu de 
$\D(\text{catégorie des $\Lambda$-modules à gauche})$; pour $X$ un site, on 
écrit $\D(X,\Lambda)$ au lieu de 
$\D($catégorie des faisceaux de $\Lambda$-modules à gauche$)$. De 
même pour $\D^-$, etc\ldots

Le signe $\R$ (resp. $\mathsf{L}$) indique un foncteur dérivé droit (resp. 
gauche). Par exemple, $\lotimes$ indique le foncteur dérivé du produit 
tensoriel: pour $K$ dans $\D^-(\Lambda^\circ)$ et $L$ dans $\D^-(\Lambda)$ 
($\Lambda^\circ$ anneau opposé à $\Lambda$), $K\lotimes L$ se calcule 
comme suit: on prend des quasi-isomorphismes $K'\to K$, $L'\to L$, avec 
$K'$ et $L$ bornés supérieurement et l'un d'eux à composantes plates, et 
le simple complexe $s(K'\otimes_\Lambda L)$ associé au complexe double 
$K'\otimes_\Lambda L'$. 





\subsection{}\label{II:4-2}

Même pour traiter le seul cas de la $\dZ/\ell^n$-cohomologie, nous aurons 
besoin de la théorie des traces d'endomorphismes de modules sur des anneaux 
non nécessairement commutatifs (des algèbres de groupes finis 
$\dZ/\ell^n[G]$). Ces traces ont été introduits par Stallings et Hattori 
\cite{ba76}. 

Soit $\Lambda$ un anneau (à unité, mais non nécessairement commutatif). 
On par $\Lambda^\natural$ le quotient du groupe additif de $\Lambda$ par le 
sous-groupe engendré par les commutateurs $a b - b a$. Si $f$ est un 
endomorphisme du $\Lambda$-module à gauche libre $\Lambda^n$, de matrice 
$f_i^j$, on note $\tr(f)$ l'image de $\sum f_i^f$ dans $\Lambda^\natural$. 
Si $f$ et $g$ sont deux homomorphismes 
$f:\Lambda^n \leftrightarrows \Lambda^m : g$, on vérifie aussitôt que 
$\tr(f g) = \tr(g f)$. 

Soit $f$ un endomorphisme d'un $\Lambda$-module projectif de type fini $P$. 
Écrivons $P$ comme facteur direct d'une module libre $\Lambda^n$: cela signifie 
choisir un module $P'$, et un isomorphisme $\alpha:P\oplus P'\iso \Lambda^n$, 
soit encore choisir un homomorphisme $a:P\to \Lambda^n$, et une rétraction 
$b:\Lambda^n\to P$ (prendre $P=\ker(b)$). Soit 
$f'=\alpha(f\oplus 0)\alpha^{-1} =af b$. On pose $\tr(f) = \tr(f')$. Cet 
élément de $\Lambda^\natural$ ne dépend que de $f$: si un autre choix 
$c:P\leftrightarrows \Lambda^m:d$ est fait, on a $a=a(dc)(ba)$, d'où 
\[
 \tr(a f b) = \tr(adcbafb) = \tr(cbafbad) = \tr(cfd) \text{.}
\]

Si $f$ est un endomorphisme d'un $\Lambda$-module projectif de type fini 
$\dZ/2$-gradué, de composantes $f_i^j:P^j\to P^i$, on pose 
\[
  \tr(f) = \tr(f_0^0) - \tr(f_1^1) \text{.}
\]
S'il y a risque d'ambiguïté, on écrira plutôt $\tr(f;P)$. Si les 
homomorphismes $f:P\leftrightarrows Q:g$ sont homogènes, on vérifie 
aussitôt par réduction au cas libre que 
\begin{equation}\label{II:eq:4-2-1}
  \tr(f g) = \tr\left((-1)^{\deg f \deg g} g f\right) \text{.}
\end{equation}

Si $f$ est un endomorphisme d'un $\Lambda$-module $\dZ/2$-gradué filtré 
$(P,F)$, de filtration finie, compatible à la graduation, et que les 
$\gr_F^i(P)$ sont projectifs de type fini, alors $P$ est projectif de type 
fini, et 
\begin{equation}\label{II:eq:4-2-2}
  \tr(f;P) = \sum \tr\left(f; \gr_F^i(P) \right)
\end{equation}
(scinder la filtration pour se ramener au cas d'une somme directe). 





\subsection{}\label{II:4-3}

Un module $\dZ$-gradué sera considéré comme $\dZ/2$-gradué par la 
parité du degré. En particulier, si $f$ est un endomorphisme d'un complexe 
borné de $\Lambda$-modules projectifs de type fini, on pose 
\begin{equation}\label{II:eq:4-3-1}
  \tr(f) = \sum (-1)^i \tr(f^i) \text{.}
\end{equation}

D'après \eqref{II:eq:4-2-1}, si $f$ est homotope à zéro, $\tr(f) = 0$: 
\begin{equation}\label{II:eq:4-3-2}
  f = d H + h D \Rightarrow \tr(f) = 0 \text{.}
\end{equation}

Soit $\mathsf{K}_\text{parf}(\Lambda)$ la catégorie d'objets les complexes 
bornés de $\Lambda$-modules projectifs de type fini, et de flèches les 
morphismes de complexes pris à homotopie près. Le foncteur 
$\mathsf{K}_\text{parf}(\Lambda)\to\D(\Lambda)$ est pleinement fidèle. On note 
$\mathsf{D}_\text{parf}(\Lambda)$ son image essentielle. Les formules 
\eqref{II:eq:4-3-1} \eqref{II:eq:4-3-2} permettent de définir $\tr(f)$ pour $f$ 
un endomorphisme de $K\in\ob \mathsf{D}_\text{parf}(\Lambda)$ (et cette trace ne 
dépend que de la classe d'isomorphe de $(K,f)$). 





\subsection{}\label{II:4-4}

Rappelons que la catégorie dérivée filtrée $\mathsf{DF}(\cA)$ d'une catégorie 
abélienne $\cA$ est la catégorie déduite de celle des complexes d'objets 
filtrés de filtration finie de $\cA$ en inversant les quasi-isomorphismes 
filtrés ($=$ morphismes de complexes filtrés induisant un quasi-isomorphisme 
sur le gradué associé). On dispose de foncteurs ``complexe sous-jacent'': 
$\mathsf{DF}(\cA)\to\D(\cA): (K,F)\mapsto K$ et ``$p$-ième composante du 
gradué'': $\mathsf{DF}(\cA)\to\D(\cA): (K,F)\mapsto \gr_F^p(K)$. 

Soit $\mathsf{KF}_\text{parf}(\Lambda)$ la catégorie d'objets les complexes 
bornés filtrés de filtration finie $(K,F)$ avec des $\gr_F^p(K^q)$ projectifs 
de type fini, et de flèches les morphismes de complexes respectant la 
filtration, pris à une homotopie respectant la filtration près. Le foncteur 
$\mathsf{KF}_\text{parf}(\Lambda)\to\mathsf{DF}(\Lambda)$ est pleinement fidèle; 
son image essentielle $\mathsf{DF}_\text{parf}(\Lambda)$ est presque tous nuls; 
si $(K,F)$ est dans $\mathsf{D}_\text{parf}(\Lambda)$, alors $K$ est dans 
$\mathsf{D}_\text{parf}(\Lambda)$. 

D'après \eqref{II:eq:4-2-2}, si $f$ est un endomorphisme de 
$(K,F)\in\ob\mathsf{DF}_\text{parf}(\Lambda)$, on a 
\begin{equation}\label{II:eq:4-4-1}
 \tr(f,K) = \sum_p \tr(f,\gr_F^p(K)) \text{.}
\end{equation}





\subsection{}\label{II:4-5}

Soient $\Lambda$ un anneau, et $K\in\ob\D^-(\Lambda)$. On dit que $K$ est de 
tor-dimension $\leqslant r$ si pour tout $\Lambda$-module à droite $N$, les 
hypertors $\h^i(N\lotimes_\Lambda K)$ sont nuls pour $i<-r$. Il est dit de 
tor-dimension finie si cette condition est vérifiée pour un $r$. 

On emploie la même terminologie pour un complexe de faisceaux 
$K\in\ob\D^-(X,\Lambda)$. 





\begin{lemma}\label{II:4-5-1} 
Soient $\Lambda$ un anneau noethérien à gauche, et $K\in\ob\D^-(\Lambda)$. 
Pour que $K$ soit dans $\mathsf{DF}_\text{parf}(\Lambda)$, il faut et il suffit 
qu'il soit de tor-dimension finie et que les $\h^i(K)$ soient de type fini.
\end{lemma}

Quitte à remplacer $K$ par un complexe isomorphe (dans $\D^-(\Lambda)$), on 
peut supposer $K$ borné supérieurement et à composantes libres de type 
fini (cf. \ref{II:4-7}, ci-dessous). Si $K$ est de tor-dimension $\leqslant r$, 
les $\h^i(K)$ sont nuls pour $i<-r$, $K^{-r}/\im(d)$ admet la résolution 
libre 
\[
  (\cdots \to K^{-r-1} \to K^{-r}) \to K^{-r}/\im(d) \text{,}
\]
et pour tout $\Lambda$-module à droite $N$, 
$\h^{-r-k}(N\lotimes_\Lambda K) = \h^{-r-k}(N\otimes_\Lambda K) 
= \tor_k(N,K^r/\im d)$ pour $k\geqslant 1$. 

Par hypothèse, ces groupes sont nuls; le module $K^r/\im d$ est donc plat de 
présentation finie, i.e. projectif de type fini. Le quotient 
\[
  0 \to K^r/\im(d) \to K^{r+1} \to K^{r+2}\to \cdots 
\]
de $K$ est quasi-isomorphe à $K$, et ceci montre que 
$K\in\ob\mathsf{DF}_\text{parf}(\Lambda)$. 





\begin{prop-def_}\label{II:4-6}
Soient $X$ un schéma noethérien, et $\Lambda$ un anneau noethérien à 
gauche. On note $\D_\textnormal{ctf}^b(X,\Lambda)$ la sous-catégorie de 
$\D^-(X,\Lambda)$ formée des complexes $\sK$ vérifient les conditions 
équivalentes suivantes:
\begin{enumerate}[\indent i)]
  \item $\sK$ est isomorphe (dans $\D^-(X,\Lambda)$) à un complexe borné e 
    faisceaux de $\Lambda$-modules plats et constructibles. 
  \item $\sK$ est de tor-dimension finie et les faisceaux $\h^i(K)$ sont 
    constructibles. 
\end{enumerate}
\end{prop-def_}

Même argument qu'en (\ref{II:4-5}), à partir du lemme suivant. 





\begin{lemma_}\label{II:4-7}
Si un complexe $\sK$ de faisceaux de $\Lambda$-modules est tel que les 
$\h^i(\sK))$ soient constructibles, et nuls pour $i$ grand, il existe un 
quasi-isomorphisme $\sK'\to \sK$, avec $\sK'$ borné supérieurement à 
composantes constructibles et plates.
\end{lemma_}

On construit $\sK'$ en se propageant de droite à gauche. Pour un $m$ tel que les 
$\h^i(\sK)$ ($i\geqslant m$) soient nuls, on prend ${\sK'}^i = 0$ pour 
$i\geqslant m$. Supposons ensuite déjà construits les ${\sK'}^i$ pour 
$i>n$:
\[\xymatrix{
  \cdots \ar[r] 
    & \sK^{n-1} \ar[r] 
    & \sK^n \ar[r] 
    & \sK^{n+1} \ar[r] 
    & \cdots \\
  & & & {\sK'}^{n+1} \ar[u] \ar[r] 
    & \cdots \text{,}
}\]
avec $\h^i(\sK')\iso \h^i(\sK)$ pour $i>n+1$, et $\ker(d)\to\h^{n+1}(\sK)$ surjectif 
en degré $n+1$. Définissons des faisceaux $\sA$ et $\sB$ par le diagramme 
cartésien 
\[\xymatrix{
  \sK^n \ar[r] 
    & \sK^n/\im(d) \ar[r] 
    & \ker(d) \\
  \sA \ar[u] \ar[r]^u 
    & \sB \ar[u]\ar[r] 
    & \ker(d) \ar[u] \text{.}
}\]

Alors, $\sB$ est constructible, $u$ est surjectif, et pour avancer d'un pas, il 
suffit de construire $K^n$ plat et constructible et $v:\sK^n\to \sA$ tel que u
$u v$ soit surjectif. Que ce soit possible est garanti par le lemme suivant. 





\begin{lemma_}\label{II:4-8}
Soit $v:\sA\to \sB$ un épimorphisme de faisceaux de $\Lambda$-modules, avec 
$\sB$ constructible. Il existe alors $u:\sK\to \sA$ avec $\sK$ plat et 
constructible et $v u$ un épimorphisme. 
\end{lemma_}

Le faisceau $\sA$ est quotient d'une somme de faisceaux de la forme 
$\varphi_! \Lambda$, pour $\varphi:U\to X$ étale. Puisque $\sB$ est noethérien 
(\cite[IX.2.10]{sga4}), une somme finie de ceux-ci s'envoie déjà sur $\sB$, 
d'où le lemme. 





\begin{theorem_}\label{II:4-9}
Soit $f:X\to Y$ un morphisme séparé de type fini de schémas noethériens. 
Si $\sK\in\ob\D_\textnormal{ctf}^b(X,\Lambda)$, alors 
$\R f_!\sK\in \ob\D_\textnormal{ctf}^b(Y,\Lambda)$. 
\end{theorem_}

Rappelons que si $j:X\hookrightarrow \bar X \xrightarrow{\bar f} Y$ est une 
compactification relative de $X/Y$, on a par définition 
$\R f_! = \R \bar f_* \circ j_!$, où $\R \bar f_*$ est le foncteur dérivé 
du foncteur $\bar f_*$. Cette formule ramène la preuve de (\ref{II:4-9}) 
au cas où $f$ est propre. Pour $f$ propre, $f_*$ est de dimension 
cohomologique finie; ceci permet de définir $\R f_*$ sur la catégorie 
dérivée tout entière, et aussi comme un foncteur 
\[
  \R f_* : \D^-(X,\Lambda) \to \D^-(Y,\Lambda) \text{.}
\]
Par composition, on définit de même en général 
\[
  \R f_! : \D^-(X,\Lambda) \to \D^-(Y,\Lambda) \text{.}
\]

Pour tout $\Lambda$-module à droite $N$, on a (preuve donnée ci-dessous) 
\begin{equation}\label{II:eq:4-9-1}
  N\lotimes \R f_! \sK \iso \R f_! (N\lotimes \sK) \text{.}
\end{equation}

Déduisons (\ref{II:4-9}) de \eqref{II:eq:4-9-1}. 
\begin{enumerate}[\indent a)]
  \item La suite spectrale 
    \[
      E_2^{p q} = \R^p f_! \h^q(\sK) \Rightarrow \h^{p+q} \R f_! \sK \text{.}
    \]
    et le théorème de finitude pour $\R f_!$, montrent que 
    $\R f_!\sK\in \ob \D^b(X,\Lambda)$ et est à cohomologie constructible. 
    Reste à prouver la tor-dimension finie (\ref{II:4-6}). 
  \item Si $\sK$ est de tor-dimension $\leqslant -r$, il en est de même pour 
    $\R f_! \sK$: on a $\h^i(N\lotimes \R f_! \sK) = \h^i(\R f_!(N\lotimes \sK))$, 
    et, dans la suite spectrale ci-dessus pour $N\lotimes \sK$, les 
    $E_2^{p q}$ sont nuls pour $q<r-s$ fortiori pour $p+q<r$. 
\end{enumerate}

Prouvons \eqref{II:eq:4-9-1}, en restant au niveau des complexes. 
\begin{enumerate}[\indent a)]
  \item Utilisant la formule de définition $\R f_! = \R f_*\circ j_!$, on se 
    ramène à supposer $f$ propre. 
  \item Représentons $\sK$ par un complexe borné 
    supérieurement à composantes acyclique pour $f_*:\R^p f_* \sK^q = 0$ 
    pour $p>0$. On a alors $\R f_* K \sim f_* \sK$. Il n'est possible de 
    travailler ainsi avec des complexes bornés supérieurement que parce que 
    $f_*$ est de dimension cohomologique finie. 
  \item Si $L$ un $\Lambda$-module à droite libre, on a 
    $\R^p f_*(L\otimes \sK^q) = L\otimes \R^p f_* \sK^q$: c'est trivial pour 
    $\Lambda$ libre de type fini, et le cas général s'en déduit par 
    passage à la limite. Le faisceau $L\otimes \sK^q$ est acyclique pour 
    $f_*$, et $f_*(L\otimes \sK^q) = L\otimes f_* \sK^q$. 
  \item Soit $N_\bullet$ un résolution libre de $N$. On a 
    $N\lotimes \sK\sim s(N_\bullet\otimes \sK)$ (complexe simple associé au 
    complexe double $N_\bullet\otimes \sK$). D'après c), 
    \[
      \R f_*(N\lotimes \sK) \sim \R f_* (s(N_\bullet\otimes \sK)) \sim f_*(s(N_\bullet\otimes \sK)) \sim s(N_\bullet\otimes f_* \sK) \sim N\lotimes \R f_* \sK \text{,}
    \]
    soit la formule annoncée. 
\end{enumerate}

Lorsque $Y$ est le spectre d'un corps séparablement clos, le foncteur $\R f_!$ 
s'identifie à un foncteur noté 
$\R \Gamma_c : \D_\text{ctf}^b(X,\Lambda)\to\D_\text{parf}(\Lambda)$. 





\begin{theorem_}\label{II:4-10}
Avec les notations du \S\ref{II:1}, soient $X_0$ un schéma séparé de type 
fini sur $\dF_q$, et $\Lambda$ un anneau noethérien de torsion, tué par un 
entier premier à $p$. Soit $\sK_0\in\ob\D_\textnormal{ctf}^b(X_0,\Lambda)$. 
Avec la définition (\ref{II:4-3}) de traces, on a pour tout $N$ 
\begin{equation}\label{II:eq:4-10}
  \sum_{x\in X^{F^n}} \tr\left({F^n}^*,\sK_x\right) = \tr\left({F^n}^*,\R\Gamma_c(X,\sK)\right) \text{.}
\end{equation}
\end{theorem_}

Dans la fin de ce \S, nous allons déduire \ref{II:3-2} de \ref{II:4-10}, et 
prouver \ref{II:2-10}. 





\subsection{Preuve de \texorpdfstring{\ref{II:2-10}}{2.10}}\label{II:4-11}

Soit $X$ un schéma séparé de type fini sur un corps algébriquement clos 
$k$, et soit $\sG$ un $\dQ_\ell$-faisceau sur $X$. D'après \ref{II:2-8} on 
peut l'écrire sous la forme $\sG=\sF\otimes\dQ_\ell$, avec $\sF$ un 
$\dZ_\ell$-faisceau sans torsion. Posons 
\[
  K_n = \R \Gamma_c(X,\sF\otimes\dZ/\ell^{n+1})\in\D(\dZ/\ell^{n+1}) \text{.}
\]
D'après \ref{II:4-9}, on a $K_n\in\ob\D_\text{parf}(\dZ/\ell^{n+1})$, et il 
résulte de \eqref{II:eq:4-9-1} que la flèche naturelle 
\begin{equation}\label{II:eq:4-11-1}
\xymatrix{
  K_n\lotimes_{\dZ/\ell^{n+1}} \dZ/\ell^n \ar[r] & K_{n+1} \text{,}
}
\end{equation}
est un isomorphisme. Le problème de la définition de cette flèche est 
discuté en \ref{II:4-12}. 

On peut maintenant invoquer \cite[XI.3.3]{sga5} (p. 31, 1-2 à la fin; ce 
passage est indépendant du reste de \cite{sga5}) pour réaliser chaque 
$K_n$ par un complexe borné de $\dZ/\ell^{n+1}$-modules libres de type fini, 
et les flèches \eqref{II:eq:4-11-1} par des \emph{isomorphismes de complexes} 
\[\xymatrix{
  K_n\otimes \dZ/\ell^n \ar[r]^-\sim 
    & K_{n-1} \text{.}
}\]
Soit $K$ le complexe $\varprojlim K_n$. C'est un complexe borné de 
$\dZ_\ell$-modules libres, et $K_n\simeq K\otimes_{\dZ_\ell}\dZ/\ell^{n+1}$. On 
a 
\[
  \h^i(K) = \varprojlim \h^i(K_n) \text{,}
\]
(tout système projectif de groupes finis vérifie en effet la condition de 
Mittag-Leffler), et $\h^i(K)$ est un $\dZ_\ell$-module de type fini. On a 
\[
  \h_c^i(X,\sG) = \left(\varprojlim \h^i(K_n) \right)\otimes_{\dZ_\ell} \dQ_\ell \text{:}
\]
c'est un $\dQ_\ell$-espace vectoriel de dimension finie. 





\subsection{La flèche \texorpdfstring{\eqref{II:eq:4-11-1}}{(4.11.1)}}\label{II:4-12}

Pour \emph{définir} la flèche \eqref{II:eq:4-11-1}, on ne peut pas 
procéder comme en \ref{II:4-9}, i.e. résoudre $\dZ/\ell^n$ par un complexe 
de $\dZ/\ell^{n+1}$-modules libres, puisqu'on veut construire on isomorphisme 
dans $\D_\text{parf}(\dZ/\ell^n)$. La flèche voulue est un cas particulier de 
la flèche d'extension des scalaires suivante. 

(*) Soit $\Lambda\to\Lambda'$ un homomorphisme d'anneaux noethériens de 
torsion. Pour $X$ un schéma séparé de type fini sur $k$ algébriquement 
clos, et $\sK\in\ob\D_\text{ctf}(X,\Lambda)$, on a un isomorphisme 
\[\xymatrix{
  \R\Gamma_c(X,\sK)\lotimes_\Lambda \Lambda' \ar[r]^-\sim & \R \Gamma_c(X,\sK\lotimes_\Lambda \Lambda') \text{.}
}\]

Des flèches plus générales sont construits dans \cite[XVII 4.2.12]{sga4}. La 
m\'
ethode est la suivante: 
\begin{enumerate}[\indent a)]
  \item La définition de $\R\Gamma_c$ est la formule 
    $j_!(\sK\otimes_\Lambda\Lambda') = (j_! \sK)\otimes_\Lambda\Lambda'$ pour une 
    immersion ouverte nous ramènent au cas où $X$ est propre. 
  \item Pour $X$ propre, il s'agit de remplacer $\sK$ par un complexe borné à 
    composantes acycliques pour le foncteur $\Gamma$, et de fibre en tout point 
    géométrique (ou seulement un tout point fermé) homotope à un 
    complexe de $\Lambda$-modules plats. Pour un tel complexe, on a 
    $\Gamma(X,\sK)\sim \R\Gamma(X,\sK)$ et 
    $K\otimes_\Lambda\Lambda'\sim \sK\lotimes_\Lambda\Lambda'$; la flèche 
    \eqref{II:eq:4-11-1} est définie comme la composition 
    \[
      \R\Gamma(X,\sK)\lotimes_\Lambda \Lambda' \to \Gamma(X,\sK)\otimes_\Lambda\Lambda' \to \Gamma(X,\sK\otimes_\Lambda\Lambda') \xleftarrow\sim \R\Gamma(X,\sK\lotimes_\Lambda\Lambda') \text{.}
    \]
  \item Il reste à montrer qu'il est possible de remplacer $\sK$ par un 
    complexe du type voulu \cite[XVII.4.2.10]{sga4}. On remplace d'abord $\sK$ par 
    un complexe borné à composantes plates (\ref{II:4-6}), puis on en prend 
    une résolution flasque canonique tronquée (ceci ne modifie pas le type 
    d'homotopie des complexes fibres, et on tronque assez loin pour que 
    vérifiée la condition d'acyclicité pour $\Gamma$ (i.e. au-delà 
    de la dimension cohomologique de $\Gamma$)). 
\end{enumerate}





\subsection{Prouve de \texorpdfstring{(\ref{II:3-2})}{(3.2)}}\label{II:4-13}

Pour simplifier les notations, nous ne traiterons que la cas $n=1$ de 
(\ref{II:3-2}). Le cas général s'en déduit en remplaçant ci-dessous 
$F$ par $F^n$. 

Soient donc $X_0$ séparé de type fini sur $\dF_q$, et $\sG_0$ un 
$\dQ_\ell$-faisceau sur $X_0$. On a $\sG_0=\sF_0\otimes\dQ_\ell$, pour 
$\dZ_\ell$-faisceau sans torsion convenable $\sF_0$. Soit 
$K_n=\R\Gamma_c(X,\sF\otimes\dZ/\ell^n)$. Les morphismes de Frobenius induisent 
des morphismes 
\[
  F^* : K_n\to K_n \qquad \text{(dans $\D_\text{parf}(\dZ/\ell^{n+1})$),}
\]
qui se déduisent les uns des autres via les isomorphismes 
\eqref{II:eq:4-11-1}. 

Réalisons comme en (\ref{II:4-11}) les $K_n$ et les isomorphismes 
\eqref{II:eq:4-11-1} su niveau des complexes. La passage déjà cité de 
\cite[XV.3.3]{sga5} permet de réaliser les morphismes $F^*$ par des 
endomorphismes de complexes, encore notés $F^*$, qui se réduisent les uns 
sur les autres. On note encore $F^*$ leur limite projective, et les 
endomorphismes qui s'en déduisent. Puisque 
$\h_c^i(X,\sG) = \h^i(K)\otimes\dQ_\ell = \h^i(K\otimes\dQ_\ell)$, on a (avec 
la convention de signe de (?.1) 
\[
  \tr\left(F^*, \h_c^\bullet(X,\sG)\right) = \tr\left(F^*,K^\bullet\otimes\dQ_\ell\right) = \varprojlim_n \tr\left(F^*,K_n^\bullet\right) = \varprojlim_n \tr\left(F^*,\R\Gamma_c(X,\sF\otimes\dZ/\ell^{n+1})\right) \text{.}
\]
Appliquons (\ref{II:4-10}) à $\sF_0\otimes\dZ/\ell^{n+1}$ (vu comme complexe 
réduit su degré $0$), pour $\Lambda=\dZ/\ell^{n+1}$. On trouve que 
\begin{align*}
  \tr\left(F^*,\R\Gamma_c(X,\sF\otimes\dZ/\ell^{n+1})\right) 
    &= \sum_{x\in X^F} \tr\left(F^*,\sF_x\otimes\dZ/\ell^{n+1}\right) \\
    &= \sum_{x\in X^F} \tr\left(F^*,\sG_x\right) \mod \ell^{n+1} \text{,}
\end{align*}
et (\ref{II:3-2}), en résulte par passage à limite. 










\section{La méthode de Nielsen-Wecken}\label{II:5}

Dans ce chapitre, nous prouvons deux cas particuliers de (\ref{II:4-10}). Le 
cas général sera considéré au chapitre suivant. 





\begin{lemma_}\label{II:5-1}
La théorème (\ref{II:4-10}) est vrai pour $X_0$ de dimension $0$. 
\end{lemma_}

En dimension $0$, la formule \eqref{II:eq:4-10} vaut pour toute 
correspondance, et non seulement pour les itérés d'un Frobenius. Elle est 
un cas particulier de l'énoncé suivent (appliqué à $X(\dF)$, $K$ et 
aux ${F^*}^n$). 

(*) Soit $X$ un ensemble fini, $F:X\to X$, $\sK\in\ob\D_\text{parf}(X,\Lambda)$ 
et $F^*:F^* \sK\to \sK$. On a 
\[
  \sum_{X^F} \tr\left(F^*,\sK_x\right) = \tr\left(F^*,\Gamma(X,K)\right) \text{.}
\]

On a $\Gamma(X,\sK)=\bigoplus_{x\in X} \sK_x$, et la formule résulte de ce que 
pour tout morphisme $u:\bigoplus_{x\in X} \sK_x\to \bigoplus_{x\in X} \sK_x$, de 
matrice $u_x^y$, on a $\tr(u)=\sum_{x\in X} \tr\left(u_x^x\right)$. 





\begin{lemma_}\label{II:5-2}
Soient $\Lambda$ un anneau noethérien de torsion, tué par une puissance du 
nombre premier $\ell\ne p$, $X_0$ une courbe projective, lisse et connexe sur 
$\dF_q$, $j:U_0\hookrightarrow X_0$ un ouvert dense de $X_0$ et $\sG_0$ un 
faisceau localement constant de $\Lambda$-modules projectifs sur $U_0$. On a 
\[
  \sum_{x\in U^F} \tr\left(F^*,\sG_x\right) = \tr\left(F^*,\R\Gamma_c(U,\sG)\right) \text{.}
\]
\end{lemma_}

Plus précisément, nous déduirons (\ref{II:5-2}) du 





\begin{theorem_}\label{II:5-3}
Soient $X$ une courbe projective non singulière sur un corps algébriquement 
clos $k$, $f$ un endomorphisme à points fixes isolés de $X$ et $v(f)$ le 
nombre de points fixes de $f$, chacun étant compté avec sa multiplicité: 
$v(f)$ est la nombre d'intersection du graphe de $f$ avec la diagonale de 
$X\times X$. Alors, pour $\ell$ premier à l'exposant caractéristique de $k$, 
on a 
\[
  v(f) = \sum (-1)^i \tr\left(f^*,\h^i(X,\dQ_\ell)\right) \text{.}
\]
\end{theorem_}

Ce théorème est dû à Weil (\cite{we48}, n$^\circ$ 43, 66 et 68) et la 
démonstration de Weil est reproduite dans Lang \cite{la83} (VI \S3 combiné 
avec VII \S2 th.3). On peut aussi le déduire du formalisme de la classe 
cohomologie associée à un cycle (\nameref{IV}, \ref{IV:3-7}). 





\begin{corollary_}\label{II:5-4}
Soit $j:U\hookrightarrow X$ un ouvert dense de $X$, tel que $f^{-1}(U)=U$. Si 
les points fixes de $f$ dans $X\setminus U$ sont de multiplicité un, alors le 
nombre $v_U(f)$ de points fixes de $f$ dans $U$ (chacun compté avec sa 
multiplicité) est 
\[
  v_U(f) = \sum (-1)^i \tr\left(f^*,\h_c^i(U,\dQ_\ell)\right) \text{.}
\]
\end{corollary_}

Soit $F$ l'ensemble fini $X\setminus U$. La suite exacte de cohomologie 
définie par la suite exacte court 
$0\to j_! \dQ_\ell\to\dQ_\ell\to{\dQ_\ell}_F\to 0$: 
\[\xymatrix{
  \ar[r]^-\partial 
    & \h_c^i(U,\dQ_\ell) \ar[r] 
    & \h^i(X,\dQ_\ell) \ar[r] 
    & \h^i(F,\dQ_\ell) \ar[r]^-\partial 
    & 
}\]
fournit, avec la convention de signe de (\ref{II:3-1}). 
\[
  \tr\left(f^*,\h_c^\bullet(U,\dQ_\ell)\right) 
    = \tr\left(f^*,\h^\bullet(X,\dQ_\ell)\right) - \tr\left(f^*,\h^\bullet(F,\dQ_\ell)\right) 
    = v(f) - \tr(f^*,\dQ_\ell^F) \text{.}
\]
La trace de $f^*$ sur $\dQ_\ell^F$ est le nombre $v_F(f)$ de points fixes de 
$f$ dans $F$; l'hypothèse de multiplicité un assure donc que 
\[
  \tr\left(f^*,\h_c^\bullet(U,\dQ_\ell)\right) 
    = v_F(f) - v_U(f) \text{,}
\]
comme promis. 

La preuve de (\ref{II:5-2}) utilise quelques lemmes que nous allons 
développer maintenant sur un cas particulier des traces non commutative 
(\ref{II:4-2}). \emph{Pour abréger, nous dirons projectif pour projectif de 
type fini}. 





\subsection{}\label{II:5-5}

Soient $\Lambda$ un anneau, et $H$ un groupe fini. L'application 
$\sum a_h \cdot a\mapsto $ classe de $a_e$, de l'algèbre de groupe 
$\Lambda[H]$ dans $\Lambda^\natural$, s'annule sur les commutateurs $ab-ba$ de 
$\Lambda[H]$. Elle se factorise donc par 
\[
  \varepsilon : \Lambda[H]^\natural \to \Lambda^\natural \text{.}
\]
Si $F$ est un endomorphisme d'un $\Lambda[H]$-module projectif $P$, on pose 
\[
  \tr_\Lambda^H(F) = \varepsilon \tr_{\Lambda[H]} (F) 
\]
où au second membre figure la trace (\ref{II:4-2}). Pour éviter des 
ambiguïtés, on écrira parfois cette trace $\tr_\Lambda^H(F,P)$. 





\begin{proposition_}\label{II:5-6}
$\tr_\Lambda(F) = |H|\tr_\Lambda^H(F)$. 
\end{proposition_}

Les propriétés formelles des traces nous ramènent à ne traiter que le 
cas où $P=\Lambda[H]$. L'endomorphisme $F$ est alors le multiplication à 
droite par un élément $\sum a_h\cdot h$ de $\Lambda[H]$, et 
$\tr_\Lambda(F) = |H| a_e$, $\tr_\Lambda^H(F) = a_e$. 





\subsection{}\label{II:5-7}

Soient $A$ un anneau commutatif et $\Lambda$ un $A$-algèbre. La multiplication 
par un élément de $A$ passe au quotient pour définir une structure de 
$A$-module sur $A^\natural$. Si $H$ est un groupe fini, $P$ un $A[H]$-module 
projectif et $M$ un $\Lambda[H]$-module, projectif en tant que $\Lambda$-module, 
on sait que le $\Lambda$-module $P\otimes_A H$, muni de l'action diagonale de 
$H$, est un $\Lambda[H]$-module projectif. En effet: 
\begin{enumerate}[\indent a)]
  \item I suffit de le vérifier par $P=A[H]$. 
  \item Le $\Lambda$-module $P\otimes_A M$, muni de l'action de $H$ définie 
    par son action sur le premier facteur, est clairement un 
    $\Lambda[H]$-module projectif; notions le $(P\otimes_A M)'$. 
  \item L'application $h\otimes m\mapsto h\otimes h^{-1} m$ induit un 
    isomorphisme 
    \begin{equation}\label{II:eq:5-7-1}
      A[H]\otimes_A M \iso (A[H]\otimes_A M)' \text{.}
    \end{equation}
\end{enumerate}





\begin{proposition_}\label{II:5-8}
Avec les notations précédentes, si $u$ est un endomorphisme de $F$ et 
$v$ un endomorphisme de $M$, on a 
\[
  \tr_\Lambda^H(u\otimes v) = \tr_A^H(u)\tr_\Lambda(v) \text{.}
\]
\end{proposition_}

On se ramène à supposer que $P=A[H]$; l'endomorphisme $u$ est alors la 
multiplication à droite $m_x$ par un élément $x=\sum a_h h$ de $A[H]$. 
L'isomorphisme \eqref{II:eq:5-7-1} transforme $u\otimes v$ en 
$\sum a_h m_h\otimes h^{-1} v$, de trace $a_e\cdot\tr_\Lambda(v)$. 





\subsection{}\label{II:5-9}

Soient $G$ un groupe extension de $\dZ$ par un groupe fini $H$
\[
  1 \to H \to G \to \dZ \to 1 \text{,}
\]
et $G^+$ l'image inverse de $\dN$. Soient $\Lambda$ un anneau, et $P$ un 
$\Lambda[G^+]$-module, projectif en tant que $\Lambda[H]$-module. Si 
$g\in G^+$, on note $Z_g$ le centralisateur de $g$ dans $H$; la multiplication 
par $g$ est un endomorphisme de $P$, vu comme $\Lambda[Z_g]$-module. $P$ étant 
$\Lambda[Z_g]$-projectif, une trace $\tr_\Lambda^{Z_g}(g)$ est définie. 
D'après \ref{II:5-6}, on a 
\begin{equation}\label{II:eq:5-9-1}
  \tr_\Lambda(g) = |Z_g| \cdot \tr_\Lambda^{Z_g}(g) \text{.}
\end{equation}

Supposons que $\Lambda$ soit une algèbre sur un anneau commutatif $A$. Pour 
$P$ un $A[G^+]$-module, projectif en tant que $A[H]$-module et $M$ un 
$\Lambda[G^+]$-module, projectif en tant que $\Lambda$-module, le 
$\Lambda[G^+]$-module $P\otimes_A M$ (action diagonale de $G^+$) est projectif 
en tant que $\Lambda[H]$-module et d'après \ref{II:5-8}, pour tout 
$g\in G^+$, on a 
\begin{equation}\label{II:eq:5-9-2}
  \tr_\Lambda^{Z_g}(g,P\otimes_A M) = \tr_A^{Z_g}(g,P) \cdot \tr_\Lambda(g,M) \text{.}
\end{equation}





\subsection{}\label{II:5-10}

Soient $P$ un $\Lambda[G^+]$-module projectif en tant que $\Lambda[H]$-module, 
et $P_H$ les coinvariants de $H$ dans $P$. C'est un $\Lambda$-module projectif. 
L'action de $g\in G^+$ passe au quotient, et définit un endomorphisme de 
$P_H$ qui ne dépend que de l'image de $g$ dans $\dN$. Nous notons $F$ 
l'action des éléments $g\in G^+$ d'image $1$. 





\begin{proposition_}\label{II:5-11}
$\tr_\Lambda(F,P_H) = \sum_{g\mapsto 1}' \tr_\Lambda^{Z_g}(g,P)$, oùu 
$\sum_{g\mapsto 1}'$ désigne une somme étendue aux classes de 
$H$-conjugaison d'éléments de $G^+$ d'image $1$ dans $\dN$. 
\end{proposition_}

Après multiplication par $|H|$, cette formule peut se vérifier comme suit: 
\[
  |H|\cdot \tr_\Lambda(F,P_H) 
    = \tr_\Lambda\left(\sum_{g\mapsto 1} g,P\right) 
    = \sum_{g\mapsto 1}' \frac{|H|}{|Z_g|}\tr_\Lambda(g,P) 
    = \sum_{g\mapsto 1}' |H|\cdot \tr_\Lambda^{Z_g}(g,P) \text{.}
\]

Choisissons un élément $g\in G^+$ d'image $1$ dans $\dN$, et soit 
$\sigma$ l'automorphisme $\Lambda[H]$ induit par l'automorphisme 
$\operatorname{ad} g$ de $H$. Il revient au même de se donner un 
$\Lambda[G^+]$-module, ou un $\Lambda[H]$--module muni d'un endomorphisme 
$\sigma$-linéaire $\gamma$: on fait agir $g^n h$ par $\gamma^n h$. Dans ce 
langage, la formule \ref{II:5-11} prend la forme: pour $P$ un 
$\Lambda[H]$-module projectif et pour $\gamma$ un endomorphisme 
$\sigma$-linéaire de $P$, on a 
\[
  \tr_\Lambda(\gamma,P_H) = \sum_h' \tr_\Lambda^{Z_g}(h\gamma,P)\text{,}
\]
où $\sum'$ est une somme sur les classes de $\operatorname{ad}g$-conjugaison 
dans $H$ (orbites de $H$ agissant sur lui-même par les 
$x\mapsto h x \operatorname{ad} g(h)^{-1}$). 

Sous cette forme, les propriétés des traces nous ramènent aussitôt au 
cas où $P=\Lambda[H]$. L'endomorphisme $Y$ a alors la forme 
$x\mapsto \sigma(x) a$, avec $a=\sum a_h h$ dans $\Lambda[H]$. Il suffit de 
traiter le cas universel où lles $a_h$ sont des indéterminées. Dans ce 
cas, $\Lambda=\dZ\langle a_h\rangle_{h\in H}$ (polynômes non commutatifs) et 
$\Lambda^\natural$ est sans torsion: on obtient \ref{II:5-11} par l'argument 
du début et division par $|H|$. On peut aussi calculer directement. 





\subsubsection{Remarque}\label{II:5-11-1}

Tout ce qui a été dit ci-dessus s'applique aussi bien aux modules 
$\dZ/2$-gradués, ou aux complexes de modules (cf. \ref{II:4-2}, \ref{II:4-3}). 





\subsection{}\label{II:5-12}

Reprenons les notations de \ref{II:5-2}, et soit $A=\dZ/\ell^n$, $n$ étant 
assez grand pour que $\Lambda$ soit une $A$-algèbre. Soit $f:V_0\to U_0$ un 
revêtement fini étale et connexe de $U_0$, galoisien de groupe de Galois 
$H$, et tel que le faisceau $f^*\sG_0$ soit constant. On note $M$ le valeur 
constants $\h^0(V_0,f^*\sG_0)$, c'est un $\Lambda[H]$-module, projectif en 
tant que $\Lambda$-module. 

Le faisceau $f_* A$ sur $U_0$, muni de l'action naturelle de $H$, est un 
faisceau localement libre (de rang un) de $A[H]$-modules. Le complexe 
$\R\Gamma_c(U,f_* A)$ est donc objet de $\D_\text{parf}\left(A[H]\right)$. Il 
est muni d'un endomorphisme de Frobenius $F^*$. 

Pour l'action naturelle de $H$ sur $f_* f^*\sG_0$, le morphisme trace 
$f_* f^*\sG_0\to\sG_0$ se factorise par un isomorphisme 
\[
  \left(f_* f^*\sG_0\right)_H \to \sG_0 \text{.}
\]
Par ailleurs, on a $f_* f^*\sG_0=f_* A\otimes_A M$ (avec action diagonale de 
$H$). L'isomorphisme 
\[\xymatrix{
  \sG_0 
    & \ar[l]_-\sim (f_* A\otimes_A M)_H = (f_* A\otimes_A M)\otimes_{\Lambda[H]} \Lambda
}\]
se dérive (\ref{II:4-12}) et un isomorphisme 
\[
  \R\Gamma_c(U,\sG) \sim \left(\R\Gamma_c(U,f_* A)\otimes_A M\right)\otimes_{\Lambda[H]} \Lambda \text{.}
\]

L'endomorphisme $F^*$ du membre de gauche se déduit de l'endomorphisme 
$F^*$ de $\R\Gamma_c(U,f_* A)$. Le complexe de $\Lambda[H]$-modules 
$\R\Gamma_c(U,f_* A)$, muni de $F^*$, peut être vu comme un complexe de 
$\Lambda[G^+]$-modules, pour $G^+=H\times \dN$. Les formules \ref{II:5-10} 
et \eqref{II:eq:5-9-2}, amplifiées par \ref{II:5-11-1}, fournissent donc 
\[
  \tr\left(F^*,\R\Gamma_c(U,\sG)\right) = \sum_h' \tr_A^{Z_g}\left(h F^*,\R\Gamma_c(U,f_* A)\right)\cdot \tr_\Lambda(h,M)
\]
où $\sum'$ est une somme sur les classes de conjugaison dans $H$, De plus, 
\[
  |Z_h|\cdot \tr_A^{Z_h}\left(h F^*,\R\Gamma_c(U,f_ A)\right) = \tr_A\left((F h^{-1})^*,\R\Gamma_c(V,A)\right) \text{.}
\]
Cette formule reste vraie pour $A=\dZ/\ell^m$, avec $m$ de plus en plus grand. 
Passant à limite, on trouve que 
$\tr_A^{Z_g}\left(h F^*,\R\Gamma_c(U,f^* A)\right)$ vaut 
\[
  \frac{1}{|Z_h|} \sum (-1)^i \tr\left((F h^{-1})^*,\h_c^i(V,\dQ_\ell)\right)
\]
(un élément de $\dZ_\ell$), et
\begin{equation}\label{II:eq:5-12-1}
\tr\left(F^*,\R\Gamma_c(U,\sG)\right) = \sum_h' \frac{1}{|Z_h|} \tr\left((F h^{-1})^*,\h_c^i(V,\dQ_\ell)\right)\cdot \tr_\Lambda(h,M) \text{.} 
\end{equation}





\subsection{}\label{II:5-13}

Il reste maintenant à appliquer \ref{II:5-4} (noter que si $\bar V$ est la 
complétion projective non singulière de $V$, les points fixes de $F h^{-1}$ 
sont tous de multiplicité un). On peut le faire sans aucun calcul: 
\begin{enumerate}[\indent A.]
  \item Si $U^F=\varnothing$, $\tr\left(F^*,\R\Gamma_c(U,\sG)\right) = 0$. \end{enumerate}

En effet, $F h^{-1}$ n'a pas de point fixe sur $V$, \ref{II:5-4} montre que 
$\sum (-1)^i \tr\left((F h^{-1})^*,\h_c^i(V,\dQ_\ell)\right)=0$ et on 
applique \eqref{II:eq:5-12-1}.
\begin{enumerate}[\indent A.]
\setcounter{enumi}{1}
  \item Dans le cas général, soit $j:U'=U\setminus U^F\hookrightarrow U$. 
    La filtration  $j_! j^* \sG_0\subset \sG_0$ de $\sG_0$ induit une 
    filtration de $\R\Gamma_c(U,\sG)$ de quotients successifs 
    $\R\Gamma_c(U',\sG)$ et $\R\Gamma_c(U^F,\sG)$. D'après \eqref{II:eq:4-4-1} 
    et \ref{II:5-1}, on a 
    \[
      \tr\left(F^*,\R\Gamma_c(U,\sG)\right) = \tr\left(F^*,\R\Gamma_c(U',\sG)\right) + \sum_{x\in U^F} \tr(F^*,\sG_x) \text{,}
    \]
    et, puisque ${U'}^F=0$, le premier terme au second membre est $0$. 
\end{enumerate}










\section{Fin de la preuve de \ref{II:4-10}}\label{II:6}

Prouvons \ref{II:4-10}, sous les hypothèses additionnelles que l'anneau 
$\Lambda$ est annulé par une puissance d'un nombre premier $\ell\ne p$, et 
que $n=1$. Le cas général en résultera: décomposer $\Lambda$ en ses 
composantes $\ell$-primaires ($\ell$ parcourant les nombre premiers), et 
remplacer $\dF_q$ par $\dF_{q^n}$, $X_0$ par $X_0\otimes_{\dF_q}\dF_{q^n}$. 

Pour $X_0$ séparé de type fini sur $\dF_q$, et 
$\sK_0\in\ob\D_\text{ctf}(X_0,\Lambda)$, posons 
\[
  T'(X_0,\sK_0) = \sum_{x\in X^F} \tr(F^*,\sK_x)
\]
et
\[
  T''(X_0,\sK_0) = \tr(F^*,\R\Gamma_c(X,\sK)) \text{.}
\]
Il faut prouver l'identité 
\[
  (1;X_0,\sK_0) \qquad T'(X_0,\sK_0) = T''(X_0,\sK_0) \text{.}
\]
On commence par observer que $T'$ et $T''$ (notés génériquement $T$) 
obéissent à un même formalisme:

(A) Si $j:U_0\hookrightarrow X_0$ est un ouvert de $X_0$, de fermé 
complémentaire $Y_0$, on a 
\[
  T(X_0,\sK_0) = T(U_0,\sK_0|U_0) + T(Y_0,\sK_0|Y_0) \text{.}
\]
C'est clair pour $T'$. Pour $T''$, on observe que la filtration 
$0\subset j_! j^* \sK_0\subset \sK_0$ de $\sK_0$ induit une filtration de 
$\R\Gamma_c(X,\sK)$, de quotients successifs $\R\Gamma_c(U,\sK)$ et 
$\R\Gamma_c(Y,\sK)$, et on invoque \eqref{II:eq:4-4-1} (plus exactement 
énoncé: $\sK_0$, muni de la filtrations dite, définit un objet de 
$\mathsf{DF}(X,\Lambda)$; par application de $\R\Gamma_c$, on a déduit un 
objet de $\mathsf{DF}_\text{parf}(\Lambda)$, auquel $\R\Gamma_c(X,\sK)$ est 
sous-jacent, et de gradué comme dit). 

(B) Si $\sK_0$ est un complexe borné à composantes constructibles et plates, 
\[
  T(X_0,\sK_0) = \sum (-1)^i T(X_0,\sK^i) \text{.}
\]
C'est clair pour $T'$; pour $T''$, on applique l'argument ci-dessus à la 
filtration bête de $\sK_0$, pour obtenir 
\[
  T''(X_0,\sK_0) = \sum T''(X_0,\sK_0^i[-i]) = \sum (-1)^i T''(X_0,\sK_0^i) \text{.}
\]

(C) Pour $f:X_0\to Y_0$ un morphisme, on a 
\[
  T''(X_0,\sK_0) = T''(Y_0,\R f_! \sK_0)
\]
et la même identité vaut pour $T'$ si les fibres de $f$ aux points 
rationnels de $Y_0$ vérifiant $\left(1;f^{-1}(y),\sK_0|f^{-1}(y)\right)$: on 
utilise que 
\[
  T'(X_0,\sK_0) = \sum_{y\in Y^F} T'\left(f^{-1}(y), \sK_0|f^{-1}(y)\right)
\]
et 
\[
  \R\Gamma_c(Y, \R f_! \sK) = \R\Gamma_c(X,\sK) \text{.}
\]

Utilisant ce formalisme, et \ref{II:4-7}, on ramène la preuve de \ref{II:4-10} 
aux deux particuliers \ref{II:5-1} et \ref{II:5-2}. 

