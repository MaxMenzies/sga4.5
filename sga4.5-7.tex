% !TEX root = sga4.5.tex


\chapter{Théorèmes de finitude en cohomologie \texorpdfstring{$\ell$}{l}-adique}\label{VII}










\section{Énoncé des théorèmes}\label{VII:1}

Dans tout cet exposé, $S$ sera un schéma noethérien et $A$ un anneau 
noethérien à gauche, de torsion annulé par un entier inversible sur $S$. 
Notre résultat principal est le suivant. 





\begin{theorem_}\label{VII:1-1}
On suppose $S$ régulier de dimension $0$ ou $1$. Soient $f:X\to Y$ un 
morphisme de $S$-schémas de type fini et $\sF$ un faisceau constructible de 
$A$-modules à gauche sur $X$. Alors, les faisceaux $\eR^i f_\ast \sF$ sont 
constructibles. 
\end{theorem_}

La preuve sera donnée au paragraphe \ref{VII:2} et en \ref{VII:3-10}. 





\subsection{Remarque}\label{VII:1-2}

En caractéristique $0$, ce résultat est moins général que 
\cite[XIX paragraphe 5]{sga4}, qui prouve la conclusion du théorème pour 
tout morphisme de type fini de schémas excellents de caractéristique $0$. 
La démonstration de \cite[XIX]{sga4} utilise d'une part la résolution des  
singularités, d'autre part que les schémas soient d'égale 
caractéristique (pour pouvoir déduire de la résolution le ``théorème 
de pureté''). 






\subsection{Remarque}\label{VII:1-3}

Nous dirons qu'un complexe $K\in\ob\eD(X,A)$ est \emph{constructible} si ses 
faisceaux de cohomologie sont constructibles et nous noterons avec un indice 
$c$ la sous-catégorie de $\eD(X,A)$ (ou $\eD^+$, $\eD^-$, $\eD^b$) formée 
des complexes constructibles. Dans le langage des catégories dérivées, 
que nous utiliserons librement, \ref{VII:1-1} dit que 
$\eR f_\ast:\eD^+(X,A) \to \eD^+(Y,A)$ envoie $\eD_c^+$ dans $\eD_c^+$. 
Explicitons: 
\begin{enumerate}[\indent a)]
  \item Pour $K$ réduit à un faisceau $\sF$ en degré $0$, on a 
    $\sH^i\eR f_\ast K = \eR^i f_\ast \sF$; l'énoncé dérivé implique 
    donc le théorème. 
  \item Dans l'autre sens, on invoque la suite spectrale 
    \[
      E_1^{p q} = \eR^q f_\ast \sH^p(K) \Rightarrow \sH^{p+q} \eR f_\ast K \text{.} 
    \]
\end{enumerate}

Parler le langage des catégories dérivées à l'avantage de remplacer par 
une simple formule de transitivité $\eR(f g)_\ast = \eR f_\ast \eR g_\ast$ 
une suite spectrale de Leray 
$E_2^{p q} = \eR^p f_\ast \eR^q g_\ast \Rightarrow \eR^{p+q} (f g)_\ast$. 

Sous les hypothèses de \ref{VII:1-1}, il résulte de \cite{sga4} que 
$\eR f_\ast$ est de dimension cohomologique finie. Cela permet ci-dessus de 
remplacer $\eD^+$ par $\eD$, $\eD^-$ ou $\eD^b$. 





\subsection{Les idées de la démonstration}\label{VII:1-4}
\begin{enumerate}[\indent a)]
  \item La dualité de Poincaré permet de traiter le cas où $X$ est 
    lisse, $\sF$ localement constant, et $Y=S$. L'hypothèse de dimension 
    appara\"it par calculer des $\rHom$ sur $S$.
  \item On factorise $f$ en $g j$: $g$ propre et $j$ plongement ouvert. On a 
    $\eR f_\ast = \eR g_\ast \eR j_\ast$ et le théorème de finitude pour 
    les morphismes propres contrôle $\eR g_\ast$. Dévissant, on se ramène 
    à supposer $x$ lisse, $\sF$ localement constant, $f$ un plongement ouvert 
    et $Y$ propre sur $S$. 
  \item Une récurrence sur $\dim X$ (avec changement de $S$) permet, grosso 
    modo, de supposer le théorème vrai en dehors d'une partie de $Y$ finie 
    sur $S$. Notant $a:X\to S$ et $b:Y\to S$ les morphismes structuraux, on 
    montre alors que si le théorème était faux pour $\sF$ et 
    $\eR f_\ast$, il le serait aussi pour $\sF$ et 
    $\eR b_\ast \eR f_\ast = \eR a_\ast$: contradiction. 
\end{enumerate}





\begin{corollary_}\label{VII:1-5}
Sous les hypothèses du théorème, les catégories $\eD_c(X,A)$ et 
$\eD_c(Y,A)$ sont transformées l'une en l'autre par les $4$ opérations 
$\eR f_\ast$, $\eR f_!$, $f^\ast$, $\eR f^!$. 
\end{corollary_}

$\eR f_\ast$ est traité ci-dessus, $\eR f_!$ en \cite[XVII 5.3]{sga4}, 
$f^\ast$ est clair. Reste $\eR f^!$. Le problème est local, ce qui permet de 
supposer que $f$ se factorise en $X\xrightarrow i Z \xrightarrow g Y$ ($i$ 
plongement fermé et $g$ lisse, purement de dimension relatif $n$). La 
dualité de Poincaré $\eR g_! K(n)[2 n]$ et la transitivité 
$\eR f^! = \eR i^! \eR g^!$ nous ramènent à prouver \ref{VII:1-5} pour $i$. 
Pour tout $L\in \eD(Z,A)$, si $j$ est l'inclusion dans $Z$ de $U=Z\setminus X$, 
$i_\ast \eR i^! L$ est le mapping cylinder de $K\to \eR j_\ast j^\ast K$ et 
\ref{VII:1-5} pour $i$ résulte de \ref{VII:1-1} pour $j$. 





\begin{corollary_}\label{VII:1-6}
Soit $X$ comme dans le théorème, et supposons $A$ commutatif. Alors, si 
$\sF$ et $\sG$ sont des faisceaux constructibles de $A$-modules sur $X$, les 
$\Ext_A^i(\sF,\sG)$ sont constructibles: $\rHom$ envoie 
$\eD_c^-(X,A)\times \eD_c^+(X,A)$ dans $\eD_c^+(X,A)$. 
\end{corollary_}

Par dévissage de $\sF$, on se ramène à supposer $\sF$ de la forme 
$j_!\sF_1$ ($j:Y\hookrightarrow X$) un plongement localement fermé et $\sF_1$ 
localement constant sur $Y$). On a alors 
\[
  \rHom(j_! \sF_1,\sG) = \eR j_\ast \Hom(\sF_1,\eR j^! \sG) \text{:} 
\]
d'après \ref{VII:1-3}, nous sommes ramenés au cas où $\sF$ est localement 
constant -- voire constant puisque le problème est local. Pour $\sF$ 
localement constant constructible, on a $\Hom(\sF,\sG)_x=\hom(\sF_x,\sG_x)$, 
et de même pour $\rHom$. Pour $\sF$ constant, on peut donc calculer les 
$\Ext^i$ à l'aide d'une résolution projective de type fini de sa valeur 
constante, et les $\Ext^i$ sont constructibles si $\sG$ l'est -- d'où le 
corollaire. 





\subsection{Remarque}\label{VII:1-7}

Puisque $\eR f_\ast$ est de dimension cohomologique finie, il transforme 
complexes de tor-dimension finie (resp. $\leqslant d$) en complexes de 
tor-dimension finie (resp. $\leqslant d$) \cite[XVII 5.2.11]{sga4}. 
\cite[XVII 5.2.10]{sga4} et la preuve de \ref{VII:1-5} montrent alors que la 
tor-dimension finie est stable par les $4$ opérations $\eR f_\ast$, 
$\eR f_!$, $f^\ast$, $\eR f^!$. Une variant de celle de \ref{VII:1-6} 
(dévisser $K$ selon une partition de $x$ pour supposer les faisceaux de 
cohomologie localement constants, puis se localiser pour remplacer $K$ par un 
complexe fini de faisceaux localement libres de type fini) montre alors que, 
pour $A$ commutatif, $\rHom$ induit 
\[
  \rHom:\eD_\text{ctf}^b(X,A)\times \eD_\text{tf}^b(X,A) \to \eD_\text{tf}^+(X,A) 
\]
(t.f.: tor-dimension finie). 





\subsection{}\label{VII:1-8}

Sous les hypothèses du théorème, la même méthode nous permet de 
prouver un théorème de bidualité locale $K\iso D D K$ (paragraphe 
\ref{VII:4}). Nous prouvons aussi un théorème de finitude pous les 
faisceaux de cycles évanescents. Pour $S$ quelconque, on obtient encore un 
théorème ``générique'': 





\begin{theorem_}\label{VII:1-9}
Soit $f:X\to Y$ un morphisme de $S$-schémas de type fini et $\sF$ un faisceau 
constructible de $A$-modules sur $X$. Il existe un ouvert dense $U$ de $S$ tel 
que 
\begin{enumerate}[\indent (i)]
  \item Au-dessus de $U$, les $\eR^i f_\ast \sF$ sont constructibles, et nuls 
    sauf un nombre fini d'entre eux. 
  \item La formation des $\eR^i f_\ast \sF$ est compatible à tout 
    changement de base $S' \to U\subset S$. 
\end{enumerate}
\end{theorem_}

Pour $S$ le spectre d'un corps, on a $U=S$. 

Si $S$ est le spectre d'un corps, un argument de passage à la limite fournit 
la compatibilité aux $S$-changements de base des $\eR^i f_\ast$ pour tout 
morphisme quasi-compact quasi-séparé de $S$-schémas et tout faisceau 
$\sF$. 





\begin{corollary_}\label{VII:1-10}
Soient $x$ un schéma de type fini sur $k$ séparablement clos et $\sF$ un 
faisceau constructible de $A$-modules. Alors, les $\h^i(X,\sF)$ sont de type 
fini. 
\end{corollary_}

C'est le cas particulier $S=\spec(k)$, $Y=S$. 

\begin{corollary_}\label{VII:1-11}
Soient $X$ et $Y$ deux schémas de type fini sur $k$ séparablement clos et 
$K\in \ob\eD^-(X,A^\circ)$, $L\in \ob\eD^-(Y,A)$ des complexes de faisceaux de 
modules à droite et à gauche. La flèche de K\"unneth 
\[
  \eR \Gamma(X,K) \lotimes \eR \Gamma(Y,L) \to \eR\Gamma(X\times Y,\pr_1^\ast K \lotimes \pr_2^\ast L) 
\]
est un isomorphisme. 
\end{corollary_}

On procède comme en \cite[XVII 5.4.3]{sga4}: 
\[\xymatrix{
  X\times Y \ar[r]^-{\text{pr}_1} \ar[d]^-{\text{pr}_2} 
    & X \ar[d]^-a \\
  Y \ar[r]^-b 
    & \spec(k) 
}\]
changeant de base par $b$, on trouve que 
$\eR\pr_{2,\ast}{} \pr_1^\ast K$ est 
$b^\ast \eR\Gamma(X,K)$. On a alors (cf. \cite[XVII 5.2.11]{sga4} et preuve) 

\begin{align*}
  \eR\Gamma(X\times Y,\pr_1^\ast K\lotimes \pr_2^\ast L) 
    &= \eR\Gamma(Y,\eR \pr_{2,\ast}(\pr_1^\ast K\lotimes \pr_2^\ast L)) \\
    &= \eR\Gamma(Y,(\eR\pr_{2,\ast}\pr_1^\ast K)\lotimes L) \\
    &= \eR\Gamma(Y,b^\ast \eR\Gamma(X,K)\lotimes L) \\
    &= \eR\Gamma(X,K)\lotimes \eR\Gamma(Y,L) 
\end{align*}





\subsection{}\label{VII:1-12}

Enfin, comme contre-partie locale de \ref{VII:1-9}(ii), nous prouverons un 
théorème d'acyclicité locale générique (\ref{VII:2-13}). 










\section{Théorèmes génériques}\label{VII:2}

Dans ce paragraphe, nous prouvons \ref{VII:1-9} et le théorème 
d'acyclicité locale générique \ref{VII:2-13}. Pour prouver \ref{VII:1-9}, 
on se ramène aussitôt à supposer $S$ intègre. Soit $\eta$ son point 
générique. 





\subsection{}\label{VII:2-1}

Nous commencerons par prouver \ref{VII:1-9} sous les hypothèses 
additionnelles suivantes: $X$ est lisse sur $S$, purement de dimension relative 
$n$, $A=\dZ/m$, $\sF$ est localement constant et $Y=S$. Soit 
$\sF'=\Hom(\sF,\dZ/m)$. Nous allons montrer que si les $\eR^i f_! \sF'$ sont 
localement constants, les conclusions de \ref{VII:1-9} valent pour $U=S$. 
Rappelons que si $\sL$ est un faisceau localement constant constructible, les 
$\Ext^i(\sL,\sG)$ se calculent fibre par fibre, et sont constructibles si $\sG$ 
l'est. Rappelons aussi que $\dZ/m$ est un $\dZ/m$-module injectif, et est le 
module dualisant, de sorte que $\sF=\rHom(\sF',\dZ/m)$. La dualité de 
Poincaré 
\[
  \eR f_\ast \rHom(K,\eR f^! L) = \rHom(\eR f_! K,L) \text{,} 
\]
pour $K=\sF'$, $L=\dZ/m$, $\eR f^! L = \dZ/m[2n](n)$, fournit donc 
\[
  \eR^{2n-i} f_\ast \sF = \Hom(\eR^i f_! \sF',\dZ/m)(-n) \text{.} 
\]

Le faisceau au second membre est localement constant, de formation compatible 
à tout changement de base -- d'où l'assertion. 





\subsection{}\label{VII:2-2}

Prouvons \ref{VII:1-9} sous les hypothèses additionnelles: $X$ est lisse sur 
$S$, $\sF$ est localement constant et $Y=S$. 
\begin{enumerate}[a)]
  \item Décomposant $X$ en composantes connexes, on se ramène à supposer 
    qu'il est purement d'une dimension relative $n$ sur $S$. 
  \item Décomposant $A$ en produit, on se ramène à supposer que 
    $\ell^m A=0$, avec $\ell$ premier inversible sur $S$; on filtre alors $\sF$ 
    par les $\ell^k \sF$ pour se ramener, par la suite spectrale 
    correspondante, au cas où $\ell\sF=0$. On peut alors remplacer $A$ par 
    $A/\ell A$ et supposer que $\ell A=0$. 
  \item Remplaçant $S$ par $U$ convenable, on peut supposer qu'il existe un 
    revêtement galoisien fini étale $X_1/X$, de groupe de Galois $G$, telle 
    que l'image réciproque $\sF_1$ de $\sF$ sur $X_1$ soit un faisceau 
    constant, de valeur constante $F$. Notant $f_1$ la projection de $X_1$ sur 
    $S$, on a alors 
    \[
      \eR^i {f_1}_\ast \sF_1 = (\eR^i {f_1}_\ast \dZ/\ell)\otimes_{\dZ/\ell} F \text{.} 
    \]
    On dispose aussi de la suite spectrale de Hochschild-Serre (ou: de Leray 
    pour le recouvrement $X_1/X$) 
    \[
      \underline\h^p(G,\eR^q {f_1}_\ast \sF_1) \Rightarrow \eR^{p+q} f_\ast \sF \text{.} 
    \]
    D'après \ref{VII:2-1}, on peut supposer, quitte à rétrécir $U$, que 
    les $\eR^i {f_1}_\ast \dZ/\ell$ sont localement constants, de formation 
    compatible à tout changement de base. La même propriété vaut alors 
    pour les $\eR^i f_\ast \sF$. Enfin, si $\bar\eta$ est un point 
    géométrique localisé au point générique $\eta$ de $S$, les 
    $\eR^i f_\ast\sF$ sont presque tous nuls, car les 
    $(\eR^i f_\ast \sF)_{\bar\eta} = \h^i(X_{\bar \eta},\sF)$ le sont. 
\end{enumerate}





\subsection{}\label{VII:2-3}

Prouvons par récurrence sur $n$ que 
\begin{equation*}\tag{$\ast_n$}\label{VII:eq:2-3-1}
  \begin{array}{c}
    \text{Les conclusions de \ref{VII:1-9} sont vraies lorsque $\dim{X_\eta}\leqslant n$} \\
    \text{et que $f$ est un plongement ouvert d'image dense.}
  \end{array}
\end{equation*}

Pour $n=0$, quitte à rétrécir $S$, on a $X=Y$: $(\ast_0)$ est évident. 
Supposons ($\ast_{n-1})$, et prouvons \eqref{VII:eq:2-3-1}. Dans 
($\ast_{n-1}$), on peut remplacer ``plongement ouvert'' par ``plongement'' 
comme on le voit en factorisant en plongement ouvert et plongement fermé. 





\begin{lemma_}\label{VII:2-4}
Quitte à rétrécir $S$, les conclusions de \ref{VII:1-9} valent au-
dessus de $Y'\subset Y$, le complément $Y_1$ de $Y'$ étant fini sur $S$. 
\end{lemma_}

L'assertion est locale sur $Y$, qu'on peut supposer affine: $Y\subset \dA_S^n$. 
L'hypothèse de récurrence ($\ast_{n-1}$) s'applique à 
\[\xymatrix{
  X \ar[r]^-f \ar[dr] 
    & Y \ar[d]^-{\pr_i} \\
  & \dA_S^1 
}\]
Il existe donc pour chaque $i$ ouvert dense $U_i$ de $\dA_S^1$ tel que les 
conclusions de \ref{VII:1-9} valent au-dessus de $\pr_i^{-1}(U_i)$; elle valent 
au-dessus de la réunion des $\pr_i^{-1}(U_i)$, et \ref{VII:2-4} en résulte. 





\addtocounter{subsection}{1} % a section is skipped in the original
\subsection{}\label{VII:2-6}

Prouvons \eqref{VII:eq:2-3-1} pour $X$ lisse sur $S$ et $\sF$ localement 
constant sur $X$. Le problème étant local sur $Y$, on peut supposer $Y$ 
affine, puis projectif (remplacer $Y$ par son adhérence dans un espace 
projectif). Soient $i:Y_1\hookrightarrow Y$ et $j:Y' \to Y$ garantis par 
\ref{VII:2-4}. 
\[\xymatrix{
  X \ar@{^{(}->}[r] \ar[dr]_-a 
    & Y \ar[d]^-b 
    & \ar[l]_-i \ar[dl]^-{b_1} Y_1 \\
  & S 
}\] 
quitte à rétrécir $S$, on sait que $j^\ast \eR f_\ast \sF$ est 
constructible, de formation compatible à tout changement de base en $S$; on 
sait aussi que $\eR a_\ast \sF = \eR b_\ast \eR f_\ast \sF$ est constructible, 
de formation compatible à tout changement de base. 

Appliquons $\eR b_\ast$ au triangle défini par la suite exacte 
\begin{equation*}\tag{1}\label{VII:eq:2-6-1}
\xymatrix{
  0 \ar[r] 
    & j_! j^\ast \eR f_\ast \sF \ar[r] 
    & \eR f_\ast \sF \ar[r] 
    & i_! i^\ast \eR f_\ast \sF \ar[r] 
    & 0 \text{:} 
}
\end{equation*}
on obtient un triangle 
\begin{equation*}\tag{2}\label{VII:eq:2-6-2}
\xymatrix{
  \ar[r] 
    & \eR b_\ast j_! j^\ast \eR f_\ast \sF \ar[r] 
    & \eR a_\ast \sF \ar[r] 
    & {b_1}_\ast i^\ast \eR f_\ast \sF \ar[r] 
    & 
}
\end{equation*}
dans lequel les deux premiers termes sont constructibles, de formation 
compatible à tout changement de base en $S$ (pour le ler, d'après le 
théorème de finitude pour le morphisme propre $b$). Il en va donc de 
même pour le troisième. Puisque $b_1$ est fini, on en déduit que 
$i^\ast \eR f_\ast \sF$ est constructible de formation compatible à tout 
changement de base en $S$, et de même pour $\eR f_\ast \sF$ par 
\eqref{VII:eq:2-6-1}. 





\subsection{}\label{VII:2-7}

Prouvons \eqref{VII:eq:2-3-1} en général. On commence par se ramener au cas 
où dans $X$ existe un ouvert dense $V$ lisse sur $S$. Pour $S$ spectre d'un 
corps parfait, il suffit de remplacer $X$ par $X_\text{red}$ (et $Y$ par 
$Y_\text{red}$). En général, il faut rapetisser $S$, faire un changement de 
base fini radicel et surjectif $S' \to S$, et remplacer $X$ et $Y$ par 
$X_\text{red}$ et $Y_\text{red}$. La topologie étale étant insensible aux 
morphismes finis radiciels et surjectifs, ceci est innocent. Quitte à 
rétrécir $V$, on peut supposer $\sF$ localement constant sur $V$ 
\[\xymatrix{
  V \ar@{^{(}->}[r]^i 
    & X \ar@{^{(}->}[r]^-f 
    & Y \text{.} 
}\]
Définissons $\Delta$ par le triangle 
\begin{equation*}\tag{1}\label{VII:eq:2-7-1}
\xymatrix{
  \ar[r] 
    & \sF \ar[r] 
    & \eR j_\ast j^\ast \sF \ar[r] 
    & \Delta \ar[r] 
    & \text{.} 
}
\end{equation*}
Les faisceaux de cohomologie e $\Delta$ sont à support dans $X\setminus V$, 
et $\dim(X\setminus V)_\eta < n$. L'hypothèse de récurrence permet donc de 
supposer que $\eR f_\ast \Delta$ est constructible. Appliquons $\eR f_\ast$ au 
triangle \eqref{VII:eq:2-7-1}; on trouve un triangle 
\[\xymatrix{
  \ar[r] 
    & \eR f_\ast \sF \ar[r] 
    & \eR(f j)_\ast j^\ast \sF \ar[r] 
    & \eR f_\ast \Delta \ar[r] 
    & 
}\]
dans lequel deux des sommets sont constructibles de formation compatible à 
tout changement de base en $S$. Le troisième l'est donc également. 





\subsection{}\label{VII:2-8}

Prouvons \ref{VII:1-9}. Le problème est local sur $Y$, qu'on peut supposer 
affine. Prenant un recouvrement affine de $X$ et invoquant la suite spectrale 
de Leray, on se ramène à avoir $X$ également affine. Tout ceci pour 
assurer qu'on puisse factoriser $f$ en un plongement ouvert suivi d'un 
morphisme propre: $f=g j$, d'où $\eR f_\ast = \eR g_\ast \eR j_\ast$. Le 
plongement ouvert est justiciable d'un \eqref{VII:eq:2-3-1}, le morphisme 
propre du théorème de finitude. 

Les corollaires suivants se prouvent comme au paragraphe 1. 





\begin{corollary_}\label{VII:2-9}
Sous les hypothèses du théorème, pour $K$ dans $\eD_c^b(X,A)$ ou 
$\eD_c^b(Y,A)$ respectivement, il existe un ouvert non vide $U$ de $S$ 
au-dessus duquel $\eR f_\ast K$, $\eR f_! K$, $f^\ast K$, $\eR f^! K$ soient 
dans $\eD_c^b(Y,A)$ ou $\eD_c^b(X,A)$, et de formation compatible à tout 
changement de base $S' \to U\subset S$. 
\end{corollary_}





\begin{corollary_}\label{VII:2-10}
Soit $X$ comme dans le théorème, et supposons $A$ commutatif. Alors, si 
$\sF$ et $\sG$ sont des faisceaux constructibles de $A$-modules sur $X$, 
au-dessus d'un ouvert dense $U$ de $S$, les $\Ext_A^i(\sF,\sG)$ sont encore 
constructibles, de formation compatible à tout changement de base 
$S' \to U\subset S$. 
\end{corollary_}





\subsection{}\label{VII:2-11}

Si $x$ est un point géométrique d'un schéma $X$, nous noterons $X_x$ 
l'henselisé strict de $X$ en $x$. Pour $f:X\to S$ et $t$ un point 
géométrique de $S$, nous noterons $X_t$ la fibre géométrique de $X$ en 
$t$. Enfin, pour $x$ un point géométrique de $X$, et $t$ un point 
géométrique de $S_{f(x)}$, $(X_x)_t$ est la fibre en $t$ de 
$X_x \to S_{f(x)}$. 





\begin{definition_}\label{VII:2-12}
Soient $f:X\to S$ et $K\in \ob\eD^+(X,A)$. On dit que $f$ est \emph{localement 
acyclique en $x$, res. $K$}, si pour tout point géométrique $t$ de 
$S_{f(x)}$ on a $K_x=\eR\Gamma(X_x,K)\iso \eR\Gamma(X_{x,t},K)$. On dit que $f$ 
est \emph{localement acyclique, rel. $K$}, si c'est vrai pour tout $x$, et 
\emph{universellement localement acyclique, rel. $K$}, si cela reste vraie 
après tout changement de base $S' \to S$.
\end{definition_}





\begin{theorem_}\label{VII:2-13}
Soit $f:X\to S$ un morphisme de type fini et $\sF$ constructible sur $X$. Il 
existe un ouvert dense $U$ de $S$ au-dessus duquel $f$ est universellement 
localement acyclique, rel. $\sF$. 
\end{theorem_}

Nous admettrons le résultat suivant, prouvé dans l'appendice. 





\begin{lemma_}\label{VII:2-14}
Soit $X\xrightarrow f S_1 \xrightarrow g S_2$. Si $g$ est lisse, et $f$ 
universellement localement acyclique rel. $K$, alors $g f$ l'est aussi. 
\end{lemma_}

On se ramène à supposer $S$ intègre, de point générique $\eta$, et 
on procède par récurrence sur $\dim{X_\eta}$. On commence par déduire de 
hypothèse de récurrence que 
\begin{equation*}\tag{A}\label{VII:eq:2-14-1}
  \begin{array}{c}
    \text{Quitte à rétrécir $S$, il existe $T\subset X$, fini sur $S$, tel que $f$} \\
    \text{soit universellement localement acyclique en dehors de $T$.} 
  \end{array}
\end{equation*}

Cette question étant locale, on se localise sur $X$ et on factorise $f$ en 
$X\xrightarrow u \dA_S^1 \xrightarrow S$. L'hypothèse de récurrence 
s'applique à $u$ et on conclut par \ref{VII:2-14} et les arguments habituels. 

Pour prouver le théorème, on peut supposer $X$ propre, puisque le 
problème est local. On rétrécit $S$ pour que les $\eR^i f_\ast \sF$ 
soient localement constants et que \eqref{VII:eq:2-14-1} soit applicable, et on 
utilise le lemme suivant. 





\begin{lemma_}\label{VII:2-15}
Soient $f:X\to S$, propre, et $\sF$ constructible. On suppose que les 
$\eR^i f_\ast \sF$ sont localement constants et que, en dehors de $T\subset X$ 
fini sur $S$, $f$ localement acyclique rel. à $\sF$. Alors, $f$ est 
localement acyclique rel. à $\sF$. 
\end{lemma_}

Soient $s$ un point géométrique de $S$, $S_s$ le localisé strict de $S$ 
en $s$ et $t$ un point géométrique de $S_s$. Soient $X_{(s)}$ l'image 
réciproque de $X$ sur $S_s$, $\bar j:X_t \to X_{(s)}$ et 
$i:X_s \hookrightarrow X_{(s)}$. Notant encore $\sF$ l'image réciproque de 
$\sF$ sur $X_{(s)}$ ou $X_t$, il faut prouver que 
$i^\ast \sF \iso i^\ast \eR \bar j_\ast \sF$. Soit $\Delta$ le mapping cylinder 
de cette application. Ses faisceaux de cohomologie sont à support dans $T_s$. 
Pour prouver que $\Delta=0$, il suffit donc de prouver que 
$\eR\Gamma(X_s,\Delta)=0$. C'est là le mapping cylinder de 
$(\eR f_\ast \sF)_s = \eR\Gamma(X_s,\sF) \to \eR\Gamma(X_s,i^\ast \eR\bar j_\ast \sF) = \eR\Gamma(X_{(s)},\eR\bar j_\ast \sF) = \eR\Gamma(X_t,\sF) = (\eR f_\ast \sF)_t$. 
Ce morphisme de spécialisation est par hypothèse un isomorphisme, et 
$\Delta=0$. 





\begin{corollary_}\label{VII:2-16}
Pour $S$ le spectre d'un corps, tout $S$-schéma $X$ est universellement 
localement acyclique, rel. $K$, quel que soit $K\in\ob\eD^+(X,A)$. 
\end{corollary_}

Se déduit de \ref{VII:2-13} par passage à la limite (possible, car $U$ de 
\ref{VII:2-13} est toujours égal à $S$. 










\section{Preuve de \texorpdfstring{\ref{VII:1-1}}{1.1} et constructibilité des faisceaux de cycles évanescents}\label{VII:3}





\subsection{}\label{VII:3-1}

Soit $S$ un trait strictement local (le spectre d'un anneau de valuation 
discrète hensélien à corps résiduel séparablement clos). On note $s$ et 
$\eta$ ses points fermé et générique, et $\bar\eta$ un point 
géométrique localisé et $\eta$ (une clôture séparable de $k(\eta)$). 
Soient $X$ sur $S$ et $\sF$ un faisceau sur $X_\eta$. Rappelons la 
définition des faisceaux de cycles évanescents $\eR^i \Psi_\eta(\sF)$ (des 
faisceaux sur $X_s$, muni d'une action de $\gal(\bar\eta/\eta)$). Soient 
$i:X_s\hookrightarrow X$, $j:X_\eta \hookrightarrow X$, et $\bar j$ le 
composé $X_{\bar\eta}\to X_\eta \hookrightarrow X$; on a 
\[
  \eR^i \Psi_\eta(\sF) = i^\ast \eR^i \bar j_\ast (\bar j^\ast \sF) \text{.} 
\]

Voici le théorème principal de ce numéro. Il améliore 
\cite[XIII 2.3.1, 2.4.2]{sga7}. 





\begin{theorem_}\label{VII:3-2}
On suppose que $X$ de type fini sur $S$ et $\sF$ constructible. Alors, les 
faisceaux de cycles évanescents $\eR^i \Psi_\eta(\sF)$ sont constructibles. 
\end{theorem_}

On procède par récurrence sur la dimension $n$ de $X_\eta$. On peut par 
ailleurs supposer -- et on suppose -- $X_\eta$ dense dans $X$ (remplacer $X$ 
par $\bar X_\eta$ ne change pas les $\eR^i \Psi$, qui sont à support dans 
$\bar X_\eta$). 





\begin{lemma_}\label{VII:3-3}
Si \ref{VII:3-2} est vrai en dimension de $X_\eta<n$, et que $\dim X_\eta=n$, 
il existe des sous-faisceaux constructibles $\sG_i$ des $\eR^i \Psi_\eta(\sF)$ 
tels que les supports des sections locales de $\eR^i\Psi_\eta(\sF)/\sG$ soient 
finis. 
\end{lemma_}

Soit $s'$ un point générique géométrique de la droite affine sur $s$, 
et soit $S'$ le localisé strict en $s'$ de la droite affine $\dA_S^1$ sur 
$S$. C'est encore un trait strictement local, et les uniformisants pour $S$ 
sont des uniformisantes sur $S'$. 
\[\xymatrix{
  & \spec(k') \ar[rr] \ar[dd] \ar[dr] 
    & & (S',\eta',s') \ar[dr] \ar[dd] \\
  \bar\eta' \ar[ur] \ar[dr] 
    & & \dA_{\bar S}^1 \ar[d]  
    & & \dA_S^1 \ar[dl] \\
  & \bar\eta\ar@{^{(}->}[r] 
    & \bar S \ar[r] 
    & (S,\eta,s) 
}\]
Soit $k'=k(\bar\eta)\otimes_{k(\eta)}k(\eta')$. Si $\bar S$ est le normalisé 
de $S$ dans $\bar\eta$, $k'$ est le corps des fractions du localisé strict 
de $\dA_{\bar S}^1$ en $s'$; c'est donc un corps, et 
$\gal(\bar\eta/\eta)=\gal(k'/\eta')$. Soit $\bar\eta'$ le spectre d'une 
clôture algébrique de $k'$. Le groupe de Galois $P=\gal(\bar\eta'/k')$ est 
un pro-$p$-groupe, pour $p$ l'exposant caractéristique de $k(s)$. 





\begin{lemma_}\label{VII:3-4}
Pour tout schéma $X'$ sur $S'$, et tout faisceau $\sF$ sur 
$X_\eta'=X_{\eta'}'$, on a entre les faisceaux de cycles évanescents pour 
$X'/S'$ et pour $X'/S$ la relation 
$\eR^i \Psi_\eta(\sF) =\eR^i\Psi_{\eta'}(\sF)^P$. 
\end{lemma_}

Le diagramme suivant compare les morphismes utilisés dans la définition de 
$\eR\Psi$ pour $X'/S'$ et $X'/S$: 
\[\xymatrix{
  X_{\bar\eta'} \ar[rr] \ar[d] 
    & & X_{\eta'}' \ar@{=}[dd] \ar@{^{(}->}[dr] 
    & & S' \\ 
  X_{\eta'}\times_{\eta'}\spec(k') \ar@{=}[d] \ar[urr] 
    & & & X \ar[ur] \ar[dr] \\
  X_\eta' \ar[rr] 
    & & X_\eta' \ar@{^{(}->}[ur] 
    & & S
}\]
et le lemme résulte de la suite spectrale de Hochschild-Serre, compte tenu de 
ce que $P$ est un pro-$p$-groupe, avec $p$ inversible dans $A$. 

Prouvons \ref{VII:3-3}. La question est locale; ceci permet de supposer $X$ 
affine, $X\subset \dA_S^n$. Soient $f$ l'une des projection 
$X\subset \dA_S^n\to \dA_S^1$. $X'$ le ``localisé'' $X_{\dA_S^1} S'$ de $x$, 
et $\sF'$ l'image inverse de $\sF$ sur $x'$ 
\[\xymatrix{
  \sF'\text{ sur }X' \ar[r] \ar[d]^-\lambda 
    & S' \ar[d]^-\lambda \\
  \sF \text{ sur }X \ar[r]^-f 
    & \dA_S^1 \ar[r] 
    & S \text{.} 
}\]
On a, sur $X_s'=X_{s'}'$, 
\[
  \lambda^\ast \eR^i\Psi_\eta(\sF) = \eR^i \Psi_\eta(\sF') = \eR^i\Psi_{\eta'}(\sF')^P \text{:} 
\]
ce faisceau est constructible, car l'hypothèse de récurrence s'applique à 
$X'/S'$, et on utilise le lemme suivant appliqué à $X_s\subset \dA_S^n$ et 
aux faisceaux de cycles évanescents. 





\begin{lemma_}\label{VII:3-5}
Soient $\dA^n$ l'espace affine type de dimension $n$ sur un corps $k$, 
$X\subset \dA^n$, $]sF$ un faisceau de $A$-modules sur $X$ et $\bar\eta$ un 
point générique géométrique de $\dA^1$. Soient $X_{\bar\eta,i}$ la 
fibre générique géométrique de 
$X\subset \dA^n\xrightarrow{\pr_i} \dA^1$ et $\sF_{\bar\eta,i}$ la restriction 
de $\sF$ à $X_{\bar\eta,i}$. Si les $\sF_{\bar\eta,i}$ sont constructibles, 
il existe $\sF'\subset \sF$, constructible, tel que les sections locales de 
$\sF/\sF'$ soient à support fini. 
\end{lemma_}

Par passage à limite, pour chaque $i$
\begin{enumerate}[a)]
  \item Il existe un voisinage étale $U$ de $\bar\eta$ et un faisceau 
    constructible $\sH$ sur $X_{U,i} = X\times_{\dA^1,\pr_i} U$ de restriction 
    $\sF_{\bar\eta,i}$ à $X_{\bar\eta,i}$. 
  \item Pour $U$ convenable, l'isomorphisme 
    $\sH_{\bar\eta,i} \iso \sF_{\bar\eta,i}$ provient de $u:\sH\to \sF_{U,i}$ 
    où $\sF_{U,i}$ est l'image réciproque de $\sF$ sur $X_{U,i}$. Soit 
    $\varphi:X_{U,i} \to X$. Le morphisme $u$ définit 
    $v_i:\varphi_! \sH \to \sF$, d'image $\sF_i'$. On a 
    $(\sF/\sF_i')_{\bar\eta,i} = 0$ et on prend pour $\sF'$ la somme des 
    $\sF_i'$. 
\end{enumerate}





\subsection{}\label{VII:3-6}

Prouvons \ref{VII:3-2} en dimension $n$. On peut supposer $X$ affine (car le 
problème est local sur $X_s$), puis projectif sur $S$ (prendre l'adhérence 
de $X$ dans un plongement projectif). On a alors une suite spectrale 
\cite[I 2.2.3]{sga7} (la suite spectrale de Leray pour $\bar j$, compte tenu du 
théorème de changement de base pour le morphisme propre $X\to S$). 
\begin{equation*}\tag{1}\label{VII:eq:3-6-1}
  E_2^{p q} = \h^p(X_s,\eR^i\Psi_\eta(\sF)) \Rightarrow \h^{p+q}(X_{\bar\eta},\sF) \text{.} 
\end{equation*}
Soit $\sG^q\subset \eR^q \Psi_\eta(\sF)$ comme en \ref{VII:3-3}, et soit 
$\sH^q$ le faisceau quotient. On a $\h^i(X_s,\sH^q)=0$ pour $i\ne 0$, et pour 
que $\sH^q$, donc $\eR^q\Psi_\eta(\sF)$ soit constructible, il suffit que 
$\h^0(X_s,\sH^q)$ soit de type fini. 

Calculons \eqref{VII:eq:3-6-1} modulo modules de type fini (i.e. dans la 
catégorie quotient de celle des $A$-modules par la sous-catégorie épaisse 
des $A$-modules de type fini). La suite exacte longue de cohomologie déduit 
de $0 \to \sG^q \to \eR^q\Psi_\eta(\sF) \to \sH^q \to 0$ fournit, par 
\ref{VII:1-10} appliqué à $\sG^q$, que $E_2^{p q} \sim \h^q(X_s,\sH^q)$. 
En particulier, $E_2^{p q}\sim 0$ pour $p\ne 0$, 
$E_2^{0 q}\sim \h^q(X_{\bar\eta},\sF) \sim 0$, et ceci achève la 
démonstration. 

Des arguments parallèles fournissent le résultat suivant, qui améliore 
\cite[XIII 2.1.12, 2.4.2]{sga7}. 





\begin{proposition_}\label{VII:3-7}
La formation des faisceaux de cycles évanescents est compatible aux 
changements de traits. 
\end{proposition_}

Soient $g:(S',\eta',s') \to (S,\eta,s)$ un morphisme (surjectif) de traits 
strictement locaux, $X/S$, $\sF$ sur $X_\eta$, de torsion premier à la 
caractéristique résiduelle et $(X',\sF')$ leurs images réciproques sur 
$S'$. Notant encore $g$ les morphismes parallèles à $g$, tels 
$X_{s'}' \to X_s$, il faut prouver que 
\begin{equation*}\tag{3.7.1}\label{VII:eq:3-7-1}
\xymatrix{
  g^\ast \eR\Psi_\eta(\sF) \ar[r]^-\sim 
    & \eR\Psi_{\eta'}(\sF') \text{.} 
}
\end{equation*}
Ce morphisme est défini par le diagramme commutatif 
\[\xymatrix{
  X_{s'}' \ar@{^{(}->}[r]^-{i'} \ar[d]^-g 
    & X' \ar[d]^-g 
    & X_{\eta'}' \ar[l]_-{\bar j'} \ar[d]^-g \\
  X_s \ar@{^{(}->}[r]^-i 
    & X 
    & \ar[l]_-{\bar j} X_{\bar\eta} 
}\]

Un passage à la limite ramène à supposer $X$ de type fini sur $S$, et on 
procède par récurrence sur $\dim{X_\eta}$. La question étant locale, on 
peut supposer $X$ affine, puis projectif. Supposons \eqref{VII:eq:3-7-1} pour 
$\dim{X_\eta}<n$. Pour $\dim{X_\eta}=n$, il résulte alors des deux 
assertions suivantes, où $\Delta$ est le mapping cylinder de 
\eqref{VII:eq:3-7-1}. 

\begin{enumerate}[(A)]
  \item Le support des sections locales des faisceaux de cohomologie de 
    $\Delta$ est fini. 
    \phantomsection\hypertarget{VII:3-A}{}
  \item $\eR\Gamma(X_s,\Delta)=0$, donc $\Delta=0$, d'après 
    (\hyperlink{VII:3-A}{A}). 
    \phantomsection\hypertarget{VII:3-B}{}
\end{enumerate}





\subsection{}\label{VII:3-8}

La preuve de (\hyperlink{VII:3-A}{A}) est parallèle à celle de 
\ref{VII:3-3}. On se localise sur $X$ et on considère des applications 
$X\to \dA_S^1$; notant $S(x)$ l'hensélisé stricte de $\dA_S^1$ au point 
générique de la fibre spéciale, on applique l'hypothèse de récurrence 
à $X_1/S(x)$ ($X_1=X\times_{\dA^1} S(x)$) et à $S'(x)$. Prenant les 
invariants par un pro-$p$-groupe, on trouve que $\Delta$ est nul sur la fibre 
générique géométrique de $X_{s'} \to \dA_s^1$. Ceci étant vrai, 
localement sur $X$ et pour tout projection, on a (\hyperlink{VII:3-A}{A}). 





\subsection{}\label{VII:3-9}

Pour (\hyperlink{VII:3-B}{B}), la propreté de $X$ sur $S$ assure que 
\[
  \eR\Gamma(X_{s'}',\eR\Psi_{\eta'}(\sF')) = \eR\Gamma(X_{\bar\eta}',\sF') 
\]
et
\[
  \eR\Gamma(X_s,\eR\Psi_\eta(\sF)) = \eR\Gamma(X_{\bar\eta},\sF) \text{.} 
\]
On a 
$\eR\Gamma(X_s,\eR\Psi_\eta(\sF))\iso \eR\Gamma(X_{s'}',g^\ast\eR\Psi_\eta(\sF))$
(invariance par changement de corps séparablement clos de base); 
$\eR\Gamma(X_{s'}', \Delta)$ est donc le mapping cylinder de 
$\eR\Gamma(X_{\bar\eta},\sF) \to \eR\Gamma(X_{\bar\eta}',\sF')$, un 
isomorphisme par la même théorème d'invariance. 





\subsection{}\label{VII:3-10}

Prouvons \ref{VII:1-1}. On se ramène à supposer $S$ connexe, donc 
intègre. Si $\dim S=0$, ($S$ spectre d'un corps); on applique \ref{VII:1-9}, 
\ref{VII:1-10}. Si $S$ est de dimension $1$, le théorème \ref{VII:1-9} 
assure que les $\eR^i f_\ast \sF$ sont constructibles au-dessus du complément 
d'un ensemble fini $T$ de points de $S$. Il reste à voir que leur restriction 
aux $Y_t$ ($t\in T$) sont constructibles. Il suffit de le voir après 
localisation en $t$: on peut supposer que $S$ est un trait strictement local. 
Le cas essentiel est le suivant. 





\begin{lemma_}\label{VII:3-11}
Soient $X$ de type fini sur $S$, $j:X_\eta\hookrightarrow X$, 
$i:X_s\hookrightarrow X$ et $\sF$ constructible sur $X_\eta$. Alors, 
$i^\ast\eR j_\ast \sF$ est constructible. 
\end{lemma_}

Soit $I=\gal(\bar\eta/\eta)$ le groupe d'inertie. On sait que c'est une 
extension de $\widehat\dZ_{p'}(1)$ par un pro-$p$-groupe $P$ ($p$ exposant 
caractéristique du corps résiduel). La suite spectrale de Hochschild-Serre 
donne 
\[
  E_2^{p q} = \sH^p(I,\eR^q\Psi_\eta(\sF))\Rightarrow i^\ast \eR^{p+q}j_\ast\sF \text{.} 
\]
Soit $R_t^q = \eR^q\Psi_\eta(\sF)^P$ ($t$ pour modéré). Si $\sigma$ est un 
générateur de $\widehat\dZ_{p'}(1)$, on a 
$E_2^{0 q} = \ker(\sigma-1,R_t^q)$, 
$E_2^{1 q} = \operatorname{coker}(\sigma-1,R_t^q)$ et $E_2^{p q} = 0$ pour 
$p\ne 0,1$. Ces faisceaux sont constructibles, puisque les faisceaux de cycles 
évanescents le sont, et \ref{VII:3-11} en résulte. 





\subsection{}\label{VII:3-12}

Traitons le cas général, $S$ étant toujours un trait strictement 
hensélien. Si $X=X_\eta$, $f$ est le composé $X\to Y_\eta \to Y$ et on 
applique \ref{VII:1-9} et \ref{VII:3-11}. Dans le cas général, soit 
$j:X_\eta\hookrightarrow X$ et soit $\Delta$ le mapping cylinder de 
$\sF \to \eR j_\ast j^\ast \sF$. Ses faisceaux de cohomologie sont à support 
dans $X_s$; d'après \ref{VII:1-9} appliqué à $X_s \to Y_{s'}$, 
$\eR f_\ast \Delta$ est donc constructible. Le complexe 
$\eR f_\ast \eR j_\ast j^\ast \sF = \eR(f j)_\ast j^\ast \sF$ l'est 
également, et on conclut par le triangle 
\[\xymatrix{
  \ar[r] 
    & \eR f_\ast \sF \ar[r] 
    & \eR f_\ast \eR j_\ast j^\ast \sF \ar[r] 
    & \eR f_\ast \Delta \ar[r] 
    & \text{.} 
}\]










\section{Bidualité locale}\label{VII:4}





\subsection{}\label{VII:4-1}

Soit $S$ régulier de dimension $0$ ou $1$ et posons $K_S=A$ (faisceau 
constant de valeur $A$). Pour $K\in\ob\eD_\text{ctf}^b(S,A)$, on pose 
$D K=\rHom(K,K_S)\in\ob\eD^+(S,A^\circ)$ (un complexe de faisceaux de 
$A$-modules à droites). On a encore $D K\in\ob\eD_\text{ctf}^b(S,A^0)$, et 
un calcul explicite, possible car $S$ est de dimension $1$ montre 
$K\iso D D K$ (pour $A=\dZ/n$, \hyperref[V]{Dualité} \ref{V:1-4}). 





\subsection{}\label{VII:4-2}

On supposera que $A$ est commutatif, hypothèse sans doute inutile. Pour 
$a:X\to S$ de type fini sur $S$, on pose $K_X=\eR a^! K_S$. Pour 
$K\in\ob\eD_\text{ctf}^b(S,A)$, on pose 
$D K=\rHom(K,K_X)\in\ob\eD_\text{ctf}^b(X,A)$. Notre résultat principal est 
le suivant 





\begin{theorem_}\label{VII:4-3}
On a $K\iso D D K$ ($K\in \ob\eD_\textnormal{ctf}^b(X,A)$). 
\end{theorem_}





\begin{lemma_}\label{VII:4-4}
Si $a$ est propre, on a $\eR a_\ast K \iso \eR a_\ast D D K$. 
\end{lemma_}

La dualité de Poincaré 
\[
  \eR a_! \rHom(K,\eR a^! K_S) = \rHom(\eR a_! K,K_S) 
\]
nous fournit un isomorphisme $\eR a_\ast D = D \eR a_! = D \eR a_\ast$. Le 
lemme résulte de la commutativité du diagramme (\hyperref[V]{Dualité}, 
\ref{V:1-2}) 
\[\xymatrix{
  \eR a_\ast K \ar[r] \ar[d]^-\wr 
    & \eR a_\ast D D K \ar[d]^-\wr \\
  D D \eR a_\ast K \ar[r]^-\sim 
    & D \eR a_\ast D K \text{.} 
}\]





\subsection{}\label{VII:4-5}

On voit par localisation qu'il suffit de traiter les cas où $S$ est spectre 
d'un corps, ou un trait. Pour $S$ spectre d'un corps, on procède par 
récurrence sur $\dim X$. Le problème étant local, on peut supposer $X$ 
propre. Par ailleurs, l'hypothèse de récurrence assure, par les arguments 
habituels, que le mapping cylinder $\Delta$ de $K\to D D K$ a des faisceaux de 
cohomologie en gratte-ciel. Le lemme assure alors que $\eR a_\ast \Delta=0$, 
donc que $\Delta=0$. 





\subsection{}\label{VII:4-6}

Pour $S$ un trait, on sait déjà que \ref{VII:4-3} vaut sur la fibre 
générique (\ref{VII:4-5}). On peut aussi supposer $X$ propre sur $S$ et on 
procède par récurrence sur $\dim{X_s}$, $X$ propre sur $S$. Le mapping 
cylinder $\delta$ a encore des faisceaux de cohomologie en gratte ciel, sur 
$X_s$, et on conclut comme plus haut. 





\subsection{}\label{VII:4-7}

Supposons que $A=\dZ/m$. On a alors une variante de la théorie 
précédente, en prenant $K$ quelconque dans $\eD_c^b(X,A)$. Le complexe 
$\eR a^! \dZ/m$ est de dimension injective finie \cite[XVIII 3.1.7]{sga4}. Ceci 
assure que $D K$ est encore dans $\eD_c^b(X,A)$. Pour prouver la bidualité 
locale sur $S$, on utilise que $\dZ/m$ est le $\dZ/m$-module dualisant. 
Ensuite, on procède comme ci-devant. 










\section{Appendice, par L.\ Illusie}\label{VII:5}

Dans cet appendice, qui reprend certaines partes de feu \cite[II]{sga5}, et 
une lettre de P.\ Deligne à l'auteur, nous prouvons \ref{VII:2-14} ainsi que 
diverses généralisations et variantes des théorèmes de spécialisation 
\cite[XVI 2.1]{sga4} et (\hyperref[I]{Cohomologie étale} \ref{I:5-1-7}). 

Les schémas et morphismes considérés seront supposés quasi-compacts et 
quasi-séparés. Si $X$ est un schéma et $x$ un point géométrique de 
$X$, on notera $X(x)$ le localisé strict de $X$ en $x$. 

On fixe un schéma de base $S$ et un faisceau d'anneaux $A$ sur $S$ (non 
nécessairement commutatif ni noethérien). Si $X$ est un $S$-schéma, on 
écrira $\eD(X)$ pour $\eD(X,A_X)$. 





\subsection{Propreté cohomologique}\label{VII:5-1}





\begin{proposition}\label{VII:5-1-1}
Soient $f:X\to Y$ un $S$-morphisme et $E\in\ob \eD^+(X)$. Les conditions 
suivantes sont équivalentes: 
\begin{enumerate}[\indent (i)]
  \item La formation de $\eR f_\ast E$ commute à tout changement de base 
    fini $S'\to S$.
  \item La formation de $\eR f_\ast E$ commute à tout changement de base 
    quasi-fini (ou limite projective de morphismes quasi-finis) $S'\to S$. 
  \item Pour tout flèche de spécialisation $i:t\to S(s)$ 
    (\hyperref[I]{Cohomologie étale}, \ref{I:5-1-2}), si l'on désigne par 
    $\widetilde S$ le normalisé dans $k(\bar t)$ du schéma intègre 
    adhérence de $i(t)$ dans $S(s)$ (d'après 
    \cite[IV, 6.15.5, 18.6.11, 18.8.16]{ega}) $\bar S$ est donc local 
    intègre, de point fermé radiciel sur $s$), $\bar f:\bar X\to \bar Y$ 
    le morphisme déduit de $f$ par le changement de base $\bar S\to S$, 
    $f_s:X_s\to Y_s$ la fibre de $f$ en $s$, alors la flèche de changement 
    de base 
    \[
      (\eR \bar f_\ast (E|\bar X))|Y_s \to \eR {f_s}_\ast(E|X_s) 
    \]
    est un isomorphisme. 
\end{enumerate}
\end{proposition}

L'équivalence de (i) et (ii) résulte du théorème principal de Zariski, 
et il est clair que (ii) implique (iii). Prouvons que (iii) implique (ii). Pour 
tout $S'\to S$, limite projective de morphismes quasi-finis, et tout point 
géométrique $s$ de $S'$, notons $C(S',s)$ le cône de la flèche de 
changement de base 
\[
  (\eR f_\ast'(E|X'))|Y_s \to \eR{f_s}_\ast (E|X_s) \text{,} 
\]
où $f':X'\to Y'$ est le morphisme déduit de $f$ par le changement de base 
$S'\to S$ et $f_s:X_s\to Y_s$ la fibre de $f$ en $s$. Il s'agit de prouver que, 
pour tout $(S',s)$ comme ci-dessus, $C(S',s)$ est acyclique. On va montrer, par 
récurrence sur $N$, que $\h^i C(S',s) = 0$ pour $i\leqslant N$. Par le Main 
Theorem, passage à la limite, et nettoyage de morphismes radiciels, on peut 
supposer que $S$ et $S'$ sont strictement locaux, $S'$ intègre, de point 
fermé $s$, $S'\to S$ fini. Soient $t$ un point géométrique générique 
de $S'$, et $\bar S$ le normalisé de $S'$ dans $k(t)$. Soit $\bar S_\bullet$ 
le schéma simplicial cosquelette de $S'\to S'$ 
($\bar S_0=\bar S$,\ldots $\bar S_n=(\bar S/S')^{n+1}$,\ldots), 
$X_\bullet=X\times_S \bar S_\bullet$. $g_\bullet:\bar X_\bullet \to X'$ la 
projection canonique. Posons $G|X'=G'$. Le bicomplexe 
${g_\bullet}_\ast g_\bullet^\ast G'$, de colonnes les 
${g_n}_\ast g_n^\ast G'$ est une résolution de $G'$ \cite[VIII 8]{sga4}. On 
peut donc représenter $C(S',s)$ par un bicomplexe dont les colonnes 
s'identifient aux $C(\bar S_n,s)$. D'après (iii), $C(\bar S_0,s)=C(\bar S,s)$ 
est acyclique. D'autre part, par $n>0$, on a $\h^i(\bar S_n,s)=0$ pour 
$i\leqslant N$ par l'hypothèse de récurrence. On en conclut que 
$\h^i C(S',s)=0$ pour $i\leqslant N+1$, ce qui achève la démonstration. 





\subsubsection{Remarque}\label{VII:5-1-2}

Soit $(f,E)$ vérifiant la condition (i) de \ref{VII:5-1-1}. 
\begin{enumerate}[a)]
  \item Si $E\in \ob \eD^b(X)$, et $f_\ast$ est de dimension cohomologique 
    finie, alors, pour tout $F\in \ob\eD^-(S,A^\circ)$, la flèche canonique 
    \begin{equation*}\tag{$*$}\label{VII:eq:5-1-1-1}
      \eR f_\ast(E)\lotimes_A q^\ast F \to \eR f_\ast(E\lotimes_A p^\ast F) \text{,} 
    \end{equation*}
    où $p:X\to S$, $q:Y\to S$ sont les projections, est un isomorphisme. 
    Même conclusion avec $E\in\ob\eD^-(X)$. 
\end{enumerate} 

Par dévissage, il suffit en effet de vérifier l'assertion pour $F$ de la 
forme $i_! A_{S'}$, avec $i:S'\to S$ quasi-fini. Par le Main Theorem, on se 
ramène à supposer $i$ fini, la conclusion équivaut alors à la 
commutation de $\eR f_\ast(E)$ au changement de base par $i$. 
\begin{enumerate}[a)]
\setcounter{enumi}{1}
  \item On suppose $A$ noethérien, constructible. Alors, p our tout 
    $F\in\ob\eD^b(S,A^\circ)$, constructible et de tor-dimension finie, la 
    flèche \eqref{VII:eq:5-1-1-1} ci-dessus est un isomorphisme. 
\end{enumerate}

On se ramène en effet, par dévissage, au cas où $F$ est de la forme 
$i_! M$, avec $i:S'\to S$ localement fermé, $A$ localement constant sur $S'$, 
$M$ constructible, de tor-dimension finie, et à cohomologie localement 
constante sur $S'$. Changeant de base de $S$ à $S'$, on est donc ramené au 
cas où $F$ est parfait \cite[I]{sga6}, donc finalement, par localisation et 
dévissage, au cas où $F=A$, pour lequel la conclusion est triviale. 





\subsubsection{}\label{VII:5-1-3}

Soient $f:X\to Y$ un $S$-morphisme, et $E\in\ob\eD^+(X)$. Nous dirons que 
$(f,E)$ est \emph{cohomologiquement propre} rel.\ à $S$ si la formation de 
$\eR f_\ast E$ commute à tout changement de base $S'\to S$. Il revient au 
même de dire qu'après tout changement de base $S'\to S$, la formation de 
$\eR f_\ast' E'$ (où $f'=f\times_S S'$, $E'=$image inverse de $E$ sur 
$X\times_S S'$) commute à tout changement de base fini $S'' \to S'$ (cf. 
\cite[XII 6.1]{sga4}). Voici quelques exemples de propreté cohomologique 
(pour une étude de la notion analogue pour les faisceaux d'ensembles ou de 
groupes non commutatifs, le lecteur se reportera à \cite[XIII]{sga1}). 
\begin{enumerate} % originally 1.3.1, 1.3.2, ...
  \item On suppose $A$ de torsion et $f$ propre. Alors, pour tout 
    $E\in\ob\eD^+(X)$, $(f,E)$ est cohomologiquement propre rel.\ à $S$ 
    (théorème de changement de base propre, \cite[XII 5.1]{sga4}). 
  \item On suppose $S$ fini, $X$ et $Y$ de type fini sur $S$, et $A$ annulé 
    par un entier inversible sur $S$. Alors, pour tout $E\in\ob\eD^+(X)$, 
    $(f,E)$ est cohomologiquement propre rel.\ à $S$ 
    (\ref{VII:1-9}). 
  \item On suppose $A$ constant, noethérien, annulé par un entier $n$ 
    inversible sur $S$, $f:X=Y\setminus D\to Y$ l'inclusion du complément 
    d'un diviseur $D$ sur $Y$, à croisements normaux rel. à $S$ 
    \cite[XIII 2.1]{sga1}. Soit $E$ un faisceau de $A$-modules sur $X$ 
    vérifiant l'une des conditions suivantes: 
    \begin{enumerate}[(i)]
      \item $E$ est localement constant constructible et modérément 
        ramifié le long de $D$; 
      \item $E$ est localement pour la topologie étale sur $Y$ et sur $S$ 
        l'image inverse d'un faisceau de $A$-modules sur $S$. 
    \end{enumerate}
    Alors $(f,E)$ est cohomologiquement propre rel.\ à $S$, et $\eR f_\ast E$ 
    constructible (comparer avec \cite[XIII 2.4]{sga1}). 
\end{enumerate} 

Esquissons la démonstration dans le cas (i). La question est locale au 
voisinage d'un point $y$ de $Y$. On peut supposer $D$ somme de diviseurs 
lisses, $D=\sum_{i=1}^m D_i$. En vertu du lemme d'Abhyankar relatif 
\cite[XIII 5.5]{sga1}, on peut, au voisinage de $y$, trivialiser $E$ par un 
revêtement fini $\widetilde Y\to Y$ de la forme 
$\widetilde Y=Y[T_1,\dots,T_r]/(T_1^{n_1}-t_1,\dots,T_r^{n_r}-t_r)$ où les 
$t_i$ sont des équations locales des diviseurs lisses passant par $y$, et les 
$n_i$ des entiers premiers à la car.\ de $k(y)$. L'image inverse 
$\widetilde D$ de $D$ dans $Y$ est encore à croisements normaux rel.\ à 
$S$, et si $g:\widetilde X=\widetilde Y\setminus \widetilde D\to X$ est la 
projection, $g^\ast E$ est constant. Comme $E$ s'injecte dans $g_\ast g^\ast E$ 
et que le quotient est modérément ramifié, un dévissage facile ramène 
au cas où $E$ est constant. Notons $p:X\to S$, $q:Y\to S$ les 
projections, et, pour $1\leqslant i\leqslant m$, $f_i:Y\setminus D_i \to D$ 
l'inclusion canonique. Du théorème de pureté relatif \cite[XVI 3.7]{sga4} 
on déduit aisément, par récurrence sur $m$, que, pour tout $A$-module 
localement constant $M$ sur $S$, la flèche canonique 
\begin{equation*}\tag{5.1.3.1}\label{VII:eq:5-1-3-3-1} % originally 1.3.3.1
  \eR f_\ast (\dZ/n)\lotimes q^\ast M\to \eR f_\ast (p^\ast M) 
\end{equation*}
est un isomorphisme, et que d'autre part on a: 
\begin{align*}\tag{5.1.3.2}\label{VII:eq:5-1-3-3-2}
  {f_i}_\ast(\dZ/n) &= \dZ/n \text{,} \\
  \eR^1 {f_i}_\ast (\dZ/n) &= (\dZ/n)(-1)_{D_i} \text{,} 
\end{align*}
un isomorphisme canonique étant donné par la classe fondamental $\cl(D_i)$ 
(\hyperref[IV]{Cycle} \ref{IV:2-1-4}), 
\begin{align*}
  \eR^q {f_i}_\ast(\dZ/n) &= 0\text{ pour $q>1$,} \\
  \eR f_\ast (\dZ/n) &= \bigotimes_i^\eL \eR {f_i}_\ast (\dZ/n) \text{ , i.e.:} \\
  \eR^1 f_\ast(\dZ/n) &= \bigoplus_{i=1}^m (\dZ/n)(-1)_{D_i} \text{,} \\
  \textstyle\bigwedge^\bullet \eR^1 f_\ast(\dZ/n) &\iso \eR^\bullet f_\ast (\dZ/n) \text{.} 
\end{align*}
La conclusion de 3, dans le cas (i), résulte donc de 
\eqref{VII:eq:5-1-3-3-1} et \eqref{VII:eq:5-1-3-3-2}. Dans le cas (ii), on se 
ramène, par passage à la limite, au cas où $E=p^\ast M$, avec $M$ 
constructible, puis, par dévissage et changement de base (utilisant (i)), au 
cas $M$ constant, déjà traité. 

Il découle de la démonstration précédente que \eqref{VII:eq:5-1-3-3-1} 
est un isomorphisme pour tout $M\in\ob\eD^+(S)$. 

On comparera les formules \eqref{VII:eq:5-1-3-3-2} aux formule analogues en 
cohomologie entière dans le cas $S=\spec(\dC)$ (voir par exemple 
\cite[3.1]{de71-2}), qui résultent de ce que, localement pour la topologie 
classique, le complément de $D$ a le type d'homotopie d'un tore. 

Par passage à la limite, on déduit de \eqref{VII:eq:5-1-3-3-2} des formules 
similaires en cohomologie $\ell$-adique, i.e. avec $\dZ/n$ remplacé par 
$\dZ_\ell$ ($\ell$ un nombre premier inversible sur $S$). Supposons 
$q:Y\to S$ propre et lisse. Il découle alors des conjectures de Weil 
\cite{de74,de80} que la suite spectrale de Leray 
\[
  E_2^{i j} = \eR^i q_\ast \eR^j f_\ast \dZ_\ell \Rightarrow \eR^\bullet p_\ast \dZ_\ell 
\]
dégénère en $E_3$ modulo torsion \cite[\S 6]{de71-1}. Le même 
phénomène se produit en cohomologie entière dans le cas 
$S=\spec(\dC)$, voir (loc.\ cit.) et \cite[3.2.13]{de71-2}. 





\subsection{Spécialisation et cospécialisation}\label{VII:5-2}

Soient $f:X\to S$ et $E\in\ob\eD^+(X)$. 





\subsubsection{}\label{VII:5-2-1}

Supposons que la formation de $\eR f_\ast E$ commute à tout changement de 
base fini. Pour toute spécialisation $u:t\to S(s)$ de points géométriques 
de $S$, on définit alors une flèche, dite \emph{flèche de 
spécialisation}, 
\begin{equation*}\tag{5.2.1.1}\label{VII:eq:5-2-1-1}
  \spe(u):\eR\Gamma(X_s,E|X_s) \to \eR\Gamma(X_t,E|X_t) \text{,}
\end{equation*}
de la manière suivante. Soient, comme en \ref{VII:5-1-1}(iii), $\bar S$ le 
normalisé dans $k(t)$ de l'adhérence de $u(t)$ dans $S(s)$, et 
$\bar f:\bar X\to \bar S$ le morphisme déduit de $f$ par le changement de 
base $\bar S\to S$. D'après la partie triviale (i)$\Rightarrow$(iii) de 
\ref{VII:5-1-1}, la flèche de changement de base
\begin{equation*}\tag{5.2.1.2}\label{VII:eq:5-2-1-2}
  \eR\Gamma(\bar X,E|\bar X)=\eR\bar f_\ast (E|\bar X)_s \to \eR\Gamma(X_s,E|X_s) 
\end{equation*}
est un isomorphisme. On définit \eqref{VII:eq:5-2-1-1} comme composée de 
l'inverse de \eqref{VII:eq:5-2-1-2} et de la flèche de restriction 
\begin{equation*}\tag{5.2.1.3}\label{VII:eq:5-2-1-3}
  \eR\Gamma(\bar X,E|\bar X) \to \eR\Gamma(X_t,E|X_t) \text{.}
\end{equation*}





\subsubsection{}\label{VII:5-2-2}

Supposons que $f$ soit localement acyclique rel.\ à $E$ (\ref{VII:2-12}). 
Pour toute spécialisation $u:t\to S(s)$ de points géométriques de $S$, on 
définit alors une flèche, dite \emph{flèche de cospécialisation}, 
\begin{equation*}\tag{5.2.2.1}\label{VII:eq:5-2-2-1}
  \cosp(u):\eR\Gamma(X_t,E|X_t) \to \eR\Gamma(X_s,E|X_s) \text{,} 
\end{equation*}
définie de la manière suivante. Soient $g:X_t\to X$, $g':\bar X\to X$ les 
flèches canoniques. Il découle de l'hypothèse d'acyclicité locale que 
la fl\`che canonique 
\begin{equation*}\tag{5.2.2.2}\label{VII:eq:5-2-2-2}
  \eR g_\ast'(E|\bar X)\to \eR g_\ast(E|X_t) 
\end{equation*}
est un isomorphisme. En effet, on peut supposer $S$ strictement local, 
intègre, de point fermé $s$, $t$ étant un point géométrique 
générique. Calculons la fibre de \eqref{VII:eq:5-2-2-2} en un point 
géométrique $x$ en $X$, d'image $y$ dans $S$. Quitte à remplacer $S$ par 
le localisé strict de $S$ en $y$, on peut supposer que $y=s$. Comme 
$\bar S\to S$ est limite projective de morphismes finis, et radiciel en $s$, on 
trouve que $\eR g_\ast'(E|\bar X)_x = E_x$ et que la fibre que 
\eqref{VII:eq:5-2-2-2} en $x$ s'identifie à la restriction 
$E_x=\eR\Gamma(X(x),E)\to \eR\Gamma(X(x)_t,E|X(x)_t)$, laquelle est un 
isomorphisme par hypothèse. Appliquant $\eR\Gamma(X,-)$ à 
\eqref{VII:eq:5-2-2-2}, on en déduit que la flèche de restriction 
\eqref{VII:eq:5-2-1-3} est un isomorphisme. On définit \eqref{VII:eq:5-2-2-1} 
comme composée de l'inverse de \eqref{VII:eq:5-2-1-3} et de la flèche de 
changement de base \eqref{VII:eq:5-2-1-2}. 





\subsubsection{}\label{VII:5-2-3}

Si $s'' \xrightarrow{u'} s' \xrightarrow u s$ sont les flèches de 
spécialisation de points géométriques de $S$, on vérifie que l'on a 
\begin{align*}
  \spe(u')\spe(u'') &= \spe(u u') \text{,} \\
  \cosp(u) \cosp(u') &= \cosp(u u') \text{,} 
\end{align*}
chaque fois que les deux membres sont définis. 





\subsubsection{}\label{VII:5-2-4}

Supposons que la formation de $\eR f_\ast E$ commute à tout changement de 
base fini, et que $f$ soit localement acyclique rel.\ à $E$. Il découle des 
définitions que, pour tout spécialisation $u:t\to S(s)$ de points 
géométriques de $S$; les flèches $\spe(u)$ et $\cosp(u)$ sont des 
isomorphismes, inverses l'un de l'autre. 

Les hypothèses précédentes sont vérifiées notamment lorsque $A$ est 
noethérien, annulé par un entier $n$ inversible sur $S$, $f$ propre et 
lisse, et $E$ à cohomologie localement constante (propreté cohomologique 
des morphismes propres (\ref{VII:5-1-3}), et locale acyclicité des morphismes 
lisses \cite[XV]{sga4}). On retrouve le ``théorème de spécialisation'' 
\cite[XVI 2.1]{sga4}. 

D'autre part, \ref{VII:5-1-1} a la conséquence suivante: 





\begin{proposition}\label{VII:5-2-5}
Supposons que $f$ soit localement acyclique relativement à $E$, et que, pour 
toute spécialisation $u:t\to S(s)$, la flèche de cospécialisation 
$\cosp(u)$ soit un isomorphisme. Alors la formation de $\eR f_\ast E$ commute 
à tout changement de base fini. 
\end{proposition}

Compte tenu de \ref{VII:5-2-4}, \ref{VII:5-2-5} généralise 
(\hyperref[I]{Cohomologie étale} \ref{I:5-1-7}). 





\begin{corollary}\label{VII:5-2-6}
Supposons que $f$ soit localement acyclique rel.\ à $E$. Soit $x$ un point 
géométrique de $X$, d'image $s$ dans $S$, notons $f_{(x)}:X(x)\to S(s)$ le 
morphisme induit par $f$. Alors la formation de $\eR {f_{(x)}}_\ast(E|X(x))$ 
commute à tout changement de base fini $S'\to S(s)$. 
\end{corollary}

Il suffit, d'après \ref{VII:5-2-5}, de vérifier que les flèches de 
cospécialisation pour $(f_{(x)},E|X(x))$ sont des isomorphismes. Par 
transitivité des flèches de cospécialisation (\ref{VII:5-2-3}), il suffit 
de montrer que, si $u:t\to S(s)$ est une spécialisation, alors 
$\cosp(u):\eR\Gamma(X(x)_t,E|X(x)_t)\to \eR\Gamma(X(x)_s,E|X(x)_s)$ est un 
isomorphisme. Mais, comme $s$ est fermé dans $S(s)$, $X(x)_s$ est strictement 
local de point fermé $x$, $\eR\Gamma(X(x)_s,E|X(x)_s)=E_x$, et 
$\cosp(u)$ est un isomorphisme, inverse de l'isomorphisme de restriction 
$E_x\to \eR\Gamma(X(x)_t,E|X(x)_t)$. 





\begin{corollary}\label{VII:5-2-7}
Soient $f:X\to Y$ un $S$-morphisme, $p:X\to S$, $q:Y\to S$ les projections, et 
$E\in\ob\eD^+(X)$. On suppose $A$ localement constant, $f$ localement acyclique 
rel.\ à $E$, et $q$ localement acyclique en tout point géométrique $y$ de 
$Y$ rel.\ à tout faisceau localement constant au voisinage de $y$. Alors $p$ 
est localement acyclique rel. à $E$. 
\end{corollary}

Soient $x$ un point géométrique de $x$, d'images $y$ et $s$ dans $Y$ et 
$S$, $t\to S(s)$ un spécialisation, $f_{(x)}:X(x)\to Y(y)$ le morphisme 
induit par $f$. Il s'agit de montrer que la restriction 
\begin{equation*}\tag{$*$}\label{VII:eq:5-2-7-1}
  E_x \to \eR\Gamma(X(x)_t,E|X(x)_t) 
\end{equation*}
es un isomorphisme. On peut supposer $A$ constant. D'après \ref{VII:5-2-4} et 
\ref{VII:5-2-5}, pour toute spécialisation $z\to Y(y)$, la flèche de 
spécialisation 
\[
  E_x = \eR {f_{(x)}}_\ast (E|X(x))_y \to \eR {f_{(x)}}_\ast (E|X(x)|_z = \eR\Gamma(X(x)_z,E|X(x)_z) 
\]
est un isomorphisme, autrement dit $\eR{f_{(x)}}_\ast(E|X(x))$ est constant de 
valeur $E_x$. L'hypothèse d'acyclicité sur $q$ entra\^ine donc que la 
restriction 
\[
  E_x \to \eR\Gamma(Y(y)_y,(E_x)|Y(y)_t) = \eR\Gamma\left(Y(y)_t,\eR {f_{(x)}}_\ast (E|X(x))|Y(y)_t\right)
\]
est un isomorphisme. Mais cette flèche s'identifie à 
\eqref{VII:eq:5-2-7-1}, d'où la conclusion. 

L'hypothèse de \ref{VII:5-2-7} sur $q$ est vérifiée notamment lorsque $q$ 
est lisse, et que $A$ est annulé par un entier $n$ inversible sur $S$ 
\cite[XV]{sga4}. D'autre part, si les hypothèses de \ref{VII:5-2-7} sont 
vérifiées après tout changement de base $S'\to S$, on conclut alors que 
$p$ est universellement localement acyclique rel.\ à $E$. De ces remarques 
découle en particulier l'énoncé \ref{VII:2-14}. 





\begin{corollary}\label{VII:5-2-8}
Soient $f:X\to S$ et $E\in \ob\eD^+(X)$. On suppose que $f$ est universellement 
localement acyclique rel.\ à $E$, que la formation de $\eR f_\ast E$ commute 
à tout changement de base fini, et que pour tout morphisme $s'\to s$, où 
$s$ est un point géométrique de $S$ et $s'$ le spectre d'un corps 
séparablement clos, la flèche de changement de base 
$\eR\Gamma(X_s,E|X_s)\to \eR\Gamma(X_{s'},E|X_{s'})$ est un isomorphisme. 
Alors $(f,E)$ est cohomologiquement propre (\ref{VII:5-1-3}). 
\end{corollary}

Soit $f':X'\to S'$ le morphisme déduit de $f$ par un changement de base 
$S'\to S$, il s'agit de voir que la formation de $\eR f_\ast'(E|X')$ commute 
à tout changement de base fini. D'après \ref{VII:5-2-5}, il suffit de 
montrer que, pour toute spécialisation $u':t'\to S'(s')$, la flèche de 
cospécialisation $\cosp(u')$: 
$\eR\Gamma(X_{t'}',E|X_{t'}') \to \eR\Gamma(X_{s'}',E|X_{s'}')$ est un 
isomorphisme. Soient $s$ (resp.\ $t$) le point géométrique image de $s$ 
(resp.\ $t$) dans $S$, $u:t\to S(s)$ la spécialisation définie par $u'$. 
On voit aisément qu'on a un carré commutatif 
\[\xymatrix{
  \eR\Gamma(X_t,E|X_t) \ar[r]^-{\cosp(u)} \ar[d] 
    & \eR\Gamma(X_s,E|X_s) \ar[d] \\
  \eR\Gamma(X_{t'}',E|X_{t'}') \ar[r]^-{\cosp(u')} 
    & \eR\Gamma(X_{s'}',E|X_{s'}') 
}\]
où les flèches verticales sont les flèches de changement de base. Par 
hypothèse, celles-ci sont des isomorphismes. D'autre part (\ref{VII:5-2-4}) 
$\cosp(u)$ est un isomorphisme, donc $\cosp(u')$ est un isomorphisme, ce qui 
achève la démonstration. 

Voici un exemple d'application de \ref{VII:5-2-8}. Supposons $A$ annulé par 
un entier $n$ inversible sur $S$, $f$ localement de type fini et 
universellement localement acyclique. Soient $x$ un point géométrique de 
$X$, d'image $s$ dans $S$, $f_{(x)}:X(x)\to S(s)$ le morphisme induit par $f$. 
Il résulte de \ref{VII:5-2-6} et \ref{VII:1-9} (par passage à la limite) 
que les hypothèses de \ref{VII:5-2-8} sont vérifiées par le couple 
$(f_{(x)},E|X(x))$, qui est donc cohomologiquement propre. 





\subsubsection{}\label{VII:5-2-9}

Supposons $A$ constant. Soient $f:X\to S$, et $E\in\ob\eD^+(X)$ le complexe de 
tor-dimension finie. Nous dirons que $f$ est \emph{fortement localement 
acyclique} rel.\ à $E$ si, pour tout point géométrique $x$ de $X$, 
d'image $s$ dans $S$, toute spécialisation $t\to S(s)$, et tout 
$A^\circ$-module $M$, la flèche de restriction 
\[
  E_x\lotimes_A M \to \eR\Gamma(X(x)_t,(E|X(x)_t)\lotimes_A M) 
\]
est un isomorphisme. J'ignore si ``localement acyclique'' implique 
``fortement localement acyclique.'' 





\begin{proposition}\label{VII:5-2-10}
Sous les hypothèses de \ref{VII:5-2-9}, supposons $A$ noethérien de 
torsion, $S$ noethérien, et $f$ fortement localement acyclique rel.\ à $E$. 
Alors, pour tout diagramme à carrés cartésiens 
\[\xymatrix{
  X \ar[d]^-f 
    & \ar[l]_-j X' \ar[d]^-{f'} 
    & \ar[l]_-{j'} X'' \ar[d]^-{f''} \\
  S 
    & \ar[l]_-i S' 
    & \ar[l]_-{i'} S'' \text{,} 
}\]
avec $i$ quasi-fini et $i'$ immersion ouverte, et tout 
$F\in\ob\eD^+(S'',A^\circ)$, la flèche canonique 
\begin{equation*}\tag{$*$}\label{VII:eq:5-2-10-1}
  j^\ast E \lotimes {f'}^\ast \eR i_\ast' F \to \eR j_\ast' ((j j')^\ast E\lotimes {f''}^\ast F) 
\end{equation*}
est un isomorphisme. 
\end{proposition}

On comparera cet énoncé avec \cite[XV 1.17]{sga4}. 

Démontrons \ref{VII:5-2-10}. On se ramène facilement à$S=S'$, 
strictement local, et $F$ constructible, concentré en degré $0$. On 
résout alors $F$ à droite par des sommes finies de faisceaux de la forme 
$u_\ast M$, où $u:t\to S$ est un point géométrique et $M$ un 
$A^\circ$-module. L'hypoth\`se implique que \eqref{VII:eq:5-2-10-1} est un 
isomorphisme pour $F=u_\ast M$, d'où le résultat. 



