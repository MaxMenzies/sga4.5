\documentclass[oneside]{book}
\usepackage{sga-style}





\title{Séminaire de Géométrie Algébrique du Bois-Marie \\ \vspace{20pt}
Cohomologie Étale \\ \vspace{20pt}
(SGA $4\frac 1 2$)}
\author{par P. Deligne \vspace{10pt}\\ 
avec la collaboration de J.F. Boutot, \\ A. Grothendieck, L. Illusie et J.L. Verdier}
\date{1977}


%% the part I'm currently working on 
%  \includeonly{sga4.5-III}


\begin{document}
\maketitle

% French characters
% à é è ê ô ù û

% other notes: there are two I.6.5's in Deligne's original - one defining sheaf 
% cohomology, the other defining torsors






\section*{Typesetters note}

This is a \LaTeX rendition of Deligne's SGA $4\frac 1 2$. It is \emph{not} the 
original document, and any errors and typos are entirely my responsibility. 
You may also be violating copyright law by downloading this file. 

\chapter*{Introduction}

Ce volume a pour but de faciliter au non-expert l'usage de la 
cohomologie $\ell$-adique. J'espère qu'il lui permettra souvent d'éviter le 
recours aux exposés touffus de SGA 4 et SGA 5. Il contient aussi quelques 
résultats nouveaux. 

Le premier exposé, édigé par J.F. Boutot, survole SGA 4. Il donne les 
principaux résultats -- avec une généralité minimale, souvent 
insuffisante pour les applications -- et une idée de leur démonstration. Pour 
des résultats complets, ou des démonstrations détaillées, SGA 4 reste 
indispensable. 

Le ``\nameref{II}'' contient une démonstration complété de 
la formule des traces pour l'endomorphisme de Frobenius. La démonstration est 
celle donnée par Grothendieck dans SGA 5, élaguée de tout détail inutile. Ce 
Rapport devrait permettre à utilisateur d'oublier SGA 5, qu'on pourra 
considérer comme une série de digression, certaines très intéressantes. Son 
existence permettra de publier prochainement SGA 5 tel quel. Il est complété 
par l'exposé ``\nameref{VI}'' qui explique comment la formule des traces permet 
l'étude de sommes trigonométriques, et donne des exemples. 

Le public visé par les autres exposés est plus limité, et leur style s'en 
ressent. L'exposé ``\nameref{III}'' est une 
généralisation ``modulaire'' du Rappoport, basée sur l'étude SGA 4 XVII 5.5 des 
puissances symétriques. L'exposé ``\nameref{IV}'' 
définit cette classes dans divers contextes, et donne la compatibilité 
entre intersections et cup-produits. Dans ``\nameref{V}'' sont rassemblés 
quelques résultats connus, pour lesquels manquait une référence, et quelques 
compatibilités. L'exposé ``\nameref{VII}'' 
est nouveau. Il donne notamment, en cohomologie sans supports, des théorèmes de 
finitude analogues à ceux connus en cohomologie à supports compacts. 

Pour plus de détails sur les exposés, je renvoie à leur introduction 
respective. 

Je remercie enfin J.L. Verdier de m'avoir permis de reproduire ici ses notes 
``Catégories dérivées (Etat $0$).'' Elles restent je crois très utiles, et 
étaien: devenues introuvables. 

Dans les références internes à ce volume, les exposés sont cités par 
un titre abrégé, indiqué entre [ ] dans la table des matières. 

\vspace{20pt}
Bures-sur-Yvette, le 20 Septembre 1976

\vspace{5pt}
Pierre Deligne
\tableofcontents
\chapter{Cohomologie étale: les points de départ\texorpdfstring{\footnote{par P. Deligne, rédigé par J.F. Boutot}}{}}




















\section{Topologies de Grothendieck}\label{I:1}

A l'origine, les topologies de Grothendieck sont apparues comme sous-jacentes 
à sa théorie de la descente (cf. SGA 1 VI, VIII); l'usage des théorie de 
cohomologie correspondantes est plus tardif. La même démarche est suivi 
ici: en formalisant les notions classiques de localisation, de propriété 
locale et de recollement (\ref{I:1-1}, \ref{I:1-2}, \ref{I:1-3}), en dégage le 
concept général de topologie de Grothendieck (\ref{I:1-6}); pour en justifier 
l'introduction en géométrie algébrique, on démontre un théorème 
de descente fidèlement plat (\ref{I:1-4}), généralisation du classique théorème 
de Hilbert (\ref{I:1-5}). 

Le lecteur trouvera une exposition plus compète, mais concise, du 
formalisme dans Giraud \cite{gi64}. Les notes de M. Artin: ``Grothendieck 
topologies'' \cite{ar62} (chapitres I à III) restent également utiles. Les 
866 pages des exposés I à V de SGA 4 sont précieuses lorsqu'on 
considère des topologies exotiques, telle celle qui donne naissance à la 
cohomologie cristalline; pour utiliser la topologie étale si proche de 
l'intuition classique, il n'est pas indispensable de les lire. 










\subsection{Cribles}\label{I:1-1}

Soient $X$ un espace topologique et $f:X\to \dR$ une fonction à valeur 
réelles sur $X$. La continuité de $f$ est une propriété de nature 
locale; autrement dit, si $f$ est continue sur tout ouvert suffisamment petit 
de $X$, $f$ est continue sur $X$ tout entier. Pour formaliser la notion de 
``propriété de nature locale,'' nous introduirons quelques définitions.

On dit qu'un ensemble $\sU$ d'ouverts de $X$ est un \emph{crible} si pour tout 
$U\in\sU$ et $V\subset U$, on a $V\in\sU$. On dit qu'un crible est 
\emph{couvrant} si la réunion de tout les ouverts appartenant à ce crible est 
égale à $X$.

Etant donnée une famille $\{U_i\}$ d'ouverts de $X$, le crible engendré par 
$\{U_i\}$ est par définition l'ensemble des ouverts $U$ de $X$ tels que $U$ 
soit contenu dans l'un des $U_i$. 

On dit qu'une propriété $P(U)$, définie pour tout ouvert $U$ de $X$, est 
\emph{locale} si, pour tout crible couvrant $\sU$ de tout ouvert $U$ de $X$, 
$P(U)$ est vraie si et seulement si $P(V)$ est vraie pour tout $V\in \sU$. Par 
exemple, étant donné $f:X\to \dR$, la propriété ``$f$ est continue sur $U$'' 
est locale. 





\subsection{Faisceaux}\label{I:1-2}

Précisons la notion de fonction donnée localement sur $X$.





\subsubsection{Point de vue des cribles}\label{I:1-2-1}

Soit $\sU$ un crible d'ouverts de $X$. On appelle fonction donnée 
$\sU$-localement sur $X$ la donnée pour tout $U\in \sU$ d'une fonction $f_U$ 
sur $U$ telle que, si $V\subset U$, on ait $f_V=f_U|V$. 





\subsubsection{Pointe de vue de Čech}\label{I:1-2-2}

Si le crible $\sU$ est engendré par une famille d'ouverts $U_i$ de $X$, se 
donner une fonction $\sU$-localement revient à se donner une fonction $f_i$ 
sur chaque $U_i$, telle que $f_i|{U_i\cap U_j} = f_j|{U_i\cap U_j}$. 

Autrement dit, si $Z=\coprod U_i$, se donner une fonction $\sU$-localement 
revient à se donner une fonction sur $Z$ qui soit constante sur les fibres de 
la projection naturelle $Z\to X$. 





\subsubsection{}\label{I:1-2-3}

Les fonctions continues forment un faisceau; cela signifie que pour tout crible 
couvrant $\sU$ d'un ouvert $V$ de $X$ et toute fonction donnée 
$\sU$-localement $\{f_U\}$ telle que chaque $f_U$ soit continue sur $U$, il 
existe une unique fonction continue $f$ sur $V$ telle que $f|U=f_U$ pour tout 
$U\in \sU$.










\subsection{Champs}\label{I:1-3}

Précisons maintenant la notion de fibré vectoriel donné localement sur $X$. 





\subsubsection{Point de vue des cribles}\label{I:1-3-1}

Soit $\sU$ un crible d'ouverts de $X$. On appelle fibré vectoriel donné 
$\sU$-localement sur $X$ les données de 
\begin{enumerate}[\indent a)]
  \item un fibré vectoriel $E_U$ sur chaque $U\in \sU$, 
  \item si $V\subset U$, un isomorphisme $\rho_{U,V} : E_V\iso E_U|V$, vérifiant 
  \item si $W\subset V\subset U$, le diagramme 
    \[\xymatrix{
      E_W \ar[r]^-{\rho_{U,W}} \ar[dr]_-{\rho_{V,W}}
        & E_U|W \\
      & E_V|W \ar[u]_-{\rho_{U,V}|W}
    }\]
    commute, c'est-à-dire 
    $\rho_{U,V} = (\mbox{$\rho_{U,V}$ restreint à $W$}) \circ \rho_{V,W}$. 
\end{enumerate}





\subsubsection{Point de vue de Čech}\label{I:1-3-2}

Si le crible $\sU$ est engendré par une famille d'ouverts $U_i$ de $X$, se 
donner un fibré vectoriel $\sU$-localement revient à se donner:
\begin{enumerate}[\indent a)]
  \item un fibré vectoriel $E_i$ sur chaque $U_i$, 
  \item si $U_{ij} = U_i\cap U_j = U_i\times_X U_j$, un isomorphisme 
    $\rho_{ji} : E_i|U_{ij} \iso E_j|U_{ij}$, de sorte que 
  \item si $U_{ijk} = U_i\times_X U_j\times_X U_k$, le diagramme 
    \[\xymatrix{
      E_i |U_{ijk} \ar[r]^-{\rho_{ki}|U_{ijk}} \ar[dr]_-{\rho_{ji}|U_{ijk}} 
        & E_k|U_{ijk} \\
      & E_j|U_{ijk} \ar[u]_-{\rho_{kj}|U_{ijk}}
    }\]
    commute, c'est-à-dire $\rho_{ki} = \rho_{kj}\circ \rho_{ji}$ sur 
    $U_{ijk}$. 
\end{enumerate}

Autrement dit, si $Z=\coprod U_i$ et si $\pi:Z\to X$ est la projection 
naturelle, se donner un fibré vectoriel $\sU$-localement revient à se donner: 
\begin{enumerate}[\indent a)]
  \item un fibré vectoriel $E$ sur $Z$, 
  \item si $x$ et $y$ sont deux points de $Z$ tels que $\pi(x) = \pi(y)$, un 
    isomorphisme $\rho_{y x}:E_x\iso E_y$ entre les fibres de $E$ en $x$ et en 
    $y$, dépendant continûment de $(x,y)$ et tel que, 
  \item si $x$, $y$ et $z$ sont trois points de $Z$ tels que 
  $\pi(x)=\pi(y)=\pi(z)$, on ait $\rho_{zx} = \rho_{zy}\circ \rho_{yx}$. 
\end{enumerate}





\subsubsection{}\label{I:1-3-3}

Une fibré vectoriel $E$ sur $X$ définit un fibré vectoriel donné 
$\sU$-localement $E_\sU$; le système des restrictions $E_U$ de $E$ aux objets 
de $\sU$. Le fait que la notion de fibré vectoriel est de nature locale peut 
s'exprimer ainsi: pour tout crible couvrant $\sU$ de $X$, le foncteur 
$E\mapsto E_\sU$, des fibrés vectoriels sur $X$ dans les fibrés vectoriels 
donnés $\sU$-localement, est une équivalence de catégories. 





\subsubsection{}\label{I:1-3-4}

Si dans \ref{I:1} on remplace ``ouvert de $X$'' par ``partie de 
$X$,'' on obtient la notion de crible de sous-espaces de $X$. Dans ce cadre 
aussi on dispose de théorèmes de recollement. Par exemple: soient $X$ un 
espace et $\sC$ un crible de sous-espaces de $X$ engendré par un recouvrement 
fermé localement fini de $X$, alors le foncteur $E\mapsto E_\sC$, des fibrés 
vectoriels sur $X$ dans les fibrés vectoriels donnés $\sC$-localement est une 
équivalence de catégories. 

En géométrie algébrique, il est utile de considérer aussi des ``cribles 
d'espaces au-dessus de $X$''; c'est ce que nous verrons au paragraphe suivant. 










\subsection{Descente fidèlement plat}\label{I:1-4}





\subsubsection{}\label{I:1-4-1}

Dans le cadre des schémas, la topologie de Zariski n'est pas assez fine pour 
l'étude des problèmes non linéaires et on est amené à remplacer dans les 
définitions précédentes les immersions ouvertes par des morphismes plus 
généraux. De ce point de vue, les techniques de descente apparaissent comme des 
techniques de localisation. Ainsi l'énoncé de descente suivant peut sexprimer 
en disant que les propriétés considérées sont de nature locale pour la 
topologie fidèlement plate (on dit qu'un morphisme de schémas est fidèlement 
plat s'il est plat est surjectif). 


\begin{proposition}\label{I:1-4-2}
Soient $A$ un anneau et $B$ une $A$-algèbre fidèlement plate. Alors:
\begin{enumerate}[(i)]
  \item Une suite $\Sigma=(M'\to M\to M'')$ de $A$-modules est exacte dés 
    que la suite $\Sigma_{(B)}$ qui s'en déduit par extension des scalaires 
    à $B$ est exacte.
  \item Un $A$-module $M$ est de type fini (resp. de présentation finie, 
    plat, localement libre de rang fini, inversible (i.e. localement libre 
    de rang un)) dés que le $B$-module $M_{(B)}$ l'est.
\end{enumerate}
\end{proposition}
\begin{proof}
(i) Le foncteur $M\mapsto M_{(B)}$ étant exact (platitude de $B$), il suffit 
de montrer que, si un $A$-module $N$ est non nul, $N_{(B)}$ est non nul. Si 
$N$ est non nul, $N$ contient un sous-module monogène non nul $A/\fa$; alors 
$N_{(B)}$ contient un sous-module monogène $(A/\fa)_{(B)} = B/\fa B$, non nul 
par surjectivité du morphisme structural $\varphi:\spec(B)\to\spec(A)$ (si 
$V(\fa)$ est non vide, $\varphi^{-1}(V(\fa))=V(\fa B)$ est non vide). 

(ii) Pour toute famille $(x_i)$ d'éléments de $M_{(B)}$, il existe un 
sous-module de type fini $M'$ de $M$ tel que $M'_{(B)}$ contienne les $x_i$. Si 
$M_{(B)}$ est de type fini et si les $x_i$ engendrent $M_{(B)}$, on a 
$M'_{(B)}=M_{(B)}$, donc $M'=M$ et $M$ est de type fini. 
\end{proof}

Si $M_{(B)}$ est de présentation finie, on peut, d'aprés ce qui précède, 
trouver une surjection $A^n\to M$. Si $N$ est le noyau de cette surjection, le 
$B$-module $N_{(B)}$ est de type fini, donc $N$ l'est, et $M$ est de 
présentation finie. L'assertion pour ``flat'' résulte aussitôt de (i); 
``localement libre de rang fini'' signifie ``plat et de présentation finie'' 
et le rang se teste par extension des scalaires à des corps.





\subsubsection{}\label{I:1-4-3}

Soient $X$ un schéma et $\sS$ une classe de $X$-schémas stable par produit 
fibré sur $X$. Une classe $\sU\subset \sS$ est un \emph{crible} sur $X$ 
(relativement à $\sS$) si, pour tout morphisme $\varphi:V\to U$ de 
$X$-schémas, avec $U,V\in \sS$ et $U\in\sU$, on a $V\in \sU$. Le crible 
\emph{engendré} par une famille $\{U_i\}$ de $X$-schémas dans $\sS$ est la 
classe des $V\in\sS$ tels qu'il existe un morphisme de $X$-schémas de $V$ dans 
l'un des $U_i$. 






\subsubsection{}\label{I:1-4-4}

Soit $\sU$ un crible sur $X$. On appelle module quasi-cohérent donné 
$\sU$-localement sur $X$ la donnée de 
\begin{enumerate}[\indent a)]
  \item un module quasi-cohérent $E_U$ sur chaque $U\in\sU$, 
  \item pour tout $U\in\sU$ et pour tout morphisme $\varphi:V\to U$ de 
    $X$-schémas dans $\sS$, un isomorphisme 
    $\rho_\varphi:E_V\iso \varphi^* E_U$, ceux-ci étant tels que 
  \item si $\psi:W\to V$ est un morphisme de $X$-schémas dans $\sS$, le 
    diagramme 
    \[\xymatrix{
      E_W \ar[r]^-{\rho_{\varphi\circ\psi}} \ar[dr]_-{\rho_\psi} 
        & \psi^*\varphi^* E_U \\
      & \psi^* E_V \ar[u]_-{\psi^*\rho_\varphi}
    }\]
    commute, c'est-à-dire 
    $\rho_{\varphi\circ\psi}=(\psi^*\rho_\varphi)\circ \rho_\psi$. 
\end{enumerate}

Si $E$ est un module quasi-cohérent sur $X$, on note $E_\sU$ le module 
donné $\sU$-localement valant $\varphi_U^* E$ sur $\varphi:U\to X$ et tel que, 
pour tout morphisme $\psi:V\to U$ l'isomorphisme de restriction $\rho_\psi$ 
soit l'isomorphisme canonique 
$E_V=(\varphi_U\circ \psi)^* E\iso \psi^* \varphi_U^* E = \psi^* E_U$. 


\begin{theorem}\label{I:1-4-5}
Soit $\{U_i\}\in\sS$ une famille finie de $X$-schémas plats sur $X$ telle que 
$X$ soit le réunion des images des $U_i$, et soit $\sU$ le crible engendré par 
$\{U_i\}$. Alors le foncteur $E\mapsto E_\sU$ est une équivalence de la 
catégorie des modules quasi-cohérents sur $X$ avec la catégorie des modules 
quasi-cohérents donnés $\sU$-localement.
\end{theorem}
\begin{proof}
Nous ne traiterons que le cas où $x$ est affine et où $\sU$ est engendré par 
un $X$-schéma affine $U$ fidèlement plat sur $X$. La réduction à ce cas est 
formelle. On pose $X=\spec(A)$ et $U=\spec(B)$. 

Si le morphisme $U\to X$ admet une section, $X$ appartient au crible $\sU$ et 
l'assertion est évidente. Nous nous réduirons à ce cas. 

Un module quasi-cohérent donné $\sU$-localement définit des modules $M'$, $M''$ 
et $M'''$ sur $U$, $U\times_X U$ et $U\times_X U\times_X U$, et des 
isomorphismes $\rho:p^* M^\bullet\simeq M^\bullet$ pour tout morphisme de 
projection $p$ entre ces espaces; c'est là un \emph{diagramme cartésien} 
\[\xymatrix{
  M^* : M' \ar@<2pt>[r] \ar@<-2pt>[r] 
    & M'' \ar[r] \ar@<4pt>[r] \ar@<-4pt>[r] 
    & M'''
}\]
au-dessus de 
\[\xymatrix{
  U_* : U 
    & U\times_X U \ar@<2pt>[l] \ar@<-2pt>[l] 
    & U\times_X U\times_X U \ar[l] \ar@<4pt>[l] \ar@<-4pt>[l] \mbox{.}
}\]
Réciproquement $M^*$ détermine le module donné $\sU$-localement: pour 
$V\in\sU$, il existe $\varphi:V\to U$ et on pose $M_U=\varphi^* M'$; 
pour $\varphi_1,\varphi_2:V\to U$, on a une identification naturelle 
$\varphi_1^* M'\simeq (\varphi_1\times \varphi_2)^* M'' \simeq \varphi_2^* M'$, 
et on voit en utilisant $M'''$ ue ces identifications sont compatibles, de 
sorte que la définition est légitime. Bref, il revient au même de se donner un 
module $\sU$-localement ou un diagramme $M^*$ cartésien sur $U_*$. 

Traduisons en termes algébriques: se donner $M^*$ revient à se donner un 
diagramme cartésien de modules 
\[\xymatrix{
  M' \ar@<3pt>[r]|-{\partial_0} \ar@<-3pt>[r]|-{\partial_1} 
    & M'' \ar@<6pt>[r]|-{\partial_0} \ar[r]|-{\partial_0} \ar@<-6pt>[r]|-{\partial_2}
    & M'''
}\]
au-dessus du diagramme d'anneaux 
\[\xymatrix{
  B \ar@<3pt>[r]|-{\partial_0} \ar@<-3pt>[r]|-{\partial_1} 
    & B\otimes_A B \ar@<6pt>[r]|-{\partial_0} \ar[r]|-{\partial_0} \ar@<-6pt>[r]|-{\partial_2} 
    & B\otimes_A B\otimes_A B
}\]
(précisons: on a $\partial_i(b m)=\partial_i(b)\cdot\partial_i(m)$, les 
identités usuelles telles que $\partial_0\partial_1=\partial_0\partial_0$ 
sont vraies, et ``cartésien'' signifie que les morphismes 
$\partial_i:M'\otimes_{B,\partial_i}(B\otimes_A B)\to M''$ et 
$M'''\otimes_{B\otimes_A B,\partial_i}(B\otimes_A B\otimes_A B)\to M'''$ sont 
des isomorphismes). 

Le foncteur $E\mapsto E_\sU$ devient le foncteur qui, à un $A$-module $M$, 
associe 
\[\xymatrix{
  M^* = (M\otimes_A B \ar@<2pt>[r] \ar@<-2pt>[r] 
    & M\otimes_A B\otimes_A B \ar@<-4pt>[r] \ar[r] \ar@<4pt>[r] 
    & M\otimes_A B\otimes_A B\otimes_A B)\text{.}
}\]
Il admet pour adjoint à droite le foncteur
\[\xymatrix{
  (M' \ar@<2pt>[r] \ar@<-2pt>[r] 
    & M'' \ar@<-4pt>[r] \ar[r] \ar@<4pt>[r]
    & M''') \ar@{|->}[r] 
    & \ker(M' \ar@<2pt>[r] \ar@<-2pt>[r] 
    & M''')\text{.}
}\]
Il nous faut prouver que les flèches d'adjonction 
\[
  M \to\ker(M\otimes_A B\rightrightarrows M\otimes_A B\times_A B)
\]
et
\[
  \ker(M''\twoheadrightarrow M'')\otimes_A B \to M'
\]
sont des isomorphismes. D'après (\ref{I:1-4-2}.i), il suffit de le prouver 
après un changement de base fidèlement plat $A\to A'$ ($B$ devenant 
$B'=B\otimes_A A'$). Prenant $A'=B$, ceci nous ramène au cas où $U\to X$ 
admet une section. 
\end{proof}










\subsection{Un cas particulier: le théorème 90 de Hilbert}\label{I:1-5}





\subsubsection{}\label{I:1-5-1}

Soient $k$ un corps, $k'$ une extension galoisienne de $k$ et $G=\gal(k'/k)$. 
Alors l'homomorphisme 
\begin{align*}
  k'\otimes_k k' &\to \oplus_{\sigma\in G} k' \\
  x\otimes y     &\mapsto \{x\cdot \sigma(y)\}_{\sigma\in G}
\end{align*}
est bijectif. 

On en déduit qu'il revient au même de se donner un module localement pour 
le crible engendré par $\spec(k')$ sur $\spec(k)$ ou de se donner un 
$k'$-espace vectoriel muni d'une action semi-linéaire de $G$, c'est-à-dire: 
\begin{enumerate}[\indent a)]
  \item un $k'$-espace vectoriel $V'$,
  \item pour tout $\sigma\in G$, un endomorphisme $\varphi_\sigma$ de la 
    structure de groupe de $V'$ tel que 
    $\varphi_\sigma(\lambda v) = \sigma(\lambda) \varphi_\sigma(v)$, pour 
    tout $v\in V'$, vérifiant la condition 
  \item pour tout $\sigma,\tau\in G$, on a 
    $\varphi_{\tau\sigma} = \varphi_\tau\circ\varphi_\sigma$. 
\end{enumerate}

Soit $V={V'}^G$ le groupe des invariants par cette action de $G$; c'est un 
$k$-espace vectoriel et, d'après le théorème \ref{I:1-4-5}, on a:

\begin{proposition}\label{I:1-5-2}
L'inclusion de $V$ dans $V'$ définit un isomorphisme 
$V\otimes_k k' \iso V'$.
\end{proposition}

En particulier, si $V'$ est de dimension $1$ et si $v'\in V$ est non nul, 
$\varphi_\sigma$ est déterminé par la constante 
$c(\sigma)\in {k'}^\times$ telle que $\varphi_\sigma(v')=c(\sigma)v'$ et la 
condition c) s'écrit 
\[
  c(\tau\sigma) = c(\tau)\cdot \tau(c(\sigma))\text{.}
\]
D'après la proposition il existe un vecteur invariant non nui 
$v=\mu v'$, $\mu\in {k'}^\times$. On a donc pour tout $\sigma\in G$, 
\[
  c(\sigma) = \mu\cdot \sigma(\mu^{-1})\text{.}
\]
Autrement dit tout $1$-cocycle de $G$ à valeurs dans ${k'}^\times$ est un 
cobord: 

\begin{corollary}\label{I:1-5-3}
On a $\h^1(G,{k'}^\times)=0$.
\end{corollary}










\subsection{Topologies de Grothendieck}\label{I:1-6}

Nous transcrivons maintenant les définitions des paragraphes précédents 
dans un cadre abstrait englobant à la fois le cas des espaces topologiques 
et celui des schémas. 





\subsubsection{}\label{I:1-6-1}

Soient $\sS$ une catégorie et $U$ un objet de $\sS$. On appelle \emph{crible} 
sur $U$ un sous-ensemble $\sU$ de $\ob(\sS/U)$ tel que si $\varphi:V\to U$ 
appartient à $\sU$ et si $\psi:W\to V$ est un morphisme dans $\sS$, alors 
$\varphi\circ\psi:W\to U$ appartient à $\sU$. 

Si $\{\varphi_i:U_i\to U\}$ est une famille de morphismes, le crible 
engendré par les $U_i$ est par définition l'ensemble des morphismes 
$\varphi:V\to U$ qui se factorisent à travers l'un des $\varphi_i$. 

Si $\sU$ est un crible sur $U$ et si $\varphi:V\to U$ est un morphisme, la 
restriction $\sU_V$ de $\sU$ à $V$ est par définition le crible sur $V$ 
constitué oar les morphismes $\psi:w\to V$ tels que 
$\varphi\circ\psi:W\to U$ appartienne à $\sU$.





\subsubsection{}\label{I:1-6-2}

La donnée d'une \emph{topologie de Grothendieck} sur $\sS$ consiste en la 
donnée pour tout objet $U$ de $\sS$ d'un ensemble $C(U)$ de cribles sur $U$, 
dits cribles couvrants, de telle sorte que les axiomes suivants soient 
satisfaits:
\begin{enumerate}[\indent a)]
  \item Le crible engendré par l'identité de $U$ est couvrant.
  \item Si $\sU$ est un crible couvrant sur $U$ et si $V\to U$ est un 
    morphisme, le crible $\sU_V$ est couvrant.
  \item Un crible localement couvrant est couvrant. Autrement dit, si $\sU$ est 
    un crible couvrant sur $U$ et si $\sU'$ est un crible sur $U$ tel que, pour 
    tout $V\to U$ appartenant à $\sU$, le crible $\sU_V'$ est couvrant, alors 
    $\sU'$ est couvrant.
\end{enumerate}

On appelle \emph{site} la donnée d'une catégorie munie d'une topologie de 
Grothendieck.





\subsubsection{}\label{I:1-6-3}

Étant donnée un site $\sS$, on appelle préfaisceau sur $\sS$ un foncteur 
contravariant $\cF$ de $\sS$ dans la catégorie des ensembles. Pour tout 
objet $U$ de $\sS$, on appelle section de $\cF$ au-dessus de $U$ les 
éléments de $\cF(U)$. Pour tout morphisme $V\to U$ et pour tout 
$s\in \cF(U)$, on note $s|V$ ($s$ restreint à $V$) l'image de $s$ dans 
$\cF(V)$.

Si $\sU$ est un crible sur $U$, on appelle section donnée $\sU$-localement la 
donnée, pour tout $V\to U$ appartenant à $\sU$, d'une section 
$s_V\in\cF(V)$ telle que, pour tout morphisme $W\to V$, on sit $s_V|W=s_W$. On 
dit que $\cF$ est un \emph{faisceau} si, pour tout objet $U$ de $\sS$, pour 
tout crible couvrant $\sU$ sur $U$ et pour tout section donnée 
$\sU$-localement $\{s_V\}$, il existe une unique section $s\in\cF(U)$ telle que 
$s|V=s_V$, pour tout $V\to U$ appartenant à $\sU$. 

On définit de manière analogue les \emph{faisceaux abéliens} en 
remplaçant la catégorie des ensembles par celle des groupes abéliens. On 
montre que la catégorie des faisceaux abéliens sur $\sS$ est une 
catégorie abélienne possédant suffisamment d'injectifs. Une suite 
$\cF\xrightarrow f \cG\xrightarrow g \cH$ de faisceaux est exacte si, pour 
tout objet $U$ de $\sS$, et pour tout $s\in \cG(U)$ telle que $g(s)=0$, 
il existe localement $t$ tel que $f(t)=s$; i.e. s'il existe un crible couvrant 
$\sU$ sur $U$ et pour tout $V\in\sU$, une section $t_V$ de $\cF$ sur $V$ telle 
que $f(t_V)=s|V$. 





\subsubsection{Exemples}\label{I:1-6-4}

Nous en avons vu deux plus haut. 

\begin{enumerate}[\indent a)]
  \item Soient $X$ un espace topologique et $\sS$ la catégorie dont les objets 
    sont les ouverts de $X$ et les morphismes les inclusions naturelles. La 
    topologique de Grothendieck sur $\sS$ correspondant à la topologie usuelle 
    de $X$ est celle pour laquelle un crible $\sU$ sur un ouvert $U$ de $X$ est 
    couvrant si la réunion des ouverts appartenant à ce crible est égale à 
    $U$. Il est clair que la catégorie des faisceaux sur $\sS$ est équivalente 
    à la catégorie des faisceaux sur $X$ au sens usuel. 
  \item Soient $X$ un schéma et $\sS$ la catégorie des schémas sur $X$. On 
    appelle topologie fpqc (fidèlement plate quasi-compacte) sur $\sS$ la 
    topologie de Grothendieck pour laquelle un crible sur un $X$-schéma $U$ est 
    couvrant s'il est engendré par une famille finie de morphismes plats dont  
    les images recouvrent $U$. 
\end{enumerate}





\subsubsection{Cohomologie}\label{I:1-6-5}

On supposera toujours que la catégorie $\sS$ a un objet final $X$. Alors on 
appelle sections globales d'un faisceaux abélien $\cF$, et on note 
$\Gamma\cF$ ou $\h^0(X,\cF)$, le groupe $\cF(X)$. Le foncteur 
$\cF\mapsto\Gamma\cF$ est un foncteur exact \'a gauche de la catégorie des 
faisceaux abéliens sur $\sS$ dans la catégorie des groupes abéliens, ou 
note $\h^i(X,-)$ ses dérivés (ou satellites). Ces groupes de cohomologie 
représentent les obstructions à passer du local au global. Par définition, si 
$0\to\cF\to\cG\to\cH\to 0$ est une suite exacte de faisceaux abéliens, on a une 
suite exacte longue de cohomologie:
\[\xymatrix{
  0 \ar[r] 
    & \h^0(X,\cF) \ar[r] 
    & \h^0(X,\cG) \ar[r] 
    & \h^0(X,\cH) \ar[r] 
    & \h^1(X,\cF) \ar[r] 
    & \cdots \\
  \ldots \ar[r] 
    & \h^n(X,\cF) \ar[r] 
    & \h^n(X,\cG) \ar[r]
    & \h^n(X,\cH) \ar[r] 
    & \h^{n+1}(X,\cF) \ar[r] 
    & \cdots
}\]





\subsubsection{}\label{I:1-6-6}

Étant donné un faisceau abélien $\cF$ sur $\sS$, on appelle 
\emph{$\cF$-torseur} un faisceau $\cG$ muni d'une action $\cF\times\cG\to\cG$ 
de $\cF$ telle que localement (après restriction à tous les objets d'un 
crible couvrant l'objet final $X$) $\cG$ muni de l'action de $\cF$ soit 
isomorphe à $\cF$ muni de l'action canonique $\cF\times \cF\to \cF$ par 
translations. 

On peut montrer que $\h^1(X,\cF)$ s'interprète comme l'ensemble des classes 
à isomorphisme près de $\cF$-torseurs. 




















\section{Topologie étale}\label{I:2}

On spécialise les définitions du chapitre précédent au cas de la 
topologie étale d'un schéma $X$ (\ref{I:2-1}, \ref{I:2-2}, \ref{I:2-3}). La 
cohomologie correspondante coïncide dans le cas où $X$ est le spectre d'un 
corps $K$ avec la cohomologie galoisienne de $K$ (\ref{I:2-4}). 










\subsection{Topologie étale}\label{I:2-1}

Nous commencerons par quelques rappels sur la notion de morphisme étale. 

\begin{definition}\label{I:2-1-1}
Soit $A$ un anneau (commutatif). On dit qu'une $A$-algèbre $B$ est étale si 
$B$ est une $A$-algèbre de présentation finie et si les conditions 
équivalentes suivantes sont vérifiées:
\begin{enumerate}[\indent a)]
  \item Pour toute $A$-algèbre $C$ et pour tout idéal de carré nul $J$ de 
    $C$, l'application canonique 
    \[
      \hom_{A\textnormal{-alg}}(B,C) \to \hom_{A\textnormal{-alg}}(B,C/J)
    \]
    est une bijection.
  \item $B$ est un $A$-module plat et $\Omega_{B/A}=0$ (on note $\Omega_{B/A}$ 
    le module des différentielles relatives).
  \item Soit $B=A[X_1,\dotsc,X_n]/I$ une présentation de $B$. Alors pour tout 
    idéal premier $\fp$ de $A[X_1,\dotsc,X_n]$ contenant $I$, il existe des 
    polynômes $P_1,\dotsc,P_n\in I$ tels que $I_\fp$ soit engendré par les 
    images de $P_1,\dotsc,P_n$ et $\det(\partial P_i/\partial X_j)\notin \fp$.  
\end{enumerate}
\end{definition}
(cf. \cite[I]{sga1} ou \cite[V]{ra70})
%[cf. SGA I, exposé I ou M. {\sc Raynoud}, \emph{Anneaux Locaux Henséliens}, chapitre V]. 

On dit qu'un morphisme de schémas $f:X\to S$ est \emph{étale} si pour tout 
$x\in X$ il existe un voisinage ouvert affine $U=\spec(A)$ de $f(x)$ et un 
voisinage ouvert affine $V=\spec(B)$ de $x$ dans $X\times_S U$ tel que $B$ soit 
une $A$-algèbre étale. 





\subsubsection{Exemples}\label{I:2-1-2}
\begin{enumerate}[\indent a)]
  \item Si $A$ est un corps, une $A$-algèbre $B$ est étale si et seulement 
    si c'est un produit fini d'extensions séparables de $A$. 
  \item Si $X$ et $S$ sont des schémas de type fini sur $\dC$, un morphisme 
    $f:X\to S$ est étale si et seulement si son analyticité 
    $f^{\text{an}}:X^{\text{an}}\to Y^{\text{an}}$ est un isomorphisme local.
\end{enumerate}





\subsubsection{Sorite}\label{I:2-1-3}
\begin{enumerate}[\indent a)]
  \item (changement de base) Si $f:X\to S$ est un morphisme étale, il en est 
    de même de $f_{S'}:X\times_S S'\to S'$ pour tout morphisme $S'\to S$. 
  \item Si $f:X\to S$ et $g:Y\to S$ sont deux morphismes étales, tout 
    $S$-morphisme de $X$ dans $Y$ est étale.
  \item (descente) Soit $f:X\to S$ un morphisme. S'il existe un morphisme 
    fidèlement plat $S'\to S$, tel que $f_{S'}:X\times_S S'\to S'$ soit 
    étale, alors $f$ est étale.
\end{enumerate}





\subsubsection{}\label{I:2-1-4}

Soit $X$ un schéma. Soit $\sS$ la catégorie des $X$-schémas étales; 
d'après (\ref{I:2-1-3}.c) tout morphisme de $\sS$ est un morphisme étale. 
On appelle \emph{topologie étale} sur $\sS$ la topologie pour laquelle un 
crible sur $U$ est couvrant s'il est engendré par une famille finie de 
morphismes $\varphi_i:U_i\to U$ tels que la réunion des images des 
$\varphi_i$ recouvre $U$. On appelle \emph{site étale} de $X$, et note 
$\et X$, le site défini par $\sS$ de la topologie étale. 










\subsection{Exemples de faisceaux}\label{I:2-2}





\subsubsection{Faisceau constant}\label{I:2-2-1}

Soit $C$ un groupe abélien et supposons pour simplifier $X$ noethérien. On 
notera $\const C_X$ (ou même $C$ s'il n'y a pas d'ambiguïté) le faisceau 
défini par $U\mapsto C^{\pi_0(U)}$, où $\pi_0(U)$ est l'ensemble (fini) des 
composantes connexes de $U$. Le cas le plus important sera $C=\dZ/n$. On a donc 
par définition 
\[
  \h^0(X,\dZ/n)=\left(\dZ/n\right)^{\pi_0(X)}\text{.}
\]
De plus $\h^1(X,\dZ/n)$ est l'ensemble des classes d'isomorphisme de 
$\dZ/n$-torseurs (\ref{I:1-6-6}), autrement dit de revêtements étales 
galoisiens de $X$ de groupe $\dZ/n$. En particulier, si $X$ est connexe et si 
$\pi_1(X)$ est son groupe fondamental pour un pointe bas choisi, on a 
\[
  \h^1(X,\dZ/n) = \hom(\pi_1(X),\dZ/n)\text{.}
\]





\subsubsection{Groupe multiplicatif}\label{I:2-2-2}

On notera $\dG_{m,X}$ (ou $\dG_m$ s'il n'y a pas d'ambiguïté) le faisceau 
défini par $U\mapsto \Gamma(U,\cO_U^\times)$; il s'agit bien d'un faisceau 
grâce au théorème de descente fidèlement plate (\ref{I:1-4-5}). On a 
par définition 
\[
  \h^0(X,\dG_m) = \h^0(X,\cO_X)^\times\text{;}
\]
en particulier si $X$ est réduit, connexe et propre sur un corps 
algébriquement clos $k$, on a:
\[
  \h^0(X,\dG_m) = k^\times\text{.}
\]

\begin{proposition}\label{I:2-2-3}
On a un isomorphisme:
\[
  \h^1(X,\dG_m) = \pic(X)\text{,}
\]
où $\pic(X)$ est le groupe des classes de faisceaux inversibles sur $X$.
\end{proposition}
\begin{proof}
Soit $*$ le foncteur qui, à un faisceau inversible $\cL$ sur $X$, associe le 
préfaisceau $\cL^*$ suivant sur $\et X$: pour $\varphi:U\to X$ étale, 
\[
  \cL^*(U)=\operatorname{Isom}_U(\cO_U,\varphi^*\cL)\text{.}
\]
% the original references here are just to (4.2) and (4.5), but the context 
% makes it seem that they refer to the results in chapter 1. Likewise a little 
% later
D'après (\ref{I:1-4-2}.i) et et (\ref{I:1-4-5}) (pleine fidélité), ce 
préfaisceau est un faisceau; c'est même un $\dG_m$-torseur. On vérifie 
aussitôt que 
\begin{enumerate}[\indent a)]
  \item le foncteur $*$ est compatible à la localisation (étale);
  \item il induit une équivalence de la catégorie des faisceaux 
    inversibles triviaux (i.e. à $\cO_X$) avec la catégorie des 
    $\dG_m$-torseurs triviaux: $\cL$ est trivial si et seulement si $\cL^*$ 
    l'est.
\end{enumerate}

De plus, d'après (\ref{I:1-4-2}.ii) et (\ref{I:1-4-5}), 
\begin{enumerate}[\indent a)]
\setcounter{enumi}{2}
  \item la notion de faisceau inversible est locale pour la topologie étale. 
\end{enumerate}

Il résulte formellement de a), b), c) que $*$ est une équivalence entre la 
catégorie des faisceaux inversibles sur $X$ est celle des $\dG_m$-torseurs 
sur $\et X$; elle induit l'isomorphisme cherché. On construit comme suit 
l'équivalence inverse: si $\cT$ est un $\dG_m$-torseur, il existe un 
recouvrement étale fini $\{U_i\}$ de $X$ tel que les torseurs $\cT/U_i$ soient 
triviaux; $\cT$ est alors trivial sur chaque $V$ étale sur $X$ appartenant au 
crible $\sU\subset \et X$ engendré par $\{U_i\}$. Sur chaque 
$V\in\sU$, $\cT|V$ correspond à un faisceau inversible $\cL_V$ (par b)) et les 
$\cL_V$ constituent un faisceau inversible donné $\sU$-localement $\cL_\sU$ 
(par a)). Par c), ce dernier provient d'un faisceau inversible $\cL(\cT)$ sur 
$X$, et $\cT\mapsto \cL(\cT)$ est l'inverse cherché de $*$. 
\end{proof}





\subsubsection{Racines de l'unité}\label{I:2-2-4}

Pour tout entier $n>0$, on appelle faisceau des racines $n$-ièmes de 
l'unité, et on note $\dmu_n$, le noyau de l'élévation à puissance 
$n$-iéme dans $\dG_m$. Si $X$ est un schéma sur un corps séparablement 
clos $k$ et si $n$ est inversible dans $k$, le choix d'une racine primitive 
$n$-ième de l'unité $\zeta\in k$ définit un isomorphisme 
$i\mapsto \zeta^i$ de $\dZ/n$ avec $\dmu_n$. 

La relation entre cohomologie à coefficients dans $\dmu_m$ et cohomologie à 
coefficients dans $\dG_m$ est donnée par la suite exacte de cohomologie déduite 
de la 






\begin{theorem}[Théorie de Kummer]\label{I:2-2-5}
Si $n$ est inversible sur $X$, l'élévation à la puissance $n$-ième dans 
$\dG_m$ est un épimorphisme de faisceaux. On a donc une suite exacte 
\[
  0 \to \dmu_n \to \dG_m \to \dG_m \to 0\text{.}
\]
\end{theorem}
\begin{proof}
Soient $U\to X$ un morphisme étale et $a\in \dG_m(U) =\Gamma(U,\cO_U^\times)$. 
Puisque $n$ est inversible sur $U$, l'équation $T^n-a = 0$ est séparable; 
autrement dit $U'=\spec\left(\cO_U[T]/(T^n-a)\right)$ est étale su-dessus de 
$U$. Par ailleurs $U'\to U$ est surjectif et a admet une racine $n$-ième sur 
$U'$, d'où résultat. 
\end{proof}










\subsection{Fibres, images directes}\label{I:2-3}





\subsubsection{}\label{I:2-3-1}

On appelle \emph{pointe géométrique} de $X$ un morphisme $\bar x\to X$, où 
$\bar x$ est le spectre d'un corps séparablement clos $k(\bar x)$. On le notera 
abusivement $\bar x$, sous-entendent le morphisme $\bar x\to X$. Si $x$ est 
l'image de $\bar x$ dans $X$, on dit que $\bar x$ est centré en $x$. Si le 
corps $k(\bar x)$ est une extension algébrique du corps résiduel $k(x)$, on dit 
que $\bar x$ est un point géométrique \emph{algébrique} de $X$. 

On appelle \emph{voisinage étale} de $\bar x$ un diagramme commutatif 
\[\xymatrix{
  & U \ar[d] \\
  \bar x \ar[ur] \ar[r] 
  & X\text{,}
}\]
où $U\to X$ est un morphisme étale. 

Le \emph{localisé stricte} de $X$ en $\bar x$ est l'anneau 
$\cO_{X,\bar x} = \varinjlim \Gamma(U,\cO_U)$, la limite inductive étant sur les 
voisinages étales de $\bar x$. C'est un anneau local strictement hensélien dont 
le corps résiduel est la clôture séparable du corps résiduel $k(x)$ de $X$ en 
$x$ dans $k(\bar x)$. Il joue le rôle d'anneau local pour la topologie étale. 





\subsubsection{}\label{I:2-3-2}

Étant donné un faisceau $\cF$ sur $\et X$, on appelle \emph{fibre} de $\cF$ en 
$\bar x$ l'ensemble (resp. le groupe,\dots) $\cF_{\bar x}=\varprojlim \cF(U)$, 
la limite inductive étant toujours prise sur les voisinages étales de  $X$. 

Pour qu'un homomorphisme de faisceaux $\cF\to \cG$ soit un 
mono-/epi-/isomorphisme il faut et il suffit qu'il en soit ainsi des morphismes 
$\cF_{\bar x}\to \cG_{\bar x}$ induit sur les fibres et tout point géométrique 
de $X$. Si $X$ est de type fini sur un corps algébriquement clos, il suffit 
qu'il en soit ainsi en les points rationnelles de $X$. 





\subsubsection{}\label{I:2-3-3}

Si $f:X\to Y$ est un morphisme de schémas et $\cF$ un faisceau sur $\et X$, 
l'\emph{image directe} $f_* \cF$ de $\cF$ par $f$ est le faisceau sur $\et Y$ 
défini par $f_* \cF(V) = \cF(X\times_Y V)$ pour tout $V$ étale sur $Y$. Le foncteur 
$f_*:\sh(\et X)\to \sh(\et Y)$ est exact à gauche. Ses foncteurs 
dérivés à droite $\R^q f_*$ s'appellent images directes supérieures. Si 
$\bar y$ est un point géométrique de $Y$, on a 
\[
  (\R^q f_* \cF)_{\bar y} = \varinjlim \h^q(V\times_Y X,\cF)\text{,}
\]
limite inductive prise sur les voisinages étales $V$ de $\bar y$. 

Soient $\cO_{Y,\bar y}$ le localisé stricte de $Y$ en $\bar y$, 
$\widetilde Y=\spec(\cO_{Y,\bar y})$ et $\widetilde X=X\times_Y \widetilde Y$. 
On peut étendre $\cF$ à $\et{\widetilde X}$ (c'est un cas particulier de la 
notion générale d'image réciproque) de la manière suivante: soit 
$\widetilde U$ un schéma étale sur $\widetilde X$, alors il existe un 
voisinage étale $V$ de $\bar y$ et un schéma étale $U$ sur $X\times_Y V$ tel 
que $\widetilde U=U\times_V \widetilde Y$; on posera 
\[
  \cF(\widetilde U)=\varinjlim \cF\left(U\times_V V'\right)\text{,}
\]
le limite inductive étant prise sur les voisinages étales $V'$ de $\bar y$ qui 
dominent $V$. Avec cette définition, on a 
\[
  \left(\R^q f_* \cF\right)_{\bar y} = \h^q(\widetilde X,\cF)\text{.}
\]

Le foncteur $f_*$ a un adjoint à gauche $f^*$, le foncteur ``image 
réciproque.'' Si $\bar x$ est un point géométrique de $X$ et $f(\bar x)$ son 
image dans $Y$, on a $(f^* \cF)_{\bar x} = \cF_{f(\bar x)}$. Cette formule montre 
que $f^*$ est un foncteur exact. Le foncteur $f_*$ transforme donc faisceau 
injectif en faisceau injectif, et la suite spectrale du foncteur compose 
$\Gamma\circ f_*$ (resp. $g_* f_*$) fournit la 





\begin{theorem}[Suite spectrale de Leray]\label{I:2-3-4}
Soient $\cF$ un faisceau abélien sur $\et X$ et $f:X\to Y$ un morphisme de 
schémas (resp. des morphismes de schémas $X\xrightarrow f Y \xrightarrow g Z$). 
On a une suite spectrale 
\begin{align*}
  E_2^{pq} &= \h^p(Y,\R^q f_* \cF) \Rightarrow \h^{p+q}(X,\cF) \\
  \text{(resp.}\qquad E_2^{pq} &= \R^p g_* \R^q f_* \cF \Rightarrow \R^{p+q}(gf)_* \cF\text{).}
\end{align*}
\end{theorem}





\begin{corollary}\label{I:2-3-5}
Si $\R^q f_* \cF=0$ pour tout $q>0$, on a $\h^p(Y,f_* \cF)=\h^p(X,\cF)$ (resp. 
$\R^p g_*(\cF_* \cF) = \R^p(g f)_* \cF$) pour tout $p\geqslant 0$. 
\end{corollary}

Cela s'applique en particulier dans le cas suivant:





\begin{proposition}\label{I:2-3-6}
Soit $f:X\to Y$ un morphisme fini (voire, par passage à la limite, un 
morphisme entier) et $\cF$ un faisceau abélien sur $X$. Alors $\R^q f_* \cF=0$, 
pour tout $q>0$.
\end{proposition}

En effet soient $\bar y$ un pointe géométrique de $Y$, $\widetilde Y$ le 
spectre du localisé strict de $Y$ en $y$ et 
$\widetilde X=X\times_Y \widetilde Y$; d'après ce qui précède, il suffit de 
montrer que $\h^q(\widetilde X,\cF) = 0$ pour tout $q>0$. Or $\widetilde X$ est le 
spectre d'un produit d'anneaux locaux strictement henséliens (cf. 
\cite[I]{ra70}), le foncteur $\Gamma(\widetilde X,-)$ est exact car tout 
$\widetilde X$-schéma étale et surjectif admet une section, d'où l'assertion. 










\subsection{Cohomologie galoisienne}\label{I:2-4}

Pour $X=\spec(K)$ le spectre d'un corps, nous allons voir que la cohomologie 
étale s'identifie à la cohomologie galoisienne. 





\subsubsection{}\label{I:2-4-1}

Commençons par une analogie topologique. Si $K$ est le corps des fonctions 
d'une variété algébrique affine intègre $Y=\spec(A)$ sur $\dC$, on a 
$K=\varprojlim_{f\in A} A[1/f]$. 

Autrement dit $X=\varprojlim U$, $U$ parcourant l'ensemble des ouverts de $Y$. 
On sait qu'il existe des ouverts de Zariski arbitrairement petits qui pour la 
topologie classique sont des $K(\pi,1)$. On ne sera donc pas surpris si l'on 
considère $\spec(K)$ lui-même comme un $K(\pi,1)$, $\pi$ étant le groupe 
fondamental (au sens algébrique) de $X$, autrement dit le groupe de Galois de 
$\bar K/K$, où $\bar K$ est clôture séparable de $K$. 





\subsubsection{}\label{I:2-4-2}

Plus précisément soient $K$ un corps, $\bar K$ un clôture séparable de $K$ et 
$G = \gal(\bar K/K)$ le groupe de Galois topologique. A toute $K$-algèbre finie 
étale $A$ (produit fini d'extensions séparables de $K$), associons l'ensemble 
fini $\hom_K(A,\bar K)$. Le groupe de Galois $G$ opère sur cet ensemble à 
travers un quotient discret (donc fini). Si $A=K[T]/(F)$, il s'identifie à 
l'ensemble des racines dans $\bar K$ du polynôme $F$. La théorie de Galois, 
sous la forme que lui a donnée Grothendieck, dit que:





\begin{proposition}\label{I:2-4-3}
Le foncteur 
\[
  \left\{\begin{array}{c}
          \textnormal{$K$-algèbres} \\ 
          \textnormal{finies étales}
        \end{array}\right\}
  \to  
  \left\{\begin{array}{c}
          \textnormal{ensembles finis sur lesquels} \\
          \textnormal{$G$ opère continûment}
        \end{array}\right\}
\]
qui à une algèbre étale $A$ associe $\hom_K(A,\bar K)$ est une 
anti-équivalence de catégories.
\end{proposition}

On en déduit une description analogue des faisceaux pour la topologie étale sur 
$\spec(K)$:





\begin{proposition}\label{I:2-4-4}
Le foncteur 
\[
  \left\{\begin{array}{c}
          \textnormal{Faisceaux étales} \\ 
          \textnormal{sur $\spec(K)$}
        \end{array}\right\}
  \to 
  \left\{\begin{array}{c}
          \textnormal{ensembles sur lesquels} \\
          \textnormal{$G$ opère continûment}
        \end{array}\right\}
\]
qui à un 
faisceau $\cF$ associe sa fibre $\cF_{\bar K}$ au point géométrique 
$\spec(\bar K)$ est une équivalence de catégories.
\end{proposition}

On dit que $G$ opère continûment sur un ensemble $E$ si le fixateur de tout 
élément de $E$ est un sous-groupe ouvert de $G$. Le foncteur en sens 
inverse est décrit de la manière évidente: soient $A$ une 
$K$-algèbre finie étale, $U=\spec(A)$ et $U(\bar K)=\hom_K(A,\bar K)$ le 
$G$-ensemble correspondant à; alors on a 
$\cF(U)= \hom_{G\text{-ens}}(U(\bar K),\cF_{\bar K})$. 

En particulier, si $X=\spec(K)$, on a $\cF(X) = \cF_{\bar K}^G$. Si l'on se 
restreint aux faisceaux abéliens, on obtient en passant aux foncteurs 
dérivés des isomorphismes canoniques 
\[
  \h^q(\et X,\cF) = \h^q(G,\cF_{\bar K})
\]





\subsubsection{Exemples}\label{I:2-4-5}

\begin{enumerate}[\indent a)]
  \item Au faisceau constant $\dZ/n$ correspond $\dZ/n$ avec action triviale de 
    $G$. 
  \item Au faisceau des racines $n$-ièmes de l'unité $\dmu_n$ correspond 
    le groupe $\dmu_n(\bar K)$ des racines $n$-ièmes de l'unité dans 
    $\bar K$, avec l'action naturelle de $G$.
  \item Au faisceau $\dG_m$ correspond le groupe $\bar K^\times$ avec l'action 
    naturelle de $G$.
\end{enumerate}




















\section{Cohomologie des courbes}\label{I:3}

Dans le cas des espaces topologiques, des dévissages utilisant la formule de 
K\"unneth et des décompositions simpliciales permettent de se ramener pour 
calculer la cohomologie à l'intervalle $I=[0,1]$ pour lequel on a 
$\h^0(I,\dZ)=\dZ$ et $\h^q(I,\dZ)=0$ pour $q>0$. 

Dans notre cas, les dévissages aboutiront à des objets plus compliqués, 
à savoir les courbes sur un corps algébriquement clos; nous allons calculer 
leur cohomologie dans ce chapitre. La situation est plus complexe que dans le 
cas topologique car les groupes de cohomologie sont nuis pour $q>2$ seulement. 
L'ingrédient essentiel des calculs est la nullité du groupe de Brauer du 
corps des fonctions d'une telle courbe (théorème de Tsen, \ref{I:3-2}). 










\subsection{Le groupe de Brauer}\label{I:3-1}

Rappelons-en tout d'abord la définition classique:





\begin{definition}\label{I:3-1-1}
Soit $K$ un corps et $A$ une $K$-algèbre de dimension finie. On dit que $A$ 
est une algèbre simple centrale sur $K$ si les conditions équivalentes 
suivantes sont vérifiées:
\begin{enumerate}[\indent a)]
  \item $A$ n'a pas d'idéal bilatère non trivial et son centre est $K$. 
  \item Il existe une extension galoisienne finie $K'/K$ telle que 
    $A_{K'} = A\otimes_K K'$ soit isomorphe à une algèbre de matrices 
    carrées sur $K'$.
  \item $A$ est $K$-isomorphe à algèbre de matrices carrées sur un corps 
    gauche de centre $K$.
\end{enumerate}
\end{definition}

Deux telles algèbres sont dites équivalentes si les corps gauches qui 
leur sont associés par c) sont $K$-isomorphes. Si ces algèbres out 
même dimension, cela revient à dire qu'elles sont $K$-isomorphes. Le 
produit tensoriel définit par passage au quotient une structure de groupe 
abélien sur l'ensemble des classes d'équivalence. C'est ce groupe que l'on 
appelle classiquement \emph{le groupe de Brauer} de $K$ et que l'on note 
$\br(K)$. 





\subsubsection{}\label{I:3-1-2}

On notera $\br(n,K)$ l'ensemble des classes de $K$-isomorphisme de 
$K$-algèbres $A$ telles qu'il existe une extension galoisienne finie $K'$ de 
$K$ pour laquelle $A_{K'}$ est isomorphe à l'algèbre $M_n(K')$ des matrices 
carrées $n\times n$ sur $K'$. Par définition $\br(K)$ est réunion des 
sous-ensembles $\br(n,K)$ pour $n\in\dN$. Soient $\bar K$ une clôture 
algébrique de $K$ et $G=\gal(\bar K/K)$. L'ensemble $\br(n,K)$ est 
l'ensemble des ``formes'' de $M_n(\bar K)$, il est donc canoniquement 
isomorphe à $\h^1\left(G,\aut(M_n(\bar K))\right)$. 

On sait que tout automorphisme de $M_n(\bar K)$ est intérieur. Par 
conséquent le groupe $\aut(M_n(\bar K))$ s'identifie au groupe linéare 
projectif $\pgl(n,\bar K)$ et on a une bijection canonique:
\[
  \theta_n : \br(n,K) \iso \h^1\left(G,\pgl(n,\bar K)\right)\text{.}
\]

D'autre part la suite exacte:
\begin{equation}\label{I:eq:3-1-2-1}
  1 \to \bar K^\times \to \gl(n,\bar K) \to \pgl(n,\bar K) \to 1\text{,}
\end{equation}
permet de définir un opérateur cobord:
\[
  \Delta_n : \h^1\left(G,\pgl(n,\bar K)\right) \to \h^2(G,\bar K^\times)\text{.}
\]

En composant $\theta_n$ et $\Delta_n$, on obtient une application:
\[
  \delta_n : \br(n,K) \to \h^2(G,\bar K^\times)\text{.}
\]
On vérifie facilement que les applications $\delta_n$ sont compatibles entre 
elles et définissent un homomorphisme de groupes:
\[
  \delta : \br(K) \to \h^2(G,\bar K^\times)\text{.}
\]





\begin{proposition}\label{I:3-1-3}
L'homomorphisme $\delta:\br(K)\to \h^2(G,\bar K^\times)$ est bijectif. 
\end{proposition}

Cela résulte des deux lemmes suivants:





\begin{lemma}\label{I:3-1-4}
L'application 
$\Delta_n:\h^1\left(G,\pgl(n,\bar K)\right)\to \h^2(G,\bar K^\times)$ est 
injective.
\end{lemma}

D'après \cite{se94}, cor. à la prop. I-44, il suffit de vérifier que chaque 
fois qu'on tord la suite exacts (\ref{I:eq:3-1-2-1}) par un élément de 
$\h^1\left(G,\pgl(n,\bar K)\right)$, le $\h^1$ du groupe médian est trivial. 
Ce groupe médian est le groupe des $\bar K$-points du groupe multiplicatif 
d'une algèbre centrale simple $A$ de rang $n^2$ sur $K$. Pour prouver que 
$\h^1(G,A_{\bar K}^\times) = 0$, n interprète $A^\times$ comme le groupe 
des automorphismes du $A$-module libre $L$ de rang $1$, et $\h^1$ comme 
l'ensemble des ``formes'' de $L$ -- des $A$-modules de rang $n^2$ sur $K$, 
automatiquement libres. 





\begin{lemma}\label{I:3-1-5}
Soient $\alpha\in \h^2(G,\bar K^\times)$, $K'$ un extension finie de $K$ 
contenue dans $\bar K$, $n=[K':K]$, et $G'=\gal(\bar K/K')$. Si l'image de 
$\alpha$ dans $\h^2(G',\bar K^\times)$ est nulle, alors, $\alpha$ appartient 
à l'image de $\Delta_n$. 
\end{lemma}

Remarquons tout d'abord qu'on a: 
\[
  \h^2(G',\bar K^\times) \simeq \h^2\left(G,(\bar K\otimes_K K')^\times\right)\text{.}
\]
(D'un point de vue géométrique si l'on note $x=\spec(K)$, $x'=\spec(K')$ et 
$\pi:x'\to x$ le morphisme canonique, on a $\R^q \pi_*(\dG_{m,x'}) = 0$ pour 
$q>0$ et par suite $\h^q(x',\dG_{m,X'})\simeq \h^q(x,\pi_*\dG_{m,X'})$ pour 
$q\geqslant 0$). 

Par ailleurs la choix d'une base de $K'$ en tant qu'espace vectoriel sur $K$ 
permet de définir un homomorphisme 
\[
  (\bar K\otimes_K K')^\times \to \gl(n,\bar K)
\]
qui, à un élément $x$, fait correspondre l'endomorphisme de 
multiplication par $x$ de $\bar K\otimes_K K'$. On a alors un diagramme 
commutatif à lignes exactes: 
\[\xymatrix{
  1 \ar[r] 
    & \bar K^\times \ar[r] \ar@{=}[d]
    & (\bar K\otimes_K K')^\times \ar[r] \ar[d] 
    & (\bar K\otimes_K K')^\times / \bar K^\times \ar[r] \ar[d] 
    & 1 \\
  1 \ar[r] 
    & \bar K^\times \ar[r] 
    & \gl(n,\bar K) \ar[r] 
    & \pgl(n,\bar K) \ar[r] 
    & 1
}\]
Le lemme résulte du diagramme commutatif que l'on en déduit en passant 
à la cohomologie:
\[\xymatrix{
  \h^1\left(G,(\bar K\otimes_K K')^\times/\bar K^\times\right) \ar[r] \ar[d] 
    & \h^2(G,\bar K^\times) \ar[r] \ar@{=}[d]
    & \h^2\left(G,(\bar K\otimes_K K')^\times\right) \\
  \h^1\left(G,\pgl(n,\bar K)\right) \ar[r]^-{\Delta_n} 
    & \h^2(G,\bar K^\times) \text{.}
}\]





\begin{proposition}\label{I:3-1-6}
Soient $K$ un corps, $\bar K$ une clôture algébrique de $K$ et 
$G=\gal(\bar K/K)$. Supposons que, pour tout extension finie $K'$ de $K$, on 
ait $\br(K')=0$. Alors on a:
\begin{enumerate}[\indent i)]
  \item $\h^q(G,\bar K^\times) = 0$ pour tout $q>0$.
  \item $\h^q(G,\cF) = 0$ pour tout $G$-module de torsion $\cF$ et pour tout 
    $q\geqslant 2$.
\end{enumerate}
\end{proposition}

(Pour la démonstration, cf. \cite{se94}).










\subsection{Le théorème de Tsen}\label{I:3-2}





\begin{definition}\label{I:3-2-1}
On dit qu'un corps $K$ est $C_1$ si tout polynôme homogène non constant 
$f(x_1,\dotsc,x_n)$ de degré $d<n$ a un zéro non trivial.
\end{definition}





\begin{proposition}\label{I:3-2-2}
Si un corps $K$ est $C_1$, on a $\br(K)=0$.
\end{proposition}

Il s'agit de montrer que tout corps gauche $D$ de centre $K$ et fini sur $K$ 
est égale à $K$. Soient $r^2$ le degré de $D$ sur $K$ et 
$\operatorname{Nrd}:D\to K$ la norme réduite. 

(Localement pour la topologie étale sur $K$, $D$ est isomorphe -- non 
canoniquement -- à une algèbre de matrices $M_r$ et la norme réduite 
coïncide avec l'application déterminant. Celle-ci est bien définie, 
indépendamment de l'isomorphisme choisi entre $D$ et $M_r$ car tout 
automorphisme de $M_r$ est intérieur et deux matrices semblables ont 
même déterminant. Cette application définie localement pour la 
topologie étale se descende, à cause de son unicité locale, en une 
application $\operatorname{Nrd}:D\to K$). 

Le seul zéro de $\operatorname{Nrd}$ est l'élément nul de $D$, car, si 
$x\ne 0$, on a $\operatorname{Nrd}(x)\cdot \operatorname{Nrd}(x^{-1}) = 1$. 
D'autre part, si $\{e_1,\dotsc,e_{r^2}\}$ est une base de $D$ sur $K$ et si 
$x=\sum x_i e_i$, la fonction $\operatorname{Nrd}(x)$ s'écrit comme un polynôme 
homogène $\operatorname{Nrd}(x_1,\dotsc,x_{r^2})$ de degré $r$ (c'est 
clair localement pour la topologie étale). Puisque $K$ est $C_1$, on a 
$r^2\leqslant r$, c'est-à-dire $r=1$ et $D=K$. 





\begin{theorem}[Tsen]\label{I:3-2-3}
Soient $k$ un corps algébriquement clos et $K$ une extension de degré de 
transcendance $1$ de $k$. Alors $K$ est $C_1$.
\end{theorem}

Supposons tout d'abord que $K=k(X)$. Soit 
\[
  f(T) = \sum a_{i_1,\dotsc i_n} T_1^{i_1} \dotsm T_n^{i_n}
\]
un polynôme homogène de degré $d<n$ à coefficients dans $k(X)$. Quitte 
à multiplier les coefficients par un dénominateur commun on peut supposer 
qu'ils sont dans $k[X]$. Soit alors $\delta=\sup\deg(a_{i_1\dotsc i_n})$. On 
cherche un zéro non trivial dans $k[X]$ par la méthode des coefficients 
indéterminés en écrivant chaque $T_i$ ($i=1,\dotsc,n$) comme un 
polynôme de degré $N$ en $X$. Alors l'équation $f(T)=0$ devient un système 
d'équations homogènes en les $n\times (N+1)$ coefficients des polynômes 
$T_i(X)$ exprimant la nullité des coefficients du polynôme en $X$ obtenu en 
remplaçant $T_i$ par $T_i(X)$. Ce polynôme est de degré $\delta+N D$ au plus, 
il y a donc $\delta+N d+1$ équations en $n\times (N+1)$ variables. Comme $k$ 
est algébriquement clos ce système a une solution non triviale si 
$n(N+1)>N d+\delta+1$, ce qui sera le cas pour $N$ assez grand si $d<n$. 

Il est clair que, pour démontrer le théorème dans la cas général, il 
suffit de le démontrer lorsque $K$ est une extension finie d'une extension 
transcendent pure $k(X)$ de $k$. Soit $f(T)=f(T_1,\dotsc,T_n)$ un 
polynôme homogène de degré $d<n$ à coefficients dans $K$. Soient 
$a=[K:k(X)]$ et $e_1,\dotsc,e_s$ une bas de $K$ sur $k(X)$. Introduisons de 
nouvelles variables $U_{i j}$, en nombre $s n$, telles que 
$T_i=\sum U_{i j}e_j$. Pour que le polynôme $f(T)$ ait un zéro 
non trivial dans $K$, il suffit que le polynôme 
$g(X_{i j}) = N_{K/k}(f( T))$ ait un zéro non trivial dans $k(X)$. 
Or $g$ est un polynôme de degré $s d$ en $s n$ variables, d'où le 
résultat. 





\begin{corollary}\label{I:3-2-4}
  Soient $k$ un corps algébriquement clos et $K$ une extension de degré 
de transcendance $1$ de $k$. Alors les groupes de cohomologie étale 
$\h^q(\spec(K),\dG_m)$ sont nuls pour tout $q>0$. 
\end{corollary}










\subsection{Cohomologie des courbes lisses}

Dorénavant, et sauf mention expresse du contraire, les groupes de 
cohomologie considérés sont les groupes des cohomologie étale. 





\begin{proposition}\label{I:3-3-1}
Soient $k$ un corps algébriquement clos et $X$ une courbe projective non 
singulière connexe sur $k$. Alors on a:
\begin{align*}
  \h^0(X,\dG_m) &= k^\times \text{,} \\
  \h^1(X,\dG_m) &= \pic(X) \text{,} \\
  \h^q(X,\dG_m) &= 0 \text{ pour $q\geqslant 2$.}
\end{align*}
\end{proposition}

Soient $\eta$ le point générique de $X$, $j:\eta\to X$ le morphisme 
canonique et $\dG_{m,X}$ le groupe multiplicatif du corps des fractions 
$K(X)$. Pour tout pointe fermé $x$ de $X$, soient $i_x:x\to X$ l'immersion 
canonique et $\dZ_x$ le faisceau constant de valeur $\dZ$ sur $x$. Ainsi 
$j_*\dG_{m,\eta}$ est le faisceau des fonctions méromorphes non nulles sur 
$X$ et $\oplus_{x\in X} i_{x_I}\dZ_x$ le faisceau des diviseurs, on a donc une 
suite exacte de faisceaux:
\begin{equation}\label{I:eq:3-3-1-1}
\xymatrix{
  0 \ar[r] 
    & \dG_m \ar[r]
    & j_* \dG_{m,\eta} \ar[r]^-{\dv} \ar[r] 
    & \displaystyle\bigoplus_{x\in X} i_{x*} \dZ_x \ar[r] 
    & 0\text{.}
}
\end{equation}





\begin{lemma}\label{I:3-3-2}
On a $\R^q j_* \dG_{m,\eta} = 0$ pour tout $q>0$.
\end{lemma}

Il suffit de montrer que la fibre de ce faisceau en tout pointe fermé $x$ 
de $X$ est nulle. Si $\tilde\cO_{X,x}$ est l'hensélisé de $X$ en $x$ et $K$ 
le corps des fractions de $\tilde\cO_{X,x}$, on a 
\[
  \spec(K) = \eta\times_X \spec(\tilde\cO_{X,x})\text{,}
\]
donc $(\R^qj_*\dG_{m,\eta})_x = \h^q(\spec(K),\dG_m)$. 

Or $K$ est une extension algébrique de $k(X)$, donc une extension de degré 
de transcendante $1$ de $k$: le lemme résulte de (\ref{I:3-2-4}). 





\begin{lemma}\label{I:3-3-3}
On a $\h^q(X, j_*\dG_{m,\eta}) = 0$ pour tout $q>0$.
\end{lemma}

En effet de (\ref{I:3-3-2}) et de la suite spectrale de Leray pour $j$, on 
déduit:
\[
  \h^q(X, j_* \dG_{m,\eta}) = \h^q(\eta,\dG_{m,\eta})
\]
pour tout $q\geqslant 0$ et le deuxième membre est nul pour $q>0$ d'après 
(\ref{I:3-2-4}). 





\begin{lemma}\label{I:3-3-4}
On a $\h^q\left(X,\bigoplus_{x\in X} i_{x*} \dZ_x\right) = 0$ pour tout $q>0$. 
\end{lemma}

En effet pour tout point fermé de $X$, on a $\R^q i_{x*}\dZ_x 0$ pour $q>0$, 
car $i_x$ est un morphisme fini (\ref{I:2-3-6}), et 
\[
  \h^q(X,i_{x*}\dZ_x) = \h^q(x,\dZ_x)\text{.}
\]
Le deuxième membre est nul pour tout $q>0$, car $x$ est la spectre d'un 
corps algébriquement clos (On voit que le lemme est vrai plus 
généralement pour tout faisceau ``gratte-ciel'' sur $X$). 

On déduit des lemmes précédents et de la suite exacte (\ref{I:eq:3-3-1-1}) 
les égalités:
\[
  \h^q(X,\dG_m) = 0 \text{ pour $q\geqslant 2$,}
\]
et une suite exacte de cohomologie en bas degré:
\[
  1 \to \h^0(X,\dG_m) \to \h^0(X, j_*\dG_{m,\eta}) \to \h^0\left(X,\bigoplus_{x\in X} i_{x*}\dZ_x\right) \to \h^1(X,\dG_m) \to 1
\]
qui n'est autre que la suite exacte:
\[
  1 \to k^\times \to k(X)^\times\to \Div(X) \to \pic(X) \to 1\text{.}
\]
De la proposition \ref{I:3-3-1} on déduit que les groupes de cohomologie de 
$X$ à valeur dans $\dZ/n$, $n$ premier à la caractéristique de $k$, ont 
une valeur raisonnable:





\begin{corollary}\label{I:3-3-5}
Si $X$ est de genre $g$ et si $n$ est inversible dans $k$, les 
$\h^q(X,\dZ/n)$ sont nuls pour $q>2$, et libres sur $\dZ/n$ de rang $1,2 g,1$ 
pour $q=0,1,2$. Remplaçant $\dZ/n$ par le groupe isomorphe $\dmu_n$, on a des 
isomorphismes canoniques 
\begin{align*}
  \h^0(X,\dmu_n) &= \dmu_n \\
  \h^1(X,\dmu_n) &= \pic^0(X)_n \\
  \h^2(X,\dmu_n) &= \dZ/n \text{.}
\end{align*}
\end{corollary}

Comme le corps $k$ est algébriquement clos, $\dZ/n$ est isomorphe (non 
canoniquement) à $\mu_n$. De la suite exacte de Kummer:
\[
  0 \to \mu_n \to \dG_m \to \dG_m \to 0\text{,}
\]
et de la proposition \ref{I:3-3-1}, on déduit les égalité:
\[
  \h^q(X,\dZ/n) = 0 \text{ pour $q>2$,}
\]
et, en bas degré, des suites exactes:
\[\xymatrix{
  0 \ar[r] 
    & \h^0(X,\dmu_n) \ar[r] 
    & k^\times \ar[r]^-n 
    & k^\times \ar[r] 
    & 0
}\]
\[\xymatrix{
  0 \ar[r] 
    & \h^1(X,\dmu_n) \ar[r] 
    & \pic(X) \ar[r]^-n 
    & \h^2(X,\dmu_n) \ar[r] 
    & 0\text{.}
}\]
De plus on a une suite exacte:
\[\xymatrix{
  0 \ar[r] 
    & \pic^0(X) \ar[r] 
    & \pic(X) \ar[r]^-{\deg}
    & \dZ \ar[r] 
    & 0 \text{,}
}\]
et $\pic^0(X)$ s'identifie au groupe des points rationnels sur $k$ d'une 
variété abélienne de dimension $g$, la jacobienne de $X$. Dans un tel 
groupe, la multiplication par $n$ est surjective et son noyau est un 
$\dZ/n\dZ$-module libre de rang $2 g$ (car $n$ est inversible dans $k$); d'où 
le corollaire. 

De dévissage astucieux, utilisant la ``méthode de la trace,'' permet 
d'obtenir en corollaire la 





\begin{proposition}[{\cite[IX 5.7]{sga4}}]\label{I:3-3-6}
Soient $k$ un corps algébriquement clos, $X$ une courbe algébrique sur $k$ 
et $\cF$ un faisceau de torsion sur $X$. Alors:
\begin{enumerate}[\indent i)]
  \item On a $\h^q(X,\cF) = 0$ pour $q>2$.
  \item Si $X$ est affine, on a même $\h^q(X,\cF) = 0$ pour $q>1$.
\end{enumerate}
\end{proposition}

Pour la démonstration, ainsi que pour l'exposé de la ``méthode de la 
trace,'' nous renvoyons à \cite[IX 5]{sga4}.










\subsection{Dévissages}\label{I:3-4}

Pour calculer la cohomologie des variétés de dimension $>1$ on emploie des 
fibrations par des courbes, ce qui permet de se ramener à étudier les 
morphismes dont les fibres sont de dimension $\leqslant 1$. Ce principe 
possède plusieurs variantes, indiquons-en quelques-unes.





\subsubsection{}\label{I:3-4-1}

Soient $A$ une $k$-algèbre de type fini et $a_1,\dotsc,a_n$ des générateurs de 
$A$. Si l'on pose $X_0=\spec(k)$, $X_i=\spec(k[a_1,\dotsc,a_i])$, 
$X_n=\spec(A)$, les inclusions canoniques 
$k[a_1,\dotsc,a_i]\to k[a_1,\dotsc,a_i,a_{i+1}]$ définissent des morphismes 
$X_n \to X_{n-1} \to \cdots \to X_1 \to X_0$ dont les fibres sont de dimension 
$\leqslant 1$. 





\subsubsection{}\label{I:3-4-2}

Dans le cas d'un morphisme lisse, on peut être plus précis. On appelle 
\emph{fibration élémentaire} un morphisme de schémas $f:X\to S$ qui peut 
être plongé dans un diagramme commutatif
\[\xymatrix{
  X \ar@{^{(}->}[r]^-j \ar[dr]_-f
    & \bar X \ar[d]^-{\bar f}
    & \ar@{_{(}->}[l]_-i \ar[dl]^-g Y \\
  & S
}\]
satisfaisant aux conditions suivantes:
\begin{enumerate}[\indent i)]
  \item $j$ est une immersion ouverte dense dans chaque fibre et 
    $X=\bar X \setminus Y$. 
  \item $\bar f$ est lisse et projectif, à fibres géométriques 
    irréductibles et de dimension $1$.
  \item $g$ est un revêtement étale et aucune fibre de $g$ n'est vide.
\end{enumerate}

On appelle \emph{bon voisinage} relatif à $S$ un $S$-schéma $X$ tel qu'il 
existe $S$-schémas $X=X_n,\dotsc,X_0=S$ et des fibrations élémentaires 
$f_i:X_i\to X_{i-1}$, $i=1,\dotsc,n$. On peut montrer \cite[XI 3.3]{sga4} que si 
$X$ est un schéma lisse sur un corps algébriquement clos $k$ tout point 
rationnel de $X$ possède un voisinage ouvert qui est un bon voisinage 
(relatif à $\spec(K)$). 





\subsubsection{}\label{I:3-4-3}

On peut dévisser un morphisme propre $f:X\to S$ de la façon suivante. 
D'après le lemme de Chow, il existe un diagramme commutatif
\[\xymatrix{
  X \ar[dr]_-f 
    & \ar[l]_-\pi \ar[d]^-{\bar f} \bar X \\
  & S
}\]
où $\pi$ et $\bar f$ sont des morphismes projectifs, $\pi$ étant de plus un 
isomorphisme au-dessus d'un ouvert dense de $X$. Localement sur $S$, $\bar X$ 
est un sous-schéma fermé d'un espace projectif type $\dP_S^n$. 

On dévisse ce dernier en considérant la projection 
$\varphi:\dP_S^n\to\dP_S^1$ qui envoie le point de coordonnées homogènes 
$(x_0,x_1,\dotsc,x_n)$ sur $(x_0,x_1)$. C'est une application rationnelle 
définie en dehors du fermé $Y\simeq \dP_S^{n-2}$ de $\dP_S^n$ d'équations 
homogènes $x_0=x_1=0$. Soit $u:P\to \dP_S^n$ l'éclatement à centre 
$Y$; les fibres de $u$ soit de dimension $\leqslant 1$. De plus il existe un 
morphisme naturel $v:P\to \dP_S^1$ qui prolonge l'application rationnelle 
$\varphi$ et $v$ fait de $P$ un $\dP_S^1$-schéma localement isomorphe à 
l'espace projectif type $\dP^{n-1}$ que l'on peut à son tour projeter sur un 
$\dP^1$, etc. 





\subsubsection{}\label{I:3-4-4}

On peut balayer une variété projective et lisse $X$ par un \emph{pinceau de 
Lefschetz}. L'éclaté $\widetilde X$ de l'intersection de l'axe du pinceau 
avec $X$ se projette sur $\dP^1$ et les fibres de cette projection sont les 
sections hyperplans de $X$ par les hyperplans du pinceau.




















\section{Théorème de changement de base pour un morphisme propre}\label{I:4}










\subsection{Introduction}\label{I:4-1}

Ce chapitre est consacré à la démonstration et aux applications du 





\begin{theorem}\label{I:4-1-1}
Soient $f:X\to S$ un morphisme propre de schémas et $\cF$ un faisceau abélien 
de torsion sur $X$. Alors, quel que soit $q\geqslant 0$, la fibre de 
$\R^q f_* \cF$ en un point géométrique $s$ de $S$ est isomorphe à la 
cohomologie $\h^q(X_s,\cF)$ de la fibre $X_s=X\otimes_S \spec k(s)$ de $f$ en 
$s$. 
\end{theorem}

Pour $f:X\to S$ une application continue propre et séparé (séparée 
signifie que la diagonale de $X\times_S X$ est fermée) entre espaces 
topologiques, et $\cF$ un faisceau abélien sur $X$, le résultat analogue est 
bien connu, et élémentaire: comme $f$ est fermée, les $f^{-1}(V)$ pour 
$V$ voisinage de $s$ forment un système fondamental de voisinages de $X_s$, et 
on vérifie que $\h^*(X_s,\cF) = \varinjlim_U \h^*(U,\cF)$, pour $U$ parcourant les 
voisinages de $X_s$. En pratique, $X_s$ a même un système fondamental 
$\sU$ de voisinages $U$ dont il est rétracte par déformation et, pour $\cF$ 
constant, on a donc $\h^*(X_s,\cF)=\h^*(U,\cF)$. En termes imagés: la fibre 
spéciale avale la fibre générale. 

Dans le cas des schémas la démonstration est plus délicate et il est 
indispensable de supposer que $\cF$ est de torsion (\cite[XII.2]{sga4}). Compte 
tenu de la description des fibres de $\R^q f_* \cF$ (\ref{I:2-3-3}), le théorème 
(\ref{I:4-1-1}) est essentiellement équivalent au 

 
  
   
    
\begin{theorem}\label{I:4-1-2}
Soient $A$ un anneau local strictement hensélien et $S=\spec(A)$. Soient 
$f:X\to S$ un morphisme propre et $X_0$ la fibre fermée de $f$. Alors, pour 
tout faisceau abélien de torsion $\cF$ sur $X$ et pour tout $q\geqslant 0$, on 
a $\h^q(X,\cF) \iso \h^q(X_0,\cF)$. 
\end{theorem} 

Par passage à la limite on voit qu'il suffit de démontrer le théorème 
lorsque $A$ est l'hensélisé strict d'une $\dZ$-algèbre  de type fini en 
un idéal premier. On traite d'abord le cas $q=0$ ou $1$ et $\cF=\dZ/n$ 
(\ref{I:4-2}). Un argument basé sur la notion de faisceau constructible 
(\ref{I:4-3}) montre d'ailleurs qu'il suffit de considérer le cas où $\cF$ est 
constant. D'autre part le dévissage (\ref{I:3-4-3}) permet de supposer que 
$X_0$ est une courbe; dans ce cas il ne reste plus qu'à démontrer le 
théorème pour $q=2$ (\ref{I:4-4}). 

Entre autres applications (\ref{I:4-6}), le théorème permet de définir la notion 
de cohomologie à support propre (\ref{I:4-5}). 










\subsection{Démonstration pour \texorpdfstring{$q=0$}{q=1} ou \texorpdfstring{$1$}{1} et \texorpdfstring{$\cF=\dZ/n$}{\cF=Z/n}}\label{I:4-2} 

Le résultat pour $q=0$ et $\cF$ constant est équivalent à la proposition 
suivante (théorème de connexion de Zariski):





\begin{proposition}\label{I:4-2-1}
Soient $A$ un anneau local hensélien noethérien et $S=\spec(A)$. 
Soient $f:X\to S$ un morphisme propre et $X_0$ la fibre fermée de $f$. Alors 
les ensembles de composantes connexes $\pi_0(X)$ et $\pi_0(X_0)$ sont en 
bijection.
\end{proposition}

Il revient au même de montrer que les ensembles de parties à la fois 
ouvertes et fermées $\operatorname{Of}(X)$ et $\operatorname{Of}(X_0)$ sont 
en bijection. On sait que l'ensemble $\operatorname{Of}(X)$ correspond 
bijectivement à l'ensemble des idempotents de $\Gamma(X,\cO_X)$, de même 
$\operatorname{Of}(X_0)$ correspond bijectivement à l'ensemble des 
idempotents de $\Gamma(X_0,\cO_{X_0})$. Il s'agit donc de montrer que 
l'application canonique 
\[
  \operatorname{Idem} \Gamma(X,\cO_X) \to \operatorname{Idem}\Gamma(X_0,\cO_{X_0})
\]
est bijective. 

On notera $\fm$ l'idéal maximal de $A$, $\Gamma(X,\cO_X)^\wedge$ le complété 
de $\Gamma(X,\cO_X)$ pour la topologie $\fm$-adique et, pour tout entier 
$n\geqslant 0$, $X_n=X\otimes_A A/\fm^{n+1}$. D'après le théorème de finitude 
pour les morphismes propres \cite[III.3.2]{ega}, $\Gamma(X,\cO_X)$ est une 
$A$-algèbre finie; comme $A$ est hensélien, il en résulte que 
l'application canonique 
\[
  \operatorname{Idem} \Gamma(X,\cO_X) \to \operatorname{Idem}\Gamma(X,\cO_X)^\wedge
\]
est bijective. En particulier l'application canonique 
\[
  \operatorname{Idem} \Gamma(X,\cO_X)^\wedge \to \varprojlim \operatorname{Idem}\Gamma(X_n,\cO_{X_n})
\]
est bijective. Mais, puisque $X_n$ et $X_0$ on même espace topologique 
sous-jacent, l'application canonique 
\[
  \operatorname{Idem}\Gamma(X_n,\cO_{X_n}) \to \operatorname{Idem} \Gamma(X_0,\cO_{X_0})
\]
est bijective pour tout $n$, ce qui achève la démonstration. 

Puisque $\h^1(X,\dZ/n)$ est en bijection avec l'ensemble des classes 
d'isomorphisme de revêtements étales galoisiens de $X$ de groupe $\dZ/n$, 
le théorème  pour $q=1$ et $\cF=\dZ/n$ résulte de la proposition suivante. 





\begin{proposition}\label{I:4-2-2}
Soient $A$ un anneau local hensélien noethérien et $S=\spec(A)$. Soient 
$f:X\to S$ un morphisme propre et $X_0$ la fibre fermée de $f$. Alors le 
foncteur de restriction 
\[
  \operatorname{Rev.et}(X) \to \operatorname{Rev.et}(X_0)
\]
est une équivalence de catégories. 
\end{proposition}

(Si $X_0$ est connexe et si l'on a choisi un point géométrique de $X_0$ 
comme point base, cela revient à dire que l'application canonique 
$\pi_1(X_0)\to \pi_1(X)$ sur les groupes fondamentaux (profinis) est 
bijective). 

La proposition (\ref{I:4-2-1}) montre que ce foncteur est pleinement fidèle. En 
effet, si $X'$ et $X''$ sont deux revêtements étales de $X$, un 
$X$-morphisme de $X'$ dans $X''$ est déterminé par son graphe qui est une 
partie ouverte et fermée de $X'\times_X X''$. 

Il s'agit donc de montrer que tout revêtement étale $X_0'$ de $X_0$ s'étend 
en un revêtement étale de $X$. On sait que les revêtements étales ne 
dépendent pas des éléments nilpotents \cite[ch.1]{sga1}, par conséquent 
$X_0'$ se relève de manière unique en un revêtement étale $X_n'$ de 
$X_n$ pour tout $n\geqslant 0$, autrement dit en un revêtement étale 
$\fX'$ du schéma formel $\fX$ complété de $X$ de long de $X_0$. D'après 
le théorème d'algébrisation des faisceaux cohérents formels de 
Grothendieck (théorème d'existence, \cite[III.5]{ega}), $\fX'$ est le 
complété formel d'un revêtement étale $\bar X'$, de 
$\bar X=X\otimes_A \hat A$. 

Par passage à la limite, il suffit de démontrer la proposition dans le cas 
où $A$ est l'hensélisé d'une $\dZ$-algèbre de type fini. On peut alors 
appliquer le théorème d'approximation d'Artin au foncteur 
$\cF:(\textnormal{$A$-algèbres}) \to (\textnormal{ensembles})$ qui, à une 
$A$-algèbre $B$, fait correspondre l'ensemble des classes d'isomorphisme de 
revêtements étales de $X\otimes_A B$. En effet ce foncteur est localement de 
présentation finie: si $B_i$ est une système inductif filtrant de 
$A$-algèbres et si $B=\varinjlim B_i$, on a $\cF(B) = \varinjlim \cF(B_i)$. 
D'après le théorème d'Artin, étant donné un élément 
$\xi\in \cF(\hat A)$, en l'occurrence la classe d'isomorphisme de $\bar X'$, il 
existe $\xi\in \cF(A)$ ayant même image que $\bar\xi$ dans $\cF(A/\fm)$. 
Autrement dit il existe un revêtement étale $X'$ de $X$ dont la restriction 
à $X_0$ est isomorphe à $X_0'$. 










\subsection{Faisceaux constructibles}\label{I:4-3}

Dans ce paragraphe, on considère un schéma \emph{noethérien} $X$ et on 
appelle faisceau sur $X$ un faisceau \emph{abélien} sur $\et X$. 





\begin{definition}\label{I:4-3-1}
On dit qu'un faisceau $\cF$ sur $X$ est \emph{localement constant constructible} 
(en abrégé l.c.c.) s'il est représenté par un revêtement étale de 
$X$. 
\end{definition}





\begin{definition}\label{I:4-3-2}
On dit qu'un faisceau $\cF$ sur $X$ est \emph{constructible} s'il vérifie les 
conditions équivalentes suivantes:
\begin{enumerate}[\indent (i)]
  \item Il existe une famille finie surjective de sous-schémas $X_i$ de $X$ 
     tels que la restriction de $\cF$ à $X_i$ soit l.c.c..
  \item Il existe une famille finie de morphismes finis $p_i:X_i'\to X$, pour 
    chaque $i$ un faisceau constant constructible ( = défini par un groupe 
    abélien fini) $C_i$ sur $X_i'$, et un monomorphisme 
    $\cF\to \prod p_{i*} C_i$. 
\end{enumerate}
\end{definition}

On vérifie facilement que la catégorie des faisceaux constructibles sur $X$ 
est une catégorie abélienne. De plus, si $u:\cF\to \cG$ est un homomorphisme de 
faisceaux et si $\cF$ est constructible, le faisceau $\im(u)$ est constructible.





\begin{lemma}\label{I:4-3-3}
Tout faisceau de torsion $\cF$ est limite inductive filtrante de faisceaux 
constructibles.
\end{lemma}

En effet, si $j:U\to X$ est un schéma étale de type fini sur $X$, un 
élément $\xi\in \cF(U)$ tel que $n\xi=0$ définit un homomorphisme de 
faisceaux $j_! \const{\dZ/n}\to \cF$ dont l'image (le lus petit sous-faisceau de 
$\cF$ dont $\xi$ soit section locale) est un sous-faisceau constructible de $\cF$. 
Il est clair que $\cF$ est limite inductive de tels sous-faisceaux.





\begin{definition}\label{I:4-3-4}
Soient $\sC$ une catégorie abélienne et $T$ un foncteur défini sur $\sC$ 
à valeurs dans la catégorie des groupes abéliens. On dira que $T$ est 
\emph{effaçable} dans $\sC$ si, pour tout objet $A$ de $\sC$ et tout 
$\alpha\in T(A)$, il existe un monomorphisme $u:A\to M$ dans $\sC$ tel que 
$T(u)\alpha = 0$. 
\end{definition}





\begin{lemma}\label{I:4-3-5}
Les foncteurs $\h^q(X,-)$ pour $q>0$ sont effaçables dans la catégorie des 
faisceaux constructibles sur $X$.
\end{lemma}

Il suffit de remarquer que, si $\cF$ est un faisceau constructible, il existe 
nécessairement un entier $n>0$ tel que $\cF$ soit un faisceau de 
$\dZ/n$-modules. Alors il existe un monomorphisme $\cF\hookrightarrow \cG$, où 
$\cG$ est un faisceau de $\dZ/n$-modules et $\h^q(X,\cG) = 0$ pour tout $q>0$.  
On peut par exemple prendre pour $\cG$ la résolution de Godement 
$\prod_{x\in X} i_{x*} \cF_{\bar x}$, où $x$ parcourt les points de $X$ et 
$i_x:\bar x\to X$ est un point géométrique centré en $X$. D'après 
(\ref{I:4-3-3}) $\cG$ est limite inductive de faisceaux constructibles, d'o`u le 
lemme, car les foncteurs $\h^q(X,-)$ commutent aux limites inductives. 





\begin{lemma}\label{I:4-3-6}
Soit $\varphi^\bullet:T^\bullet\to {T'}^\bullet$ un morphisme de foncteurs 
cohomologiques définis sur une catégorie abélienne $\sC$ et à valeurs 
dans la catégorie des groupes abéliens. Supposons que $T^q$ est effaçable 
pour $q>0$ et soit $\sE$ un sous-ensemble d'objets de $\sC$ tel que tout objet 
de $\sC$ soit contenu dans un objet appartenant à $\sE$. Alors les 
conditions suivantes sont équivalentes:
\begin{enumerate}[\indent (i)]
  \item $\varphi^q(A)$ est bijectif pour tout $q\geqslant 0$ et tout 
    $A\in\ob \sC$.
  \item $\varphi^0(M)$ est bijectif et $\varphi^q(M)$ surjectif pour tout 
    $q>0$ et tout $M\in\sE$. 
  \item $\varphi^0(A)$ est bijectif pour tout $A\in\ob \sC$ et ${T'}^q$ est 
    effaçable pour tout $q>0$. 
\end{enumerate}
\end{lemma}

La démonstration se fait par récurrence sur $q$ et ne présente pas de 
difficultés. 





\begin{proposition}\label{I:4-3-7}
Soit $X_0$ un sous-schéma de $X$. Supposons que, pour tout $n\geqslant 0$ et 
pour tout schéma $X'$ fini sur $X$, l'application canonique 
\[
  \h^q(X',\dZ/n) \to \h^q(X_0',\dZ/n)\text{,}
\]
où $X_0' = X'\times_X X_0$, est bijective pour $q=0$ et surjective pour 
$q>0$. Alors, pour tout faisceau de torsion $\cF$ sur $X$ et pour tout 
$q\geqslant 0$, l'application canonique 
\[
  \h^q(X,\cF) \to \h^q(X_0,\cF)
\]
est bijective.
\end{proposition}

Par passage à la limite, il suffit de démontrer l'assertion pour $\cF$ 
constructible. On applique le lemme (\ref{I:4-3-6}) en prenant pour $\sC$ la 
catégorie des faisceaux constructibles sur $X$, $T^q=\h^q(X,-)$, 
${T'}^q=\h^q(X_0,-)$, et $\sE$ l'ensemble des faisceaux constructibles de la 
forme $\prod p_{i*} C_i$, où $p_i:X_i'\to X$ est un morphisme fini et $C_i$ 
un faisceau constant fini sur $X_i'$. 










\subsection{Fin de la démonstration}\label{I:4-4}

Par la méthode de fibration par des courbes (\ref{I:3-4-3}), on se ramène 
à démontrer le théorème en dimension relative $\leqslant 1$. D'après 
le paragraphe précédent, il suffira de montrer que, si $S$ est le spectre 
d'un anneau local noethérien strictement hensélien, $f:X\to S$ un 
morphisme propre dont la fibre fermée $X_0$ est de dimension $\leqslant 1$ 
et $n$ un entier $\geqslant 0$, l'homomorphisme canonique 
\[
  \h^q(X,\dZ/n) \to \h^q(X_0,\dZ/n)
\]
est bijectif pour $q=0$ et surjectif pour $q>0$. 

Les cas $q=0$ et $1$ ont été vus plus haut et on a $\h^q(X_0,\dZ/n) = 0$ 
pour $q\geqslant 3$; il suffit donc de traiter le cas $q=2$. On peut 
évidemment supposer que $n$ est une puissance d'un nombre premier. Si 
$n=p^r$, où $p$ est la caractéristique du corps résiduel de $S$, la 
théorie d'Artin-Schreier montre qu'un a $\h^2(X_0,\dZ/p^r) = 0$. Si 
$n=\ell^r$, $\ell\ne p$, on déduit de la théorie de Kummer un diagramme 
commutatif 
\[\xymatrix{
  \pic(X) \ar[d] \ar[r]^-\alpha 
    & \h^2(X,\dZ/\ell^r) \ar[d] \\
  \pic(X_0) \ar[r]^-\beta 
    & \h^2(X_0,\dZ/\ell^r)
}\]
où l'application $\beta$ est surjective (On l'a vu au chapitre \ref{I:3} pour 
une courbe lisse sur un corps algébriquement clos, mais des arguments 
similaires s'appliquent à n'importe quelle courbe sur un corps 
séparablement clos). 

Pour conclure, il suffira donc de montrer:





\begin{proposition}\label{I:4-4-1}
Soient $S$ le spectre d'un anneau local noethérien hensélien et $f:X\to S$ 
un morphisme propre dont la fibre fermée $X_0$ est de dimension 
$\leqslant 1$. Alors l'application canonique de restriction 
\[
  \pic(X)\to \pic(X_0)
\]
est surjective (Il suffit d'ailleurs que le morphisme $f$ soit 
\emph{séparé} de type fini). 
\end{proposition}

Pour simplifier la démonstration, nous supposerons que $X$ est intègre, 
bien que cela ne soit pas nécessaire. Tout faisceau inversible sur $X_0$ est 
associé à un diviseur de Cartier (car $X_0$ est une courbe, donc 
quasi-projectif), il suffit donc de montrer que l'application canonique 
$\Div(X)\to\Div(X_0)$ est surjective. 

Tout diviseur sur $X_0$ est combinaison linéaire de diviseurs dont le 
support est concentré en un seul point fermé non isolé de $X_0$. Soient 
$x$ un tel point, $t_0\in\cO_{X_0,x}$ un élément régulier non inversible 
de $\cO_{X_0,x}$ et $D_0$ le diviseur concentré en $x$ d'équation locale 
$t_0$. Soit $U$ un voisinage ouvert de $x$ dans $X$ tel qu'il existe un section 
$t\in \Gamma(U,\cO_U)$ relevant $t_0$. Soit $Y$ le fermé de $U$ d'équation 
$t=0$; quitte à prendre $U$ assez petit, on peut supposer que $x$ est le 
seul point de $Y\cap X_0$. Alors $Y$ est quasi-fini au-dessus de $S$ en $x$; 
puisque $S$ est le spectre d'un anneau local hensélien, on en déduit que 
$Y=Y_1\amalg Y_2$, où $Y_1$ est fini sur $S$ et où $Y_2$ ne rencontre pas 
$X_0$. De plus, comme $X$ est séparé sur $S$, $Y_1$ est fermé dans $X$. 

Quitte à remplacer $U$ par un voisinage ouvert plus petit de $x$, on peut 
supposer que $Y=Y_1$, autrement dit que $Y$ est fermé dans $X$. On définit 
alors un diviseur $D$ sur $X$ relevant $D_0$ en posant $D|X\setminus Y=0$ et 
$D|U=\dv(t)$ ce qui a un sens car $t$ est inversible sur $U\setminus Y$. 





\subsubsection{Remarque}\label{I:4-4-2}

Dans le cas où $f$ est propre, on pourrait aussi faire une démonstration 
du même style que celle de la proposition (\ref{I:4-2-2}). En effet, comme 
$X_0$ est une courbe, il n'y a pas d'obstruction à relever un faisceau 
inversible sur $X_0$ aux voisinages infinitésimaux $X_n$ de $X_0$, donc au 
complété formel $\fX$ de $X$ le long de $X_0$. On conclut alors en 
appliquant successivement le théorème d'existence de Grothendieck et le 
théorème d;approximation d'Artin. 










\subsection{Cohomologie à support propre}\label{I:4-5}





\begin{definition}\label{I:4-5-1}
Soit $X$ un schéma séparé de type fini sur un corps $k$. D'après un 
théorème de Nagata, il existe un schéma $\bar X$ propre sur $k$ et une 
immersion ouverte $j:X\to\bar X$. Pour tout faisceau de torsion $\cF$ sur $X$ on 
note $j_! \cF$ le prolongement par $0$ de $\cF$ à $\bar X$ et on définit les 
groupes de \emph{cohomologie à support propre} 
\[
  \h_c^q(X,\cF)=\h^q(\bar X, j_! \cF) \text{.}
\]
\end{definition}

Montrons que cette définition est indépendante de la compactification 
$j:X\to\bar X$ choisie. Soient $j_1:X\to \bar X_1$ et $j_2:X\to \bar X_2$ deux 
compactifications. Alors $X$ s'envoie dans $\bar X_1\times \bar X_2$ par 
$x\mapsto (j_1(x),j_2(x))$ et l'image fermée $\bar X_3$ de $X$ par cette 
application est une compactification de $X$. On a ainsi un diagramme commutatif 
\[\xymatrix{
  & \bar X_1 \\
  X \ar[ur]^-{j_1} \ar[rr]^-{(j_1,j_2)} \ar[dr]_-{j_2} 
    & & \bar X_3 \ar[ul]_-{p_1} \ar[dl]^-{p_2} \\
  & \bar X_2
}\]
où $p_1$ et $p_2$, les restrictions des projections naturelles à 
$\bar X_3$, sont des morphismes propres. Il suffit donc de traiter le cas où 
on a un diagramme commutatif 
\[\xymatrix{
  X \ar[r]^-{j_2} \ar[dr]_-{j_1} 
    & \bar X_2 \ar[d]^-p \\
  & \bar X_1
}\]
avec $p$ un morphisme propre. 





\begin{lemma}\label{I:4-5-2}
On a $p_*(j_{2!} \cF) = j_{1!} \cF$ et $\R^q p_*(j_{2!} \cF) = 0$, pour $q>0$. 
\end{lemma}

Notons tout de suite que lemme suffit pour conclure. En utilisant la suite 
spectrale de Leray du morphisme $p$, on en déduit qu'on a, pour tout 
$q\geqslant 0$, 
\[
  \h^q(\bar X_2, j_{2!} \cF) = \h^q(\bar X_1, j_{1!} \cF) \text{.}
\]

Pour démontrer le lemme, on raisonne fibre par fibre en utilisant le théorème 
de changement de base (\ref{I:4-1-1}) pour $p$. Le résultat est immédiat, car, 
au-dessus d'un point de $X$, $p$ est un isomorphe et, au-dessus d'un point de 
$\bar X_1\setminus X$, $j_{2!} \cF$ est nul sur la fibre de $p$. 





\subsubsection{}\label{I:4-5-3}

De même, si $f:X\to S$ est un morphisme séparé de type fini de schémas 
noethériens, il existe un morphisme propre $\bar f:\bar X\to S$ et un 
immersion ouverte $j:X\to \bar X$. On définit alors les images directes s
supérieures à support propre $\R^q f_!$ en posant pour tout faisceau de 
torsion $\cF$ sur $X$
\[
  \R^q f_! \cF = \R^q f_*(j_! \cF) \text{.}
\]
On vérifie comme précédemment que cette définition est indépendante de 
la compactification choisie. 





\begin{theorem}\label{I:4-5-4}
Soient $f:X\to S$ un morphisme séparé de type fini de schémas noethériens 
et $\cF$ un faisceau de torsion sur $X$. Alors la fibre de $\R^q f_! \cF$ en un 
point géométrique $s$ de $S$ est isomorphe à la cohomologie à support 
propre $\h_c^q(X_s,\cF)$ de la fibre $X_s$ de $f$ en $s$. 
\end{theorem}

C'est une simple variante du théorème de changement de base pour un morphisme 
propre (\ref{I:4-1-1}). Plus généralement, si 
\[\xymatrix{
  X \ar[d]^-f
    & X' \ar[l]_-{g'} \ar[d]^-{f'} \\
  S 
    & S' \ar[l]^-g
}\]
est un diagramme cartésien, on a un isomorphe canonique 
\begin{equation}\label{I:eq:4-5-4-1}
  g^* (\R^q f_! \cF) \simeq \R^q f_!'({g'}^* \cF) \text{.}
\end{equation}










\subsection{Applications}\label{I:4-6}





\begin{theorem}[d'annulation]\label{I:4-6-1}
Soient $f:X\to S$ un morphisme séparé de type fini dont les fibres sont de 
dimension $\leqslant n$ et $\cF$ un faisceau de torsion sur $X$. Alors on a 
$\R^q f_! \cF = 0$ pour $q>2 n$. 
\end{theorem}

D'après le théorème de changement de base, on peut supposer que $S$ est 
le spectre d'un corps séparablement clos. Si $\dim X = n$, il existe un 
ouvert affine $U$ de $X$ tel que $\dim(X\setminus U) < n$; on a alors un suite 
exacte $0\to \cF_U \to \cF \to \cF_{X\setminus U} \to 0$ et, par récurrence sur $n$, 
il suffit de démontrer le théorème pour $X=U$ affine. Puis la méthode de 
fibration par des courbes (\ref{I:3-4-1}) et le théorème de changement de 
base permettent de se ramener à une courbe sur un corps séparablement clos 
pour laquelle on déduit le résultat voulu du théorème de Tsen 
(\ref{I:3-3-6}). 





\begin{theorem}[de finitude]\label{I:4-6-2}
Soient $f:X\to S$ un morphisme séparé de type fini et $\cF$ un faisceau 
constructible sur $X$. Alors les faisceaux $\R^q f_! \cF$ sont constructibles. 
\end{theorem}

Nous ne considérons que le cas où $\cF$ est annulé par un entier inversible 
sur $X$. 

Par démontrer le théorème on se ramène au cas o`u $\cF$ est un faisceau 
constant $\const{\dZ/n}$ et où $f:X\to S$ est un morphisme propre et lisse 
dont les fibres sont des courbes géométriquement connexes de genre $g$. 
Pour $n$ inversible sur $X$, les faisceaux $\R^q f_* \cF$ sont alors localement 
libres de rang fini, nuls pour $q>2$ (\ref{I:4-6-1}). Remplaçant $\dZ/n$ par le 
faisceau localement isomorphe (sur $S$) $\dmu_n$, on a canoniquement 
\begin{equation}\label{I:eq:4-6-2-1}
\begin{aligned}
  \R^0 f_* \dmu_n &= \dmu_n \\
  \R^1 f_* \dmu_n &= \const\pic(X/S)_n \\
  \R^2 f_* \dmu_n &= \dZ/n \text{.}
\end{aligned}
\end{equation}





\begin{theorem}[comparaison avec la cohomologie classique]\label{I:4-6-3}
Soient $f:X\to S$ un morphisme séparé de schémas de type fini sur $\dC$, 
et $\cF$ un faisceau de torsion sur $X$. Notons par un exposant $\an{(-)}$ le 
foncteur de passage aux espaces topologiques usuels, et par $\R^q \an f_!$ les 
foncteurs dérivés du foncteur image directe à support propre par $\an f$. On a 
\[
  \an{(\R^q f_! \cF)} \simeq \R^q \an f_! \an \cF \text{.}
\]
En particulier, pour $S=$ un point et $\cF$ le faisceau constant $\dZ/n$, 
\[
  \h_c^q(X,\dZ/n) \simeq \h_c^q(\an X,\dZ/n)\text{.}
\]
\end{theorem}

Des dévissages utilisant le théorème de changement de bas nous ramènent au 
cas où $X$ est une courbe propre et lisse, où $S=$ un point, et où 
$\cF=\dZ/n$. Les groupes de cohomologie considérés sont alors nuls pour 
$q\ne 0,1,2$, et on invoque \cite{se55}: en effet, si $X$ est propre sur $\dC$, 
on a $\pi_0(X) = \pi_0(\an X)$ et $\pi_1(X) = $ complété profini de 
$\pi_1(\an X)$, d'où l'assertion pour $q=0,1$. Pour $q=2$, on utilise la 
suite exacte de Kummer et le fait que, par \cite{se55} encore, 
$\pic(X)=\pic(\an X)$. 





\begin{theorem}[dimension cohomologique des schémas affines]\label{I:4-6-4}
Soient $X$ un schéma affine de type fini sur un corps séparablement clos et 
$\cF$ un faisceau de torsion sur $X$. Alors on a $\h^1(X,\cF) = 0$ pour 
$q>\dim(X)$. 
\end{theorem}

Pour la très jolie démonstration nous revoyons à \cite{sga4}, XIV \S 2 et 3. 





\subsubsection{Remarque}\label{I:4-6-5}

Ce théorème est en quelque sorte un substitut pour la théorie de Morse. 
Considérons en effet le cas classique où $X$ est lisse et affine sur $\dC$ 
plongé dans un espace affine type $\dC^N$. Alors, pour presque tout point 
$p\in\dC^N$, la fonction ``distance à $p$'' sur $X$ est une fonction de Morse 
et les indices de ses points critiques sont plus petits que $\dim(X)$. Ainsi 
$X$ est obtenu par recollement d'anses d'indice plus petit que $\dim(X)$, d'où 
l'analogue classique de (\ref{I:4-6-4}). 




















\section{Acyclicité locale des morphismes lisses}\label{I:5}

Soient $X$ une variété analytique complexe et $f:X\to D$ un morphisme de 
$X$ dans le disque. On note $[0,t]$ le segment de droit fermé d'extrémités 
$0$ et $t$ dans $D$ et $(0,t]$ le segment semi-ouvert. Si $f$ est \emph{lisse}, 
l'inclusion 
\[
  j : f^{-1}\left((0,t]\right) \hookrightarrow f^{-1}\left([0,t]\right)
\]
est une équivalence d'homotopie; en peut pousser la fibre spécial 
$X_0=f^{-1}(0)$ dans $f^{-1}([0,t])$. 

En pratique, pour $t$ assez petit, $f^{-1}((0,t])$ sera un fibré sur 
$(0,t]$ de sorte que l'inclusion 
\[
  X_t = f^{-1}(t) \hookrightarrow f^{-1}\left((0,t]\right)
\]
sera également une équivalence d'homotopie. On appelle alors \emph{morphisme 
de cospécialisation} la classe d'homotopie d'applications: 
\[
  \cosp : X_0 \hookrightarrow f^{-1}\left([0,t]\right) \xleftarrow\sim f^{-1}\left((0,t]\right) \xleftarrow\sim X_t \text{.}
\]
On peut exprimer cette construction en termes imagés en disant que, pour un 
morphisme lisse, la fibre générale avale la fibre spéciale. 

Ne supposons plus $f$ nécessairement lisse (mais supposons que $f^{-1}((0,t])$ 
soit un fibré sur $(0,t]$). On peut encore définir un morphisme 
$\cosp^\bullet$ en cohomologie dès que $j_* \dZ=\dZ$ et 
$\R^q j_* \dZ = 0$ pour $q>0$. Sous ces hypothèses, la suite spectrale de 
Leray pour $j$ montre qu'on a 
\[
  \h^\bullet\left(f^{-1}\left([0,t]\right),\dZ\right) \iso \h^\bullet\left(f^{-1}\left((0,t]\right),\dZ\right)
\]
et $\cosp^\bullet$ est le morphisme composé: 
\[
  \cosp^\bullet : \h^\bullet(X_t,\dZ)
    \xleftarrow\sim \h^\bullet\left(f^{-1}\left((0,t]\right),\dZ\right) 
    \xleftarrow\sim \h^\bullet\left(f^{-1}\left([0,t]\right),\dZ\right) 
    \to \h^\bullet(X_0,\dZ)
\]
La fibre de $\R^q j_* \dZ$ en un point $x\in X_0$ se calcule comme suit. On 
prend dans un espace ambiant une boule $B_\varepsilon$de centre $x$ et de rayon 
$\varepsilon$ assez petit, et pour $\eta$ assez petit, on pose 
$E=X\cap B_\varepsilon \cap f^{-1}(\eta t)$; c'est la \emph{variété des 
cycles évanescents} en $x$. On a 
\[
  (\R^q j_* \dZ)_x \xleftarrow\sim \h^q\left(X\cap B_\varepsilon\cap f^{-1}\left((0,\eta t]\right),\dZ\right) \xrightarrow\sim \h^q(E,\dZ)
\]
et le morphisme de cospécialisation est défini en cohomologie dés que les 
variétés de cycles évanescents sont acycliques ($\h^0(E,\dZ)=\dZ$ et 
$\h^q(E,\dZ) = 0$ pour $q>0$), ce qui s'exprime en disant que $f$ est 
\emph{localement acyclique}. 

Ce chapitre est consacré à l'analogue de cette situation pour un morphisme 
lisse de schémas pour la cohomologie étale. Cependant il est indispensable 
dans ce cadre de se limiter aux coefficients de torsion et d'ordre \emph{premier aux caractéristiques résiduelles}. La paragraphe \ref{I:5-1} est consacré 
à des généralités sur les morphismes localement acycliques et les 
flèches de cospécialisation. Dans le paragraphe \ref{I:5-2}, on démontre 
qu'un morphisme lisse est localement acyclique. Dans le paragraphe \ref{I:5-3}, on 
joint ce résultat à ceux du chapitre précédent pour en déduire deux 
applications: un théorème de spécialisation des groupes de cohomologie (la 
cohomologie des fibres géométriques d'un morphisme propre et lisse est 
localement constant) et un théorème de changement de base par un morphisme 
lisse. 

Dans tout ce qui suit, on fixe un entier $n$ et ``schéma'' signifie ``schéma 
sur lequel $n$ est inversible.'' ``Point géométrique'' signifiera toujours 
``point géométrique algébrique'' (\ref{I:2-3-1}) $x:\spec(k)\to X$, avec $k$ 
algébriquement clos. 










\subsection{Morphismes localement acycliques}\label{I:5-1}





\subsubsection{Notation}\label{I:5-1-1}

Étant donnés un schéma $S$ et un point géométrique $s$ de $S$, on 
notera $\widetilde S^s$ le spectre du localisé stricte de $S$ en $s$. 





\begin{definition}\label{I:5-1-2}
On dit qu'un point géométrique $t$ de $S$ est une \emph{générisation} de 
$s$ s'il est défini par une clôture algébrique du corps résiduel d'un 
point de $\widetilde S^s$. On dit aussi que $s$ est une \emph{spécialisation} 
de $t$ et on appelle \emph{flèche de spécialisation} le $S$-morphisme 
$t\to \widetilde S^s$. 
\end{definition}





\begin{definition}\label{I:5-1-3}
Soit $f:X\to S$ un morphisme de schémas. Soient $s$ un point géométrique de 
$S$, $t$ une générisation de $s$, $x$ un point géométrique de $X$ 
su-dessus de $s$ et 
$\widetilde X_t^x = \widetilde X^x\times_{\widetilde S^s} t$. Alors on dit que 
$\widetilde X_t^x$ est une \emph{variété de cycles évanescents} de $f$ au 
point $x$. 

On dit que $f$ est \emph{localement acyclique} si, la cohomologie réduite de 
toute variété de cycles évanescents $\widetilde X_t^x$ est nulle:
\begin{equation}\label{I:eq:5-1-3-1}
  \hat\h^\bullet(\widetilde X_t^x,\dZ/n) = 0\text{,}
\end{equation}
i.e. $\h^0\left(X_t^x,\dZ/n\right) = \dZ/n$ et 
$\h^q(\widetilde X_t^x,\dZ/n) = 0$ pour $q>0$. 
\end{definition}





\begin{lemma}\label{I:5-1-4}
Soient $f:X\to S$ un morphisme localement acyclique et $g:S'\to S$ un morphisme 
quasi-fini (ou limite projective de morphismes quasi-finis). Alors le 
morphisme $f':X'\to S'$ déduit de $f$ par changement de base est localement 
acyclique.
\end{lemma}

On vérifie en effet que tout variété de cycles évanescents de $f'$ est 
une variété de cycles évanescents de $f$.  	





\begin{lemma}\label{I:5-1-5}
Soit $f:X\to S$ un morphisme localement acyclique. Pour tout point géométrique 
$t$ de $S$, donnant lieu à un diagramme cartésien 
\[\xymatrix{
  X_t \ar[r]^-{\varepsilon'} \ar[d] 
    & X \ar[d]^-f \\
  t \ar[r]^-\varepsilon 
    & S
}\]
On a $\varepsilon_*' \dZ/n = f^*\varepsilon_* \dZ/n$ et $\R^q \varepsilon_*'\dZ/n=0$ 
pour $q>0$. 
\end{lemma}

Soient $\bar S$ l'adhérence de $\varepsilon(t)$, $S'$ le normalisé de $\bar S$ 
dans $k(t)$, et le diagramme cartésien 
\[\xymatrix{
  X_t \ar[r]^-{f'} \ar[d] 
    & X' \ar[r]^-{\alpha'} \ar[d]^-{f'} 
    & X \ar[d]^-f \\
  t \ar[r]^-i 
    & S' \ar[r]^-\alpha 
    & S
}\]
Les anneaux locaux de $S'$ sont normaux à corps de fractions séparablement 
clos. Ils sont donc strictement henséliens, et l'acyclicité locale de $f'$ 
(\ref{I:5-1-4}) fournit $f_*'\dZ/n = \dZ/n$, $\R^qi_*'\dZ/n = 0$ pour $q>0$. 
Puisque $\alpha$ est entier, on a alors 
\[
  \R^q\varepsilon_*'\dZ/n 
    = \alpha_*' \R^q i_*' \dZ/n 
    = \alpha_*' {f'}^* \R^q i_* \dZ/n 
    = f^* \alpha_* \R^q i_* \dZ/n 
    = f^* \R^q \varepsilon_* \dZ/n \text{,}
\]
et le lemme. 





\subsubsection{}\label{I:5-1-6}

Etant donnés un morphisme localement acyclique $f:X\to S$ et une flèche de 
spécialisation $t\to \widetilde S^s$, nous allons définir des 
homomorphismes canoniques, dits flèches de cospécialisation 
\[
  \cosp^\bullet : \h^\bullet(X_t,\dZ/n) \to \h^\bullet(X_s,\dZ/n)\text{,}
\]
reliant la cohomologie de la fibre générale $X_t=X\times_S t$ et celle de la 
fibre spéciale $X_s=X\times_X s$. 

Considérons le diagramme cartésien 
\[\xymatrix{
  X_t \ar[d] \ar[r]^-{\varepsilon'} 
    & \widetilde X \ar[d]^-{f'} 
    & \ar[l] X_s \ar[d] \\
  t \ar[r]^-\varepsilon 
    & \widetilde S^s 
    & \ar[l] s
}\]
déduit de $f$ par changement de base. D'après (\ref{I:5-1-4}), $f'$ est encore 
localement acyclique. De la définition de l'acyclicité locale, on tire 
aussitôt que la restriction à $X_s$ des faisceaux $\R^q \varepsilon_*' \dZ/n$ 
est $\dZ/n$ pour $q=0$, et $0$ pour $q>0$. Par (\ref{I:5-1-5}) on sait même que 
$\R^q \varepsilon_*'\dZ/n = 0$ pour $q>0$. On définit $\cosp^\bullet$ comme la 
flèche composée 
\begin{equation}\label{I:eq:5-1-6-1}
  \h^\bullet(X_t,\dZ/n) \simeq \h^\bullet(X,\varepsilon_*' \dZ/n) \to \h^\bullet(X_s,\dZ/n) \text{.}
\end{equation}

\paragraph{Variante}
Soient $\bar S$ l'adhérence de $\varepsilon(t)$ dans $\widetilde S^s$, $S'$ le 
normalisé de $\bar S$ dans $k(t)$ et $X'/S'$ déduit de $X/S$ par changement 
de base. Le diagramme \eqref{I:eq:5-1-6-1} peut encore s'écrire 
\[
  \h^\bullet(X_t,\dZ/n) \simeq \h^\bullet(X',\dZ/n) \to \h^\bullet(X_s,\dZ/n) \text{.}
\]





\begin{theorem}\label{I:5-1-7}
Soient $S$ un schéma localement noethérien, $s$ un point géométrique de 
$S$ et $f:X\to S$ un morphisme. On suppose 
\begin{enumerate}[\indent a)]
  \item le morphisme $f$ est localement acyclique, 
  \item pour tout morphisme de spécialisation $t\to \widetilde S^s$ et pour 
    tout $q\geqslant 0$, les flèches de cospécialisation 
    $\h^q(X_t,\dZ/n) \to \h^q(X_s,\dZ/n)$ sont bijectives.
\end{enumerate}

Alors l'homomorphisme canonique $(\R^q f_*\dZ/n)_s\to \h^q(X_s,\dZ/n)$ est 
bijectif pour tout $q\geqslant 0$. 
\end{theorem}

Pour démontrer le théorème, il est clair qu'on peut supposer 
$S=\widetilde S^s$. On va et fait montrer que, pour tout faisceau de 
$\dZ/n\dZ$-modules $\cF$ sur $S$, l'homomorphisme canonique 
$\varphi^q(\cF):(\R^q f_* f^* \cF)_s \to \h^q(X_s,f^*\cF)$ est bijectif. 

Tout faisceau de $\dZ/n\dZ$ est limite inductive filtrante de faisceaux 
constructibles de $\dZ/n\dZ$-modules (\ref{I:4-3-3}). De plus tout faisceau 
constructible de $\dZ/n\dZ$-modules se plonge dans un faisceau de la forme 
$\prod i_{\lambda *} C_\lambda$, où $i_\lambda:t_\lambda\to S$ est une 
famille finie de générisations de $s$ et $C_\lambda$ un $\dZ/n\dZ$-module 
libre de rang fini sur $t_\lambda$. D''après la définition des flèches de 
cospécialisation, la condition b) signifie que les homomorphismes 
$\varphi^q(\cF)$ sont bijectifs si $\cF$ est de cette forme. 

On conclut à l'aide d'une variant du lemme (\ref{I:4-3-6}):





\begin{lemma}\label{I:5-1-8}
Soit $\sC$ un catégorie abélienne dans laquelle les limites inductives 
filtrantes existent. Soit $\varphi^\bullet:T^\bullet\to{T'}^\bullet$ un 
morphisme de foncteurs cohomologiques commutant aux limites inductives 
filtrantes, définis sur $\sC$ et à valeurs dans la catégorie des groupes 
abéliens. Supposons qu'il existe deux sous-ensembles $\sD$ et $\sE$ d'objets 
de $\sC$ tels que:
\begin{enumerate}[\indent a)]
  \item tout objet de $\sC$ est limite inductive filtrante d'objets appartenant 
    à $\sD$, 
  \item tout objet appartenant à $\sD$ est contenu dans un objet appartenant 
    à $\sE$.
\end{enumerate}

Alors les conditions suivantes sont équivalentes: 
\begin{enumerate}[\indent (i)]
  \item $\varphi^q(A)$ est bijectif pour tout $q\geqslant 0$ et tout 
    $A\in\ob\sC$.
  \item $\varphi^q(M)$ est bijectif pour tout $q\geqslant 0$ et tout $M\in\sE$. 
\end{enumerate}
\end{lemma}

La démonstration du lemme se fait par passage à la limite inductive, 
récurrence sur $q$ et application répétée du lemme des cinq au diagramme 
de suites exactes de cohomologie déduit d'une suite exacte 
$0\to A\to M\to A'\to 0$, avec $A\in\sD$, $M\in\sE$, $A'\in \ob\sC$. 





\begin{corollary}\label{I:5-1-9}
Soient $S$ le spectre d'un anneau local noethérien strictement hensélien et 
$f:X\to S$ un morphisme localement acyclique. Supposons que, pour tout point 
géométrique $t$ de $S$ on dit $\h^0(X_t,\dZ/n) = \dZ/n$ et 
$\h^q(X_t,\dZ/n) = 0$ pour $q>0$ (autrement dit les fibres géométriques de 
$f$ sont acycliques). Alors on a $f_* \dZ/n\dZ/n$ et $\R^qf_*\dZ/n = 0$ pour 
$q>0$. 
\end{corollary}





\begin{corollary}\label{I:5-1-10}
Soient $f:X\to Y$ et $g:Y\to X$ des morphismes de schémas localement 
noethériens. Alors, si $f$ et $g$ sont localement acycliques, il en est de 
même de $g\circ f$. 
\end{corollary}

On peut supposer que $X$, $Y$ et $Z$ sont strictement locaux et que $f$ et $g$ 
sont des morphismes locaux. Il s'agit alors de montrer que, si $z$ est un point 
géométrique algébrique de $Z$, on a $\h^0(X_z,\dZ/n) = \dZ/n$ et 
$\h^q(X_z,\dZ/n) = 0$ pour $q>0$. 

Puisque $g$ est localement acyclique, on a $\h^0(Y_z,\dZ/n) = \dZ/n$, et 
$\h^q(Y_z,\dZ/n) = 0$ pour $q>0$. Par ailleurs le morphisme $f_z:X_z\to Y_z$ est 
localement acyclique (\ref{I:5-1-4}) et ses fibres géométriques sont 
acycliques, car ce sont des variétés de cycles évanescents de $f$. 
D'après (\ref{I:5-1-9}), on a donc $\R^q {f_z}_*\dZ/n = 0$ pour $q>0$. De plus 
${f_z}_*\dZ/n$ est constant de fibre $\dZ/n$ sur $Y_z$. On conclut à l'aide de 
la suite spectrale de Leray de $f_z$. 










\subsection{Acyclicité locale d'un morphisme lisse}\label{I:5-2}





\begin{theorem}\label{I:5-2-1}
Un morphisme lisse est localement acyclique.
\end{theorem}

Soit $f:X\to S$ un morphisme lisse. L'assertion est locale pour la topologie 
étale sur $X$ et $S$, on peut donc supposer que $X$ est l'espace affine type 
de dimension $d$ sur $S$. Par passage à la limite, on peut supposer que $S$ 
est noethérien et la transitivité de l'acyclicité locale (\ref{I:5-1-10}) montre 
qu'il suffit de traiter le cas $d=1$. 

Soient $s$ un point géométrique de $S$ et $x$ un point géométrique de 
$X$ centré en un point fermé de $X_s$. Il s'agit de montrer que les fibres 
géométriques du morphisme $\widetilde X^x\to \widetilde S^s$ sont 
acycliques. On posera désormais $S=\widetilde S^s=\spec(A)$ et 
$X=\widetilde X^x$. On a $X\simeq \spec A\{T\}$, où $A\{T\}$ est l'hensélise 
de $A[T]$ au point $T=0$ au-dessus de $s$. 

Si $t$ est un point géométrique de $S$, la fibre $X_t$ est limite 
projective de courbes affines et lisses sur $t$. On a donc $\h^q(X_t,\dZ/n) = 0$ 
pour $q\geqslant 0$ et il suffit de montrer que $\h^0(X_t,\dZ/n) = \dZ/n$ et 
$\h^1(X_t,\dZ/n) = 0$ pour $n$ premier à la caractéristique résiduelle de 
$S$. Cela résulte des deux propositions suivantes. 





\begin{proposition}\label{I:5-2-2}
Soient $A$ un anneau local strictement hensélien, $S=\spec(A)$ et 
$X=\spec A\{T\}$. Alors les fibres géométrique de $X\to S$ sont connexes. 
\end{proposition}

On peut se ramener par passage à limite au cas où $A$ est un hensélisé 
strict d'une $\dZ$-algèbre de type fini. 

Soient $\bar t$ un point géométrique de $S$, localisé en $t$, et $k'$ une 
extension finie séparable de $k(t)$ dans $k(\bar t)$. On pose $t'=\spec(k')$ 
et $X_{t'} = X\times_S\spec(k')$. Il nous faut vérifier que, quelques soient 
$\bar t$ et $t'$, $X_{t'}$ est connexe (par qui on entend connexe et non vide). 
Soit $A'$ le normalisé de $A$ dans $k'$, i.e. l'anneau des éléments de 
$k'$ entiers sur l'image de $A$ dans $k(t)$. On a 
$A\{t\}\otimes_A A'\iso A'\{T\}$: le membre de gauche est en effet hensélien 
local (car $A'$ est fini sur $A$, et local) et limite d'algèbres locales 
étales sur $A'\{T\}=A\{T\}\otimes_A A'$. Le schéma $X_{t'}$ est donc encore 
la fibre en $t'$ de $X=\spec\left(A'\{T\}\right)$ sur $S'=\spec(A')$. Le 
schéma local $X'$ est normal, donc intègre; son localisé $X_{t'}$ est 
encore intègre, a fortiori connexe. 





\begin{proposition}\label{I:5-2-3}
Soient $A$ un anneau local strictement hensélien, $S=\spec(A)$ et 
$X=\spec\left(A\{T\}\right)$. Soient $\bar t$ un point géométrique de $S$ 
et $X_{\bar t}$ la fibre géométrique correspondante. Alors tout revêtement 
étale galoisien de $X_{\bar t}$ d'ordre premier à la caractéristique du 
corps résiduel de $A$ est trivial. 
\end{proposition}





\begin{lemma}[théorème de pureté de Zariski-Nagata en dimension $2$] \label{I:5-2-4} % originally 5.2.3.1
Soient $C$ un anneau local régulier de dimension $2$ et $C'$ une $C$-algèbre 
finie normale étale au--dessus de l'ouvert complémentaire du point fermé 
de $\spec(C)$. Alors $C'$ est étale sur $C$.
\end{lemma}

En effet $C'$ est normal de dimension $2$, donc $\operatorname{prof}(C')=2$. 
Puisque $\operatorname{prof}(C')+\dim\operatorname{proj}(C') = \dim(C)=2$, on 
en conclut que $C'$ est libre sur $C$. Alors l'ensemble des points de $C$ où 
$C'$ est ramifié est défini par une équations, le discriminant; puisqu'il 
ne contient pas de point de hauteur $1$, il est vide. 





\begin{lemma}[cas particulier du lemme d'Abhyankar]\label{I:5-2-5} % originally 5.2.3.2
Soient $S=\spec(V)$ un trait, $\pi$ une uniformisante, $\eta$ le point 
générique de $S$, $X$ lisse sur $S$, irréductible, de dimension 
relative $1$, $\widetilde X_\eta$ un revêtement étale galoisien de $X_\eta$, 
de degré $n$ inversible sur $S$, et $S_1=\spec\left(V[\pi^{1/n}]\right)$. 
Notons par un indice $(-)_1$ le changement de base de $S$ à $S_1$. Alors, 
$\widetilde X_{1\eta}$ se prolonge en un revêtement étale de $X_1$. 
\end{lemma}

Soit $\widetilde X_1$ le normalisé de $X_1$ de $\widetilde X_{1\eta}$. Vu la 
structure des groupes d'inertie modérée des anneaux de valuation discrète 
localisés de $X$ aux points génériques de la fibre spéciale $X_s$, 
$\widetilde X_1$ est étale sur $X_1$ sur la fibre générale, et aux points 
génériques de la fibre spéciale. Par (\ref{I:5-2-4}), il est étale 
partout. 





\subsubsection{}\label{I:5-2-6} % originally 5.2.3.3

Notons $t$ le point en lequel $\bar t$ est localisé. Il nous est loisible de 
remplacer $A$ par le normalisé de $A$ dans une extension finie séparable 
$k(t')$ de $k(t)$ dans $k(\bar t)$ (cf. \ref{I:5-2-2}). Ceci, et un passage à 
la limite préliminaire, nous permettent de supposer que 
\begin{enumerate}[\indent a)]
  \item $A$ est normal noethérien, et $t$ est le point générique de $S$.
  \item Le revêtement étale considéré de $X_{\bar t}$ provient d'un 
    revêtement étale de $X_t$. 
  \item Il provient d'un revêtement étale $\widetilde X_U$ de l'image 
    réciproque $X_U$ d'un ouvert non vide $U$ de $S$ (résulte de b): $t$ 
    est la limite des $U$).
  \item La complément de $U$ est de codimension $\geqslant 2$ (ceci au prix 
    d'agrandir $k(t)$, par application de (\ref{I:5-2-5}) aux anneaux de 
    valuation discrète localisés de $S$ en les points de $S\setminus U$ de 
    codimension $1$ dans $S$; ces points sont en nombre fini). 
  \item Le revêtement $\widetilde X_U$ est trivial su-dessus du sous-schéma 
    $T=0$. (Ceci quitte à encore agrandir $k(t)$). 
\end{enumerate}

Lorsque ces conditions sont remplies, nous allons voir que le revêtement 
$\widetilde X_U$ est trivial. 





\begin{lemma}\label{I:5-2-7} % originally 5.2.3.4
Soient $A$ un anneau local strictement hensélien normal et noethérien, $U$ 
un ouvert de $\spec(A)$ dont le complément est de codimension $\geqslant 2$, 
$V$ son image réciproque dans $X=\spec\left(A\{T\}\right)$ et $V'$ un 
revêtement étale de $V$. Si $V'$ est trivial au-dessus de $T=0$, alors $V'$ 
est trivial. 
\end{lemma}

Soit $B=\Gamma(V',\cO)$. Puisque $V'$ est l'image inverse de $V$ dans 
$\spec(B)$, il suffit de voir que $B$ est fini étale sur $A\{T\}$ (donc 
décomposé, puisque $A\{T\}$ est strictement hensélien). Soit 
$\widehat X=A\llbracket T\rrbracket$, et notons par $(-)^\wedge$ le changement 
de base de $X$ à $\widehat X$. Le schéma $\widehat X$ est fidèlement plat 
sur $X$. On a donc 
$\Gamma(\widehat V',\cO) = B\otimes_{A\{T\}} A\llbracket T\rrbracket$, et il 
suffit de voir que cet anneau $\widehat B$ est fini étale sur 
$A\llbracket T\rrbracket$. 

Soit $V_m$ (resp. $V_m'$) le sous-schéma de $\widehat V$ (resp. $\widehat V'$) 
d'équation $T^{m+1} = 0$. Par hypothèse, $V_0'$ est un revêtement trivial 
de $V_0$: une somme de $n$ copies de $V_0$. De même pour $V_m'/V_m$, puisque 
les revêtements étales sont insensibles aux nilpotents. On en tire 
\[
  \varphi : \Gamma(\widehat V',\cO) \to \varprojlim_m \Gamma(V_m',\cO) = \left(\varprojlim_m \Gamma(V,\cO)\right)^n \text{.}
\]
Par hypothèse, le complément de $U$ est de profondeur $\geqslant 2$: on a 
$\Gamma(V_m,\cO)=A[T]/(T^{m+1})$, et $\varphi$ est un homomorphisme de 
$\widehat B$ dans $A\llbracket T\rrbracket^n$. Au-dessus de $U$, il fournit $n$ 
sections distinctes de $\widehat V'/\widehat V$: $\widehat V'$ est donc trivial, 
somme de $n$ copies de $\widehat V$. Le complément de $\widehat V$ dans 
$\widehat X$ étant encore de codimension $\geqslant 2$ (donc de profondeur 
$\geqslant 2$), on en déduit que $\widehat B=A\llbracket T\rrbracket^n$, 
d'où le lemme. 










\subsection{Applications}\label{I:5-3}





\begin{theorem}[spécialisation des groupes de cohomologie]\label{I:5-3-1}
Soit $f:X\to S$ un morphisme propre et localement acyclique, par exemple un 
morphisme propre et lisse. Alors les faisceaux $\R^q f_*\dZ/n$ sont localement 
constants constructibles et pour toute flèche de spécialisation 
$t\to\widetilde S^s$, les flèches de cospécialisation 
$\h^q(X_t,\dZ/n)\to\h^q(X_s,\dZ/n)$ sont bijectives.
\end{theorem}

Cela résulte immédiatement de la définition des flèches de 
cospécialisation et des théorèmes de finitude et de changement de base pour 
les propres. 





\begin{theorem}[changement de base par un morphisme lisse]\label{I:5-3-2}
Soit un diagramme cartésien 
\[\xymatrix{
  X' \ar[r]^-{g'} \ar[d]^-{f'} 
    & X \ar[d]^-f \\
  S' \ar[r]^-{g} 
    & S
}\]
avec $g$ lisse. Pour tout faisceau $\cF$ sur $X$, de torsion et premier aux 
caractéristiques résiduelles de $S$, on a 
\[
  g^* \R^q f_* \cF \iso \R^q f_*'({g'}^* \cF) \text{.}
\]
\end{theorem}

En prenant un recouvrement ouvert de $X$< on se ramène au cas où $X$ est 
affine, puis par un passage à la limite au case \`i $X$ est de type fini sur 
$S$. Alors $f$ se factorise en une immersion ouverte $j:X\to \bar X$ et un 
morphisme propre $\bar f:\bar X\to S$. De la suite spectrale de Leray pour 
$\bar f\circ j$ et du théorème de changement de base pour les morphismes 
propres, on déduit qu'il suffit de démontrer le théorème dans le cas 
où $X\to S$ est une immersion ouverte. 

Dans ce cas, si $\cF$ est de la forme $\varepsilon_* C$, où 
$\varepsilon:t\to X$ est un point géométrique de $X$< le théorème est 
corollaire de (\ref{I:5-1-5}). Le cas général en résulte par le lemme 
(\ref{I:5-1-8}). 





\begin{corollary}\label{I:5-3-3}
Soient $K/k$ une extension de corps séparablement clos, $X$ un $k$-schéma 
et $n$ un entier premier à la charactéristique de $k$. Alors l'application 
canonique $\h^q(X,\dZ/n) \to \h^q(X_K,\dZ/n)$ est bijective pour tout 
$q\geqslant 0$. 
\end{corollary}

Il suffit de remarquer que $\bar K$ est limite inductive de $\bar k$-algèbres 
lisses. 





\begin{theorem}[pureté relative]\label{I:5-3-4}
Soit un diagramme commutatif 
\begin{equation}\label{I:eq:5-3-4-1}\xymatrix{
  U \ar@{^{(}->}[r]^-j \ar[dr]
    & X \ar[d]^-f 
    & \ar@{_{(}->}[l]_-i Y \ar[dl]^-h \\
  & S
}\end{equation}
avec $f$ lisse purement de dimension relative $N$, $h$ lisse purement de 
dimension relative $N-1$, $i$ un plongement fermé et $U=X\setminus Y$. Pour 
$n$ premier aux charactéristiques résiduelles de $S$, on a 
\begin{equation*}
\begin{aligned}
  j_* \dZ/n     &= \dZ/n \\
  \R^1 j_*\dZ/n &= \dZ/n(-1)_Y \\
  \R^q j_*\dZ/n &= 0 \qquad \text{pour $q\geqslant 2$}
\end{aligned}
\end{equation*}
\end{theorem}

Dans ces formules, $\dZ/n(-1)$ désigne le $\dZ/n$-duel de $\dmu_n$. Si $t$ 
est une équation locale pour $Y$, l'isomorphisme 
$\R^1j_*\dZ/n\simeq\dZ/n(-1)_Y$ est défini par l'application 
$a:\dZ/n\to \R^1j_*\dmu_n$ qui envoie $1$ sur la classe du $\dmu_n$-torseur des 
racines $n$-ièmes de $t$. 

La question est de nature locale. Ceci permet de remplacer $(X,Y)$ par un 
couple localement isomorphe, par exemple 
\[\xymatrix{
  \dA_T^1 \ar@{^{(}->}[r]^-j \ar[dr]_-g 
    & \dP_T^1 \ar[d]^-f 
    & \ar@{_{(}->}[l]_-i T \ar[dl] \\
  & T
}\]
avec $T=\dA_S^{n-1}$ et $i=$ section à l'infini. Le corollaire (\ref{I:5-1-9}) 
s'applique à $g$, et fournit $\R^q g_ \dZ/n=\dZ/n$ pour $q=0$, $0$ pour $q>0$. 
Pour $f$, on a par ailleurs \eqref{I:eq:4-6-2-1}
\[
  \R^q f_* \dZ/n = \dZ/n,0,\dZ/n(-1),0 \text{ pour $q>2$.}
\]
On vérifie facilement que $j_* \dZ/n = \dZ/n$, et que les $\R^q j_*\dZ/n = 0$ 
sont concentrés sur $i(T)$ pour $q>0$. La suite spectrale de Leray 
$E_2^{pq} = \R^p f_* \R^q j_*\dZ/n \Rightarrow \R^{p+q}g_* \dZ/n$ se réduit 
donc à 
\[\xymatrix{
  i^* \R^q j_* \dZ/n & 0 & \cdots \\
  i^* \R^1 j_* \dZ/n \ar[drr]^-{d_2} & 0 & \cdots \\
  \dZ/n              & 0 & \dZ/n(-1) & 0 & \cdots
}\]
$\R^q j_*\dZ/n=0$ pour $q\geqslant 0$, et $\R^q j_*\dZ/n$ est le prolongement 
par zéro d'un faisceau localement libre de rang un sur $T$ (isomorphe, via 
$d_2$, à $\dZ/n(-1)$). L'application $a$, définie plus haut, étant 
injective (ainsi qu'on le vérifie fibre par fibre), c'est un isomorphisme, et 
ceci prouve (\ref{I:5-3-4}). 





\subsubsection{}\label{I:5-3-5}

Nous renvoyons à \cite[XVI.4,5]{sga4} pour la démonstration des 
applications suivantes du théorème d'acyclicité (\ref{I:5-2-1}). 





\begin{theorem}\label{I:5-3-6} % originally 5.3.5.1
Soient $f:X\to S$ un morphisme de schémas de type finie sur $\dC$ et $\cF$ un 
faisceau constructible sur $X$. Alors 
\[
  \an{(\R^q f_*\cF)} \sim \R^q \an f_* (\an\cF)
\]
(cf. \ref{I:4-6-3}: en cohomologie ordinaire, il est nécessaire de supposer 
$\cF$ est constructible et non seulement de torsion). 
\end{theorem}





\begin{theorem}\label{I:5-3-7} % originally 5.3.5.2
Soient $f:X\to S$ un morphisme de schémas de type fini sur un corps $k$ de 
charactéristique $0$ et $\cF$ un faisceau constructible sur $X$. Alors, les 
$\R^q f_* \cF$ sont également constructibles.
\end{theorem}

La preuve utilise la résolution des singularités et (\ref{I:5-3-4}). Elle est 
généralisée au cas d'un morphisme de type fini de schémas excellents de 
charactéristique $0$ dans \cite[XIX.5]{sga4}. Une autre démonstration, 
indépendante de la résolution, est donnée dans ce volume 
(\nameref{VII}, \ref{VII:1-1}) 
% NOTE: originally (Th. finitude, 1.1),
elle s'applique à un morphisme de schémas de type fini sur un corps ou sur 
un anneau de Dedekind. 




















\section{Dualité de Poincaré}










\subsection{Introduction}\label{I:6-1}

Soit $X$ un variété topologique orientée, purement de dimension $N$, et 
supposons que $X$ admette un recouvrement ouvert fini 
$\sU=(U_i)_{1\leqslant i\leqslant K}$, tel que les intersections non vides 
d'ouverts $U_i$ soient homéomorphes à des boules. Pour une telle variété, 
le théorème de dualité de Poincaré peut se présenter ainsi: 

\begin{enumerate}[\indent A]
  \item La cohomologie de $X$ est la cohomologie de \v Cech correspondant au 
    revêtement $\sU$. C'est la cohomologie du complexe 
    \begin{equation}\label{I:eq:6-1}
      0 \to \dZ^{A_0} \to \dZ^{A_1} \to \cdots
    \end{equation}
    où $A_k=\{(i_0,\dotsc,i_k) : i_0<\cdots<i_k\text{ et } U_{i_0}\cap \cdots \cap U_{i_k} \ne \varnothing\}$. 
    \item Pour $a\in A_k$, $a=(i_0,\dotsc,i_k)$, soit 
      $U_a=U_{i_1}\cap \cdots \cap U_{i_k}$ et soit $j_a$ l'inclusion de $U_a$ 
      dans $X$. Le faisceau constant $\dZ$ sur $X$ admet la résolution (à 
      gauche) 
      \begin{equation}\label{I:eq:6-2}
      \xymatrix{
        \cdots \ar[r]
          & \displaystyle\bigoplus_{a\in A_1} j_{a!}\dZ \ar[r] 
          & \displaystyle\bigoplus_{a\in A_0} j_{a!} \dZ \ar[r] \ar[d] 
          & 0 \\
        & & \dZ \text{.}
      }
      \end{equation}
      La cohomologie à support compact $\h_c^\bullet(X,j_{a!}\dZ)$ n'est autre 
      que la cohomologie à support propre de la boule (orientée) $U_a$: 
        \[
          \h_c^i(X,j_{a!}\dZ) = \begin{cases}
                                  0   & \text{si $i\ne N$} \\
                                  \dZ & \text{si $i = N$}
                                \end{cases}
        \]
\end{enumerate}

La suite spectrale d'hypercohomologie, pour le complexe \eqref{I:eq:6-2}, et la 
cohomologie \`a support propre, affirme donc que $\h_c^i(X)$ est le 
$(i-N)$-i\`eme groupe de cohomologie d'un complexe 
\begin{equation}\label{I:eq:6-3}
  \cdots \to \dZ^{A_1} \to \dZ^{A_0} \to 0 \text{.}
\end{equation}
Ce complexe est le dual du complexe \eqref{I:eq:6-1}, d'o\`u la dualit\'e de 
Poincare.

Les points essentiels de cette construction sont 
\begin{enumerate}[\indent a)]
  \item l'existence d'une théorie de cohomologie à support propre: 
  \item le fait que tout point $x$ d'une varété $X$ purement de dimension 
    $N$ a un système fondamental de voisinages ouverts $U$ pour laquels 
    \begin{equation}\label{I:eq:6-4}
      \h_c^i(U) = \begin{cases}
                    0   & \text{pour $i\ne N$,} \\
                    \dZ & \text{pour $i=N$.}
                  \end{cases}
    \end{equation}
\end{enumerate}

La dualité de Poincaré en cohomologie étale peut se construire sur ce 
modèle. Pour $X$ lisse purement de dimension $N$ sur $k$ algébriquement clos, 
$n$ inversible sur $X$ et $x$ un point fermé de $X$, le point clef est de 
calculer la limite projective, étendue aux voisinages étales $U$ de $x$ 
\begin{equation}\label{I:eq:6-5}
  \varprojlim \h_c^i(U,\dZ/n) 
    = \begin{cases}
        0     & \text{si $i\ne 2N$} \\
        \dZ/n & \text{si $i=2N$.}
      \end{cases} 
\end{equation}

De même que pour les variétés topologiques il faut d'abord traiter 
directement le cas d'une boule ouverte (voire simplement celui de l'intervalle 
$(0,1)$), ici il faut d'abord traiter directement le cas des courbes 
(\ref{I:6-2}). Le théorème d'acyclicité locale des morphismes lisses 
permet ensuite de ramener \eqref{I:eq:6-5} à ce cas particulier (\ref{I:6-3}). 

Les isomorphismes \eqref{I:eq:6-4} et \eqref{I:eq:6-5} ne sont pas canoniques: 
ils dépendent du choix d'une \emph{orientation} de $X$. Pour $n$ inversible 
sur un schéma $X$, $\dmu_n$ est un faisceau de $\dZ/n$-modules libres de rang 
un. On note $\dZ/n(N)$ sa puissance tensorielle $N$-ième ($N\in\dZ$). La 
forme intrinsèque de la duixième ligne de \eqref{I:eq:6-5} est 
\begin{equation}\label{I:eq:6-5'}
  \varprojlim \h_c^{2N} \left(U,\dZ/n(N)\right) = \dZ/n \text{,}
\end{equation}
et $\dZ/n(N)$ s'appelle le \emph{faisceau d'orientation} de $X$; le faisceau 
$\dZ/n(N)$ étant constant, isomorphe à $\dZ/n$, on peut le faire sortir 
du signe $N$ et écrire plutôt
\begin{equation}\label{I:eq:6-5''}
  \varprojlim \h_c^{2N} (U,\dZ/n) = \dZ/n(-N) \text{,}
\end{equation}
et la dualité de Poincaré prendra la forme d'une dualité parfaite, à 
valeurs dans $\dZ/n(_N)$, entre $\h^i(X,\dZ/n)$ et $\h_c^{2N-1}(X,\dZ/n)$. 










\subsection{Le cas des courbes}\label{I:6-2}





\subsubsection{}\label{I:6-2-1}

Soient $\bar X$ une courbe projective et lisse sur $k$ algébriquement clos, et 
$n$ inversible sur $\bar X$. La preuve de (\ref{I:3-3-5}) fournit, pour 
$\bar X$ connexe, un isomorphisme canonique 
\[\xymatrix{
  \h^2(\bar X,\dmu_n) = \pic(\bar X)/n\pic(\bar X) \ar[r]^-{\deg}_-{\sim}
    & \dZ/n
}\]
Soient $D$ un diviseur réduit de $\bar X$ et $X=\bar X\setminus D$ 
\[\xymatrix{
  X \ar@{^{(}->}[r]^-j 
    & \bar X 
    & \ar@{_{(}->}[l]_-i D \text{.}
}\]
La suite exacte $0\to j_! \dmu_n \to \dmu_n\to i_* \dmu_n\to 0$ et le fait que 
$\h^i(\bar X,i_*\dmu_n) = \h^i(D,\dmu_n)=0$ pour $i>0$ fournissent un 
isomorphisme 
\[\xymatrix{
  \h_c^2(X,\dmu_n) = \h^2(\bar X,j_! \dmu_n) \ar[r]^-\sim 
    & \h^2(\bar X,\dmu_n) \ar[r]^-\sim 
    & \dZ/n \text{.}
}\]
Pour $\bar X$ disconnexe, on a de même 
\[
  \h_c^2(X,\dmu_n) = (\dZ/n)^{\pi_0(X)}
\]
et on définit le \emph{morphisme trace} comme la somme 
\[\xymatrix{
  \tr : \h_c^2(X,\dmu_n) \simeq (\dZ/n)^{\pi_0(X)} \ar[r]^-\Sigma
    & \dZ/n\text{.}
}\]





\begin{theorem}\label{I:6-2-2}
La forme $\tr(a\smallsmile b)$ identifie chacun des deux groupes $\h^i(X,\dZ/n)$ 
et $\h_c^{2-i}(X,\dmu_n)$ au dual (à valeurs dans $\dZ/n$) de l'autre. 
\end{theorem}

\emph{Preuve transcendante.} Si $\bar X$ est une courbe projective et lisse sur 
un trait $S$, et que $j:X\hookrightarrow \bar X$ est l'inclusion du 
complémentaire d'un diviseur $D$ étale sur $S$< les cohomologies (resp. 
les cohomologies à support propre) des fibres géométriques spéciales et 
génériques de $X/S$ sont ``les mêmes,'' i.e. les fibres de faisceaux 
localement constants sur $S$. Ceci se déduit des faits analogues pour 
$\bar X$ et $D$, via la suite exacte $0\to j_! \dZ/n\to\dZ/n\to {\dZ/n}_D\to 0$ 
(pour la cohomologie à supports propres) et les formules $j_*\dZ/n = \dZ/n$, 
$\R^1 j_* \dZ/n \simeq {\dZ/n}_D(-1)$, $\R^i j_*\dZ/n = 0$ ($i\geqslant 2$) 
(pour la cohomologie ordinaire) (\ref{I:5-3-4}). 

Ce principe de spécialisation ramène (\ref{I:6-2-2}) au cas où $k$ est de 
charactéristique $0$. Par (\ref{I:5-3-3}), ce cas se ramène à celui où 
$k=\dC$. Enfin, pour $k=\dC$, les groupes $\h^\bullet(X,\dZ/n)$ et 
$\h_c^\bullet(X,\dmu_n)$ coïncident avec les groupes de même nom, calculés 
pour l'espace topologique classique $X_{\textnormal{cl}}$ et, via 
l'isomorphisme $\dZ/n\to \dmu_n: x\mapsto \exp\left(\frac{2\pi i x}{n}\right)$, 
le morphisme trace s'identifie à ``intégration sur la classe fondamentale,'' 
de sorte que (\ref{I:6-2-2}) résulte de la dualité de Poincaré pour 
$X_{\textnormal{cl}}$. 





\subsubsection{Preuve algébrique}\label{I:6-2-3}

Pour une preuve très économique, voir (\nameref{V}, \S\ref{V:2}). 
% originally (Dualité \S 2)
% NOTE: should be hyperlinked V:2
En voici une autre, liée à l'autodualité de la jacobienne. 

Reprenons les notations de (\ref{I:6-2-1}). On peut supposer -- et on suppose -- 
que $X$ est connexe. Les cas $i=0$ et $i=2$ étant triviaux, on suppose aussi 
que $i=1$. Définissons $_D\dG_m$ par la suite exacte 
$0\to _D\dG_m \to \dG_m \to i_*\dG_m\to 0$ (sections de $\dG_m$ congrues à 
$1\mod D$). Le groupe $\h^1(\bar X, _D\dG_m)$ classifie les faisceaux 
inversibles sur $\bar X$ trivialisés sur $D$. C'est le groupe des points de 
$\pic_D(\bar X)$, une extension de $\dZ$ (le degré) par le groupe des points 
d'une jacobienne généralisée de Rosenlicht (correspondant su conducteur 
$1$ en chaque point de $D$) $\pic_D^0(\bar X)$, elle-même extension de la 
variété abélienne $\pic^0(\bar X)$ par le tore 
$\dG_m^D/(\text{$\dG_m$ diagonal})$. 

\begin{enumerate}[\indent a)]
  \item La suite exacte 
    $0\to j_!\dmu_n \to _D\dG_m \xrightarrow{n} _D\dG_m\to 0$ fournit un 
    isomorphisme 
    \begin{equation}\label{I:eq:6-2-3-1}
      \h_c^1(X,\dmu_n) = \pic_D^0(\bar X)_n \text{.}
    \end{equation}
  \item L'application qui à $x\in X(k)$ associe la classe du faisceau 
    inversible $\cO(x)$ sur $\bar X$, trivialisé par $1$ sur $D$, provient 
    d'un morphisme 
    \[
      f:X \to \pic_D(\bar X) \text{.}
    \]
\end{enumerate}
Pour la suite, on fixe un point base $0$ et on pose $f_0(x)=f(x)-f(0)$. Pour 
tout homomorphisme $v:\pic_D^0(\bar X)_n \to \dZ/n$, soit 
$\bar v\in \h^1\left(\pic_D^0(\bar X),\dZ/n\right)$ l'image par $v$ de la classe 
dans $\h^1\left(\pic_D^0(\bar X),\pic_D^0(\bar X)_n\right)$ du torseur défini 
par l'extension 
\[\xymatrix{
  0 \ar[r] 
    & \pic_D^0(\bar X)_n \ar[r] 
    & \pic_D^0(\bar X) \ar[r]^-n 
    & \pic_D^0(\bar X) \ar[r] 
    & 0
}\]
La théorie du corps de classes géométrique (telle qu'exposée dans 
Serre \cite{se59}) montre que l'application $v\mapsto f_0^*(\bar v)$: 
\begin{equation}\label{I:eq:6-2-3-2}
  \hom\left(\pic_D^0(\bar X)_n,\dZ/n\right) \to \h^1(X,\dZ/n)
\end{equation}
est un isomorphisme. Pour déduire (\ref{I:6-2-2}) de \eqref{I:eq:6-2-3-1} et 
\eqref{I:eq:6-2-3-2}, il reste à savoir que 
\begin{equation}\label{I:eq:6-2-3-3}
  \tr\left(u \smallsmile f_0^*(\bar v)\right) = -v(u) \text{.}
\end{equation}

% originally (Dualité, 3.2.4).










\subsection{Le cas général}\label{I:6-3}

Soient $X$ un variété algébrique lisse et purement de dimension $N$, sur 
$k$ algébriquement clos. Pour énoncer le théorème de dualité de 
Poincaré, il faut tout d'abord définir le morphisme trace 
\[
  \tr : \h_c^{2N}\left(X,\dZ/n(N)\right) \to \dZ/n\text{.}
\]
La définition est un dévissage pénible à partir dur cas des courbes 
\cite[XVIII \S 2]{sga4}. On a alors 





\begin{theorem}\label{I:6-3-1}
La forme $\tr(a\smallsmile b)$ identifie chacun des groupes 
$\h_c^i\left(X,\dZ/n(N)\right)$ et $\h^{2 N-i}(X,\dZ/n)$ au dual de l'autre.
\end{theorem}

Soient $x\in X$ un point fermé et $X_x$ le localisé strict de $X$ en $x$. 
Nous posons, pour $U$ parcourant les voisinages étales de $x$ 
\begin{equation}\label{I:eq:6-3-1-1}
  \h_c^\bullet(X_x,\dZ/n) = \varprojlim \h_c^\bullet(U,\dZ/n) \text{.}
\end{equation}
Il serait préférable de considérer plutôt le pro-objet 
$\text{``}\varprojlim\text{''} \h_c^\bullet(U,\dZ/n)$ mais, les groupes en jeu 
étant finis, la différence est inessentielle. Comme on a tenté de 
l'expliquer dans l'introduction, (\ref{I:6-3-1}) résulte de ce que 
\begin{equation}\label{I:eq:6-3-1-2}
\begin{aligned}
  \h_c^i(X_x,\dZ/n) &= 0 \text{ pour $i\ne 2N$ et que } \\
  \tr : \h_c^{2 N}\left(X_x,  \dZ/n(N)\right) &\iso \dZ/n \text{ est un isomorphisme.} 
\end{aligned}
\end{equation}

Le cas où $N=0$ est trivial. Si $N>0$, soit $Y_y$ le localisé strict en un 
point fermé d'un schéma $Y$ lisse purement de dimension $N-1$, et soit 
$f:X_x\to Y_y$ un morphisme essentiellement lisse (de dimension relative un). 
La démonstration utilise la suite spectrale de Leray en cohomologie à 
support propre pour $f$, pour se ramener au cas des courbes. Les ``cohomologie 
à support propre'' considérées étant définies comme des limites 
\eqref{I:eq:6-3-1-1}, l'existence d'une telle suite spectrale pose divers 
problèmes de passage à la limite, traités avec trop de détailes dans 
\cite[XVIII]{sga4}. Pour tout point géométrique $z$ de $Y_y$, on a 
\[
  (\R^if_! \dZ/n)_z = \h_c^i(f^{-1}(z),\dZ/n) \text{.}
\]
La fibre géométrique $f^{-1}(z)$ est une limite projective de courbes lisses 
sur un corps algébriquement clos. Elle vérifie la dualité de Poincaré. 
Sa cohomologie ordinaire est donnée par le théorème d'acyclicité locale 
pour les morphismes lisses: 
\[
  \h^i(f^{-1}(z),\dZ/n) 
    = \begin{cases}
        \dZ/n & \text{pour $i=0$} \\
        0     & \text{pour $i>0$.}
      \end{cases}
\]
Par dualité, on a 
\[
  \h_c^i(f^{-1}(z),\dZ/n) 
    = \begin{cases}
        \dZ/n(-1) & \text{pour $i=2$} \\
        0         & \text{pour $i\ne 2$.}
      \end{cases}
\]
et la suite spectrale de Leray s'écrit 
\[
  \h_c^i\left(X_x,\dZ/n(N)\right) = \h_c^{i-2}\left(Y_y,\dZ/n(N-1)\right) \text{.}
\]
On conclut par récurrence sur $N$. 










\subsection{Variantes et applications}\label{I:6-4}

On peut construire, en cohomologie étale, un ``formalisme de dualité'' 
($=$ des foncteurs $\R f_!$, $\R f_!$, $\R f^!$ satisfaisant diverses 
compatibilités et formules d'adjonction) parallèle à celui qui existe en 
cohomologie cohérente. Dans ce language, les résultats du paragraphe 
précédents se transcrivent comme suit: si $f:X\to S$ est lisse et purement 
de dimension relative $N$, et que $S=\spec(k)$, avec $k$ algébriquement clos, 
alors 
\[
  \R f^! \dZ/n = \dZ/n [2N](N) \text{.}
\]
Ce résultat vaut sans hypothèse sur $S$. Il admet le 





\begin{corollary}\label{I:6-4-1}
Si $f$ est lisse et purement de dimension relative $N$, et que les faisceaux 
$\R^i f_! \dZ/n$ sont localement constants, alors les faisceaux 
$\R^i f_* \dZ/n$ sont aussi localement constants, et 
\[
  \R^i f_* \dZ/n = \const\hom\left(\R^{2 N-i} f_! \dZ/n(N),\dZ/n\right)\text{.}
\]
\end{corollary}

En particulier, sous les hypothèses du corollaire, les faisceaux 
$\R^i f_* \dZ/n$ sont constructibles. Partant de là, on peut montrer que, si 
$S$ est de type fini sur le spectre d'un corps ou d'un anneau de Dedekind, 
alors, pour tout morphisme de type fini $f:X\to S$ et tout faisceau 
constructible $\cF$ sur $X$, les faisceaux $\R^i f_* \cF$ sont constructibles 
(\nameref{VII}, \ref{VII:1-1}). 
% originally (Th. finitude, 1.1).
% !TEX root = sga4.5.tex

\chapter{Rapport sur la formule des traces}\label{II}

Dans ce texte, j'ai tenté d'exposer de façon aussi directe que possible la 
théorie cohomologique de Grothendieck des fonctions $L$. Je suis de très près 
certains des exposés donnés par Grothendieck à l'IHES au printemps 1966. Dans 
l'esprit de ce volume, il ne sera pas fait appel à SGA 5 -- saui deux 
références à des passages de l'exposé XI, indépendant au reste de ce 
séminaire. 

Dans le \S 10 (p.81-90) de \cite{de73}, le lecteur trouvera un résume de la 
théorie, dans le cas crucial des courbes. 










\section{Quelques rappels}\label{II:1}





\subsection{Notations}\label{II:1-1}

$p,q,\dF,\dF_q$: $p$ est un nombre premier, $q=p^f$ une puissance de $p$ et 
$\dF$ un clôture algébrique du corps premier $\dF_p$. Pour toute puissance 
$p^n$ de $p$, on note $\dF_{p^n}$ le sous-corps à $p^n$ éléments de $\dF$. 

$X_0,X$: On désignera souvent par un symbole affecte d'un indice $0$ un objet 
sur $\dF_q$. Le même symbole, sans $0$, désigne alors l'objet qui s'en déduit 
par extension des scalaires à $\dF$. Par exemple, si $\cF_0$ est un faisceau 
sur un schéma $X_0$ sur $\dF_q$, on note $\cF$ son image réciproque sur 
$X=X_0\otimes_{\dF_q} \dF$. 

$F$: Si $X_0$ est un schéma sur $\dF_q$, on appelle \emph{morphisme de 
Frobenius} et on note $F$ l'endomorphisme de $X_0$ qui ``envoie le point de 
coordonnées $x$ dans celui de coordonnées $x^q$'': c'est l'identifie sur 
l'espace topologique sous-jacent et, pour $x$ une section locale du faisceau 
structurel, $F^* x = x^q$. 

On désigne encore par $F$ l'endomorphisme de $X$ qui s'en déduit par 
extension des scalaires. Sur l'ensemble $X(\dF) = X_0(\dF)$ des points 
rationnels de $X$, $F$ agit comme la substitution de Frobenius 
$\varphi\in\gal(\dF/\dF_q)$ définie par $\varphi(x)=x^q$. En particulier, 
$X^F=X_0(\dF_q)$. 

En cas d'ambiguïté, on écrira $F_{(q)}$ plutôt que $F$. 










\subsection{Correspondance de Frobenius}\label{II:1-2}

Soient $X_0$ un schéma sur $\dF_q$, et $\cF_0$ un faisceau sur $X_0$. J'aime 
voir comme suit la \emph{correspondance de Frobenius} (un $F$-endomorphisme de 
$\cF_0$): 
\begin{equation}\label{II:eq:1-2-1}
  F^* : F^*\cF_0 \iso \cF_0
\end{equation}

On regarde $\cF_0$ comme le faisceau des sections locales d'un espace 
$[\cF_0]$ étale sur $X_0$ ($[\cF_0]$ est un espace algébrique au sens de M. 
Artin). La fonctorialité de $F$ fournit un diagramme commutatif 
\[\xymatrix{
  [\cF_0] \ar[r]^-F \ar[d]
    & [\cF_0] \ar[d] \\
  X_0 \ar[r]^F 
    & X_0
}\]
Parce que $[\cF_0]$ est étale sur $X_0$, ce diagramme est cartésien; il fournit 
un isomorphisme $[\cF_0]\iso F^*[\cF_0]$ d'inverse \eqref{II:eq:1-2-1}. On note 
encore $F^*$ les correspondances ou morphismes déduits de $F$ par extension 
des scalaires ou par fonctorialité, tels 
$F^*:\h_c^\bullet(X,\cF)\to \h_c^\bullet(X,\cF)$. 

Pour une définition en forme, je renvoie à \cite[XI.1,2]{sga5}. On n'y c
considère que le cas où $q=p$, mais le cas général se traite de même. 





\subsection{}\label{II:1-3}

La correspondance de Frobenius est fonctorielle en $(X_0,\cF_0)$. Si 
$f_0:X_0\to Y_0$ est un morphisme de $\dF_q$-schémas, $\cF_0$ un faisceau sur 
$X_0$, $\cG_0$ un faisceau sur $Y_0$ et $u\in\hom(f_0^*\cG_0,\cF_0)$ un 
$f_0$-morphisme de $\cG_0$ dans $\cF_0$, on a un diagramme commutatif d'espaces 
$f_0 F = F f_0$, et au-dessus un diagramme commutatif de faisceaux 
\[\xymatrix{
  X_0 \ar[r]^-{f_0} \ar[d]^-F
    & Y_0 \ar[d]^-F \\
  X_0 \ar[r]^-{f_0} 
    & Y_0
}\qquad\qquad
\xymatrix{
  \cF_0 
    & \ar[l]_-u \cG_0 \\
  \cF_0 \ar[u]_-{F^*} 
    & \ar[l] \cG_0 \ar[u]^-{F^*}
}\]
En particulier, la correspondance sur $f_{0*}\cF_0$ déduite la correspondance 
de Frobenius de $(X_0,\cF_0)$ est la correspondance de Frobenius de 
$(U_0,f_{0*}\cF_0)$. 





\subsection{}\label{II:1-4}

Le morphisme naturel $Y\to Y_0$ est limite de morphismes étales. Il en 
résulte que pour tout faisceau abélien $\cF_0$ sur $X_0$, les images 
réciproques sur $Y$ des $\R^if_0\cF_0$ sont les $\R^if_*\cF$. 

Sur $\R^if_*\cF$, on dispose de 
\begin{enumerate}[\indent a)]
  \item la correspondance 
    $F^*\R^i f_*\cF \iso \R^i f_*\cF \xrightarrow{F^*} \R^if_*\cF$ déduite 
    par fonctorialité de la correspondance Frobenius; 
  \item la correspondance $F^*\R^if_*\cF\to\R^if_*\cF$ image réciproque sur 
    $Y$ de la correspondance de Frobenius sur $\R^ii f_{0*}\cF_0$. 
\end{enumerate}

On déduit de (\ref{II:1-3}) que ces correspondances coïncident. Elles 
induisent la terme initial d'une action de Frobenius sur toute la suite 
spectrale 
\[
  \h^p(Y,\R^q f_*\cF) \Rightarrow \h^{p+q}(X,\cF) \text{.}
\]
De même en cohomologie à support propre, et pour la suite spectrale 
\[
  \h_c^p(Y,\R^q f_* \cF) \Rightarrow \h_c^{p+q}(X,\cF)
\]





\subsection{Changement de corps fini}\label{II:1-5}

Soient $(X_0,\cF_0)$ sur $\dF_q$, $n$ un entier, et $(X_1,\cF_1)$ sur 
$\dF_{q^n}$ déduit de $(X_0,\cF_0)$ par extension des scalaires. On dispose 
d'une part de la correspondance de Frobenius $F_{(q)}:X_0\to X_0)$, 
$F_{(q)}^*:F_{(q)}^*\cF_0\to \cF_0$, d'autre part de 
$F_{(q^n)}:X_1\to X_1$, $F_{(q^n)}^*:F_{(q^n)}^*\cF_1\to\cF_1$. On vérifie 
facilement que $F_{(q^n)}$ se déduit de la puissance $n$-ième de $F_{(q)}$ 
par extension des scalaires de $\dF_q$ à $\dF_{q^n}$; après extension des 
scalaires à $\dF$, on a encore la même identité $F_{(q^n)}=F_{(q)}^n$ 
entre endomorphismes de $X$ et correspondances sur $X$. 

Si $x$ est un point fermé de $X_0$, avec $[k(x),\dF_q]=n$, et que 
$\bar x\in X_0(\dF) = X(\dF)$ est localisé en $x$, $\bar x$ est un point 
fixe de $F_{(q^n)}=F_{(q)}^n$. On note $F_x^*$ l'endomorphisme de 
$\cF_{\bar x}$ induit par $F_{(q^n)}^*$. A isomorphisme près, 
$(F_{\bar x}^*,\cF_{\bar x})$ ne dépend pas du choix de $\bar x$. La trace, 
etc. de $F_x^*$ se désigne par une notation telle que 
$\tr(F_x^*,\cF)$. 





\subsection{Fonction \texorpdfstring{$L$}{L}}\label{II:1-6}

Soit $X_0$ un schéma de type fini sur $\dF_q$. On note $|X_0|$ l'ensemble de 
ses points fermés et, pour $x\in |X_0|$, on pose $\deg(x)=[k(x),\dF_p]$. Si 
$\cF_0$ est un $\dQ_\ell$-faisceau sur $X_0$ (pour cette notion, voir 
ci-dessous) -- ou un faisceau constructible et plat de modules sur un 
anneau noethérien commutatif $A$, on note $L(X_0,\cF_0)$ la série formelle 
de terme constant $1$, dans $\dQ_\ell\llbracket t\rrbracket$ ou dans 
$A\llbracket t\rrbracket$, 
\[
  L(X_0,\cF_0) = \prod_{x\in|X_0|} \det\left(1-F_x^* t^{\deg(x)},\cF\right)^{-1} \text{.}
\]





\subsection{Remarque}\label{II:1-7}

Dans la définition (\ref{II:1-6}), on a $\deg(x)=[k(x),\dF_p]$ (degré 
absolu) et non $\deg(x)=[k(x),\dF_q]$ (degré relatif à $\dF_q$). Ceci à 
pour effet que $L(X_0,\cF_0)$ ne dépend que du schéma $X_0$ et du faisceau 
$\cF_0$, non de $q$ et du morphisme structurel $X_0\to\spec(\dF_q)$. 
L'indéterminée $t$ est un avatar de $p^{-s}$. 





\subsection{Le point de vue galoisien}\label{II:1-8}

Ce n$^\circ$ ne serra pas utilisé dans la suite du rapport. Pour $\cF_0$ un 
faisceau abélien sur $X_0$, le groupe de Galois $\gal(\dF/\dF_q)$ agit sur 
$\dF$, et par transport de structure sur $\h_c^i(X,\cF)$. En particulier, la 
substitution de Frobenius $\varphi$ (\ref{II:1-1}) induit un automorphisme, 
encore noté $\varphi$, de $\h_c^i(X,\cF)$. La correspondance de Frobenius 
définit de son côté un endomorphisme $F^*$ de $\h_c^i(X,\cF)$. On a 
\begin{equation}\label{II:eq:1-8-1}
  {F^*}^{-1} = \varphi \qquad \text{(sur $\h_c^i(X,\cF)$).}
\end{equation}

Pour le voir, appliquons (\ref{II:1-4}) au cas où $Y_0$ est un point: 
$Y_0=\spec(\dF_q)$. On trouve que $\h_c^i(X,\cF)$ est la fibre, au point 
géométrique $\spec(\dF)$ de $Y_0$, de $\R^i f_! \cF_0$, et que $F^*$ est 
la fibre de la correspondance de Frobenius de $\R^i f_!\cF_0$. Soit 
$[\R^i f_! \cF_0]$ comme en (\ref{II:1-2}). On a 
$\h_c^i(X,\cF) = [\R^i f_!\cF_0](\dF)$ et 
\begin{enumerate}[\indent a)]
  \item $\varphi$ agit par transport de structure, via son action sur $\dF$, 
  \item $F^*$ est l'inverse de 
    $F:[\R^i f_!\cF_0](\dF) \to [\R^i f_!\cF_0](\dF)$. 
\end{enumerate}

On conclut par l'identité $F=\varphi$ de (\ref{II:1-1}). 

Un des intérêts du point de vue galoisien est qu'il permet de raisonner par 
transport de structure. 










\section{\texorpdfstring{$\dQ_\ell$}{Ql}-faisceaux}\label{II:2}

La notion de $\dQ_\ell$-faisceaux est développée en détail dans 
\cite[V,VI]{sga5}. Nous n'utiliserons que les définitions et résultats 
ci-dessous. 





\begin{definition_}\label{II:2-1}
Soit $X$ un schéma noethérien. Un $\dZ_\ell$-faisceau $\cF$ sur $X$ est un 
système projectif de faisceaux $\cF_n$ ($n\geqslant 0$), avec $\cF_n$ un 
faisceau de $\dZ/\ell^{n+1}$-modules constructibles \cite[IX.2]{sga4}, et tel que 
le morphisme de transition $\cF_n\to\cF_{n-1}$ se factorise par un isomorphe 
\[
  \cF_n\otimes_{\dZ/\ell^{n+1}} \dZ/\ell^n \iso \cF_{n-1}\text{.}
\]
On dit que $\cF$ est \emph{lisse} si les $\cF_n$ sont localement constants. 
\end{definition_}

La terminologie de \cite{sga5} est différents, on y dit ``$\dZ_\ell$-faisceau 
constructible'' pour ``$\dZ_\ell$-faisceau,'' et ``constant tordu 
constructible'' pour ``lisse.'' 

De même, ci-dessous, pour les $\dQ_\ell$-faisceaux. 





\subsection{}\label{II:2-2}

Si $k$ est un corps séparablement clos, un faisceau de $\dZ/\ell^n$-modules 
sur $\spec(k)$ s'identifie à un $\dZ/\ell^n$-module, et le foncteur 
$\cF\mapsto \varprojlim \cF_n$ identifie les $\dZ_\ell$-faisceaux sur 
$\spec(k)$ sux $\dZ_\ell$-modules de type fini. Ceci justifie le définition 
suivante: la \emph{fibre} du $\dZ_\ell$-faisceau $\cF$ en le point 
géométrique $x$ de $X$ est le $\dZ_\ell$-module $\varprojlim (\cF_n)_x$. 





\subsection{}\label{II:2-3}

Supposons $X$ connexe, et soit $x$ un point géométrique de $X$. On sait que 
le foncteur ``fibre en $x$'' est une équivalence de la catégorie des 
faisceaux localement constants constructibles de $\dZ/\ell^n$-modules avec la 
catégorie des $\dZ/\ell^n$-modules de type fini munis d'une action continue 
de $\pi_1(X,x)$. Par passage à la limite, on en déduit la proposition 
suivante. 





\begin{proposition_}\label{II:2-4}
Sous les hypothèses ci-dessus, le foncteur ``fibre en $x$'' est une 
équivalence de la catégorie des $\dZ_\ell$-faisceaux lisse sur $X$ avec 
celle des $\dZ_\ell$-modules de type fini munis d'une action continue de 
$\pi_1(X,x)$. 
\end{proposition_}





\begin{proposition_}\label{II:2-5}
Soit $\cF$ un $\dZ_\ell$-faisceau sur un schéma noethérien $X$. Il existe 
une partition finie de $X$ en parties localement fermées $X_i$ telles que 
les $\cF|X_i$ soient lisses.
\end{proposition_}

Posons $\gr_\ell^n\cF=\ell^n\cF_m/\ell^{m+1}\cF_m$ pour $m\geqslant n$, et 
$\gr_\ell\cF=\bigoplus \gr_\ell^n\cF$. C'est un faisceau de modules sur 
l'anneau noethérien $\gr_\ell\dZ=\dZ/\ell[t]$, gradué de $\dZ$ pour la 
filtration $\ell$-adique. C'est un quotient de 
$\gr_\ell^0\cF \otimes_{\dZ/\ell} \dZ/\ell[t]$, donc un faisceau de 
$\gr_\ell\dZ$-modules constructible \cite[IX.2]{sga4}. et il existe une partition 
finie de $X$ en parties localement fermées $X_i$, telle que 
$\gr_\ell \cF|X_i$ sont localement constant. Chacun des faisceaux 
$\gr_\ell^n\cF|X_i$ est alors localement constant ainsi que les $\cF_m|X_i$, 
qui en sont des extensions successives. 





\subsection{}\label{II:2-6}

Si $\cF$ et $\cG$ sont deux $\dZ_\ell$-faisceaux sur $X$, on pose 
$\hom(\cF,\cG) = \varprojlim(\cF_n,\cG_n)$; c'est un $\dZ_\ell$-module. 
Puisque $\hom(\cF_n,\cG_n) = \hom(\cF_m,\cG_n)$ pour $m\geqslant n$, on a 
encore 
\[
  \hom(\cF,\cG) = \varprojlim_n \varinjlim_m \hom(\cF_m,\cG_n) \text{,}
\]
de sorte que le foncteur $\cF\mapsto \text{``}\varprojlim\text{''}\cF_n$ est 
pleinement fidèle, de la catégorie des $\dZ_\ell$-faisceaux dans celle 
pro-faisceaux sur $X$ ($=$ pro-objets de la catégorie des faisceaux sur $X$). 
On l'utilisera pour identifier les $\dZ_\ell$-faisceaux à certains 
pro-faisceaux (ceux tels que chaque $\cF_n=\cF/\ell^{n+1}\cF$ soit un faisceau, 
et que $\cF\iso \text{``}\varprojlim\text{''}\cF_n$). On vérifie aisément 
qui si $\cF+\text{``}\varprojlim\text{''}\cF_n$, avec $\cF_n$ un faisceau 
constructible de $\dZ/\ell^n$-modules, pour que $\cF$ soit un 
$\dZ_\ell$-faisceau, il faut et il suffit que les systèmes projectifs (en $n$) 
$\cF_n/\ell^{k+1}\cF_n$ soient essentiellement constants: le système 
projectif des $\cF_k'=\varprojlim_n\cF_n/\ell^{k+1}\cF_n$ vérifie alors les 
conditions de (\ref{II:2-1}), et $\cF=\text{``}\varprojlim\text{''}\cF_k'$. 

Dans ce qui suit, nous abandonnons la notation $\cF_n$ de (\ref{II:2-1}); si 
$\cF$ est un $\dZ_\ell$-faisceau, le faisceau $\cF_n$ de (\ref{II:2-1}) sera 
noté $\cF\otimes\dZ/\ell^{n+1}$. 





\subsection{}\label{II:2-7}

Dans la catégorie abélienne des pro-faisceaux, la sous-catégorie des 
$\dZ_\ell$-faisceaux est stable par noyau, conoyau et extension. C'est claire 
pour les conoyaux. Pour les noyaux, si $f:\cF\to\cG$ est un morphisme de 
$\dZ_\ell$-faisceaux, on déduit aisément de (\ref{II:2-4}, \ref{II:2-5}) 
et de Artin-Rees que le système projectif des 
$\ker(f:\cF\otimes\dZ/\ell^n\to\cG\otimes\dZ/\ell^n)$ vérifie le critère 
donné et (\ref{II:2-6}) ci-dessus. Il existe même un entier $r$ tel que pour 
tout $n$ 
\[
  \ker(f)\otimes\dZ/\ell^n = \ker(f:\cF\otimes\dZ/\ell^{n+r}\to\cG\otimes\dZ/\ell^{n+r})\otimes\dZ/\ell^n \text{.}
\]
Le cas des extensions est laissé au lecteur. 





\subsection{}\label{II:2-8}

Un $\dZ_\ell$-faisceau $\cF$ est \emph{sans torsion} si 
$\ker(\ell:\cF\to\cF) = 0$. Chaque $\cF\otimes\dZ/\ell^n$ est alors plat sur 
$\dZ/\ell^n$ (i.e. à fibres des $\dZ/\ell^n$-modules libres de type fini). On 
déduit de (\ref{II:2-4}, \ref{II:2-5}) que pour tout $\dZ_\ell$-faisceau 
$\cF$ il existe un entier $n$ tel que $\cF/\ker(\ell^n)$ soit sans torsion. 





\subsection{}\label{II:2-9}

La catégorie abélienne des \emph{$\dQ_\ell$-faisceaux} sur $X$ est le 
quotient de celle des $\dZ_\ell$-faisceaux par la sous-catégorie de 
$\dZ_\ell$-faisceaux de torsion. En termes équivalents, ses objets sont les 
$\dZ_\ell$-faisceaux et, notant $\cF\otimes\dZ_\ell$ le $\dZ_\ell$-faisceaux 
$\cF$, vu comme objet de le catégorie des $\dQ_\ell$-faisceaux, on a 
\[
  \hom(\cF\otimes\dQ_\ell,\cG\otimes\dQ_\ell) = \hom(\cF,\cG)\otimes_{\dZ_\ell}\dQ_\ell \text{.}
\]

La \emph{fibre} d'un $\dQ_\ell$-faisceau $\cF\otimes\dQ_\ell$ en un point 
géométrique $x$ de $X$ est la $\dQ_\ell$-espace vectoriel de dimension 
finie $\cF_x\otimes_{\dZ_\ell}\dQ_\ell$. 





\subsection{}\label{II:2-10}

Soient $X$ un schéma séparé de type fini sur un corps algébriquement 
clos $k$ et $\cF$ un $\dQ_\ell$-faisceau constructible sur $X$. Soit $\cF'$ un 
$\dZ_\ell$-faisceau constructible tel que 
$\cF\sim \cF'\otimes_{\dZ_\ell}\dQ_\ell$. Nous verrons en (\ref{II:4-11}) que 
\[
  \left(\varprojlim \h_c^q(X,\cF'\otimes \dZ/\ell^n)\right)\otimes_{\dZ_\ell}\dQ_\ell
\]
est un $\dQ_\ell$-espace vectoriel de dimension finie. Il ne dépend que de 
$\cF'$, et se note $\h_c^q(X,\cF)$. La preuve n'utilise pas que $\ell$ soit 
premier à la caractéristique, mais seul ce cas nous intéresse. 





\subsection{}\label{II:2-11}

Dans cet exposé, tous les calculs importants se feront au niveau des 
$\dZ/\ell^n$-faisceaux, le passage aux $\dQ_\ell$-faisceaux ne se faisant qu'en 
tout dernier lieu. En pratique, il est toutefois souvent commode de travailler 
systématiquement avec des $\dQ_\ell$-faisceaux. Dans les exposés 
V et VI de \cite{sga5}, Jouenolou donne le formalisme qui rend cela possible. 
Un des résultats essentiels (VI 2.2) est que si $f:X\to Y$ est un morphisme 
séparé de type fini de schémas noethériens, et $\cF$ un 
$\dZ_\ell$-faisceau sur $Y$, alors, pour chaque $q$, le pro-faisceau 
$\text{``}\varprojlim\text{''}\R^q f_! (\cF\otimes\dZ/\ell^n)$ est un 
$\dZ_\ell$-faisceau sur $Y$. Il démontrer même une propriété de 
régularité plus forte (à la Artin-Rees) pour le système projectif des 
$\R^q f_!(\cF\otimes\dZ/\ell^n)$. 

Jouanolou se place même dans un cadre plus général, englobant le cas 
très utile des $E_\lambda$-faisceau, pour $E_\lambda$ une extension finie de 
$\dQ_\ell$. Ils sont définis comme l'ont été les $\dQ_\ell$-faisceaux, 
avec $\dQ_\ell$ remplacé par $E_\lambda$, $\dZ_\ell$ par l'anneau 
$\cO_\lambda$ de la valuation $\lambda$ et $\dZ/\ell^n$ par 
$\cO_\lambda/\pi^n$, pour $\pi$ une uniformisante. Il reviendrait au même de 
définir un $E_\lambda$-faisceau comme un $\dQ_\ell$-faisceau $\cF$, muni 
d'une homomorphisme $E_\lambda\to\operatorname{End}(\cF)$. 










\section{Formule des traces et fonctions \texorpdfstring{$L$}{L}}\label{II:3}

\emph{On utilise les notations du \S\ref{II:1}, et $\ell$ est un nombre premier 
$\ne p$.}

Soient $X_0$ un schéma séparé de type fini sur $\dF_q$ ($q=p^f$) et 
$\cF_0$ un $\dQ_\ell$-faisceau constructible sur $X_0$. L'interprétation 
cohomologique de Grothendieck des fonctions $L$ est le théorème suivant: 





\begin{theorem_}\label{II:3-1}
$L(X_0,\cF_0) = \prod_i \det\left(1-F^* t^f, \h_c^i(X,\cF)\right)^{(-1)^{i+1}}$. 
\end{theorem_}

Il est obtenu comme conséquence de la formule des traces:





\begin{theorem_}\label{II:3-2}
Pour tout entier $n$, 
$\sum_{x\in X^{F^n}} \tr\left({F^n}^*,\cF_x\right) = \sum_i (-1)^i \tr\left({F^n}^*,\h_c^i(X,\cF)\right)$. 
\end{theorem_}





Rappelons brièvement la méthode classique pour déduire (\ref{II:3-1}) 
de (\ref{II:3-2}). Posons $T=t^f$; les deux membres sont des séries formelles 
en $T$, de terme constant $1$. Puisque $\dQ_\ell$ est de caractéristique 
$0$, il suffit de prouver que les dérivées logarithmiques 
$T\frac{d}{dT}\log$ des deux membres sont égales. 

Pour calculer ces dérivées logarithmiques, nous utiliserons la formule 
\[
  T \frac{d}{dT}\log\det(1-f T)^{-1} = \sum_{n>0} \tr(f^n) T^n \text{,}
\]
% NOTE: the following lines are not typesetting correctly
dont la démonstration (dans un cadre un peu plus général) est rappelée 
ci-dessous. Au premier nombre, on trouva 
\begin{align*}
  T\frac{d}{dT}\log L(X_0,\cF_0) 
    &= \sum_{x\in |X_0|} \sum_n [k(x):\dF_q] \tr\left(F_x^n,\cF\right) T^{n\cdot[k(x):\dF_q]} \\
    &= \sum_{n>0} T^n \sum_{x\in X^{F^n}} \tr\left({F^n}^*,\cF_x\right)
\end{align*}
et su second membre 
\[
  \sum_{n>0} T^n \sum (-1)^i \tr\left({F^n}^*,\h_c^i(X,\cF)\right) \text{ ; }
\]
on compare terme à terme. 





\begin{proposition_}\label{II:3-3}
Soient $A$ un anneau commutatif, $M$ un $A$-module projectif de type fini et 
$f$ un endomorphisme de $M$. Pour tout série formelle inversible $Q$, on pose 
$\frac{d}{dT}\log Q = Q^{-1}\frac{d}{dT}Q$. On a 
\begin{equation}\label{II:eq:3-3-1}
  T\frac{d}{dT}\log\det(1-f T)^{-1} = \sum_{n>0} \tr(f^n) T^n \text{.}
\end{equation}
\end{proposition_}

Soit $M'$ un $A$-module tel que $M\oplus M'$ soit libre de rang fini. 
Remplacer $(M,f)$ par $(M\oplus M',f\oplus 0)$ ne change pas les deux membres 
de (\ref{II:3-3}), et nous ramène au cas où $M$ est libre. Pour $M=A^d$, 
\eqref{II:eq:3-3-1} est une identité algébrique portant sur les coefficients 
$f_i^j$ de $f$. Par le principe de prolongement des identités algébriques, 
il suffit de la vérifier pour $A$ un corps algébriquement clos de 
caractéristique $0$. Les deux membres étant additifs en $M$ dans une 
extension $0\to M'\to M\to M''\to 0$, il suffit même de ne traiter que le cas 
où $M$ est de rang $1$. Si $f$ est la multiplication par $a$, 
\eqref{II:eq:3-3-1} se réduit alors à la formule 
\[
  T\frac{d}{dT}\log\left((1-a T)^{-1}\right) = \sum_{n>0} a^n T^n \text{.}
\]





\begin{corollary_}\label{II:3-4}
Soient $n$ un entier, et $f^{(n)}$ l'endomorphisme 
$(x_1,\dotsc,x_n)\mapsto \left(f(x_n),x_1,\dotsc,x_{n-1}\right)$ de $M^n$. 
On a 
\begin{equation}\label{II:eq:3-4-1}
  \det(1-f T^n) = \det(1-f^{(n)} T) \text{.}
\end{equation}
\end{corollary_}

Procédant comme en (\ref{II:3-3}), on se ramène à supposer que $A$ est de 
caractéristique $0$. Il suffit alors de vérifier que les dérivées 
logarithmiques $-T\frac{d}{dT}\log$ des deux membres sont égales. 
\begin{align*}
  -T\frac{d}{dT}\log\det(1-f T^n) &= - n T^n \frac{d}{dT^n} \log\det(1-f T^n) = n \sum_{m>0} \tr(f^m) T^{n m} \text{,} \\
  -T \frac{d}{dT}\log\det\left(1-f^{(n)} T\right) &= \sum_{m>0} \tr\left(f^{(n) m}\right) T^m \text{,}
\end{align*}
et on observe que 
\begin{align*}
  \tr(f^{(n) m}) &= 0 \qquad \text{si $n\nmid m$} \\
  \tr(f^{(n) n m}) &= n \tr(f^m) \text{.}
\end{align*}





\subsection{Remarque}\label{II:3-5}

Pour $A=\dQ_\ell$, \eqref{II:eq:3-4-1} est le cas particulier 
$X_0=\spec(\dF_{q^n})$ de (\ref{II:3-1}). Utilisant cette identité, il est 
facile de vérifier directement que le second membre de (\ref{II:3-1}) est 
indépendant de $q$, et du morphisme structural $X_0\to\spec(\dF_q)$. 





\subsection{Remarque}\label{II:3-6}

Si $j:X_0\hookrightarrow \bar X_0$ est une compactification de $X_0$, on a 
$L(X_0,\cF_0) = L(\bar X_0, j_!\cF_0)$ et 
$\h_c^i(X,\cF) = \h^i(\bar X,j_!\cF)$. Ceci ramène (\ref{II:3-1}) au cas 
propre, et explique l'usage de la cohomologie à supports propres. Pour $X$ 
propre, (\ref{II:3-2}) a la forme d'une formule des traces de Lefschetz. On 
notera que les termes locaux $\tr(F^{n *},\cF_x)$ ($x\in X^{F^n}$) ne 
dépendent que de la fibre de $\cF$ en $x$, et ne sont pas affectés de 
multiplicité. Pour $X$ et $\cF$ lisses, cela correspond au fait que le graphe 
de $F$ est transverse à la diagonale (puisque $d F = 0$). 





\subsection{Remarque}\label{II:3-7}

Si on admet le formalisme de $\dQ_\ell$-faisceaux (cf. \ref{II:2-11}; il faut 
disposer de la suite spectrale de Leray et du théorème de changement de 
base pour les images directes supérieures $\R^q f_!$), il est facile de 
ramener la preuve de (\ref{II:3-1}, \ref{II:3-2}), au cas où $X_0$ est 
une courbe lisse et où $\cF_0$ est lisse. Ceci est clairement expliqué 
dans \cite[\S 5]{gr95} (pour \ref{II:3-1}; \ref{II:3-2} se traite de même). 





\subsection{}\label{II:3-8}

On dispose de deux méthodes pour prouver (\ref{II:3-2}). 


\paragraph{Lefschetz-Verdier}
Si $X_0$ est propre (cf. \ref{II:3-6}), la formule des traces générale de 
Lefschetz-Verdier permet d'exprimer le second membre de (\ref{II:3-2}) comme 
une somme de termes locaux, un pour chaque point de $X^{F^n}$. Dans la version 
originale de \cite{sga5}, cette formule n'était prouvée que modulo la 
résolution des singularités. Le lecteur trouvera une preuve inconditionnelle 
dans la version définitive. Dans le cas des courbes, cas auquel on peut se 
réduire (\ref{II:3-7}), les ingrédients de la démonstration étaient 
d'ailleurs tous disponibles. 

Pour déduire (\ref{II:3-2}) de la formule de Lefschetz-Verdier, il faut 
pouvoir en calculer les termes locaux. Pour une courbe et l'endomorphisme de 
Frobenius, cela avait été fait par Artin et Verdier \cite{ve67}. 


\paragraph{Nielson-Wecken}
Un méthode inspirée dex travaux de Nielsen et Wecken permet de ramener 
(\ref{II:3-2}), à un cas particulier prouvé par Weil; c'est ce qui sera 
expliqué dans les paragraphes suivants. 










\section{Réduction à des théorèmes en \texorpdfstring{$\dZ/\ell^n$}{Z/l n}-cohomologie}\label{II:4}

Nous prouverons (\ref{II:3-2}) par passage à la limite à partir d'une 
théorème analogue en $\dZ/\ell^n$-cohomologie. L'énoncer présente la 
difficulté suivante. Soient $X_0$ un schéma séparé de type fini sur 
$\dF_q$, et $\cF_0$ un faisceau plat et constructible de $\dZ/\ell^n$-modules. 
Ses fibres sont des $\dZ/\ell^n$-modules libres de type fini, et le membre de 
gauche de (\ref{II:3-2}) a donc un sens (c'est un élément de $\dZ/\ell^n$). 
Par contre, les $\h_c^i(X,\cF)$ ne seront en général pas libres, de sorte 
que le membre de droite n'est pas défini. Obvier à cette difficulté est 
l'un des usages essentiels faits par Grothendieck de la théorie des 
catégories dérivées.





\subsection{}\label{II:4-1}

Pour la définition de la catégorie dérivée $\D(\sA)$ d'une catégorie 
abélienne $\sA$, et celle des sous-catégorie $\D^+(\sA)$, $\D^-(\sA)$, 
$\D^b(\sA)$, je renvoie au texte de Verdier dans ce volume. Pour $\Lambda$ 
un anneau, on écrit $\D(\Lambda)$ au lieu de 
$\D(\text{catégorie des $\Lambda$-modules à gauche})$; pour $X$ un site, on 
écrit $\D(X,\Lambda)$ au lieu de 
$\D($catégorie des faisceaux de $\Lambda$-modules à gauche$)$. De 
même pour $\D^-$, etc\ldots

Le signe $\R$ (resp. $\mathsf{L}$) indique un foncteur dérivé droit (resp. 
gauche). Par exemple, $\lotimes$ indique le foncteur dérivé du produit 
tensoriel: pour $K$ dans $\D^-(\Lambda^\circ)$ et $L$ dans $\D^-(\Lambda)$ 
($\Lambda^\circ$ anneau opposé à $\Lambda$), $K\lotimes L$ se calcule 
comme suit: on prend des quasi-isomorphismes $K'\to K$, $L'\to L$, avec 
$K'$ et $L$ bornés supérieurement et l'un d'eux à composantes plates, et 
le simple complexe $s(K'\otimes_\Lambda L)$ associé au complexe double 
$K'\otimes_\Lambda L'$. 





\subsection{}\label{II:4-2}

Même pour traiter le seul cas de la $\dZ/\ell^n$-cohomologie, nous aurons 
besoin de la théorie des traces d'endomorphismes de modules sur des anneaux 
non nécessairement commutatifs (des algèbres de groupes finis 
$\dZ/\ell^n[G]$). Ces traces ont été introduits par Stallings et Hattori 
\cite{ba76}. 

Soit $\Lambda$ un anneau (à unité, mais non nécessairement commutatif). 
On par $\Lambda^\natural$ le quotient du groupe additif de $\Lambda$ par le 
sous-groupe engendré par les commutateurs $a b - b a$. Si $f$ est un 
endomorphisme du $\Lambda$-module à gauche libre $\Lambda^n$, de matrice 
$f_i^j$, on note $\tr(f)$ l'image de $\sum f_i^f$ dans $\Lambda^\natural$. 
Si $f$ et $g$ sont deux homomorphismes 
$f:\Lambda^n \leftrightarrows \Lambda^m : g$, on vérifie aussitôt que 
$\tr(f g) = \tr(g f)$. 

Soit $f$ un endomorphisme d'un $\Lambda$-module projectif de type fini $P$. 
Écrivons $P$ comme facteur direct d'une module libre $\Lambda^n$: cela signifie 
choisir un module $P'$, et un isomorphisme $\alpha:P\oplus P'\iso \Lambda^n$, 
soit encore choisir un homomorphisme $a:P\to \Lambda^n$, et une rétraction 
$b:\Lambda^n\to P$ (prendre $P=\ker(b)$). Soit 
$f'=\alpha(f\oplus 0)\alpha^{-1} =af b$. On pose $\tr(f) = \tr(f')$. Cet 
élément de $\Lambda^\natural$ ne dépend que de $f$: si un autre choix 
$c:P\leftrightarrows \Lambda^m:d$ est fait, on a $a=a(dc)(ba)$, d'où 
\[
 \tr(a f b) = \tr(adcbafb) = \tr(cbafbad) = \tr(cfd) \text{.}
\]

Si $f$ est un endomorphisme d'un $\Lambda$-module projectif de type fini 
$\dZ/2$-gradué, de composantes $f_i^j:P^j\to P^i$, on pose 
\[
  \tr(f) = \tr(f_0^0) - \tr(f_1^1) \text{.}
\]
S'il y a risque d'ambiguïté, on écrira plutôt $\tr(f;P)$. Si les 
homomorphismes $f:P\leftrightarrows Q:g$ sont homogènes, on vérifie 
aussitôt par réduction au cas libre que 
\begin{equation}\label{II:eq:4-2-1}
  \tr(f g) = \tr\left((-1)^{\deg f \deg g} g f\right) \text{.}
\end{equation}

Si $f$ est un endomorphisme d'un $\Lambda$-module $\dZ/2$-gradué filtré 
$(P,F)$, de filtration finie, compatible à la graduation, et que les 
$\gr_F^i(P)$ sont projectifs de type fini, alors $P$ est projectif de type 
fini, et 
\begin{equation}\label{II:eq:4-2-2}
  \tr(f;P) = \sum \tr\left(f; \gr_F^i(P) \right)
\end{equation}
(scinder la filtration pour se ramener au cas d'une somme directe). 





\subsection{}\label{II:4-3}

Un module $\dZ$-gradué sera considéré comme $\dZ/2$-gradué par la 
parité du degré. En particulier, si $f$ est un endomorphisme d'un complexe 
borné de $\Lambda$-modules projectifs de type fini, on pose 
\begin{equation}\label{II:eq:4-3-1}
  \tr(f) = \sum (-1)^i \tr(f^i) \text{.}
\end{equation}

D'après \eqref{II:eq:4-2-1}, si $f$ est homotope à zéro, $\tr(f) = 0$: 
\begin{equation}\label{II:eq:4-3-2}
  f = d H + h D \Rightarrow \tr(f) = 0 \text{.}
\end{equation}

Soit $\mathsf{K}_\text{parf}(\Lambda)$ la catégorie d'objets les complexes 
bornés de $\Lambda$-modules projectifs de type fini, et de flèches les 
morphismes de complexes pris à homotopie près. Le foncteur 
$\mathsf{K}_\text{parf}(\Lambda)\to\D(\Lambda)$ est pleinement fidèle. On note 
$\mathsf{D}_\text{parf}(\Lambda)$ son image essentielle. Les formules 
\eqref{II:eq:4-3-1} \eqref{II:eq:4-3-2} permettent de définir $\tr(f)$ pour $f$ 
un endomorphisme de $K\in\ob \mathsf{D}_\text{parf}(\Lambda)$ (et cette trace ne 
dépend que de la classe d'isomorphe de $(K,f)$). 





\subsection{}\label{II:4-4}

Rappelons que la catégorie dérivée filtrée $\mathsf{DF}(\sA)$ d'une catégorie 
abélienne $\sA$ est la catégorie déduite de celle des complexes d'objets 
filtrés de filtration finie de $\sA$ en inversant les quasi-isomorphismes 
filtrés ($=$ morphismes de complexes filtrés induisant un quasi-isomorphisme 
sur le gradué associé). On dispose de foncteurs ``complexe sous-jacent'': 
$\mathsf{DF}(\sA)\to\D(\sA): (K,F)\mapsto K$ et ``$p$-ième composante du 
gradué'': $\mathsf{DF}(\sA)\to\D(\sA): (K,F)\mapsto \gr_F^p(K)$. 

Soit $\mathsf{KF}_\text{parf}(\Lambda)$ la catégorie d'objets les complexes 
bornés filtrés de filtration finie $(K,F)$ avec des $\gr_F^p(K^q)$ projectifs 
de type fini, et de flèches les morphismes de complexes respectant la 
filtration, pris à une homotopie respectant la filtration près. Le foncteur 
$\mathsf{KF}_\text{parf}(\Lambda)\to\mathsf{DF}(\Lambda)$ est pleinement fidèle; 
son image essentielle $\mathsf{DF}_\text{parf}(\Lambda)$ est presque tous nuls; 
si $(K,F)$ est dans $\mathsf{D}_\text{parf}(\Lambda)$, alors $K$ est dans 
$\mathsf{D}_\text{parf}(\Lambda)$. 

D'après \eqref{II:eq:4-2-2}, si $f$ est un endomorphisme de 
$(K,F)\in\ob\mathsf{DF}_\text{parf}(\Lambda)$, on a 
\begin{equation}\label{II:eq:4-4-1}
 \tr(f,K) = \sum_p \tr(f,\gr_F^p(K)) \text{.}
\end{equation}





\subsection{}\label{II:4-5}

Soient $\Lambda$ un anneau, et $K\in\ob\D^-(\Lambda)$. On dit que $K$ est de 
tor-dimension $\leqslant r$ si pour tout $\Lambda$-module à droite $N$, les 
hypertors $\h^i(N\lotimes_\Lambda K)$ sont nuls pour $i<-r$. Il est dit de 
tor-dimension finie si cette condition est vérifiée pour un $r$. 

On emploie la même terminologie pour un complexe de faisceaux 
$K\in\ob\D^-(X,\Lambda)$. 





\begin{lemma}\label{II:4-5-1} 
Soient $\Lambda$ un anneau noethérien à gauche, et $K\in\ob\D^-(\Lambda)$. 
Pour que $K$ soit dans $\mathsf{DF}_\text{parf}(\Lambda)$, il faut et il suffit 
qu'il soit de tor-dimension finie et que les $\h^i(K)$ soient de type fini.
\end{lemma}

Quitte à remplacer $K$ par un complexe isomorphe (dans $\D^-(\Lambda)$), on 
peut supposer $K$ borné supérieurement et à composantes libres de type 
fini (cf. \ref{II:4-7}, ci-dessous). Si $K$ est de tor-dimension $\leqslant r$, 
les $\h^i(K)$ sont nuls pour $i<-r$, $K^{-r}/\im(d)$ admet la résolution 
libre 
\[
  (\cdots \to K^{-r-1} \to K^{-r}) \to K^{-r}/\im(d) \text{,}
\]
et pour tout $\Lambda$-module à droite $N$, 
$\h^{-r-k}(N\lotimes_\Lambda K) = \h^{-r-k}(N\otimes_\Lambda K) 
= \tor_k(N,K^r/\im d)$ pour $k\geqslant 1$. 

Par hypothèse, ces groupes sont nuls; le module $K^r/\im d$ est donc plat de 
présentation finie, i.e. projectif de type fini. Le quotient 
\[
  0 \to K^r/\im(d) \to K^{r+1} \to K^{r+2}\to \cdots 
\]
de $K$ est quasi-isomorphe à $K$, et ceci montre que 
$K\in\ob\mathsf{DF}_\text{parf}(\Lambda)$. 





\begin{prop-def_}\label{II:4-6}
Soient $X$ un schéma noethérien, et $\Lambda$ un anneau noethérien à 
gauche. On note $\D_\textnormal{ctf}^b(X,\Lambda)$ la sous-catégorie de 
$\D^-(X,\Lambda)$ formée des complexes $\cK$ vérifient les conditions 
équivalentes suivantes:
\begin{enumerate}[\indent i)]
  \item $\cK$ est isomorphe (dans $\D^-(X,\Lambda)$) à un complexe borné e 
    faisceaux de $\Lambda$-modules plats et constructibles. 
  \item $\cK$ est de tor-dimension finie et les faisceaux $\h^i(K)$ sont 
    constructibles. 
\end{enumerate}
\end{prop-def_}

Même argument qu'en (\ref{II:4-5}), à partir du lemme suivant. 





\begin{lemma_}\label{II:4-7}
Si un complexe $\cK$ de faisceaux de $\Lambda$-modules est tel que les 
$\h^i(\cK))$ soient constructibles, et nuls pour $i$ grand, il existe un 
quasi-isomorphisme $\cK'\to \cK$, avec $\cK'$ borné supérieurement à 
composantes constructibles et plates.
\end{lemma_}

On construit $\cK'$ en se propageant de droite à gauche. Pour un $m$ tel que les 
$\h^i(\cK)$ ($i\geqslant m$) soient nuls, on prend ${\cK'}^i = 0$ pour 
$i\geqslant m$. Supposons ensuite déjà construits les ${\cK'}^i$ pour 
$i>n$:
\[\xymatrix{
  \cdots \ar[r] 
    & \cK^{n-1} \ar[r] 
    & \cK^n \ar[r] 
    & \cK^{n+1} \ar[r] 
    & \cdots \\
  & & & {\cK'}^{n+1} \ar[u] \ar[r] 
    & \cdots \text{,}
}\]
avec $\h^i(\cK')\iso \h^i(\cK)$ pour $i>n+1$, et $\ker(d)\to\h^{n+1}(\cK)$ surjectif 
en degré $n+1$. Définissons des faisceaux $\cA$ et $\cB$ par le diagramme 
cartésien 
\[\xymatrix{
  \cK^n \ar[r] 
    & \cK^n/\im(d) \ar[r] 
    & \ker(d) \\
  \cA \ar[u] \ar[r]^u 
    & \cB \ar[u]\ar[r] 
    & \ker(d) \ar[u] \text{.}
}\]

Alors, $\cB$ est constructible, $u$ est surjectif, et pour avancer d'un pas, il 
suffit de construire $K^n$ plat et constructible et $v:\cK^n\to \cA$ tel que u
$u v$ soit surjectif. Que ce soit possible est garanti par le lemme suivant. 





\begin{lemma_}\label{II:4-8}
Soit $v:\cA\to \cB$ un épimorphisme de faisceaux de $\Lambda$-modules, avec 
$\cB$ constructible. Il existe alors $u:\cK\to \cA$ avec $\cK$ plat et 
constructible et $v u$ un épimorphisme. 
\end{lemma_}

Le faisceau $\cA$ est quotient d'une somme de faisceaux de la forme 
$\varphi_! \Lambda$, pour $\varphi:U\to X$ étale. Puisque $\cB$ est noethérien 
(\cite[IX.2.10]{sga4}), une somme finie de ceux-ci s'envoie déjà sur $\cB$, 
d'où le lemme. 





\begin{theorem_}\label{II:4-9}
Soit $f:X\to Y$ un morphisme séparé de type fini de schémas noethériens. 
Si $\cK\in\ob\D_\textnormal{ctf}^b(X,\Lambda)$, alors 
$\R f_!\cK\in \ob\D_\textnormal{ctf}^b(Y,\Lambda)$. 
\end{theorem_}

Rappelons que si $j:X\hookrightarrow \bar X \xrightarrow{\bar f} Y$ est une 
compactification relative de $X/Y$, on a par définition 
$\R f_! = \R \bar f_* \circ j_!$, où $\R \bar f_*$ est le foncteur dérivé 
du foncteur $\bar f_*$. Cette formule ramène la preuve de (\ref{II:4-9}) 
au cas où $f$ est propre. Pour $f$ propre, $f_*$ est de dimension 
cohomologique finie;; ceci permet de définir $\R f_*$ sur la catégorie 
dérivée tout entière, et aussi comme un foncteur 
\[
  \R f_* : \D^-(X,\Lambda) \to \D^-(Y,\Lambda) \text{.}
\]
Par composition, on définit de même en général 
\[
  \R f_! : \D^-(X,\Lambda) \to \D^-(Y,\Lambda) \text{.}
\]

Pour tout $\Lambda$-module à droite $N$, on a (preuve donnée ci-dessous) 
\begin{equation}\label{II:eq:4-9-1}
  N\lotimes \R f_! \cK \iso \R f_! (N\lotimes \cK) \text{.}
\end{equation}

Déduisons (\ref{II:4-9}) de \eqref{II:eq:4-9-1}. 
\begin{enumerate}[\indent a)]
  \item La suite spectrale 
    \[
      E_2^{p q} = \R^p f_! \h^q(\cK) \Rightarrow \h^{p+q} \R f_! \cK \text{.}
    \]
    et le théorème de finitude pour $\R f_!$, montrent que 
    $\R f_!\cK\in \ob \D^b(X,\Lambda)$ et est à cohomologie constructible. 
    Reste à prouver la tor-dimension finie (\ref{II:4-6}). 
  \item Si $\cK$ est de tor-dimension $\leqslant -r$, il en est de même pour 
    $\R f_! \cK$: on a $\h^i(N\lotimes \R f_! \cK) = \h^i(\R f_!(N\lotimes \cK))$, 
    et, dans la suite spectrale ci-dessus pour $N\lotimes \cK$, les 
    $E_2^{p q}$ sont nuls pour $q<r-s$ fortiori pour $p+q<r$. 
\end{enumerate}

Prouvons \eqref{II:eq:4-9-1}, en restant au niveau des complexes. 
\begin{enumerate}[\indent a)]
  \item Utilisant la formule de définition $\R f_! = \R f_*\circ j_!$, on se 
    ramène à supposer $f$ propre. 
  \item Représentons $\cK$ par un complexe borné 
    supérieurement à composantes acyclique pour $f_*:\R^p f_* \cK^q = 0$ 
    pour $p>0$. On a alors $\R f_* K \sim f_* \cK$. Il n'est possible de 
    travailler ainsi avec des complexes bornés supérieurement que parce que 
    $f_*$ est de dimension cohomologique finie. 
  \item Si $L$ un $\Lambda$-module à droite libre, on a 
    $\R^p f_*(L\otimes \cK^q) = L\otimes \R^p f_* \cK^q$: c'est trivial pour 
    $\Lambda$ libre de type fini, et le cas général s'en déduit par 
    passage à la limite. Le faisceau $L\otimes \cK^q$ est acyclique pour 
    $f_*$, et $f_*(L\otimes \cK^q) = L\otimes f_* \cK^q$. 
  \item Soit $N_\bullet$ un résolution libre de $N$. On a 
    $N\lotimes \cK\sim s(N_\bullet\otimes \cK)$ (complexe simple associé au 
    complexe double $N_\bullet\otimes \cK$). D'après c), 
    \[
      \R f_*(N\lotimes \cK) \sim \R f_* (s(N_\bullet\otimes \cK)) \sim f_*(s(N_\bullet\otimes \cK)) \sim s(N_\bullet\otimes f_* \cK) \sim N\lotimes \R f_* \cK \text{,}
    \]
    soit la formule annoncée. 
\end{enumerate}

Lorsque $Y$ est le spectre d'un corps séparablement clos, le foncteur $\R f_!$ 
s'identifie à un foncteur noté 
$\R \Gamma_c : \D_\text{ctf}^b(X,\Lambda)\to\D_\text{parf}(\Lambda)$. 





\begin{theorem_}\label{II:4-10}
Avec les notations du \S\ref{II:1}, soient $X_0$ un schéma séparé de type 
fini sur $\dF_q$, et $\Lambda$ un anneau noethérien de torsion, tué par un 
entier premier à $p$. Soit $\cK_0\in\ob\D_\textnormal{ctf}^b(X_0,\Lambda)$. 
Avec la définition (\ref{II:4-3}) de traces, on a pour tout $N$ 
\begin{equation}\label{II:eq:4-10}
  \sum_{x\in X^{F^n}} \tr\left({F^n}^*,\cK_x\right) = \tr\left({F^n}^*,\R\Gamma_c(X,\cK)\right) \text{.}
\end{equation}
\end{theorem_}

Dans la fin de ce \S, nous allons déduire \ref{II:3-2} de \ref{II:4-10}, et 
prouver \ref{II:2-10}. 





\subsection{Preuve de \texorpdfstring{\ref{II:2-10}}{2.10}}\label{II:4-11}

Soit $X$ un schéma séparé de type fini sur un corps algébriquement clos 
$k$, et soit $\cG$ un $\dQ_\ell$-faisceau sur $X$. D'après \ref{II:2-8} on 
peut l'écrire sous la forme $\cG=\cF\otimes\dQ_\ell$, avec $\cF$ un 
$\dZ_\ell$-faisceau sans torsion. Posons 
\[
  K_n = \R \Gamma_c(X,\cF\otimes\dZ/\ell^{n+1})\in\D(\dZ/\ell^{n+1}) \text{.}
\]
D'après \ref{II:4-9}, on a $K_n\in\ob\D_\text{parf}(\dZ/\ell^{n+1})$, et il 
résulte de \eqref{II:eq:4-9-1} que la flèche naturelle 
\begin{equation}\label{II:eq:4-11-1}
\xymatrix{
  K_n\lotimes_{\dZ/\ell^{n+1}} \dZ/\ell^n \ar[r] & K_{n+1} \text{,}
}
\end{equation}
est un isomorphisme. Le problème de la définition de cette flèche est 
discuté en \ref{II:4-12}. 

On peut maintenant invoquer \cite[XI.3.3]{sga5} (p. 31, 1-2 à la fin; ce 
passage est indépendant du reste de \cite{sga5}) pour réaliser chaque 
$K_n$ par un complexe borné de $\dZ/\ell^{n+1}$-modules libres de type fini, 
et les flèches \eqref{II:eq:4-11-1} par des \emph{isomorphismes de complexes} 
\[\xymatrix{
  K_n\otimes \dZ/\ell^n \ar[r]^-\sim 
    & K_{n-1} \text{.}
}\]
Soit $K$ le complexe $\varprojlim K_n$. C'est un complexe borné de 
$\dZ_\ell$-modules libres, et $K_n\simeq K\otimes_{\dZ_\ell}\dZ/\ell^{n+1}$. On 
a 
\[
  \h^i(K) = \varprojlim \h^i(K_n) \text{,}
\]
(tout système projectif de groupes finis vérifie en effet la condition de 
Mittag-Leffler), et $\h^i(K)$ est un $\dZ_\ell$-module de type fini. On a 
\[
  \h_c^i(X,\cG) = \left(\varprojlim \h^i(K_n) \right)\otimes_{\dZ_\ell} \dQ_\ell \text{:}
\]
c'est un $\dQ_\ell$-espace vectoriel de dimension finie. 





\subsection{La flèche \texorpdfstring{\eqref{II:eq:4-11-1}}{(4.11.1)}}\label{II:4-12}

Pour \emph{définir} la flèche \eqref{II:eq:4-11-1}, on ne peut pas 
procéder comme en \ref{II:4-9}, i.e. résoudre $\dZ/\ell^n$ par un complexe 
de $\dZ/\ell^{n+1}$-modules libres, puisqu'on veut construire on isomorphisme 
dans $\D_\text{parf}(\dZ/\ell^n)$. La flèche voulue est un cas particulier de 
la flèche d'extension des scalaires suivante. 

(*) Soit $\Lambda\to\Lambda'$ un homomorphisme d'anneaux noethériens de 
torsion. Pour $X$ un schéma séparé de type fini sur $k$ algébriquement 
clos, et $\cK\in\ob\D_\text{ctf}(X,\Lambda)$, on a un isomorphisme 
\[\xymatrix{
  \R\Gamma_c(X,\cK)\lotimes_\Lambda \Lambda' \ar[r]^-\sim & \R \Gamma_c(X,\cK\lotimes_\Lambda \Lambda') \text{.}
}\]

Des flèches plus générales sont construits dans \cite[XVII 4.2.12]{sga4}. La 
m\'
ethode est la suivante: 
\begin{enumerate}[\indent a)]
  \item La définition de $\R\Gamma_c$ est la formule 
    $j_!(\cK\otimes_\Lambda\Lambda') = (j_! \cK)\otimes_\Lambda\Lambda'$ pour une 
    immersion ouverte nous ramènent au cas où $X$ est propre. 
  \item Pour $X$ propre, il s'agit de remplacer $\cK$ par un complexe borné à 
    composantes acycliques pour le foncteur $\Gamma$, et de fibre en tout point 
    géométrique (ou seulement un tout point fermé) homotope à un 
    complexe de $\Lambda$-modules plats. Pour un tel complexe, on a 
    $\Gamma(X,\cK)\sim \R\Gamma(X,\cK)$ et 
    $K\otimes_\Lambda\Lambda'\sim \cK\lotimes_\Lambda\Lambda'$; la flèche 
    \eqref{II:eq:4-11-1} est définie comme la composition 
    \[
      \R\Gamma(X,\cK)\lotimes_\Lambda \Lambda' \to \Gamma(X,\cK)\otimes_\Lambda\Lambda' \to \Gamma(X,\cK\otimes_\Lambda\Lambda') \xleftarrow\sim \R\Gamma(X,\cK\lotimes_\Lambda\Lambda') \text{.}
    \]
  \item Il reste à montrer qu'il est possible de remplacer $\cK$ par un 
    complexe du type voulu \cite[XVII.4.2.10]{sga4}. On remplace d'abord $\cK$ par 
    un complexe borné à composantes plates (\ref{II:4-6}), puis on en prend 
    une résolution flasque canonique tronquée (ceci ne modifie pas le type 
    d'homotopie des complexes fibres, et on tronque assez loin pour que 
    vérifiée la condition d'acyclicité pour $\Gamma$ (i.e. au-delà 
    de la dimension cohomologique de $\Gamma$)). 
\end{enumerate}





\subsection{Prouve de \texorpdfstring{(\ref{II:3-2})}{(3.2)}}\label{II:4-13}

Pour simplifier les notations, nous ne traiterons que la cas $n=1$ de 
(\ref{II:3-2}). Le cas général s'en déduit en remplaçant ci-dessous 
$F$ par $F^n$. 

Soient donc $X_0$ séparé de type fini sur $\dF_q$, et $\cG_0$ un 
$\dQ_\ell$-faisceau sur $X_0$. On a $\cG_0=\cF_0\otimes\dQ_\ell$, pour 
$\dZ_\ell$-faisceau sans torsion convenable $\cF_0$. Soit 
$K_n=\R\Gamma_c(X,\cF\otimes\dZ/\ell^n)$. Les morphismes de Frobenius induisent 
des morphismes 
\[
  F^* : K_n\to K_n \qquad \text{(dans $\D_\text{parf}(\dZ/\ell^{n+1})$),}
\]
qui se déduisent les uns des autres via les isomorphismes 
\eqref{II:eq:4-11-1}. 

Réalisons comme en (\ref{II:4-11}) les $K_n$ et les isomorphismes 
\eqref{II:eq:4-11-1} su niveau des complexes. La passage déjà cité de 
\cite[XV.3.3]{sga5} permet de réaliser les morphismes $F^*$ par des 
endomorphismes de complexes, encore notés $F^*$, qui se réduisent les uns 
sur les autres. On note encore $F^*$ leur limite projective, et les 
endomorphismes qui s'en déduisent. Puisque 
$\h_c^i(X,\cG) = \h^i(K)\otimes\dQ_\ell = \h^i(K\otimes\dQ_\ell)$, on a (avec 
la convention de signe de (?.1) 
\[
  \tr\left(F^*, \h_c^\bullet(X,\cG)\right) = \tr\left(F^*,K^\bullet\otimes\dQ_\ell\right) = \varprojlim_n \tr\left(F^*,K_n^\bullet\right) = \varprojlim_n \tr\left(F^*,\R\Gamma_c(X,\cF\otimes\dZ/\ell^{n+1})\right) \text{.}
\]
Appliquons (\ref{II:4-10}) à $\cF_0\otimes\dZ/\ell^{n+1}$ (vu comme complexe 
réduit su degré $0$), pour $\Lambda=\dZ/\ell^{n+1}$. On trouve que 
\begin{align*}
  \tr\left(F^*,\R\Gamma_c(X,\cF\otimes\dZ/\ell^{n+1})\right) 
    &= \sum_{x\in X^F} \tr\left(F^*,\cF_x\otimes\dZ/\ell^{n+1}\right) \\
    &= \sum_{x\in X^F} \tr\left(F^*,\cG_x\right) \mod \ell^{n+1} \text{,}
\end{align*}
et (\ref{II:3-2}), en résulte par passage à limite. 










\section{La méthode de Nielsen-Wecken}\label{II:5}

Dans ce chapitre, nous prouvons deux cas particuliers de (\ref{II:4-10}). Le 
cas général sera considéré au chapitre suivant. 





\begin{lemma_}\label{II:5-1}
La théorème (\ref{II:4-10}) est vrai pour $X_0$ de dimension $0$. 
\end{lemma_}

En dimension $0$, la formule \eqref{II:eq:4-10} vaut pour toute 
correspondance, et non seulement pour les itérés d'un Frobenius. Elle est 
un cas particulier de l'énoncé suivent (appliqué à $X(\dF)$, $K$ et 
aux ${F^*}^n$). 

(*) Soit $X$ un ensemble fini, $F:X\to X$, $\cK\in\ob\D_\text{parf}(X,\Lambda)$ 
et $F^*:F^* \cK\to \cK$. On a 
\[
  \sum_{X^F} \tr\left(F^*,\cK_x\right) = \tr\left(F^*,\Gamma(X,K)\right) \text{.}
\]

On a $\Gamma(X,\cK)=\bigoplus_{x\in X} \cK_x$, et la formule résulte de ce que 
pour tout morphisme $u:\bigoplus_{x\in X} \cK_x\to \bigoplus_{x\in X} \cK_x$, de 
matrice $u_x^y$, on a $\tr(u)=\sum_{x\in X} \tr\left(u_x^x\right)$. 





\begin{lemma_}\label{II:5-2}
Soient $\Lambda$ un anneau noethérien de torsion, tué par une puissance du 
nombre premier $\ell\ne p$, $X_0$ une courbe projective, lisse et connexe sur 
$\dF_q$, $j:U_0\hookrightarrow X_0$ un ouvert dense de $X_0$ et $\cG_0$ un 
faisceau localement constant de $\Lambda$-modules projectifs sur $U_0$. On a 
\[
  \sum_{x\in U^F} \tr\left(F^*,\cG_x\right) = \tr\left(F^*,\R\Gamma_c(U,\cG)\right) \text{.}
\]
\end{lemma_}

Plus précisément, nous déduirons (\ref{II:5-2}) du 





\begin{theorem_}\label{II:5-3}
Soient $X$ une courbe projective non singulière sur un corps algébriquement 
clos $k$, $f$ un endomorphisme à points fixes isolés de $X$ et $v(f)$ le 
nombre de points fixes de $f$, chacun étant compté avec sa multiplicité: 
$v(f)$ est la nombre d'intersection du graphe de $f$ avec la diagonale de 
$X\times X$. Alors, pour $\ell$ premier à l'exposant caractéristique de $k$, 
on a 
\[
  v(f) = \sum (-1)^i \tr\left(f^*,\h^i(X,\dQ_\ell)\right) \text{.}
\]
\end{theorem_}

Ce théorème est dû à Weil (\cite{we48}, n$^\circ$ 43, 66 et 68) et la 
démonstration de Weil est reproduite dans Lang \cite{la83} (VI \S3 combiné 
avec VII \S2 th.3). On peut aussi le déduire du formalisme de la classe 
cohomologie associée à un cycle (\nameref{IV}, \ref{IV:3-7}). 





\begin{corollary_}\label{II:5-4}
Soit $j:U\hookrightarrow X$ un ouvert dense de $X$, tel que $f^{-1}(U)=U$. Si 
les points fixes de $f$ dans $X\setminus U$ sont de multiplicité un, alors le 
nombre $v_U(f)$ de points fixes de $f$ dans $U$ (chacun compté avec sa 
multiplicité) est 
\[
  v_U(f) = \sum (-1)^i \tr\left(f^*,\h_c^i(U,\dQ_\ell)\right) \text{.}
\]
\end{corollary_}

Soit $F$ l'ensemble fini $X\setminus U$. La suite exacte de cohomologie 
définie par la suite exacte court 
$0\to j_! \dQ_\ell\to\dQ_\ell\to{\dQ_\ell}_F\to 0$: 
\[\xymatrix{
  \ar[r]^-\partial 
    & \h_c^i(U,\dQ_\ell) \ar[r] 
    & \h^i(X,\dQ_\ell) \ar[r] 
    & \h^i(F,\dQ_\ell) \ar[r]^-\partial 
    & 
}\]
fournit, avec la convention de signe de (\ref{II:3-1}). 
\[
  \tr\left(f^*,\h_c^\bullet(U,\dQ_\ell)\right) 
    = \tr\left(f^*,\h^\bullet(X,\dQ_\ell)\right) - \tr\left(f^*,\h^\bullet(F,\dQ_\ell)\right) 
    = v(f) - \tr(f^*,\dQ_\ell^F) \text{.}
\]
La trace de $f^*$ sur $\dQ_\ell^F$ est le nombre $v_F(f)$ de points fixes de 
$f$ dans $F$; l'hypothèse de multiplicité un assure donc que 
\[
  \tr\left(f^*,\h_c^\bullet(U,\dQ_\ell)\right) 
    = v_F(f) - v_U(f) \text{,}
\]
comme promis. 

La preuve de (\ref{II:5-2}) utilise quelques lemmes que nous allons 
développer maintenant sur un cas particulier des traces non commutative 
(\ref{II:4-2}). \emph{Pour abréger, nous dirons projectif pour projectif de 
type fini}. 





\subsection{}\label{II:5-5}

Soient $\Lambda$ un anneau, et $H$ un groupe fini. L'application 
$\sum a_h \cdot a\mapsto $ classe de $a_e$, de l'algèbre de groupe 
$\Lambda[H]$ dans $\Lambda^\natural$, s'annule sur les commutateurs $ab-ba$ de 
$\Lambda[H]$. Elle se factorise donc par 
\[
  \varepsilon : \Lambda[H]^\natural \to \Lambda^\natural \text{.}
\]
Si $F$ est un endomorphisme d'un $\Lambda[H]$-module projectif $P$, on pose 
\[
  \tr_\Lambda^H(F) = \varepsilon \tr_{\Lambda[H]} (F) 
\]
où au second membre figure la trace (\ref{II:4-2}). Pour éviter des 
ambiguïtés, on écrira parfois cette trace $\tr_\Lambda^H(F,P)$. 





\begin{proposition_}\label{II:5-6}
$\tr_\Lambda(F) = |H|\tr_\Lambda^H(F)$. 
\end{proposition_}

Les propriétés formelles des traces nous ramènent à ne traiter que le 
cas où $P=\Lambda[H]$. L'endomorphisme $F$ est alors le multiplication à 
droite par un élément $\sum a_h\cdot h$ de $\Lambda[H]$, et 
$\tr_\Lambda(F) = |H| a_e$, $\tr_\Lambda^H(F) = a_e$. 





\subsection{}\label{II:5-7}

Soient $A$ un anneau commutatif et $\Lambda$ un $A$-algèbre. La multiplication 
par un élément de $A$ passe au quotient pour définir une structure de 
$A$-module sur $A^\natural$. Si $H$ est un groupe fini, $P$ un $A[H]$-module 
projectif et $M$ un $\Lambda[H]$-module, projectif en tant que $\Lambda$-module, 
on sait que le $\Lambda$-module $P\otimes_A H$, muni de l'action diagonale de 
$H$, est un $\Lambda[H]$-module projectif. En effet: 
\begin{enumerate}[\indent a)]
  \item I suffit de le vérifier par $P=A[H]$. 
  \item Le $\Lambda$-module $P\otimes_A M$, muni de l'action de $H$ définie 
    par son action sur le premier facteur, est clairement un 
    $\Lambda[H]$-module projectif; notions le $(P\otimes_A M)'$. 
  \item L'application $h\otimes m\mapsto h\otimes h^{-1} m$ induit un 
    isomorphisme 
    \begin{equation}\label{II:eq:5-7-1}
      A[H]\otimes_A M \iso (A[H]\otimes_A M)' \text{.}
    \end{equation}
\end{enumerate}





\begin{proposition_}\label{II:5-8}
Avec les notations précédentes, si $u$ est un endomorphisme de $F$ et 
$v$ un endomorphisme de $M$, on a 
\[
  \tr_\Lambda^H(u\otimes v) = \tr_A^H(u)\tr_\Lambda(v) \text{.}
\]
\end{proposition_}

On se ramène à supposer que $P=A[H]$; l'endomorphisme $u$ est alors la 
multiplication à droite $m_x$ par un élément $x=\sum a_h h$ de $A[H]$. 
L'isomorphisme \eqref{II:eq:5-7-1} transforme $u\otimes v$ en 
$\sum a_h m_h\otimes h^{-1} v$, de trace $a_e\cdot\tr_\Lambda(v)$. 





\subsection{}\label{II:5-9}

Soient $G$ un groupe extension de $\dZ$ par un groupe fini $H$
\[
  1 \to H \to G \to \dZ \to 1 \text{,}
\]
et $G^+$ l'image inverse de $\dN$. Soient $\Lambda$ un anneau, et $P$ un 
$\Lambda[G^+]$-module, projectif en tant que $\Lambda[H]$-module. Si 
$g\in G^+$, on note $Z_g$ le centralisateur de $g$ dans $H$; la multiplication 
par $g$ est un endomorphisme de $P$, vu comme $\Lambda[Z_g]$-module. $P$ étant 
$\Lambda[Z_g]$-projectif, une trace $\tr_\Lambda^{Z_g}(g)$ est définie. 
D'après \ref{II:5-6}, on a 
\begin{equation}\label{II:eq:5-9-1}
  \tr_\Lambda(g) = |Z_g| \cdot \tr_\Lambda^{Z_g}(g) \text{.}
\end{equation}

Supposons que $\Lambda$ soit une algèbre sur un anneau commutatif $A$. Pour 
$P$ un $A[G^+]$-module, projectif en tant que $A[H]$-module et $M$ un 
$\Lambda[G^+]$-module, projectif en tant que $\Lambda$-module, le 
$\Lambda[G^+]$-module $P\otimes_A M$ (action diagonale de $G^+$) est projectif 
en tant que $\Lambda[H]$-module et d'après \ref{II:5-8}, pour tout 
$g\in G^+$, on a 
\begin{equation}\label{II:eq:5-9-2}
  \tr_\Lambda^{Z_g}(g,P\otimes_A M) = \tr_A^{Z_g}(g,P) \cdot \tr_\Lambda(g,M) \text{.}
\end{equation}





\subsection{}\label{II:5-10}

Soient $P$ un $\Lambda[G^+]$-module projectif en tant que $\Lambda[H]$-module, 
et $P_H$ les coinvariants de $H$ dans $P$. C'est un $\Lambda$-module projectif. 
L'action de $g\in G^+$ passe au quotient, et définit un endomorphisme de 
$P_H$ qui ne dépend que de l'image de $g$ dans $\dN$. Nous notons $F$ 
l'action des éléments $g\in G^+$ d'image $1$. 





\begin{proposition_}\label{II:5-11}
$\tr_\Lambda(F,P_H) = \sum_{g\mapsto 1}' \tr_\Lambda^{Z_g}(g,P)$, oùu 
$\sum_{g\mapsto 1}'$ désigne une somme étendue aux classes de 
$H$-conjugaison d'éléments de $G^+$ d'image $1$ dans $\dN$. 
\end{proposition_}

Après multiplication par $|H|$, cette formule peut se vérifier comme suit: 
\[
  |H|\cdot \tr_\Lambda(F,P_H) 
    = \tr_\Lambda\left(\sum_{g\mapsto 1} g,P\right) 
    = \sum_{g\mapsto 1}' \frac{|H|}{|Z_g|}\tr_\Lambda(g,P) 
    = \sum_{g\mapsto 1}' |H|\cdot \tr_\Lambda^{Z_g}(g,P) \text{.}
\]

Choisissons un élément $g\in G^+$ d'image $1$ dans $\dN$, et soit 
$\sigma$ l'automorphisme $\Lambda[H]$ induit par l'automorphisme 
$\operatorname{ad} g$ de $H$. Il revient au même de se donner un 
$\Lambda[G^+]$-module, ou un $\Lambda[H]$--module muni d'un endomorphisme 
$\sigma$-linéaire $\gamma$: on fait agir $g^n h$ par $\gamma^n h$. Dans ce 
langage, la formule \ref{II:5-11} prend la forme: pour $P$ un 
$\Lambda[H]$-module projectif et pour $\gamma$ un endomorphisme 
$\sigma$-linéaire de $P$, on a 
\[
  \tr_\Lambda(\gamma,P_H) = \sum_h' \tr_\Lambda^{Z_g}(h\gamma,P)\text{,}
\]
où $\sum'$ est une somme sur les classes de $\operatorname{ad}g$-conjugaison 
dans $H$ (orbites de $H$ agissant sur lui-même par les 
$x\mapsto h x \operatorname{ad} g(h)^{-1}$). 

Sous cette forme, les propriétés des traces nous ramènent aussitôt au 
cas où $P=\Lambda[H]$. L'endomorphisme $Y$ a alors la forme 
$x\mapsto \sigma(x) a$, avec $a=\sum a_h h$ dans $\Lambda[H]$. Il suffit de 
traiter le cas universel où lles $a_h$ sont des indéterminées. Dans ce 
cas, $\Lambda=\dZ\langle a_h\rangle_{h\in H}$ (polynômes non commutatifs) et 
$\Lambda^\natural$ est sans torsion: on obtient \ref{II:5-11} par l'argument 
du début et division par $|H|$. On peut aussi calculer directement. 





\subsubsection{Remarque}\label{II:5-11-1}

Tout ce qui a été dit ci-dessus s'applique aussi bien aux modules 
$\dZ/2$-gradués, ou aux complexes de modules (cf. \ref{II:4-2}, \ref{II:4-3}). 





\subsection{}\label{II:5-12}

Reprenons les notations de \ref{II:5-2}, et soit $A=\dZ/\ell^n$, $n$ étant 
assez grand pour que $\Lambda$ soit une $A$-algèbre. Soit $f:V_0\to U_0$ un 
revêtement fini étale et connexe de $U_0$, galoisien de groupe de Galois 
$H$, et tel que le faisceau $f^*\cG_0$ soit constant. On note $M$ le valeur 
constants $\h^0(V_0,f^*\cG_0)$, c'est un $\Lambda[H]$-module, projectif en 
tant que $\Lambda$-module. 

Le faisceau $f_* A$ sur $U_0$, muni de l'action naturelle de $H$, est un 
faisceau localement libre (de rang un) de $A[H]$-modules. Le complexe 
$\R\Gamma_c(U,f_* A)$ est donc objet de $\D_\text{parf}\left(A[H]\right)$. Il 
est muni d'un endomorphisme de Frobenius $F^*$. 

Pour l'action naturelle de $H$ sur $f_* f^*\cG_0$, le morphisme trace 
$f_* f^*\cG_0\to\cG_0$ se factorise par un isomorphisme 
\[
  \left(f_* f^*\cG_0\right)_H \to \cG_0 \text{.}
\]
Par ailleurs, on a $f_* f^*\cG_0=f_* A\otimes_A M$ (avec action diagonale de 
$H$). L'isomorphisme 
\[\xymatrix{
  \cG_0 
    & \ar[l]_-\sim (f_* A\otimes_A M)_H = (f_* A\otimes_A M)\otimes_{\Lambda[H]} \Lambda
}\]
se dérive (\ref{II:4-12}) et un isomorphisme 
\[
  \R\Gamma_c(U,\cG) \sim \left(\R\Gamma_c(U,f_* A)\otimes_A M\right)\otimes_{\Lambda[H]} \Lambda \text{.}
\]

L'endomorphisme $F^*$ du membre de gauche se déduit de l'endomorphisme 
$F^*$ de $\R\Gamma_c(U,f_* A)$. Le complexe de $\Lambda[H]$-modules 
$\R\Gamma_c(U,f_* A)$, muni de $F^*$, peut être vu comme un complexe de 
$\Lambda[G^+]$-modules, pour $G^+=H\times \dN$. Les formules \ref{II:5-10} 
et \eqref{II:eq:5-9-2}, amplifiées par \ref{II:5-11-1}, fournissent donc 
\[
  \tr\left(F^*,\R\Gamma_c(U,\cG)\right) = \sum_h' \tr_A^{Z_g}\left(h F^*,\R\Gamma_c(U,f_* A)\right)\cdot \tr_\Lambda(h,M)
\]
où $\sum'$ est une somme sur les classes de conjugaison dans $H$, De plus, 
\[
  |Z_h|\cdot \tr_A^{Z_h}\left(h F^*,\R\Gamma_c(U,f_ A)\right) = \tr_A\left((F h^{-1})^*,\R\Gamma_c(V,A)\right) \text{.}
\]
Cette formule reste vraie pour $A=\dZ/\ell^m$, avec $m$ de plus en plus grand. 
Passant à limite, on trouve que 
$\tr_A^{Z_g}\left(h F^*,\R\Gamma_c(U,f^* A)\right)$ vaut 
\[
  \frac{1}{|Z_h|} \sum (-1)^i \tr\left((F h^{-1})^*,\h_c^i(V,\dQ_\ell)\right)
\]
(un élément de $\dZ_\ell$), et
\begin{equation}\label{II:eq:5-12-1}
\tr\left(F^*,\R\Gamma_c(U,\cG)\right) = \sum_h' \frac{1}{|Z_h|} \tr\left((F h^{-1})^*,\h_c^i(V,\dQ_\ell)\right)\cdot \tr_\Lambda(h,M) \text{.} 
\end{equation}





\subsection{}\label{II:5-13}

Il reste maintenant à appliquer \ref{II:5-4} (noter que si $\bar V$ est la 
complétion projective non singulière de $V$, les points fixes de $F h^{-1}$ 
sont tous de multiplicité un). On peut le faire sans aucun calcul: 
\begin{enumerate}[\indent A.]
  \item Si $U^F=\varnothing$, $\tr\left(F^*,\R\Gamma_c(U,\cG)\right) = 0$. \end{enumerate}

En effet, $F h^{-1}$ n'a pas de point fixe sur $V$, \ref{II:5-4} montre que 
$\sum (-1)^i \tr\left((F h^{-1})^*,\h_c^i(V,\dQ_\ell)\right)=0$ et on 
applique \eqref{II:eq:5-12-1}.
\begin{enumerate}[\indent A.]
\setcounter{enumi}{1}
  \item Dans le cas général, soit $j:U'=U\setminus U^F\hookrightarrow U$. 
    La filtration  $j_! j^* \cG_0\subset \cG_0$ de $\cG_0$ induit une 
    filtration de $\R\Gamma_c(U,\cG)$ de quotients successifs 
    $\R\Gamma_c(U',\cG)$ et $\R\Gamma_c(U^F,\cG)$. D'après \eqref{II:eq:4-4-1} 
    et \ref{II:5-1}, on a 
    \[
      \tr\left(F^*,\R\Gamma_c(U,\cG)\right) = \tr\left(F^*,\R\Gamma_c(U',\cG)\right) + \sum_{x\in U^F} \tr(F^*,\cG_x) \text{,}
    \]
    et, puisque ${U'}^F=0$, le premier terme au second membre est $0$. 
\end{enumerate}










\section{Fin de la preuve de \ref{II:4-10}}\label{II:6}

Prouvons \ref{II:4-10}, sous les hypothèses additionnelles que l'anneau 
$\Lambda$ est annulé par une puissance d'un nombre premier $\ell\ne p$, et 
que $n=1$. Le cas général en résultera: décomposer $\Lambda$ en ses 
composantes $\ell$-primaires ($\ell$ parcourant les nombre premiers), et 
remplacer $\dF_q$ par $\dF_{q^n}$, $X_0$ par $X_0\otimes_{\dF_q}\dF_{q^n}$. 

Pour $X_0$ séparé de type fini sur $\dF_q$, et 
$\cK_0\in\ob\D_\text{ctf}(X_0,\Lambda)$, posons 
\[
  T'(X_0,\cK_0) = \sum_{x\in X^F} \tr(F^*,\cK_x)
\]
et
\[
  T''(X_0,\cK_0) = \tr(F^*,\R\Gamma_c(X,\cK)) \text{.}
\]
Il faut prouver l'identité 
\[
  (1;X_0,\cK_0) \qquad T'(X_0,\cK_0) = T''(X_0,\cK_0) \text{.}
\]
On commence par observer que $T'$ et $T''$ (notés génériquement $T$) 
obéissent à un même formalisme:

(A) Si $j:U_0\hookrightarrow X_0$ est un ouvert de $X_0$, de fermé 
complémentaire $Y_0$, on a 
\[
  T(X_0,\cK_0) = T(U_0,\cK_0|U_0) + T(Y_0,\cK_0|Y_0) \text{.}
\]
C'est clair pour $T'$. Pour $T''$, on observe que la filtration 
$0\subset j_! j^* \cK_0\subset \cK_0$ de $\cK_0$ induit une filtration de 
$\R\Gamma_c(X,\cK)$, de quotients successifs $\R\Gamma_c(U,\cK)$ et 
$\R\Gamma_c(Y,\cK)$, et on invoque \eqref{II:eq:4-4-1} (plus exactement 
énoncé: $\cK_0$, muni de la filtrations dite, définit un objet de 
$\mathsf{DF}(X,\Lambda)$; par application de $\R\Gamma_c$, on a déduit un 
objet de $\mathsf{DF}_\text{parf}(\Lambda)$, auquel $\R\Gamma_c(X,\cK)$ est 
sous-jacent, et de gradué comme dit). 

(B) Si $\cK_0$ est un complexe borné à composantes constructibles et plates, 
\[
  T(X_0,\cK_0) = \sum (-1)^i T(X_0,\cK^i) \text{.}
\]
C'est clair pour $T'$; pour $T''$, on applique l'argument ci-dessus à la 
filtration bête de $\cK_0$, pour obtenir 
\[
  T''(X_0,\cK_0) = \sum T''(X_0,\cK_0^i[-i]) = \sum (-1)^i T''(X_0,\cK_0^i) \text{.}
\]

(C) Pour $f:X_0\to Y_0$ un morphisme, on a 
\[
  T''(X_0,\cK_0) = T''(Y_0,\R f_! \cK_0)
\]
et la même identité vaut pour $T'$ si les fibres de $f$ aux points 
rationnels de $Y_0$ vérifiant $\left(1;f^{-1}(y),\cK_0|f^{-1}(y)\right)$: on 
utilise que 
\[
  T'(X_0,\cK_0) = \sum_{y\in Y^F} T'\left(f^{-1}(y), \cK_0|f^{-1}(y)\right)
\]
et 
\[
  \R\Gamma_c(Y, \R f_! \cK) = \R\Gamma_c(X,\cK) \text{.}
\]

Utilisant ce formalisme, et \ref{II:4-7}, on ramène la preuve de \ref{II:4-10} 
aux deux particuliers \ref{II:5-1} et \ref{II:5-2}. 



\chapter{Fonctions \texorpdfstring{$L$}{L} modulo \texorpdfstring{$\ell^n$}{l n} et modulo \texorpdfstring{$p$}{p}}\label{III}

Dans cet expos\'e, je prouve une formule analogue au th\'eor\`eme du rapport sur 
la formule des traces (ce volume; cit\'e ``Rapport'' par la suite) pour un 
anneau de coefficients $A$ qui soit de torsion premier \`a $p$, ou un corps de 
caract\'eristique $p$. Un usage essentiel sera fait de \cite[XVII 5.5]{sga4}. 










\section{Tenseurs sym\'etriques}\label{III:1}





\subsection{}\label{III:1-1}

Soit $A$ un anneau commutatif. Pour une liste de propri\'et\'es des foncteurs 
$\Gamma^n:(\text{$A$-modules})\to (\text{$A$-modules})$, je renvoie \`a 
\cite[XVII 5.5.1 et 2]{sga4}. Rappelons seulement que pour $M$ plat (en fait, seul 
le cas o\`u $M$ est projectif de type fini nous int\'eresse), on a 
\[
  \Gamma^n M = \left(M^{\otimes n}\right)^{S_n} \qquad \text{(tenseurs sym\'etriques de degr\'e $n$).}
\]





\subsection{}\label{III:1-2}

Soient $X$ un $S$-sch\'ema, $x^n/S$ la puissance fibr\'ee $n$-i\`eme de $X$ et 
$\pi$ la projection de $X^n/S$ sur $(X^n/S)/S_n=\operatorname{Sym}_S^n(X)$ 
(suppos\'e exister). Si $\cG$ est un faisceau de $A$-modules sur $X$, le 
produit tensoriel externe de $n$ copies de $\cG$; $\cG^{\boxtimes n}$, est un 
faisceau $S_n$-\'equivariant sur $X^n/S$. Nous aurons \`a faire usage du 
foncteur $\Gamma^n$ \emph{externe} de \cite[XVII 5.5.7 \`a 9]{sga4}. Rappelons 
seulement que pour $\cG$ un faisceau de $A$-modules plats (en fait, seul le cas 
constructible et plat nous int\'eresse), on a 
\[
  \Gamma_\text{ext}^n(\cG) = \left(\pi_*\cG^{\boxtimes n}\right)^{S_n} \text{.}
\]





\begin{lemma_}\label{III:1-3}
Pour $S=\spec(k)$, avec $k$ alg\'ebriquement clos, $X$ une somme finie de 
copies de $S$, et $G$ un faisceau plat de $A$-modules sur $X$, on a 
\[
  \Gamma\left(\operatorname{Sym}_S^n(X),\Gamma_\text{ext}(\cG)\right) = \Gamma^n \Gamma(X,\cG)
\]
\end{lemma_}

En effet, 
\begin{align*}
  \Gamma\left(\operatorname{Sym}_S^n(X),\Gamma_\text{ext}^n(\cG)\right) 
    &= \Gamma\left(\operatorname{Sym}_S^n(X),\pi_*\cG^{\boxtimes n}\right)^{S_n} \\
    &= \Gamma\left(X^n/S,\cG^{\boxtimes n}\right)^{S_n} 
    = \left(\Gamma(X,\cG)^{\otimes n}\right)^{S_n} 
    = \Gamma^n \Gamma(X,\cG) \text{.}
\end{align*}





\subsection{}\label{III:1-4}

Nous aurons \`a utiliser les d\'eriv\'es des foncteurs non additifs $\Gamma^n$ 
et $\Gamma_\text{ext}^n$. 

La th\'eorie de tela foncteurs d\'eriv\'es est due \`a Dold et Puppe 
\cite{do61}. Pour des rappels plus complets que ceux donn\'es ci-dessous, je 
renvoie \`a \cite[XVII 5.5.3]{sga4}. 

Soit $\sA$ une cat\'egorie ab\'elienne (voire une cat\'egorie additive o\`u 
tout endomorphisme idempotent soit la projection sur un facteur direct). Notons 
$N$ le foncteur de normalisation: $N:($complexes cosimpliciaux d'objets de 
$\sA)\to($complexes différentiables d'objets de $\sA$, avec $K^i$ pour 
$i<0)$. C'est une \'equivalence de cat\'egories. Ainsi que son inverse, elle 
transforme homotopes en morphismes homotopes. Si $\Gamma$ est un foncteur 
$\sA\to\sB$, et si $K$ est un complexe d'objets de $\sA$, \`a degr\'es 
positifs ($K^i=0$ pour $i<0$), on pose $T K=N T N^{-1} K$. 





\subsection{Exemple}\label{III:1-5}

Si $K$ est r\'eduit \`a $M$ en degr\'e $0$, $T K$ st r\'eduit \`a $T M$ en 
degr\'e $0$. 





\subsection{}\label{III:1-6}

Si $K$ est un complexe born\'e, \`a degr\'es positifs, de $A$-module projectifs 
de type fini, le complexe de $A$-modules $\Gamma^n K$ a les m\^emes 
propri\'et\'es. 

Notons $\D_\text{parf}^{\geqslant 0}(A)$ la sous-cat\'egorie de 
$\D_\text{parf}(A)$ (Rapport, \ref{II:4-3}) form\'ee des complexes de 
tor-dimension $\leqslant 0$. On voit que $\Gamma^n$ induit un foncteur 
\[
  \eL\Gamma^n : \D_\text{parf}^{\geqslant 0}(A) \to \D_\text{parf}^{\geqslant 0}(A) \text{.}
\]

Pour la d\'efinition analogue, mais plus compliqu\'ee, de 
$\eL\Gamma_\text{ext}^n$, je renvoie \`a \cite[XVII 5.5.14]{sga4}. 





\subsection{}\label{III:1-7}

Soient $K$ un complexe born\'e de $A$-modules projectifs de type fini, et $u$ 
un endomorphisme de $K$. On note $\det(1-u t,K)$ la s\'erie formelle de terme 
constant $1$ 
\[
  \det(1-u t,K) = \prod_i \det(1-u t,K^i)^{(-1)^i} \text{.}
\]

On laisse au lecteur lla v\'erification des propri\'et\'es suivantes. 


\subsubsection{}\label{III:1-7-1}

Soit $F$ une filtration finie stable par $u$, telle que les $\gr_F^p(K^q)$ 
soient projectifs de type fini. Alors 
\[
  \det(1-u t,K) = \prod_p \det(1-u t,\gr_F^p(K)) \text{.}
\]


\subsubsection{}\label{III:1-7-2}

Si on translate $K$ de $q$ crans vers la gauche, on a 
\[
  \det(1-u t,K[q]) = \det(1-u t,K)^{(-1)^q} \text{.}
\]





\begin{proposition_}\label{III:1-8}
Soit $u$ un endomorphisme d'un complexe born\'e, \`a degr\'es positifs, de 
$A$-modules projectifs de type fini. On a \
\begin{equation}\label{III:eq:1-8-1}
  \det(1-u t,K)^{-1} = \sum_n \tr(u,\Gamma^n K) t^n \text{.}
\end{equation}
\end{proposition_}

Notons $D(u,K)$ le second membre de \eqref{III:eq:1-8-1}. 





\begin{lemma_}\label{III:1-9}
Soit $u$ un endomorphisme d'une suite exacte courte de complexes born\'es \`a 
degr\'es $\geqslant 0$ de $A$-modules projectifs de type fini 
\[
  0 \to K' \to K \to K'' \to 0 \text{.}
\]
On a $D(u,K)=D(u,K')\cdot D(u,K'')$.
\end{lemma_}

Si $0\to M'\to M\to M''\to 0$ est une suite exacte courte scindable de 
$A$-modules, $\Gamma^n M$ a une filtration naturelle, de quotients successifs 
les $\Gamma^i M'\otimes \Gamma^j M''$ ($i+j=n$). D\`es lors, $\Gamma^n K$ admet 
une filtration naturelle, de quotients successifs les complexes 
$N(N^{-1} \Gamma^i K'\otimes N^{-1}\Gamma^j K'')$ ($i+j=n$), canoniquement 
homotopes aux complexes simples $s(\Gamma^i K'\otimes \Gamma^i K'')$ d\'eduit 
des complexes doubles $\Gamma^i K'\otimes \Gamma^i K''$, et 
\[
  \tr(u,\Gamma^n K) = \sum_{i+j=n} \tr(u,\Gamma^i K')\cdot \tr(u,\Gamma^j K'') \text{,}
\]
d'o\`u le lemme. De \ref{III:1-9}, on tire par r\'ecurrence. 





\begin{lemma_}\label{III:1-10}
Soient $F$ une filtration finie stable par $u$ de $K$, telle que les 
$\gr_F^p(K^q)$ soient projectifs de type fini. Alors, 
\[
  D(u,K) = \prod_p D\left(u,\gr_F^p(K)\right) \text{.}
\]
\end{lemma_}





\begin{lemma_}\label{III:1-11}
Soient $u$ un endomorphisme d'un $A$-module projectif de type fini $M$, et 
$M[-q]$ le complexe r\'eduit \`a $M$ en degr\'e $q$. On a 
\[
  D\left(u,M[-q]\right) = \left(\sum \tr(u,\Gamma^n M) t^n\right)^{(-1)^q} \qquad \text{($q\geqslant 0$.)}
\]
\end{lemma_}





D'apr\`es \ref{III:1-5}, c'est vrai pour $q=0$. Prec\'edons par r\'ecurrence, il 
reste \`a montrer que 
\[
  D\left(u,M[-q]\right) \cdot D\left(u,M[-(q+1)]\right) = 1 \text{.}
\]
Cette formule r\'esulte de \ref{III:1-9}, appliqu\'e ou complexe $K$ r\'eduit a 
$M$ en degr\'es $q$ et $q+1$ (avec $d=\operatorname{id}$), et \`a sa 
filtration b\^ete:
\[
  (\cdots 0 \to M \cdots) \to (\cdots M \to M) \to (\cdots M \to 0 \to 0 \cdots) \text{.}
\]
En effet, $K$ est homotope \`a $0$. Dans la cat\'eegorie d\'eriv\'ee, on a donc 
$\Gamma^n(K)=\Gamma^n(0)$; $0$ pour $n>0$ et $A$ pour $n=0$, et $D(u,K)=1$. 





\subsection{}\label{III:1-12}

Prouvons \ref{III:1-8}. D'apr\`es \ref{III:1-10} appliqu\'e \`a la filtration 
``b\^ete'' de $K$ par les sous-complexes 
\[
  0 \to \cdots \to 0 \to K^q \to K^{q+1} \to \cdots \text{,}
\]
on a 
\[
  D(u,K) = \prod_q D\left(u,K^q[-q]\right) =_{\text{(\ref{III:1-10})}} \prod_q \left(\sum \tr\left(u,\Gamma^n(K^q)\right) t^n \right)^{(-1)^q} \text{.}
\]
Il reste \`a prouver que pour $u$ un endomorphisme d'un module projectif de type 
fini $M$, on a 
\begin{equation}\label{III:eq:1-12-1}
  \det\left(1-u t,M\right)^{-1} = \sum \tr(u,\Gamma^n M) t^n \text{.}
\end{equation}
Si $u=0$, les deux membres valent $1$. Ajoutant \`a $M$ un module $N$ sur lequel 
$u=0$, et appliquent \ref{III:1-9}, on se ram\`ene \`a supposer que $M$ est libre. 
Prenons-en un base. Il s'agit alors de prouver des identit\'es alg\'ebriques 
entres les coordonn\'ees de $u$. Le principe de prolongement des identit\'es 
alg\'ebriques nous ram\`ene \`a supposer que $A$ est un corps alg\'ebriquement 
clos. Le module $M$ admet alors une filtration stable par $u$ \`a quotients 
successifs de rang $1$, et \ref{III:1-10} nous ram\`ene au case o\`u $M$ est de 
rang $1$. La formule \eqref{III:eq:1-12-1} se ram\`ene alors \`a l'identit\'e 
\[
  \frac{1}{1-a t} = \sum a^n t^n \text{.}
\]





\begin{corollary_}\label{III:1-13}
Soient $u$ et $u'$ des endomorphismes de complexes born\'es $K$ et $K'$ de 
$A$-modules projectifs de type fini. Si, dans la cat\'egorie d\'eriv\'ee, $K$, 
muni de $u$, est isomorphe \`a $K'$, muni de $u'$, alors 
\[
  \det(1-u t,K) = \det(1-u' t,K') \text{.}
\]
\end{corollary_}

Appliquent \ref{III:1-7-2}, on se ram\`ene \`a supposer que $K$ et $K'$ sont \`a 
degr\'es positifs. On observe alors que le second membre de \ref{III:1-8} 
poss\`ede la propri\'et\'e d'invariance voulue. 





\subsection{}\label{III:1-14}

Ce corollaire permet de d\'efinir $\det(1-u t,K)$ pour $u$ un endomorphisme de 
$K\in \ob\D_\text{parf}(A)$. Cette construction est ad hoc; en fait, pour $v$ un 
automorphisme de $L\in\ob\D_\text{parf}(\Lambda)$, on peut d\'efinir 
$\det(v)\in\Lambda^\times$, et $\det(1-u t,K)$ s'obtient pour $v=1-u t$, 
$\Lambda=A\llbracket t\rrbracket$ et $L=K\otimes_A\Lambda$. 










\section{Le Th\'eor\`eme}\label{III:2}





\subsection{}\label{III:2-1}

Reprenons les notations de (Rapport, \S\ref{II:1}). Pour $A$ un anneau 
noeth\'erien commutatif de torsion, $X_0$ un sch\'ema s\'epar\'e de type fini sur 
$\dF_q$ et $\cK_0\in\ob\D_\text{ctf}(X_0;A)$, on pose 
\[
  L(X_0,\cK_0) = \prod_{x\in |X_0|} \det\left(1-F_x^* t^{\deg(x)}, \cK\right)^{-1}
\]
(\ref{III:1-7}, \ref{III:1-13} et Rapport, \ref{II:1-5}, \ref{II:1-6}). 





\begin{theorem_}\label{III:2-2}
Dans chacun des deux cas suivants
\begin{enumerate}[\indent a)]
  \item $A$ est de torsion premi\`re \`a $p$;
  \item $A$ est r\'eduit et de caract\'eristique $p$
\end{enumerate}
on a 
\begin{equation}\label{III:eq:2-2-1}
  L(X_0,\cK_0) = \det\left(1-F^* t^f, \R\Gamma_c(X,\cK)\right)^{-1} \text{.}
\end{equation}
\end{theorem_}





\subsection{Remarque}\label{III:2-3}

Dans le cas a), la m\'ethode de Rapport, \S\ref{II:3} permet de d\'eduire de 
la formule des traces (Rapport \ref{II:4-10}) que les d\'eriv\'ees logarithmiques 
des deux membres de \eqref{III:eq:2-2-1} sont \'egales. L'anneau $A$ \'etant de 
torsion, cela ne suffit pas pour prouver \ref{III:2-2}. 





\subsection{Remarque}\label{III:2-4}

Si $A$ est un corps, on a 
\[
  L(X_0,\cK_0) = \prod_i L\left(X_0,\underline\h^i(\cK_0)\right)^{(-1)^i}
\]
et la suite spectrale 
$E_2^{p q} = \h_c^p(X,\underline\h^q(\cK))\Rightarrow \h_c^{p+q}(X,\cK)$ montre 
que 
\begin{align*}
  \det\left(1-F^* t, \R\Gamma_c(X,\cK)\right) 
    &= \prod_n \det\left(1-F^* t^f,\h_c^n(X,\cK)\right)^{(-1)^n} \\
    &= \prod_{p,q} \det\left(1-F^* t,\h_c^p(X,\underline\h^q(\cK))\right)^{(-1)^{p+1}} \text{.}
\end{align*}
Ceci ram\`ene \ref{III:2-2} (pour $A$ un corps) au cas particulier suivant. 


\subsubsection{}\label{III:2-4-1}

Si $\cG_0$ est un faisceau constructible de $A$-espaces vectoriels, on a 
\[
  L(X_0,\cG_0) = \prod_n \det\left(1-F^* t^f,\h_c^n(X,\cG)\right)^{(-1)^{n+1}} \text{.}
\]










\chapter{La classe de cohomologie associée à un cycle}\label{IV}

\section{Application: la formule des traces de Lefschetz dans le cas propre et lisse}\label{IV:3}

\begin{corollary_}\label{IV:3-7}
foo
\end{corollary_}

\chapter{Dualité}\label{V}

\section{La dualité de Poincaré pour les courbes}\label{V:2}

\chapter{Applications de la formule des traces aux sommes trigonométriques}\label{VI}

\chapter{Théorèmes de finitude en cohomologie \texorpdfstring{$\ell$}{l}-adique}\label{VII}

\section{Énoncé des théorèmes}\label{VII:1}

\begin{theorem_}\label{VII:1-1}
foo
\end{theorem_}










% to use BibTeX
\bibliographystyle{amsplain}
\bibliography{sga-sources}










\end{document}
