\documentclass{book}


\usepackage[utf8]{inputenc}
\usepackage[all]{xy}
\usepackage[stable]{footmisc}
\usepackage{amsmath,amssymb,amsthm,bm,enumerate,fullpage,mathrsfs,stmaryrd} %,titlesec} to have inline subsubsections
\usepackage[colorlinks,backref=page]{hyperref}


% operators
  \DeclareMathOperator{\ab}{Ab}
  \DeclareMathOperator{\aut}{Aut}
  \DeclareMathOperator{\br}{Br}
  \DeclareMathOperator{\cosp}{cosp}
  \DeclareMathOperator{\dv}{div}
  \DeclareMathOperator{\Div}{Div}
  \DeclareMathOperator{\gal}{Gal}
  \DeclareMathOperator{\gl}{GL}
  \DeclareMathOperator{\h}{H}
  \DeclareMathOperator{\im}{im}
  \DeclareMathOperator{\nrd}{Nrd}
  \DeclareMathOperator{\ob}{Ob}
  \DeclareMathOperator{\pgl}{PGL}
  \DeclareMathOperator{\pic}{Pic}
  \DeclareMathOperator{\sh}{Sh}
  \DeclareMathOperator{\spec}{Spec}


% special characters
  \newcommand{\cF}{\mathcal{F}}
  \newcommand{\cG}{\mathcal{G}}
  \newcommand{\cH}{\mathcal{H}}
  \newcommand{\cL}{\mathcal{L}}
  \newcommand{\cO}{\mathcal{O}}
  \newcommand{\cT}{\mathcal{T}}

  \newcommand{\dC}{\mathbb{C}}
  \newcommand{\dG}{\mathbb{G}}
  \newcommand{\dmu}{{\bm\mu}}%{\boldsymbol{\mu}}
  \newcommand{\dN}{\mathbb{N}}
  \newcommand{\dP}{\mathbb{P}}
  \newcommand{\dR}{\mathbb{R}}
  \newcommand{\dZ}{\mathbb{Z}}

  \newcommand{\fa}{\mathfrak{a}}
  \newcommand{\fm}{\mathfrak{m}}
  \newcommand{\fr}{\mathfrak{r}}
  \newcommand{\fX}{\mathfrak{X}}
  
  \newcommand{\sC}{\mathscr{C}}
  \newcommand{\sD}{\mathscr{D}}
  \newcommand{\sE}{\mathscr{E}}
  \newcommand{\sS}{\mathscr{S}}
  \newcommand{\sU}{\mathscr{U}}


% miscelaneous
  \newcommand{\an}[1]{{#1}^{\textnormal{an}}}
  \newcommand{\const}[1]{\underline{#1}}
  \newcommand{\et}[1]{{#1}_{\textnormal{et}}}
  \newcommand{\iso}{\xrightarrow\sim}
  \newcommand{\R}{\mathsf{R}}


% theorem types
  \newtheorem{proposition}[subsubsection]{Proposition}
  \newtheorem{corollary}[subsubsection]{Corollaire}
  \newtheorem{definition}[subsubsection]{Définition}
  \newtheorem{theorem}[subsubsection]{Théorème}
  \newtheorem{lemma}[subsubsection]{Lemme}
  \setcounter{tocdepth}{2}


\renewcommand*\thesection{\arabic{section}}
\setcounter{secnumdepth}{3}


\title{Séminaire de Géométrie Algébrique du Bois-Marie SGA $4\frac 1 2$}
\author{par P. Deligne ave la collaboration de J.F. Boutot, \\ A. Grothendieck, L. Illusie et J.L. Verdier}
\date{1977}


\begin{document}
\maketitle

% French characters
% à é è ê ô ù û

% other notes: there are two I.6.5's in Deligne's original - one defining sheaf 
% cohomology, the other defining torsors










\chapter*{Introduction}

Ce volume a pour but de faciliter au non-expert l'usage de la 
cohomologie $\ell$-adique. J'espère qu'il lui permettra souvent d'éviter le 
recours aux exposés touffus de SGA 4 et SGA 5. Il contient aussi quelques 
résultats nouveaux. 

Le premier exposé, édigé par J.F. Boutot, survole SGA 4. Il donne les 
principaux résultats -- avec une généralité minimale, souvent 
insuffisante pour les applications -- et une idée de leur démonstration. Pour 
des résultats complets, ou des démonstrations détaillées, SGA 4 reste 
indispensable. 

Le ``Rapport sur la formule des traces'' contient une démonstration complété de 
la formule des traces pour l'endomorphisme de Frobenius. La démonstration est 
celle donnée par Grothendieck dans SGA 5, élaguée de tout détail inutile. Ce 
Rapport devrait permettre à utilisateur d'oublier SGA 5, qu'on pourra 
considérer comme une série de digression, certaines très intéressantes. Son 
existence permettra de publier prochainement SGA 5 tel quel. Il est complété 
par l'exposé ``Applications de la formule des traces aux sommes 
trigonométriques'' qui explique comment la formule des traces permet l'étude de 
sommes trigonométriques, et donne des exemples. 

Le public visé par les autres exposés est plus limité, et leur style s'en 
ressent. L'exposé ``Fonctions $L$ modulo $\ell^n$ et modulo $p$'' est une 
généralisation ``modulaire'' du Rappoport, basée sur l'étude SGA 4 XVII 5.5 des 
puissances symétriques. L'exposé ``La classe de cohomologie associée à un 
cycle'' définit cette classes dans divers contextes, et donne la compatibilité 
entre intersections et cup-produits. Dans ``Dualité'' sont rassemblés quelques 
résultats connus, pour lesquels manquait une référence, et quelques 
compatibilités. L'exposé ``Théorèmes de finitude en cohomologie $\ell$-adique'' 
est nouveau. Il donne notamment, en cohomologie sans supports, des théorèmes de 
finitude analogues à ceux connus en cohomologie à supports compacts. 

Pour plus de détails sur les exposés, je renvoie à leur introduction 
respective. 

Je remercie enfin J.L. Verdier de m'avoir permis de reproduire ici ses notes 
``Catégories dérivées (Etat $0$).'' Elles restent je crois très utiles, et 
étaien: devenues introuvables. 

Dans les références internes à ce volume, les exposés sont cités par 
un titre abrégé, indiqué entre [ ] dans la table des matières. 

Bures-sur-Yvette, le 20 Septembre 1976

Pierre Deligne











\tableofcontents




















\chapter{Cohomologie étale: les points de départ\texorpdfstring{\footnote{rédigé par J.F. Boutot}}{}}




















\section{Topologies de Grothendieck}\label{1}

A l'origine, les topologies de Grothendieck sont apparues comme sous-jacentes 
à sa théorie de la descente (cf. SGA 1 VI, VIII); l'usage des théorie de 
cohomologie correspondantes est plus tardif. La même démarche est suivi 
ici: en formalisant les notions classiques de localisation, de propriété 
locale et de recollement (\ref{1-1}, \ref{1-2}, \ref{1-3}), en dégage le 
concept général de topologie de Grothendieck (\ref{1-6}); pour en justifier 
l'introduction en géométrie algébrique, on démontre un théorème 
de descente fidèlement plat (\ref{1-4}), généralisation du classique théorème 
de Hilbert (\ref{1-5}). 

Le lecteur trouvera une exposition plus compète, mais concise, du 
formalisme dans Giraud \cite{Gi}. Les notes de M. Artin: ``Grothendieck 
topologies'' \cite{Ar} (chapitres I à III) restent également utiles. Les 
866 pages des exposés I à V de SGA 4 sont précieuses lorsqu'on 
considère des topologies exotiques, telle celle qui donne naissance à la 
cohomologie cristalline; pour utiliser la topologie étale si proche de 
l'intuition classique, il n'est pas indispensable de les lire. 










\subsection{Cribles}\label{1-1}

Soient $X$ un espace topologique et $f:X\to \dR$ une fonction à valeur 
réelles sur $X$. La continuité de $f$ est une propriété de nature 
locale; autrement dit, si $f$ est continue sur tout ouvert suffisamment petit 
de $X$, $f$ est continue sur $X$ tout entier. Pour formaliser la notion de 
``propriété de nature locale,'' nous introduirons quelques définitions.

On dit qu'un ensemble $\sU$ d'ouverts de $X$ est un \emph{crible} si pour tout 
$U\in\sU$ et $V\subset U$, on a $V\in\sU$. On dit qu'un crible est 
\emph{couvrant} si la réunion de tout les ouverts appartenant à ce crible est 
égale à $X$.

Etant donnée une famille $\{U_i\}$ d'ouverts de $X$, le crible engendré par 
$\{U_i\}$ est par définition l'ensemble des ouverts $U$ de $X$ tels que $U$ 
soit contenu dans l'un des $U_i$. 

On dit qu'une propriété $P(U)$, définie pour tout ouvert $U$ de $X$, est 
\emph{locale} si, pour tout crible couvrant $\sU$ de tout ouvert $U$ de $X$, 
$P(U)$ est vraie si et seulement si $P(V)$ est vraie pour tout $V\in \sU$. Par 
exemple, étant donné $f:X\to \dR$, la propriété ``$f$ est continue sur $U$'' 
est locale. 





\subsection{Faisceaux}\label{1-2}

Précisons la notion de fonction donnée localement sur $X$.





\subsubsection{Point de vue des cribles}\label{1-2-1}

Soit $\sU$ un crible d'ouverts de $X$. On appelle fonction donnée 
$\sU$-localement sur $X$ la donnée pour tout $U\in \sU$ d'une fonction $f_U$ 
sur $U$ telle que, si $V\subset U$, on ait $f_V=f_U|V$. 





\subsubsection{Pointe de vue de Čech}\label{1-2-2}

Si le crible $\sU$ est engendré par une famille d'ouverts $U_i$ de $X$, se 
donner une fonction $\sU$-localement revient à se donner une fonction $f_i$ 
sur chaque $U_i$, telle que $f_i|{U_i\cap U_j} = f_j|{U_i\cap U_j}$. 

Autrement dit, si $Z=\coprod U_i$, se donner une fonction $\sU$-localement 
revient à se donner une fonction sur $Z$ qui soit constante sur les fibres de 
la projection naturelle $Z\to X$. 





\subsubsection{}\label{1-2-3}

Les fonctions continues forment un faisceau; cela signifie que pour tout crible 
couvrant $\sU$ d'un ouvert $V$ de $X$ et toute fonction donnée 
$\sU$-localement $\{f_U\}$ telle que chaque $f_U$ soit continue sur $U$, il 
existe une unique fonction continue $f$ sur $V$ telle que $f|U=f_U$ pour tout 
$U\in \sU$.










\subsection{Champs}\label{1-3}

Précisons maintenant la notion de fibré vectoriel donné localement sur $X$. 





\subsubsection{Point de vue des cribles}\label{1-3-1}

Soit $\sU$ un crible d'ouverts de $X$. On appelle fibré vectoriel donné 
$\sU$-localement sur $X$ les données de 
\begin{enumerate}[\indent a)]
  \item un fibré vectoriel $E_U$ sur chaque $U\in \sU$, 
  \item si $V\subset U$, un isomorphisme $\rho_{U,V} : E_V\iso E_U|V$, vérifiant 
  \item si $W\subset V\subset U$, le diagramme 
    \[\xymatrix{
      E_W \ar[r]^-{\rho_{U,W}} \ar[dr]_-{\rho_{V,W}}
        & E_U|W \\
      & E_V|W \ar[u]_-{\rho_{U,V}|W}
    }\]
    commute, c'est-à-dire 
    $\rho_{U,V} = (\mbox{$\rho_{U,V}$ restreint à $W$}) \circ \rho_{V,W}$. 
\end{enumerate}





\subsubsection{Point de vue de Čech}\label{1-3-2}

Si le crible $\sU$ est engendré par une famille d'ouverts $U_i$ de $X$, se 
donner un fibré vectoriel $\sU$-localement revient à se donner:
\begin{enumerate}[\indent a)]
  \item un fibré vectoriel $E_i$ sur chaque $U_i$, 
  \item si $U_{ij} = U_i\cap U_j = U_i\times_X U_j$, un isomorphisme 
    $\rho_{ji} : E_i|U_{ij} \iso E_j|U_{ij}$, de sorte que 
  \item si $U_{ijk} = U_i\times_X U_j\times_X U_k$, le diagramme 
    \[\xymatrix{
      E_i |U_{ijk} \ar[r]^-{\rho_{ki}|U_{ijk}} \ar[dr]_-{\rho_{ji}|U_{ijk}} 
        & E_k|U_{ijk} \\
      & E_j|U_{ijk} \ar[u]_-{\rho_{kj}|U_{ijk}}
    }\]
    commute, c'est-à-dire $\rho_{ki} = \rho_{kj}\circ \rho_{ji}$ sur 
    $U_{ijk}$. 
\end{enumerate}

Autrement dit, si $Z=\coprod U_i$ et si $\pi:Z\to X$ est la projection 
naturelle, se donner un fibré vectoriel $\sU$-localement revient à se donner: 
\begin{enumerate}[\indent a)]
  \item un fibré vectoriel $E$ sur $Z$, 
  \item si $x$ et $y$ sont deux points de $Z$ tels que $\pi(x) = \pi(y)$, un 
    isomorphisme $\rho_{y x}:E_x\iso E_y$ entre les fibres de $E$ en $x$ et en 
    $y$, dépendant continûment de $(x,y)$ et tel que, 
  \item si $x$, $y$ et $z$ sont trois points de $Z$ tels que 
  $\pi(x)=\pi(y)=\pi(z)$, on ait $\rho_{zx} = \rho_{zy}\circ \rho_{yx}$. 
\end{enumerate}





\subsubsection{}\label{1-3-3}

Une fibré vectoriel $E$ sur $X$ définit un fibré vectoriel donné 
$\sU$-localement $E_\sU$; le système des restrictions $E_U$ de $E$ aux objets 
de $\sU$. Le fait que la notion de fibré vectoriel est de nature locale peut 
s'exprimer ainsi: pour tout crible couvrant $\sU$ de $X$, le foncteur 
$E\mapsto E_\sU$, des fibrés vectoriels sur $X$ dans les fibrés vectoriels 
donnés $\sU$-localement, est une équivalence de catégories. 





\subsubsection{}\label{1-3-4}

Si dans \ref{1} on remplace ``ouvert de $X$'' par ``partie de 
$X$,'' on obtient la notion de crible de sous-espaces de $X$. Dans ce cadre 
aussi on dispose de théorèmes de recollement. Par exemple: soient $X$ un 
espace et $\sC$ un crible de sous-espaces de $X$ engendré par un recouvrement 
fermé localement fini de $X$, alors le foncteur $E\mapsto E_\sC$, des fibrés 
vectoriels sur $X$ dans les fibrés vectoriels donnés $\sC$-localement est une 
équivalence de catégories. 

En géométrie algébrique, il est utile de considérer aussi des ``cribles 
d'espaces au-dessus de $X$''; c'est ce que nous verrons au paragraphe suivant. 










\subsection{Descente fidèlement plat}\label{1-4}





\subsubsection{}\label{1-4-1}

Dans le cadre des schémas, la topologie de Zariski n'est pas assez fine pour 
l'étude des problèmes non linéaires et on est amené à remplacer dans les 
définitions précédentes les immersions ouvertes par des morphismes plus 
généraux. De ce point de vue, les techniques de descente apparaissent comme des 
techniques de localisation. Ainsi l'énoncé de descente suivant peut sexprimer 
en disant que les propriétés considérées sont de nature locale pour la 
topologie fidèlement plate (on dit qu'un morphisme de schémas est fidèlement 
plat s'il est plat est surjectif). 


\begin{proposition}\label{1-4-2}
Soient $A$ un anneau et $B$ une $A$-algèbre fidèlement plate. Alors:
\begin{enumerate}[(i)]
  \item Une suite $\Sigma=(M'\to M\to M'')$ de $A$-modules est exacte dés 
    que la suite $\Sigma_{(B)}$ qui s'en déduit par extension des scalaires 
    à $B$ est exacte.
  \item Un $A$-module $M$ est de type fini (resp. de présentation finie, 
    plat, localement libre de rang fini, inversible (i.e. localement libre 
    de rang un)) dés que le $B$-module $M_{(B)}$ l'est.
\end{enumerate}
\end{proposition}
\begin{proof}
(i) Le foncteur $M\mapsto M_{(B)}$ étant exact (platitude de $B$), il suffit 
de montrer que, si un $A$-module $N$ est non nul, $N_{(B)}$ est non nul. Si 
$N$ est non nul, $N$ contient un sous-module monogène non nul $A/\fa$; alors 
$N_{(B)}$ contient un sous-module monogène $(A/\fa)_{(B)} = B/\fa B$, non nul 
par surjectivité du morphisme structural $\varphi:\spec(B)\to\spec(A)$ (si 
$V(\fa)$ est non vide, $\varphi^{-1}(V(\fa))=V(\fa B)$ est non vide). 

(ii) Pour toute famille $(x_i)$ d'éléments de $M_{(B)}$, il existe un 
sous-module de type fini $M'$ de $M$ tel que $M'_{(B)}$ contienne les $x_i$. Si 
$M_{(B)}$ est de type fini et si les $x_i$ engendrent $M_{(B)}$, on a 
$M'_{(B)}=M_{(B)}$, donc $M'=M$ et $M$ est de type fini. 
\end{proof}

Si $M_{(B)}$ est de présentation finie, on peut, d'aprés ce qui précède, 
trouver une surjection $A^n\to M$. Si $N$ est le noyau de cette surjection, le 
$B$-module $N_{(B)}$ est de type fini, donc $N$ l'est, et $M$ est de 
présentation finie. L'assertion pour ``flat'' résulte aussitôt de (i); 
``localement libre de rang fini'' signifie ``plat et de présentation finie'' 
et le rang se teste par extension des scalaires à des corps.





\subsubsection{}\label{1-4-3}

Soient $X$ un schéma et $\sS$ une classe de $X$-schémas stable par produit 
fibré sur $X$. Une classe $\sU\subset \sS$ est un \emph{crible} sur $X$ 
(relativement à $\sS$) si, pour tout morphisme $\varphi:V\to U$ de 
$X$-schémas, avec $U,V\in \sS$ et $U\in\sU$, on a $V\in \sU$. Le crible 
\emph{engendré} par une famille $\{U_i\}$ de $X$-schémas dans $\sS$ est la 
classe des $V\in\sS$ tels qu'il existe un morphisme de $X$-schémas de $V$ dans 
l'un des $U_i$. 






\subsubsection{}\label{1-4-4}

Soit $\sU$ un crible sur $X$. On appelle module quasi-cohérent donné 
$\sU$-localement sur $X$ la donnée de 
\begin{enumerate}[\indent a)]
  \item un module quasi-cohérent $E_U$ sur chaque $U\in\sU$, 
  \item pour tout $U\in\sU$ et pour tout morphisme $\varphi:V\to U$ de 
    $X$-schémas dans $\sS$, un isomorphisme 
    $\rho_\varphi:E_V\iso \varphi^* E_U$, ceux-ci étant tels que 
  \item si $\psi:W\to V$ est un morphisme de $X$-schémas dans $\sS$, le 
    diagramme 
    \[\xymatrix{
      E_W \ar[r]^-{\rho_{\varphi\circ\psi}} \ar[dr]_-{\rho_\psi} 
        & \psi^*\varphi^* E_U \\
      & \psi^* E_V \ar[u]_-{\psi^*\rho_\varphi}
    }\]
    commute, c'est-à-dire 
    $\rho_{\varphi\circ\psi}=(\psi^*\rho_\varphi)\circ \rho_\psi$. 
\end{enumerate}

Si $E$ est un module quasi-cohérent sur $X$, on note $E_\sU$ le module 
donné $\sU$-localement valant $\varphi_U^* E$ sur $\varphi:U\to X$ et tel que, 
pour tout morphisme $\psi:V\to U$ l'isomorphisme de restriction $\rho_\psi$ 
soit l'isomorphisme canonique 
$E_V=(\varphi_U\circ \psi)^* E\iso \psi^* \varphi_U^* E = \psi^* E_U$. 


\begin{theorem}\label{1-4-5}
Soit $\{U_i\}\in\sS$ une famille finie de $X$-schémas plats sur $X$ telle que 
$X$ soit le réunion des images des $U_i$, et soit $\sU$ le crible engendré par 
$\{U_i\}$. Alors le foncteur $E\mapsto E_\sU$ est une équivalence de la 
catégorie des modules quasi-cohérents sur $X$ avec la catégorie des modules 
quasi-cohérents donnés $\sU$-localement.
\end{theorem}
\begin{proof}
Nous ne traiterons que le cas où $x$ est affine et où $\sU$ est engendré par 
un $X$-schéma affine $U$ fidèlement plat sur $X$. La réduction à ce cas est 
formelle. On pose $X=\spec(A)$ et $U=\spec(B)$. 

Si le morphisme $U\to X$ admet une section, $X$ appartient au crible $\sU$ et 
l'assertion est évidente. Nous nous réduirons à ce cas. 

Un module quasi-cohérent donné $\sU$-localement définit des modules $M'$, $M''$ 
et $M'''$ sur $U$, $U\times_X U$ et $U\times_X U\times_X U$, et des 
isomorphismes $\rho:p^* M^\bullet\simeq M^\bullet$ pour tout morphisme de 
projection $p$ entre ces espaces; c'est là un \emph{diagramme cartésien} 
\[\xymatrix{
  M^* : M' \ar@<2pt>[r] \ar@<-2pt>[r] 
    & M'' \ar[r] \ar@<4pt>[r] \ar@<-4pt>[r] 
    & M'''
}\]
au-dessus de 
\[\xymatrix{
  U_* : U 
    & U\times_X U \ar@<2pt>[l] \ar@<-2pt>[l] 
    & U\times_X U\times_X U \ar[l] \ar@<4pt>[l] \ar@<-4pt>[l] \mbox{.}
}\]
Réciproquement $M^*$ détermine le module donné $\sU$-localement: pour 
$V\in\sU$, il existe $\varphi:V\to U$ et on pose $M_U=\varphi^* M'$; 
pour $\varphi_1,\varphi_2:V\to U$, on a une identification naturelle 
$\varphi_1^* M'\simeq (\varphi_1\times \varphi_2)^* M'' \simeq \varphi_2^* M'$, 
et on voit en utilisant $M'''$ ue ces identifications sont compatibles, de 
sorte que la définition est légitime. Bref, il revient au même de se donner un 
module $\sU$-localement ou un diagramme $M^*$ cartésien sur $U_*$. 

Traduisons en termes algébriques: se donner $M^*$ revient à se donner un 
diagramme cartésien de modules 
\[\xymatrix{
  M' \ar@<3pt>[r]|-{\partial_0} \ar@<-3pt>[r]|-{\partial_1} 
    & M'' \ar@<6pt>[r]|-{\partial_0} \ar[r]|-{\partial_0} \ar@<-6pt>[r]|-{\partial_2}
    & M'''
}\]
au-dessus du diagramme d'anneaux 
\[\xymatrix{
  B \ar@<3pt>[r]|-{\partial_0} \ar@<-3pt>[r]|-{\partial_1} 
    & B\otimes_A B \ar@<6pt>[r]|-{\partial_0} \ar[r]|-{\partial_0} \ar@<-6pt>[r]|-{\partial_2} 
    & B\otimes_A B\otimes_A B
}\]
(précisons: on a $\partial_i(b m)=\partial_i(b)\cdot\partial_i(m)$, les 
identités usuelles telles que $\partial_0\partial_1=\partial_0\partial_0$ 
sont vraies, et ``cartésien'' signifie que les morphismes 
$\partial_i:M'\otimes_{B,\partial_i}(B\otimes_A B)\to M''$ et 
$M'''\otimes_{B\otimes_A B,\partial_i}(B\otimes_A B\otimes_A B)\to M'''$ sont 
des isomorphismes). 

Le foncteur $E\mapsto E_\sU$ devient le foncteur qui, à un $A$-module $M$, 
associe 
\[\xymatrix{
  M^* = (M\otimes_A B \ar@<2pt>[r] \ar@<-2pt>[r] 
    & M\otimes_A B\otimes_A B \ar@<-4pt>[r] \ar[r] \ar@<4pt>[r] 
    & M\otimes_A B\otimes_A B\otimes_A B)\text{.}
}\]
Il admet pour adjoint à droite le foncteur
\[\xymatrix{
  (M' \ar@<2pt>[r] \ar@<-2pt>[r] 
    & M'' \ar@<-4pt>[r] \ar[r] \ar@<4pt>[r]
    & M''') \ar@{|->}[r] 
    & \ker(M' \ar@<2pt>[r] \ar@<-2pt>[r] 
    & M''')\text{.}
}\]
Il nous faut prouver que les flèches d'adjonction 
\[
  M \to\ker(M\otimes_A B\rightrightarrows M\otimes_A B\times_A B)
\]
et
\[
  \ker(M''\twoheadrightarrow M'')\otimes_A B \to M'
\]
sont des isomorphismes. D'après (\ref{1-4-2}.i), il suffit de le prouver 
après un changement de base fidèlement plat $A\to A'$ ($B$ devenant 
$B'=B\otimes_A A'$). Prenant $A'=B$, ceci nous ramène au cas où $U\to X$ 
admet une section. 
\end{proof}










\subsection{Un cas particulier: le théorème 90 de Hilbert}\label{1-5}





\subsubsection{}\label{1-5-1}

Soient $k$ un corps, $k'$ une extension galoisienne de $k$ et $G=\gal(k'/k)$. 
Alors l'homomorphisme 
\begin{align*}
  k'\otimes_k k' &\to \oplus_{\sigma\in G} k' \\
  x\otimes y     &\mapsto \{x\cdot \sigma(y)\}_{\sigma\in G}
\end{align*}
est bijectif. 

On en déduit qu'il revient au même de se donner un module localement pour 
le crible engendré par $\spec(k')$ sur $\spec(k)$ ou de se donner un 
$k'$-espace vectoriel muni d'une action semi-linéaire de $G$, c'est-à-dire: 
\begin{enumerate}[\indent a)]
  \item un $k'$-espace vectoriel $V'$,
  \item pour tout $\sigma\in G$, un endomorphisme $\varphi_\sigma$ de la 
    structure de groupe de $V'$ tel que 
    $\varphi_\sigma(\lambda v) = \sigma(\lambda) \varphi_\sigma(v)$, pour 
    tout $v\in V'$, vérifiant la condition 
  \item pour tout $\sigma,\tau\in G$, on a 
    $\varphi_{\tau\sigma} = \varphi_\tau\circ\varphi_\sigma$. 
\end{enumerate}

Soit $V={V'}^G$ le groupe des invariants par cette action de $G$; c'est un 
$k$-espace vectoriel et, d'après le théorème (\ref{1-4-5}), on a:

\begin{proposition}\label{1-5-2}
L'inclusion de $V$ dans $V'$ définit un isomorphisme 
$V\otimes_k k' \iso V'$.
\end{proposition}

En particulier, si $V'$ est de dimension $1$ et si $v'\in V$ est non nul, 
$\varphi_\sigma$ est déterminé par la constante 
$c(\sigma)\in {k'}^\times$ telle que $\varphi_\sigma(v')=c(\sigma)v'$ et la 
condition c) s'écrit 
\[
  c(\tau\sigma) = c(\tau)\cdot \tau(c(\sigma))\text{.}
\]
D'après la proposition il existe un vecteur invariant non nui 
$v=\mu v'$, $\mu\in {k'}^\times$. On a donc pour tout $\sigma\in G$, 
\[
  c(\sigma) = \mu\cdot \sigma(\mu^{-1})\text{.}
\]
Autrement dit tout $1$-cocycle de $G$ à valeurs dans ${k'}^\times$ est un 
cobord: 

\begin{corollary}\label{1-5-3}
On a $\h^1(G,{k'}^\times)=0$.
\end{corollary}










\subsection{Topologies de Grothendieck}\label{1-6}

Nous transcrivons maintenant les définitions des paragraphes précédents 
dans un cadre abstrait englobant à la fois le cas des espaces topologiques 
et celui des schémas. 





\subsubsection{}\label{1-6-1}

Soient $\sS$ une catégorie et $U$ un objet de $\sS$. On appelle \emph{crible} 
sur $U$ un sous-ensemble $\sU$ de $\ob(\sS/U)$ tel que si $\varphi:V\to U$ 
appartient à $\sU$ et si $\psi:W\to V$ est un morphisme dans $\sS$, alors 
$\varphi\circ\psi:W\to U$ appartient à $\sU$. 

Si $\{\varphi_i:U_i\to U\}$ est une famille de morphismes, le crible 
engendré par les $U_i$ est par définition l'ensemble des morphismes 
$\varphi:V\to U$ qui se factorisent à travers l'un des $\varphi_i$. 

Si $\sU$ est un crible sur $U$ et si $\varphi:V\to U$ est un morphisme, la 
restriction $\sU_V$ de $\sU$ à $V$ est par définition le crible sur $V$ 
constitué oar les morphismes $\psi:w\to V$ tels que 
$\varphi\circ\psi:W\to U$ appartienne à $\sU$.





\subsubsection{}\label{1-6-2}

La donnée d'une \emph{topologie de Grothendieck} sur $\sS$ consiste en la 
donnée pour tout objet $U$ de $\sS$ d'un ensemble $C(U)$ de cribles sur $U$, 
dits cribles couvrants, de telle sorte que les axiomes suivants soient 
satisfaits:
\begin{enumerate}[\indent a)]
  \item Le crible engendré par l'identité de $U$ est couvrant.
  \item Si $\sU$ est un crible couvrant sur $U$ et si $V\to U$ est un 
    morphisme, le crible $\sU_V$ est couvrant.
  \item Un crible localement couvrant est couvrant. Autrement dit, si $\sU$ est 
    un crible couvrant sur $U$ et si $\sU'$ est un crible sur $U$ tel que, pour 
    tout $V\to U$ appartenant à $\sU$, le crible $\sU_V'$ est couvrant, alors 
    $\sU'$ est couvrant.
\end{enumerate}

On appelle \emph{site} la donnée d'une catégorie munie d'une topologie de 
Grothendieck.





\subsubsection{}\label{1-6-3}

Étant donnée un site $\sS$, on appelle préfaisceau sur $\sS$ un foncteur 
contravariant $\cF$ de $\sS$ dans la catégorie des ensembles. Pour tout 
objet $U$ de $\sS$, on appelle section de $\cF$ au-dessus de $U$ les 
éléments de $\cF(U)$. Pour tout morphisme $V\to U$ et pour tout 
$s\in \cF(U)$, on note $s|V$ ($s$ restreint à $V$) l'image de $s$ dans 
$\cF(V)$.

Si $\sU$ est un crible sur $U$, on appelle section donnée $\sU$-localement la 
donnée, pour tout $V\to U$ appartenant à $\sU$, d'une section 
$s_V\in\cF(V)$ telle que, pour tout morphisme $W\to V$, on sit $s_V|W=s_W$. On 
dit que $\cF$ est un \emph{faisceau} si, pour tout objet $U$ de $\sS$, pour 
tout crible couvrant $\sU$ sur $U$ et pour tout section donnée 
$\sU$-localement $\{s_V\}$, il existe une unique section $s\in\cF(U)$ telle que 
$s|V=s_V$, pour tout $V\to U$ appartenant à $\sU$. 

On définit de manière analogue les \emph{faisceaux abéliens} en 
remplaçant la catégorie des ensembles par celle des groupes abéliens. On 
montre que la catégorie des faisceaux abéliens sur $\sS$ est une 
catégorie abélienne possédant suffisamment d'injectifs. Une suite 
$\cF\xrightarrow f \cG\xrightarrow g \cH$ de faisceaux est exacte si, pour 
tout objet $U$ de $\sS$, et pour tout $s\in \cG(U)$ telle que $g(s)=0$, 
il existe localement $t$ tel que $f(t)=s$; i.e. s'il existe un crible couvrant 
$\sU$ sur $U$ et pour tout $V\in\sU$, une section $t_V$ de $\cF$ sur $V$ telle 
que $f(t_V)=s|V$. 





\subsubsection{Exemples}\label{1-6-4}

Nous en avons vu deux plus haut. 

\begin{enumerate}[\indent a)]
  \item Soient $X$ un espace topologique et $\sS$ la catégorie dont les objets 
    sont les ouverts de $X$ et les morphismes les inclusions naturelles. La 
    topologique de Grothendieck sur $\sS$ correspondant à la topologie usuelle 
    de $X$ est celle pour laquelle un crible $\sU$ sur un ouvert $U$ de $X$ est 
    couvrant si la réunion des ouverts appartenant à ce crible est égale à 
    $U$. Il est clair que la catégorie des faisceaux sur $\sS$ est équivalente 
    à la catégorie des faisceaux sur $X$ au sens usuel. 
  \item Soient $X$ un schéma et $\sS$ la catégorie des schémas sur $X$. On 
    appelle topologie fpqc (fidèlement plate quasi-compacte) sur $\sS$ la 
    topologie de Grothendieck pour laquelle un crible sur un $X$-schéma $U$ est 
    couvrant s'il est engendré par une famille finie de morphismes plats dont  
    les images recouvrent $U$. 
\end{enumerate}





\subsubsection{Cohomologie}\label{1-6-5}

On supposera toujours que la catégorie $\sS$ a un objet final $X$. Alors on 
appelle sections globales d'un faisceaux abélien $\cF$, et on note 
$\Gamma\cF$ ou $\h^0(X,\cF)$, le groupe $\cF(X)$. Le foncteur 
$\cF\mapsto\Gamma\cF$ est un foncteur exact \'a gauche de la catégorie des 
faisceaux abéliens sur $\sS$ dans la catégorie des groupes abéliens, ou 
note $\h^i(X,-)$ ses dérivés (ou satellites). Ces groupes de cohomologie 
représentent les obstructions à passer du local au global. Par définition, si 
$0\to\cF\to\cG\to\cH\to 0$ est une suite exacte de faisceaux abéliens, on a une 
suite exacte longue de cohomologie:
\[\xymatrix{
  0 \ar[r] 
    & \h^0(X,\cF) \ar[r] 
    & \h^0(X,\cG) \ar[r] 
    & \h^0(X,\cH) \ar[r] 
    & \h^1(X,\cF) \ar[r] 
    & \cdots \\
  \ldots \ar[r] 
    & \h^n(X,\cF) \ar[r] 
    & \h^n(X,\cG) \ar[r]
    & \h^n(X,\cH) \ar[r] 
    & \h^{n+1}(X,\cF) \ar[r] 
    & \cdots
}\]





\subsubsection{}\label{1-6-6}

Étant donné un faisceau abélien $\cF$ sur $\sS$, on appelle 
\emph{$\cF$-torseur} un faisceau $\cG$ muni d'une action $\cF\times\cG\to\cG$ 
de $\cF$ telle que localement (après restriction à tous les objets d'un 
crible couvrant l'objet final $X$) $\cG$ muni de l'action de $\cF$ soit 
isomorphe à $\cF$ muni de l'action canonique $\cF\times \cF\to \cF$ par 
translations. 

On peut montrer que $\h^1(X,\cF)$ s'interprète comme l'ensemble des classes 
à isomorphisme près de $\cF$-torseurs. 




















\section{Topologie étale}\label{2}

On spécialise les définitions du chapitre précédent au cas de la 
topologie étale d'un schéma $X$ (\ref{2-1}, \ref{2-2}, \ref{2-3}). La 
cohomologie correspondante coïncide dans le cas où $X$ est le spectre d'un 
corps $K$ avec la cohomologie galoisienne de $K$ (\ref{2-4}). 










\subsection{Topologie étale}\label{2-1}

Nous commencerons par quelques rappels sur la notion de morphisme étale. 

\begin{definition}\label{2-1-1}
Soit $A$ un anneau (commutatif). On dit qu'une $A$-algèbre $B$ est étale si 
$B$ est une $A$-algèbre de présentation finie et si les conditions 
équivalentes suivantes sont vérifiées:
\begin{enumerate}[\indent a)]
  \item Pour toute $A$-algèbre $C$ et pour tout idéal de carré nul $J$ de 
    $C$, l'application canonique 
    \[
      \hom_{A\textnormal{-alg}}(B,C) \to \hom_{A\textnormal{-alg}}(B,C/J)
    \]
    est une bijection.
  \item $B$ est un $A$-module plat et $\Omega_{B/A}=0$ (on note $\Omega_{B/A}$ 
    le module des différentielles relatives).
  \item Soit $B=A[X_1,\dotsc,X_n]/I$ une présentation de $B$. Alors pour tout 
    idéal premier $\fr$ de $A[X_1,\dotsc,X_n]$ contenant $I$, il existe des 
    polynômes $P_1,\dotsc,P_n\in I$ tels que $I_\fr$ soit engendré par les 
    images de $P_1,\dotsc,P_n$ et $\det(\partial P_i/\partial X_j)\notin \fr$.  
\end{enumerate}
\end{definition}
(cf. \cite[I]{7} ou \cite[V]{11})
%[cf. SGA I, exposé I ou M. {\sc Raynoud}, \emph{Anneaux Locaux Henséliens}, chapitre V]. 

On dit qu'un morphisme de schémas $f:X\to S$ est \emph{étale} si pour tout 
$x\in X$ il existe un voisinage ouvert affine $U=\spec(A)$ de $f(x)$ et un 
voisinage ouvert affine $V=\spec(B)$ de $x$ dans $X\times_S U$ tel que $B$ soit 
une $A$-algèbre étale. 





\subsubsection{Exemples}\label{2-1-2}
\begin{enumerate}[\indent a)]
  \item Si $A$ est un corps, une $A$-algèbre $B$ est étale si et seulement 
    si c'est un produit fini d'extensions séparables de $A$. 
  \item Si $X$ et $S$ sont des schémas de type fini sur $\dC$, un morphisme 
    $f:X\to S$ est étale si et seulement si son analyticité 
    $f^{\text{an}}:X^{\text{an}}\to Y^{\text{an}}$ est un isomorphisme local.
\end{enumerate}





\subsubsection{Sorite}\label{2-1-3}
\begin{enumerate}[\indent a)]
  \item (changement de base) Si $f:X\to S$ est un morphisme étale, il en est 
    de même de $f_{S'}:X\times_S S'\to S'$ pour tout morphisme $S'\to S$. 
  \item Si $f:X\to S$ et $g:Y\to S$ sont deux morphismes étales, tout 
    $S$-morphisme de $X$ dans $Y$ est étale.
  \item (descente) Soit $f:X\to S$ un morphisme. S'il existe un morphisme 
    fidèlement plat $S'\to S$, tel que $f_{S'}:X\times_S S'\to S'$ soit 
    étale, alors $f$ est étale.
\end{enumerate}





\subsubsection{}\label{2-1-4}

Soit $X$ un schéma. Soit $\sS$ la catégorie des $X$-schémas étales; 
d'après (\ref{2-1-3}.c) tout morphisme de $\sS$ est un morphisme étale. 
On appelle \emph{topologie étale} sur $\sS$ la topologie pour laquelle un 
crible sur $U$ est couvrant s'il est engendré par une famille finie de 
morphismes $\varphi_i:U_i\to U$ tels que la réunion des images des 
$\varphi_i$ recouvre $U$. On appelle \emph{site étale} de $X$, et note 
$\et X$, le site défini par $\sS$ de la topologie étale. 










\subsection{Exemples de faisceaux}\label{2-2}





\subsubsection{Faisceau constant}\label{2-2-1}

Soit $C$ un groupe abélien et supposons pour simplifier $X$ noethérien. On 
notera $\const C_X$ (ou même $C$ s'il n'y a pas d'ambiguïté) le faisceau 
défini par $U\mapsto C^{\pi_0(U)}$, où $\pi_0(U)$ est l'ensemble (fini) des 
composantes connexes de $U$. Le cas le plus important sera $C=\dZ/n$. On a donc 
par définition 
\[
  \h^0(X,\dZ/n)=\left(\dZ/n\right)^{\pi_0(X)}\text{.}
\]
De plus $\h^1(X,\dZ/n)$ est l'ensemble des classes d'isomorphisme de 
$\dZ/n$-torseurs (\ref{1-6-6}), autrement dit de revêtements étales 
galoisiens de $X$ de groupe $\dZ/n$. En particulier, si $X$ est connexe et si 
$\pi_1(X)$ est son groupe fondamental pour un pointe bas choisi, on a 
\[
  \h^1(X,\dZ/n) = \hom(\pi_1(X),\dZ/n)\text{.}
\]





\subsubsection{Groupe multiplicatif}\label{2-2-2}

On notera $\dG_{m,X}$ (ou $\dG_m$ s'il n'y a pas d'ambiguïté) le faisceau 
défini par $U\mapsto \Gamma(U,\cO_U^\times)$; il s'agit bien d'un faisceau 
grâce au théorème de descente fidèlement plate (\ref{1-4-5}). On a 
par définition 
\[
  \h^0(X,\dG_m) = \h^0(X,\cO_X)^\times\text{;}
\]
en particulier si $X$ est réduit, connexe et propre sur un corps 
algébriquement clos $k$, on a:
\[
  \h^0(X,\dG_m) = k^\times\text{.}
\]

\begin{proposition}\label{2-2-3}
On a un isomorphisme:
\[
  \h^1(X,\dG_m) = \pic(X)\text{,}
\]
où $\pic(X)$ est le groupe des classes de faisceaux inversibles sur $X$.
\end{proposition}
\begin{proof}
Soit $*$ le foncteur qui, à un faisceau inversible $\cL$ sur $X$, associe le 
préfaisceau $\cL^*$ suivant sur $\et X$: pour $\varphi:U\to X$ étale, 
\[
  \cL^*(U)=\operatorname{Isom}_U(\cO_U,\varphi^*\cL)\text{.}
\]
% the original references here are just to (4.2) and (4.5), but the context 
% makes it seem that they refer to the results in chapter 1. Likewise a little 
% later
D'après (\ref{1-4-2}.i) et et (\ref{1-4-5}) (pleine fidélité), ce 
préfaisceau est un faisceau; c'est même un $\dG_m$-torseur. On vérifie 
aussitôt que 
\begin{enumerate}[\indent a)]
  \item le foncteur $*$ est compatible à la localisation (étale);
  \item il induit une équivalence de la catégorie des faisceaux 
    inversibles triviaux (i.e. à $\cO_X$) avec la catégorie des 
    $\dG_m$-torseurs triviaux: $\cL$ est trivial si et seulement si $\cL^*$ 
    l'est.
\end{enumerate}

De plus, d'après (\ref{1-4-2}.ii) et (\ref{1-4-5}), 
\begin{enumerate}[\indent a)]
\setcounter{enumi}{2}
  \item la notion de faisceau inversible est locale pour la topologie étale. 
\end{enumerate}

Il résulte formellement de a), b), c) que $*$ est une équivalence entre la 
catégorie des faisceaux inversibles sur $X$ est celle des $\dG_m$-torseurs 
sur $\et X$; elle induit l'isomorphisme cherché. On construit comme suit 
l'équivalence inverse: si $\cT$ est un $\dG_m$-torseur, il existe un 
recouvrement étale fini $\{U_i\}$ de $X$ tel que les torseurs $\cT/U_i$ soient 
triviaux; $\cT$ est alors trivial sur chaque $V$ étale sur $X$ appartenant au 
crible $\sU\subset \et X$ engendré par $\{U_i\}$. Sur chaque 
$V\in\sU$, $\cT|V$ correspond à un faisceau inversible $\cL_V$ (par b)) et les 
$\cL_V$ constituent un faisceau inversible donné $\sU$-localement $\cL_\sU$ 
(par a)). Par c), ce dernier provient d'un faisceau inversible $\cL(\cT)$ sur 
$X$, et $\cT\mapsto \cL(\cT)$ est l'inverse cherché de $*$. 
\end{proof}





\subsubsection{Racines de l'unité}\label{2-2-4}

Pour tout entier $n>0$, on appelle faisceau des racines $n$-ièmes de 
l'unité, et on note $\dmu_n$, le noyau de l'élévation à puissance 
$n$-iéme dans $\dG_m$. Si $X$ est un schéma sur un corps séparablement 
clos $k$ et si $n$ est inversible dans $k$, le choix d'une racine primitive 
$n$-ième de l'unité $\zeta\in k$ définit un isomorphisme 
$i\mapsto \zeta^i$ de $\dZ/n$ avec $\dmu_n$. 

La relation entre cohomologie à coefficients dans $\dmu_m$ et cohomologie à 
coefficients dans $\dG_m$ est donnée par la suite exacte de cohomologie déduite 
de la 






\begin{theorem}[Théorie de Kummer]\label{2-2-5}
Si $n$ est inversible sur $X$, l'élévation à la puissance $n$-ième dans 
$\dG_m$ est un épimorphisme de faisceaux. On a donc une suite exacte 
\[
  0 \to \dmu_n \to \dG_m \to \dG_m \to 0\text{.}
\]
\end{theorem}
\begin{proof}
Soient $U\to X$ un morphisme étale et $a\in \dG_m(U) =\Gamma(U,\cO_U^\times)$. 
Puisque $n$ est inversible sur $U$, l'équation $T^n-a = 0$ est séparable; 
autrement dit $U'=\spec\left(\cO_U[T]/(T^n-a)\right)$ est étale su-dessus de 
$U$. Par ailleurs $U'\to U$ est surjectif et a admet une racine $n$-ième sur 
$U'$, d'où résultat. 
\end{proof}










\subsection{Fibres, images directes}\label{2-3}





\subsubsection{}\label{2-3-1}

On appelle \emph{pointe géométrique} de $X$ un morphisme $\bar x\to X$, où 
$\bar x$ est le spectre d'un corps séparablement clos $k(\bar x)$. On le notera 
abusivement $\bar x$, sous-entendent le morphisme $\bar x\to X$. Si $x$ est 
l'image de $\bar x$ dans $X$, on dit que $\bar x$ est centré en $x$. Si le 
corps $k(\bar x)$ est une extension algébrique du corps résiduel $k(x)$, on dit 
que $\bar x$ est un point géométrique \emph{algébrique} de $X$. 

On appelle \emph{voisinage étale} de $\bar x$ un diagramme commutatif 
\[\xymatrix{
  & U \ar[d] \\
  \bar x \ar[ur] \ar[r] 
  & X\text{,}
}\]
où $U\to X$ est un morphisme étale. 

Le \emph{localisé stricte} de $X$ en $\bar x$ est l'anneau 
$\cO_{X,\bar x} = \varinjlim \Gamma(U,\cO_U)$, la limite inductive étant sur les 
voisinages étales de $\bar x$. C'est un anneau local strictement hensélien dont 
le corps résiduel est la clôture séparable du corps résiduel $k(x)$ de $X$ en 
$x$ dans $k(\bar x)$. Il joue le rôle d'anneau local pour la topologie étale. 





\subsubsection{}\label{2-3-2}

Étant donné un faisceau $\cF$ sur $\et X$, on appelle \emph{fibre} de $\cF$ en 
$\bar x$ l'ensemble (resp. le groupe,\dots) $\cF_{\bar x}=\varprojlim \cF(U)$, 
la limite inductive étant toujours prise sur les voisinages étales de  $X$. 

Pour qu'un homomorphisme de faisceaux $\cF\to \cG$ soit un 
mono-/epi-/isomorphisme il faut et il suffit qu'il en soit ainsi des morphismes 
$\cF_{\bar x}\to \cG_{\bar x}$ induit sur les fibres et tout point géométrique 
de $X$. Si $X$ est de type fini sur un corps algébriquement clos, il suffit 
qu'il en soit ainsi en les points rationnelles de $X$. 





\subsubsection{}\label{2-3-3}

Si $f:X\to Y$ est un morphisme de schémas et $\cF$ un faisceau sur $\et X$, 
l'\emph{image directe} $f_* \cF$ de $\cF$ par $f$ est le faisceau sur $\et Y$ 
défini par $f_* \cF(V) = \cF(X\times_Y V)$ pour tout $V$ étale sur $Y$. Le foncteur 
$f_*:\ab\sh(\et X)\to \ab\sh(\et Y)$ est exact à gauche. Ses foncteurs 
dérivés à droite $\R^q f_*$ s'appellent images directes supérieures. Si 
$\bar y$ est un point géométrique de $Y$, on a 
\[
  (\R^q f_* \cF)_{\bar y} = \varinjlim \h^q(V\times_Y X,\cF)\text{,}
\]
limite inductive prise sur les voisinages étales $V$ de $\bar y$. 

Soient $\cO_{Y,\bar y}$ le localisé stricte de $Y$ en $\bar y$, 
$\widetilde Y=\spec(\cO_{Y,\bar y})$ et $\widetilde X=X\times_Y \widetilde Y$. 
On peut étendre $\cF$ à $\et{\widetilde X}$ (c'est un cas particulier de la 
notion générale d'image réciproque) de la manière suivante: soit 
$\widetilde U$ un schéma étale sur $\widetilde X$, alors il existe un 
voisinage étale $V$ de $\bar y$ et un schéma étale $U$ sur $X\times_Y V$ tel 
que $\widetilde U=U\times_V \widetilde Y$; on posera 
\[
  \cF(\widetilde U)=\varinjlim \cF\left(U\times_V V'\right)\text{,}
\]
le limite inductive étant prise sur les voisinages étales $V'$ de $\bar y$ qui 
dominent $V$. Avec cette définition, on a 
\[
  \left(\R^q f_* \cF\right)_{\bar y} = \h^q(\widetilde X,\cF)\text{.}
\]

Le foncteur $f_*$ a un adjoint à gauche $f^*$, le foncteur ``image 
réciproque.'' Si $\bar x$ est un point géométrique de $X$ et $f(\bar x)$ son 
image dans $Y$, on a $(f^* \cF)_{\bar x} = \cF_{f(\bar x)}$. Cette formule montre 
que $f^*$ est un foncteur exact. Le foncteur $f_*$ transforme donc faisceau 
injectif en faisceau injectif, et la suite spectrale du foncteur compose 
$\Gamma\circ f_*$ (resp. $g_* f_*$) fournit la 





\begin{theorem}[Suite spectrale de Leray]\label{2-3-4}
Soient $\cF$ un faisceau abélien sur $\et X$ et $f:X\to Y$ un morphisme de 
schémas (resp. des morphismes de schémas $X\xrightarrow f Y \xrightarrow g Z$). 
On a une suite spectrale 
\begin{align*}
  E_2^{pq} &= \h^p(Y,\R^q f_* \cF) \Rightarrow \h^{p+q}(X,\cF) \\
  \text{(resp.}\qquad E_2^{pq} &= \R^p g_* \R^q f_* \cF \Rightarrow \R^{p+q}(gf)_* \cF\text{).}
\end{align*}
\end{theorem}





\begin{corollary}\label{2-3-5}
Si $\R^q f_* \cF=0$ pour tout $q>0$, on a $\h^p(Y,f_* \cF)=\h^p(X,\cF)$ (resp. 
$\R^p g_*(\cF_* \cF) = \R^p(g f)_* \cF$) pour tout $p\geqslant 0$. 
\end{corollary}

Cela s'applique en particulier dans le cas suivant:





\begin{proposition}\label{2-3-6}
Soit $f:X\to Y$ un morphisme fini (voire, par passage à la limite, un 
morphisme entier) et $\cF$ un faisceau abélien sur $X$. Alors $\R^q f_* \cF=0$, 
pour tout $q>0$.
\end{proposition}

En effet soient $\bar y$ un pointe géométrique de $Y$, $\widetilde Y$ le 
spectre du localisé strict de $Y$ en $y$ et 
$\widetilde X=X\times_Y \widetilde Y$; d'après ce qui précède, il suffit de 
montrer que $\h^q(\widetilde X,\cF) = 0$ pour tout $q>0$. Or $\widetilde X$ est le 
spectre d'un produit d'anneaux locaux strictement henséliens (cf. 
\cite[I]{11}), le foncteur $\Gamma(\widetilde X,-)$ est exact car tout 
$\widetilde X$-schéma étale et surjectif admet une section, d'où l'assertion. 










\subsection{Cohomologie galoisienne}\label{2-4}

Pour $X=\spec(K)$ le spectre d'un corps, nous allons voir que la cohomologie 
étale s'identifie à la cohomologie galoisienne. 





\subsubsection{}\label{2-4-1}

Commençons par une analogie topologique. Si $K$ est le corps des fonctions 
d'une variété algébrique affine intègre $Y=\spec(A)$ sur $\dC$, on a 
$K=\varprojlim_{f\in A} A[1/f]$. 

Autrement dit $X=\varprojlim U$, $U$ parcourant l'ensemble des ouverts de $Y$. 
On sait qu'il existe des ouverts de Zariski arbitrairement petits qui pour la 
topologie classique sont des $K(\pi,1)$. On ne sera donc pas surpris si l'on 
considère $\spec(K)$ lui-même comme un $K(\pi,1)$, $\pi$ étant le groupe 
fondamental (au sens algébrique) de $X$, autrement dit le groupe de Galois de 
$\bar K/K$, où $\bar K$ est clôture séparable de $K$. 





\subsubsection{}\label{2-4-2}

Plus précisément soient $K$ un corps, $\bar K$ un clôture séparable de $K$ et 
$G = \gal(\bar K/K)$ le groupe de Galois topologique. A toute $K$-algèbre finie 
étale $A$ (produit fini d'extensions séparables de $K$), associons l'ensemble 
fini $\hom_K(A,\bar K)$. Le groupe de Galois $G$ opère sur cet ensemble à 
travers un quotient discret (donc fini). Si $A=K[T]/(F)$, il s'identifie à 
l'ensemble des racines dans $\bar K$ du polynôme $F$. La théorie de Galois, 
sous la forme que lui a donnée Grothendieck, dit que:





\begin{proposition}\label{2-4-3}
Le foncteur 
\[
  \left\{\begin{array}{c}
          \textnormal{$K$-algèbres} \\ 
          \textnormal{finies étales}
        \end{array}\right\}
  \to 
  \left\{\begin{array}{c}
          \textnormal{ensembles finis sur lesquels} \\
          \textnormal{$G$ opère continûment}
        \end{array}\right\}
\]
qui à une algèbre étale $A$ associe $\hom_K(A,\bar K)$ est une 
anti-équivalence de catégories.
\end{proposition}

On en déduit une description analogue des faisceaux pour la topologie étale sur 
$\spec(K)$:





\begin{proposition}\label{2-4-4}
Le foncteur 
\[
  \left\{\begin{array}{c}
          \textnormal{Faisceaux étales} \\ 
          \textnormal{sur $\spec(K)$}
        \end{array}\right\}
  \to 
  \left\{\begin{array}{c}
          \textnormal{ensembles sur lesquels} \\
          \textnormal{$G$ opère continûment}
        \end{array}\right\}
\]
qui à un 
faisceau $\cF$ associe sa fibre $\cF_{\bar K}$ au point géométrique 
$\spec(\bar K)$ est une équivalence de catégories.
\end{proposition}

On dit que $G$ opère continûment sur un ensemble $E$ si le fixateur de tout 
élément de $E$ est un sous-groupe ouvert de $G$. Le foncteur en sens 
inverse est décrit de la manière évidente: soient $A$ une 
$K$-algèbre finie étale, $U=\spec(A)$ et $U(\bar K)=\hom_K(A,\bar K)$ le 
$G$-ensemble correspondant à; alors on a 
$\cF(U)= \hom_{G\text{-ens}}(U(\bar K),\cF_{\bar K})$. 

En particulier, si $X=\spec(K)$, on a $\cF(X) = \cF_{\bar K}^G$. Si l'on se 
restreint aux faisceaux abéliens, on obtient en passant aux foncteurs 
dérivés des isomorphismes canoniques 
\[
  \h^q(\et X,\cF) = \h^q(G,\cF_{\bar K})
\]





\subsubsection{Exemples}\label{2-4-5}

\begin{enumerate}[\indent a)]
  \item Au faisceau constant $\dZ/n$ correspond $\dZ/n$ avec action triviale de 
    $G$. 
  \item Au faisceau des racines $n$-ièmes de l'unité $\dmu_n$ correspond 
    le groupe $\dmu_n(\bar K)$ des racines $n$-ièmes de l'unité dans 
    $\bar K$, avec l'action naturelle de $G$.
  \item Au faisceau $\dG_m$ correspond le groupe $\bar K^\times$ avec l'action 
    naturelle de $G$.
\end{enumerate}




















\section{Cohomologie des courbes}\label{3}

Dans le cas des espaces topologiques, des dévissages utilisant la formule de 
K\"unneth et des décompositions simpliciales permettent de se ramener pour 
calculer la cohomologie à l'intervalle $I=[0,1]$ pour lequel on a 
$\h^0(I,\dZ)=\dZ$ et $\h^q(I,\dZ)=0$ pour $q>0$. 

Dans notre cas, les dévissages aboutiront à des objets plus compliqués, 
à savoir les courbes sur un corps algébriquement clos; nous allons calculer 
leur cohomologie dans ce chapitre. La situation est plus complexe que dans le 
cas topologique car les groupes de cohomologie sont nuis pour $q>2$ seulement. 
L'ingrédient essentiel des calculs est la nullité du groupe de Brauer du 
corps des fonctions d'une telle courbe (théorème de Tsen, \ref{3-2}). 










\subsection{Le groupe de Brauer}\label{3-1}

Rappelons-en tout d'abord la définition classique:





\begin{definition}\label{3-1-1}
Soit $K$ un corps et $A$ une $K$-algèbre de dimension finie. On dit que $A$ 
est une algèbre simple centrale sur $K$ si les conditions équivalentes 
suivantes sont vérifiées:
\begin{enumerate}[\indent a)]
  \item $A$ n'a pas d'idéal bilatère non trivial et son centre est $K$. 
  \item Il existe une extension galoisienne finie $K'/K$ telle que 
    $A_{K'} = A\otimes_K K'$ soit isomorphe à une algèbre de matrices 
    carrées sur $K'$.
  \item $A$ est $K$-isomorphe à algèbre de matrices carrées sur un corps 
    gauche de centre $K$.
\end{enumerate}
\end{definition}

Deux telles algèbres sont dites équivalentes si les corps gauches qui 
leur sont associés par c) sont $K$-isomorphes. Si ces algèbres out 
même dimension, cela revient à dire qu'elles sont $K$-isomorphes. Le 
produit tensoriel définit par passage au quotient une structure de groupe 
abélien sur l'ensemble des classes d'équivalence. C'est ce groupe que l'on 
appelle classiquement \emph{le groupe de Brauer} de $K$ et que l'on note 
$\br(K)$. 





\subsubsection{}\label{3-1-2}

On notera $\br(n,K)$ l'ensemble des classes de $K$-isomorphisme de 
$K$-algèbres $A$ telles qu'il existe une extension galoisienne finie $K'$ de 
$K$ pour laquelle $A_{K'}$ est isomorphe à l'algèbre $M_n(K')$ des matrices 
carrées $n\times n$ sur $K'$. Par définition $\br(K)$ est réunion des 
sous-ensembles $\br(n,K)$ pour $n\in\dN$. Soient $\bar K$ une clôture 
algébrique de $K$ et $G=\gal(\bar K/K)$. L'ensemble $\br(n,K)$ est 
l'ensemble des ``formes'' de $M_n(\bar K)$, il est donc canoniquement 
isomorphe à $\h^1\left(G,\aut(M_n(\bar K))\right)$. 

On sait que tout automorphisme de $M_n(\bar K)$ est intérieur. Par 
conséquent le groupe $\aut(M_n(\bar K))$ s'identifie au groupe linéare 
projectif $\pgl(n,\bar K)$ et on a une bijection canonique:
\[
  \theta_n : \br(n,K) \iso \h^1\left(G,\pgl(n,\bar K)\right)\text{.}
\]

D'autre part la suite exacte:
\begin{equation}\label{eq:1}
  1 \to \bar K^\times \to \gl(n,\bar K) \to \pgl(n,\bar K) \to 1\text{,}
\end{equation}
permet de définir un opérateur cobord:
\[
  \Delta_n : \h^1\left(G,\pgl(n,\bar K)\right) \to \h^2(G,\bar K^\times)\text{.}
\]

En composant $\theta_n$ et $\Delta_n$, on obtient une application:
\[
  \delta_n : \br(n,K) \to \h^2(G,\bar K^\times)\text{.}
\]
On vérifie facilement que les applications $\delta_n$ sont compatibles entre 
elles et définissent un homomorphisme de groupes:
\[
  \delta : \br(K) \to \h^2(G,\bar K^\times)\text{.}
\]





\begin{proposition}\label{3-1-3}
L'homomorphisme $\delta:\br(K)\to \h^2(G,\bar K^\times)$ est bijectif. 
\end{proposition}

Cela résulte des deux lemmes suivants:





\begin{lemma}\label{3-1-4}
L'application 
$\Delta_n:\h^1\left(G,\pgl(n,\bar K)\right)\to \h^2(G,\bar K^\times)$ est 
injective.
\end{lemma}

D'après \cite{14}, cor. à la prop. I-44, il suffit de vérifier que chaque 
fois qu'on tord la suite exacts (\ref{eq:1}) par un élément de 
$\h^1\left(G,\pgl(n,\bar K)\right)$, le $\h^1$ du groupe médian est trivial. 
Ce groupe médian est le groupe des $\bar K$-points du groupe multiplicatif 
d'une algèbre centrale simple $A$ de rang $n^2$ sur $K$. Pour prouver que 
$\h^1(G,A_{\bar K}^\times) = 0$, n interprète $A^\times$ comme le groupe 
des automorphismes du $A$-module libre $L$ de rang $1$, et $\h^1$ comme 
l'ensemble des ``formes'' de $L$ -- des $A$-modules de rang $n^2$ sur $K$, 
automatiquement libres. 





\begin{lemma}\label{3-1-5}
Soient $\alpha\in \h^2(G,\bar K^\times)$, $K'$ un extension finie de $K$ 
contenue dans $\bar K$, $n=[K':K]$, et $G'=\gal(\bar K/K')$. Si l'image de 
$\alpha$ dans $\h^2(G',\bar K^\times)$ est nulle, alors, $\alpha$ appartient 
à l'image de $\Delta_n$. 
\end{lemma}

Remarquons tout d'abord qu'on a: 
\[
  \h^2(G',\bar K^\times) \simeq \h^2\left(G,(\bar K\otimes_K K')^\times\right)\text{.}
\]
(D'un point de vue géométrique si l'on note $x=\spec(K)$, $x'=\spec(K')$ et 
$\pi:x'\to x$ le morphisme canonique, on a $\R^q \pi_*(\dG_{m,x'}) = 0$ pour 
$q>0$ et par suite $\h^q(x',\dG_{m,X'})\simeq \h^q(x,\pi_*\dG_{m,X'})$ pour 
$q\geqslant 0$). 

Par ailleurs la choix d'une base de $K'$ en tant qu'espace vectoriel sur $K$ 
permet de définir un homomorphisme 
\[
  (\bar K\otimes_K K')^\times \to \gl(n,\bar K)
\]
qui, à un élément $x$, fait correspondre l'endomorphisme de 
multiplication par $x$ de $\bar K\otimes_K K'$. On a alors un diagramme 
commutatif à lignes exactes: 
\[\xymatrix{
  1 \ar[r] 
    & \bar K^\times \ar[r] \ar@{=}[d]
    & (\bar K\otimes_K K')^\times \ar[r] \ar[d] 
    & (\bar K\otimes_K K')^\times / \bar K^\times \ar[r] \ar[d] 
    & 1 \\
  1 \ar[r] 
    & \bar K^\times \ar[r] 
    & \gl(n,\bar K) \ar[r] 
    & \pgl(n,\bar K) \ar[r] 
    & 1
}\]
Le lemme résulte du diagramme commutatif que l'on en déduit en passant 
à la cohomologie:
\[\xymatrix{
  \h^1\left(G,(\bar K\otimes_K K')^\times/\bar K^\times\right) \ar[r] \ar[d] 
    & \h^2(G,\bar K^\times) \ar[r] \ar@{=}[d]
    & \h^2\left(G,(\bar K\otimes_K K')^\times\right) \\
  \h^1\left(G,\pgl(n,\bar K)\right) \ar[r]^-{\Delta_n} 
    & \h^2(G,\bar K^\times) \text{.}
}\]





\begin{proposition}\label{3-1-6}
Soient $K$ un corps, $\bar K$ une clôture algébrique de $K$ et 
$G=\gal(\bar K/K)$. Supposons que, pour tout extension finie $K'$ de $K$, on 
ait $\br(K')=0$. Alors on a:
\begin{enumerate}[\indent i)]
  \item $\h^q(G,\bar K^\times) = 0$ pour tout $q>0$.
  \item $\h^q(G,\cF) = 0$ pour tout $G$-module de torsion $\cF$ et pour tout 
    $q\geqslant 2$.
\end{enumerate}
\end{proposition}

(Pour la démonstration, cf. \cite{14}).










\subsection{Le théorème de Tsen}\label{3-2}





\begin{definition}\label{3-2-1}
On dit qu'un corps $K$ est $C_1$ si tout polynôme homogène non constant 
$f(x_1,\dotsc,x_n)$ de degré $d<n$ a un zéro non trivial.
\end{definition}





\begin{proposition}\label{3-2-2}
Si un corps $K$ est $C_1$, on a $\br(K)=0$.
\end{proposition}

Il s'agit de montrer que tout corps gauche $D$ de centre $K$ et fini sur $K$ 
est égale à $K$. Soient $r^2$ le degré de $D$ sur $K$ et $\nrd:D\to K$ la 
norme réduite. 

(Localement pour la topologie étale sur $K$, $D$ est isomorphe -- non 
canoniquement -- à une algèbre de matrices $M_r$ et la norme réduite 
coïncide avec l'application déterminant. Celle-ci est bien définie, 
indépendamment de l'isomorphisme choisi entre $D$ et $M_r$ car tout 
automorphisme de $M_r$ est intérieur et deux matrices semblables ont 
même déterminant. Cette application définie localement pour la 
topologie étale se descende, à cause de son unicité locale, en une 
application $\nrd:D\to K$). 

Le seul zéro de $\nrd$ est l'élément nul de $D$, car, si $x\ne 0$, on a 
$\nrd(x)\cdot \nrd(x^{-1}) = 1$. D'autre part, si $\{e_1,\dotsc,e_{r^2}\}$ est 
une base de $D$ sur $K$ et si $x=\sum x_i e_i$, la fonction $\nrd(x)$ s'écrit 
comme un polynôme homogène $\nrd(x_1,\dotsc,x_{r^2})$ de degré $r$ (c'est 
clair localement pour la topologie étale). Puisque $K$ est $C_1$, on a 
$r^2\leqslant r$, c'est-à-dire $r=1$ et $D=K$. 





\begin{theorem}[Tsen]\label{3-2-3}
Soient $k$ un corps algébriquement clos et $K$ une extension de degré de 
transcendance $1$ de $k$. Alors $K$ est $C_1$.
\end{theorem}

Supposons tout d'abord que $K=k(X)$. Soit 
\[
  f(T) = \sum a_{i_1,\dotsc i_n} T_1^{i_1} \dotsm T_n^{i_n}
\]
un polynôme homogène de degré $d<n$ à coefficients dans $k(X)$. Quitte 
à multiplier les coefficients par un dénominateur commun on peut supposer 
qu'ils sont dans $k[X]$. Soit alors $\delta=\sup\deg(a_{i_1\dotsc i_n})$. On 
cherche un zéro non trivial dans $k[X]$ par la méthode des coefficients 
indéterminés en écrivant chaque $T_i$ ($i=1,\dotsc,n$) comme un 
polynôme de degré $N$ en $X$. Alors l'équation $f(T)=0$ devient un système 
d'équations homogènes en les $n\times (N+1)$ coefficients des polynômes 
$T_i(X)$ exprimant la nullité des coefficients du polynôme en $X$ obtenu en 
remplaçant $T_i$ par $T_i(X)$. Ce polynôme est de degré $\delta+N D$ au plus, 
il y a donc $\delta+N d+1$ équations en $n\times (N+1)$ variables. Comme $k$ 
est algébriquement clos ce système a une solution non triviale si 
$n(N+1)>N d+\delta+1$, ce qui sera le cas pour $N$ assez grand si $d<n$. 

Il est clair que, pour démontrer le théorème dans la cas général, il 
suffit de le démontrer lorsque $K$ est une extension finie d'une extension 
transcendent pure $k(X)$ de $k$. Soit $f(T)=f(T_1,\dotsc,T_n)$ un 
polynôme homogène de degré $d<n$ à coefficients dans $K$. Soient 
$a=[K:k(X)]$ et $e_1,\dotsc,e_s$ une bas de $K$ sur $k(X)$. Introduisons de 
nouvelles variables $U_{i j}$, en nombre $s n$, telles que 
$T_i=\sum U_{i j}e_j$. Pour que le polynôme $f(T)$ ait un zéro 
non trivial dans $K$, il suffit que le polynôme 
$g(X_{i j}) = N_{K/k}(f( T))$ ait un zéro non trivial dans $k(X)$. 
Or $g$ est un polynôme de degré $s d$ en $s n$ variables, d'où le 
résultat. 





\begin{corollary}\label{3-2-4}
  Soient $k$ un corps algébriquement clos et $K$ une extension de degré 
de transcendance $1$ de $k$. Alors les groupes de cohomologie étale 
$\h^q(\spec(K),\dG_m)$ sont nuls pour tout $q>0$. 
\end{corollary}










\subsection{Cohomologie des courbes lisses}

Dorénavant, et sauf mention expresse du contraire, les groupes de 
cohomologie considérés sont les groupes des cohomologie étale. 





\begin{proposition}\label{3-3-1}
Soient $k$ un corps algébriquement clos et $X$ une courbe projective non 
singulière connexe sur $k$. Alors on a:
\begin{align*}
  \h^0(X,\dG_m) &= k^\times \text{,} \\
  \h^1(X,\dG_m) &= \pic(X) \text{,} \\
  \h^q(X,\dG_m) &= 0 \text{ pour $q\geqslant 2$.}
\end{align*}
\end{proposition}

Soient $\eta$ le point générique de $X$, $j:\eta\to X$ le morphisme 
canonique et $\dG_{m,X}$ le groupe multiplicatif du corps des fractions 
$K(X)$. Pour tout pointe fermé $x$ de $X$, soient $i_x:x\to X$ l'immersion 
canonique et $\dZ_x$ le faisceau constant de valeur $\dZ$ sur $x$. Ainsi 
$j_*\dG_{m,\eta}$ est le faisceau des fonctions méromorphes non nulles sur 
$X$ et $\oplus_{x\in X} i_{x_I}\dZ_x$ le faisceau des diviseurs, on a donc une 
suite exacte de faisceaux:
\begin{equation}\label{eq:2}
\xymatrix{
  0 \ar[r] 
    & \dG_m \ar[r]
    & j_* \dG_{m,\eta} \ar[r]^-{\dv} \ar[r] 
    & \bigoplus_{x\in X} i_{x*} \dZ_x \ar[r] 
    & 0\text{.}
}
\end{equation}





\begin{lemma}\label{3-3-2}
On a $\R^q j_* \dG_{m,\eta} = 0$ pour tout $q>0$.
\end{lemma}

Il suffit de montrer que la fibre de ce faisceau en tout pointe fermé $x$ 
de $X$ est nulle. Si $\tilde\cO_{X,x}$ est l'hensélisé de $X$ en $x$ et $K$ 
le corps des fractions de $\tilde\cO_{X,x}$, on a 
\[
  \spec(K) = \eta\times_X \spec(\tilde\cO_{X,x})\text{,}
\]
donc $(\R^qj_*\dG_{m,\eta})_x = \h^q(\spec(K),\dG_m)$. 

Or $K$ est une extension algébrique de $k(X)$, donc une extension de degré 
de transcendante $1$ de $k$: le lemme résulte de (\ref{3-2-4}). 





\begin{lemma}\label{3-3-3}
On a $\h^q(X, j_*\dG_{m,\eta}) = 0$ pour tout $q>0$.
\end{lemma}

En effet de (\ref{3-3-2}) et de la suite spectrale de Leray pour $j$, on 
déduit:
\[
  \h^q(X, j_* \dG_{m,\eta}) = \h^q(\eta,\dG_{m,\eta})
\]
pour tout $q\geqslant 0$ et le deuxième membre est nul pour $q>0$ d'après 
(\ref{3-2-4}). 





\begin{lemma}\label{3-3-4}
On a $\h^q\left(X,\bigoplus_{x\in X} i_{x*} \dZ_x\right) = 0$ pour tout $q>0$. 
\end{lemma}

En effet pour tout point fermé de $X$, on a $\R^q i_{x*}\dZ_x 0$ pour $q>0$, 
car $i_x$ est un morphisme fini (\ref{2-3-6}), et 
\[
  \h^q(X,i_{x*}\dZ_x) = \h^q(x,\dZ_x)\text{.}
\]
Le deuxième membre est nul pour tout $q>0$, car $x$ est la spectre d'un 
corps algébriquement clos (On voit que le lemme est vrai plus 
généralement pour tout faisceau ``gratte-ciel'' sur $X$). 

On déduit des lemmes précédents et de la suite exacte (\ref{eq:2}) 
les égalités:
\[
  \h^q(X,\dG_m) = 0 \text{ pour $q\geqslant 2$,}
\]
et une suite exacte de cohomologie en bas degré:
\[
  1 \to \h^0(X,\dG_m) \to \h^0(X, j_*\dG_{m,\eta}) \to \h^0\left(X,\bigoplus_{x\in X} i_{x*}\dZ_x\right) \to \h^1(X,\dG_m) \to 1
\]
qui n'est autre que la suite exacte:
\[
  1 \to k^\times \to k(X)^\times\to \Div(X) \to \pic(X) \to 1\text{.}
\]
De la proposition (\ref{3-3-1}) on déduit que les groupes de cohomologie de 
$X$ à valeur dans $\dZ/n$, $n$ premier à la caractéristique de $k$, ont 
une valeur raisonnable:





\begin{corollary}\label{3-3-5}
Si $X$ est de genre $g$ et si $n$ est inversible dans $k$, les 
$\h^q(X,\dZ/n)$ sont nuls pour $q>2$, et libres sur $\dZ/n$ de rang $1,2 g,1$ 
pour $q=0,1,2$. Remplaçant $\dZ/n$ par le groupe isomorphe $\dmu_n$, on a des 
isomorphismes canoniques 
\begin{align*}
  \h^0(X,\dmu_n) &= \dmu_n \\
  \h^1(X,\dmu_n) &= \pic^0(X)_n \\
  \h^2(X,\dmu_n) &= \dZ/n \text{.}
\end{align*}
\end{corollary}

Comme le corps $k$ est algébriquement clos, $\dZ/n$ est isomorphe (non 
canoniquement) à $\mu_n$. De la suite exacte de Kummer:
\[
  0 \to \mu_n \to \dG_m \to \dG_m \to 0\text{,}
\]
et de la proposition (\ref{3-3-1}), on déduit les égalité:
\[
  \h^q(X,\dZ/n) = 0 \text{ pour $q>2$,}
\]
et, en bas degré, des suites exactes:
\[\xymatrix{
  0 \ar[r] 
    & \h^0(X,\dmu_n) \ar[r] 
    & k^\times \ar[r]^-n 
    & k^\times \ar[r] 
    & 0
}\]
\[\xymatrix{
  0 \ar[r] 
    & \h^1(X,\dmu_n) \ar[r] 
    & \pic(X) \ar[r]^-n 
    & \h^2(X,\dmu_n) \ar[r] 
    & 0\text{.}
}\]
De plus on a une suite exacte:
\[\xymatrix{
  0 \ar[r] 
    & \pic^0(X) \ar[r] 
    & \pic(X) \ar[r]^-{\deg}
    & \dZ \ar[r] 
    & 0 \text{,}
}\]
et $\pic^0(X)$ s'identifie au groupe des points rationnels sur $k$ d'une 
variété abélienne de dimension $g$, la jacobienne de $X$. Dans un tel 
groupe, la multiplication par $n$ est surjective et son noyau est un 
$\dZ/n\dZ$-module libre de rang $2 g$ (car $n$ est inversible dans $k$); d'où 
le corollaire. 

De dévissage astucieux, utilisant la ``méthode de la trace,'' permet 
d'obtenir en corollaire la 





\begin{proposition}[{\cite[IX 5.7]{4}}]\label{3-3-6}
Soient $k$ un corps algébriquement clos, $X$ une courbe algébrique sur $k$ 
et $\cF$ un faisceau de torsion sur $X$. Alors:
\begin{enumerate}[\indent i)]
  \item On a $\h^q(X,\cF) = 0$ pour $q>2$.
  \item Si $X$ est affine, on a même $\h^q(X,\cF) = 0$ pour $q>1$.
\end{enumerate}
\end{proposition}

Pour la démonstration, ainsi que pour l'exposé de la ``méthode de la 
trace,'' nous renvoyons à \cite[IX 5]{4}.










\subsection{Dévissages}\label{3-4}

Pour calculer la cohomologie des variétés de dimension $>1$ on emploie des 
fibrations par des courbes, ce qui permet de se ramener à étudier les 
morphismes dont les fibres sont de dimension $\leqslant 1$. Ce principe 
possède plusieurs variantes, indiquons-en quelques-unes.





\subsubsection{}\label{3-4-1}

Soient $A$ une $k$-algèbre de type fini et $a_1,\dotsc,a_n$ des générateurs de 
$A$. Si l'on pose $X_0=\spec(k)$, $X_i=\spec(k[a_1,\dotsc,a_i])$, 
$X_n=\spec(A)$, les inclusions canoniques 
$k[a_1,\dotsc,a_i]\to k[a_1,\dotsc,a_i,a_{i+1}]$ définissent des morphismes 
$X_n \to X_{n-1} \to \cdots \to X_1 \to X_0$ dont les fibres sont de dimension 
$\leqslant 1$. 





\subsubsection{}\label{3-4-2}

Dans le cas d'un morphisme lisse, on peut être plus précis. On appelle 
\emph{fibration élémentaire} un morphisme de schémas $f:X\to S$ qui peut 
être plongé dans un diagramme commutatif
\[\xymatrix{
  X \ar@{^{(}->}[r]^-j \ar[dr]_-f
    & \bar X \ar[d]^-{\bar f}
    & \ar@{_{(}->}[l]_-i \ar[dl]^-g Y \\
  & S
}\]
satisfaisant aux conditions suivantes:
\begin{enumerate}[\indent i)]
  \item $j$ est une immersion ouverte dense dans chaque fibre et 
    $X=\bar X \setminus Y$. 
  \item $\bar f$ est lisse et projectif, à fibres géométriques 
    irréductibles et de dimension $1$.
  \item $g$ est un revêtement étale et aucune fibre de $g$ n'est vide.
\end{enumerate}

On appelle \emph{bon voisinage} relatif à $S$ un $S$-schéma $X$ tel qu'il 
existe $S$-schémas $X=X_n,\dotsc,X_0=S$ et des fibrations élémentaires 
$f_i:X_i\to X_{i-1}$, $i=1,\dotsc,n$. On peut montrer \cite[XI, 3.3]{4} que si 
$X$ est un schéma lisse sur un corps algébriquement clos $k$ tout point 
rationnel de $X$ possède un voisinage ouvert qui est un bon voisinage 
(relatif à $\spec(K)$). 





\subsubsection{}\label{3-4-3}

On peut dévisser un morphisme propre $f:X\to S$ de la façon suivante. 
D'après le lemme de Chow, il existe un diagramme commutatif
\[\xymatrix{
  X \ar[dr]_-f 
    & \ar[l]_-\pi \ar[d]^-{\bar f} \bar X \\
  & S
}\]
où $\pi$ et $\bar f$ sont des morphismes projectifs, $\pi$ étant de plus un 
isomorphisme au-dessus d'un ouvert dense de $X$. Localement sur $S$, $\bar X$ 
est un sous-schéma fermé d'un espace projectif type $\dP_S^n$. 

On dévisse ce dernier en considérant la projection 
$\varphi:\dP_S^n\to\dP_S^1$ qui envoie le point de coordonnées homogènes 
$(x_0,x_1,\dotsc,x_n)$ sur $(x_0,x_1)$. C'est une application rationnelle 
définie en dehors du fermé $Y\simeq \dP_S^{n-2}$ de $\dP_S^n$ d'équations 
homogènes $x_0=x_1=0$. Soit $u:P\to \dP_S^n$ l'éclatement à centre 
$Y$; les fibres de $u$ soit de dimension $\leqslant 1$. De plus il existe un 
morphisme naturel $v:P\to \dP_S^1$ qui prolonge l'application rationnelle 
$\varphi$ et $v$ fait de $P$ un $\dP_S^1$-schéma localement isomorphe à 
l'espace projectif type $\dP^{n-1}$ que l'on peut à son tour projeter sur un 
$\dP^1$, etc. 





\subsubsection{}\label{3-4-4}

On peut balayer une variété projective et lisse $X$ par un \emph{pinceau de 
Lefschetz}. L'éclaté $\widetilde X$ de l'intersection de l'axe du pinceau 
avec $X$ se projette sur $\dP^1$ et les fibres de cette projection sont les 
sections hyperplans de $X$ par les hyperplans du pinceau.




















\section{Théorème de changement de base pour un morphisme propre}\label{4}










\subsection{Introduction}\label{4-1}

Ce chapitre est consacré à la démonstration et aux applications du 





\begin{theorem}\label{4-1-1}
Soient $f:X\to S$ un morphisme propre de schémas et $\cF$ un faisceau abélien 
de torsion sur $X$. Alors, quel que soit $q\geqslant 0$, la fibre de 
$\R^q f_* \cF$ en un point géométrique $s$ de $S$ est isomorphe à la 
cohomologie $\h^q(X_s,\cF)$ de la fibre $X_s=X\otimes_S \spec k(s)$ de $f$ en 
$s$. 
\end{theorem}

Pour $f:X\to S$ une application continue propre et séparé (séparée 
signifie que la diagonale de $X\times_S X$ est fermée) entre espaces 
topologiques, et $\cF$ un faisceau abélien sur $X$, le résultat analogue est 
bien connu, et élémentaire: comme $f$ est fermée, les $f^{-1}(V)$ pour 
$V$ voisinage de $s$ forment un système fondamental de voisinages de $X_s$, et 
on vérifie que $\h^*(X_s,\cF) = \varinjlim_U \h^*(U,\cF)$, pour $U$ parcourant les 
voisinages de $X_s$. En pratique, $X_s$ a même un système fondamental 
$\sU$ de voisinages $U$ dont il est rétracte par déformation et, pour $\cF$ 
constant, on a donc $\h^*(X_s,\cF)=\h^*(U,\cF)$. En termes imagés: la fibre 
spéciale avale la fibre générale. 

Dans le cas des schémas la démonstration est plus délicate et il est 
indispensable de supposer que $\cF$ est de torsion (\cite[XII.2]{4}). Compte 
tenu de la description des fibres de $\R^q f_* \cF$ (\ref{2-3-3}), le théorème 
(\ref{4-1-1}) est essentiellement équivalent au 

 
  
   
    
\begin{theorem}\label{4-1-2}
Soient $A$ un anneau local strictement hensélien et $S=\spec(A)$. Soient 
$f:X\to S$ un morphisme propre et $X_0$ la fibre fermée de $f$. Alors, pour 
tout faisceau abélien de torsion $\cF$ sur $X$ et pour tout $q\geqslant 0$, on 
a $\h^q(X,\cF) \iso \h^q(X_0,\cF)$. 
\end{theorem} 

Par passage à la limite on voit qu'il suffit de démontrer le théorème 
lorsque $A$ est l'hensélisé strict d'une $\dZ$-algèbre  de type fini en 
un idéal premier. On traite d'abord le cas $q=0$ ou $1$ et $\cF=\dZ/n$ 
(\ref{4-2}). Un argument basé sur la notion de faisceau constructible 
(\ref{4-3}) montre d'ailleurs qu'il suffit de considérer le cas où $\cF$ est 
constant. D'autre part le dévissage (\ref{3-4-3}) permet de supposer que 
$X_0$ est une courbe; dans ce cas il ne reste plus qu'à démontrer le 
théorème pour $q=2$ (\ref{4-4}). 

Entre autres applications (\ref{4-6}), le théorème permet de définir la notion 
de cohomologie à support propre (\ref{4-5}). 










\subsection{Démonstration pour \texorpdfstring{$q=0$}{q=1} or \texorpdfstring{$1$}{1} et \texorpdfstring{$\cF=\dZ/n$}{\cF=Z/n}}\label{4-2} 

Le résultat pour $q=0$ et $\cF$ constant est équivalent à la proposition 
suivante (théorème de connexion de Zariski):





\begin{proposition}\label{4-2-1}
Soient $A$ un anneau local hensélien noethérien et $S=\spec(A)$. 
Soient $f:X\to S$ un morphisme propre et $X_0$ la fibre fermée de $f$. Alors 
les ensembles de composantes connexes $\pi_0(X)$ et $\pi_0(X_0)$ sont en 
bijection.
\end{proposition}

Il revient au même de montrer que les ensembles de parties à la fois 
ouvertes et fermées $\operatorname{Of}(X)$ et $\operatorname{Of}(X_0)$ sont 
en bijection. On sait que l'ensemble $\operatorname{Of}(X)$ correspond 
bijectivement à l'ensemble des idempotents de $\Gamma(X,\cO_X)$, de même 
$\operatorname{Of}(X_0)$ correspond bijectivement à l'ensemble des 
idempotents de $\Gamma(X_0,\cO_{X_0})$. Il s'agit donc de montrer que 
l'application canonique 
\[
  \operatorname{Idem} \Gamma(X,\cO_X) \to \operatorname{Idem}\Gamma(X_0,\cO_{X_0})
\]
est bijective. 

On notera $\fm$ l'idéal maximal de $A$, $\Gamma(X,\cO_X)^\wedge$ le complété 
de $\Gamma(X,\cO_X)$ pour la topologie $\fm$-adique et, pour tout entier 
$n\geqslant 0$, $X_n=X\otimes_A A/\fm^{n+1}$. D'après le théorème de finitude 
pour les morphismes propres \cite[III.3.2]{8}, $\Gamma(X,\cO_X)$ est une 
$A$-algèbre finie; comme $A$ est hensélien, il en résulte que 
l'application canonique 
\[
  \operatorname{Idem} \Gamma(X,\cO_X) \to \operatorname{Idem}\Gamma(X,\cO_X)^\wedge
\]
est bijective. En particulier l'application canonique 
\[
  \operatorname{Idem} \Gamma(X,\cO_X)^\wedge \to \varprojlim \operatorname{Idem}\Gamma(X_n,\cO_{X_n})
\]
est bijective. Mais, puisque $X_n$ et $X_0$ on même espace topologique 
sous-jacent, l'application canonique 
\[
  \operatorname{Idem}\Gamma(X_n,\cO_{X_n}) \to \operatorname{Idem} \Gamma(X_0,\cO_{X_0})
\]
est bijective pour tout $n$, ce qui achève la démonstration. 

Puisque $\h^1(X,\dZ/n)$ est en bijection avec l'ensemble des classes 
d'isomorphisme de revêtements étales galoisiens de $X$ de groupe $\dZ/n$, 
le théorème  pour $q=1$ et $\cF=\dZ/n$ résulte de la proposition suivante. 





\begin{proposition}\label{4-2-2}
Soient $A$ un anneau local hensélien noethérien et $S=\spec(A)$. Soient 
$f:X\to S$ un morphisme propre et $X_0$ la fibre fermée de $f$. Alors le 
foncteur de restriction 
\[
  \operatorname{Rev.et}(X) \to \operatorname{Rev.et}(X_0)
\]
est une équivalence de catégories. 
\end{proposition}

(Si $X_0$ est connexe et si l'on a choisi un point géométrique de $X_0$ 
comme point base, cela revient à dire que l'application canonique 
$\pi_1(X_0)\to \pi_1(X)$ sur les groupes fondamentaux (profinis) est 
bijective). 

La proposition (\ref{4-2-1}) montre que ce foncteur est pleinement fidèle. En 
effet, si $X'$ et $X''$ sont deux revêtements étales de $X$, un 
$X$-morphisme de $X'$ dans $X''$ est déterminé par son graphe qui est une 
partie ouverte et fermée de $X'\times_X X''$. 

Il s'agit donc de montrer que tout revêtement étale $X_0'$ de $X_0$ s'étend 
en un revêtement étale de $X$. On sait que les revêtements étales ne 
dépendent pas des éléments nilpotents \cite[chap. 1]{7}, par conséquent 
$X_0'$ se relève de manière unique en un revêtement étale $X_n'$ de 
$X_n$ pour tout $n\geqslant 0$, autrement dit en un revêtement étale 
$\fX'$ du schéma formel $\fX$ complété de $X$ de long de $X_0$. D'après 
le théorème d'algébrisation des faisceaux cohérents formels de 
Grothendieck (théorème d'existence, \cite[III.5]{8}), $\fX'$ est le 
complété formel d'un revêtement étale $\bar X'$, de 
$\bar X=X\otimes_A \hat A$. 

Par passage à la limite, il suffit de démontrer la proposition dans le cas 
où $A$ est l'hensélisé d'une $\dZ$-algèbre de type fini. On peut alors 
appliquer le théorème d'approximation d'Artin au foncteur 
$\cF:(\textnormal{$A$-algèbres}) \to (\textnormal{ensembles})$ qui, à une 
$A$-algèbre $B$, fait correspondre l'ensemble des classes d'isomorphisme de 
revêtements étales de $X\otimes_A B$. En effet ce foncteur est localement de 
présentation finie: si $B_i$ est une système inductif filtrant de 
$A$-algèbres et si $B=\varinjlim B_i$, on a $\cF(B) = \varinjlim \cF(B_i)$. 
D'après le théorème d'Artin, étant donné un élément 
$\xi\in \cF(\hat A)$, en l'occurrence la classe d'isomorphisme de $\bar X'$, il 
existe $\xi\in \cF(A)$ ayant même image que $\bar\xi$ dans $\cF(A/\fm)$. 
Autrement dit il existe un revêtement étale $X'$ de $X$ dont la restriction 
à $X_0$ est isomorphe à $X_0'$. 










\subsection{Faisceaux constructibles}\label{4-3}

Dans ce paragraphe, on considère un schéma \emph{noethérien} $X$ et on 
appelle faisceau sur $X$ un faisceau \emph{abélien} sur $\et X$. 





\begin{definition}\label{4-3-1}
On dit qu'un faisceau $\cF$ sur $X$ est \emph{localement constant constructible} 
(en abrégé l.c.c.) s'il est représenté par un revêtement étale de 
$X$. 
\end{definition}





\begin{definition}\label{4-3-2}
On dit qu'un faisceau $\cF$ sur $X$ est \emph{constructible} s'il vérifie les 
conditions équivalentes suivantes:
\begin{enumerate}[\indent (i)]
  \item Il existe une famille finie surjective de sous-schémas $X_i$ de $X$ 
     tels que la restriction de $\cF$ à $X_i$ soit l.c.c..
  \item Il existe une famille finie de morphismes finis $p_i:X_i'\to X$, pour 
    chaque $i$ un faisceau constant constructible ( = défini par un groupe 
    abélien fini) $C_i$ sur $X_i'$, et un monomorphisme 
    $\cF\to \prod p_{i*} C_i$. 
\end{enumerate}
\end{definition}

On vérifie facilement que la catégorie des faisceaux constructibles sur $X$ 
est une catégorie abélienne. De plus, si $u:\cF\to \cG$ est un homomorphisme de 
faisceaux et si $\cF$ est constructible, le faisceau $\im(u)$ est constructible.





\begin{lemma}\label{4-3-3}
Tout faisceau de torsion $\cF$ est limite inductive filtrante de faisceaux 
constructibles.
\end{lemma}

En effet, si $j:U\to X$ est un schéma étale de type fini sur $X$, un 
élément $\xi\in \cF(U)$ tel que $n\xi=0$ définit un homomorphisme de 
faisceaux $j_! \const{\dZ/n}\to \cF$ dont l'image (le lus petit sous-faisceau de 
$\cF$ dont $\xi$ soit section locale) est un sous-faisceau constructible de $\cF$. 
Il est clair que $\cF$ est limite inductive de tels sous-faisceaux.





\begin{definition}\label{4-3-4}
Soient $\sC$ une catégorie abélienne et $T$ un foncteur défini sur $\sC$ 
à valeurs dans la catégorie des groupes abéliens. On dira que $T$ est 
\emph{effaçable} dans $\sC$ si, pour tout objet $A$ de $\sC$ et tout 
$\alpha\in T(A)$, il existe un monomorphisme $u:A\to M$ dans $\sC$ tel que 
$T(u)\alpha = 0$. 
\end{definition}





\begin{lemma}\label{4-3-5}
Les foncteurs $\h^q(X,-)$ pour $q>0$ sont effaçables dans la catégorie des 
faisceaux constructibles sur $X$.
\end{lemma}

Il suffit de remarquer que, si $\cF$ est un faisceau constructible, il existe 
nécessairement un entier $n>0$ tel que $\cF$ soit un faisceau de 
$\dZ/n$-modules. Alors il existe un monomorphisme $\cF\hookrightarrow \cG$, où 
$\cG$ est un faisceau de $\dZ/n$-modules et $\h^q(X,\cG) = 0$ pour tout $q>0$.  
On peut par exemple prendre pour $\cG$ la résolution de Godement 
$\prod_{x\in X} i_{x*} \cF_{\bar x}$, où $x$ parcourt les points de $X$ et 
$i_x:\bar x\to X$ est un point géométrique centré en $X$. D'après 
(\ref{4-3-3}) $\cG$ est limite inductive de faisceaux constructibles, d'o`u le 
lemme, car les foncteurs $\h^q(X,-)$ commutent aux limites inductives. 





\begin{lemma}\label{4-3-6}
Soit $\varphi^\bullet:T^\bullet\to {T'}^\bullet$ un morphisme de foncteurs 
cohomologiques définis sur une catégorie abélienne $\sC$ et à valeurs 
dans la catégorie des groupes abéliens. Supposons que $T^q$ est effaçable 
pour $q>0$ et soit $\sE$ un sous-ensemble d'objets de $\sC$ tel que tout objet 
de $\sC$ soit contenu dans un objet appartenant à $\sE$. Alors les 
conditions suivantes sont équivalentes:
\begin{enumerate}[\indent (i)]
  \item $\varphi^q(A)$ est bijectif pour tout $q\geqslant 0$ et tout 
    $A\in\ob \sC$.
  \item $\varphi^0(M)$ est bijectif et $\varphi^q(M)$ surjectif pour tout 
    $q>0$ et tout $M\in\sE$. 
  \item $\varphi^0(A)$ est bijectif pour tout $A\in\ob \sC$ et ${T'}^q$ est 
    effaçable pour tout $q>0$. 
\end{enumerate}
\end{lemma}

La démonstration se fait par récurrence sur $q$ et ne présente pas de 
difficultés. 





\begin{proposition}\label{4-3-7}
Soit $X_0$ un sous-schéma de $X$. Supposons que, pour tout $n\geqslant 0$ et 
pour tout schéma $X'$ fini sur $X$, l'application canonique 
\[
  \h^q(X',\dZ/n) \to \h^q(X_0',\dZ/n)\text{,}
\]
où $X_0' = X'\times_X X_0$, est bijective pour $q=0$ et surjective pour 
$q>0$. Alors, pour tout faisceau de torsion $\cF$ sur $X$ et pour tout 
$q\geqslant 0$, l'application canonique 
\[
  \h^q(X,\cF) \to \h^q(X_0,\cF)
\]
est bijective.
\end{proposition}

Par passage à la limite, il suffit de démontrer l'assertion pour $\cF$ 
constructible. On applique le lemme (\ref{4-3-6}) en prenant pour $\sC$ la 
catégorie des faisceaux constructibles sur $X$, $T^q=\h^q(X,-)$, 
${T'}^q=\h^q(X_0,-)$, et $\sE$ l'ensemble des faisceaux constructibles de la 
forme $\prod p_{i*} C_i$, où $p_i:X_i'\to X$ est un morphisme fini et $C_i$ 
un faisceau constant fini sur $X_i'$. 










\subsection{Fin de la démonstration}\label{4-4}

Par la méthode de fibration par des courbes (\ref{3-4-3}), on se ramène 
à démontrer le théorème en dimension relative $\leqslant 1$. D'après 
le paragraphe précédent, il suffira de montrer que, si $S$ est le spectre 
d'un anneau local noethérien strictement hensélien, $f:X\to S$ un 
morphisme propre dont la fibre fermée $X_0$ est de dimension $\leqslant 1$ 
et $n$ un entier $\geqslant 0$, l'homomorphisme canonique 
\[
  \h^q(X,\dZ/n) \to \h^q(X_0,\dZ/n)
\]
est bijectif pour $q=0$ et surjectif pour $q>0$. 

Les cas $q=0$ et $1$ ont été vus plus haut et on a $\h^q(X_0,\dZ/n) = 0$ 
pour $q\geqslant 3$; il suffit donc de traiter le cas $q=2$. On peut 
évidemment supposer que $n$ est une puissance d'un nombre premier. Si 
$n=p^r$, où $p$ est la caractéristique du corps résiduel de $S$, la 
théorie d'Artin-Schreier montre qu'un a $\h^2(X_0,\dZ/p^r) = 0$. Si 
$n=\ell^r$, $\ell\ne p$, on déduit de la théorie de Kummer un diagramme 
commutatif 
\[\xymatrix{
  \pic(X) \ar[d] \ar[r]^-\alpha 
    & \h^2(X,\dZ/\ell^r) \ar[d] \\
  \pic(X_0) \ar[r]^-\beta 
    & \h^2(X_0,\dZ/\ell^r)
}\]
où l'application $\beta$ est surjective (On l'a vu au chapitre \ref{3} pour 
une courbe lisse sur un corps algébriquement clos, mais des arguments 
similaires s'appliquent à n'importe quelle courbe sur un corps 
séparablement clos). 

Pour conclure, il suffira donc de montrer:





\begin{proposition}\label{4-4-1}
Soient $S$ le spectre d'un anneau local noethérien hensélien et $f:X\to S$ 
un morphisme propre dont la fibre fermée $X_0$ est de dimension 
$\leqslant 1$. Alors l'application canonique de restriction 
\[
  \pic(X)\to \pic(X_0)
\]
est surjective (Il suffit d'ailleurs que le morphisme $f$ soit 
\emph{séparé} de type fini). 
\end{proposition}

Pour simplifier la démonstration, nous supposerons que $X$ est intègre, 
bien que cela ne soit pas nécessaire. Tout faisceau inversible sur $X_0$ est 
associé à un diviseur de Cartier (car $X_0$ est une courbe, donc 
quasi-projectif), il suffit donc de montrer que l'application canonique 
$\Div(X)\to\Div(X_0)$ est surjective. 

Tout diviseur sur $X_0$ est combinaison linéaire de diviseurs dont le 
support est concentré en un seul point fermé non isolé de $X_0$. Soient 
$x$ un tel point, $t_0\in\cO_{X_0,x}$ un élément régulier non inversible 
de $\cO_{X_0,x}$ et $D_0$ le diviseur concentré en $x$ d'équation locale 
$t_0$. Soit $U$ un voisinage ouvert de $x$ dans $X$ tel qu'il existe un section 
$t\in \Gamma(U,\cO_U)$ relevant $t_0$. Soit $Y$ le fermé de $U$ d'équation 
$t=0$; quitte à prendre $U$ assez petit, on peut supposer que $x$ est le 
seul point de $Y\cap X_0$. Alors $Y$ est quasi-fini au-dessus de $S$ en $x$; 
puisque $S$ est le spectre d'un anneau local hensélien, on en déduit que 
$Y=Y_1\amalg Y_2$, où $Y_1$ est fini sur $S$ et où $Y_2$ ne rencontre pas 
$X_0$. De plus, comme $X$ est séparé sur $S$, $Y_1$ est fermé dans $X$. 

Quitte à remplacer $U$ par un voisinage ouvert plus petit de $x$, on peut 
supposer que $Y=Y_1$, autrement dit que $Y$ est fermé dans $X$. On définit 
alors un diviseur $D$ sur $X$ relevant $D_0$ en posant $D|X\setminus Y=0$ et 
$D|U=\dv(t)$ ce qui a un sens car $t$ est inversible sur $U\setminus Y$. 





\subsubsection{Remarque}\label{4-4-2}

Dans le cas où $f$ est propre, on pourrait aussi faire une démonstration 
du même style que celle de la proposition (\ref{4-2-2}). En effet, comme 
$X_0$ est une courbe, il n'y a pas d'obstruction à relever un faisceau 
inversible sur $X_0$ aux voisinages infinitésimaux $X_n$ de $X_0$, donc au 
complété formel $\fX$ de $X$ le long de $X_0$. On conclut alors en 
appliquant successivement le théorème d'existence de Grothendieck et le 
théorème d;approximation d'Artin. 










\subsection{Cohomologie à support propre}\label{4-5}





\begin{definition}\label{4-5-1}
Soit $X$ un schéma séparé de type fini sur un corps $k$. D'après un 
théorème de Nagata, il existe un schéma $\bar X$ propre sur $k$ et une 
immersion ouverte $j:X\to\bar X$. Pour tout faisceau de torsion $\cF$ sur $X$ on 
note $j_! \cF$ le prolongement par $0$ de $\cF$ à $\bar X$ et on définit les 
groupes de \emph{cohomologie à support propre} 
\[
  \h_c^q(X,\cF)=\h^q(\bar X, j_! \cF) \text{.}
\]
\end{definition}

Montrons que cette définition est indépendante de la compactification 
$j:X\to\bar X$ choisie. Soient $j_1:X\to \bar X_1$ et $j_2:X\to \bar X_2$ deux 
compactifications. Alors $X$ s'envoie dans $\bar X_1\times \bar X_2$ par 
$x\mapsto (j_1(x),j_2(x))$ et l'image fermée $\bar X_3$ de $X$ par cette 
application est une compactification de $X$. On a ainsi un diagramme commutatif 
\[\xymatrix{
  & \bar X_1 \\
  X \ar[ur]^-{j_1} \ar[rr]^-{(j_1,j_2)} \ar[dr]_-{j_2} 
    & & \bar X_3 \ar[ul]_-{p_1} \ar[dl]^-{p_2} \\
  & \bar X_2
}\]
où $p_1$ et $p_2$, les restrictions des projections naturelles à 
$\bar X_3$, sont des morphismes propres. Il suffit donc de traiter le cas où 
on a un diagramme commutatif 
\[\xymatrix{
  X \ar[r]^-{j_2} \ar[dr]_-{j_1} 
    & \bar X_2 \ar[d]^-p \\
  & \bar X_1
}\]
avec $p$ un morphisme propre. 





\begin{lemma}\label{4-5-2}
On a $p_*(j_{2!} \cF) = j_{1!} \cF$ et $\R^q p_*(j_{2!} \cF) = 0$, pour $q>0$. 
\end{lemma}

Notons tout de suite que lemme suffit pour conclure. En utilisant la suite 
spectrale de Leray du morphisme $p$, on en déduit qu'on a, pour tout 
$q\geqslant 0$, 
\[
  \h^q(\bar X_2, j_{2!} \cF) = \h^q(\bar X_1, j_{1!} \cF) \text{.}
\]

Pour démontrer le lemme, on raisonne fibre par fibre en utilisant le théorème 
de changement de base (\ref{4-1-1}) pour $p$. Le résultat est immédiat, car, 
au-dessus d'un point de $X$, $p$ est un isomorphe et, au-dessus d'un point de 
$\bar X_1\setminus X$, $j_{2!} \cF$ est nul sur la fibre de $p$. 





\subsubsection{}\label{4-5-3}

De même, si $f:X\to S$ est un morphisme séparé de type fini de schémas 
noethériens, il existe un morphisme propre $\bar f:\bar X\to S$ et un 
immersion ouverte $j:X\to \bar X$. On définit alors les images directes s
supérieures à support propre $\R^q f_!$ en posant pour tout faisceau de 
torsion $\cF$ sur $X$
\[
  \R^q f_! \cF = \R^q f_*(j_! \cF) \text{.}
\]
On vérifie comme précédemment que cette définition est indépendante de 
la compactification choisie. 





\begin{theorem}\label{4-5-4}
Soient $f:X\to S$ un morphisme séparé de type fini de schémas noethériens 
et $\cF$ un faisceau de torsion sur $X$. Alors la fibre de $\R^q f_! \cF$ en un 
point géométrique $s$ de $S$ est isomorphe à la cohomologie à support 
propre $\h_c^q(X_s,\cF)$ de la fibre $X_s$ de $f$ en $s$. 
\end{theorem}

C'est une simple variante du théorème de changement de base pour un morphisme 
propre (\ref{4-1-1}). Plus généralement, si 
\[\xymatrix{
  X \ar[d]^-f
    & X' \ar[l]_-{g'} \ar[d]^-{f'} \\
  S 
    & S' \ar[l]^-g
}\]
est un diagramme cartésien, on a un isomorphe canonique 
\begin{equation}\label{eq:3}
  g^* (\R^q f_! \cF) \simeq \R^q f_!'({g'}^* \cF) \text{.}
\end{equation}










\subsection{Applications}\label{4-6}





\begin{theorem}[d'annulation]\label{4-6-1}
Soient $f:X\to S$ un morphisme séparé de type fini dont les fibres sont de 
dimension $\leqslant n$ et $\cF$ un faisceau de torsion sur $X$. Alors on a 
$\R^q f_! \cF = 0$ pour $q>2 n$. 
\end{theorem}

D'après le théorème de changement de base, on peut supposer que $S$ est 
le spectre d'un corps séparablement clos. Si $\dim X = n$, il existe un 
ouvert affine $U$ de $X$ tel que $\dim(X\setminus U) < n$; on a alors un suite 
exacte $0\to \cF_U \to \cF \to \cF_{X\setminus U} \to 0$ et, par récurrence sur $n$, 
il suffit de démontrer le théorème pour $X=U$ affine. Puis la méthode de 
fibration par des courbes (\ref{3-4-1}) et le théorème de changement de 
base permettent de se ramener à une courbe sur un corps séparablement clos 
pour laquelle on déduit le résultat voulu du théorème de Tsen 
(\ref{3-3-6}). 





\begin{theorem}[de finitude]\label{4-6-2}
Soient $f:X\to S$ un morphisme séparé de type fini et $\cF$ un faisceau 
constructible sur $X$. Alors les faisceaux $\R^q f_! \cF$ sont constructibles. 
\end{theorem}

Nous ne considérons que le cas où $\cF$ est annulé par un entier inversible 
sur $X$. 

Par démontrer le théorème on se ramène au cas o`u $\cF$ est un faisceau 
constant $\const{\dZ/n}$ et où $f:X\to S$ est un morphisme propre et lisse 
dont les fibres sont des courbes géométriquement connexes de genre $g$. 
Pour $n$ inversible sur $X$, les faisceaux $\R^q f_* \cF$ sont alors localement 
libres de rang fini, nuls pour $q>2$ (\ref{4-6-1}). Remplaçant $\dZ/n$ par le 
faisceau localement isomorphe (sur $S$) $\dmu_n$, on a canoniquement 
\begin{equation}\label{eq:4}
\begin{aligned}
  \R^0 f_* \dmu_n &= \dmu_n \\
  \R^1 f_* \dmu_n &= \const\pic(X/S)_n \\
  \R^2 f_* \dmu_n &= \dZ/n \text{.}
\end{aligned}
\end{equation}





\begin{theorem}[comparaison avec la cohomologie classique]\label{4-6-3}
Soient $f:X\to S$ un morphisme séparé de schémas de type fini sur $\dC$, 
et $\cF$ un faisceau de torsion sur $X$. Notons par un exposant $\an{(-)}$ le 
foncteur de passage aux espaces topologiques usuels, et par $\R^q \an f_!$ les 
foncteurs dérivés du foncteur image directe à support propre par $\an f$. On a 
\[
  \an{(\R^q f_! \cF)} \simeq \R^q \an f_! \an \cF \text{.}
\]
En particulier, pour $S=$ un point et $\cF$ le faisceau constant $\dZ/n$, 
\[
  \h_c^q(X,\dZ/n) \simeq \h_c^q(\an X,\dZ/n)\text{.}
\]
\end{theorem}

Des dévissages utilisant le théorème de changement de bas nous ramènent au 
cas où $X$ est une courbe propre et lisse, où $S=$ un point, et où 
$\cF=\dZ/n$. Les groupes de cohomologie considérés sont alors nuls pour 
$q\ne 0,1,2$, et on invoque \cite{12}: en effet, si $X$ est propre sur $\dC$, 
on a $\pi_0(X) = \pi_0(\an X)$ et $\pi_1(X) = $ complété profini de 
$\pi_1(\an X)$, d'où l'assertion pour $q=0,1$. Pour $q=2$, on utilise la 
suite exacte de Kummer et le fait que, par \cite{12} encore, 
$\pic(X)=\pic(\an X)$. 





\begin{theorem}[dimension cohomologique des schémas affines]\label{4-6-4}
Soient $X$ un schéma affine de type fini sur un corps séparablement clos et 
$\cF$ un faisceau de torsion sur $X$. Alors on a $\h^1(X,\cF) = 0$ pour 
$q>\dim(X)$. 
\end{theorem}

Pour la très jolie démonstration nous revoyons à \cite{4}, XIV \S 2 et 3. 





\subsubsection{Remarque}\label{4-6-5}

Ce théorème est en quelque sorte un substitut pour la théorie de Morse. 
Considérons en effet le cas classique où $X$ est lisse et affine sur $\dC$ 
plongé dans un espace affine type $\dC^N$. Alors, pour presque tout point 
$p\in\dC^N$, la fonction ``distance à $p$'' sur $X$ est une fonction de Morse 
et les indices de ses points critiques sont plus petits que $\dim(X)$. Ainsi 
$X$ est obtenu par recollement d'anses d'indice plus petit que $\dim(X)$, d'où 
l'analogue classique de (\ref{4-6-4}). 




















\section{Acyclicité locale des morphismes lisses}\label{5}

Soient $X$ une variété analytique complexe et $f:X\to D$ un morphisme de 
$X$ dans le disque. On note $[0,t]$ le segment de droit fermé d'extrémités 
$0$ et $t$ dans $D$ et $(0,t]$ le segment semi-ouvert. Si $f$ est \emph{lisse}, 
l'inclusion 
\[
  j : f^{-1}\left((0,t]\right) \hookrightarrow f^{-1}\left([0,t]\right)
\]
est une équivalence d'homotopie; en peut pousser la fibre spécial 
$X_0=f^{-1}(0)$ dans $f^{-1}([0,t])$. 

En pratique, pour $t$ assez petit, $f^{-1}((0,t])$ sera un fibré sur 
$(0,t]$ de sorte que l'inclusion 
\[
  X_t = f^{-1}(t) \hookrightarrow f^{-1}\left((0,t]\right)
\]
sera également une équivalence d'homotopie. On appelle alors \emph{morphisme 
de cospécialisation} la classe d'homotopie d'applications: 
\[
  \cosp : X_0 \hookrightarrow f^{-1}\left([0,t]\right) \xleftarrow\sim f^{-1}\left((0,t]\right) \xleftarrow\sim X_t \text{.}
\]
On peut exprimer cette construction en termes imagés en disant que, pour un 
morphisme lisse, la fibre générale avale la fibre spéciale. 

Ne supposons plus $f$ nécessairement lisse (mais supposons que $f^{-1}((0,t])$ 
soit un fibré sur $(0,t]$). On peut encore définir un morphisme 
$\cosp^\bullet$ en cohomologie dès que $j_* \dZ=\dZ$ et 
$\R^q j_* \dZ = 0$ pour $q>0$. Sous ces hypothèses, la suite spectrale de 
Leray pour $j$ montre qu'on a 
\[
  \h^\bullet\left(f^{-1}\left([0,t]\right),\dZ\right) \iso \h^\bullet\left(f^{-1}\left((0,t]\right),\dZ\right)
\]
et $\cosp^\bullet$ est le morphisme composé: 
\[
  \cosp^\bullet : \h^\bullet(X_t,\dZ)
    \xleftarrow\sim \h^\bullet\left(f^{-1}\left((0,t]\right),\dZ\right) 
    \xleftarrow\sim \h^\bullet\left(f^{-1}\left([0,t]\right),\dZ\right) 
    \to \h^\bullet(X_0,\dZ)
\]
La fibre de $\R^q j_* \dZ$ en un point $x\in X_0$ se calcule comme suit. On 
prend dans un espace ambiant une boule $B_\varepsilon$de centre $x$ et de rayon 
$\varepsilon$ assez petit, et pour $\eta$ assez petit, on pose 
$E=X\cap B_\varepsilon \cap f^{-1}(\eta t)$; c'est la \emph{variété des 
cycles évanescents} en $x$. On a 
\[
  (\R^q j_* \dZ)_x \xleftarrow\sim \h^q\left(X\cap B_\varepsilon\cap f^{-1}\left((0,\eta t]\right),\dZ\right) \xrightarrow\sim \h^q(E,\dZ)
\]
et le morphisme de cospécialisation est défini en cohomologie dés que les 
variétés de cycles évanescents sont acycliques ($\h^0(E,\dZ)=\dZ$ et 
$\h^q(E,\dZ) = 0$ pour $q>0$), ce qui s'exprime en disant que $f$ est 
\emph{localement acyclique}. 

Ce chapitre est consacré à l'analogue de cette situation pour un morphisme 
lisse de schémas pour la cohomologie étale. Cependant il est indispensable 
dans ce cadre de se limiter aux coefficients de torsion et d'ordre \emph{premier aux caractéristiques résiduelles}. La paragraphe \ref{5-1} est consacré 
à des généralités sur les morphismes localement acycliques et les 
flèches de cospécialisation. Dans le paragraphe \ref{5-2}, on démontre 
qu'un morphisme lisse est localement acyclique. Dans le paragraphe \ref{5-3}, on 
joint ce résultat à ceux du chapitre précédent pour en déduire deux 
applications: un théorème de spécialisation des groupes de cohomologie (la 
cohomologie des fibres géométriques d'un morphisme propre et lisse est 
localement constant) et un théorème de changement de base par un morphisme 
lisse. 

Dans tout ce qui suit, on fixe un entier $n$ et ``schéma'' signifie ``schéma 
sur lequel $n$ est inversible.'' ``Point géométrique'' signifiera toujours 
``point géométrique algébrique'' (\ref{2-3-1}) $x:\spec(k)\to X$, avec $k$ 
algébriquement clos. 










\subsection{Morphismes localement acycliques}\label{5-1}





\subsubsection{Notation}\label{5-1-1}

Étant donnés un schéma $S$ et un point géométrique $s$ de $S$, on 
notera $\widetilde S^s$ le spectre du localisé stricte de $S$ en $s$. 





\begin{definition}\label{5-1-2}
On dit qu'un point géométrique $t$ de $S$ est une \emph{générisation} de 
$s$ s'il est défini par une clôture algébrique du corps résiduel d'un 
point de $\widetilde S^s$. On dit aussi que $s$ est une \emph{spécialisation} 
de $t$ et on appelle \emph{flèche de spécialisation} le $S$-morphisme 
$t\to \widetilde S^s$. 
\end{definition}





\begin{definition}\label{5-1-3}
Soit $f:X\to S$ un morphisme de schémas. Soient $s$ un point géométrique de 
$S$, $t$ une générisation de $s$, $x$ un point géométrique de $X$ 
su-dessus de $s$ et 
$\widetilde X_t^x = \widetilde X^x\times_{\widetilde S^s} t$. Alors on dit que 
$\widetilde X_t^x$ est une \emph{variété de cycles évanescents} de $f$ au 
point $x$. 

On dit que $f$ est \emph{localement acyclique} si, la cohomologie réduite de 
toute variété de cycles évanescents $\widetilde X_t^x$ est nulle:
\begin{equation}\label{eq:5}
  \hat\h^\bullet(\widetilde X_t^x,\dZ/n) = 0\text{,}
\end{equation}
i.e. $\h^0\left(X_t^x,\dZ/n\right) = \dZ/n$ et 
$\h^q(\widetilde X_t^x,\dZ/n) = 0$ pour $q>0$. 
\end{definition}





\begin{lemma}\label{5-1-4}
Soient $f:X\to S$ un morphisme localement acyclique et $g:S'\to S$ un morphisme 
quasi-fini (ou limite projective de morphismes quasi-finis). Alors le 
morphisme $f':X'\to S'$ déduit de $f$ par changement de base est localement 
acyclique.
\end{lemma}

On vérifie en effet que tout variété de cycles évanescents de $f'$ est 
une variété de cycles évanescents de $f$.  	





\begin{lemma}\label{5-1-5}
Soit $f:X\to S$ un morphisme localement acyclique. Pour tout point géométrique 
$t$ de $S$, donnant lieu à un diagramme cartésien 
\[\xymatrix{
  X_t \ar[r]^-{\varepsilon'} \ar[d] 
    & X \ar[d]^-f \\
  t \ar[r]^-\varepsilon 
    & S
}\]
On a $\varepsilon_*' \dZ/n = f^*\varepsilon_* \dZ/n$ et $\R^q \varepsilon_*'\dZ/n=0$ 
pour $q>0$. 
\end{lemma}

Soient $\bar S$ l'adhérence de $\varepsilon(t)$, $S'$ le normalisé de $\bar S$ 
dans $k(t)$, et le diagramme cartésien 
\[\xymatrix{
  X_t \ar[r]^-{f'} \ar[d] 
    & X' \ar[r]^-{\alpha'} \ar[d]^-{f'} 
    & X \ar[d]^-f \\
  t \ar[r]^-i 
    & S' \ar[r]^-\alpha 
    & S
}\]
Les anneaux locaux de $S'$ sont normaux à corps de fractions séparablement 
clos. Ils sont donc strictement henséliens, et l'acyclicité locale de $f'$ 
(\ref{5-1-4}) fournit $f_*'\dZ/n = \dZ/n$, $\R^qi_*'\dZ/n = 0$ pour $q>0$. 
Puisque $\alpha$ est entier, on a alors 
\[
  \R^q\varepsilon_*'\dZ/n 
    = \alpha_*' \R^q i_*' \dZ/n 
    = \alpha_*' {f'}^* \R^q i_* \dZ/n 
    = f^* \alpha_* \R^q i_* \dZ/n 
    = f^* \R^q \varepsilon_* \dZ/n \text{,}
\]
et le lemme. 





\subsubsection{}\label{5-1-6}

Etant donnés un morphisme localement acyclique $f:X\to S$ et une flèche de 
spécialisation $t\to \widetilde S^s$, nous allons définir des 
homomorphismes canoniques, dits flèches de cospécialisation 
\[
  \cosp^\bullet : \h^\bullet(X_t,\dZ/n) \to \h^\bullet(X_s,\dZ/n)\text{,}
\]
reliant la cohomologie de la fibre générale $X_t=X\times_S t$ et celle de la 
fibre spéciale $X_s=X\times_X s$. 

Considérons le diagramme cartésien 
\[\xymatrix{
  X_t \ar[d] \ar[r]^-{\varepsilon'} 
    & \widetilde X \ar[d]^-{f'} 
    & \ar[l] X_s \ar[d] \\
  t \ar[r]^-\varepsilon 
    & \widetilde S^s 
    & \ar[l] s
}\]
déduit de $f$ par changement de base. D'après (\ref{5-1-4}), $f'$ est encore 
localement acyclique. De la définition de l'acyclicité locale, on tire 
aussitôt que la restriction à $X_s$ des faisceaux $\R^q \varepsilon_*' \dZ/n$ 
est $\dZ/n$ pour $q=0$, et $0$ pour $q>0$. Par (\ref{5-1-5}) on sait même que 
$\R^q \varepsilon_*'\dZ/n = 0$ pour $q>0$. On définit $\cosp^\bullet$ comme la 
flèche composée 
\begin{equation}\label{eq:6}
  \h^\bullet(X_t,\dZ/n) \simeq \h^\bullet(X,\varepsilon_*' \dZ/n) \to \h^\bullet(X_s,\dZ/n) \text{.}
\end{equation}

\paragraph{Variante}
Soient $\bar S$ l'adhérence de $\varepsilon(t)$ dans $\widetilde S^s$, $S'$ le 
normalisé de $\bar S$ dans $k(t)$ et $X'/S'$ déduit de $X/S$ par changement 
de base. Le diagramme \eqref{eq:6} peut encore s'écrire 
\[
  \h^\bullet(X_t,\dZ/n) \simeq \h^\bullet(X',\dZ/n) \to \h^\bullet(X_s,\dZ/n) \text{.}
\]





\begin{theorem}\label{5-1-7}
Soient $S$ un schéma localement noethérien, $s$ un point géométrique de 
$S$ et $f:X\to S$ un morphisme. On suppose 
\begin{enumerate}[\indent a)]
  \item le morphisme $f$ est localement acyclique, 
  \item pour tout morphisme de spécialisation $t\to \widetilde S^s$ et pour 
    tout $q\geqslant 0$, les flèches de cospécialisation 
    $\h^q(X_t,\dZ/n) \to \h^q(X_s,\dZ/n)$ sont bijectives.
\end{enumerate}

Alors l'homomorphisme canonique $(\R^q f_*\dZ/n)_s\to \h^q(X_s,\dZ/n)$ est 
bijectif pour tout $q\geqslant 0$. 
\end{theorem}

Pour démontrer le théorème, il est clair qu'on peut supposer 
$S=\widetilde S^s$. On va et fait montrer que, pour tout faisceau de 
$\dZ/n\dZ$-modules $\cF$ sur $S$, l'homomorphisme canonique 
$\varphi^q(\cF):(\R^q f_* f^* \cF)_s \to \h^q(X_s,f^*\cF)$ est bijectif. 

Tout faisceau de $\dZ/n\dZ$ est limite inductive filtrante de faisceaux 
constructibles de $\dZ/n\dZ$-modules (\ref{4-3-3}). De plus tout faisceau 
constructible de $\dZ/n\dZ$-modules se plonge dans un faisceau de la forme 
$\prod i_{\lambda *} C_\lambda$, où $i_\lambda:t_\lambda\to S$ est une 
famille finie de générisations de $s$ et $C_\lambda$ un $\dZ/n\dZ$-module 
libre de rang fini sur $t_\lambda$. D''après la définition des flèches de 
cospécialisation, la condition b) signifie que les homomorphismes 
$\varphi^q(\cF)$ sont bijectifs si $\cF$ est de cette forme. 

On conclut à l'aide d'une variant du lemme (\ref{4-3-6}):





\begin{lemma}\label{5-1-8}
Soit $\sC$ un catégorie abélienne dans laquelle les limites inductives 
filtrantes existent. Soit $\varphi^\bullet:T^\bullet\to{T'}^\bullet$ un 
morphisme de foncteurs cohomologiques commutant aux limites inductives 
filtrantes, définis sur $\sC$ et à valeurs dans la catégorie des groupes 
abéliens. Supposons qu'il existe deux sous-ensembles $\sD$ et $\sE$ d'objets 
de $\sC$ tels que:
\begin{enumerate}[\indent a)]
  \item tout objet de $\sC$ est limite inductive filtrante d'objets appartenant 
    à $\sD$, 
  \item tout objet appartenant à $\sD$ est contenu dans un objet appartenant 
    à $\sE$.
\end{enumerate}

Alors les conditions suivantes sont équivalentes: 
\begin{enumerate}[\indent (i)]
  \item $\varphi^q(A)$ est bijectif pour tout $q\geqslant 0$ et tout 
    $A\in\ob\sC$.
  \item $\varphi^q(M)$ est bijectif pour tout $q\geqslant 0$ et tout $M\in\sE$. 
\end{enumerate}
\end{lemma}

La démonstration du lemme se fait par passage à la limite inductive, 
récurrence sur $q$ et application répétée du lemme des cinq au diagramme 
de suites exactes de cohomologie déduit d'une suite exacte 
$0\to A\to M\to A'\to 0$, avec $A\in\sD$, $M\in\sE$, $A'\in \ob\sC$. 





\begin{corollary}\label{5-1-9}
Soient $S$ le spectre d'un anneau local noethérien strictement hensélien et 
$f:X\to S$ un morphisme localement acyclique. Supposons que, pour tout point 
géométrique $t$ de $S$ on dit $\h^0(X_t,\dZ/n) = \dZ/n$ et 
$\h^q(X_t,\dZ/n) = 0$ pour $q>0$ (autrement dit les fibres géométriques de 
$f$ sont acycliques). Alors on a $f_* \dZ/n\dZ/n$ et $\R^qf_*\dZ/n = 0$ pour 
$q>0$. 
\end{corollary}





\begin{corollary}\label{5-1-10}
Soient $f:X\to Y$ et $g:Y\to X$ des morphismes de schémas localement 
noethériens. Alors, si $f$ et $g$ sont localement acycliques, il en est de 
même de $g\circ f$. 
\end{corollary}

On peut supposer que $X$, $Y$ et $Z$ sont strictement locaux et que $f$ et $g$ 
sont des morphismes locaux. Il s'agit alors de montrer que, si $z$ est un point 
géométrique algébrique de $Z$, on a $\h^0(X_z,\dZ/n) = \dZ/n$ et 
$\h^q(X_z,\dZ/n) = 0$ pour $q>0$. 

Puisque $g$ est localement acyclique, on a $\h^0(Y_z,\dZ/n) = \dZ/n$, et 
$\h^q(Y_z,\dZ/n) = 0$ pour $q>0$. Par ailleurs le morphisme $f_z:X_z\to Y_z$ est 
localement acyclique (\ref{5-1-4}) et ses fibres géométriques sont 
acycliques, car ce sont des variétés de cycles évanescents de $f$. 
D'après (\ref{5-1-9}), on a donc $\R^q {f_z}_*\dZ/n = 0$ pour $q>0$. De plus 
${f_z}_*\dZ/n$ est constant de fibre $\dZ/n$ sur $Y_z$. On conclut à l'aide de 
la suite spectrale de Leray de $f_z$. 










\subsection{Acyclicité locale d'un morphisme lisse}\label{5-2}





\begin{theorem}\label{5-2-1}
Un morphisme lisse est localement acyclique.
\end{theorem}

Soit $f:X\to S$ un morphisme lisse. L'assertion est locale pour la topologie 
étale sur $X$ et $S$, on peut donc supposer que $X$ est l'espace affine type 
de dimension $d$ sur $S$. Par passage à la limite, on peut supposer que $S$ 
est noethérien et la transitivité de l'acyclicité locale (\ref{5-1-10}) montre 
qu'il suffit de traiter le cas $d=1$. 

Soient $s$ un point géométrique de $S$ et $x$ un point géométrique de 
$X$ centré en un point fermé de $X_s$. Il s'agit de montrer que les fibres 
géométriques du morphisme $\widetilde X^x\to \widetilde S^s$ sont 
acycliques. On posera désormais $S=\widetilde S^s=\spec(A)$ et 
$X=\widetilde X^x$. On a $X\simeq \spec A\{T\}$, où $A\{T\}$ est l'hensélise 
de $A[T]$ au point $T=0$ au-dessus de $s$. 

Si $t$ est un point géométrique de $S$, la fibre $X_t$ est limite 
projective de courbes affines et lisses sur $t$. On a donc $\h^q(X_t,\dZ/n) = 0$ 
pour $q\geqslant 0$ et il suffit de montrer que $\h^0(X_t,\dZ/n) = \dZ/n$ et 
$\h^1(X_t,\dZ/n) = 0$ pour $n$ premier à la caractéristique résiduelle de 
$S$. Cela résulte des deux propositions suivantes. 





\begin{proposition}\label{5-2-2}
Soient $A$ un anneau local strictement hensélien, $S=\spec(A)$ et 
$X=\spec A\{T\}$. Alors les fibres géométrique de $X\to S$ sont connexes. 
\end{proposition}

On peut se ramener par passage à limite au cas où $A$ est un hensélisé 
strict d'une $\dZ$-algèbre de type fini. 

Soient $\bar t$ un point géométrique de $S$, localisé en $t$, et $k'$ une 
extension finie séparable de $k(t)$ dans $k(\bar t)$. On pose $t'=\spec(k')$ 
et $X_{t'} = X\times_S\spec(k')$. Il nous faut vérifier que, quelques soient 
$\bar t$ et $t'$, $X_{t'}$ est connexe (par qui on entend connexe et non vide). 
Soit $A'$ le normalisé de $A$ dans $k'$, i.e. l'anneau des éléments de 
$k'$ entiers sur l'image de $A$ dans $k(t)$. On a 
$A\{t\}\otimes_A A'\iso A'\{T\}$: le membre de gauche est en effet hensélien 
local (car $A'$ est fini sur $A$, et local) et limite d'algèbres locales 
étales sur $A'\{T\}=A\{T\}\otimes_A A'$. Le schéma $X_{t'}$ est donc encore 
la fibre en $t'$ de $X=\spec\left(A'\{T\}\right)$ sur $S'=\spec(A')$. Le 
schéma local $X'$ est normal, donc intègre; son localisé $X_{t'}$ est 
encore intègre, a fortiori connexe. 





\begin{proposition}\label{5-2-3}
Soient $A$ un anneau local strictement hensélien, $S=\spec(A)$ et 
$X=\spec\left(A\{T\}\right)$. Soient $\bar t$ un point géométrique de $S$ 
et $X_{\bar t}$ la fibre géométrique correspondante. Alors tout revêtement 
étale galoisien de $X_{\bar t}$ d'ordre premier à la caractéristique du 
corps résiduel de $A$ est trivial. 
\end{proposition}





\begin{lemma}[théorème de pureté de Zariski-Nagata en dimension $2$] \label{5-2-4} % originally 5.2.3.1
Soient $C$ un anneau local régulier de dimension $2$ et $C'$ une $C$-algèbre 
finie normale étale au--dessus de l'ouvert complémentaire du point fermé 
de $\spec(C)$. Alors $C'$ est étale sur $C$.
\end{lemma}

En effet $C'$ est normal de dimension $2$, donc $\operatorname{prof}(C')=2$. 
Puisque $\operatorname{prof}(C')+\dim\operatorname{proj}(C') = \dim(C)=2$, on 
en conclut que $C'$ est libre sur $C$. Alors l'ensemble des points de $C$ où 
$C'$ est ramifié est défini par une équations, le discriminant; puisqu'il 
ne contient pas de point de hauteur $1$, il est vide. 





\begin{lemma}[cas particulier du lemme d'Abhyankar]\label{5-2-5} % originally 5.2.3.2
Soient $S=\spec(V)$ un trait, $\pi$ une uniformisante, $\eta$ le point 
générique de $S$, $X$ lisse sur $S$, irréductible, de dimension 
relative $1$, $\widetilde X_\eta$ un revêtement étale galoisien de $X_\eta$, 
de degré $n$ inversible sur $S$, et $S_1=\spec\left(V[\pi^{1/n}]\right)$. 
Notons par un indice $(-)_1$ le changement de base de $S$ à $S_1$. Alors, 
$\widetilde X_{1\eta}$ se prolonge en un revêtement étale de $X_1$. 
\end{lemma}

Soit $\widetilde X_1$ le normalisé de $X_1$ de $\widetilde X_{1\eta}$. Vu la 
structure des groupes d'inertie modérée des anneaux de valuation discrète 
localisés de $X$ aux points génériques de la fibre spéciale $X_s$, 
$\widetilde X_1$ est étale sur $X_1$ sur la fibre générale, et aux points 
génériques de la fibre spéciale. Par (\ref{5-2-4}), il est étale 
partout. 





\subsubsection{}\label{5-2-6} % originally 5.2.3.3

Notons $t$ le point en lequel $\bar t$ est localisé. Il nous est loisible de 
remplacer $A$ par le normalisé de $A$ dans une extension finie séparable 
$k(t')$ de $k(t)$ dans $k(\bar t)$ (cf. \ref{5-2-2}). Ceci, et un passage à 
la limite préliminaire, nous permettent de supposer que 
\begin{enumerate}[\indent a)]
  \item $A$ est normal noethérien, et $t$ est le point générique de $S$.
  \item Le revêtement étale considéré de $X_{\bar t}$ provient d'un 
    revêtement étale de $X_t$. 
  \item Il provient d'un revêtement étale $\widetilde X_U$ de l'image 
    réciproque $X_U$ d'un ouvert non vide $U$ de $S$ (résulte de b): $t$ 
    est la limite des $U$).
  \item La complément de $U$ est de codimension $\geqslant 2$ (ceci au prix 
    d'agrandir $k(t)$, par application de (\ref{5-2-5}) aux anneaux de 
    valuation discrète localisés de $S$ en les points de $S\setminus U$ de 
    codimension $1$ dans $S$; ces points sont en nombre fini). 
  \item Le revêtement $\widetilde X_U$ est trivial su-dessus du sous-schéma 
    $T=0$. (Ceci quitte à encore agrandir $k(t)$). 
\end{enumerate}

Lorsque ces conditions sont remplies, nous allons voir que le revêtement 
$\widetilde X_U$ est trivial. 





\begin{lemma}\label{5-2-7} % originally 5.2.3.4
Soient $A$ un anneau local strictement hensélien normal et noethérien, $U$ 
un ouvert de $\spec(A)$ dont le complément est de codimension $\geqslant 2$, 
$V$ son image réciproque dans $X=\spec\left(A\{T\}\right)$ et $V'$ un 
revêtement étale de $V$. Si $V'$ est trivial au-dessus de $T=0$, alors $V'$ 
est trivial. 
\end{lemma}

Soit $B=\Gamma(V',\cO)$. Puisque $V'$ est l'image inverse de $V$ dans 
$\spec(B)$, il suffit de voir que $B$ est fini étale sur $A\{T\}$ (donc 
décomposé, puisque $A\{T\}$ est strictement hensélien). Soit 
$\widehat X=A\llbracket T\rrbracket$, et notons par $(-)^\wedge$ le changement 
de base de $X$ à $\widehat X$. Le schéma $\widehat X$ est fidèlement plat 
sur $X$. On a donc 
$\Gamma(\widehat V',\cO) = B\otimes_{A\{T\}} A\llbracket T\rrbracket$, et il 
suffit de voir que cet anneau $\widehat B$ est fini étale sur 
$A\llbracket T\rrbracket$. 

Soit $V_m$ (resp. $V_m'$) le sous-schéma de $\widehat V$ (resp. $\widehat V'$) 
d'équation $T^{m+1} = 0$. Par hypothèse, $V_0'$ est un revêtement trivial 
de $V_0$: une somme de $n$ copies de $V_0$. De même pour $V_m'/V_m$, puisque 
les revêtements étales sont insensibles aux nilpotents. On en tire 
\[
  \varphi : \Gamma(\widehat V',\cO) \to \varprojlim_m \Gamma(V_m',\cO) = \left(\varprojlim_m \Gamma(V,\cO)\right)^n \text{.}
\]
Par hypothèse, le complément de $U$ est de profondeur $\geqslant 2$: on a 
$\Gamma(V_m,\cO)=A[T]/(T^{m+1})$, et $\varphi$ est un homomorphisme de 
$\widehat B$ dans $A\llbracket T\rrbracket^n$. Au-dessus de $U$, il fournit $n$ 
sections distinctes de $\widehat V'/\widehat V$: $\widehat V'$ est donc trivial, 
somme de $n$ copies de $\widehat V$. Le complément de $\widehat V$ dans 
$\widehat X$ étant encore de codimension $\geqslant 2$ (donc de profondeur 
$\geqslant 2$), on en déduit que $\widehat B=A\llbracket T\rrbracket^n$, 
d'où le lemme. 










\subsection{Applications}\label{5-3}





\begin{theorem}[spécialisation des groupes de cohomologie]\label{5-3-1}
Soit $f:X\to S$ un morphisme propre et localement acyclique, par exemple un 
morphisme propre et lisse. Alors les faisceaux $\R^q f_*\dZ/n$ sont localement 
constants constructibles et pour toute flèche de spécialisation 
$t\to\widetilde S^s$, les flèches de cospécialisation 
$\h^q(X_t,\dZ/n)\to\h^q(X_s,\dZ/n)$ sont bijectives.
\end{theorem}










\begin{thebibliography}{ab}
  \bibitem{Ar} M. Artin, \emph{Grothendieck topologies}, (Harvard University 1962).

  \bibitem{Gi} J. Giraud, \emph{Analysis situs}, Séminaire Bourbaki 256, mai 1963 -- Benjamin. 

  \bibitem{4} M. Artin, A. Grothendieck, J.L. Verdier. \emph{Théorie des Topos et Cohomologie Étale des Schémas}, SGA 4, Springer nos.269, 270 et 305, 1972/73.

  \bibitem{7} A. Grothendieck, \emph{Revêtements Étales et Groupe Fondamental}, SGA 1, Springer no.224, 1971.
  
  \bibitem{8} A. Grothendieck et J. Dieudonné, \emph{Éléments de géométrie algébrique}, Pub. Math. IHES, 1960/67. 

  \bibitem{11} M. Raynaud, \emph{Anneaux Locaux Henséliens}, Springer no.169, 1970. 
  
  \bibitem{12} J.-P. Serre, \emph{Géométrie Algébrique et Géométrie Analytique}, Ann. de l'Institut Fourier, 1956. 

  \bibitem{14} J.-P. Serre, \emph{Cohomologie galoisienne}, Springer no.5, 1965.

\end{thebibliography}





\end{document}
