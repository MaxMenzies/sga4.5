\documentclass[oneside]{book} % add option draft to highlight overful hbox
\usepackage{sga-style}





\title{Séminaire de Géométrie Algébrique du Bois-Marie \\ \vspace{20pt}
Cohomologie Étale \\ \vspace{20pt}
(SGA $4\frac 1 2$)}
\author{par P. Deligne \vspace{10pt}\\ 
avec la collaboration de J.F. Boutot, \\ A. Grothendieck, L. Illusie et J.L. Verdier}
\date{1977}


%% the part I'm currently working on 
%  \includeonly{sga4.5-ch6}
% NOTE: could go from \setminus to \smallsetminus, not clear which looks better


\begin{document}
\pagenumbering{alph} % title page # will be a
\maketitle

% French characters
% à é è ê ô ù û

% other notes: there are two I.6.5's in Deligne's original - one defining sheaf 
% cohomology, the other defining torsors






\pagenumbering{roman} % intro will be i, ii, ...
\section*{Typesetters note}

This is a \LaTeX rendition of Deligne's SGA $4\frac 1 2$. It is \emph{not} the 
original document, and any errors and typos are entirely my responsibility. 
You may also be violating copyright law by downloading this file. 

\chapter*{Introduction}

Ce volume a pour but de faciliter au non-expert l'usage de la 
cohomologie $\ell$-adique. J'espère qu'il lui permettra souvent d'éviter le 
recours aux exposés touffus de SGA 4 et SGA 5. Il contient aussi quelques 
résultats nouveaux. 

Le premier exposé, édigé par J.F. Boutot, survole SGA 4. Il donne les 
principaux résultats -- avec une généralité minimale, souvent 
insuffisante pour les applications -- et une idée de leur démonstration. Pour 
des résultats complets, ou des démonstrations détaillées, SGA 4 reste 
indispensable. 

Le ``\nameref{II}'' contient une démonstration complété de 
la formule des traces pour l'endomorphisme de Frobenius. La démonstration est 
celle donnée par Grothendieck dans SGA 5, élaguée de tout détail inutile. Ce 
Rapport devrait permettre à utilisateur d'oublier SGA 5, qu'on pourra 
considérer comme une série de digression, certaines très intéressantes. Son 
existence permettra de publier prochainement SGA 5 tel quel. Il est complété 
par l'exposé ``\nameref{VI}'' qui explique comment la formule des traces permet 
l'étude de sommes trigonométriques, et donne des exemples. 

Le public visé par les autres exposés est plus limité, et leur style s'en 
ressent. L'exposé ``\nameref{III}'' est une 
généralisation ``modulaire'' du Rappoport, basée sur l'étude SGA 4 XVII 5.5 des 
puissances symétriques. L'exposé ``\nameref{IV}'' 
définit cette classes dans divers contextes, et donne la compatibilité 
entre intersections et cup-produits. Dans ``\nameref{V}'' sont rassemblés 
quelques résultats connus, pour lesquels manquait une référence, et quelques 
compatibilités. L'exposé ``\nameref{VII}'' 
est nouveau. Il donne notamment, en cohomologie sans supports, des théorèmes de 
finitude analogues à ceux connus en cohomologie à supports compacts. 

Pour plus de détails sur les exposés, je renvoie à leur introduction 
respective. 

Je remercie enfin J.L. Verdier de m'avoir permis de reproduire ici ses notes 
``Catégories dérivées (Etat $0$).'' Elles restent je crois très utiles, et 
étaien: devenues introuvables. 

Dans les références internes à ce volume, les exposés sont cités par 
un titre abrégé, indiqué entre [ ] dans la table des matières. 

\vspace{20pt}
Bures-sur-Yvette, le 20 Septembre 1976

\vspace{5pt}
Pierre Deligne
\tableofcontents
\pagenumbering{arabic} % rest of pages are numbered the usual way
% !TEX root = sga4.5.tex

\chapter{Cohomologie étale: les points de départ}\label{I}
\footnote{par P.\ Deligne, rédigé par J.F.\ Boutot}




















\section{Topologies de Grothendieck}\label{I:1}

A l'origine, les topologies de Grothendieck sont apparues comme sous-jacentes 
à sa théorie de la descente (cf. SGA 1 VI, VIII); l'usage des théorie de 
cohomologie correspondantes est plus tardif. La même démarche est suivi 
ici: en formalisant les notions classiques de localisation, de propriété 
locale et de recollement (\ref{I:1-1}, \ref{I:1-2}, \ref{I:1-3}), en dégage le 
concept général de topologie de Grothendieck (\ref{I:1-6}); pour en justifier 
l'introduction en géométrie algébrique, on démontre un théorème 
de descente fidèlement plat (\ref{I:1-4}), généralisation du classique théorème 
de Hilbert (\ref{I:1-5}). 

Le lecteur trouvera une exposition plus compète, mais concise, du 
formalisme dans Giraud \cite{gi64}. Les notes de M. Artin: ``Grothendieck 
topologies'' \cite{ar62} (chapitres I à III) restent également utiles. Les 
866 pages des exposés I à V de SGA 4 sont précieuses lorsqu'on 
considère des topologies exotiques, telle celle qui donne naissance à la 
cohomologie cristalline; pour utiliser la topologie étale si proche de 
l'intuition classique, il n'est pas indispensable de les lire. 










\subsection{Cribles}\label{I:1-1}

Soient $X$ un espace topologique et $f:X\to \dR$ une fonction à valeur 
réelles sur $X$. La continuité de $f$ est une propriété de nature 
locale; autrement dit, si $f$ est continue sur tout ouvert suffisamment petit 
de $X$, $f$ est continue sur $X$ tout entier. Pour formaliser la notion de 
``propriété de nature locale,'' nous introduirons quelques définitions.

On dit qu'un ensemble $\cU$ d'ouverts de $X$ est un \emph{crible} si pour tout 
$U\in\cU$ et $V\subset U$, on a $V\in\cU$. On dit qu'un crible est 
\emph{couvrant} si la réunion de tout les ouverts appartenant à ce crible est 
égale à $X$.

Etant donnée une famille $\{U_i\}$ d'ouverts de $X$, le crible engendré par 
$\{U_i\}$ est par définition l'ensemble des ouverts $U$ de $X$ tels que $U$ 
soit contenu dans l'un des $U_i$. 

On dit qu'une propriété $P(U)$, définie pour tout ouvert $U$ de $X$, est 
\emph{locale} si, pour tout crible couvrant $\cU$ de tout ouvert $U$ de $X$, 
$P(U)$ est vraie si et seulement si $P(V)$ est vraie pour tout $V\in \cU$. Par 
exemple, étant donné $f:X\to \dR$, la propriété ``$f$ est continue sur $U$'' 
est locale. 





\subsection{Faisceaux}\label{I:1-2}

Précisons la notion de fonction donnée localement sur $X$.





\subsubsection{Point de vue des cribles}\label{I:1-2-1}

Soit $\cU$ un crible d'ouverts de $X$. On appelle fonction donnée 
$\cU$-localement sur $X$ la donnée pour tout $U\in \cU$ d'une fonction $f_U$ 
sur $U$ telle que, si $V\subset U$, on ait $f_V=f_U|V$. 





\subsubsection{Pointe de vue de Čech}\label{I:1-2-2}

Si le crible $\cU$ est engendré par une famille d'ouverts $U_i$ de $X$, se 
donner une fonction $\cU$-localement revient à se donner une fonction $f_i$ 
sur chaque $U_i$, telle que $f_i|{U_i\cap U_j} = f_j|{U_i\cap U_j}$. 

Autrement dit, si $Z=\coprod U_i$, se donner une fonction $\cU$-localement 
revient à se donner une fonction sur $Z$ qui soit constante sur les fibres de 
la projection naturelle $Z\to X$. 





\subsubsection{}\label{I:1-2-3}

Les fonctions continues forment un faisceau; cela signifie que pour tout crible 
couvrant $\cU$ d'un ouvert $V$ de $X$ et toute fonction donnée 
$\cU$-localement $\{f_U\}$ telle que chaque $f_U$ soit continue sur $U$, il 
existe une unique fonction continue $f$ sur $V$ telle que $f|U=f_U$ pour tout 
$U\in \cU$.










\subsection{Champs}\label{I:1-3}

Précisons maintenant la notion de fibré vectoriel donné localement sur $X$. 





\subsubsection{Point de vue des cribles}\label{I:1-3-1}

Soit $\cU$ un crible d'ouverts de $X$. On appelle fibré vectoriel donné 
$\cU$-localement sur $X$ les données de 
\begin{enumerate}[\indent a)]
  \item un fibré vectoriel $E_U$ sur chaque $U\in \cU$, 
  \item si $V\subset U$, un isomorphisme $\rho_{U,V} : E_V\iso E_U|V$, vérifiant 
  \item si $W\subset V\subset U$, le diagramme 
    \[\xymatrix{
      E_W \ar[r]^-{\rho_{U,W}} \ar[dr]_-{\rho_{V,W}}
        & E_U|W \\
      & E_V|W \ar[u]_-{\rho_{U,V}|W}
    }\]
    commute, c'est-à-dire 
    $\rho_{U,V} = (\mbox{$\rho_{U,V}$ restreint à $W$}) \circ \rho_{V,W}$. 
\end{enumerate}





\subsubsection{Point de vue de Čech}\label{I:1-3-2}

Si le crible $\cU$ est engendré par une famille d'ouverts $U_i$ de $X$, se 
donner un fibré vectoriel $\cU$-localement revient à se donner:
\begin{enumerate}[\indent a)]
  \item un fibré vectoriel $E_i$ sur chaque $U_i$, 
  \item si $U_{ij} = U_i\cap U_j = U_i\times_X U_j$, un isomorphisme 
    $\rho_{ji} : E_i|U_{ij} \iso E_j|U_{ij}$, de sorte que 
  \item si $U_{ijk} = U_i\times_X U_j\times_X U_k$, le diagramme 
    \[\xymatrix{
      E_i |U_{ijk} \ar[r]^-{\rho_{ki}|U_{ijk}} \ar[dr]_-{\rho_{ji}|U_{ijk}} 
        & E_k|U_{ijk} \\
      & E_j|U_{ijk} \ar[u]_-{\rho_{kj}|U_{ijk}}
    }\]
    commute, c'est-à-dire $\rho_{ki} = \rho_{kj}\circ \rho_{ji}$ sur 
    $U_{ijk}$. 
\end{enumerate}

Autrement dit, si $Z=\coprod U_i$ et si $\pi:Z\to X$ est la projection 
naturelle, se donner un fibré vectoriel $\cU$-localement revient à se donner: 
\begin{enumerate}[\indent a)]
  \item un fibré vectoriel $E$ sur $Z$, 
  \item si $x$ et $y$ sont deux points de $Z$ tels que $\pi(x) = \pi(y)$, un 
    isomorphisme $\rho_{y x}:E_x\iso E_y$ entre les fibres de $E$ en $x$ et en 
    $y$, dépendant continûment de $(x,y)$ et tel que, 
  \item si $x$, $y$ et $z$ sont trois points de $Z$ tels que 
  $\pi(x)=\pi(y)=\pi(z)$, on ait $\rho_{zx} = \rho_{zy}\circ \rho_{yx}$. 
\end{enumerate}





\subsubsection{}\label{I:1-3-3}

Une fibré vectoriel $E$ sur $X$ définit un fibré vectoriel donné 
$\cU$-localement $E_\cU$; le système des restrictions $E_U$ de $E$ aux objets 
de $\cU$. Le fait que la notion de fibré vectoriel est de nature locale peut 
s'exprimer ainsi: pour tout crible couvrant $\cU$ de $X$, le foncteur 
$E\mapsto E_\cU$, des fibrés vectoriels sur $X$ dans les fibrés vectoriels 
donnés $\cU$-localement, est une équivalence de catégories. 





\subsubsection{}\label{I:1-3-4}

Si dans \ref{I:1} on remplace ``ouvert de $X$'' par ``partie de 
$X$,'' on obtient la notion de crible de sous-espaces de $X$. Dans ce cadre 
aussi on dispose de théorèmes de recollement. Par exemple: soient $X$ un 
espace et $\cC$ un crible de sous-espaces de $X$ engendré par un recouvrement 
fermé localement fini de $X$, alors le foncteur $E\mapsto E_\cC$, des fibrés 
vectoriels sur $X$ dans les fibrés vectoriels donnés $\cC$-localement est une 
équivalence de catégories. 

En géométrie algébrique, il est utile de considérer aussi des ``cribles 
d'espaces au-dessus de $X$''; c'est ce que nous verrons au paragraphe suivant. 










\subsection{Descente fidèlement plat}\label{I:1-4}





\subsubsection{}\label{I:1-4-1}

Dans le cadre des schémas, la topologie de Zariski n'est pas assez fine pour 
l'étude des problèmes non linéaires et on est amené à remplacer dans les 
définitions précédentes les immersions ouvertes par des morphismes plus 
généraux. De ce point de vue, les techniques de descente apparaissent comme des 
techniques de localisation. Ainsi l'énoncé de descente suivant peut sexprimer 
en disant que les propriétés considérées sont de nature locale pour la 
topologie fidèlement plate (on dit qu'un morphisme de schémas est fidèlement 
plat s'il est plat est surjectif). 


\begin{proposition}\label{I:1-4-2}
Soient $A$ un anneau et $B$ une $A$-algèbre fidèlement plate. Alors:
\begin{enumerate}[(i)]
  \item Une suite $\Sigma=(M'\to M\to M'')$ de $A$-modules est exacte dés 
    que la suite $\Sigma_{(B)}$ qui s'en déduit par extension des scalaires 
    à $B$ est exacte.
  \item Un $A$-module $M$ est de type fini (resp. de présentation finie, 
    plat, localement libre de rang fini, inversible (i.e. localement libre 
    de rang un)) dés que le $B$-module $M_{(B)}$ l'est.
\end{enumerate}
\end{proposition}
\begin{proof}
(i) Le foncteur $M\mapsto M_{(B)}$ étant exact (platitude de $B$), il suffit 
de montrer que, si un $A$-module $N$ est non nul, $N_{(B)}$ est non nul. Si 
$N$ est non nul, $N$ contient un sous-module monogène non nul $A/\fa$; alors 
$N_{(B)}$ contient un sous-module monogène $(A/\fa)_{(B)} = B/\fa B$, non nul 
par surjectivité du morphisme structural $\varphi:\spec(B)\to\spec(A)$ (si 
$V(\fa)$ est non vide, $\varphi^{-1}(V(\fa))=V(\fa B)$ est non vide). 

(ii) Pour toute famille $(x_i)$ d'éléments de $M_{(B)}$, il existe un 
sous-module de type fini $M'$ de $M$ tel que $M'_{(B)}$ contienne les $x_i$. Si 
$M_{(B)}$ est de type fini et si les $x_i$ engendrent $M_{(B)}$, on a 
$M'_{(B)}=M_{(B)}$, donc $M'=M$ et $M$ est de type fini. 
\end{proof}

Si $M_{(B)}$ est de présentation finie, on peut, d'aprés ce qui précède, 
trouver une surjection $A^n\to M$. Si $N$ est le noyau de cette surjection, le 
$B$-module $N_{(B)}$ est de type fini, donc $N$ l'est, et $M$ est de 
présentation finie. L'assertion pour ``flat'' résulte aussitôt de (i); 
``localement libre de rang fini'' signifie ``plat et de présentation finie'' 
et le rang se teste par extension des scalaires à des corps.





\subsubsection{}\label{I:1-4-3}

Soient $X$ un schéma et $\cS$ une classe de $X$-schémas stable par produit 
fibré sur $X$. Une classe $\cU\subset \cS$ est un \emph{crible} sur $X$ 
(relativement à $\cS$) si, pour tout morphisme $\varphi:V\to U$ de 
$X$-schémas, avec $U,V\in \cS$ et $U\in\cU$, on a $V\in \cU$. Le crible 
\emph{engendré} par une famille $\{U_i\}$ de $X$-schémas dans $\cS$ est la 
classe des $V\in\cS$ tels qu'il existe un morphisme de $X$-schémas de $V$ dans 
l'un des $U_i$. 






\subsubsection{}\label{I:1-4-4}

Soit $\cU$ un crible sur $X$. On appelle module quasi-cohérent donné 
$\cU$-localement sur $X$ la donnée de 
\begin{enumerate}[\indent a)]
  \item un module quasi-cohérent $E_U$ sur chaque $U\in\cU$, 
  \item pour tout $U\in\cU$ et pour tout morphisme $\varphi:V\to U$ de 
    $X$-schémas dans $\cS$, un isomorphisme 
    $\rho_\varphi:E_V\iso \varphi^* E_U$, ceux-ci étant tels que 
  \item si $\psi:W\to V$ est un morphisme de $X$-schémas dans $\cS$, le 
    diagramme 
    \[\xymatrix{
      E_W \ar[r]^-{\rho_{\varphi\circ\psi}} \ar[dr]_-{\rho_\psi} 
        & \psi^*\varphi^* E_U \\
      & \psi^* E_V \ar[u]_-{\psi^*\rho_\varphi}
    }\]
    commute, c'est-à-dire 
    $\rho_{\varphi\circ\psi}=(\psi^*\rho_\varphi)\circ \rho_\psi$. 
\end{enumerate}

Si $E$ est un module quasi-cohérent sur $X$, on note $E_\cU$ le module 
donné $\cU$-localement valant $\varphi_U^* E$ sur $\varphi:U\to X$ et tel que, 
pour tout morphisme $\psi:V\to U$ l'isomorphisme de restriction $\rho_\psi$ 
soit l'isomorphisme canonique 
$E_V=(\varphi_U\circ \psi)^* E\iso \psi^* \varphi_U^* E = \psi^* E_U$. 


\begin{theorem}\label{I:1-4-5}
Soit $\{U_i\}\in\cS$ une famille finie de $X$-schémas plats sur $X$ telle que 
$X$ soit le réunion des images des $U_i$, et soit $\cU$ le crible engendré par 
$\{U_i\}$. Alors le foncteur $E\mapsto E_\cU$ est une équivalence de la 
catégorie des modules quasi-cohérents sur $X$ avec la catégorie des modules 
quasi-cohérents donnés $\cU$-localement.
\end{theorem}
\begin{proof}
Nous ne traiterons que le cas où $x$ est affine et où $\cU$ est engendré par 
un $X$-schéma affine $U$ fidèlement plat sur $X$. La réduction à ce cas est 
formelle. On pose $X=\spec(A)$ et $U=\spec(B)$. 

Si le morphisme $U\to X$ admet une section, $X$ appartient au crible $\cU$ et 
l'assertion est évidente. Nous nous réduirons à ce cas. 

Un module quasi-cohérent donné $\cU$-localement définit des modules $M'$, $M''$ 
et $M'''$ sur $U$, $U\times_X U$ et $U\times_X U\times_X U$, et des 
isomorphismes $\rho:p^* M^\bullet\simeq M^\bullet$ pour tout morphisme de 
projection $p$ entre ces espaces; c'est là un \emph{diagramme cartésien} 
\[\xymatrix{
  M^* : M' \ar@<2pt>[r] \ar@<-2pt>[r] 
    & M'' \ar[r] \ar@<4pt>[r] \ar@<-4pt>[r] 
    & M'''
}\]
au-dessus de 
\[\xymatrix{
  U_* : U 
    & U\times_X U \ar@<2pt>[l] \ar@<-2pt>[l] 
    & U\times_X U\times_X U \ar[l] \ar@<4pt>[l] \ar@<-4pt>[l] \mbox{.}
}\]
Réciproquement $M^*$ détermine le module donné $\cU$-localement: pour 
$V\in\cU$, il existe $\varphi:V\to U$ et on pose $M_U=\varphi^* M'$; 
pour $\varphi_1,\varphi_2:V\to U$, on a une identification naturelle 
$\varphi_1^* M'\simeq (\varphi_1\times \varphi_2)^* M'' \simeq \varphi_2^* M'$, 
et on voit en utilisant $M'''$ ue ces identifications sont compatibles, de 
sorte que la définition est légitime. Bref, il revient au même de se donner un 
module $\cU$-localement ou un diagramme $M^*$ cartésien sur $U_*$. 

Traduisons en termes algébriques: se donner $M^*$ revient à se donner un 
diagramme cartésien de modules 
\[\xymatrix{
  M' \ar@<3pt>[r]|-{\partial_0} \ar@<-3pt>[r]|-{\partial_1} 
    & M'' \ar@<6pt>[r]|-{\partial_0} \ar[r]|-{\partial_1} \ar@<-6pt>[r]|-{\partial_2}
    & M'''
}\]
au-dessus du diagramme d'anneaux 
\[\xymatrix{
  B \ar@<3pt>[r]|-{\partial_0} \ar@<-3pt>[r]|-{\partial_1} 
    & B\otimes_A B \ar@<6pt>[r]|-{\partial_0} \ar[r]|-{\partial_1} \ar@<-6pt>[r]|-{\partial_2} 
    & B\otimes_A B\otimes_A B
}\]
(précisons: on a $\partial_i(b m)=\partial_i(b)\cdot\partial_i(m)$, les 
identités usuelles telles que $\partial_0\partial_1=\partial_0\partial_0$ 
sont vraies, et ``cartésien'' signifie que les morphismes 
$\partial_i:M'\otimes_{B,\partial_i}(B\otimes_A B)\to M''$ et 
$M''\otimes_{B\otimes_A B,\partial_i}(B\otimes_A B\otimes_A B)\to M'''$ sont 
des isomorphismes). 

Le foncteur $E\mapsto E_\cU$ devient le foncteur qui, à un $A$-module $M$, 
associe 
\[\xymatrix{
  M^* = (M\otimes_A B \ar@<2pt>[r] \ar@<-2pt>[r] 
    & M\otimes_A B\otimes_A B \ar@<-4pt>[r] \ar[r] \ar@<4pt>[r] 
    & M\otimes_A B\otimes_A B\otimes_A B)\text{.}
}\]
Il admet pour adjoint à droite le foncteur
\[\xymatrix{
  (M' \ar@<2pt>[r] \ar@<-2pt>[r] 
    & M'' \ar@<-4pt>[r] \ar[r] \ar@<4pt>[r]
    & M''') \ar@{|->}[r] 
    & \ker(M' \ar@<2pt>[r] \ar@<-2pt>[r] 
    & M''')\text{.}
}\]
Il nous faut prouver que les flèches d'adjonction 
\[
  M \to\ker(M\otimes_A B\rightrightarrows M\otimes_A B\otimes_A B)
\]
et
\[
  \ker(M'\rightrightarrows M'')\otimes_A B \to M'
\]
sont des isomorphismes. D'après (\ref{I:1-4-2}.i), il suffit de le prouver 
après un changement de base fidèlement plat $A\to A'$ ($B$ devenant 
$B'=B\otimes_A A'$). Prenant $A'=B$, ceci nous ramène au cas où $U\to X$ 
admet une section. 
\end{proof}










\subsection{Un cas particulier: le théorème 90 de Hilbert}\label{I:1-5}





\subsubsection{}\label{I:1-5-1}

Soient $k$ un corps, $k'$ une extension galoisienne de $k$ et $G=\gal(k'/k)$. 
Alors l'homo\~morphisme 
\begin{align*}
  k'\otimes_k k' &\to \oplus_{\sigma\in G} k' \\
  x\otimes y     &\mapsto \{x\cdot \sigma(y)\}_{\sigma\in G}
\end{align*}
est bijectif. 

On en déduit qu'il revient au même de se donner un module localement pour 
le crible engendré par $\spec(k')$ sur $\spec(k)$ ou de se donner un 
$k'$-espace vectoriel muni d'une action semi-linéaire de $G$, c'est-à-dire: 
\begin{enumerate}[\indent a)]
  \item un $k'$-espace vectoriel $V'$,
  \item pour tout $\sigma\in G$, un endomorphisme $\varphi_\sigma$ de la 
    structure de groupe de $V'$ tel que 
    $\varphi_\sigma(\lambda v) = \sigma(\lambda) \varphi_\sigma(v)$, pour 
    tout $\lambda\in k'$ et $v\in V'$, vérifiant la condition
  \item pour tout $\sigma,\tau\in G$, on a 
    $\varphi_{\tau\sigma} = \varphi_\tau\circ\varphi_\sigma$. 
\end{enumerate}

Soit $V={V'}^G$ le groupe des invariants par cette action de $G$; c'est un 
$k$-espace vectoriel et, d'après le théorème \ref{I:1-4-5}, on a:

\begin{proposition}\label{I:1-5-2}
L'inclusion de $V$ dans $V'$ définit un isomorphisme 
$V\otimes_k k' \iso V'$.
\end{proposition}

En particulier, si $V'$ est de dimension $1$ et si $v'\in V$ est non nul, 
$\varphi_\sigma$ est déterminé par la constante 
$c(\sigma)\in {k'}^\times$ telle que $\varphi_\sigma(v')=c(\sigma)v'$ et la 
condition c) s'écrit 
\[
  c(\tau\sigma) = c(\tau)\cdot \tau(c(\sigma))\text{.}
\]
D'après la proposition il existe un vecteur invariant non nui 
$v=\mu v'$, $\mu\in {k'}^\times$. On a donc pour tout $\sigma\in G$, 
\[
  c(\sigma) = \mu\cdot \sigma(\mu^{-1})\text{.}
\]
Autrement dit tout $1$-cocycle de $G$ à valeurs dans ${k'}^\times$ est un 
cobord: 

\begin{corollary}\label{I:1-5-3}
On a $\h^1(G,{k'}^\times)=0$.
\end{corollary}










\subsection{Topologies de Grothendieck}\label{I:1-6}

Nous transcrivons maintenant les définitions des paragraphes précédents 
dans un cadre abstrait englobant à la fois le cas des espaces topologiques 
et celui des schémas. 





\subsubsection{}\label{I:1-6-1}

Soient $\cS$ une catégorie et $U$ un objet de $\cS$. On appelle \emph{crible} 
sur $U$ un sous-ensemble $\cU$ de $\ob(\cS/U)$ tel que si $\varphi:V\to U$ 
appartient à $\cU$ et si $\psi:W\to V$ est un morphisme dans $\cS$, alors 
$\varphi\circ\psi:W\to U$ appartient à $\cU$. 

Si $\{\varphi_i:U_i\to U\}$ est une famille de morphismes, le crible 
engendré par les $U_i$ est par définition l'ensemble des morphismes 
$\varphi:V\to U$ qui se factorisent à travers l'un des $\varphi_i$. 

Si $\cU$ est un crible sur $U$ et si $\varphi:V\to U$ est un morphisme, la 
restriction $\cU_V$ de $\cU$ à $V$ est par définition le crible sur $V$ 
constitué oar les morphismes $\psi:w\to V$ tels que 
$\varphi\circ\psi:W\to U$ appartienne à $\cU$.





\subsubsection{}\label{I:1-6-2}

La donnée d'une \emph{topologie de Grothendieck} sur $\cS$ consiste en la 
donnée pour tout objet $U$ de $\cS$ d'un ensemble $C(U)$ de cribles sur $U$, 
dits cribles couvrants, de telle sorte que les axiomes suivants soient 
satisfaits:
\begin{enumerate}[\indent a)]
  \item Le crible engendré par l'identité de $U$ est couvrant.
  \item Si $\cU$ est un crible couvrant sur $U$ et si $V\to U$ est un 
    morphisme, le crible $\cU_V$ est couvrant.
  \item Un crible localement couvrant est couvrant. Autrement dit, si $\cU$ est 
    un crible couvrant sur $U$ et si $\cU'$ est un crible sur $U$ tel que, pour 
    tout $V\to U$ appartenant à $\cU$, le crible $\cU_V'$ est couvrant, alors 
    $\cU'$ est couvrant.
\end{enumerate}

On appelle \emph{site} la donnée d'une catégorie munie d'une topologie de 
Grothendieck.





\subsubsection{}\label{I:1-6-3}

Étant donnée un site $\cS$, on appelle préfaisceau sur $\cS$ un foncteur 
contravariant $\sF$ de $\cS$ dans la catégorie des ensembles. Pour tout 
objet $U$ de $\cS$, on appelle section de $\sF$ au-dessus de $U$ les 
éléments de $\sF(U)$. Pour tout morphisme $V\to U$ et pour tout 
$s\in \sF(U)$, on note $s|V$ ($s$ restreint à $V$) l'image de $s$ dans 
$\sF(V)$.

Si $\cU$ est un crible sur $U$, on appelle section donnée $\cU$-localement la 
donnée, pour tout $V\to U$ appartenant à $\cU$, d'une section 
$s_V\in\sF(V)$ telle que, pour tout morphisme $W\to V$, on sit $s_V|W=s_W$. On 
dit que $\sF$ est un \emph{faisceau} si, pour tout objet $U$ de $\cS$, pour 
tout crible couvrant $\cU$ sur $U$ et pour tout section donnée 
$\cU$-localement $\{s_V\}$, il existe une unique section $s\in\sF(U)$ telle que 
$s|V=s_V$, pour tout $V\to U$ appartenant à $\cU$. 

On définit de manière analogue les \emph{faisceaux abéliens} en 
remplaçant la catégorie des ensembles par celle des groupes abéliens. On 
montre que la catégorie des faisceaux abéliens sur $\cS$ est une 
catégorie abélienne possédant suffisamment d'injectifs. Une suite 
$\sF\xrightarrow f \sG\xrightarrow g \sH$ de faisceaux est exacte si, pour 
tout objet $U$ de $\cS$, et pour tout $s\in \sG(U)$ telle que $g(s)=0$, 
il existe localement $t$ tel que $f(t)=s$; i.e. s'il existe un crible couvrant 
$\cU$ sur $U$ et pour tout $V\in\cU$, une section $t_V$ de $\sF$ sur $V$ telle 
que $f(t_V)=s|V$. 





\subsubsection{Exemples}\label{I:1-6-4}

Nous en avons vu deux plus haut. 

\begin{enumerate}[\indent a)]
  \item Soient $X$ un espace topologique et $\cS$ la catégorie dont les objets 
    sont les ouverts de $X$ et les morphismes les inclusions naturelles. La 
    topologique de Grothendieck sur $\cS$ correspondant à la topologie usuelle 
    de $X$ est celle pour laquelle un crible $\cU$ sur un ouvert $U$ de $X$ est 
    couvrant si la réunion des ouverts appartenant à ce crible est égale à 
    $U$. Il est clair que la catégorie des faisceaux sur $\cS$ est équivalente 
    à la catégorie des faisceaux sur $X$ au sens usuel. 
  \item Soient $X$ un schéma et $\cS$ la catégorie des schémas sur $X$. On 
    appelle topologie fpqc (fidèlement plate quasi-compacte) sur $\cS$ la 
    topologie de Grothendieck pour laquelle un crible sur un $X$-schéma $U$ est 
    couvrant s'il est engendré par une famille finie de morphismes plats dont  
    les images recouvrent $U$. 
\end{enumerate}





\subsubsection{Cohomologie}\label{I:1-6-5}

On supposera toujours que la catégorie $\cS$ a un objet final $X$. Alors on 
appelle sections globales d'un faisceaux abélien $\sF$, et on note 
$\Gamma\sF$ ou $\h^0(X,\sF)$, le groupe $\sF(X)$. Le foncteur 
$\sF\mapsto\Gamma\sF$ est un foncteur exact \'a gauche de la catégorie des 
faisceaux abéliens sur $\cS$ dans la catégorie des groupes abéliens, ou 
note $\h^i(X,-)$ ses dérivés (ou satellites). Ces groupes de cohomologie 
représentent les obstructions à passer du local au global. Par définition, si 
$0\to\sF\to\sG\to\sH\to 0$ est une suite exacte de faisceaux abéliens, on a une 
suite exacte longue de cohomologie:
\[\xymatrix{
  0 \ar[r] 
    & \h^0(X,\sF) \ar[r] 
    & \h^0(X,\sG) \ar[r] 
    & \h^0(X,\sH) \ar[r] 
    & \h^1(X,\sF) \ar[r] 
    & \cdots \\
  \ldots \ar[r] 
    & \h^n(X,\sF) \ar[r] 
    & \h^n(X,\sG) \ar[r]
    & \h^n(X,\sH) \ar[r] 
    & \h^{n+1}(X,\sF) \ar[r] 
    & \cdots
}\]





\subsubsection{}\label{I:1-6-6}

Étant donné un faisceau abélien $\sF$ sur $\cS$, on appelle 
\emph{$\sF$-torseur} un faisceau $\sG$ muni d'une action $\sF\times\sG\to\sG$ 
de $\sF$ telle que localement (après restriction à tous les objets d'un 
crible couvrant l'objet final $X$) $\sG$ muni de l'action de $\sF$ soit 
isomorphe à $\sF$ muni de l'action canonique $\sF\times \sF\to \sF$ par 
translations. 

On peut montrer que $\h^1(X,\sF)$ s'interprète comme l'ensemble des classes 
à isomorphisme près de $\sF$-torseurs. 




















\section{Topologie étale}\label{I:2}

On spécialise les définitions du chapitre précédent au cas de la 
topologie étale d'un schéma $X$ (\ref{I:2-1}, \ref{I:2-2}, \ref{I:2-3}). La 
cohomologie correspondante coïncide dans le cas où $X$ est le spectre d'un 
corps $K$ avec la cohomologie galoisienne de $K$ (\ref{I:2-4}). 










\subsection{Topologie étale}\label{I:2-1}

Nous commencerons par quelques rappels sur la notion de morphisme étale. 

\begin{definition}\label{I:2-1-1}
Soit $A$ un anneau (commutatif). On dit qu'une $A$-algèbre $B$ est étale si 
$B$ est une $A$-algèbre de présentation finie et si les conditions 
équivalentes suivantes sont vérifiées:
\begin{enumerate}[\indent a)]
  \item Pour toute $A$-algèbre $C$ et pour tout idéal de carré nul $J$ de 
    $C$, l'application canonique 
    \[
      \hom_{A\textnormal{-alg}}(B,C) \to \hom_{A\textnormal{-alg}}(B,C/J)
    \]
    est une bijection.
  \item $B$ est un $A$-module plat et $\Omega_{B/A}=0$ (on note $\Omega_{B/A}$ 
    le module des différentielles relatives).
  \item Soit $B=A[X_1,\dotsc,X_n]/I$ une présentation de $B$. Alors pour tout 
    idéal premier $\fp$ de $A[X_1,\dotsc,X_n]$ contenant $I$, il existe des 
    polynômes $P_1,\dotsc,P_n\in I$ tels que $I_\fp$ soit engendré par les 
    images de $P_1,\dotsc,P_n$ et $\det(\partial P_i/\partial X_j)\notin \fp$.  
\end{enumerate}
\end{definition}
(cf. \cite[I]{sga1} ou \cite[V]{ra70})
%[cf. SGA I, exposé I ou M. {\sc Raynoud}, \emph{Anneaux Locaux Henséliens}, chapitre V]. 

On dit qu'un morphisme de schémas $f:X\to S$ est \emph{étale} si pour tout 
$x\in X$ il existe un voisinage ouvert affine $U=\spec(A)$ de $f(x)$ et un 
voisinage ouvert affine $V=\spec(B)$ de $x$ dans $X\times_S U$ tel que $B$ soit 
une $A$-algèbre étale. 





\subsubsection{Exemples}\label{I:2-1-2}
\begin{enumerate}[\indent a)]
  \item Si $A$ est un corps, une $A$-algèbre $B$ est étale si et seulement 
    si c'est un produit fini d'extensions séparables de $A$. 
  \item Si $X$ et $S$ sont des schémas de type fini sur $\dC$, un morphisme 
    $f:X\to S$ est étale si et seulement si son analyticité 
    $f^{\text{an}}:X^{\text{an}}\to Y^{\text{an}}$ est un isomorphisme local.
\end{enumerate}





\subsubsection{Sorite}\label{I:2-1-3}
\begin{enumerate}[\indent a)]
  \item (changement de base) Si $f:X\to S$ est un morphisme étale, il en est 
    de même de $f_{S'}:X\times_S S'\to S'$ pour tout morphisme $S'\to S$. 
  \item Si $f:X\to S$ et $g:Y\to S$ sont deux morphismes étales, tout 
    $S$-morphisme de $X$ dans $Y$ est étale.
  \item (descente) Soit $f:X\to S$ un morphisme. S'il existe un morphisme 
    fidèlement plat $S'\to S$, tel que $f_{S'}:X\times_S S'\to S'$ soit 
    étale, alors $f$ est étale.
\end{enumerate}





\subsubsection{}\label{I:2-1-4}

Soit $X$ un schéma. Soit $\cS$ la catégorie des $X$-schémas étales; 
d'après (\ref{I:2-1-3}.c) tout morphisme de $\cS$ est un morphisme étale. 
On appelle \emph{topologie étale} sur $\cS$ la topologie pour laquelle un 
crible sur $U$ est couvrant s'il est engendré par une famille finie de 
morphismes $\varphi_i:U_i\to U$ tels que la réunion des images des 
$\varphi_i$ recouvre $U$. On appelle \emph{site étale} de $X$, et note 
$\et X$, le site défini par $\cS$ de la topologie étale. 










\subsection{Exemples de faisceaux}\label{I:2-2}





\subsubsection{Faisceau constant}\label{I:2-2-1}

Soit $C$ un groupe abélien et supposons pour simplifier $X$ noethérien. On 
notera $\const C_X$ (ou même $C$ s'il n'y a pas d'ambiguïté) le faisceau 
défini par $U\mapsto C^{\pi_0(U)}$, où $\pi_0(U)$ est l'ensemble (fini) des 
composantes connexes de $U$. Le cas le plus important sera $C=\dZ/n$. On a donc 
par définition 
\[
  \h^0(X,\dZ/n)=\left(\dZ/n\right)^{\pi_0(X)}\text{.}
\]
De plus $\h^1(X,\dZ/n)$ est l'ensemble des classes d'isomorphisme de 
$\dZ/n$-torseurs (\ref{I:1-6-6}), autrement dit de revêtements étales 
galoisiens de $X$ de groupe $\dZ/n$. En particulier, si $X$ est connexe et si 
$\pi_1(X)$ est son groupe fondamental pour un pointe bas choisi, on a 
\[
  \h^1(X,\dZ/n) = \hom(\pi_1(X),\dZ/n)\text{.}
\]





\subsubsection{Groupe multiplicatif}\label{I:2-2-2}

On notera $\dG_{m,X}$ (ou $\dG_m$ s'il n'y a pas d'ambiguïté) le faisceau 
défini par $U\mapsto \Gamma(U,\sO_U^\times)$; il s'agit bien d'un faisceau 
grâce au théorème de descente fidèlement plate (\ref{I:1-4-5}). On a 
par définition 
\[
  \h^0(X,\dG_m) = \h^0(X,\sO_X)^\times\text{;}
\]
en particulier si $X$ est réduit, connexe et propre sur un corps 
algébriquement clos $k$, on a:
\[
  \h^0(X,\dG_m) = k^\times\text{.}
\]

\begin{proposition}\label{I:2-2-3}
On a un isomorphisme:
\[
  \h^1(X,\dG_m) = \pic(X)\text{,}
\]
où $\pic(X)$ est le groupe des classes de faisceaux inversibles sur $X$.
\end{proposition}
\begin{proof}
Soit $*$ le foncteur qui, à un faisceau inversible $\sL$ sur $X$, associe le 
préfaisceau $\sL^*$ suivant sur $\et X$: pour $\varphi:U\to X$ étale, 
\[
  \sL^*(U)=\operatorname{Isom}_U(\sO_U,\varphi^*\sL)\text{.}
\]
% the original references here are just to (4.2) and (4.5), but the context 
% makes it seem that they refer to the results in chapter 1. Likewise a little 
% later
D'après (\ref{I:1-4-2}.i) et (\ref{I:1-4-5}) (pleine fidélité), ce 
préfaisceau est un faisceau; c'est même un $\dG_m$-torseur. On vérifie 
aussitôt que 
\begin{enumerate}[\indent a)]
  \item le foncteur $*$ est compatible à la localisation (étale);
  \item il induit une équivalence de la catégorie des faisceaux 
    inversibles triviaux (i.e. à $\sO_X$) avec la catégorie des 
    $\dG_m$-torseurs triviaux: $\sL$ est trivial si et seulement si $\sL^*$ 
    l'est.
\end{enumerate}

De plus, d'après (\ref{I:1-4-2}.ii) et (\ref{I:1-4-5}), 
\begin{enumerate}[\indent a)]
\setcounter{enumi}{2}
  \item la notion de faisceau inversible est locale pour la topologie étale. 
\end{enumerate}

Il résulte formellement de a), b), c) que $*$ est une équivalence entre la 
catégorie des faisceaux inversibles sur $X$ est celle des $\dG_m$-torseurs 
sur $\et X$; elle induit l'isomorphisme cherché. On construit comme suit 
l'équivalence inverse: si $\sT$ est un $\dG_m$-torseur, il existe un 
recouvrement étale fini $\{U_i\}$ de $X$ tel que les torseurs $\sT/U_i$ soient 
triviaux; $\sT$ est alors trivial sur chaque $V$ étale sur $X$ appartenant au 
crible $\cU\subset \et X$ engendré par $\{U_i\}$. Sur chaque 
$V\in\cU$, $\sT|V$ correspond à un faisceau inversible $\sL_V$ (par b)) et les 
$\sL_V$ constituent un faisceau inversible donné $\cU$-localement $\sL_\cU$ 
(par a)). Par c), ce dernier provient d'un faisceau inversible $\sL(\sT)$ sur 
$X$, et $\sT\mapsto \sL(\sT)$ est l'inverse cherché de $*$. 
\end{proof}





\subsubsection{Racines de l'unité}\label{I:2-2-4}

Pour tout entier $n>0$, on appelle faisceau des racines $n$-ièmes de 
l'unité, et on note $\dmu_n$, le noyau de l'élévation à puissance 
$n$-iéme dans $\dG_m$. Si $X$ est un schéma sur un corps séparablement 
clos $k$ et si $n$ est inversible dans $k$, le choix d'une racine primitive 
$n$-ième de l'unité $\zeta\in k$ définit un isomorphisme 
$i\mapsto \zeta^i$ de $\dZ/n$ avec $\dmu_n$. 

La relation entre cohomologie à coefficients dans $\dmu_m$ et cohomologie à 
coefficients dans $\dG_m$ est donnée par la suite exacte de cohomologie déduite 
de la 






\begin{theorem}[Théorie de Kummer]\label{I:2-2-5}
Si $n$ est inversible sur $X$, l'élévation à la puissance $n$-ième dans 
$\dG_m$ est un épimorphisme de faisceaux. On a donc une suite exacte 
\[
  0 \to \dmu_n \to \dG_m \to \dG_m \to 0\text{.}
\]
\end{theorem}
\begin{proof}
Soient $U\to X$ un morphisme étale et $a\in \dG_m(U) =\Gamma(U,\sO_U^\times)$. 
Puisque $n$ est inversible sur $U$, l'équation $T^n-a = 0$ est séparable; 
autrement dit $U'=\spec\left(\sO_U[T]/(T^n-a)\right)$ est étale su-dessus de 
$U$. Par ailleurs $U'\to U$ est surjectif et a admet une racine $n$-ième sur 
$U'$, d'où résultat. 
\end{proof}










\subsection{Fibres, images directes}\label{I:2-3}





\subsubsection{}\label{I:2-3-1}

On appelle \emph{pointe géométrique} de $X$ un morphisme $\bar x\to X$, où 
$\bar x$ est le spectre d'un corps séparablement clos $k(\bar x)$. On le notera 
abusivement $\bar x$, sous-entendent le morphisme $\bar x\to X$. Si $x$ est 
l'image de $\bar x$ dans $X$, on dit que $\bar x$ est centré en $x$. Si le 
corps $k(\bar x)$ est une extension algébrique du corps résiduel $k(x)$, on dit 
que $\bar x$ est un point géométrique \emph{algébrique} de $X$. 

On appelle \emph{voisinage étale} de $\bar x$ un diagramme commutatif 
\[\xymatrix{
  & U \ar[d] \\
  \bar x \ar[ur] \ar[r] 
  & X\text{,}
}\]
où $U\to X$ est un morphisme étale. 

Le \emph{localisé stricte} de $X$ en $\bar x$ est l'anneau 
$\sO_{X,\bar x} = \varinjlim \Gamma(U,\sO_U)$, la limite inductive étant sur les 
voisinages étales de $\bar x$. C'est un anneau local strictement hensélien dont 
le corps résiduel est la clôture séparable du corps résiduel $k(x)$ de $X$ en 
$x$ dans $k(\bar x)$. Il joue le rôle d'anneau local pour la topologie étale. 





\subsubsection{}\label{I:2-3-2}

Étant donné un faisceau $\sF$ sur $\et X$, on appelle \emph{fibre} de $\sF$ en 
$\bar x$ l'ensemble (resp. le groupe,\dots) $\sF_{\bar x}=\varinjlim \sF(U)$, 
la limite inductive étant toujours prise sur les voisinages étales de  $X$. 

Pour qu'un homomorphisme de faisceaux $\sF\to \sG$ soit un 
mono-/epi-/isomorphisme il faut et il suffit qu'il en soit ainsi des morphismes 
$\sF_{\bar x}\to \sG_{\bar x}$ induit sur les fibres et tout point géométrique 
de $X$. Si $X$ est de type fini sur un corps algébriquement clos, il suffit 
qu'il en soit ainsi en les points rationnelles de $X$. 





\subsubsection{}\label{I:2-3-3}

Si $f:X\to Y$ est un morphisme de schémas et $\sF$ un faisceau sur $\et X$, 
l'\emph{image directe} $f_* \sF$ de $\sF$ par $f$ est le faisceau sur $\et Y$ 
défini par $f_* \sF(V) = \sF(X\times_Y V)$ pour tout $V$ étale sur $Y$. Le foncteur 
$f_*:\sh(\et X)\to \sh(\et Y)$ est exact à gauche. Ses foncteurs 
dérivés à droite $\R^q f_*$ s'appellent images directes supérieures. Si 
$\bar y$ est un point géométrique de $Y$, on a 
\[
  (\R^q f_* \sF)_{\bar y} = \varinjlim \h^q(V\times_Y X,\sF)\text{,}
\]
limite inductive prise sur les voisinages étales $V$ de $\bar y$. 

Soient $\sO_{Y,\bar y}$ le localisé stricte de $Y$ en $\bar y$, 
$\widetilde Y=\spec(\sO_{Y,\bar y})$ et $\widetilde X=X\times_Y \widetilde Y$. 
On peut étendre $\sF$ à $\et{\widetilde X}$ (c'est un cas particulier de la 
notion générale d'image réciproque) de la manière suivante: soit 
$\widetilde U$ un schéma étale sur $\widetilde X$, alors il existe un 
voisinage étale $V$ de $\bar y$ et un schéma étale $U$ sur $X\times_Y V$ tel 
que $\widetilde U=U\times_V \widetilde Y$; on posera 
\[
  \sF(\widetilde U)=\varinjlim \sF\left(U\times_V V'\right)\text{,}
\]
le limite inductive étant prise sur les voisinages étales $V'$ de $\bar y$ qui 
dominent $V$. Avec cette définition, on a 
\[
  \left(\R^q f_* \sF\right)_{\bar y} = \h^q(\widetilde X,\sF)\text{.}
\]

Le foncteur $f_*$ a un adjoint à gauche $f^*$, le foncteur ``image 
réciproque.'' Si $\bar x$ est un point géométrique de $X$ et $f(\bar x)$ son 
image dans $Y$, on a $(f^* \sF)_{\bar x} = \sF_{f(\bar x)}$. Cette formule montre 
que $f^*$ est un foncteur exact. Le foncteur $f_*$ transforme donc faisceau 
injectif en faisceau injectif, et la suite spectrale du foncteur compose 
$\Gamma\circ f_*$ (resp. $g_* f_*$) fournit la 





\begin{theorem}[Suite spectrale de Leray]\label{I:2-3-4}
Soient $\sF$ un faisceau abélien sur $\et X$ et $f:X\to Y$ un morphisme de 
schémas (resp. des morphismes de schémas $X\xrightarrow f Y \xrightarrow g Z$). 
On a une suite spectrale 
\begin{align*}
  E_2^{pq} &= \h^p(Y,\R^q f_* \sF) \Rightarrow \h^{p+q}(X,\sF) \\
  \text{(resp.}\qquad E_2^{pq} &= \R^p g_* \R^q f_* \sF \Rightarrow \R^{p+q}(gf)_* \sF\text{).}
\end{align*}
\end{theorem}





\begin{corollary}\label{I:2-3-5}
Si $\R^q f_* \sF=0$ pour tout $q>0$, on a $\h^p(Y,f_* \sF)=\h^p(X,\sF)$ (resp. 
$\R^p g_*(\sF_* \sF) = \R^p(g f)_* \sF$) pour tout $p\geqslant 0$. 
\end{corollary}

Cela s'applique en particulier dans le cas suivant:





\begin{proposition}\label{I:2-3-6}
Soit $f:X\to Y$ un morphisme fini (voire, par passage à la limite, un 
morphisme entier) et $\sF$ un faisceau abélien sur $X$. Alors $\R^q f_* \sF=0$, 
pour tout $q>0$.
\end{proposition}

En effet soient $\bar y$ un pointe géométrique de $Y$, $\widetilde Y$ le 
spectre du localisé strict de $Y$ en $y$ et 
$\widetilde X=X\times_Y \widetilde Y$; d'après ce qui précède, il suffit de 
montrer que $\h^q(\widetilde X,\sF) = 0$ pour tout $q>0$. Or $\widetilde X$ est le 
spectre d'un produit d'anneaux locaux strictement henséliens (cf. 
\cite[I]{ra70}), le foncteur $\Gamma(\widetilde X,-)$ est exact car tout 
$\widetilde X$-schéma étale et surjectif admet une section, d'où l'assertion. 










\subsection{Cohomologie galoisienne}\label{I:2-4}

Pour $X=\spec(K)$ le spectre d'un corps, nous allons voir que la cohomologie 
étale s'identifie à la cohomologie galoisienne. 





\subsubsection{}\label{I:2-4-1}

Commençons par une analogie topologique. Si $K$ est le corps des fonctions 
d'une variété algébrique affine intègre $Y=\spec(A)$ sur $\dC$, on a 
$K=\varinjlim_{f\in A} A[1/f]$. 

Autrement dit $X=\varprojlim U$, $U$ parcourant l'ensemble des ouverts de $Y$. 
On sait qu'il existe des ouverts de Zariski arbitrairement petits qui pour la 
topologie classique sont des $K(\pi,1)$. On ne sera donc pas surpris si l'on 
considère $\spec(K)$ lui-même comme un $K(\pi,1)$, $\pi$ étant le groupe 
fondamental (au sens algébrique) de $X$, autrement dit le groupe de Galois de 
$\bar K/K$, où $\bar K$ est clôture séparable de $K$. 





\subsubsection{}\label{I:2-4-2}

Plus précisément soient $K$ un corps, $\bar K$ un clôture séparable de $K$ et 
$G = \gal(\bar K/K)$ le groupe de Galois topologique. A toute $K$-algèbre finie 
étale $A$ (produit fini d'extensions séparables de $K$), associons l'ensemble 
fini $\hom_K(A,\bar K)$. Le groupe de Galois $G$ opère sur cet ensemble à 
travers un quotient discret (donc fini). Si $A=K[T]/(F)$, il s'identifie à 
l'ensemble des racines dans $\bar K$ du polynôme $F$. La théorie de Galois, 
sous la forme que lui a donnée Grothendieck, dit que:





\begin{proposition}\label{I:2-4-3}
Le foncteur 
\[
  \left\{\begin{array}{c}
          \textnormal{$K$-algèbres} \\ 
          \textnormal{finies étales}
        \end{array}\right\}
  \to  
  \left\{\begin{array}{c}
          \textnormal{ensembles finis sur lesquels} \\
          \textnormal{$G$ opère continûment}
        \end{array}\right\}
\]
qui à une algèbre étale $A$ associe $\hom_K(A,\bar K)$ est une 
anti-équivalence de catégories.
\end{proposition}

On en déduit une description analogue des faisceaux pour la topologie étale sur 
$\spec(K)$:





\begin{proposition}\label{I:2-4-4}
Le foncteur 
\[
  \left\{\begin{array}{c}
          \textnormal{Faisceaux étales} \\ 
          \textnormal{sur $\spec(K)$}
        \end{array}\right\}
  \to 
  \left\{\begin{array}{c}
          \textnormal{ensembles sur lesquels} \\
          \textnormal{$G$ opère continûment}
        \end{array}\right\}
\]
qui à un 
faisceau $\sF$ associe sa fibre $\sF_{\bar K}$ au point géométrique 
$\spec(\bar K)$ est une équivalence de catégories.
\end{proposition}

On dit que $G$ opère continûment sur un ensemble $E$ si le fixateur de tout 
élément de $E$ est un sous-groupe ouvert de $G$. Le foncteur en sens 
inverse est décrit de la manière évidente: soient $A$ une 
$K$-algèbre finie étale, $U=\spec(A)$ et $U(\bar K)=\hom_K(A,\bar K)$ le 
$G$-ensemble correspondant à; alors on a 
$\sF(U)= \hom_{G\text{-ens}}(U(\bar K),\sF_{\bar K})$. 

En particulier, si $X=\spec(K)$, on a $\sF(X) = \sF_{\bar K}^G$. Si l'on se 
restreint aux faisceaux abéliens, on obtient en passant aux foncteurs 
dérivés des isomorphismes canoniques 
\[
  \h^q(\et X,\sF) = \h^q(G,\sF_{\bar K})
\]





\subsubsection{Exemples}\label{I:2-4-5}

\begin{enumerate}[\indent a)]
  \item Au faisceau constant $\dZ/n$ correspond $\dZ/n$ avec action triviale de 
    $G$. 
  \item Au faisceau des racines $n$-ièmes de l'unité $\dmu_n$ correspond 
    le groupe $\dmu_n(\bar K)$ des racines $n$-ièmes de l'unité dans 
    $\bar K$, avec l'action naturelle de $G$.
  \item Au faisceau $\dG_m$ correspond le groupe $\bar K^\times$ avec l'action 
    naturelle de $G$.
\end{enumerate}




















\section{Cohomologie des courbes}\label{I:3}

Dans le cas des espaces topologiques, des dévissages utilisant la formule de 
K\"unneth et des décompositions simpliciales permettent de se ramener pour 
calculer la cohomologie à l'intervalle $I=[0,1]$ pour lequel on a 
$\h^0(I,\dZ)=\dZ$ et $\h^q(I,\dZ)=0$ pour $q>0$. 

Dans notre cas, les dévissages aboutiront à des objets plus compliqués, 
à savoir les courbes sur un corps algébriquement clos; nous allons calculer 
leur cohomologie dans ce chapitre. La situation est plus complexe que dans le 
cas topologique car les groupes de cohomologie sont nuis pour $q>2$ seulement. 
L'ingrédient essentiel des calculs est la nullité du groupe de Brauer du 
corps des fonctions d'une telle courbe (théorème de Tsen, \ref{I:3-2}). 










\subsection{Le groupe de Brauer}\label{I:3-1}

Rappelons-en tout d'abord la définition classique:





\begin{definition}\label{I:3-1-1}
Soit $K$ un corps et $A$ une $K$-algèbre de dimension finie. On dit que $A$ 
est une algèbre simple centrale sur $K$ si les conditions équivalentes 
suivantes sont vérifiées:
\begin{enumerate}[\indent a)]
  \item $A$ n'a pas d'idéal bilatère non trivial et son centre est $K$. 
  \item Il existe une extension galoisienne finie $K'/K$ telle que 
    $A_{K'} = A\otimes_K K'$ soit isomorphe à une algèbre de matrices 
    carrées sur $K'$.
  \item $A$ est $K$-isomorphe à algèbre de matrices carrées sur un corps 
    gauche de centre $K$.
\end{enumerate}
\end{definition}

Deux telles algèbres sont dites équivalentes si les corps gauches qui 
leur sont associés par c) sont $K$-isomorphes. Si ces algèbres out 
même dimension, cela revient à dire qu'elles sont $K$-isomorphes. Le 
produit tensoriel définit par passage au quotient une structure de groupe 
abélien sur l'ensemble des classes d'équivalence. C'est ce groupe que l'on 
appelle classiquement \emph{le groupe de Brauer} de $K$ et que l'on note 
$\br(K)$. 





\subsubsection{}\label{I:3-1-2}

On notera $\br(n,K)$ l'ensemble des classes de $K$-isomorphisme de 
$K$-algèbres $A$ telles qu'il existe une extension galoisienne finie $K'$ de 
$K$ pour laquelle $A_{K'}$ est isomorphe à l'algèbre $M_n(K')$ des matrices 
carrées $n\times n$ sur $K'$. Par définition $\br(K)$ est réunion des 
sous-ensembles $\br(n,K)$ pour $n\in\dN$. Soient $\bar K$ une clôture 
algébrique de $K$ et $G=\gal(\bar K/K)$. L'ensemble $\br(n,K)$ est 
l'ensemble des ``formes'' de $M_n(\bar K)$, il est donc canoniquement 
isomorphe à $\h^1\left(G,\aut(M_n(\bar K))\right)$. 

On sait que tout automorphisme de $M_n(\bar K)$ est intérieur. Par 
conséquent le groupe $\aut(M_n(\bar K))$ s'identifie au groupe linéare 
projectif $\pgl(n,\bar K)$ et on a une bijection canonique:
\[
  \theta_n : \br(n,K) \iso \h^1\left(G,\pgl(n,\bar K)\right)\text{.}
\]

D'autre part la suite exacte:
\begin{equation*}\tag{3.1.2.1}\label{I:eq:3-1-2-1}
  1 \to \bar K^\times \to \gl(n,\bar K) \to \pgl(n,\bar K) \to 1\text{,}
\end{equation*}
permet de définir un opérateur cobord:
\[
  \Delta_n : \h^1\left(G,\pgl(n,\bar K)\right) \to \h^2(G,\bar K^\times)\text{.}
\]

En composant $\theta_n$ et $\Delta_n$, on obtient une application:
\[
  \delta_n : \br(n,K) \to \h^2(G,\bar K^\times)\text{.}
\]
On vérifie facilement que les applications $\delta_n$ sont compatibles entre 
elles et définissent un homomorphisme de groupes:
\[
  \delta : \br(K) \to \h^2(G,\bar K^\times)\text{.}
\]





\begin{proposition}\label{I:3-1-3}
L'homomorphisme $\delta:\br(K)\to \h^2(G,\bar K^\times)$ est bijectif. 
\end{proposition}

Cela résulte des deux lemmes suivants:





\begin{lemma}\label{I:3-1-4}
L'application 
$\Delta_n:\h^1\left(G,\pgl(n,\bar K)\right)\to \h^2(G,\bar K^\times)$ est 
injective.
\end{lemma}

D'après \cite{se94}, cor. à la prop. I-44, il suffit de vérifier que chaque 
fois qu'on tord la suite exacts (\ref{I:eq:3-1-2-1}) par un élément de 
$\h^1\left(G,\pgl(n,\bar K)\right)$, le $\h^1$ du groupe médian est trivial. 
Ce groupe médian est le groupe des $\bar K$-points du groupe multiplicatif 
d'une algèbre centrale simple $A$ de rang $n^2$ sur $K$. Pour prouver que 
$\h^1(G,A_{\bar K}^\times) = 0$, n interprète $A^\times$ comme le groupe 
des automorphismes du $A$-module libre $L$ de rang $1$, et $\h^1$ comme 
l'ensemble des ``formes'' de $L$ -- des $A$-modules de rang $n^2$ sur $K$, 
automatiquement libres. 





\begin{lemma}\label{I:3-1-5}
Soient $\alpha\in \h^2(G,\bar K^\times)$, $K'$ un extension finie de $K$ 
contenue dans $\bar K$, $n=[K':K]$, et $G'=\gal(\bar K/K')$. Si l'image de 
$\alpha$ dans $\h^2(G',\bar K^\times)$ est nulle, alors, $\alpha$ appartient 
à l'image de $\Delta_n$. 
\end{lemma}

Remarquons tout d'abord qu'on a: 
\[
  \h^2(G',\bar K^\times) \simeq \h^2\left(G,(\bar K\otimes_K K')^\times\right)\text{.}
\]
(D'un point de vue géométrique si l'on note $x=\spec(K)$, $x'=\spec(K')$ et 
$\pi:x'\to x$ le morphisme canonique, on a $\R^q \pi_*(\dG_{m,x'}) = 0$ pour 
$q>0$ et par suite $\h^q(x',\dG_{m,X'})\simeq \h^q(x,\pi_*\dG_{m,X'})$ pour 
$q\geqslant 0$). 

Par ailleurs la choix d'une base de $K'$ en tant qu'espace vectoriel sur $K$ 
permet de définir un homomorphisme 
\[
  (\bar K\otimes_K K')^\times \to \gl(n,\bar K)
\]
qui, à un élément $x$, fait correspondre l'endomorphisme de 
multiplication par $x$ de $\bar K\otimes_K K'$. On a alors un diagramme 
commutatif à lignes exactes: 
\[\xymatrix{
  1 \ar[r] 
    & \bar K^\times \ar[r] \ar@{=}[d]
    & (\bar K\otimes_K K')^\times \ar[r] \ar[d] 
    & (\bar K\otimes_K K')^\times / \bar K^\times \ar[r] \ar[d] 
    & 1 \\
  1 \ar[r] 
    & \bar K^\times \ar[r] 
    & \gl(n,\bar K) \ar[r] 
    & \pgl(n,\bar K) \ar[r] 
    & 1
}\]
Le lemme résulte du diagramme commutatif que l'on en déduit en passant 
à la cohomologie:
\[\xymatrix{
  \h^1\left(G,(\bar K\otimes_K K')^\times/\bar K^\times\right) \ar[r] \ar[d] 
    & \h^2(G,\bar K^\times) \ar[r] \ar@{=}[d]
    & \h^2\left(G,(\bar K\otimes_K K')^\times\right) \\
  \h^1\left(G,\pgl(n,\bar K)\right) \ar[r]^-{\Delta_n} 
    & \h^2(G,\bar K^\times) \text{.}
}\]





\begin{proposition}\label{I:3-1-6}
Soient $K$ un corps, $\bar K$ une clôture algébrique de $K$ et 
$G=\gal(\bar K/K)$. Supposons que, pour tout extension finie $K'$ de $K$, on 
ait $\br(K')=0$. Alors on a:
\begin{enumerate}[\indent i)]
  \item $\h^q(G,\bar K^\times) = 0$ pour tout $q>0$.
  \item $\h^q(G,\sF) = 0$ pour tout $G$-module de torsion $\sF$ et pour tout 
    $q\geqslant 2$.
\end{enumerate}
\end{proposition}

(Pour la démonstration, cf. \cite{se94}).










\subsection{Le théorème de Tsen}\label{I:3-2}





\begin{definition}\label{I:3-2-1}
On dit qu'un corps $K$ est $C_1$ si tout polynôme homogène non constant 
$f(x_1,\dotsc,x_n)$ de degré $d<n$ a un zéro non trivial.
\end{definition}





\begin{proposition}\label{I:3-2-2}
Si un corps $K$ est $C_1$, on a $\br(K)=0$.
\end{proposition}

Il s'agit de montrer que tout corps gauche $D$ de centre $K$ et fini sur $K$ 
est égale à $K$. Soient $r^2$ le degré de $D$ sur $K$ et 
$\operatorname{Nrd}:D\to K$ la norme réduite. 

(Localement pour la topologie étale sur $K$, $D$ est isomorphe -- non 
canoniquement -- à une algèbre de matrices $M_r$ et la norme réduite 
coïncide avec l'application déterminant. Celle-ci est bien définie, 
indépendamment de l'isomorphisme choisi entre $D$ et $M_r$ car tout 
automorphisme de $M_r$ est intérieur et deux matrices semblables ont 
même déterminant. Cette application définie localement pour la 
topologie étale se descende, à cause de son unicité locale, en une 
application $\operatorname{Nrd}:D\to K$). 

Le seul zéro de $\operatorname{Nrd}$ est l'élément nul de $D$, car, si 
$x\ne 0$, on a $\operatorname{Nrd}(x)\cdot \operatorname{Nrd}(x^{-1}) = 1$. 
D'autre part, si $\{e_1,\dotsc,e_{r^2}\}$ est une base de $D$ sur $K$ et si 
$x=\sum x_i e_i$, la fonction $\operatorname{Nrd}(x)$ s'écrit comme un polynôme 
homogène $\operatorname{Nrd}(x_1,\dotsc,x_{r^2})$ de degré $r$ (c'est 
clair localement pour la topologie étale). Puisque $K$ est $C_1$, on a 
$r^2\leqslant r$, c'est-à-dire $r=1$ et $D=K$. 





\begin{theorem}[Tsen]\label{I:3-2-3}
Soient $k$ un corps algébriquement clos et $K$ une extension de degré de 
transcendance $1$ de $k$. Alors $K$ est $C_1$.
\end{theorem}

Supposons tout d'abord que $K=k(X)$. Soit 
\[
  f(T) = \sum a_{i_1,\dotsc i_n} T_1^{i_1} \dotsm T_n^{i_n}
\]
un polynôme homogène de degré $d<n$ à coefficients dans $k(X)$. Quitte 
à multiplier les coefficients par un dénominateur commun on peut supposer 
qu'ils sont dans $k[X]$. Soit alors $\delta=\sup\deg(a_{i_1\dotsc i_n})$. On 
cherche un zéro non trivial dans $k[X]$ par la méthode des coefficients 
indéterminés en écrivant chaque $T_i$ ($i=1,\dotsc,n$) comme un 
polynôme de degré $N$ en $X$. Alors l'équation $f(T)=0$ devient un système 
d'équations homogènes en les $n\times (N+1)$ coefficients des polynômes 
$T_i(X)$ exprimant la nullité des coefficients du polynôme en $X$ obtenu en 
remplaçant $T_i$ par $T_i(X)$. Ce polynôme est de degré $\delta+N D$ au plus, 
il y a donc $\delta+N d+1$ équations en $n\times (N+1)$ variables. Comme $k$ 
est algébriquement clos ce système a une solution non triviale si 
$n(N+1)>N d+\delta+1$, ce qui sera le cas pour $N$ assez grand si $d<n$. 

Il est clair que, pour démontrer le théorème dans la cas général, il 
suffit de le démontrer lorsque $K$ est une extension finie d'une extension 
transcendent pure $k(X)$ de $k$. Soit $f(T)=f(T_1,\dotsc,T_n)$ un 
polynôme homogène de degré $d<n$ à coefficients dans $K$. Soient 
$a=[K:k(X)]$ et $e_1,\dotsc,e_s$ une bas de $K$ sur $k(X)$. Introduisons de 
nouvelles variables $U_{i j}$, en nombre $s n$, telles que 
$T_i=\sum U_{i j}e_j$. Pour que le polynôme $f(T)$ ait un zéro 
non trivial dans $K$, il suffit que le polynôme 
$g(X_{i j}) = N_{K/k}(f( T))$ ait un zéro non trivial dans $k(X)$. 
Or $g$ est un polynôme de degré $s d$ en $s n$ variables, d'où le 
résultat. 





\begin{corollary}\label{I:3-2-4}
  Soient $k$ un corps algébriquement clos et $K$ une extension de degré 
de transcendance $1$ de $k$. Alors les groupes de cohomologie étale 
$\h^q(\spec(K),\dG_m)$ sont nuls pour tout $q>0$. 
\end{corollary}










\subsection{Cohomologie des courbes lisses}

Dorénavant, et sauf mention expresse du contraire, les groupes de 
cohomologie considérés sont les groupes des cohomologie étale. 





\begin{proposition}\label{I:3-3-1}
Soient $k$ un corps algébriquement clos et $X$ une courbe projective non 
singulière connexe sur $k$. Alors on a:
\begin{align*}
  \h^0(X,\dG_m) &= k^\times \text{,} \\
  \h^1(X,\dG_m) &= \pic(X) \text{,} \\
  \h^q(X,\dG_m) &= 0 \text{ pour $q\geqslant 2$.}
\end{align*}
\end{proposition}

Soient $\eta$ le point générique de $X$, $j:\eta\to X$ le morphisme 
canonique et $\dG_{m,X}$ le groupe multiplicatif du corps des fractions 
$K(X)$. Pour tout pointe fermé $x$ de $X$, soient $i_x:x\to X$ l'immersion 
canonique et $\dZ_x$ le faisceau constant de valeur $\dZ$ sur $x$. Ainsi 
$j_*\dG_{m,\eta}$ est le faisceau des fonctions méromorphes non nulles sur 
$X$ et $\oplus_{x\in X} i_{x_I}\dZ_x$ le faisceau des diviseurs, on a donc une 
suite exacte de faisceaux:
\begin{equation*}\tag{3.3.1.1}\label{I:eq:3-3-1-1}
\xymatrix{
  0 \ar[r] 
    & \dG_m \ar[r]
    & j_* \dG_{m,\eta} \ar[r]^-{\dv} \ar[r] 
    & \displaystyle\bigoplus_{x\in X} i_{x*} \dZ_x \ar[r] 
    & 0\text{.}
}
\end{equation*}





\begin{lemma}\label{I:3-3-2}
On a $\R^q j_* \dG_{m,\eta} = 0$ pour tout $q>0$.
\end{lemma}

Il suffit de montrer que la fibre de ce faisceau en tout pointe fermé $x$ 
de $X$ est nulle. Si $\tilde\sO_{X,x}$ est l'hensélisé de $X$ en $x$ et $K$ 
le corps des fractions de $\tilde\sO_{X,x}$, on a 
\[
  \spec(K) = \eta\times_X \spec(\tilde\sO_{X,x})\text{,}
\]
donc $(\R^qj_*\dG_{m,\eta})_x = \h^q(\spec(K),\dG_m)$. 

Or $K$ est une extension algébrique de $k(X)$, donc une extension de degré 
de transcendante $1$ de $k$: le lemme résulte de (\ref{I:3-2-4}). 





\begin{lemma}\label{I:3-3-3}
On a $\h^q(X, j_*\dG_{m,\eta}) = 0$ pour tout $q>0$.
\end{lemma}

En effet de (\ref{I:3-3-2}) et de la suite spectrale de Leray pour $j$, on 
déduit:
\[
  \h^q(X, j_* \dG_{m,\eta}) = \h^q(\eta,\dG_{m,\eta})
\]
pour tout $q\geqslant 0$ et le deuxième membre est nul pour $q>0$ d'après 
(\ref{I:3-2-4}). 





\begin{lemma}\label{I:3-3-4}
On a $\h^q\left(X,\bigoplus_{x\in X} i_{x*} \dZ_x\right) = 0$ pour tout $q>0$. 
\end{lemma}

En effet pour tout point fermé de $X$, on a $\R^q i_{x*}\dZ_x 0$ pour $q>0$, 
car $i_x$ est un morphisme fini (\ref{I:2-3-6}), et 
\[
  \h^q(X,i_{x*}\dZ_x) = \h^q(x,\dZ_x)\text{.}
\]
Le deuxième membre est nul pour tout $q>0$, car $x$ est la spectre d'un 
corps algébriquement clos (On voit que le lemme est vrai plus 
généralement pour tout faisceau ``gratte-ciel'' sur $X$). 

On déduit des lemmes précédents et de la suite exacte (\ref{I:eq:3-3-1-1}) 
les égalités:
\[
  \h^q(X,\dG_m) = 0 \text{ pour $q\geqslant 2$,}
\]
et une suite exacte de cohomologie en bas degré:
\[
  1 \to \h^0(X,\dG_m) \to \h^0(X, j_*\dG_{m,\eta}) \to \h^0\left(X,\bigoplus_{x\in X} i_{x*}\dZ_x\right) \to \h^1(X,\dG_m) \to 1
\]
qui n'est autre que la suite exacte:
\[
  1 \to k^\times \to k(X)^\times\to \Div(X) \to \pic(X) \to 1\text{.}
\]
De la proposition \ref{I:3-3-1} on déduit que les groupes de cohomologie de 
$X$ à valeur dans $\dZ/n$, $n$ premier à la caractéristique de $k$, ont 
une valeur raisonnable:





\begin{corollary}\label{I:3-3-5}
Si $X$ est de genre $g$ et si $n$ est inversible dans $k$, les 
$\h^q(X,\dZ/n)$ sont nuls pour $q>2$, et libres sur $\dZ/n$ de rang $1,2 g,1$ 
pour $q=0,1,2$. Remplaçant $\dZ/n$ par le groupe isomorphe $\dmu_n$, on a des 
isomorphismes canoniques 
\begin{align*}
  \h^0(X,\dmu_n) &= \dmu_n \\
  \h^1(X,\dmu_n) &= \pic^0(X)_n \\
  \h^2(X,\dmu_n) &= \dZ/n \text{.}
\end{align*}
\end{corollary}

Comme le corps $k$ est algébriquement clos, $\dZ/n$ est isomorphe (non 
canoniquement) à $\mu_n$. De la suite exacte de Kummer:
\[
  0 \to \mu_n \to \dG_m \to \dG_m \to 0\text{,}
\]
et de la proposition \ref{I:3-3-1}, on déduit les égalité:
\[
  \h^q(X,\dZ/n) = 0 \text{ pour $q>2$,}
\]
et, en bas degré, des suites exactes:
\[\xymatrix{
  0 \ar[r] 
    & \h^0(X,\dmu_n) \ar[r] 
    & k^\times \ar[r]^-n 
    & k^\times \ar[r] 
    & 0
}\]
\[\xymatrix{
  0 \ar[r] 
    & \h^1(X,\dmu_n) \ar[r] 
    & \pic(X) \ar[r]^-n 
    & \h^2(X,\dmu_n) \ar[r] 
    & 0\text{.}
}\]
De plus on a une suite exacte:
\[\xymatrix{
  0 \ar[r] 
    & \pic^0(X) \ar[r] 
    & \pic(X) \ar[r]^-{\deg}
    & \dZ \ar[r] 
    & 0 \text{,}
}\]
et $\pic^0(X)$ s'identifie au groupe des points rationnels sur $k$ d'une 
variété abélienne de dimension $g$, la jacobienne de $X$. Dans un tel 
groupe, la multiplication par $n$ est surjective et son noyau est un 
$\dZ/n\dZ$-module libre de rang $2 g$ (car $n$ est inversible dans $k$); d'où 
le corollaire. 

De dévissage astucieux, utilisant la ``méthode de la trace,'' permet 
d'obtenir en corollaire la 





\begin{proposition}[{\cite[IX 5.7]{sga4}}]\label{I:3-3-6}
Soient $k$ un corps algébriquement clos, $X$ une courbe algébrique sur $k$ 
et $\sF$ un faisceau de torsion sur $X$. Alors:
\begin{enumerate}[\indent i)]
  \item On a $\h^q(X,\sF) = 0$ pour $q>2$.
  \item Si $X$ est affine, on a même $\h^q(X,\sF) = 0$ pour $q>1$.
\end{enumerate}
\end{proposition}

Pour la démonstration, ainsi que pour l'exposé de la ``méthode de la 
trace,'' nous renvoyons à \cite[IX 5]{sga4}.










\subsection{Dévissages}\label{I:3-4}

Pour calculer la cohomologie des variétés de dimension $>1$ on emploie des 
fibrations par des courbes, ce qui permet de se ramener à étudier les 
morphismes dont les fibres sont de dimension $\leqslant 1$. Ce principe 
possède plusieurs variantes, indiquons-en quelques-unes.





\subsubsection{}\label{I:3-4-1}

Soient $A$ une $k$-algèbre de type fini et $a_1,\dotsc,a_n$ des générateurs de 
$A$. Si l'on pose $X_0=\spec(k)$, $X_i=\spec(k[a_1,\dotsc,a_i])$, 
$X_n=\spec(A)$, les inclusions canoniques 
$k[a_1,\dotsc,a_i]\to k[a_1,\dotsc,a_i,a_{i+1}]$ définissent des morphismes 
$X_n \to X_{n-1} \to \cdots \to X_1 \to X_0$ dont les fibres sont de dimension 
$\leqslant 1$. 





\subsubsection{}\label{I:3-4-2}

Dans le cas d'un morphisme lisse, on peut être plus précis. On appelle 
\emph{fibration élémentaire} un morphisme de schémas $f:X\to S$ qui peut 
être plongé dans un diagramme commutatif
\[\xymatrix{
  X \ar@{^{(}->}[r]^-j \ar[dr]_-f
    & \bar X \ar[d]^-{\bar f}
    & \ar@{_{(}->}[l]_-i \ar[dl]^-g Y \\
  & S
}\]
satisfaisant aux conditions suivantes:
\begin{enumerate}[\indent i)]
  \item $j$ est une immersion ouverte dense dans chaque fibre et 
    $X=\bar X \setminus Y$. 
  \item $\bar f$ est lisse et projectif, à fibres géométriques 
    irréductibles et de dimension $1$.
  \item $g$ est un revêtement étale et aucune fibre de $g$ n'est vide.
\end{enumerate}

On appelle \emph{bon voisinage} relatif à $S$ un $S$-schéma $X$ tel qu'il 
existe $S$-schémas $X=X_n,\dotsc,X_0=S$ et des fibrations élémentaires 
$f_i:X_i\to X_{i-1}$, $i=1,\dotsc,n$. On peut montrer \cite[XI 3.3]{sga4} que si 
$X$ est un schéma lisse sur un corps algébriquement clos $k$ tout point 
rationnel de $X$ possède un voisinage ouvert qui est un bon voisinage 
(relatif à $\spec(K)$). 





\subsubsection{}\label{I:3-4-3}

On peut dévisser un morphisme propre $f:X\to S$ de la façon suivante. 
D'après le lemme de Chow, il existe un diagramme commutatif
\[\xymatrix{
  X \ar[dr]_-f 
    & \ar[l]_-\pi \ar[d]^-{\bar f} \bar X \\
  & S
}\]
où $\pi$ et $\bar f$ sont des morphismes projectifs, $\pi$ étant de plus un 
isomorphisme au-dessus d'un ouvert dense de $X$. Localement sur $S$, $\bar X$ 
est un sous-schéma fermé d'un espace projectif type $\dP_S^n$. 

On dévisse ce dernier en considérant la projection 
$\varphi:\dP_S^n\to\dP_S^1$ qui envoie le point de coordonnées homogènes 
$(x_0,x_1,\dotsc,x_n)$ sur $(x_0,x_1)$. C'est une application rationnelle 
définie en dehors du fermé $Y\simeq \dP_S^{n-2}$ de $\dP_S^n$ d'équations 
homogènes $x_0=x_1=0$. Soit $u:P\to \dP_S^n$ l'éclatement à centre 
$Y$; les fibres de $u$ soit de dimension $\leqslant 1$. De plus il existe un 
morphisme naturel $v:P\to \dP_S^1$ qui prolonge l'application rationnelle 
$\varphi$ et $v$ fait de $P$ un $\dP_S^1$-schéma localement isomorphe à 
l'espace projectif type $\dP^{n-1}$ que l'on peut à son tour projeter sur un 
$\dP^1$, etc. 





\subsubsection{}\label{I:3-4-4}

On peut balayer une variété projective et lisse $X$ par un \emph{pinceau de 
Lefschetz}. L'éclaté $\widetilde X$ de l'intersection de l'axe du pinceau 
avec $X$ se projette sur $\dP^1$ et les fibres de cette projection sont les 
sections hyperplans de $X$ par les hyperplans du pinceau.




















\section{Théorème de changement de base pour un morphisme propre}\label{I:4}










\subsection{Introduction}\label{I:4-1}

Ce chapitre est consacré à la démonstration et aux applications du 





\begin{theorem}\label{I:4-1-1}
Soient $f:X\to S$ un morphisme propre de schémas et $\sF$ un faisceau abélien 
de torsion sur $X$. Alors, quel que soit $q\geqslant 0$, la fibre de 
$\R^q f_* \sF$ en un point géométrique $s$ de $S$ est isomorphe à la 
cohomologie $\h^q(X_s,\sF)$ de la fibre $X_s=X\otimes_S \spec k(s)$ de $f$ en 
$s$. 
\end{theorem}

Pour $f:X\to S$ une application continue propre et séparé (séparée 
signifie que la diagonale de $X\times_S X$ est fermée) entre espaces 
topologiques, et $\sF$ un faisceau abélien sur $X$, le résultat analogue est 
bien connu, et élémentaire: comme $f$ est fermée, les $f^{-1}(V)$ pour 
$V$ voisinage de $s$ forment un système fondamental de voisinages de $X_s$, et 
on vérifie que $\h^\bullet(X_s,\sF) = \varinjlim_U \h^\bullet(U,\sF)$, pour $U$ parcourant les 
voisinages de $X_s$. En pratique, $X_s$ a même un système fondamental 
$\cU$ de voisinages $U$ dont il est rétracte par déformation et, pour $\sF$ 
constant, on a donc $\h^\bullet(X_s,\sF)=\h^\bullet(U,\sF)$. En termes imagés: la fibre 
spéciale avale la fibre générale. 

Dans le cas des schémas la démonstration est plus délicate et il est 
indispensable de supposer que $\sF$ est de torsion (\cite[XII.2]{sga4}). Compte 
tenu de la description des fibres de $\R^q f_* \sF$ (\ref{I:2-3-3}), le théorème 
(\ref{I:4-1-1}) est essentiellement équivalent au 

 
  
   
    
\begin{theorem}\label{I:4-1-2}
Soient $A$ un anneau local strictement hensélien et $S=\spec(A)$. Soient 
$f:X\to S$ un morphisme propre et $X_0$ la fibre fermée de $f$. Alors, pour 
tout faisceau abélien de torsion $\sF$ sur $X$ et pour tout $q\geqslant 0$, on 
a $\h^q(X,\sF) \iso \h^q(X_0,\sF)$. 
\end{theorem} 

Par passage à la limite on voit qu'il suffit de démontrer le théorème 
lorsque $A$ est l'hensélisé strict d'une $\dZ$-algèbre  de type fini en 
un idéal premier. On traite d'abord le cas $q=0$ ou $1$ et $\sF=\dZ/n$ 
(\ref{I:4-2}). Un argument basé sur la notion de faisceau constructible 
(\ref{I:4-3}) montre d'ailleurs qu'il suffit de considérer le cas où $\sF$ est 
constant. D'autre part le dévissage (\ref{I:3-4-3}) permet de supposer que 
$X_0$ est une courbe; dans ce cas il ne reste plus qu'à démontrer le 
théorème pour $q=2$ (\ref{I:4-4}). 

Entre autres applications (\ref{I:4-6}), le théorème permet de définir la notion 
de cohomologie à support propre (\ref{I:4-5}). 










\subsection{Démonstration pour \texorpdfstring{$q=0$}{q=1} ou \texorpdfstring{$1$}{1} et \texorpdfstring{$\sF=\dZ/n$}{\sF=Z/n}}\label{I:4-2} 

Le résultat pour $q=0$ et $\sF$ constant est équivalent à la proposition 
suivante (théorème de connexion de Zariski):





\begin{proposition}\label{I:4-2-1}
Soient $A$ un anneau local hensélien noethérien et $S=\spec(A)$. 
Soient $f:X\to S$ un morphisme propre et $X_0$ la fibre fermée de $f$. Alors 
les ensembles de composantes connexes $\pi_0(X)$ et $\pi_0(X_0)$ sont en 
bijection.
\end{proposition}

Il revient au même de montrer que les ensembles de parties à la fois 
ouvertes et fermées $\operatorname{Of}(X)$ et $\operatorname{Of}(X_0)$ sont 
en bijection. On sait que l'ensemble $\operatorname{Of}(X)$ correspond 
bijectivement à l'ensemble des idempotents de $\Gamma(X,\sO_X)$, de même 
$\operatorname{Of}(X_0)$ correspond bijectivement à l'ensemble des 
idempotents de $\Gamma(X_0,\sO_{X_0})$. Il s'agit donc de montrer que 
l'application canonique 
\[
  \operatorname{Idem} \Gamma(X,\sO_X) \to \operatorname{Idem}\Gamma(X_0,\sO_{X_0})
\]
est bijective. 

On notera $\fm$ l'idéal maximal de $A$, $\Gamma(X,\sO_X)^\wedge$ le complété 
de $\Gamma(X,\sO_X)$ pour la topologie $\fm$-adique et, pour tout entier 
$n\geqslant 0$, $X_n=X\otimes_A A/\fm^{n+1}$. D'après le théorème de finitude 
pour les morphismes propres \cite[III.3.2]{ega}, $\Gamma(X,\sO_X)$ est une 
$A$-algèbre finie; comme $A$ est hensélien, il en résulte que 
l'application canonique 
\[
  \operatorname{Idem} \Gamma(X,\sO_X) \to \operatorname{Idem}\Gamma(X,\sO_X)^\wedge
\]
est bijective. En particulier l'application canonique 
\[
  \operatorname{Idem} \Gamma(X,\sO_X)^\wedge \to \varprojlim \operatorname{Idem}\Gamma(X_n,\sO_{X_n})
\]
est bijective. Mais, puisque $X_n$ et $X_0$ on même espace topologique 
sous-jacent, l'application canonique 
\[
  \operatorname{Idem}\Gamma(X_n,\sO_{X_n}) \to \operatorname{Idem} \Gamma(X_0,\sO_{X_0})
\]
est bijective pour tout $n$, ce qui achève la démonstration. 

Puisque $\h^1(X,\dZ/n)$ est en bijection avec l'ensemble des classes 
d'isomorphisme de revêtements étales galoisiens de $X$ de groupe $\dZ/n$, 
le théorème  pour $q=1$ et $\sF=\dZ/n$ résulte de la proposition suivante. 





\begin{proposition}\label{I:4-2-2}
Soient $A$ un anneau local hensélien noethérien et $S=\spec(A)$. Soient 
$f:X\to S$ un morphisme propre et $X_0$ la fibre fermée de $f$. Alors le 
foncteur de restriction 
\[
  \operatorname{Rev.et}(X) \to \operatorname{Rev.et}(X_0)
\]
est une équivalence de catégories. 
\end{proposition}

(Si $X_0$ est connexe et si l'on a choisi un point géométrique de $X_0$ 
comme point base, cela revient à dire que l'application canonique 
$\pi_1(X_0)\to \pi_1(X)$ sur les groupes fondamentaux (profinis) est 
bijective). 

La proposition (\ref{I:4-2-1}) montre que ce foncteur est pleinement fidèle. En 
effet, si $X'$ et $X''$ sont deux revêtements étales de $X$, un 
$X$-morphisme de $X'$ dans $X''$ est déterminé par son graphe qui est une 
partie ouverte et fermée de $X'\times_X X''$. 

Il s'agit donc de montrer que tout revêtement étale $X_0'$ de $X_0$ s'étend 
en un revêtement étale de $X$. On sait que les revêtements étales ne 
dépendent pas des éléments nilpotents \cite[ch.1]{sga1}, par conséquent 
$X_0'$ se relève de manière unique en un revêtement étale $X_n'$ de 
$X_n$ pour tout $n\geqslant 0$, autrement dit en un revêtement étale 
$\fX'$ du schéma formel $\fX$ complété de $X$ de long de $X_0$. D'après 
le théorème d'algébrisation des faisceaux cohérents formels de 
Grothendieck (théorème d'existence, \cite[III.5]{ega}), $\fX'$ est le 
complété formel d'un revêtement étale $\bar X'$, de 
$\bar X=X\otimes_A \hat A$. 

Par passage à la limite, il suffit de démontrer la proposition dans le cas 
où $A$ est l'hensélisé d'une $\dZ$-algèbre de type fini. On peut alors 
appliquer le théorème d'approximation d'Artin au foncteur 
$\sF:(\textnormal{$A$-algèbres}) \to (\textnormal{ensembles})$ qui, à une 
$A$-algèbre $B$, fait correspondre l'ensemble des classes d'isomorphisme de 
revêtements étales de $X\otimes_A B$. En effet ce foncteur est localement de 
présentation finie: si $B_i$ est une système inductif filtrant de 
$A$-algèbres et si $B=\varinjlim B_i$, on a $\sF(B) = \varinjlim \sF(B_i)$. 
D'après le théorème d'Artin, étant donné un élément 
$\xi\in \sF(\hat A)$, en l'occurrence la classe d'isomorphisme de $\bar X'$, il 
existe $\xi\in \sF(A)$ ayant même image que $\bar\xi$ dans $\sF(A/\fm)$. 
Autrement dit il existe un revêtement étale $X'$ de $X$ dont la restriction 
à $X_0$ est isomorphe à $X_0'$. 










\subsection{Faisceaux constructibles}\label{I:4-3}

Dans ce paragraphe, on considère un schéma \emph{noethérien} $X$ et on 
appelle faisceau sur $X$ un faisceau \emph{abélien} sur $\et X$. 





\begin{definition}\label{I:4-3-1}
On dit qu'un faisceau $\sF$ sur $X$ est \emph{localement constant constructible} 
(en abrégé l.c.c.) s'il est représenté par un revêtement étale de 
$X$. 
\end{definition}





\begin{definition}\label{I:4-3-2}
On dit qu'un faisceau $\sF$ sur $X$ est \emph{constructible} s'il vérifie les 
conditions équivalentes suivantes:
\begin{enumerate}[\indent (i)]
  \item Il existe une famille finie surjective de sous-schémas $X_i$ de $X$ 
     tels que la restriction de $\sF$ à $X_i$ soit l.c.c..
  \item Il existe une famille finie de morphismes finis $p_i:X_i'\to X$, pour 
    chaque $i$ un faisceau constant constructible ( = défini par un groupe 
    abélien fini) $C_i$ sur $X_i'$, et un monomorphisme 
    $\sF\to \prod p_{i*} C_i$. 
\end{enumerate}
\end{definition}

On vérifie facilement que la catégorie des faisceaux constructibles sur $X$ 
est une catégorie abélienne. De plus, si $u:\sF\to \sG$ est un homomorphisme de 
faisceaux et si $\sF$ est constructible, le faisceau $\im(u)$ est constructible.





\begin{lemma}\label{I:4-3-3}
Tout faisceau de torsion $\sF$ est limite inductive filtrante de faisceaux 
constructibles.
\end{lemma}

En effet, si $j:U\to X$ est un schéma étale de type fini sur $X$, un 
élément $\xi\in \sF(U)$ tel que $n\xi=0$ définit un homomorphisme de 
faisceaux $j_! \const{\dZ/n}\to \sF$ dont l'image (le lus petit sous-faisceau de 
$\sF$ dont $\xi$ soit section locale) est un sous-faisceau constructible de $\sF$. 
Il est clair que $\sF$ est limite inductive de tels sous-faisceaux.





\begin{definition}\label{I:4-3-4}
Soient $\cC$ une catégorie abélienne et $T$ un foncteur défini sur $\cC$ 
à valeurs dans la catégorie des groupes abéliens. On dira que $T$ est 
\emph{effaçable} dans $\cC$ si, pour tout objet $A$ de $\cC$ et tout 
$\alpha\in T(A)$, il existe un monomorphisme $u:A\to M$ dans $\cC$ tel que 
$T(u)\alpha = 0$. 
\end{definition}





\begin{lemma}\label{I:4-3-5}
Les foncteurs $\h^q(X,-)$ pour $q>0$ sont effaçables dans la catégorie des 
faisceaux constructibles sur $X$.
\end{lemma}

Il suffit de remarquer que, si $\sF$ est un faisceau constructible, il existe 
nécessairement un entier $n>0$ tel que $\sF$ soit un faisceau de 
$\dZ/n$-modules. Alors il existe un monomorphisme $\sF\hookrightarrow \sG$, où 
$\sG$ est un faisceau de $\dZ/n$-modules et $\h^q(X,\sG) = 0$ pour tout $q>0$.  
On peut par exemple prendre pour $\sG$ la résolution de Godement 
$\prod_{x\in X} i_{x*} \sF_{\bar x}$, où $x$ parcourt les points de $X$ et 
$i_x:\bar x\to X$ est un point géométrique centré en $X$. D'après 
(\ref{I:4-3-3}) $\sG$ est limite inductive de faisceaux constructibles, d'o`u le 
lemme, car les foncteurs $\h^q(X,-)$ commutent aux limites inductives. 





\begin{lemma}\label{I:4-3-6}
Soit $\varphi^\bullet:T^\bullet\to {T'}^\bullet$ un morphisme de foncteurs 
cohomologiques définis sur une catégorie abélienne $\cC$ et à valeurs 
dans la catégorie des groupes abéliens. Supposons que $T^q$ est effaçable 
pour $q>0$ et soit $\cE$ un sous-ensemble d'objets de $\cC$ tel que tout objet 
de $\cC$ soit contenu dans un objet appartenant à $\cE$. Alors les 
conditions suivantes sont équivalentes:
\begin{enumerate}[\indent (i)]
  \item $\varphi^q(A)$ est bijectif pour tout $q\geqslant 0$ et tout 
    $A\in\ob \cC$.
  \item $\varphi^0(M)$ est bijectif et $\varphi^q(M)$ surjectif pour tout 
    $q>0$ et tout $M\in\cE$. 
  \item $\varphi^0(A)$ est bijectif pour tout $A\in\ob \cC$ et ${T'}^q$ est 
    effaçable pour tout $q>0$. 
\end{enumerate}
\end{lemma}

La démonstration se fait par récurrence sur $q$ et ne présente pas de 
difficultés. 





\begin{proposition}\label{I:4-3-7}
Soit $X_0$ un sous-schéma de $X$. Supposons que, pour tout $n\geqslant 0$ et 
pour tout schéma $X'$ fini sur $X$, l'application canonique 
\[
  \h^q(X',\dZ/n) \to \h^q(X_0',\dZ/n)\text{,}
\]
où $X_0' = X'\times_X X_0$, est bijective pour $q=0$ et surjective pour 
$q>0$. Alors, pour tout faisceau de torsion $\sF$ sur $X$ et pour tout 
$q\geqslant 0$, l'application canonique 
\[
  \h^q(X,\sF) \to \h^q(X_0,\sF)
\]
est bijective.
\end{proposition}

Par passage à la limite, il suffit de démontrer l'assertion pour $\sF$ 
constructible. On applique le lemme (\ref{I:4-3-6}) en prenant pour $\cC$ la 
catégorie des faisceaux constructibles sur $X$, $T^q=\h^q(X,-)$, 
${T'}^q=\h^q(X_0,-)$, et $\cE$ l'ensemble des faisceaux constructibles de la 
forme $\prod p_{i*} C_i$, où $p_i:X_i'\to X$ est un morphisme fini et $C_i$ 
un faisceau constant fini sur $X_i'$. 










\subsection{Fin de la démonstration}\label{I:4-4}

Par la méthode de fibration par des courbes (\ref{I:3-4-3}), on se ramène 
à démontrer le théorème en dimension relative $\leqslant 1$. D'après 
le paragraphe précédent, il suffira de montrer que, si $S$ est le spectre 
d'un anneau local noethérien strictement hensélien, $f:X\to S$ un 
morphisme propre dont la fibre fermée $X_0$ est de dimension $\leqslant 1$ 
et $n$ un entier $\geqslant 0$, l'homomorphisme canonique 
\[
  \h^q(X,\dZ/n) \to \h^q(X_0,\dZ/n)
\]
est bijectif pour $q=0$ et surjectif pour $q>0$. 

Les cas $q=0$ et $1$ ont été vus plus haut et on a $\h^q(X_0,\dZ/n) = 0$ 
pour $q\geqslant 3$; il suffit donc de traiter le cas $q=2$. On peut 
évidemment supposer que $n$ est une puissance d'un nombre premier. Si 
$n=p^r$, où $p$ est la caractéristique du corps résiduel de $S$, la 
théorie d'Artin-Schreier montre qu'un a $\h^2(X_0,\dZ/p^r) = 0$. Si 
$n=\ell^r$, $\ell\ne p$, on déduit de la théorie de Kummer un diagramme 
commutatif 
\[\xymatrix{
  \pic(X) \ar[d] \ar[r]^-\alpha 
    & \h^2(X,\dZ/\ell^r) \ar[d] \\
  \pic(X_0) \ar[r]^-\beta 
    & \h^2(X_0,\dZ/\ell^r)
}\]
où l'application $\beta$ est surjective (On l'a vu au chapitre \ref{I:3} pour 
une courbe lisse sur un corps algébriquement clos, mais des arguments 
similaires s'appliquent à n'importe quelle courbe sur un corps 
séparablement clos). 

Pour conclure, il suffira donc de montrer:





\begin{proposition}\label{I:4-4-1}
Soient $S$ le spectre d'un anneau local noethérien hensélien et $f:X\to S$ 
un morphisme propre dont la fibre fermée $X_0$ est de dimension 
$\leqslant 1$. Alors l'application canonique de restriction 
\[
  \pic(X)\to \pic(X_0)
\]
est surjective (Il suffit d'ailleurs que le morphisme $f$ soit 
\emph{séparé} de type fini). 
\end{proposition}

Pour simplifier la démonstration, nous supposerons que $X$ est intègre, 
bien que cela ne soit pas nécessaire. Tout faisceau inversible sur $X_0$ est 
associé à un diviseur de Cartier (car $X_0$ est une courbe, donc 
quasi-projectif), il suffit donc de montrer que l'application canonique 
$\Div(X)\to\Div(X_0)$ est surjective. 

Tout diviseur sur $X_0$ est combinaison linéaire de diviseurs dont le 
support est concentré en un seul point fermé non isolé de $X_0$. Soient 
$x$ un tel point, $t_0\in\sO_{X_0,x}$ un élément régulier non inversible 
de $\sO_{X_0,x}$ et $D_0$ le diviseur concentré en $x$ d'équation locale 
$t_0$. Soit $U$ un voisinage ouvert de $x$ dans $X$ tel qu'il existe un section 
$t\in \Gamma(U,\sO_U)$ relevant $t_0$. Soit $Y$ le fermé de $U$ d'équation 
$t=0$; quitte à prendre $U$ assez petit, on peut supposer que $x$ est le 
seul point de $Y\cap X_0$. Alors $Y$ est quasi-fini au-dessus de $S$ en $x$; 
puisque $S$ est le spectre d'un anneau local hensélien, on en déduit que 
$Y=Y_1\amalg Y_2$, où $Y_1$ est fini sur $S$ et où $Y_2$ ne rencontre pas 
$X_0$. De plus, comme $X$ est séparé sur $S$, $Y_1$ est fermé dans $X$. 

Quitte à remplacer $U$ par un voisinage ouvert plus petit de $x$, on peut 
supposer que $Y=Y_1$, autrement dit que $Y$ est fermé dans $X$. On définit 
alors un diviseur $D$ sur $X$ relevant $D_0$ en posant $D|X\setminus Y=0$ et 
$D|U=\dv(t)$ ce qui a un sens car $t$ est inversible sur $U\setminus Y$. 





\subsubsection{Remarque}\label{I:4-4-2}

Dans le cas où $f$ est propre, on pourrait aussi faire une démonstration 
du même style que celle de la proposition (\ref{I:4-2-2}). En effet, comme 
$X_0$ est une courbe, il n'y a pas d'obstruction à relever un faisceau 
inversible sur $X_0$ aux voisinages infinitésimaux $X_n$ de $X_0$, donc au 
complété formel $\fX$ de $X$ le long de $X_0$. On conclut alors en 
appliquant successivement le théorème d'existence de Grothendieck et le 
théorème d'approximation d'Artin. 










\subsection{Cohomologie à support propre}\label{I:4-5}





\begin{definition}\label{I:4-5-1}
Soit $X$ un schéma séparé de type fini sur un corps $k$. D'après un 
théorème de Nagata, il existe un schéma $\bar X$ propre sur $k$ et une 
immersion ouverte $j:X\to\bar X$. Pour tout faisceau de torsion $\sF$ sur $X$ on 
note $j_! \sF$ le prolongement par $0$ de $\sF$ à $\bar X$ et on définit les 
groupes de \emph{cohomologie à support propre} 
\[
  \h_c^q(X,\sF)=\h^q(\bar X, j_! \sF) \text{.}
\]
\end{definition}

Montrons que cette définition est indépendante de la compactification 
$j:X\to\bar X$ choisie. Soient $j_1:X\to \bar X_1$ et $j_2:X\to \bar X_2$ deux 
compactifications. Alors $X$ s'envoie dans $\bar X_1\times \bar X_2$ par 
$x\mapsto (j_1(x),j_2(x))$ et l'image fermée $\bar X_3$ de $X$ par cette 
application est une compactification de $X$. On a ainsi un diagramme commutatif 
\[\xymatrix{
  & \bar X_1 \\
  X \ar[ur]^-{j_1} \ar[rr]^-{(j_1,j_2)} \ar[dr]_-{j_2} 
    & & \bar X_3 \ar[ul]_-{p_1} \ar[dl]^-{p_2} \\
  & \bar X_2
}\]
où $p_1$ et $p_2$, les restrictions des projections naturelles à 
$\bar X_3$, sont des morphismes propres. Il suffit donc de traiter le cas où 
on a un diagramme commutatif 
\[\xymatrix{
  X \ar[r]^-{j_2} \ar[dr]_-{j_1} 
    & \bar X_2 \ar[d]^-p \\
  & \bar X_1
}\]
avec $p$ un morphisme propre. 





\begin{lemma}\label{I:4-5-2}
On a $p_*(j_{2!} \sF) = j_{1!} \sF$ et $\R^q p_*(j_{2!} \sF) = 0$, pour $q>0$. 
\end{lemma}

Notons tout de suite que lemme suffit pour conclure. En utilisant la suite 
spectrale de Leray du morphisme $p$, on en déduit qu'on a, pour tout 
$q\geqslant 0$, 
\[
  \h^q(\bar X_2, j_{2!} \sF) = \h^q(\bar X_1, j_{1!} \sF) \text{.}
\]

Pour démontrer le lemme, on raisonne fibre par fibre en utilisant le théorème 
de changement de base (\ref{I:4-1-1}) pour $p$. Le résultat est immédiat, car, 
au-dessus d'un point de $X$, $p$ est un isomorphe et, au-dessus d'un point de 
$\bar X_1\setminus X$, $j_{2!} \sF$ est nul sur la fibre de $p$. 





\subsubsection{}\label{I:4-5-3}

De même, si $f:X\to S$ est un morphisme séparé de type fini de schémas 
noethériens, il existe un morphisme propre $\bar f:\bar X\to S$ et un 
immersion ouverte $j:X\to \bar X$. On définit alors les images directes s
supérieures à support propre $\R^q f_!$ en posant pour tout faisceau de 
torsion $\sF$ sur $X$
\[
  \R^q f_! \sF = \R^q f_*(j_! \sF) \text{.}
\]
On vérifie comme précédemment que cette définition est indépendante de 
la compactification choisie. 





\begin{theorem}\label{I:4-5-4}
Soient $f:X\to S$ un morphisme séparé de type fini de schémas noethériens 
et $\sF$ un faisceau de torsion sur $X$. Alors la fibre de $\R^q f_! \sF$ en un 
point géométrique $s$ de $S$ est isomorphe à la cohomologie à support 
propre $\h_c^q(X_s,\sF)$ de la fibre $X_s$ de $f$ en $s$. 
\end{theorem}

C'est une simple variante du théorème de changement de base pour un morphisme 
propre (\ref{I:4-1-1}). Plus généralement, si 
\[\xymatrix{
  X \ar[d]^-f
    & X' \ar[l]_-{g'} \ar[d]^-{f'} \\
  S 
    & S' \ar[l]^-g
}\]
est un diagramme cartésien, on a un isomorphe canonique 
\begin{equation*}\tag{4.5.4.1}\label{I:eq:4-5-4-1}
  g^* (\R^q f_! \sF) \simeq \R^q f_!'({g'}^* \sF) \text{.}
\end{equation*}










\subsection{Applications}\label{I:4-6}





\begin{theorem}[d'annulation]\label{I:4-6-1}
Soient $f:X\to S$ un morphisme séparé de type fini dont les fibres sont de 
dimension $\leqslant n$ et $\sF$ un faisceau de torsion sur $X$. Alors on a 
$\R^q f_! \sF = 0$ pour $q>2 n$. 
\end{theorem}

D'après le théorème de changement de base, on peut supposer que $S$ est 
le spectre d'un corps séparablement clos. Si $\dim X = n$, il existe un 
ouvert affine $U$ de $X$ tel que $\dim(X\setminus U) < n$; on a alors un suite 
exacte $0\to \sF_U \to \sF \to \sF_{X\setminus U} \to 0$ et, par récurrence sur $n$, 
il suffit de démontrer le théorème pour $X=U$ affine. Puis la méthode de 
fibration par des courbes (\ref{I:3-4-1}) et le théorème de changement de 
base permettent de se ramener à une courbe sur un corps séparablement clos 
pour laquelle on déduit le résultat voulu du théorème de Tsen 
(\ref{I:3-3-6}). 





\begin{theorem}[de finitude]\label{I:4-6-2}
Soient $f:X\to S$ un morphisme séparé de type fini et $\sF$ un faisceau 
constructible sur $X$. Alors les faisceaux $\R^q f_! \sF$ sont constructibles. 
\end{theorem}

Nous ne considérons que le cas où $\sF$ est annulé par un entier inversible 
sur $X$. 

Par démontrer le théorème on se ramène au cas o`u $\sF$ est un faisceau 
constant $\const{\dZ/n}$ et où $f:X\to S$ est un morphisme propre et lisse 
dont les fibres sont des courbes géométriquement connexes de genre $g$. 
Pour $n$ inversible sur $X$, les faisceaux $\R^q f_* \sF$ sont alors localement 
libres de rang fini, nuls pour $q>2$ (\ref{I:4-6-1}). Remplaçant $\dZ/n$ par le 
faisceau localement isomorphe (sur $S$) $\dmu_n$, on a canoniquement 
\begin{equation*}\tag{4.6.2.1}\label{I:eq:4-6-2-1}
\begin{aligned}
  \R^0 f_* \dmu_n &= \dmu_n \\
  \R^1 f_* \dmu_n &= \const\pic(X/S)_n \\
  \R^2 f_* \dmu_n &= \dZ/n \text{.}
\end{aligned}
\end{equation*}





\begin{theorem}[comparaison avec la cohomologie classique]\label{I:4-6-3}
Soient $f:X\to S$ un morphisme séparé de schémas de type fini sur $\dC$, 
et $\sF$ un faisceau de torsion sur $X$. Notons par un exposant $\an{(-)}$ le 
foncteur de passage aux espaces topologiques usuels, et par $\R^q \an f_!$ les 
foncteurs dérivés du foncteur image directe à support propre par $\an f$. On a 
\[
  \an{(\R^q f_! \sF)} \simeq \R^q \an f_! \an \sF \text{.}
\]
En particulier, pour $S=$ un point et $\sF$ le faisceau constant $\dZ/n$, 
\[
  \h_c^q(X,\dZ/n) \simeq \h_c^q(\an X,\dZ/n)\text{.}
\]
\end{theorem}

Des dévissages utilisant le théorème de changement de bas nous ramènent au 
cas où $X$ est une courbe propre et lisse, où $S=$ un point, et où 
$\sF=\dZ/n$. Les groupes de cohomologie considérés sont alors nuls pour 
$q\ne 0,1,2$, et on invoque \cite{se55}: en effet, si $X$ est propre sur $\dC$, 
on a $\pi_0(X) = \pi_0(\an X)$ et $\pi_1(X) = $ complété profini de 
$\pi_1(\an X)$, d'où l'assertion pour $q=0,1$. Pour $q=2$, on utilise la 
suite exacte de Kummer et le fait que, par \cite{se55} encore, 
$\pic(X)=\pic(\an X)$. 





\begin{theorem}[dimension cohomologique des schémas affines]\label{I:4-6-4}
Soient $X$ un schéma affine de type fini sur un corps séparablement clos et 
$\sF$ un faisceau de torsion sur $X$. Alors on a $\h^1(X,\sF) = 0$ pour 
$q>\dim(X)$. 
\end{theorem}

Pour la très jolie démonstration nous revoyons à \cite{sga4}, XIV \S 2 et 3. 





\subsubsection{Remarque}\label{I:4-6-5}

Ce théorème est en quelque sorte un substitut pour la théorie de Morse. 
Considérons en effet le cas classique où $X$ est lisse et affine sur $\dC$ 
plongé dans un espace affine type $\dC^N$. Alors, pour presque tout point 
$p\in\dC^N$, la fonction ``distance à $p$'' sur $X$ est une fonction de Morse 
et les indices de ses points critiques sont plus petits que $\dim(X)$. Ainsi 
$X$ est obtenu par recollement d'anses d'indice plus petit que $\dim(X)$, d'où 
l'analogue classique de (\ref{I:4-6-4}). 




















\section{Acyclicité locale des morphismes lisses}\label{I:5}

Soient $X$ une variété analytique complexe et $f:X\to D$ un morphisme de 
$X$ dans le disque. On note $[0,t]$ le segment de droit fermé d'extrémités 
$0$ et $t$ dans $D$ et $(0,t]$ le segment semi-ouvert. Si $f$ est \emph{lisse}, 
l'inclusion 
\[
  j : f^{-1}\left((0,t]\right) \hookrightarrow f^{-1}\left([0,t]\right)
\]
est une équivalence d'homotopie; en peut pousser la fibre spécial 
$X_0=f^{-1}(0)$ dans $f^{-1}([0,t])$. 

En pratique, pour $t$ assez petit, $f^{-1}((0,t])$ sera un fibré sur 
$(0,t]$ de sorte que l'inclusion 
\[
  X_t = f^{-1}(t) \hookrightarrow f^{-1}\left((0,t]\right)
\]
sera également une équivalence d'homotopie. On appelle alors \emph{morphisme 
de cospécialisation} la classe d'homotopie d'applications: 
\[
  \cosp : X_0 \hookrightarrow f^{-1}\left([0,t]\right) \xleftarrow\sim f^{-1}\left((0,t]\right) \xleftarrow\sim X_t \text{.}
\]
On peut exprimer cette construction en termes imagés en disant que, pour un 
morphisme lisse, la fibre générale avale la fibre spéciale. 

Ne supposons plus $f$ nécessairement lisse (mais supposons que $f^{-1}((0,t])$ 
soit un fibré sur $(0,t]$). On peut encore définir un morphisme 
$\cosp^\bullet$ en cohomologie dès que $j_* \dZ=\dZ$ et 
$\R^q j_* \dZ = 0$ pour $q>0$. Sous ces hypothèses, la suite spectrale de 
Leray pour $j$ montre qu'on a 
\[
  \h^\bullet\left(f^{-1}\left([0,t]\right),\dZ\right) \iso \h^\bullet\left(f^{-1}\left((0,t]\right),\dZ\right)
\]
et $\cosp^\bullet$ est le morphisme composé: 
\[
  \cosp^\bullet : \h^\bullet(X_t,\dZ)
    \xleftarrow\sim \h^\bullet\left(f^{-1}\left((0,t]\right),\dZ\right) 
    \xleftarrow\sim \h^\bullet\left(f^{-1}\left([0,t]\right),\dZ\right) 
    \to \h^\bullet(X_0,\dZ)
\]
La fibre de $\R^q j_* \dZ$ en un point $x\in X_0$ se calcule comme suit. On 
prend dans un espace ambiant une boule $B_\varepsilon$de centre $x$ et de rayon 
$\varepsilon$ assez petit, et pour $\eta$ assez petit, on pose 
$E=X\cap B_\varepsilon \cap f^{-1}(\eta t)$; c'est la \emph{variété des 
cycles évanescents} en $x$. On a 
\[
  (\R^q j_* \dZ)_x \xleftarrow\sim \h^q\left(X\cap B_\varepsilon\cap f^{-1}\left((0,\eta t]\right),\dZ\right) \xrightarrow\sim \h^q(E,\dZ)
\]
et le morphisme de cospécialisation est défini en cohomologie dés que les 
variétés de cycles évanescents sont acycliques ($\h^0(E,\dZ)=\dZ$ et 
$\h^q(E,\dZ) = 0$ pour $q>0$), ce qui s'exprime en disant que $f$ est 
\emph{localement acyclique}. 

Ce chapitre est consacré à l'analogue de cette situation pour un morphisme 
lisse de schémas pour la cohomologie étale. Cependant il est indispensable 
dans ce cadre de se limiter aux coefficients de torsion et d'ordre \emph{premier aux caractéristiques résiduelles}. La paragraphe \ref{I:5-1} est consacré 
à des généralités sur les morphismes localement acycliques et les 
flèches de cospécialisation. Dans le paragraphe \ref{I:5-2}, on démontre 
qu'un morphisme lisse est localement acyclique. Dans le paragraphe \ref{I:5-3}, on 
joint ce résultat à ceux du chapitre précédent pour en déduire deux 
applications: un théorème de spécialisation des groupes de cohomologie (la 
cohomologie des fibres géométriques d'un morphisme propre et lisse est 
localement constant) et un théorème de changement de base par un morphisme 
lisse. 

Dans tout ce qui suit, on fixe un entier $n$ et ``schéma'' signifie ``schéma 
sur lequel $n$ est inversible.'' ``Point géométrique'' signifiera toujours 
``point géométrique algébrique'' (\ref{I:2-3-1}) $x:\spec(k)\to X$, avec $k$ 
algébriquement clos. 










\subsection{Morphismes localement acycliques}\label{I:5-1}





\subsubsection{Notation}\label{I:5-1-1}

Étant donnés un schéma $S$ et un point géométrique $s$ de $S$, on 
notera $\widetilde S^s$ le spectre du localisé stricte de $S$ en $s$. 





\begin{definition}\label{I:5-1-2}
On dit qu'un point géométrique $t$ de $S$ est une \emph{générisation} de 
$s$ s'il est défini par une clôture algébrique du corps résiduel d'un 
point de $\widetilde S^s$. On dit aussi que $s$ est une \emph{spécialisation} 
de $t$ et on appelle \emph{flèche de spécialisation} le $S$-morphisme 
$t\to \widetilde S^s$. 
\end{definition}





\begin{definition}\label{I:5-1-3}
Soit $f:X\to S$ un morphisme de schémas. Soient $s$ un point géométrique de 
$S$, $t$ une générisation de $s$, $x$ un point géométrique de $X$ 
su-dessus de $s$ et 
$\widetilde X_t^x = \widetilde X^x\times_{\widetilde S^s} t$. Alors on dit que 
$\widetilde X_t^x$ est une \emph{variété de cycles évanescents} de $f$ au 
point $x$. 

On dit que $f$ est \emph{localement acyclique} si, la cohomologie réduite de 
toute variété de cycles évanescents $\widetilde X_t^x$ est nulle:
\begin{equation*}\tag{5.1.3.1}\label{I:eq:5-1-3-1}
  \hat\h^\bullet(\widetilde X_t^x,\dZ/n) = 0\text{,}
\end{equation*}
i.e. $\h^0\left(X_t^x,\dZ/n\right) = \dZ/n$ et 
$\h^q(\widetilde X_t^x,\dZ/n) = 0$ pour $q>0$. 
\end{definition}





\begin{lemma}\label{I:5-1-4}
Soient $f:X\to S$ un morphisme localement acyclique et $g:S'\to S$ un morphisme 
quasi-fini (ou limite projective de morphismes quasi-finis). Alors le 
morphisme $f':X'\to S'$ déduit de $f$ par changement de base est localement 
acyclique.
\end{lemma}

On vérifie en effet que tout variété de cycles évanescents de $f'$ est 
une variété de cycles évanescents de $f$.  	





\begin{lemma}\label{I:5-1-5}
Soit $f:X\to S$ un morphisme localement acyclique. Pour tout point géométrique 
$t$ de $S$, donnant lieu à un diagramme cartésien 
\[\xymatrix{
  X_t \ar[r]^-{\varepsilon'} \ar[d] 
    & X \ar[d]^-f \\
  t \ar[r]^-\varepsilon 
    & S
}\]
On a $\varepsilon_*' \dZ/n = f^*\varepsilon_* \dZ/n$ et $\R^q \varepsilon_*'\dZ/n=0$ 
pour $q>0$. 
\end{lemma}

Soient $\bar S$ l'adhérence de $\varepsilon(t)$, $S'$ le normalisé de $\bar S$ 
dans $k(t)$, et le diagramme cartésien 
\[\xymatrix{
  X_t \ar[r]^-{f'} \ar[d] 
    & X' \ar[r]^-{\alpha'} \ar[d]^-{f'} 
    & X \ar[d]^-f \\
  t \ar[r]^-i 
    & S' \ar[r]^-\alpha 
    & S
}\]
Les anneaux locaux de $S'$ sont normaux à corps de fractions séparablement 
clos. Ils sont donc strictement henséliens, et l'acyclicité locale de $f'$ 
(\ref{I:5-1-4}) fournit $f_*'\dZ/n = \dZ/n$, $\R^qi_*'\dZ/n = 0$ pour $q>0$. 
Puisque $\alpha$ est entier, on a alors 
\[
  \R^q\varepsilon_*'\dZ/n 
    = \alpha_*' \R^q i_*' \dZ/n 
    = \alpha_*' {f'}^* \R^q i_* \dZ/n 
    = f^* \alpha_* \R^q i_* \dZ/n 
    = f^* \R^q \varepsilon_* \dZ/n \text{,}
\]
et le lemme. 





\subsubsection{}\label{I:5-1-6}

Etant donnés un morphisme localement acyclique $f:X\to S$ et une flèche de 
spécialisation $t\to \widetilde S^s$, nous allons définir des 
homomorphismes canoniques, dits flèches de cospécialisation 
\[
  \cosp^\bullet : \h^\bullet(X_t,\dZ/n) \to \h^\bullet(X_s,\dZ/n)\text{,}
\]
reliant la cohomologie de la fibre générale $X_t=X\times_S t$ et celle de la 
fibre spéciale $X_s=X\times_X s$. 

Considérons le diagramme cartésien 
\[\xymatrix{
  X_t \ar[d] \ar[r]^-{\varepsilon'} 
    & \widetilde X \ar[d]^-{f'} 
    & \ar[l] X_s \ar[d] \\
  t \ar[r]^-\varepsilon 
    & \widetilde S^s 
    & \ar[l] s
}\]
déduit de $f$ par changement de base. D'après (\ref{I:5-1-4}), $f'$ est encore 
localement acyclique. De la définition de l'acyclicité locale, on tire 
aussitôt que la restriction à $X_s$ des faisceaux $\R^q \varepsilon_*' \dZ/n$ 
est $\dZ/n$ pour $q=0$, et $0$ pour $q>0$. Par (\ref{I:5-1-5}) on sait même que 
$\R^q \varepsilon_*'\dZ/n = 0$ pour $q>0$. On définit $\cosp^\bullet$ comme la 
flèche composée 
\begin{equation*}\tag{5.1.6.1}\label{I:eq:5-1-6-1}
  \h^\bullet(X_t,\dZ/n) \simeq \h^\bullet(X,\varepsilon_*' \dZ/n) \to \h^\bullet(X_s,\dZ/n) \text{.}
\end{equation*}

\paragraph{Variante}
Soient $\bar S$ l'adhérence de $\varepsilon(t)$ dans $\widetilde S^s$, $S'$ le 
normalisé de $\bar S$ dans $k(t)$ et $X'/S'$ déduit de $X/S$ par changement 
de base. Le diagramme \eqref{I:eq:5-1-6-1} peut encore s'écrire 
\[
  \h^\bullet(X_t,\dZ/n) \simeq \h^\bullet(X',\dZ/n) \to \h^\bullet(X_s,\dZ/n) \text{.}
\]





\begin{theorem}\label{I:5-1-7}
Soient $S$ un schéma localement noethérien, $s$ un point géométrique de 
$S$ et $f:X\to S$ un morphisme. On suppose 
\begin{enumerate}[\indent a)]
  \item le morphisme $f$ est localement acyclique, 
  \item pour tout morphisme de spécialisation $t\to \widetilde S^s$ et pour 
    tout $q\geqslant 0$, les flèches de cospécialisation 
    $\h^q(X_t,\dZ/n) \to \h^q(X_s,\dZ/n)$ sont bijectives.
\end{enumerate}

Alors l'homomorphisme canonique $(\R^q f_*\dZ/n)_s\to \h^q(X_s,\dZ/n)$ est 
bijectif pour tout $q\geqslant 0$. 
\end{theorem}

Pour démontrer le théorème, il est clair qu'on peut supposer 
$S=\widetilde S^s$. On va et fait montrer que, pour tout faisceau de 
$\dZ/n\dZ$-modules $\sF$ sur $S$, l'homomorphisme canonique 
$\varphi^q(\sF):(\R^q f_* f^* \sF)_s \to \h^q(X_s,f^*\sF)$ est bijectif. 

Tout faisceau de $\dZ/n\dZ$ est limite inductive filtrante de faisceaux 
constructibles de $\dZ/n\dZ$-modules (\ref{I:4-3-3}). De plus tout faisceau 
constructible de $\dZ/n\dZ$-modules se plonge dans un faisceau de la forme 
$\prod i_{\lambda *} C_\lambda$, où $i_\lambda:t_\lambda\to S$ est une 
famille finie de générisations de $s$ et $C_\lambda$ un $\dZ/n\dZ$-module 
libre de rang fini sur $t_\lambda$. D''après la définition des flèches de 
cospécialisation, la condition b) signifie que les homomorphismes 
$\varphi^q(\sF)$ sont bijectifs si $\sF$ est de cette forme. 

On conclut à l'aide d'une variant du lemme (\ref{I:4-3-6}):





\begin{lemma}\label{I:5-1-8}
Soit $\cC$ un catégorie abélienne dans laquelle les limites inductives 
filtrantes existent. Soit $\varphi^\bullet:T^\bullet\to{T'}^\bullet$ un 
morphisme de foncteurs cohomologiques commutant aux limites inductives 
filtrantes, définis sur $\cC$ et à valeurs dans la catégorie des groupes 
abéliens. Supposons qu'il existe deux sous-ensembles $\cD$ et $\cE$ d'objets 
de $\cC$ tels que:
\begin{enumerate}[\indent a)]
  \item tout objet de $\cC$ est limite inductive filtrante d'objets appartenant 
    à $\cD$, 
  \item tout objet appartenant à $\cD$ est contenu dans un objet appartenant 
    à $\cE$.
\end{enumerate}

Alors les conditions suivantes sont équivalentes: 
\begin{enumerate}[\indent (i)]
  \item $\varphi^q(A)$ est bijectif pour tout $q\geqslant 0$ et tout 
    $A\in\ob\cC$.
  \item $\varphi^q(M)$ est bijectif pour tout $q\geqslant 0$ et tout $M\in\cE$. 
\end{enumerate}
\end{lemma}

La démonstration du lemme se fait par passage à la limite inductive, 
récurrence sur $q$ et application répétée du lemme des cinq au diagramme 
de suites exactes de cohomologie déduit d'une suite exacte 
$0\to A\to M\to A'\to 0$, avec $A\in\cD$, $M\in\cE$, $A'\in \ob\cC$. 





\begin{corollary}\label{I:5-1-9}
Soient $S$ le spectre d'un anneau local noethérien strictement hensélien et 
$f:X\to S$ un morphisme localement acyclique. Supposons que, pour tout point 
géométrique $t$ de $S$ on dit $\h^0(X_t,\dZ/n) = \dZ/n$ et 
$\h^q(X_t,\dZ/n) = 0$ pour $q>0$ (autrement dit les fibres géométriques de 
$f$ sont acycliques). Alors on a $f_* \dZ/n\dZ/n$ et $\R^qf_*\dZ/n = 0$ pour 
$q>0$. 
\end{corollary}





\begin{corollary}\label{I:5-1-10}
Soient $f:X\to Y$ et $g:Y\to X$ des morphismes de schémas localement 
noethériens. Alors, si $f$ et $g$ sont localement acycliques, il en est de 
même de $g\circ f$. 
\end{corollary}

On peut supposer que $X$, $Y$ et $Z$ sont strictement locaux et que $f$ et $g$ 
sont des morphismes locaux. Il s'agit alors de montrer que, si $z$ est un point 
géométrique algébrique de $Z$, on a $\h^0(X_z,\dZ/n) = \dZ/n$ et 
$\h^q(X_z,\dZ/n) = 0$ pour $q>0$. 

Puisque $g$ est localement acyclique, on a $\h^0(Y_z,\dZ/n) = \dZ/n$, et 
$\h^q(Y_z,\dZ/n) = 0$ pour $q>0$. Par ailleurs le morphisme $f_z:X_z\to Y_z$ est 
localement acyclique (\ref{I:5-1-4}) et ses fibres géométriques sont 
acycliques, car ce sont des variétés de cycles évanescents de $f$. 
D'après (\ref{I:5-1-9}), on a donc $\R^q {f_z}_*\dZ/n = 0$ pour $q>0$. De plus 
${f_z}_*\dZ/n$ est constant de fibre $\dZ/n$ sur $Y_z$. On conclut à l'aide de 
la suite spectrale de Leray de $f_z$. 










\subsection{Acyclicité locale d'un morphisme lisse}\label{I:5-2}





\begin{theorem}\label{I:5-2-1}
Un morphisme lisse est localement acyclique.
\end{theorem}

Soit $f:X\to S$ un morphisme lisse. L'assertion est locale pour la topologie 
étale sur $X$ et $S$, on peut donc supposer que $X$ est l'espace affine type 
de dimension $d$ sur $S$. Par passage à la limite, on peut supposer que $S$ 
est noethérien et la transitivité de l'acyclicité locale (\ref{I:5-1-10}) montre 
qu'il suffit de traiter le cas $d=1$. 

Soient $s$ un point géométrique de $S$ et $x$ un point géométrique de 
$X$ centré en un point fermé de $X_s$. Il s'agit de montrer que les fibres 
géométriques du morphisme $\widetilde X^x\to \widetilde S^s$ sont 
acycliques. On posera désormais $S=\widetilde S^s=\spec(A)$ et 
$X=\widetilde X^x$. On a $X\simeq \spec A\{T\}$, où $A\{T\}$ est l'hensélise 
de $A[T]$ au point $T=0$ au-dessus de $s$. 

Si $t$ est un point géométrique de $S$, la fibre $X_t$ est limite 
projective de courbes affines et lisses sur $t$. On a donc $\h^q(X_t,\dZ/n) = 0$ 
pour $q\geqslant 0$ et il suffit de montrer que $\h^0(X_t,\dZ/n) = \dZ/n$ et 
$\h^1(X_t,\dZ/n) = 0$ pour $n$ premier à la caractéristique résiduelle de 
$S$. Cela résulte des deux propositions suivantes. 





\begin{proposition}\label{I:5-2-2}
Soient $A$ un anneau local strictement hensélien, $S=\spec(A)$ et 
$X=\spec A\{T\}$. Alors les fibres géométrique de $X\to S$ sont connexes. 
\end{proposition}

On peut se ramener par passage à limite au cas où $A$ est un hensélisé 
strict d'une $\dZ$-algèbre de type fini. 

Soient $\bar t$ un point géométrique de $S$, localisé en $t$, et $k'$ une 
extension finie séparable de $k(t)$ dans $k(\bar t)$. On pose $t'=\spec(k')$ 
et $X_{t'} = X\times_S\spec(k')$. Il nous faut vérifier que, quelques soient 
$\bar t$ et $t'$, $X_{t'}$ est connexe (par qui on entend connexe et non vide). 
Soit $A'$ le normalisé de $A$ dans $k'$, i.e. l'anneau des éléments de 
$k'$ entiers sur l'image de $A$ dans $k(t)$. On a 
$A\{t\}\otimes_A A'\iso A'\{T\}$: le membre de gauche est en effet hensélien 
local (car $A'$ est fini sur $A$, et local) et limite d'algèbres locales 
étales sur $A'\{T\}=A\{T\}\otimes_A A'$. Le schéma $X_{t'}$ est donc encore 
la fibre en $t'$ de $X=\spec\left(A'\{T\}\right)$ sur $S'=\spec(A')$. Le 
schéma local $X'$ est normal, donc intègre; son localisé $X_{t'}$ est 
encore intègre, a fortiori connexe. 





\begin{proposition}\label{I:5-2-3}
Soient $A$ un anneau local strictement hensélien, $S=\spec(A)$ et 
$X=\spec\left(A\{T\}\right)$. Soient $\bar t$ un point géométrique de $S$ 
et $X_{\bar t}$ la fibre géométrique correspondante. Alors tout revêtement 
étale galoisien de $X_{\bar t}$ d'ordre premier à la caractéristique du 
corps résiduel de $A$ est trivial. 
\end{proposition}





\begin{lemma}[théorème de pureté de Zariski-Nagata en dimension $2$] \label{I:5-2-4} % originally 5.2.3.1
Soient $C$ un anneau local régulier de dimension $2$ et $C'$ une $C$-algèbre 
finie normale étale au--dessus de l'ouvert complémentaire du point fermé 
de $\spec(C)$. Alors $C'$ est étale sur $C$.
\end{lemma}

En effet $C'$ est normal de dimension $2$, donc $\operatorname{prof}(C')=2$. 
Puisque $\operatorname{prof}(C')+\dim\operatorname{proj}(C') = \dim(C)=2$, on 
en conclut que $C'$ est libre sur $C$. Alors l'ensemble des points de $C$ où 
$C'$ est ramifié est défini par une équations, le discriminant; puisqu'il 
ne contient pas de point de hauteur $1$, il est vide. 





\begin{lemma}[cas particulier du lemme d'Abhyankar]\label{I:5-2-5} % originally 5.2.3.2
Soient $S=\spec(V)$ un trait, $\pi$ une uniformisante, $\eta$ le point 
générique de $S$, $X$ lisse sur $S$, irréductible, de dimension 
relative $1$, $\widetilde X_\eta$ un revêtement étale galoisien de $X_\eta$, 
de degré $n$ inversible sur $S$, et $S_1=\spec\left(V[\pi^{1/n}]\right)$. 
Notons par un indice $(-)_1$ le changement de base de $S$ à $S_1$. Alors, 
$\widetilde X_{1\eta}$ se prolonge en un revêtement étale de $X_1$. 
\end{lemma}

Soit $\widetilde X_1$ le normalisé de $X_1$ de $\widetilde X_{1\eta}$. Vu la 
structure des groupes d'inertie modérée des anneaux de valuation discrète 
localisés de $X$ aux points génériques de la fibre spéciale $X_s$, 
$\widetilde X_1$ est étale sur $X_1$ sur la fibre générale, et aux points 
génériques de la fibre spéciale. Par (\ref{I:5-2-4}), il est étale 
partout. 





\subsubsection{}\label{I:5-2-6} % originally 5.2.3.3

Notons $t$ le point en lequel $\bar t$ est localisé. Il nous est loisible de 
remplacer $A$ par le normalisé de $A$ dans une extension finie séparable 
$k(t')$ de $k(t)$ dans $k(\bar t)$ (cf. \ref{I:5-2-2}). Ceci, et un passage à 
la limite préliminaire, nous permettent de supposer que 
\begin{enumerate}[\indent a)]
  \item $A$ est normal noethérien, et $t$ est le point générique de $S$.
  \item Le revêtement étale considéré de $X_{\bar t}$ provient d'un 
    revêtement étale de $X_t$. 
  \item Il provient d'un revêtement étale $\widetilde X_U$ de l'image 
    réciproque $X_U$ d'un ouvert non vide $U$ de $S$ (résulte de b): $t$ 
    est la limite des $U$).
  \item La complément de $U$ est de codimension $\geqslant 2$ (ceci au prix 
    d'agrandir $k(t)$, par application de (\ref{I:5-2-5}) aux anneaux de 
    valuation discrète localisés de $S$ en les points de $S\setminus U$ de 
    codimension $1$ dans $S$; ces points sont en nombre fini). 
  \item Le revêtement $\widetilde X_U$ est trivial su-dessus du sous-schéma 
    $T=0$. (Ceci quitte à encore agrandir $k(t)$). 
\end{enumerate}

Lorsque ces conditions sont remplies, nous allons voir que le revêtement 
$\widetilde X_U$ est trivial. 





\begin{lemma}\label{I:5-2-7} % originally 5.2.3.4
Soient $A$ un anneau local strictement hensélien normal et noethérien, $U$ 
un ouvert de $\spec(A)$ dont le complément est de codimension $\geqslant 2$, 
$V$ son image réciproque dans $X=\spec\left(A\{T\}\right)$ et $V'$ un 
revêtement étale de $V$. Si $V'$ est trivial au-dessus de $T=0$, alors $V'$ 
est trivial. 
\end{lemma}

Soit $B=\Gamma(V',\sO)$. Puisque $V'$ est l'image inverse de $V$ dans 
$\spec(B)$, il suffit de voir que $B$ est fini étale sur $A\{T\}$ (donc 
décomposé, puisque $A\{T\}$ est strictement hensélien). Soit 
$\widehat X=A\llbracket T\rrbracket$, et notons par $(-)^\wedge$ le changement 
de base de $X$ à $\widehat X$. Le schéma $\widehat X$ est fidèlement plat 
sur $X$. On a donc 
$\Gamma(\widehat V',\sO) = B\otimes_{A\{T\}} A\llbracket T\rrbracket$, et il 
suffit de voir que cet anneau $\widehat B$ est fini étale sur 
$A\llbracket T\rrbracket$. 

Soit $V_m$ (resp. $V_m'$) le sous-schéma de $\widehat V$ (resp. $\widehat V'$) 
d'équation $T^{m+1} = 0$. Par hypothèse, $V_0'$ est un revêtement trivial 
de $V_0$: une somme de $n$ copies de $V_0$. De même pour $V_m'/V_m$, puisque 
les revêtements étales sont insensibles aux nilpotents. On en tire 
\[
  \varphi : \Gamma(\widehat V',\sO) \to \varprojlim_m \Gamma(V_m',\sO) = \left(\varprojlim_m \Gamma(V,\sO)\right)^n \text{.}
\]
Par hypothèse, le complément de $U$ est de profondeur $\geqslant 2$: on a 
$\Gamma(V_m,\sO)=A[T]/(T^{m+1})$, et $\varphi$ est un homomorphisme de 
$\widehat B$ dans $A\llbracket T\rrbracket^n$. Au-dessus de $U$, il fournit $n$ 
sections distinctes de $\widehat V'/\widehat V$: $\widehat V'$ est donc trivial, 
somme de $n$ copies de $\widehat V$. Le complément de $\widehat V$ dans 
$\widehat X$ étant encore de codimension $\geqslant 2$ (donc de profondeur 
$\geqslant 2$), on en déduit que $\widehat B=A\llbracket T\rrbracket^n$, 
d'où le lemme. 










\subsection{Applications}\label{I:5-3}





\begin{theorem}[spécialisation des groupes de cohomologie]\label{I:5-3-1}
Soit $f:X\to S$ un morphisme propre et localement acyclique, par exemple un 
morphisme propre et lisse. Alors les faisceaux $\R^q f_*\dZ/n$ sont localement 
constants constructibles et pour toute flèche de spécialisation 
$t\to\widetilde S^s$, les flèches de cospécialisation 
$\h^q(X_t,\dZ/n)\to\h^q(X_s,\dZ/n)$ sont bijectives.
\end{theorem}

Cela résulte immédiatement de la définition des flèches de 
cospécialisation et des théorèmes de finitude et de changement de base pour 
les propres. 





\begin{theorem}[changement de base par un morphisme lisse]\label{I:5-3-2}
Soit un diagramme cartésien 
\[\xymatrix{
  X' \ar[r]^-{g'} \ar[d]^-{f'} 
    & X \ar[d]^-f \\
  S' \ar[r]^-{g} 
    & S
}\]
avec $g$ lisse. Pour tout faisceau $\sF$ sur $X$, de torsion et premier aux 
caractéristiques résiduelles de $S$, on a 
\[
  g^* \R^q f_* \sF \iso \R^q f_*'({g'}^* \sF) \text{.}
\]
\end{theorem}

En prenant un recouvrement ouvert de $X$< on se ramène au cas où $X$ est 
affine, puis par un passage à la limite au case \`i $X$ est de type fini sur 
$S$. Alors $f$ se factorise en une immersion ouverte $j:X\to \bar X$ et un 
morphisme propre $\bar f:\bar X\to S$. De la suite spectrale de Leray pour 
$\bar f\circ j$ et du théorème de changement de base pour les morphismes 
propres, on déduit qu'il suffit de démontrer le théorème dans le cas 
où $X\to S$ est une immersion ouverte. 

Dans ce cas, si $\sF$ est de la forme $\varepsilon_* C$, où 
$\varepsilon:t\to X$ est un point géométrique de $X$< le théorème est 
corollaire de (\ref{I:5-1-5}). Le cas général en résulte par le lemme 
(\ref{I:5-1-8}). 





\begin{corollary}\label{I:5-3-3}
Soient $K/k$ une extension de corps séparablement clos, $X$ un $k$-schéma 
et $n$ un entier premier à la charactéristique de $k$. Alors l'application 
canonique $\h^q(X,\dZ/n) \to \h^q(X_K,\dZ/n)$ est bijective pour tout 
$q\geqslant 0$. 
\end{corollary}

Il suffit de remarquer que $\bar K$ est limite inductive de $\bar k$-algèbres 
lisses. 





\begin{theorem}[pureté relative]\label{I:5-3-4}
Soit un diagramme commutatif 
\begin{equation*}\tag{5.3.4.1}\label{I:eq:5-3-4-1}\xymatrix{
  U \ar@{^{(}->}[r]^-j \ar[dr]
    & X \ar[d]^-f 
    & \ar@{_{(}->}[l]_-i Y \ar[dl]^-h \\
  & S
}\end{equation*}
avec $f$ lisse purement de dimension relative $N$, $h$ lisse purement de 
dimension relative $N-1$, $i$ un plongement fermé et $U=X\setminus Y$. Pour 
$n$ premier aux charactéristiques résiduelles de $S$, on a 
\begin{equation*}
\begin{aligned}
  j_* \dZ/n     &= \dZ/n \\
  \R^1 j_*\dZ/n &= \dZ/n(-1)_Y \\
  \R^q j_*\dZ/n &= 0 \qquad \text{pour $q\geqslant 2$}
\end{aligned}
\end{equation*}
\end{theorem}

Dans ces formules, $\dZ/n(-1)$ désigne le $\dZ/n$-duel de $\dmu_n$. Si $t$ 
est une équation locale pour $Y$, l'isomorphisme 
$\R^1j_*\dZ/n\simeq\dZ/n(-1)_Y$ est défini par l'application 
$a:\dZ/n\to \R^1j_*\dmu_n$ qui envoie $1$ sur la classe du $\dmu_n$-torseur des 
racines $n$-ièmes de $t$. 

La question est de nature locale. Ceci permet de remplacer $(X,Y)$ par un 
couple localement isomorphe, par exemple 
\[\xymatrix{
  \dA_T^1 \ar@{^{(}->}[r]^-j \ar[dr]_-g 
    & \dP_T^1 \ar[d]^-f 
    & \ar@{_{(}->}[l]_-i T \ar[dl] \\
  & T
}\]
avec $T=\dA_S^{n-1}$ et $i=$ section à l'infini. Le corollaire (\ref{I:5-1-9}) 
s'applique à $g$, et fournit $\R^q g_ \dZ/n=\dZ/n$ pour $q=0$, $0$ pour $q>0$. 
Pour $f$, on a par ailleurs \eqref{I:eq:4-6-2-1}
\[
  \R^q f_* \dZ/n = \dZ/n,0,\dZ/n(-1),0 \text{ pour $q>2$.}
\]
On vérifie facilement que $j_* \dZ/n = \dZ/n$, et que les $\R^q j_*\dZ/n = 0$ 
sont concentrés sur $i(T)$ pour $q>0$. La suite spectrale de Leray 
$E_2^{pq} = \R^p f_* \R^q j_*\dZ/n \Rightarrow \R^{p+q}g_* \dZ/n$ se réduit 
donc à 
\[\xymatrix{
  i^* \R^q j_* \dZ/n & 0 & \cdots \\
  i^* \R^1 j_* \dZ/n \ar[drr]^-{d_2} & 0 & \cdots \\
  \dZ/n              & 0 & \dZ/n(-1) & 0 & \cdots
}\]
$\R^q j_*\dZ/n=0$ pour $q\geqslant 0$, et $\R^q j_*\dZ/n$ est le prolongement 
par zéro d'un faisceau localement libre de rang un sur $T$ (isomorphe, via 
$d_2$, à $\dZ/n(-1)$). L'application $a$, définie plus haut, étant 
injective (ainsi qu'on le vérifie fibre par fibre), c'est un isomorphisme, et 
ceci prouve (\ref{I:5-3-4}). 





\subsubsection{}\label{I:5-3-5}

Nous renvoyons à \cite[XVI.4,5]{sga4} pour la démonstration des 
applications suivantes du théorème d'acyclicité (\ref{I:5-2-1}). 





\begin{theorem}\label{I:5-3-6} % originally 5.3.5.1
Soient $f:X\to S$ un morphisme de schémas de type finie sur $\dC$ et $\sF$ un 
faisceau constructible sur $X$. Alors 
\[
  \an{(\R^q f_*\sF)} \sim \R^q \an f_* (\an\sF)
\]
(cf. \ref{I:4-6-3}: en cohomologie ordinaire, il est nécessaire de supposer 
$\sF$ est constructible et non seulement de torsion). 
\end{theorem}





\begin{theorem}\label{I:5-3-7} % originally 5.3.5.2
Soient $f:X\to S$ un morphisme de schémas de type fini sur un corps $k$ de 
charactéristique $0$ et $\sF$ un faisceau constructible sur $X$. Alors, les 
$\R^q f_* \sF$ sont également constructibles.
\end{theorem}

La preuve utilise la résolution des singularités et (\ref{I:5-3-4}). Elle est 
généralisée au cas d'un morphisme de type fini de schémas excellents de 
charactéristique $0$ dans \cite[XIX.5]{sga4}. Une autre démonstration, 
indépendante de la résolution, est donnée dans ce volume 
(\nameref{VII}, \ref{VII:1-1}) 
% NOTE: originally (Th. finitude, 1.1),
elle s'applique à un morphisme de schémas de type fini sur un corps ou sur 
un anneau de Dedekind. 




















\section{Dualité de Poincaré}










\subsection{Introduction}\label{I:6-1}

Soit $X$ un variété topologique orientée, purement de dimension $N$, et 
supposons que $X$ admette un recouvrement ouvert fini 
$\cU=(U_i)_{1\leqslant i\leqslant K}$, tel que les intersections non vides 
d'ouverts $U_i$ soient homéomorphes à des boules. Pour une telle variété, 
le théorème de dualité de Poincaré peut se présenter ainsi: 

\begin{enumerate}[\indent A]
  \item La cohomologie de $X$ est la cohomologie de \v Cech correspondant au 
    revêtement $\cU$. C'est la cohomologie du complexe 
    \begin{equation}\label{I:eq:6-1}
      0 \to \dZ^{A_0} \to \dZ^{A_1} \to \cdots
    \end{equation}
    où $A_k=\{(i_0,\dotsc,i_k) : i_0<\cdots<i_k\text{ et } U_{i_0}\cap \cdots \cap U_{i_k} \ne \varnothing\}$. 
    \item Pour $a\in A_k$, $a=(i_0,\dotsc,i_k)$, soit 
      $U_a=U_{i_1}\cap \cdots \cap U_{i_k}$ et soit $j_a$ l'inclusion de $U_a$ 
      dans $X$. Le faisceau constant $\dZ$ sur $X$ admet la résolution (à 
      gauche) 
      \begin{equation}\label{I:eq:6-2}
      \xymatrix{
        \cdots \ar[r]
          & \displaystyle\bigoplus_{a\in A_1} j_{a!}\dZ \ar[r] 
          & \displaystyle\bigoplus_{a\in A_0} j_{a!} \dZ \ar[r] \ar[d] 
          & 0 \\
        & & \dZ \text{.}
      }
      \end{equation}
      La cohomologie à support compact $\h_c^\bullet(X,j_{a!}\dZ)$ n'est autre 
      que la cohomologie à support propre de la boule (orientée) $U_a$: 
        \[
          \h_c^i(X,j_{a!}\dZ) = \begin{cases}
                                  0   & \text{si $i\ne N$} \\
                                  \dZ & \text{si $i = N$}
                                \end{cases}
        \]
\end{enumerate}

La suite spectrale d'hypercohomologie, pour le complexe \eqref{I:eq:6-2}, et la 
cohomologie \`a support propre, affirme donc que $\h_c^i(X)$ est le 
$(i-N)$-i\`eme groupe de cohomologie d'un complexe 
\begin{equation}\label{I:eq:6-3}
  \cdots \to \dZ^{A_1} \to \dZ^{A_0} \to 0 \text{.}
\end{equation}
Ce complexe est le dual du complexe \eqref{I:eq:6-1}, d'o\`u la dualit\'e de 
Poincare.

Les points essentiels de cette construction sont 
\begin{enumerate}[\indent a)]
  \item l'existence d'une théorie de cohomologie à support propre: 
  \item le fait que tout point $x$ d'une varété $X$ purement de dimension 
    $N$ a un système fondamental de voisinages ouverts $U$ pour laquels 
    \begin{equation}\label{I:eq:6-4}
      \h_c^i(U) = \begin{cases}
                    0   & \text{pour $i\ne N$,} \\
                    \dZ & \text{pour $i=N$.}
                  \end{cases}
    \end{equation}
\end{enumerate}

La dualité de Poincaré en cohomologie étale peut se construire sur ce 
modèle. Pour $X$ lisse purement de dimension $N$ sur $k$ algébriquement clos, 
$n$ inversible sur $X$ et $x$ un point fermé de $X$, le point clef est de 
calculer la limite projective, étendue aux voisinages étales $U$ de $x$ 
\begin{equation}\label{I:eq:6-5}
  \varprojlim \h_c^i(U,\dZ/n) 
    = \begin{cases}
        0     & \text{si $i\ne 2N$} \\
        \dZ/n & \text{si $i=2N$.}
      \end{cases} 
\end{equation}

De même que pour les variétés topologiques il faut d'abord traiter 
directement le cas d'une boule ouverte (voire simplement celui de l'intervalle 
$(0,1)$), ici il faut d'abord traiter directement le cas des courbes 
(\ref{I:6-2}). Le théorème d'acyclicité locale des morphismes lisses 
permet ensuite de ramener \eqref{I:eq:6-5} à ce cas particulier (\ref{I:6-3}). 

Les isomorphismes \eqref{I:eq:6-4} et \eqref{I:eq:6-5} ne sont pas canoniques: 
ils dépendent du choix d'une \emph{orientation} de $X$. Pour $n$ inversible 
sur un schéma $X$, $\dmu_n$ est un faisceau de $\dZ/n$-modules libres de rang 
un. On note $\dZ/n(N)$ sa puissance tensorielle $N$-ième ($N\in\dZ$). La 
forme intrinsèque de la duixième ligne de \eqref{I:eq:6-5} est 
\begin{equation}\label{I:eq:6-5'}
  \varprojlim \h_c^{2N} \left(U,\dZ/n(N)\right) = \dZ/n \text{,}
\end{equation}
et $\dZ/n(N)$ s'appelle le \emph{faisceau d'orientation} de $X$; le faisceau 
$\dZ/n(N)$ étant constant, isomorphe à $\dZ/n$, on peut le faire sortir 
du signe $N$ et écrire plutôt
\begin{equation}\label{I:eq:6-5''}
  \varprojlim \h_c^{2N} (U,\dZ/n) = \dZ/n(-N) \text{,}
\end{equation}
et la dualité de Poincaré prendra la forme d'une dualité parfaite, à 
valeurs dans $\dZ/n(-N)$, entre $\h^i(X,\dZ/n)$ et $\h_c^{2N-i}(X,\dZ/n)$. 










\subsection{Le cas des courbes}\label{I:6-2}





\subsubsection{}\label{I:6-2-1}

Soient $\bar X$ une courbe projective et lisse sur $k$ algébriquement clos, et 
$n$ inversible sur $\bar X$. La preuve de (\ref{I:3-3-5}) fournit, pour 
$\bar X$ connexe, un isomorphisme canonique 
\[\xymatrix{
  \h^2(\bar X,\dmu_n) = \pic(\bar X)/n\pic(\bar X) \ar[r]^-{\deg}_-{\sim}
    & \dZ/n
}\]
Soient $D$ un diviseur réduit de $\bar X$ et $X=\bar X\setminus D$ 
\[\xymatrix{
  X \ar@{^{(}->}[r]^-j 
    & \bar X 
    & \ar@{_{(}->}[l]_-i D \text{.}
}\]
La suite exacte $0\to j_! \dmu_n \to \dmu_n\to i_* \dmu_n\to 0$ et le fait que 
$\h^i(\bar X,i_*\dmu_n) = \h^i(D,\dmu_n)=0$ pour $i>0$ fournissent un 
isomorphisme 
\[\xymatrix{
  \h_c^2(X,\dmu_n) = \h^2(\bar X,j_! \dmu_n) \ar[r]^-\sim 
    & \h^2(\bar X,\dmu_n) \ar[r]^-\sim 
    & \dZ/n \text{.}
}\]
Pour $\bar X$ disconnexe, on a de même 
\[
  \h_c^2(X,\dmu_n) = (\dZ/n)^{\pi_0(X)}
\]
et on définit le \emph{morphisme trace} comme la somme 
\[\xymatrix{
  \tr : \h_c^2(X,\dmu_n) \simeq (\dZ/n)^{\pi_0(X)} \ar[r]^-\Sigma
    & \dZ/n\text{.}
}\]





\begin{theorem}\label{I:6-2-2}
La forme $\tr(a\smallsmile b)$ identifie chacun des deux groupes $\h^i(X,\dZ/n)$ 
et $\h_c^{2-i}(X,\dmu_n)$ au dual (à valeurs dans $\dZ/n$) de l'autre. 
\end{theorem}

\emph{Preuve transcendante.} Si $\bar X$ est une courbe projective et lisse sur 
un trait $S$, et que $j:X\hookrightarrow \bar X$ est l'inclusion du 
complémentaire d'un diviseur $D$ étale sur $S$< les cohomologies (resp. 
les cohomologies à support propre) des fibres géométriques spéciales et 
génériques de $X/S$ sont ``les mêmes,'' i.e. les fibres de faisceaux 
localement constants sur $S$. Ceci se déduit des faits analogues pour 
$\bar X$ et $D$, via la suite exacte $0\to j_! \dZ/n\to\dZ/n\to {\dZ/n}_D\to 0$ 
(pour la cohomologie à supports propres) et les formules $j_*\dZ/n = \dZ/n$, 
$\R^1 j_* \dZ/n \simeq {\dZ/n}_D(-1)$, $\R^i j_*\dZ/n = 0$ ($i\geqslant 2$) 
(pour la cohomologie ordinaire) (\ref{I:5-3-4}). 

Ce principe de spécialisation ramène (\ref{I:6-2-2}) au cas où $k$ est de 
charactéristique $0$. Par (\ref{I:5-3-3}), ce cas se ramène à celui où 
$k=\dC$. Enfin, pour $k=\dC$, les groupes $\h^\bullet(X,\dZ/n)$ et 
$\h_c^\bullet(X,\dmu_n)$ coïncident avec les groupes de même nom, calculés 
pour l'espace topologique classique $X_{\textnormal{cl}}$ et, via 
l'isomorphisme $\dZ/n\to \dmu_n: x\mapsto \exp\left(\frac{2\pi i x}{n}\right)$, 
le morphisme trace s'identifie à ``intégration sur la classe fondamentale,'' 
de sorte que (\ref{I:6-2-2}) résulte de la dualité de Poincaré pour 
$X_{\textnormal{cl}}$. 





\subsubsection{Preuve algébrique}\label{I:6-2-3}

Pour une preuve très économique, voir (\nameref{V}, \S\ref{V:2}). 
% originally (Dualité \S 2)
% NOTE: should be hyperlinked V:2
En voici une autre, liée à l'autodualité de la jacobienne. 

Reprenons les notations de (\ref{I:6-2-1}). On peut supposer -- et on suppose -- 
que $X$ est connexe. Les cas $i=0$ et $i=2$ étant triviaux, on suppose aussi 
que $i=1$. Définissons $_D\dG_m$ par la suite exacte 
$0\to _D\dG_m \to \dG_m \to i_*\dG_m\to 0$ (sections de $\dG_m$ congrues à 
$1\mod D$). Le groupe $\h^1(\bar X, _D\dG_m)$ classifie les faisceaux 
inversibles sur $\bar X$ trivialisés sur $D$. C'est le groupe des points de 
$\pic_D(\bar X)$, une extension de $\dZ$ (le degré) par le groupe des points 
d'une jacobienne généralisée de Rosenlicht (correspondant su conducteur 
$1$ en chaque point de $D$) $\pic_D^0(\bar X)$, elle-même extension de la 
variété abélienne $\pic^0(\bar X)$ par le tore 
$\dG_m^D/(\text{$\dG_m$ diagonal})$. 

\begin{enumerate}[\indent a)]
  \item La suite exacte 
    $0\to j_!\dmu_n \to _D\dG_m \xrightarrow{n} _D\dG_m\to 0$ fournit un 
    isomorphisme 
    \begin{equation*}\tag{6.2.3.1}\label{I:eq:6-2-3-1}
      \h_c^1(X,\dmu_n) = \pic_D^0(\bar X)_n \text{.}
    \end{equation*}
  \item L'application qui à $x\in X(k)$ associe la classe du faisceau 
    inversible $\sO(x)$ sur $\bar X$, trivialisé par $1$ sur $D$, provient 
    d'un morphisme 
    \[
      f:X \to \pic_D(\bar X) \text{.}
    \]
\end{enumerate}
Pour la suite, on fixe un point base $0$ et on pose $f_0(x)=f(x)-f(0)$. Pour 
tout homomorphisme $v:\pic_D^0(\bar X)_n \to \dZ/n$, soit 
$\bar v\in \h^1\left(\pic_D^0(\bar X),\dZ/n\right)$ l'image par $v$ de la classe 
dans $\h^1\left(\pic_D^0(\bar X),\pic_D^0(\bar X)_n\right)$ du torseur défini 
par l'extension 
\[\xymatrix{
  0 \ar[r] 
    & \pic_D^0(\bar X)_n \ar[r] 
    & \pic_D^0(\bar X) \ar[r]^-n 
    & \pic_D^0(\bar X) \ar[r] 
    & 0
}\]
La théorie du corps de classes géométrique (telle qu'exposée dans 
Serre \cite{se59}) montre que l'application $v\mapsto f_0^*(\bar v)$: 
\begin{equation*}\tag{6.2.3.2}\label{I:eq:6-2-3-2}
  \hom\left(\pic_D^0(\bar X)_n,\dZ/n\right) \to \h^1(X,\dZ/n)
\end{equation*}
est un isomorphisme. Pour déduire (\ref{I:6-2-2}) de \eqref{I:eq:6-2-3-1} et 
\eqref{I:eq:6-2-3-2}, il reste à savoir que 
\begin{equation*}\tag{6.2.2.3}\label{I:eq:6-2-3-3}
  \tr\left(u \smallsmile f_0^*(\bar v)\right) = -v(u) \text{.}
\end{equation*}

% originally (Dualité, 3.2.4).










\subsection{Le cas général}\label{I:6-3}

Soient $X$ un variété algébrique lisse et purement de dimension $N$, sur 
$k$ algébriquement clos. Pour énoncer le théorème de dualité de 
Poincaré, il faut tout d'abord définir le morphisme trace 
\[
  \tr : \h_c^{2N}\left(X,\dZ/n(N)\right) \to \dZ/n\text{.}
\]
La définition est un dévissage pénible à partir dur cas des courbes 
\cite[XVIII \S 2]{sga4}. On a alors 





\begin{theorem}\label{I:6-3-1}
La forme $\tr(a\smallsmile b)$ identifie chacun des groupes 
$\h_c^i\left(X,\dZ/n(N)\right)$ et $\h^{2 N-i}(X,\dZ/n)$ au dual de l'autre.
\end{theorem}

Soient $x\in X$ un point fermé et $X_x$ le localisé strict de $X$ en $x$. 
Nous posons, pour $U$ parcourant les voisinages étales de $x$ 
\begin{equation*}\tag{6.3.1.1}\label{I:eq:6-3-1-1}
  \h_c^\bullet(X_x,\dZ/n) = \varprojlim \h_c^\bullet(U,\dZ/n) \text{.}
\end{equation*}
Il serait préférable de considérer plutôt le pro-objet 
$\text{``}\varprojlim\text{''} \h_c^\bullet(U,\dZ/n)$ mais, les groupes en jeu 
étant finis, la différence est inessentielle. Comme on a tenté de 
l'expliquer dans l'introduction, (\ref{I:6-3-1}) résulte de ce que 
\begin{equation*}\tag{6.3.1.2}\label{I:eq:6-3-1-2}
\begin{aligned}
  \h_c^i(X_x,\dZ/n) &= 0 \text{ pour $i\ne 2N$ et que } \\
  \tr : \h_c^{2 N}\left(X_x,  \dZ/n(N)\right) &\iso \dZ/n \text{ est un isomorphisme.} 
\end{aligned}
\end{equation*}

Le cas où $N=0$ est trivial. Si $N>0$, soit $Y_y$ le localisé strict en un 
point fermé d'un schéma $Y$ lisse purement de dimension $N-1$, et soit 
$f:X_x\to Y_y$ un morphisme essentiellement lisse (de dimension relative un). 
La démonstration utilise la suite spectrale de Leray en cohomologie à 
support propre pour $f$, pour se ramener au cas des courbes. Les ``cohomologie 
à support propre'' considérées étant définies comme des limites 
\eqref{I:eq:6-3-1-1}, l'existence d'une telle suite spectrale pose divers 
problèmes de passage à la limite, traités avec trop de détailes dans 
\cite[XVIII]{sga4}. Pour tout point géométrique $z$ de $Y_y$, on a 
\[
  (\R^if_! \dZ/n)_z = \h_c^i(f^{-1}(z),\dZ/n) \text{.}
\]
La fibre géométrique $f^{-1}(z)$ est une limite projective de courbes lisses 
sur un corps algébriquement clos. Elle vérifie la dualité de Poincaré. 
Sa cohomologie ordinaire est donnée par le théorème d'acyclicité locale 
pour les morphismes lisses: 
\[
  \h^i(f^{-1}(z),\dZ/n) 
    = \begin{cases}
        \dZ/n & \text{pour $i=0$} \\
        0     & \text{pour $i>0$.}
      \end{cases}
\]
Par dualité, on a 
\[
  \h_c^i(f^{-1}(z),\dZ/n) 
    = \begin{cases}
        \dZ/n(-1) & \text{pour $i=2$} \\
        0         & \text{pour $i\ne 2$.}
      \end{cases}
\]
et la suite spectrale de Leray s'écrit 
\[
  \h_c^i\left(X_x,\dZ/n(N)\right) = \h_c^{i-2}\left(Y_y,\dZ/n(N-1)\right) \text{.}
\]
On conclut par récurrence sur $N$. 










\subsection{Variantes et applications}\label{I:6-4}

On peut construire, en cohomologie étale, un ``formalisme de dualité'' 
($=$ des foncteurs $\R f_!$, $\R f_!$, $\R f^!$ satisfaisant diverses 
compatibilités et formules d'adjonction) parallèle à celui qui existe en 
cohomologie cohérente. Dans ce language, les résultats du paragraphe 
précédents se transcrivent comme suit: si $f:X\to S$ est lisse et purement 
de dimension relative $N$, et que $S=\spec(k)$, avec $k$ algébriquement clos, 
alors 
\[
  \R f^! \dZ/n = \dZ/n [2N](N) \text{.}
\]
Ce résultat vaut sans hypothèse sur $S$. Il admet le 





\begin{corollary}\label{I:6-4-1}
Si $f$ est lisse et purement de dimension relative $N$, et que les faisceaux 
$\R^i f_! \dZ/n$ sont localement constants, alors les faisceaux 
$\R^i f_* \dZ/n$ sont aussi localement constants, et 
\[
  \R^i f_* \dZ/n = \const\hom\left(\R^{2 N-i} f_! \dZ/n(N),\dZ/n\right)\text{.}
\]
\end{corollary}

En particulier, sous les hypothèses du corollaire, les faisceaux 
$\R^i f_* \dZ/n$ sont constructibles. Partant de là, on peut montrer que, si 
$S$ est de type fini sur le spectre d'un corps ou d'un anneau de Dedekind, 
alors, pour tout morphisme de type fini $f:X\to S$ et tout faisceau 
constructible $\sF$ sur $X$, les faisceaux $\R^i f_* \sF$ sont constructibles 
(\nameref{VII}, \ref{VII:1-1}). 
% originally (Th. finitude, 1.1).

% !TEX root = sga4.5.tex

\chapter{Rapport sur la formule des traces}\label{II}

Dans ce texte, j'ai tenté d'exposer de façon aussi directe que possible la 
théorie cohomologique de Grothendieck des fonctions $L$. Je suis de très près 
certains des exposés donnés par Grothendieck à l'IHES au printemps 1966. Dans 
l'esprit de ce volume, il ne sera pas fait appel à SGA 5 -- saui deux 
références à des passages de l'exposé XI, indépendant au reste de ce 
séminaire. 

Dans le \S 10 (p.81-90) de \cite{de73}, le lecteur trouvera un résume de la 
théorie, dans le cas crucial des courbes. 










\section{Quelques rappels}\label{II:1}





\subsection{Notations}\label{II:1-1}

$p,q,\dF,\dF_q$: $p$ est un nombre premier, $q=p^f$ une puissance de $p$ et 
$\dF$ un clôture algébrique du corps premier $\dF_p$. Pour toute puissance 
$p^n$ de $p$, on note $\dF_{p^n}$ le sous-corps à $p^n$ éléments de $\dF$. 

$X_0,X$: On désignera souvent par un symbole affecte d'un indice $0$ un objet 
sur $\dF_q$. Le même symbole, sans $0$, désigne alors l'objet qui s'en déduit 
par extension des scalaires à $\dF$. Par exemple, si $\sF_0$ est un faisceau 
sur un schéma $X_0$ sur $\dF_q$, on note $\sF$ son image réciproque sur 
$X=X_0\otimes_{\dF_q} \dF$. 

$F$: Si $X_0$ est un schéma sur $\dF_q$, on appelle \emph{morphisme de 
Frobenius} et on note $F$ l'endomorphisme de $X_0$ qui ``envoie le point de 
coordonnées $x$ dans celui de coordonnées $x^q$'': c'est l'identifie sur 
l'espace topologique sous-jacent et, pour $x$ une section locale du faisceau 
structurel, $F^* x = x^q$. 

On désigne encore par $F$ l'endomorphisme de $X$ qui s'en déduit par 
extension des scalaires. Sur l'ensemble $X(\dF) = X_0(\dF)$ des points 
rationnels de $X$, $F$ agit comme la substitution de Frobenius 
$\varphi\in\gal(\dF/\dF_q)$ définie par $\varphi(x)=x^q$. En particulier, 
$X^F=X_0(\dF_q)$. 

En cas d'ambiguïté, on écrira $F_{(q)}$ plutôt que $F$. 










\subsection{Correspondance de Frobenius}\label{II:1-2}

Soient $X_0$ un schéma sur $\dF_q$, et $\sF_0$ un faisceau sur $X_0$. J'aime 
voir comme suit la \emph{correspondance de Frobenius} (un $F$-endomorphisme de 
$\sF_0$): 
\begin{equation}\label{II:eq:1-2-1}
  F^* : F^*\sF_0 \iso \sF_0
\end{equation}

On regarde $\sF_0$ comme le faisceau des sections locales d'un espace 
$[\sF_0]$ étale sur $X_0$ ($[\sF_0]$ est un espace algébrique au sens de M. 
Artin). La fonctorialité de $F$ fournit un diagramme commutatif 
\[\xymatrix{
  [\sF_0] \ar[r]^-F \ar[d]
    & [\sF_0] \ar[d] \\
  X_0 \ar[r]^F 
    & X_0
}\]
Parce que $[\sF_0]$ est étale sur $X_0$, ce diagramme est cartésien; il fournit 
un isomorphisme $[\sF_0]\iso F^*[\sF_0]$ d'inverse \eqref{II:eq:1-2-1}. On note 
encore $F^*$ les correspondances ou morphismes déduits de $F$ par extension 
des scalaires ou par fonctorialité, tels 
$F^*:\h_c^\bullet(X,\sF)\to \h_c^\bullet(X,\sF)$. 

Pour une définition en forme, je renvoie à \cite[XI.1,2]{sga5}. On n'y c
considère que le cas où $q=p$, mais le cas général se traite de même. 





\subsection{}\label{II:1-3}

La correspondance de Frobenius est fonctorielle en $(X_0,\sF_0)$. Si 
$f_0:X_0\to Y_0$ est un morphisme de $\dF_q$-schémas, $\sF_0$ un faisceau sur 
$X_0$, $\sG_0$ un faisceau sur $Y_0$ et $u\in\hom(f_0^*\sG_0,\sF_0)$ un 
$f_0$-morphisme de $\sG_0$ dans $\sF_0$, on a un diagramme commutatif d'espaces 
$f_0 F = F f_0$, et au-dessus un diagramme commutatif de faisceaux 
\[\xymatrix{
  X_0 \ar[r]^-{f_0} \ar[d]^-F
    & Y_0 \ar[d]^-F \\
  X_0 \ar[r]^-{f_0} 
    & Y_0
}\qquad\qquad
\xymatrix{
  \sF_0 
    & \ar[l]_-u \sG_0 \\
  \sF_0 \ar[u]_-{F^*} 
    & \ar[l] \sG_0 \ar[u]^-{F^*}
}\]
En particulier, la correspondance sur $f_{0*}\sF_0$ déduite la correspondance 
de Frobenius de $(X_0,\sF_0)$ est la correspondance de Frobenius de 
$(U_0,f_{0*}\sF_0)$. 





\subsection{}\label{II:1-4}

Le morphisme naturel $Y\to Y_0$ est limite de morphismes étales. Il en 
résulte que pour tout faisceau abélien $\sF_0$ sur $X_0$, les images 
réciproques sur $Y$ des $\R^if_0\sF_0$ sont les $\R^if_*\sF$. 

Sur $\R^if_*\sF$, on dispose de 
\begin{enumerate}[\indent a)]
  \item la correspondance 
    $F^*\R^i f_*\sF \iso \R^i f_*\sF \xrightarrow{F^*} \R^if_*\sF$ déduite 
    par fonctorialité de la correspondance Frobenius; 
  \item la correspondance $F^*\R^if_*\sF\to\R^if_*\sF$ image réciproque sur 
    $Y$ de la correspondance de Frobenius sur $\R^ii f_{0*}\sF_0$. 
\end{enumerate}

On déduit de (\ref{II:1-3}) que ces correspondances coïncident. Elles 
induisent la terme initial d'une action de Frobenius sur toute la suite 
spectrale 
\[
  \h^p(Y,\R^q f_*\sF) \Rightarrow \h^{p+q}(X,\sF) \text{.}
\]
De même en cohomologie à support propre, et pour la suite spectrale 
\[
  \h_c^p(Y,\R^q f_* \sF) \Rightarrow \h_c^{p+q}(X,\sF)
\]





\subsection{Changement de corps fini}\label{II:1-5}

Soient $(X_0,\sF_0)$ sur $\dF_q$, $n$ un entier, et $(X_1,\sF_1)$ sur 
$\dF_{q^n}$ déduit de $(X_0,\sF_0)$ par extension des scalaires. On dispose 
d'une part de la correspondance de Frobenius $F_{(q)}:X_0\to X_0)$, 
$F_{(q)}^*:F_{(q)}^*\sF_0\to \sF_0$, d'autre part de 
$F_{(q^n)}:X_1\to X_1$, $F_{(q^n)}^*:F_{(q^n)}^*\sF_1\to\sF_1$. On vérifie 
facilement que $F_{(q^n)}$ se déduit de la puissance $n$-ième de $F_{(q)}$ 
par extension des scalaires de $\dF_q$ à $\dF_{q^n}$; après extension des 
scalaires à $\dF$, on a encore la même identité $F_{(q^n)}=F_{(q)}^n$ 
entre endomorphismes de $X$ et correspondances sur $X$. 

Si $x$ est un point fermé de $X_0$, avec $[k(x),\dF_q]=n$, et que 
$\bar x\in X_0(\dF) = X(\dF)$ est localisé en $x$, $\bar x$ est un point 
fixe de $F_{(q^n)}=F_{(q)}^n$. On note $F_x^*$ l'endomorphisme de 
$\sF_{\bar x}$ induit par $F_{(q^n)}^*$. A isomorphisme près, 
$(F_{\bar x}^*,\sF_{\bar x})$ ne dépend pas du choix de $\bar x$. La trace, 
etc. de $F_x^*$ se désigne par une notation telle que 
$\tr(F_x^*,\sF)$. 





\subsection{Fonction \texorpdfstring{$L$}{L}}\label{II:1-6}

Soit $X_0$ un schéma de type fini sur $\dF_q$. On note $|X_0|$ l'ensemble de 
ses points fermés et, pour $x\in |X_0|$, on pose $\deg(x)=[k(x),\dF_p]$. Si 
$\sF_0$ est un $\dQ_\ell$-faisceau sur $X_0$ (pour cette notion, voir 
ci-dessous) -- ou un faisceau constructible et plat de modules sur un 
anneau noethérien commutatif $A$, on note $L(X_0,\sF_0)$ la série formelle 
de terme constant $1$, dans $\dQ_\ell\llbracket t\rrbracket$ ou dans 
$A\llbracket t\rrbracket$, 
\[
  L(X_0,\sF_0) = \prod_{x\in|X_0|} \det\left(1-F_x^* t^{\deg(x)},\sF\right)^{-1} \text{.}
\]





\subsection{Remarque}\label{II:1-7}

Dans la définition (\ref{II:1-6}), on a $\deg(x)=[k(x),\dF_p]$ (degré 
absolu) et non $\deg(x)=[k(x),\dF_q]$ (degré relatif à $\dF_q$). Ceci à 
pour effet que $L(X_0,\sF_0)$ ne dépend que du schéma $X_0$ et du faisceau 
$\sF_0$, non de $q$ et du morphisme structurel $X_0\to\spec(\dF_q)$. 
L'indéterminée $t$ est un avatar de $p^{-s}$. 





\subsection{Le point de vue galoisien}\label{II:1-8}

Ce n$^\circ$ ne serra pas utilisé dans la suite du rapport. Pour $\sF_0$ un 
faisceau abélien sur $X_0$, le groupe de Galois $\gal(\dF/\dF_q)$ agit sur 
$\dF$, et par transport de structure sur $\h_c^i(X,\sF)$. En particulier, la 
substitution de Frobenius $\varphi$ (\ref{II:1-1}) induit un automorphisme, 
encore noté $\varphi$, de $\h_c^i(X,\sF)$. La correspondance de Frobenius 
définit de son côté un endomorphisme $F^*$ de $\h_c^i(X,\sF)$. On a 
\begin{equation}\label{II:eq:1-8-1}
  {F^*}^{-1} = \varphi \qquad \text{(sur $\h_c^i(X,\sF)$).}
\end{equation}

Pour le voir, appliquons (\ref{II:1-4}) au cas où $Y_0$ est un point: 
$Y_0=\spec(\dF_q)$. On trouve que $\h_c^i(X,\sF)$ est la fibre, au point 
géométrique $\spec(\dF)$ de $Y_0$, de $\R^i f_! \sF_0$, et que $F^*$ est 
la fibre de la correspondance de Frobenius de $\R^i f_!\sF_0$. Soit 
$[\R^i f_! \sF_0]$ comme en (\ref{II:1-2}). On a 
$\h_c^i(X,\sF) = [\R^i f_!\sF_0](\dF)$ et 
\begin{enumerate}[\indent a)]
  \item $\varphi$ agit par transport de structure, via son action sur $\dF$, 
  \item $F^*$ est l'inverse de 
    $F:[\R^i f_!\sF_0](\dF) \to [\R^i f_!\sF_0](\dF)$. 
\end{enumerate}

On conclut par l'identité $F=\varphi$ de (\ref{II:1-1}). 

Un des intérêts du point de vue galoisien est qu'il permet de raisonner par 
transport de structure. 










\section{\texorpdfstring{$\dQ_\ell$}{Ql}-faisceaux}\label{II:2}

La notion de $\dQ_\ell$-faisceaux est développée en détail dans 
\cite[V,VI]{sga5}. Nous n'utiliserons que les définitions et résultats 
ci-dessous. 





\begin{definition_}\label{II:2-1}
Soit $X$ un schéma noethérien. Un $\dZ_\ell$-faisceau $\sF$ sur $X$ est un 
système projectif de faisceaux $\sF_n$ ($n\geqslant 0$), avec $\sF_n$ un 
faisceau de $\dZ/\ell^{n+1}$-modules constructibles \cite[IX.2]{sga4}, et tel que 
le morphisme de transition $\sF_n\to\sF_{n-1}$ se factorise par un isomorphe 
\[
  \sF_n\otimes_{\dZ/\ell^{n+1}} \dZ/\ell^n \iso \sF_{n-1}\text{.}
\]
On dit que $\sF$ est \emph{lisse} si les $\sF_n$ sont localement constants. 
\end{definition_}

La terminologie de \cite{sga5} est différents, on y dit ``$\dZ_\ell$-faisceau 
constructible'' pour ``$\dZ_\ell$-faisceau,'' et ``constant tordu 
constructible'' pour ``lisse.'' 

De même, ci-dessous, pour les $\dQ_\ell$-faisceaux. 





\subsection{}\label{II:2-2}

Si $k$ est un corps séparablement clos, un faisceau de $\dZ/\ell^n$-modules 
sur $\spec(k)$ s'identifie à un $\dZ/\ell^n$-module, et le foncteur 
$\sF\mapsto \varprojlim \sF_n$ identifie les $\dZ_\ell$-faisceaux sur 
$\spec(k)$ sux $\dZ_\ell$-modules de type fini. Ceci justifie le définition 
suivante: la \emph{fibre} du $\dZ_\ell$-faisceau $\sF$ en le point 
géométrique $x$ de $X$ est le $\dZ_\ell$-module $\varprojlim (\sF_n)_x$. 





\subsection{}\label{II:2-3}

Supposons $X$ connexe, et soit $x$ un point géométrique de $X$. On sait que 
le foncteur ``fibre en $x$'' est une équivalence de la catégorie des 
faisceaux localement constants constructibles de $\dZ/\ell^n$-modules avec la 
catégorie des $\dZ/\ell^n$-modules de type fini munis d'une action continue 
de $\pi_1(X,x)$. Par passage à la limite, on en déduit la proposition 
suivante. 





\begin{proposition_}\label{II:2-4}
Sous les hypothèses ci-dessus, le foncteur ``fibre en $x$'' est une 
équivalence de la catégorie des $\dZ_\ell$-faisceaux lisse sur $X$ avec 
celle des $\dZ_\ell$-modules de type fini munis d'une action continue de 
$\pi_1(X,x)$. 
\end{proposition_}





\begin{proposition_}\label{II:2-5}
Soit $\sF$ un $\dZ_\ell$-faisceau sur un schéma noethérien $X$. Il existe 
une partition finie de $X$ en parties localement fermées $X_i$ telles que 
les $\sF|X_i$ soient lisses.
\end{proposition_}

Posons $\gr_\ell^n\sF=\ell^n\sF_m/\ell^{m+1}\sF_m$ pour $m\geqslant n$, et 
$\gr_\ell\sF=\bigoplus \gr_\ell^n\sF$. C'est un faisceau de modules sur 
l'anneau noethérien $\gr_\ell\dZ=\dZ/\ell[t]$, gradué de $\dZ$ pour la 
filtration $\ell$-adique. C'est un quotient de 
$\gr_\ell^0\sF \otimes_{\dZ/\ell} \dZ/\ell[t]$, donc un faisceau de 
$\gr_\ell\dZ$-modules constructible \cite[IX.2]{sga4}. et il existe une partition 
finie de $X$ en parties localement fermées $X_i$, telle que 
$\gr_\ell \sF|X_i$ sont localement constant. Chacun des faisceaux 
$\gr_\ell^n\sF|X_i$ est alors localement constant ainsi que les $\sF_m|X_i$, 
qui en sont des extensions successives. 





\subsection{}\label{II:2-6}

Si $\sF$ et $\sG$ sont deux $\dZ_\ell$-faisceaux sur $X$, on pose 
$\hom(\sF,\sG) = \varprojlim(\sF_n,\sG_n)$; c'est un $\dZ_\ell$-module. 
Puisque $\hom(\sF_n,\sG_n) = \hom(\sF_m,\sG_n)$ pour $m\geqslant n$, on a 
encore 
\[
  \hom(\sF,\sG) = \varprojlim_n \varinjlim_m \hom(\sF_m,\sG_n) \text{,}
\]
de sorte que le foncteur $\sF\mapsto \text{``}\varprojlim\text{''}\sF_n$ est 
pleinement fidèle, de la catégorie des $\dZ_\ell$-faisceaux dans celle 
pro-faisceaux sur $X$ ($=$ pro-objets de la catégorie des faisceaux sur $X$). 
On l'utilisera pour identifier les $\dZ_\ell$-faisceaux à certains 
pro-faisceaux (ceux tels que chaque $\sF_n=\sF/\ell^{n+1}\sF$ soit un faisceau, 
et que $\sF\iso \text{``}\varprojlim\text{''}\sF_n$). On vérifie aisément 
qui si $\sF+\text{``}\varprojlim\text{''}\sF_n$, avec $\sF_n$ un faisceau 
constructible de $\dZ/\ell^n$-modules, pour que $\sF$ soit un 
$\dZ_\ell$-faisceau, il faut et il suffit que les systèmes projectifs (en $n$) 
$\sF_n/\ell^{k+1}\sF_n$ soient essentiellement constants: le système 
projectif des $\sF_k'=\varprojlim_n\sF_n/\ell^{k+1}\sF_n$ vérifie alors les 
conditions de (\ref{II:2-1}), et $\sF=\text{``}\varprojlim\text{''}\sF_k'$. 

Dans ce qui suit, nous abandonnons la notation $\sF_n$ de (\ref{II:2-1}); si 
$\sF$ est un $\dZ_\ell$-faisceau, le faisceau $\sF_n$ de (\ref{II:2-1}) sera 
noté $\sF\otimes\dZ/\ell^{n+1}$. 





\subsection{}\label{II:2-7}

Dans la catégorie abélienne des pro-faisceaux, la sous-catégorie des 
$\dZ_\ell$-faisceaux est stable par noyau, conoyau et extension. C'est claire 
pour les conoyaux. Pour les noyaux, si $f:\sF\to\sG$ est un morphisme de 
$\dZ_\ell$-faisceaux, on déduit aisément de (\ref{II:2-4}, \ref{II:2-5}) 
et de Artin-Rees que le système projectif des 
$\ker(f:\sF\otimes\dZ/\ell^n\to\sG\otimes\dZ/\ell^n)$ vérifie le critère 
donné et (\ref{II:2-6}) ci-dessus. Il existe même un entier $r$ tel que pour 
tout $n$ 
\[
  \ker(f)\otimes\dZ/\ell^n = \ker(f:\sF\otimes\dZ/\ell^{n+r}\to\sG\otimes\dZ/\ell^{n+r})\otimes\dZ/\ell^n \text{.}
\]
Le cas des extensions est laissé au lecteur. 





\subsection{}\label{II:2-8}

Un $\dZ_\ell$-faisceau $\sF$ est \emph{sans torsion} si 
$\ker(\ell:\sF\to\sF) = 0$. Chaque $\sF\otimes\dZ/\ell^n$ est alors plat sur 
$\dZ/\ell^n$ (i.e. à fibres des $\dZ/\ell^n$-modules libres de type fini). On 
déduit de (\ref{II:2-4}, \ref{II:2-5}) que pour tout $\dZ_\ell$-faisceau 
$\sF$ il existe un entier $n$ tel que $\sF/\ker(\ell^n)$ soit sans torsion. 





\subsection{}\label{II:2-9}

La catégorie abélienne des \emph{$\dQ_\ell$-faisceaux} sur $X$ est le 
quotient de celle des $\dZ_\ell$-faisceaux par la sous-catégorie de 
$\dZ_\ell$-faisceaux de torsion. En termes équivalents, ses objets sont les 
$\dZ_\ell$-faisceaux et, notant $\sF\otimes\dZ_\ell$ le $\dZ_\ell$-faisceaux 
$\sF$, vu comme objet de le catégorie des $\dQ_\ell$-faisceaux, on a 
\[
  \hom(\sF\otimes\dQ_\ell,\sG\otimes\dQ_\ell) = \hom(\sF,\sG)\otimes_{\dZ_\ell}\dQ_\ell \text{.}
\]

La \emph{fibre} d'un $\dQ_\ell$-faisceau $\sF\otimes\dQ_\ell$ en un point 
géométrique $x$ de $X$ est la $\dQ_\ell$-espace vectoriel de dimension 
finie $\sF_x\otimes_{\dZ_\ell}\dQ_\ell$. 





\subsection{}\label{II:2-10}

Soient $X$ un schéma séparé de type fini sur un corps algébriquement 
clos $k$ et $\sF$ un $\dQ_\ell$-faisceau constructible sur $X$. Soit $\sF'$ un 
$\dZ_\ell$-faisceau constructible tel que 
$\sF\sim \sF'\otimes_{\dZ_\ell}\dQ_\ell$. Nous verrons en (\ref{II:4-11}) que 
\[
  \left(\varprojlim \h_c^q(X,\sF'\otimes \dZ/\ell^n)\right)\otimes_{\dZ_\ell}\dQ_\ell
\]
est un $\dQ_\ell$-espace vectoriel de dimension finie. Il ne dépend que de 
$\sF'$, et se note $\h_c^q(X,\sF)$. La preuve n'utilise pas que $\ell$ soit 
premier à la caractéristique, mais seul ce cas nous intéresse. 





\subsection{}\label{II:2-11}

Dans cet exposé, tous les calculs importants se feront au niveau des 
$\dZ/\ell^n$-faisceaux, le passage aux $\dQ_\ell$-faisceaux ne se faisant qu'en 
tout dernier lieu. En pratique, il est toutefois souvent commode de travailler 
systématiquement avec des $\dQ_\ell$-faisceaux. Dans les exposés 
V et VI de \cite{sga5}, Jouenolou donne le formalisme qui rend cela possible. 
Un des résultats essentiels (VI 2.2) est que si $f:X\to Y$ est un morphisme 
séparé de type fini de schémas noethériens, et $\sF$ un 
$\dZ_\ell$-faisceau sur $Y$, alors, pour chaque $q$, le pro-faisceau 
$\text{``}\varprojlim\text{''}\R^q f_! (\sF\otimes\dZ/\ell^n)$ est un 
$\dZ_\ell$-faisceau sur $Y$. Il démontrer même une propriété de 
régularité plus forte (à la Artin-Rees) pour le système projectif des 
$\R^q f_!(\sF\otimes\dZ/\ell^n)$. 

Jouanolou se place même dans un cadre plus général, englobant le cas 
très utile des $E_\lambda$-faisceau, pour $E_\lambda$ une extension finie de 
$\dQ_\ell$. Ils sont définis comme l'ont été les $\dQ_\ell$-faisceaux, 
avec $\dQ_\ell$ remplacé par $E_\lambda$, $\dZ_\ell$ par l'anneau 
$\sO_\lambda$ de la valuation $\lambda$ et $\dZ/\ell^n$ par 
$\sO_\lambda/\pi^n$, pour $\pi$ une uniformisante. Il reviendrait au même de 
définir un $E_\lambda$-faisceau comme un $\dQ_\ell$-faisceau $\sF$, muni 
d'une homomorphisme $E_\lambda\to\operatorname{End}(\sF)$. 










\section{Formule des traces et fonctions \texorpdfstring{$L$}{L}}\label{II:3}

\emph{On utilise les notations du \S\ref{II:1}, et $\ell$ est un nombre premier 
$\ne p$.}

Soient $X_0$ un schéma séparé de type fini sur $\dF_q$ ($q=p^f$) et 
$\sF_0$ un $\dQ_\ell$-faisceau constructible sur $X_0$. L'interprétation 
cohomologique de Grothendieck des fonctions $L$ est le théorème suivant: 





\begin{theorem_}\label{II:3-1}
$L(X_0,\sF_0) = \prod_i \det\left(1-F^* t^f, \h_c^i(X,\sF)\right)^{(-1)^{i+1}}$. 
\end{theorem_}

Il est obtenu comme conséquence de la formule des traces:





\begin{theorem_}\label{II:3-2}
Pour tout entier $n$, 
$\sum_{x\in X^{F^n}} \tr\left({F^n}^*,\sF_x\right) = \sum_i (-1)^i \tr\left({F^n}^*,\h_c^i(X,\sF)\right)$. 
\end{theorem_}





Rappelons brièvement la méthode classique pour déduire (\ref{II:3-1}) 
de (\ref{II:3-2}). Posons $T=t^f$; les deux membres sont des séries formelles 
en $T$, de terme constant $1$. Puisque $\dQ_\ell$ est de caractéristique 
$0$, il suffit de prouver que les dérivées logarithmiques 
$T\frac{d}{dT}\log$ des deux membres sont égales. 

Pour calculer ces dérivées logarithmiques, nous utiliserons la formule 
\[
  T \frac{d}{dT}\log\det(1-f T)^{-1} = \sum_{n>0} \tr(f^n) T^n \text{,}
\]
% NOTE: the following lines are not typesetting correctly
dont la démonstration (dans un cadre un peu plus général) est rappelée 
ci-dessous. Au premier nombre, on trouva 
\begin{align*}
  T\frac{d}{dT}\log L(X_0,\sF_0) 
    &= \sum_{x\in |X_0|} \sum_n [k(x):\dF_q] \tr\left(F_x^n,\sF\right) T^{n\cdot[k(x):\dF_q]} \\
    &= \sum_{n>0} T^n \sum_{x\in X^{F^n}} \tr\left({F^n}^*,\sF_x\right)
\end{align*}
et su second membre 
\[
  \sum_{n>0} T^n \sum (-1)^i \tr\left({F^n}^*,\h_c^i(X,\sF)\right) \text{ ; }
\]
on compare terme à terme. 





\begin{proposition_}\label{II:3-3}
Soient $A$ un anneau commutatif, $M$ un $A$-module projectif de type fini et 
$f$ un endomorphisme de $M$. Pour tout série formelle inversible $Q$, on pose 
$\frac{d}{dT}\log Q = Q^{-1}\frac{d}{dT}Q$. On a 
\begin{equation}\label{II:eq:3-3-1}
  T\frac{d}{dT}\log\det(1-f T)^{-1} = \sum_{n>0} \tr(f^n) T^n \text{.}
\end{equation}
\end{proposition_}

Soit $M'$ un $A$-module tel que $M\oplus M'$ soit libre de rang fini. 
Remplacer $(M,f)$ par $(M\oplus M',f\oplus 0)$ ne change pas les deux membres 
de (\ref{II:3-3}), et nous ramène au cas où $M$ est libre. Pour $M=A^d$, 
\eqref{II:eq:3-3-1} est une identité algébrique portant sur les coefficients 
$f_i^j$ de $f$. Par le principe de prolongement des identités algébriques, 
il suffit de la vérifier pour $A$ un corps algébriquement clos de 
caractéristique $0$. Les deux membres étant additifs en $M$ dans une 
extension $0\to M'\to M\to M''\to 0$, il suffit même de ne traiter que le cas 
où $M$ est de rang $1$. Si $f$ est la multiplication par $a$, 
\eqref{II:eq:3-3-1} se réduit alors à la formule 
\[
  T\frac{d}{dT}\log\left((1-a T)^{-1}\right) = \sum_{n>0} a^n T^n \text{.}
\]





\begin{corollary_}\label{II:3-4}
Soient $n$ un entier, et $f^{(n)}$ l'endomorphisme 
$(x_1,\dotsc,x_n)\mapsto \left(f(x_n),x_1,\dotsc,x_{n-1}\right)$ de $M^n$. 
On a 
\begin{equation}\label{II:eq:3-4-1}
  \det(1-f T^n) = \det(1-f^{(n)} T) \text{.}
\end{equation}
\end{corollary_}

Procédant comme en (\ref{II:3-3}), on se ramène à supposer que $A$ est de 
caractéristique $0$. Il suffit alors de vérifier que les dérivées 
logarithmiques $-T\frac{d}{dT}\log$ des deux membres sont égales. 
\begin{align*}
  -T\frac{d}{dT}\log\det(1-f T^n) &= - n T^n \frac{d}{dT^n} \log\det(1-f T^n) = n \sum_{m>0} \tr(f^m) T^{n m} \text{,} \\
  -T \frac{d}{dT}\log\det\left(1-f^{(n)} T\right) &= \sum_{m>0} \tr\left(f^{(n) m}\right) T^m \text{,}
\end{align*}
et on observe que 
\begin{align*}
  \tr(f^{(n) m}) &= 0 \qquad \text{si $n\nmid m$} \\
  \tr(f^{(n) n m}) &= n \tr(f^m) \text{.}
\end{align*}





\subsection{Remarque}\label{II:3-5}

Pour $A=\dQ_\ell$, \eqref{II:eq:3-4-1} est le cas particulier 
$X_0=\spec(\dF_{q^n})$ de (\ref{II:3-1}). Utilisant cette identité, il est 
facile de vérifier directement que le second membre de (\ref{II:3-1}) est 
indépendant de $q$, et du morphisme structural $X_0\to\spec(\dF_q)$. 





\subsection{Remarque}\label{II:3-6}

Si $j:X_0\hookrightarrow \bar X_0$ est une compactification de $X_0$, on a 
$L(X_0,\sF_0) = L(\bar X_0, j_!\sF_0)$ et 
$\h_c^i(X,\sF) = \h^i(\bar X,j_!\sF)$. Ceci ramène (\ref{II:3-1}) au cas 
propre, et explique l'usage de la cohomologie à supports propres. Pour $X$ 
propre, (\ref{II:3-2}) a la forme d'une formule des traces de Lefschetz. On 
notera que les termes locaux $\tr(F^{n *},\sF_x)$ ($x\in X^{F^n}$) ne 
dépendent que de la fibre de $\sF$ en $x$, et ne sont pas affectés de 
multiplicité. Pour $X$ et $\sF$ lisses, cela correspond au fait que le graphe 
de $F$ est transverse à la diagonale (puisque $d F = 0$). 





\subsection{Remarque}\label{II:3-7}

Si on admet le formalisme de $\dQ_\ell$-faisceaux (cf. \ref{II:2-11}; il faut 
disposer de la suite spectrale de Leray et du théorème de changement de 
base pour les images directes supérieures $\R^q f_!$), il est facile de 
ramener la preuve de (\ref{II:3-1}, \ref{II:3-2}), au cas où $X_0$ est 
une courbe lisse et où $\sF_0$ est lisse. Ceci est clairement expliqué 
dans \cite[\S 5]{gr95} (pour \ref{II:3-1}; \ref{II:3-2} se traite de même). 





\subsection{}\label{II:3-8}

On dispose de deux méthodes pour prouver (\ref{II:3-2}). 


\paragraph{Lefschetz-Verdier}
Si $X_0$ est propre (cf. \ref{II:3-6}), la formule des traces générale de 
Lefschetz-Verdier permet d'exprimer le second membre de (\ref{II:3-2}) comme 
une somme de termes locaux, un pour chaque point de $X^{F^n}$. Dans la version 
originale de \cite{sga5}, cette formule n'était prouvée que modulo la 
résolution des singularités. Le lecteur trouvera une preuve inconditionnelle 
dans la version définitive. Dans le cas des courbes, cas auquel on peut se 
réduire (\ref{II:3-7}), les ingrédients de la démonstration étaient 
d'ailleurs tous disponibles. 

Pour déduire (\ref{II:3-2}) de la formule de Lefschetz-Verdier, il faut 
pouvoir en calculer les termes locaux. Pour une courbe et l'endomorphisme de 
Frobenius, cela avait été fait par Artin et Verdier \cite{ve67}. 


\paragraph{Nielson-Wecken}
Un méthode inspirée dex travaux de Nielsen et Wecken permet de ramener 
(\ref{II:3-2}), à un cas particulier prouvé par Weil; c'est ce qui sera 
expliqué dans les paragraphes suivants. 










\section{Réduction à des théorèmes en \texorpdfstring{$\dZ/\ell^n$}{Z/l n}-cohomologie}\label{II:4}

Nous prouverons (\ref{II:3-2}) par passage à la limite à partir d'une 
théorème analogue en $\dZ/\ell^n$-cohomologie. L'énoncer présente la 
difficulté suivante. Soient $X_0$ un schéma séparé de type fini sur 
$\dF_q$, et $\sF_0$ un faisceau plat et constructible de $\dZ/\ell^n$-modules. 
Ses fibres sont des $\dZ/\ell^n$-modules libres de type fini, et le membre de 
gauche de (\ref{II:3-2}) a donc un sens (c'est un élément de $\dZ/\ell^n$). 
Par contre, les $\h_c^i(X,\sF)$ ne seront en général pas libres, de sorte 
que le membre de droite n'est pas défini. Obvier à cette difficulté est 
l'un des usages essentiels faits par Grothendieck de la théorie des 
catégories dérivées.





\subsection{}\label{II:4-1}

Pour la définition de la catégorie dérivée $\D(\cA)$ d'une catégorie 
abélienne $\cA$, et celle des sous-catégorie $\D^+(\cA)$, $\D^-(\cA)$, 
$\D^b(\cA)$, je renvoie au texte de Verdier dans ce volume. Pour $\Lambda$ 
un anneau, on écrit $\D(\Lambda)$ au lieu de 
$\D(\text{catégorie des $\Lambda$-modules à gauche})$; pour $X$ un site, on 
écrit $\D(X,\Lambda)$ au lieu de 
$\D($catégorie des faisceaux de $\Lambda$-modules à gauche$)$. De 
même pour $\D^-$, etc\ldots

Le signe $\R$ (resp. $\mathsf{L}$) indique un foncteur dérivé droit (resp. 
gauche). Par exemple, $\lotimes$ indique le foncteur dérivé du produit 
tensoriel: pour $K$ dans $\D^-(\Lambda^\circ)$ et $L$ dans $\D^-(\Lambda)$ 
($\Lambda^\circ$ anneau opposé à $\Lambda$), $K\lotimes L$ se calcule 
comme suit: on prend des quasi-isomorphismes $K'\to K$, $L'\to L$, avec 
$K'$ et $L$ bornés supérieurement et l'un d'eux à composantes plates, et 
le simple complexe $s(K'\otimes_\Lambda L)$ associé au complexe double 
$K'\otimes_\Lambda L'$. 





\subsection{}\label{II:4-2}

Même pour traiter le seul cas de la $\dZ/\ell^n$-cohomologie, nous aurons 
besoin de la théorie des traces d'endomorphismes de modules sur des anneaux 
non nécessairement commutatifs (des algèbres de groupes finis 
$\dZ/\ell^n[G]$). Ces traces ont été introduits par Stallings et Hattori 
\cite{ba76}. 

Soit $\Lambda$ un anneau (à unité, mais non nécessairement commutatif). 
On par $\Lambda^\natural$ le quotient du groupe additif de $\Lambda$ par le 
sous-groupe engendré par les commutateurs $a b - b a$. Si $f$ est un 
endomorphisme du $\Lambda$-module à gauche libre $\Lambda^n$, de matrice 
$f_i^j$, on note $\tr(f)$ l'image de $\sum f_i^f$ dans $\Lambda^\natural$. 
Si $f$ et $g$ sont deux homomorphismes 
$f:\Lambda^n \leftrightarrows \Lambda^m : g$, on vérifie aussitôt que 
$\tr(f g) = \tr(g f)$. 

Soit $f$ un endomorphisme d'un $\Lambda$-module projectif de type fini $P$. 
Écrivons $P$ comme facteur direct d'une module libre $\Lambda^n$: cela signifie 
choisir un module $P'$, et un isomorphisme $\alpha:P\oplus P'\iso \Lambda^n$, 
soit encore choisir un homomorphisme $a:P\to \Lambda^n$, et une rétraction 
$b:\Lambda^n\to P$ (prendre $P=\ker(b)$). Soit 
$f'=\alpha(f\oplus 0)\alpha^{-1} =af b$. On pose $\tr(f) = \tr(f')$. Cet 
élément de $\Lambda^\natural$ ne dépend que de $f$: si un autre choix 
$c:P\leftrightarrows \Lambda^m:d$ est fait, on a $a=a(dc)(ba)$, d'où 
\[
 \tr(a f b) = \tr(adcbafb) = \tr(cbafbad) = \tr(cfd) \text{.}
\]

Si $f$ est un endomorphisme d'un $\Lambda$-module projectif de type fini 
$\dZ/2$-gradué, de composantes $f_i^j:P^j\to P^i$, on pose 
\[
  \tr(f) = \tr(f_0^0) - \tr(f_1^1) \text{.}
\]
S'il y a risque d'ambiguïté, on écrira plutôt $\tr(f;P)$. Si les 
homomorphismes $f:P\leftrightarrows Q:g$ sont homogènes, on vérifie 
aussitôt par réduction au cas libre que 
\begin{equation}\label{II:eq:4-2-1}
  \tr(f g) = \tr\left((-1)^{\deg f \deg g} g f\right) \text{.}
\end{equation}

Si $f$ est un endomorphisme d'un $\Lambda$-module $\dZ/2$-gradué filtré 
$(P,F)$, de filtration finie, compatible à la graduation, et que les 
$\gr_F^i(P)$ sont projectifs de type fini, alors $P$ est projectif de type 
fini, et 
\begin{equation}\label{II:eq:4-2-2}
  \tr(f;P) = \sum \tr\left(f; \gr_F^i(P) \right)
\end{equation}
(scinder la filtration pour se ramener au cas d'une somme directe). 





\subsection{}\label{II:4-3}

Un module $\dZ$-gradué sera considéré comme $\dZ/2$-gradué par la 
parité du degré. En particulier, si $f$ est un endomorphisme d'un complexe 
borné de $\Lambda$-modules projectifs de type fini, on pose 
\begin{equation}\label{II:eq:4-3-1}
  \tr(f) = \sum (-1)^i \tr(f^i) \text{.}
\end{equation}

D'après \eqref{II:eq:4-2-1}, si $f$ est homotope à zéro, $\tr(f) = 0$: 
\begin{equation}\label{II:eq:4-3-2}
  f = d H + h D \Rightarrow \tr(f) = 0 \text{.}
\end{equation}

Soit $\mathsf{K}_\text{parf}(\Lambda)$ la catégorie d'objets les complexes 
bornés de $\Lambda$-modules projectifs de type fini, et de flèches les 
morphismes de complexes pris à homotopie près. Le foncteur 
$\mathsf{K}_\text{parf}(\Lambda)\to\D(\Lambda)$ est pleinement fidèle. On note 
$\mathsf{D}_\text{parf}(\Lambda)$ son image essentielle. Les formules 
\eqref{II:eq:4-3-1} \eqref{II:eq:4-3-2} permettent de définir $\tr(f)$ pour $f$ 
un endomorphisme de $K\in\ob \mathsf{D}_\text{parf}(\Lambda)$ (et cette trace ne 
dépend que de la classe d'isomorphe de $(K,f)$). 





\subsection{}\label{II:4-4}

Rappelons que la catégorie dérivée filtrée $\mathsf{DF}(\cA)$ d'une catégorie 
abélienne $\cA$ est la catégorie déduite de celle des complexes d'objets 
filtrés de filtration finie de $\cA$ en inversant les quasi-isomorphismes 
filtrés ($=$ morphismes de complexes filtrés induisant un quasi-isomorphisme 
sur le gradué associé). On dispose de foncteurs ``complexe sous-jacent'': 
$\mathsf{DF}(\cA)\to\D(\cA): (K,F)\mapsto K$ et ``$p$-ième composante du 
gradué'': $\mathsf{DF}(\cA)\to\D(\cA): (K,F)\mapsto \gr_F^p(K)$. 

Soit $\mathsf{KF}_\text{parf}(\Lambda)$ la catégorie d'objets les complexes 
bornés filtrés de filtration finie $(K,F)$ avec des $\gr_F^p(K^q)$ projectifs 
de type fini, et de flèches les morphismes de complexes respectant la 
filtration, pris à une homotopie respectant la filtration près. Le foncteur 
$\mathsf{KF}_\text{parf}(\Lambda)\to\mathsf{DF}(\Lambda)$ est pleinement fidèle; 
son image essentielle $\mathsf{DF}_\text{parf}(\Lambda)$ est presque tous nuls; 
si $(K,F)$ est dans $\mathsf{D}_\text{parf}(\Lambda)$, alors $K$ est dans 
$\mathsf{D}_\text{parf}(\Lambda)$. 

D'après \eqref{II:eq:4-2-2}, si $f$ est un endomorphisme de 
$(K,F)\in\ob\mathsf{DF}_\text{parf}(\Lambda)$, on a 
\begin{equation}\label{II:eq:4-4-1}
 \tr(f,K) = \sum_p \tr(f,\gr_F^p(K)) \text{.}
\end{equation}





\subsection{}\label{II:4-5}

Soient $\Lambda$ un anneau, et $K\in\ob\D^-(\Lambda)$. On dit que $K$ est de 
tor-dimension $\leqslant r$ si pour tout $\Lambda$-module à droite $N$, les 
hypertors $\h^i(N\lotimes_\Lambda K)$ sont nuls pour $i<-r$. Il est dit de 
tor-dimension finie si cette condition est vérifiée pour un $r$. 

On emploie la même terminologie pour un complexe de faisceaux 
$K\in\ob\D^-(X,\Lambda)$. 





\begin{lemma}\label{II:4-5-1} 
Soient $\Lambda$ un anneau noethérien à gauche, et $K\in\ob\D^-(\Lambda)$. 
Pour que $K$ soit dans $\mathsf{DF}_\text{parf}(\Lambda)$, il faut et il suffit 
qu'il soit de tor-dimension finie et que les $\h^i(K)$ soient de type fini.
\end{lemma}

Quitte à remplacer $K$ par un complexe isomorphe (dans $\D^-(\Lambda)$), on 
peut supposer $K$ borné supérieurement et à composantes libres de type 
fini (cf. \ref{II:4-7}, ci-dessous). Si $K$ est de tor-dimension $\leqslant r$, 
les $\h^i(K)$ sont nuls pour $i<-r$, $K^{-r}/\im(d)$ admet la résolution 
libre 
\[
  (\cdots \to K^{-r-1} \to K^{-r}) \to K^{-r}/\im(d) \text{,}
\]
et pour tout $\Lambda$-module à droite $N$, 
$\h^{-r-k}(N\lotimes_\Lambda K) = \h^{-r-k}(N\otimes_\Lambda K) 
= \tor_k(N,K^r/\im d)$ pour $k\geqslant 1$. 

Par hypothèse, ces groupes sont nuls; le module $K^r/\im d$ est donc plat de 
présentation finie, i.e. projectif de type fini. Le quotient 
\[
  0 \to K^r/\im(d) \to K^{r+1} \to K^{r+2}\to \cdots 
\]
de $K$ est quasi-isomorphe à $K$, et ceci montre que 
$K\in\ob\mathsf{DF}_\text{parf}(\Lambda)$. 





\begin{prop-def_}\label{II:4-6}
Soient $X$ un schéma noethérien, et $\Lambda$ un anneau noethérien à 
gauche. On note $\D_\textnormal{ctf}^b(X,\Lambda)$ la sous-catégorie de 
$\D^-(X,\Lambda)$ formée des complexes $\sK$ vérifient les conditions 
équivalentes suivantes:
\begin{enumerate}[\indent i)]
  \item $\sK$ est isomorphe (dans $\D^-(X,\Lambda)$) à un complexe borné e 
    faisceaux de $\Lambda$-modules plats et constructibles. 
  \item $\sK$ est de tor-dimension finie et les faisceaux $\h^i(K)$ sont 
    constructibles. 
\end{enumerate}
\end{prop-def_}

Même argument qu'en (\ref{II:4-5}), à partir du lemme suivant. 





\begin{lemma_}\label{II:4-7}
Si un complexe $\sK$ de faisceaux de $\Lambda$-modules est tel que les 
$\h^i(\sK))$ soient constructibles, et nuls pour $i$ grand, il existe un 
quasi-isomorphisme $\sK'\to \sK$, avec $\sK'$ borné supérieurement à 
composantes constructibles et plates.
\end{lemma_}

On construit $\sK'$ en se propageant de droite à gauche. Pour un $m$ tel que les 
$\h^i(\sK)$ ($i\geqslant m$) soient nuls, on prend ${\sK'}^i = 0$ pour 
$i\geqslant m$. Supposons ensuite déjà construits les ${\sK'}^i$ pour 
$i>n$:
\[\xymatrix{
  \cdots \ar[r] 
    & \sK^{n-1} \ar[r] 
    & \sK^n \ar[r] 
    & \sK^{n+1} \ar[r] 
    & \cdots \\
  & & & {\sK'}^{n+1} \ar[u] \ar[r] 
    & \cdots \text{,}
}\]
avec $\h^i(\sK')\iso \h^i(\sK)$ pour $i>n+1$, et $\ker(d)\to\h^{n+1}(\sK)$ surjectif 
en degré $n+1$. Définissons des faisceaux $\sA$ et $\sB$ par le diagramme 
cartésien 
\[\xymatrix{
  \sK^n \ar[r] 
    & \sK^n/\im(d) \ar[r] 
    & \ker(d) \\
  \sA \ar[u] \ar[r]^u 
    & \sB \ar[u]\ar[r] 
    & \ker(d) \ar[u] \text{.}
}\]

Alors, $\sB$ est constructible, $u$ est surjectif, et pour avancer d'un pas, il 
suffit de construire $K^n$ plat et constructible et $v:\sK^n\to \sA$ tel que u
$u v$ soit surjectif. Que ce soit possible est garanti par le lemme suivant. 





\begin{lemma_}\label{II:4-8}
Soit $v:\sA\to \sB$ un épimorphisme de faisceaux de $\Lambda$-modules, avec 
$\sB$ constructible. Il existe alors $u:\sK\to \sA$ avec $\sK$ plat et 
constructible et $v u$ un épimorphisme. 
\end{lemma_}

Le faisceau $\sA$ est quotient d'une somme de faisceaux de la forme 
$\varphi_! \Lambda$, pour $\varphi:U\to X$ étale. Puisque $\sB$ est noethérien 
(\cite[IX.2.10]{sga4}), une somme finie de ceux-ci s'envoie déjà sur $\sB$, 
d'où le lemme. 





\begin{theorem_}\label{II:4-9}
Soit $f:X\to Y$ un morphisme séparé de type fini de schémas noethériens. 
Si $\sK\in\ob\D_\textnormal{ctf}^b(X,\Lambda)$, alors 
$\R f_!\sK\in \ob\D_\textnormal{ctf}^b(Y,\Lambda)$. 
\end{theorem_}

Rappelons que si $j:X\hookrightarrow \bar X \xrightarrow{\bar f} Y$ est une 
compactification relative de $X/Y$, on a par définition 
$\R f_! = \R \bar f_* \circ j_!$, où $\R \bar f_*$ est le foncteur dérivé 
du foncteur $\bar f_*$. Cette formule ramène la preuve de (\ref{II:4-9}) 
au cas où $f$ est propre. Pour $f$ propre, $f_*$ est de dimension 
cohomologique finie; ceci permet de définir $\R f_*$ sur la catégorie 
dérivée tout entière, et aussi comme un foncteur 
\[
  \R f_* : \D^-(X,\Lambda) \to \D^-(Y,\Lambda) \text{.}
\]
Par composition, on définit de même en général 
\[
  \R f_! : \D^-(X,\Lambda) \to \D^-(Y,\Lambda) \text{.}
\]

Pour tout $\Lambda$-module à droite $N$, on a (preuve donnée ci-dessous) 
\begin{equation}\label{II:eq:4-9-1}
  N\lotimes \R f_! \sK \iso \R f_! (N\lotimes \sK) \text{.}
\end{equation}

Déduisons (\ref{II:4-9}) de \eqref{II:eq:4-9-1}. 
\begin{enumerate}[\indent a)]
  \item La suite spectrale 
    \[
      E_2^{p q} = \R^p f_! \h^q(\sK) \Rightarrow \h^{p+q} \R f_! \sK \text{.}
    \]
    et le théorème de finitude pour $\R f_!$, montrent que 
    $\R f_!\sK\in \ob \D^b(X,\Lambda)$ et est à cohomologie constructible. 
    Reste à prouver la tor-dimension finie (\ref{II:4-6}). 
  \item Si $\sK$ est de tor-dimension $\leqslant -r$, il en est de même pour 
    $\R f_! \sK$: on a $\h^i(N\lotimes \R f_! \sK) = \h^i(\R f_!(N\lotimes \sK))$, 
    et, dans la suite spectrale ci-dessus pour $N\lotimes \sK$, les 
    $E_2^{p q}$ sont nuls pour $q<r-s$ fortiori pour $p+q<r$. 
\end{enumerate}

Prouvons \eqref{II:eq:4-9-1}, en restant au niveau des complexes. 
\begin{enumerate}[\indent a)]
  \item Utilisant la formule de définition $\R f_! = \R f_*\circ j_!$, on se 
    ramène à supposer $f$ propre. 
  \item Représentons $\sK$ par un complexe borné 
    supérieurement à composantes acyclique pour $f_*:\R^p f_* \sK^q = 0$ 
    pour $p>0$. On a alors $\R f_* K \sim f_* \sK$. Il n'est possible de 
    travailler ainsi avec des complexes bornés supérieurement que parce que 
    $f_*$ est de dimension cohomologique finie. 
  \item Si $L$ un $\Lambda$-module à droite libre, on a 
    $\R^p f_*(L\otimes \sK^q) = L\otimes \R^p f_* \sK^q$: c'est trivial pour 
    $\Lambda$ libre de type fini, et le cas général s'en déduit par 
    passage à la limite. Le faisceau $L\otimes \sK^q$ est acyclique pour 
    $f_*$, et $f_*(L\otimes \sK^q) = L\otimes f_* \sK^q$. 
  \item Soit $N_\bullet$ un résolution libre de $N$. On a 
    $N\lotimes \sK\sim s(N_\bullet\otimes \sK)$ (complexe simple associé au 
    complexe double $N_\bullet\otimes \sK$). D'après c), 
    \[
      \R f_*(N\lotimes \sK) \sim \R f_* (s(N_\bullet\otimes \sK)) \sim f_*(s(N_\bullet\otimes \sK)) \sim s(N_\bullet\otimes f_* \sK) \sim N\lotimes \R f_* \sK \text{,}
    \]
    soit la formule annoncée. 
\end{enumerate}

Lorsque $Y$ est le spectre d'un corps séparablement clos, le foncteur $\R f_!$ 
s'identifie à un foncteur noté 
$\R \Gamma_c : \D_\text{ctf}^b(X,\Lambda)\to\D_\text{parf}(\Lambda)$. 





\begin{theorem_}\label{II:4-10}
Avec les notations du \S\ref{II:1}, soient $X_0$ un schéma séparé de type 
fini sur $\dF_q$, et $\Lambda$ un anneau noethérien de torsion, tué par un 
entier premier à $p$. Soit $\sK_0\in\ob\D_\textnormal{ctf}^b(X_0,\Lambda)$. 
Avec la définition (\ref{II:4-3}) de traces, on a pour tout $N$ 
\begin{equation}\label{II:eq:4-10}
  \sum_{x\in X^{F^n}} \tr\left({F^n}^*,\sK_x\right) = \tr\left({F^n}^*,\R\Gamma_c(X,\sK)\right) \text{.}
\end{equation}
\end{theorem_}

Dans la fin de ce \S, nous allons déduire \ref{II:3-2} de \ref{II:4-10}, et 
prouver \ref{II:2-10}. 





\subsection{Preuve de \texorpdfstring{\ref{II:2-10}}{2.10}}\label{II:4-11}

Soit $X$ un schéma séparé de type fini sur un corps algébriquement clos 
$k$, et soit $\sG$ un $\dQ_\ell$-faisceau sur $X$. D'après \ref{II:2-8} on 
peut l'écrire sous la forme $\sG=\sF\otimes\dQ_\ell$, avec $\sF$ un 
$\dZ_\ell$-faisceau sans torsion. Posons 
\[
  K_n = \R \Gamma_c(X,\sF\otimes\dZ/\ell^{n+1})\in\D(\dZ/\ell^{n+1}) \text{.}
\]
D'après \ref{II:4-9}, on a $K_n\in\ob\D_\text{parf}(\dZ/\ell^{n+1})$, et il 
résulte de \eqref{II:eq:4-9-1} que la flèche naturelle 
\begin{equation}\label{II:eq:4-11-1}
\xymatrix{
  K_n\lotimes_{\dZ/\ell^{n+1}} \dZ/\ell^n \ar[r] & K_{n+1} \text{,}
}
\end{equation}
est un isomorphisme. Le problème de la définition de cette flèche est 
discuté en \ref{II:4-12}. 

On peut maintenant invoquer \cite[XI.3.3]{sga5} (p. 31, 1-2 à la fin; ce 
passage est indépendant du reste de \cite{sga5}) pour réaliser chaque 
$K_n$ par un complexe borné de $\dZ/\ell^{n+1}$-modules libres de type fini, 
et les flèches \eqref{II:eq:4-11-1} par des \emph{isomorphismes de complexes} 
\[\xymatrix{
  K_n\otimes \dZ/\ell^n \ar[r]^-\sim 
    & K_{n-1} \text{.}
}\]
Soit $K$ le complexe $\varprojlim K_n$. C'est un complexe borné de 
$\dZ_\ell$-modules libres, et $K_n\simeq K\otimes_{\dZ_\ell}\dZ/\ell^{n+1}$. On 
a 
\[
  \h^i(K) = \varprojlim \h^i(K_n) \text{,}
\]
(tout système projectif de groupes finis vérifie en effet la condition de 
Mittag-Leffler), et $\h^i(K)$ est un $\dZ_\ell$-module de type fini. On a 
\[
  \h_c^i(X,\sG) = \left(\varprojlim \h^i(K_n) \right)\otimes_{\dZ_\ell} \dQ_\ell \text{:}
\]
c'est un $\dQ_\ell$-espace vectoriel de dimension finie. 





\subsection{La flèche \texorpdfstring{\eqref{II:eq:4-11-1}}{(4.11.1)}}\label{II:4-12}

Pour \emph{définir} la flèche \eqref{II:eq:4-11-1}, on ne peut pas 
procéder comme en \ref{II:4-9}, i.e. résoudre $\dZ/\ell^n$ par un complexe 
de $\dZ/\ell^{n+1}$-modules libres, puisqu'on veut construire on isomorphisme 
dans $\D_\text{parf}(\dZ/\ell^n)$. La flèche voulue est un cas particulier de 
la flèche d'extension des scalaires suivante. 

(*) Soit $\Lambda\to\Lambda'$ un homomorphisme d'anneaux noethériens de 
torsion. Pour $X$ un schéma séparé de type fini sur $k$ algébriquement 
clos, et $\sK\in\ob\D_\text{ctf}(X,\Lambda)$, on a un isomorphisme 
\[\xymatrix{
  \R\Gamma_c(X,\sK)\lotimes_\Lambda \Lambda' \ar[r]^-\sim & \R \Gamma_c(X,\sK\lotimes_\Lambda \Lambda') \text{.}
}\]

Des flèches plus générales sont construits dans \cite[XVII 4.2.12]{sga4}. La 
m\'
ethode est la suivante: 
\begin{enumerate}[\indent a)]
  \item La définition de $\R\Gamma_c$ est la formule 
    $j_!(\sK\otimes_\Lambda\Lambda') = (j_! \sK)\otimes_\Lambda\Lambda'$ pour une 
    immersion ouverte nous ramènent au cas où $X$ est propre. 
  \item Pour $X$ propre, il s'agit de remplacer $\sK$ par un complexe borné à 
    composantes acycliques pour le foncteur $\Gamma$, et de fibre en tout point 
    géométrique (ou seulement un tout point fermé) homotope à un 
    complexe de $\Lambda$-modules plats. Pour un tel complexe, on a 
    $\Gamma(X,\sK)\sim \R\Gamma(X,\sK)$ et 
    $K\otimes_\Lambda\Lambda'\sim \sK\lotimes_\Lambda\Lambda'$; la flèche 
    \eqref{II:eq:4-11-1} est définie comme la composition 
    \[
      \R\Gamma(X,\sK)\lotimes_\Lambda \Lambda' \to \Gamma(X,\sK)\otimes_\Lambda\Lambda' \to \Gamma(X,\sK\otimes_\Lambda\Lambda') \xleftarrow\sim \R\Gamma(X,\sK\lotimes_\Lambda\Lambda') \text{.}
    \]
  \item Il reste à montrer qu'il est possible de remplacer $\sK$ par un 
    complexe du type voulu \cite[XVII.4.2.10]{sga4}. On remplace d'abord $\sK$ par 
    un complexe borné à composantes plates (\ref{II:4-6}), puis on en prend 
    une résolution flasque canonique tronquée (ceci ne modifie pas le type 
    d'homotopie des complexes fibres, et on tronque assez loin pour que 
    vérifiée la condition d'acyclicité pour $\Gamma$ (i.e. au-delà 
    de la dimension cohomologique de $\Gamma$)). 
\end{enumerate}





\subsection{Prouve de \texorpdfstring{(\ref{II:3-2})}{(3.2)}}\label{II:4-13}

Pour simplifier les notations, nous ne traiterons que la cas $n=1$ de 
(\ref{II:3-2}). Le cas général s'en déduit en remplaçant ci-dessous 
$F$ par $F^n$. 

Soient donc $X_0$ séparé de type fini sur $\dF_q$, et $\sG_0$ un 
$\dQ_\ell$-faisceau sur $X_0$. On a $\sG_0=\sF_0\otimes\dQ_\ell$, pour 
$\dZ_\ell$-faisceau sans torsion convenable $\sF_0$. Soit 
$K_n=\R\Gamma_c(X,\sF\otimes\dZ/\ell^n)$. Les morphismes de Frobenius induisent 
des morphismes 
\[
  F^* : K_n\to K_n \qquad \text{(dans $\D_\text{parf}(\dZ/\ell^{n+1})$),}
\]
qui se déduisent les uns des autres via les isomorphismes 
\eqref{II:eq:4-11-1}. 

Réalisons comme en (\ref{II:4-11}) les $K_n$ et les isomorphismes 
\eqref{II:eq:4-11-1} su niveau des complexes. La passage déjà cité de 
\cite[XV.3.3]{sga5} permet de réaliser les morphismes $F^*$ par des 
endomorphismes de complexes, encore notés $F^*$, qui se réduisent les uns 
sur les autres. On note encore $F^*$ leur limite projective, et les 
endomorphismes qui s'en déduisent. Puisque 
$\h_c^i(X,\sG) = \h^i(K)\otimes\dQ_\ell = \h^i(K\otimes\dQ_\ell)$, on a (avec 
la convention de signe de (?.1) 
\[
  \tr\left(F^*, \h_c^\bullet(X,\sG)\right) = \tr\left(F^*,K^\bullet\otimes\dQ_\ell\right) = \varprojlim_n \tr\left(F^*,K_n^\bullet\right) = \varprojlim_n \tr\left(F^*,\R\Gamma_c(X,\sF\otimes\dZ/\ell^{n+1})\right) \text{.}
\]
Appliquons (\ref{II:4-10}) à $\sF_0\otimes\dZ/\ell^{n+1}$ (vu comme complexe 
réduit su degré $0$), pour $\Lambda=\dZ/\ell^{n+1}$. On trouve que 
\begin{align*}
  \tr\left(F^*,\R\Gamma_c(X,\sF\otimes\dZ/\ell^{n+1})\right) 
    &= \sum_{x\in X^F} \tr\left(F^*,\sF_x\otimes\dZ/\ell^{n+1}\right) \\
    &= \sum_{x\in X^F} \tr\left(F^*,\sG_x\right) \mod \ell^{n+1} \text{,}
\end{align*}
et (\ref{II:3-2}), en résulte par passage à limite. 










\section{La méthode de Nielsen-Wecken}\label{II:5}

Dans ce chapitre, nous prouvons deux cas particuliers de (\ref{II:4-10}). Le 
cas général sera considéré au chapitre suivant. 





\begin{lemma_}\label{II:5-1}
La théorème (\ref{II:4-10}) est vrai pour $X_0$ de dimension $0$. 
\end{lemma_}

En dimension $0$, la formule \eqref{II:eq:4-10} vaut pour toute 
correspondance, et non seulement pour les itérés d'un Frobenius. Elle est 
un cas particulier de l'énoncé suivent (appliqué à $X(\dF)$, $K$ et 
aux ${F^*}^n$). 

(*) Soit $X$ un ensemble fini, $F:X\to X$, $\sK\in\ob\D_\text{parf}(X,\Lambda)$ 
et $F^*:F^* \sK\to \sK$. On a 
\[
  \sum_{X^F} \tr\left(F^*,\sK_x\right) = \tr\left(F^*,\Gamma(X,K)\right) \text{.}
\]

On a $\Gamma(X,\sK)=\bigoplus_{x\in X} \sK_x$, et la formule résulte de ce que 
pour tout morphisme $u:\bigoplus_{x\in X} \sK_x\to \bigoplus_{x\in X} \sK_x$, de 
matrice $u_x^y$, on a $\tr(u)=\sum_{x\in X} \tr\left(u_x^x\right)$. 





\begin{lemma_}\label{II:5-2}
Soient $\Lambda$ un anneau noethérien de torsion, tué par une puissance du 
nombre premier $\ell\ne p$, $X_0$ une courbe projective, lisse et connexe sur 
$\dF_q$, $j:U_0\hookrightarrow X_0$ un ouvert dense de $X_0$ et $\sG_0$ un 
faisceau localement constant de $\Lambda$-modules projectifs sur $U_0$. On a 
\[
  \sum_{x\in U^F} \tr\left(F^*,\sG_x\right) = \tr\left(F^*,\R\Gamma_c(U,\sG)\right) \text{.}
\]
\end{lemma_}

Plus précisément, nous déduirons (\ref{II:5-2}) du 





\begin{theorem_}\label{II:5-3}
Soient $X$ une courbe projective non singulière sur un corps algébriquement 
clos $k$, $f$ un endomorphisme à points fixes isolés de $X$ et $v(f)$ le 
nombre de points fixes de $f$, chacun étant compté avec sa multiplicité: 
$v(f)$ est la nombre d'intersection du graphe de $f$ avec la diagonale de 
$X\times X$. Alors, pour $\ell$ premier à l'exposant caractéristique de $k$, 
on a 
\[
  v(f) = \sum (-1)^i \tr\left(f^*,\h^i(X,\dQ_\ell)\right) \text{.}
\]
\end{theorem_}

Ce théorème est dû à Weil (\cite{we48}, n$^\circ$ 43, 66 et 68) et la 
démonstration de Weil est reproduite dans Lang \cite{la83} (VI \S3 combiné 
avec VII \S2 th.3). On peut aussi le déduire du formalisme de la classe 
cohomologie associée à un cycle (\nameref{IV}, \ref{IV:3-7}). 





\begin{corollary_}\label{II:5-4}
Soit $j:U\hookrightarrow X$ un ouvert dense de $X$, tel que $f^{-1}(U)=U$. Si 
les points fixes de $f$ dans $X\setminus U$ sont de multiplicité un, alors le 
nombre $v_U(f)$ de points fixes de $f$ dans $U$ (chacun compté avec sa 
multiplicité) est 
\[
  v_U(f) = \sum (-1)^i \tr\left(f^*,\h_c^i(U,\dQ_\ell)\right) \text{.}
\]
\end{corollary_}

Soit $F$ l'ensemble fini $X\setminus U$. La suite exacte de cohomologie 
définie par la suite exacte court 
$0\to j_! \dQ_\ell\to\dQ_\ell\to{\dQ_\ell}_F\to 0$: 
\[\xymatrix{
  \ar[r]^-\partial 
    & \h_c^i(U,\dQ_\ell) \ar[r] 
    & \h^i(X,\dQ_\ell) \ar[r] 
    & \h^i(F,\dQ_\ell) \ar[r]^-\partial 
    & 
}\]
fournit, avec la convention de signe de (\ref{II:3-1}). 
\[
  \tr\left(f^*,\h_c^\bullet(U,\dQ_\ell)\right) 
    = \tr\left(f^*,\h^\bullet(X,\dQ_\ell)\right) - \tr\left(f^*,\h^\bullet(F,\dQ_\ell)\right) 
    = v(f) - \tr(f^*,\dQ_\ell^F) \text{.}
\]
La trace de $f^*$ sur $\dQ_\ell^F$ est le nombre $v_F(f)$ de points fixes de 
$f$ dans $F$; l'hypothèse de multiplicité un assure donc que 
\[
  \tr\left(f^*,\h_c^\bullet(U,\dQ_\ell)\right) 
    = v_F(f) - v_U(f) \text{,}
\]
comme promis. 

La preuve de (\ref{II:5-2}) utilise quelques lemmes que nous allons 
développer maintenant sur un cas particulier des traces non commutative 
(\ref{II:4-2}). \emph{Pour abréger, nous dirons projectif pour projectif de 
type fini}. 





\subsection{}\label{II:5-5}

Soient $\Lambda$ un anneau, et $H$ un groupe fini. L'application 
$\sum a_h \cdot a\mapsto $ classe de $a_e$, de l'algèbre de groupe 
$\Lambda[H]$ dans $\Lambda^\natural$, s'annule sur les commutateurs $ab-ba$ de 
$\Lambda[H]$. Elle se factorise donc par 
\[
  \varepsilon : \Lambda[H]^\natural \to \Lambda^\natural \text{.}
\]
Si $F$ est un endomorphisme d'un $\Lambda[H]$-module projectif $P$, on pose 
\[
  \tr_\Lambda^H(F) = \varepsilon \tr_{\Lambda[H]} (F) 
\]
où au second membre figure la trace (\ref{II:4-2}). Pour éviter des 
ambiguïtés, on écrira parfois cette trace $\tr_\Lambda^H(F,P)$. 





\begin{proposition_}\label{II:5-6}
$\tr_\Lambda(F) = |H|\tr_\Lambda^H(F)$. 
\end{proposition_}

Les propriétés formelles des traces nous ramènent à ne traiter que le 
cas où $P=\Lambda[H]$. L'endomorphisme $F$ est alors le multiplication à 
droite par un élément $\sum a_h\cdot h$ de $\Lambda[H]$, et 
$\tr_\Lambda(F) = |H| a_e$, $\tr_\Lambda^H(F) = a_e$. 





\subsection{}\label{II:5-7}

Soient $A$ un anneau commutatif et $\Lambda$ un $A$-algèbre. La multiplication 
par un élément de $A$ passe au quotient pour définir une structure de 
$A$-module sur $A^\natural$. Si $H$ est un groupe fini, $P$ un $A[H]$-module 
projectif et $M$ un $\Lambda[H]$-module, projectif en tant que $\Lambda$-module, 
on sait que le $\Lambda$-module $P\otimes_A H$, muni de l'action diagonale de 
$H$, est un $\Lambda[H]$-module projectif. En effet: 
\begin{enumerate}[\indent a)]
  \item I suffit de le vérifier par $P=A[H]$. 
  \item Le $\Lambda$-module $P\otimes_A M$, muni de l'action de $H$ définie 
    par son action sur le premier facteur, est clairement un 
    $\Lambda[H]$-module projectif; notions le $(P\otimes_A M)'$. 
  \item L'application $h\otimes m\mapsto h\otimes h^{-1} m$ induit un 
    isomorphisme 
    \begin{equation}\label{II:eq:5-7-1}
      A[H]\otimes_A M \iso (A[H]\otimes_A M)' \text{.}
    \end{equation}
\end{enumerate}





\begin{proposition_}\label{II:5-8}
Avec les notations précédentes, si $u$ est un endomorphisme de $F$ et 
$v$ un endomorphisme de $M$, on a 
\[
  \tr_\Lambda^H(u\otimes v) = \tr_A^H(u)\tr_\Lambda(v) \text{.}
\]
\end{proposition_}

On se ramène à supposer que $P=A[H]$; l'endomorphisme $u$ est alors la 
multiplication à droite $m_x$ par un élément $x=\sum a_h h$ de $A[H]$. 
L'isomorphisme \eqref{II:eq:5-7-1} transforme $u\otimes v$ en 
$\sum a_h m_h\otimes h^{-1} v$, de trace $a_e\cdot\tr_\Lambda(v)$. 





\subsection{}\label{II:5-9}

Soient $G$ un groupe extension de $\dZ$ par un groupe fini $H$
\[
  1 \to H \to G \to \dZ \to 1 \text{,}
\]
et $G^+$ l'image inverse de $\dN$. Soient $\Lambda$ un anneau, et $P$ un 
$\Lambda[G^+]$-module, projectif en tant que $\Lambda[H]$-module. Si 
$g\in G^+$, on note $Z_g$ le centralisateur de $g$ dans $H$; la multiplication 
par $g$ est un endomorphisme de $P$, vu comme $\Lambda[Z_g]$-module. $P$ étant 
$\Lambda[Z_g]$-projectif, une trace $\tr_\Lambda^{Z_g}(g)$ est définie. 
D'après \ref{II:5-6}, on a 
\begin{equation}\label{II:eq:5-9-1}
  \tr_\Lambda(g) = |Z_g| \cdot \tr_\Lambda^{Z_g}(g) \text{.}
\end{equation}

Supposons que $\Lambda$ soit une algèbre sur un anneau commutatif $A$. Pour 
$P$ un $A[G^+]$-module, projectif en tant que $A[H]$-module et $M$ un 
$\Lambda[G^+]$-module, projectif en tant que $\Lambda$-module, le 
$\Lambda[G^+]$-module $P\otimes_A M$ (action diagonale de $G^+$) est projectif 
en tant que $\Lambda[H]$-module et d'après \ref{II:5-8}, pour tout 
$g\in G^+$, on a 
\begin{equation}\label{II:eq:5-9-2}
  \tr_\Lambda^{Z_g}(g,P\otimes_A M) = \tr_A^{Z_g}(g,P) \cdot \tr_\Lambda(g,M) \text{.}
\end{equation}





\subsection{}\label{II:5-10}

Soient $P$ un $\Lambda[G^+]$-module projectif en tant que $\Lambda[H]$-module, 
et $P_H$ les coinvariants de $H$ dans $P$. C'est un $\Lambda$-module projectif. 
L'action de $g\in G^+$ passe au quotient, et définit un endomorphisme de 
$P_H$ qui ne dépend que de l'image de $g$ dans $\dN$. Nous notons $F$ 
l'action des éléments $g\in G^+$ d'image $1$. 





\begin{proposition_}\label{II:5-11}
$\tr_\Lambda(F,P_H) = \sum_{g\mapsto 1}' \tr_\Lambda^{Z_g}(g,P)$, oùu 
$\sum_{g\mapsto 1}'$ désigne une somme étendue aux classes de 
$H$-conjugaison d'éléments de $G^+$ d'image $1$ dans $\dN$. 
\end{proposition_}

Après multiplication par $|H|$, cette formule peut se vérifier comme suit: 
\[
  |H|\cdot \tr_\Lambda(F,P_H) 
    = \tr_\Lambda\left(\sum_{g\mapsto 1} g,P\right) 
    = \sum_{g\mapsto 1}' \frac{|H|}{|Z_g|}\tr_\Lambda(g,P) 
    = \sum_{g\mapsto 1}' |H|\cdot \tr_\Lambda^{Z_g}(g,P) \text{.}
\]

Choisissons un élément $g\in G^+$ d'image $1$ dans $\dN$, et soit 
$\sigma$ l'automorphisme $\Lambda[H]$ induit par l'automorphisme 
$\operatorname{ad} g$ de $H$. Il revient au même de se donner un 
$\Lambda[G^+]$-module, ou un $\Lambda[H]$--module muni d'un endomorphisme 
$\sigma$-linéaire $\gamma$: on fait agir $g^n h$ par $\gamma^n h$. Dans ce 
langage, la formule \ref{II:5-11} prend la forme: pour $P$ un 
$\Lambda[H]$-module projectif et pour $\gamma$ un endomorphisme 
$\sigma$-linéaire de $P$, on a 
\[
  \tr_\Lambda(\gamma,P_H) = \sum_h' \tr_\Lambda^{Z_g}(h\gamma,P)\text{,}
\]
où $\sum'$ est une somme sur les classes de $\operatorname{ad}g$-conjugaison 
dans $H$ (orbites de $H$ agissant sur lui-même par les 
$x\mapsto h x \operatorname{ad} g(h)^{-1}$). 

Sous cette forme, les propriétés des traces nous ramènent aussitôt au 
cas où $P=\Lambda[H]$. L'endomorphisme $Y$ a alors la forme 
$x\mapsto \sigma(x) a$, avec $a=\sum a_h h$ dans $\Lambda[H]$. Il suffit de 
traiter le cas universel où lles $a_h$ sont des indéterminées. Dans ce 
cas, $\Lambda=\dZ\langle a_h\rangle_{h\in H}$ (polynômes non commutatifs) et 
$\Lambda^\natural$ est sans torsion: on obtient \ref{II:5-11} par l'argument 
du début et division par $|H|$. On peut aussi calculer directement. 





\subsubsection{Remarque}\label{II:5-11-1}

Tout ce qui a été dit ci-dessus s'applique aussi bien aux modules 
$\dZ/2$-gradués, ou aux complexes de modules (cf. \ref{II:4-2}, \ref{II:4-3}). 





\subsection{}\label{II:5-12}

Reprenons les notations de \ref{II:5-2}, et soit $A=\dZ/\ell^n$, $n$ étant 
assez grand pour que $\Lambda$ soit une $A$-algèbre. Soit $f:V_0\to U_0$ un 
revêtement fini étale et connexe de $U_0$, galoisien de groupe de Galois 
$H$, et tel que le faisceau $f^*\sG_0$ soit constant. On note $M$ le valeur 
constants $\h^0(V_0,f^*\sG_0)$, c'est un $\Lambda[H]$-module, projectif en 
tant que $\Lambda$-module. 

Le faisceau $f_* A$ sur $U_0$, muni de l'action naturelle de $H$, est un 
faisceau localement libre (de rang un) de $A[H]$-modules. Le complexe 
$\R\Gamma_c(U,f_* A)$ est donc objet de $\D_\text{parf}\left(A[H]\right)$. Il 
est muni d'un endomorphisme de Frobenius $F^*$. 

Pour l'action naturelle de $H$ sur $f_* f^*\sG_0$, le morphisme trace 
$f_* f^*\sG_0\to\sG_0$ se factorise par un isomorphisme 
\[
  \left(f_* f^*\sG_0\right)_H \to \sG_0 \text{.}
\]
Par ailleurs, on a $f_* f^*\sG_0=f_* A\otimes_A M$ (avec action diagonale de 
$H$). L'isomorphisme 
\[\xymatrix{
  \sG_0 
    & \ar[l]_-\sim (f_* A\otimes_A M)_H = (f_* A\otimes_A M)\otimes_{\Lambda[H]} \Lambda
}\]
se dérive (\ref{II:4-12}) et un isomorphisme 
\[
  \R\Gamma_c(U,\sG) \sim \left(\R\Gamma_c(U,f_* A)\otimes_A M\right)\otimes_{\Lambda[H]} \Lambda \text{.}
\]

L'endomorphisme $F^*$ du membre de gauche se déduit de l'endomorphisme 
$F^*$ de $\R\Gamma_c(U,f_* A)$. Le complexe de $\Lambda[H]$-modules 
$\R\Gamma_c(U,f_* A)$, muni de $F^*$, peut être vu comme un complexe de 
$\Lambda[G^+]$-modules, pour $G^+=H\times \dN$. Les formules \ref{II:5-10} 
et \eqref{II:eq:5-9-2}, amplifiées par \ref{II:5-11-1}, fournissent donc 
\[
  \tr\left(F^*,\R\Gamma_c(U,\sG)\right) = \sum_h' \tr_A^{Z_g}\left(h F^*,\R\Gamma_c(U,f_* A)\right)\cdot \tr_\Lambda(h,M)
\]
où $\sum'$ est une somme sur les classes de conjugaison dans $H$, De plus, 
\[
  |Z_h|\cdot \tr_A^{Z_h}\left(h F^*,\R\Gamma_c(U,f_ A)\right) = \tr_A\left((F h^{-1})^*,\R\Gamma_c(V,A)\right) \text{.}
\]
Cette formule reste vraie pour $A=\dZ/\ell^m$, avec $m$ de plus en plus grand. 
Passant à limite, on trouve que 
$\tr_A^{Z_g}\left(h F^*,\R\Gamma_c(U,f^* A)\right)$ vaut 
\[
  \frac{1}{|Z_h|} \sum (-1)^i \tr\left((F h^{-1})^*,\h_c^i(V,\dQ_\ell)\right)
\]
(un élément de $\dZ_\ell$), et
\begin{equation}\label{II:eq:5-12-1}
\tr\left(F^*,\R\Gamma_c(U,\sG)\right) = \sum_h' \frac{1}{|Z_h|} \tr\left((F h^{-1})^*,\h_c^i(V,\dQ_\ell)\right)\cdot \tr_\Lambda(h,M) \text{.} 
\end{equation}





\subsection{}\label{II:5-13}

Il reste maintenant à appliquer \ref{II:5-4} (noter que si $\bar V$ est la 
complétion projective non singulière de $V$, les points fixes de $F h^{-1}$ 
sont tous de multiplicité un). On peut le faire sans aucun calcul: 
\begin{enumerate}[\indent A.]
  \item Si $U^F=\varnothing$, $\tr\left(F^*,\R\Gamma_c(U,\sG)\right) = 0$. \end{enumerate}

En effet, $F h^{-1}$ n'a pas de point fixe sur $V$, \ref{II:5-4} montre que 
$\sum (-1)^i \tr\left((F h^{-1})^*,\h_c^i(V,\dQ_\ell)\right)=0$ et on 
applique \eqref{II:eq:5-12-1}.
\begin{enumerate}[\indent A.]
\setcounter{enumi}{1}
  \item Dans le cas général, soit $j:U'=U\setminus U^F\hookrightarrow U$. 
    La filtration  $j_! j^* \sG_0\subset \sG_0$ de $\sG_0$ induit une 
    filtration de $\R\Gamma_c(U,\sG)$ de quotients successifs 
    $\R\Gamma_c(U',\sG)$ et $\R\Gamma_c(U^F,\sG)$. D'après \eqref{II:eq:4-4-1} 
    et \ref{II:5-1}, on a 
    \[
      \tr\left(F^*,\R\Gamma_c(U,\sG)\right) = \tr\left(F^*,\R\Gamma_c(U',\sG)\right) + \sum_{x\in U^F} \tr(F^*,\sG_x) \text{,}
    \]
    et, puisque ${U'}^F=0$, le premier terme au second membre est $0$. 
\end{enumerate}










\section{Fin de la preuve de \ref{II:4-10}}\label{II:6}

Prouvons \ref{II:4-10}, sous les hypothèses additionnelles que l'anneau 
$\Lambda$ est annulé par une puissance d'un nombre premier $\ell\ne p$, et 
que $n=1$. Le cas général en résultera: décomposer $\Lambda$ en ses 
composantes $\ell$-primaires ($\ell$ parcourant les nombre premiers), et 
remplacer $\dF_q$ par $\dF_{q^n}$, $X_0$ par $X_0\otimes_{\dF_q}\dF_{q^n}$. 

Pour $X_0$ séparé de type fini sur $\dF_q$, et 
$\sK_0\in\ob\D_\text{ctf}(X_0,\Lambda)$, posons 
\[
  T'(X_0,\sK_0) = \sum_{x\in X^F} \tr(F^*,\sK_x)
\]
et
\[
  T''(X_0,\sK_0) = \tr(F^*,\R\Gamma_c(X,\sK)) \text{.}
\]
Il faut prouver l'identité 
\[
  (1;X_0,\sK_0) \qquad T'(X_0,\sK_0) = T''(X_0,\sK_0) \text{.}
\]
On commence par observer que $T'$ et $T''$ (notés génériquement $T$) 
obéissent à un même formalisme:

(A) Si $j:U_0\hookrightarrow X_0$ est un ouvert de $X_0$, de fermé 
complémentaire $Y_0$, on a 
\[
  T(X_0,\sK_0) = T(U_0,\sK_0|U_0) + T(Y_0,\sK_0|Y_0) \text{.}
\]
C'est clair pour $T'$. Pour $T''$, on observe que la filtration 
$0\subset j_! j^* \sK_0\subset \sK_0$ de $\sK_0$ induit une filtration de 
$\R\Gamma_c(X,\sK)$, de quotients successifs $\R\Gamma_c(U,\sK)$ et 
$\R\Gamma_c(Y,\sK)$, et on invoque \eqref{II:eq:4-4-1} (plus exactement 
énoncé: $\sK_0$, muni de la filtrations dite, définit un objet de 
$\mathsf{DF}(X,\Lambda)$; par application de $\R\Gamma_c$, on a déduit un 
objet de $\mathsf{DF}_\text{parf}(\Lambda)$, auquel $\R\Gamma_c(X,\sK)$ est 
sous-jacent, et de gradué comme dit). 

(B) Si $\sK_0$ est un complexe borné à composantes constructibles et plates, 
\[
  T(X_0,\sK_0) = \sum (-1)^i T(X_0,\sK^i) \text{.}
\]
C'est clair pour $T'$; pour $T''$, on applique l'argument ci-dessus à la 
filtration bête de $\sK_0$, pour obtenir 
\[
  T''(X_0,\sK_0) = \sum T''(X_0,\sK_0^i[-i]) = \sum (-1)^i T''(X_0,\sK_0^i) \text{.}
\]

(C) Pour $f:X_0\to Y_0$ un morphisme, on a 
\[
  T''(X_0,\sK_0) = T''(Y_0,\R f_! \sK_0)
\]
et la même identité vaut pour $T'$ si les fibres de $f$ aux points 
rationnels de $Y_0$ vérifiant $\left(1;f^{-1}(y),\sK_0|f^{-1}(y)\right)$: on 
utilise que 
\[
  T'(X_0,\sK_0) = \sum_{y\in Y^F} T'\left(f^{-1}(y), \sK_0|f^{-1}(y)\right)
\]
et 
\[
  \R\Gamma_c(Y, \R f_! \sK) = \R\Gamma_c(X,\sK) \text{.}
\]

Utilisant ce formalisme, et \ref{II:4-7}, on ramène la preuve de \ref{II:4-10} 
aux deux particuliers \ref{II:5-1} et \ref{II:5-2}. 


% !TEX root = sga4.5.tex

\chapter{Fonctions \texorpdfstring{$L$}{L} modulo \texorpdfstring{$\ell^n$}{l n} et modulo \texorpdfstring{$p$}{p}}\label{III}

Dans cet exposé, je prouve une formule analogue au théorème du rapport sur 
la formule des traces (ce volume; cité ``Rapport'' par la suite) pour un 
anneau de coefficients $A$ qui soit de torsion premier à $p$, ou un corps de 
caractéristique $p$. Un usage essentiel sera fait de \cite[XVII 5.5]{sga4}. 










\section{Tenseurs symétriques}\label{III:1}





\subsection{}\label{III:1-1}

Soit $A$ un anneau commutatif. Pour une liste de propriétés des foncteurs 
$\Gamma^n:(\text{$A$-modules})\to (\text{$A$-modules})$, je renvoie à 
\cite[XVII 5.5.1 et 2]{sga4}. Rappelons seulement que pour $M$ plat (en fait, seul 
le cas où $M$ est projectif de type fini nous intéresse), on a 
\[
  \Gamma^n M = \left(M^{\otimes n}\right)^{S_n} \qquad \text{(tenseurs symétriques de degré $n$).}
\]





\subsection{}\label{III:1-2}

Soient $X$ un $S$-schéma, $X^n/S$ la puissance fibrée $n$-ième de $X$ et 
$\pi$ la projection de $X^n/S$ sur $(X^n/S)/S_n=\operatorname{Sym}_S^n(X)$ 
(supposé exister). Si $\sG$ est un faisceau de $A$-modules sur $X$, le 
produit tensoriel externe de $n$ copies de $\sG$; $\sG^{\boxtimes n}$, est un 
faisceau $S_n$-équivariant sur $X^n/S$. Nous aurons à faire usage du 
foncteur $\Gamma^n$ \emph{externe} de \cite[XVII 5.5.7 à 9]{sga4}. Rappelons 
seulement que pour $\sG$ un faisceau de $A$-modules plats (en fait, seul le cas 
constructible et plat nous intéresse), on a 
\[
  \Gamma_\text{ext}^n(\sG) = \left(\pi_*\sG^{\boxtimes n}\right)^{S_n} \text{.}
\]





\begin{lemma_}\label{III:1-3}
Pour $S=\spec(k)$, avec $k$ algébriquement clos, $X$ une somme finie de 
copies de $S$, et $G$ un faisceau plat de $A$-modules sur $X$, on a 
\[
  \Gamma\left(\operatorname{Sym}_S^n(X),\Gamma_\text{ext}(\sG)\right) = \Gamma^n \Gamma(X,\sG)
\]
\end{lemma_}

En effet, 
\begin{align*}
  \Gamma\left(\operatorname{Sym}_S^n(X),\Gamma_\text{ext}^n(\sG)\right) 
    &= \Gamma\left(\operatorname{Sym}_S^n(X),\pi_*\sG^{\boxtimes n}\right)^{S_n} \\
    &= \Gamma\left(X^n/S,\sG^{\boxtimes n}\right)^{S_n} 
    = \left(\Gamma(X,\sG)^{\otimes n}\right)^{S_n} 
    = \Gamma^n \Gamma(X,\sG) \text{.}
\end{align*}





\subsection{}\label{III:1-4}

Nous aurons à utiliser les dérivés des foncteurs non additifs $\Gamma^n$ 
et $\Gamma_\text{ext}^n$. 

La théorie de tela foncteurs dérivés est due à Dold et Puppe 
\cite{do61}. Pour des rappels plus complets que ceux donnés ci-dessous, je 
renvoie à \cite[XVII 5.5.3]{sga4}. 

Soit $\cA$ une catégorie abélienne (voire une catégorie additive où 
tout endomorphisme idempotent soit la projection sur un facteur direct). Notons 
$N$ le foncteur de normalisation: $N:($complexes cosimpliciaux d'objets de 
$\cA)\to($complexes différentiables d'objets de $\cA$, avec $K^i$ pour 
$i<0)$. C'est une équivalence de catégories. Ainsi que son inverse, elle 
transforme homotopes en morphismes homotopes. Si $\Gamma$ est un foncteur 
$\cA\to\cB$, et si $K$ est un complexe d'objets de $\cA$, à degrés 
positifs ($K^i=0$ pour $i<0$), on pose $T K=N T N^{-1} K$. 





\subsection{Exemple}\label{III:1-5}

Si $K$ est réduit à $M$ en degré $0$, $T K$ st réduit à $T M$ en 
degré $0$. 





\subsection{}\label{III:1-6}

Si $K$ est un complexe borné, à degrés positifs, de $A$-module projectifs 
de type fini, le complexe de $A$-modules $\Gamma^n K$ a les mêmes 
propriétés. 

Notons $\D_\text{parf}^{\geqslant 0}(A)$ la sous-catégorie de 
$\D_\text{parf}(A)$ (Rapport, \ref{II:4-3}) formée des complexes de 
tor-dimension $\leqslant 0$. On voit que $\Gamma^n$ induit un foncteur 
\[
  \eL\Gamma^n : \D_\text{parf}^{\geqslant 0}(A) \to \D_\text{parf}^{\geqslant 0}(A) \text{.}
\]

Pour la définition analogue, mais plus compliquée, de 
$\eL\Gamma_\text{ext}^n$, je renvoie à \cite[XVII 5.5.14]{sga4}. 





\subsection{}\label{III:1-7}

Soient $K$ un complexe borné de $A$-modules projectifs de type fini, et $u$ 
un endomorphisme de $K$. On note $\det(1-u t,K)$ la série formelle de terme 
constant $1$ 
\[
  \det(1-u t,K) = \prod_i \det(1-u t,K^i)^{(-1)^i} \text{.}
\]

On laisse au lecteur la vérification des propriétés suivantes. 


\subsubsection{}\label{III:1-7-1}

Soit $F$ une filtration finie stable par $u$, telle que les $\gr_F^p(K^q)$ 
soient projectifs de type fini. Alors 
\[
  \det(1-u t,K) = \prod_p \det(1-u t,\gr_F^p(K)) \text{.}
\]


\subsubsection{}\label{III:1-7-2}

Si on translate $K$ de $q$ crans vers la gauche, on a 
\[
  \det(1-u t,K[q]) = \det(1-u t,K)^{(-1)^q} \text{.}
\]





\begin{proposition_}\label{III:1-8}
Soit $u$ un endomorphisme d'un complexe borné, à degrés positifs, de 
$A$-modules projectifs de type fini. On a \
\begin{equation}\label{III:eq:1-8-1}
  \det(1-u t,K)^{-1} = \sum_n \tr(u,\Gamma^n K) t^n \text{.}
\end{equation}
\end{proposition_}

Notons $D(u,K)$ le second membre de \eqref{III:eq:1-8-1}. 





\begin{lemma_}\label{III:1-9}
Soit $u$ un endomorphisme d'une suite exacte courte de complexes bornés à 
degrés $\geqslant 0$ de $A$-modules projectifs de type fini 
\[
  0 \to K' \to K \to K'' \to 0 \text{.}
\]
On a $D(u,K)=D(u,K')\cdot D(u,K'')$.
\end{lemma_}

Si $0\to M'\to M\to M''\to 0$ est une suite exacte courte scindable de 
$A$-modules, $\Gamma^n M$ a une filtration naturelle, de quotients successifs 
les $\Gamma^i M'\otimes \Gamma^j M''$ ($i+j=n$). Dès lors, $\Gamma^n K$ admet 
une filtration naturelle, de quotients successifs les complexes 
$N(N^{-1} \Gamma^i K'\otimes N^{-1}\Gamma^j K'')$ ($i+j=n$), canoniquement 
homotopes aux complexes simples $s(\Gamma^i K'\otimes \Gamma^i K'')$ déduit 
des complexes doubles $\Gamma^i K'\otimes \Gamma^i K''$, et 
\[
  \tr(u,\Gamma^n K) = \sum_{i+j=n} \tr(u,\Gamma^i K')\cdot \tr(u,\Gamma^j K'') \text{,}
\]
d'où le lemme. De \ref{III:1-9}, on tire par récurrence. 





\begin{lemma_}\label{III:1-10}
Soient $F$ une filtration finie stable par $u$ de $K$, telle que les 
$\gr_F^p(K^q)$ soient projectifs de type fini. Alors, 
\[
  D(u,K) = \prod_p D\left(u,\gr_F^p(K)\right) \text{.}
\]
\end{lemma_}





\begin{lemma_}\label{III:1-11}
Soient $u$ un endomorphisme d'un $A$-module projectif de type fini $M$, et 
$M[-q]$ le complexe réduit à $M$ en degré $q$. On a 
\[
  D\left(u,M[-q]\right) = \left(\sum \tr(u,\Gamma^n M) t^n\right)^{(-1)^q} \qquad \text{($q\geqslant 0$.)}
\]
\end{lemma_}





D'après \ref{III:1-5}, c'est vrai pour $q=0$. Précédons par récurrence, il 
reste à montrer que 
\[
  D\left(u,M[-q]\right) \cdot D\left(u,M[-(q+1)]\right) = 1 \text{.}
\]
Cette formule résulte de \ref{III:1-9}, appliqué ou complexe $K$ réduit a 
$M$ en degrés $q$ et $q+1$ (avec $d=\operatorname{id}$), et à sa 
filtration bête:
\[
  (\cdots 0 \to M \cdots) \to (\cdots M \to M) \to (\cdots M \to 0 \to 0 \cdots) \text{.}
\]
En effet, $K$ est homotope à $0$. Dans la catégorie dérivée, on a donc 
$\Gamma^n(K)=\Gamma^n(0)$; $0$ pour $n>0$ et $A$ pour $n=0$, et $D(u,K)=1$. 





\subsection{}\label{III:1-12}

Prouvons \ref{III:1-8}. D'après \ref{III:1-10} appliqué à la filtration 
``bête'' de $K$ par les sous-complexes 
\[
  0 \to \cdots \to 0 \to K^q \to K^{q+1} \to \cdots \text{,}
\]
on a 
\[
  D(u,K) = \prod_q D\left(u,K^q[-q]\right) =_{\text{(\ref{III:1-10})}} \prod_q \left(\sum \tr\left(u,\Gamma^n(K^q)\right) t^n \right)^{(-1)^q} \text{.}
\]
Il reste à prouver que pour $u$ un endomorphisme d'un module projectif de type 
fini $M$, on a 
\begin{equation}\label{III:eq:1-12-1}
  \det\left(1-u t,M\right)^{-1} = \sum \tr(u,\Gamma^n M) t^n \text{.}
\end{equation}
Si $u=0$, les deux membres valent $1$. Ajoutant à $M$ un module $N$ sur lequel 
$u=0$, et appliquent \ref{III:1-9}, on se ramène à supposer que $M$ est libre. 
Prenons-en un base. Il s'agit alors de prouver des identités algébriques 
entres les coordonnées de $u$. Le principe de prolongement des identités 
algébriques nous ramène à supposer que $A$ est un corps algébriquement 
clos. Le module $M$ admet alors une filtration stable par $u$ à quotients 
successifs de rang $1$, et \ref{III:1-10} nous ramène au case où $M$ est de 
rang $1$. La formule \eqref{III:eq:1-12-1} se ramène alors à l'identité 
\[
  \frac{1}{1-a t} = \sum a^n t^n \text{.}
\]





\begin{corollary_}\label{III:1-13}
Soient $u$ et $u'$ des endomorphismes de complexes bornés $K$ et $K'$ de 
$A$-modules projectifs de type fini. Si, dans la catégorie dérivée, $K$, 
muni de $u$, est isomorphe à $K'$, muni de $u'$, alors 
\[
  \det(1-u t,K) = \det(1-u' t,K') \text{.}
\]
\end{corollary_}

Appliquent \ref{III:1-7-2}, on se ramène à supposer que $K$ et $K'$ sont à 
degrés positifs. On observe alors que le second membre de \ref{III:1-8} 
possède la propriété d'invariance voulue. 





\subsection{}\label{III:1-14}

Ce corollaire permet de définir $\det(1-u t,K)$ pour $u$ un endomorphisme de 
$K\in \ob\D_\text{parf}(A)$. Cette construction est ad hoc; en fait, pour $v$ un 
automorphisme de $L\in\ob\D_\text{parf}(\Lambda)$, on peut définir 
$\det(v)\in\Lambda^\times$, et $\det(1-u t,K)$ s'obtient pour $v=1-u t$, 
$\Lambda=A\llbracket t\rrbracket$ et $L=K\otimes_A\Lambda$. 










\section{Le Théorème}\label{III:2}





\subsection{}\label{III:2-1}

Reprenons les notations de (Rapport, \S\ref{II:1}). Pour $A$ un anneau 
noethérien commutatif de torsion, $X_0$ un schéma séparé de type fini sur 
$\dF_q$ et $\sK_0\in\ob\D_\text{ctf}(X_0;A)$, on pose 
\[
  L(X_0,\sK_0) = \prod_{x\in |X_0|} \det\left(1-F_x^* t^{\deg(x)}, \sK\right)^{-1}
\]
(\ref{III:1-7}, \ref{III:1-13} et Rapport, \ref{II:1-5}, \ref{II:1-6}). 





\begin{theorem_}\label{III:2-2}
Dans chacun des deux cas suivants
\begin{enumerate}[\indent a)]
  \item $A$ est de torsion première à $p$;
  \item $A$ est réduit et de caractéristique $p$
\end{enumerate}
on a 
\begin{equation}\label{III:eq:2-2-1}
  L(X_0,\sK_0) = \det\left(1-F^* t^f, \R\Gamma_c(X,\sK)\right)^{-1} \text{.}
\end{equation}
\end{theorem_}





\subsection{Remarque}\label{III:2-3}

Dans le cas a), la méthode de Rapport, \S\ref{II:3} permet de déduire de 
la formule des traces (Rapport \ref{II:4-10}) que les dérivées logarithmiques 
des deux membres de \eqref{III:eq:2-2-1} sont égales. L'anneau $A$ étant de 
torsion, cela ne suffit pas pour prouver \ref{III:2-2}. 





\subsection{Remarque}\label{III:2-4}

Si $A$ est un corps, on a 
\[
  L(X_0,\sK_0) = \prod_i L\left(X_0,\underline\h^i(\sK_0)\right)^{(-1)^i}
\]
et la suite spectrale 
$E_2^{p q} = \h_c^p(X,\underline\h^q(\sK))\Rightarrow \h_c^{p+q}(X,\sK)$ montre 
que 
\begin{align*}
  \det\left(1-F^* t, \R\Gamma_c(X,\sK)\right) 
    &= \prod_n \det\left(1-F^* t^f,\h_c^n(X,\sK)\right)^{(-1)^n} \\
    &= \prod_{p,q} \det\left(1-F^* t,\h_c^p(X,\underline\h^q(\sK))\right)^{(-1)^{p+1}} \text{.}
\end{align*}
Ceci ramène \ref{III:2-2} (pour $A$ un corps) au cas particulier suivant. 


\subsubsection{}\label{III:2-4-1}

Si $\sG_0$ est un faisceau constructible de $A$-espaces vectoriels, on a 
\[
  L(X_0,\sG_0) = \prod_n \det\left(1-F^* t^f,\h_c^n(X,\sG)\right)^{(-1)^{n+1}} \text{.}
\]





\subsection{Remarque}\label{III:2-5}

Il est facile de réduire le cas b) à celui où $A$ est un corps (voire 
même un corps fini) de caractéristique $p$, cf. \ref{III:4-2}. 





\subsection{Remarque}\label{III:2-6}

Le cas particulier de b) où $A=\dZ/p$, et où $X$ est réduit su faisceau 
constant $\dZ/p$ est degré $0$;
\[
  Z(X_0,t) = \prod_{x\in |X_0|} \left(1-t^{\deg(x)}\right)^{-1}
      \equiv \prod_n \det\left(1-F^* t^f,\h_c^n(X,\dZ/p)\right)^{(-1)^{n+1}} \pmod p
\]
avait été prouvé par N. Katz \cite[XXII 3.1]{sga7}. Sa méthode, toute 
différente de celle suivie ici, part de la théorie $p$-adique de Dwork de 
$Z(X_0,t)$. 





\subsection{Remarque}\label{III:2-7}


Contrairement à ce que j'avais imprudemment affirmé à des amis, on ne 
peut pas dans b) omettre l'hypothèse ``$A$ réduit.'' Pour un contre 
exemple, cf. \ref{III:4-5}. 





\subsection{Plan de la preuve de \ref{III:2-2}}\label{III:2-8}

Une translation sur les degrés \ref{III:1-7-2} nous ramène à supposer 
que $\sK_0$ est un complexe borné, à degré positifs, de faisceaux de 
$A$-modules constructibles et plats. On a alors 
$\R\Gamma_c(X,\sK)\in \ob\D_\text{parf}^{\geqslant 0}(A)$ (Rapport, 
preuve \ref{II:4-9}, b). Nous allons utiliser \ref{III:1-8} pour développer 
les deux membres de \eqref{III:eq:2-2-1} en série de puissances de $T=t^f$. 
L'égalité des coefficients de $T^n$ de déduira de 
\cite[XVII 5.5.21]{sga4} et d'une formule des traces sur 
$\operatorname{Sym}^n(X)$; dans le cas a), le formule (Rapport, 
\ref{II:4-10}), et dans le cas b) une formule qui sera prouvée dans le 
chapitre suivant. Il y a lieu de se ramener au préalable au cas où les 
puissances symétriques de $X_0$ existent (par exemple au cas où $X$ est 
affine) à l'aide des propriétés de multiplicativité en $X_0$ des deux 
membres de \eqref{III:eq:2-2-1}: pour $U_0$ un ouvert de $X_0$, de fermé 
complémentaire $Y_0$, on a $L(X_0,\sK_0)=L(U_0,\sK_0)\cdot L(Y_0,\sK_0)$ et 
la formule analogue pour le membre de droite résulte de \ref{III:1-7-1} 
par la méthode de (Rapport \S\ref{II:6}A). 





\subsection{Le cas où $X_0$ est fini}\label{III:2-9}

La multiplicativité en $X_0$ des deux membres de \eqref{III:eq:2-2-1} 
ramène ce cas à celui où $X_0=\spec(\dF_{q^k})$. Revenant aux 
définitions, on voit que \eqref{III:eq:2-2-1} équivaut alors à 
Rapport \ref{II:3-4}. 





\subsection{Le premier membre}\label{III:2-10}

D'après \ref{III:1-8}, on a 
\[
  L(X_0,\sK_0) = \prod_{x\in |X_0|} \sum_n \tr\left(F_x^*,\Gamma^n\sK\right) t^{m\deg(x)} \text{.}
\]
Le coefficient de $T^n$ est donc la somme, étendue à celles des 
combinaisons linéaires formelles $\sum_{x\in |X_0|} n_x\cdot x$ 
($n_x\geqslant 0$, les $n_x$ presque tous nuls) d'éléments de $|X_0|$ 
telles que $\sum n_x [k(x):\dF_q] = n$, des produits 
\begin{equation}\label{III:eq:2-10-1}
  \prod_x \tr\left(F_x^*,\Gamma^{n_x} \sK\right) \text{.}
\end{equation}

Pour $x\in |X_0|$, notons $(x)$ le $0$-cycle des $[k(x):\dF_q]$ points de $X$ 
au-dessus de $x$ (une orbite de $F$). L'application 
$\sum n_x\mapsto \sum n_x\cdot (x)$ identifie les combinaisons linéaires 
formelles comme ci-dessus, avec $\sum n_x [k(x):\dF_q]=n$, aux $0$-cycles 
effectifs de degré $n$ de $X$ fixes sous $F$, soit encore aux points fixes de 
$F$ dans $\operatorname{Sym}^n(X)$. Le coefficient de $T^n$ dans $L(X_0,\sK_0)$ 
apparait ainsi comme une somme de produits \eqref{III:eq:2-10-1}, indenée 
par $\operatorname{Sym}^n(X)^F$. Plus précisément, 





\begin{lemma_}\label{III:2-11}
 $L(X_0,\sK_0) = \sum_n \sum_{x\in\operatorname{Sym}^n(X)^F} \tr\left(F_x^*,\eL\Gamma_\textnormal{ext}^n \sK\right) T^n$.
\end{lemma_}

Pour $Y_0$ un sous-schéma fini de plus en plus grand de $X_0$, on a 
\[
  L(X_0,\sK_0) = \lim L(Y_0,\sK_0) \text{,}
\]
i.e., pour tout $m$, il existe un sous-schéma fini $Z_0\subset X_0$ tel que 
si $Y_0\supset Z_0$, $L(X_0,\sK_0)$ et $L(Y_0,\sK_0|Y_0)$ soient congrus 
$\mod T^m$. 
Il suffit que $Z_0$ contienne les points fermés de degré $\leqslant m$. 
Le même fait valant pour le membre de droite, il suffit de prouver 
\ref{III:2-11} pour $X_0$ fini. On a alors 
\begin{align*}
  L(X_0,\sK_0) 
    &=_\text{\ref{III:2-9}} \det\left(1-F^* T,\Gamma(X,\sK)\right)^{-1} \\
    &=_\text{\ref{III:1-8}} \sum_n \tr\left(F^*,\Gamma^n \Gamma(X,\sK)\right) T^n \\
    &=_\text{\ref{III:1-3}} \sum_n \tr\left(F^*,\Gamma(\operatorname{Sym}^n(X),\Gamma_\text{ext}^n(\sK)\right) T^n \\
    &= \sum_n \sum_{x\in \operatorname{Sym}^n(X)^F} \tr\left(F_x^*,\Gamma_\text{ext}^n(\sK)\right) T^n
\end{align*}
(par la formule des traces dans le cas trivial d'un ensemble fini). 





\subsection{Le second membre}\label{III:2-12}

D'après \ref{III:1-8} et \cite[XVII 5.5.12]{sga4}, on a 
\begin{align*}
  \det\left(1-F^* T,\R\Gamma_c(X,\sK)\right)^{-1} 
    &= \sum_n \tr\left(F^*,\eL\Gamma^n\R\Gamma_c(X,\sK)\right) T^n \\
    &= \sum_n \tr\left(F^*,\R\Gamma_c(\operatorname{Sym}^n(X),\Gamma_\text{ext}^n\sK)\right) T^n \text{.}
\end{align*}





\subsection{}\label{III:2-13}

Le théorème équivaut donc aux formules des traces 
\[
  \tr\left(F^*,\R\Gamma_c(\operatorname{Sym}^n(X,\Gamma_\text{ext}^n \sK)\right) = \sum_{x\in\operatorname{Sym}^n(X)^F} \tr\left(F_x^*,\Gamma_\text{ext}^n\sK\right) \text{.}
\]
Dans le cas a), elles résultent de Rapport \ref{II:4-10}. La formule 
analogue dans le cas b) sera prouvée au \S\ref{III:4}. 










\section{Théorie d'Artin-Scheier}\label{III:3}





\subsection{}\label{III:3-1}

Dans ce qui suit, un faisceau cohérent sur un schéma $S$ sera toujours 
regardé comme un faisceau sur $S_\text{ét}$. Soit $S$ un schéma de 
caractéristique $p>0$. Si $G$ un faisceau localement constant de 
$\dZ/p$-vectoriels de rang fini, $\sG=G\otimes_{\dZ/p}\sO$ est un faisceau 
cohérent localement libre sur $S$. On notera $\Phi$ tant l'endomorphisme 
$f\mapsto f^p$ de $\sO$ que son produit tensoriel avec l'identité de 
$G$: $\phi:\sG\to\sG$. La théorie d'Artin-Scheier \cite[IX 3.5]{sga4} 
affirme que la suite 
\[\xymatrix{
  0 \ar[r] 
    & \dZ/p \ar[r] 
    & \sO \ar[r]^-{\Phi-1} 
    & \sO \ar[r] 
    & 0
}\]
est exacte. Par tensorisation avec $G$, on trouve un suite exacte 
\[\xymatrix{
  0 \ar[r] 
    & G \ar[r] 
    & \sG \ar[r]^-{\Phi-1} 
    & \sG \ar[r] 
    & 0
}\]





\subsection{}\label{III:3-2}

Supposons que $S$ soit un ouvert dans un schéma noethérien $\bar S$. 
Soient $j:S\hookrightarrow \bar S$ le morphisme d'inclusion, et $I$ un 
faisceau d'idéaux qui définit le fermé complémentaire 
$\bar S\setminus S$. Notons encore $\Phi$ le prolongement par fonctorialité 
de $\Phi$ à $j_*\sG$. 





\begin{lemma_}\label{III:3-3}
Il existe des prolongements cohérents $\bar\sG\subset j_*\sG$ de $\sG$ tels 
que 
\begin{equation}\label{III:eq:3-3-1}
  \Phi(\bar\sG)\subset I\cdot \bar\sG\text{.}
\end{equation}
Si $\bar\sG$ est un tel prolongement, la suite 
\begin{equation}\label{III:eq:3-3-2}
\xymatrix{
  0 \ar[r] 
    & j_! G \ar[r] 
    & \bar\sG \ar[r]^-{\Phi-1} 
    & \bar\sG \ar[r] 
    & 0
}
\end{equation}
ext exacte.
\end{lemma_}
\begin{enumerate}[a)]
  \item \emph{Existence}: Soit $\sG'\subset j_*\sG$ un prolongement cohérent 
    de $\sG$ à $\bar S$. Pour tout $n$, on a 
    $\Phi(I^n \sG')\subset I^{p^n}\Phi(\sG')$. Soit $n$ assez grand pour 
    que $I^{n(p-1)-1}\Phi(\sG')\subset \sG'$, et posons $\bar\sG=I^n\sG'$. 
    On a $\Phi(\bar\sG)\subset I^{p n}\Phi(\sG') 
    =I\cdot I^n\cdot I^{n(p-1)-1}\Phi(\sG')\subset I\cdot I^n\sG' 
    = I\bar\sG$. 
  \item \emph{Noyau de $\Phi-1$}: Sur $S$, l'assertion résulte de 
    \ref{III:3-1}; il reste à vérifier qu'une section de $x$ de 
    $\ker(\Phi-1)$, sur $U$ étale sur $\bar S$, s'annule dans un voisinage de 
    l'image réciproque de $\bar S\setminus S$. Puisque $x=\Phi(x)$, on a en 
    effet par \eqref{III:eq:3-3-1} $x\in I\bar\sG$; de plus, 
    $x\in I^n\bar\sG\Rightarrow x=\Phi(x)\in\Phi(I^n\bar\sG)\subset I^{n p}\bar\sG$, et $x$ est dans tous les $I^n\bar\sG$, donc nul au voisinage de 
    $\bar S\setminus S$.
  \item \emph{Surjectivité de $\Phi-1$}: On peut supposer $\bar S$ affine, en 
    écrire $\sG$ comme quotient de $\sO^n$; $\Phi$ se relève alors en une 
    application $p$-linéaire $\widetilde\Phi:\sO^n\to\sO^n$, et il suffit de 
    vérifier la surjectivité de $\widetilde\Phi-1$:
    \[\xymatrix{
      \sO^n \ar[r]^-{\widetilde \Phi-1} \ar[d] 
        & \sO^n \ar[d] \\
      \sG \ar[r]^-{\Phi-1} 
        & \sG 
    }\]
    Le faisceau $\sO^n$ est le faisceau des sections locales du schéma en 
    groupes $\dG_a^n$, et $\widetilde\Phi-1$ est induit par un homomorphisme 
    $\dG_a^n$ dans $\dG_a^n$. Cet homomorphisme est étale. Pour vérifier 
    qu'il est surjectif, il suffit de le voir après tout changement de base 
    $\spec(k)\to\bar S$ ($k$ algébriquement clos) et sur $k$ algébriquement 
    clos, tout homomorphisme étale entre groupes connexes de même dimension 
    est surjectif. Enfin, un morphisme étale et surjectif de 
    $\bar S$-schémas induit un épimorphisme entre les faisceaux 
    correspondant sur $\et{\bar S}$.
\end{enumerate}





\subsection{Frobenius et Frobenius}\label{III:3-4}

Le morphisme de Frobenius absolu $F_\text{abs}:S\to S$ est l'identité sur 
l'espace topologique sous-jacent à $S$, et $f\mapsto f^p$ sur le faisceau 
structural. Pour $U/S$ étale, on a un isomorphisme canonique 
$F_\text{abs}^* U=U$: le diagramme 
\[\xymatrix{
  U \ar[r]^-{F_\text{abs}} \ar[d] 
    & S \ar[d] \\
  S \ar[r]^-{F_\text{abs}} 
    & S
}\]
est cartésien. L'endomorphisme de site annelé 
$(\et{\bar S},\sO)\to(\et{\bar S},\sO)$ défini par $F_\text{abs}$ peut donc 
être décrit comme suit:
\begin{center}
  $F_\text{abs}^*$ est l'identité sur $\et S$; l'homomorphisme 
  $F_\text{abs}^*\sO=\sO\to\sO$ est $\Phi$.
\end{center}
Avec cette description, pour tout faisceau étale $\sG$ sur $S$, la 
correspondance de Frobenius $F_\text{abs}^*\sG\to\sG$ est l'identité. 

Se donner un morphisme $p$-linéaire de faisceaux quasi-cohérents 
$u:\sM\to\sN$ sur $S$ revient à se donner un morphisme linéaire 
$u':F_\text{abs}^*\sM\to\sN$. En particulier, avec les notations de 
\ref{III:3-5}, l'homomorphisme $\Phi:\sG\to\sG$ correspond à 
$\Phi':F_\text{abs}^*\sG\to\sG$. 





\begin{lemma_}\label{III:3-5}
Avec les notations de \ref{III:3-4}, le diagramme 
\[\xymatrix{
  F_\textnormal{abs}^* G \ar[r] \ar[d] 
    & G \ar[d] \\
  F_\textnormal{abs}^*\sG \ar[r]^-{\Phi'} 
    & \sG 
}\]
est commutatif.
\end{lemma_}

Il suffit de le vérifier localement. On peut donc supposer $G$ constant, 
auquel cas le lemme est évident. 

La même théorie vaut pour les puissances de $F_\text{abs}$.





\subsection{}\label{III:3-6}

Supposons maintenant que $\bar S$ soit un schéma sur $\dF_q$ ($q=p^f$) et 
conformément aux conventions de Rapport \S\ref{II:1}, écrivons 
$S_0,\bar S_0,G_0,\sG_0,\bar\sG_0$ au lieu de $S,\dots$. Le lemme 
\ref{III:3-5}, pour $F_\text{abs}^F:S_0\to S_0$, donne un diagramme commutatif 
\[\xymatrix{
  F^* G_0 \ar[r] \ar[d] 
    & G_0 \ar[d] \\
  F^*\sG_0 \ar[r]^-{(\Phi^f)'} \ar[r] 
    & \sG_0 
}\qquad\text{, puis}\qquad 
\xymatrix{
  F^*(j_! G_0) \ar[r] \ar[d] 
    & j_!G_0 \ar[d] \\
  F^*\bar\sG_0 \ar[r]^-{(\Phi^f)'}
    & \sG_0
}\text{,}
\]
puis, par extension des scalaires de $\dF_q$ à $\dF$ et passage à la 
cohomologie 
\[\xymatrix{
  F^*(j_! G) \ar[r]^-{F_*} \ar[d] 
    & j_!G \ar[d] \\
  F^*\bar\sG \ar[r]
    & \bar\sG
}\qquad\text{et}\qquad 
\xymatrix{
  \h^\bullet(\bar S,j_! G) \ar[r]^-{F^*} \ar[d] 
    & \h^\bullet(\bar S,j_!G) \ar[d] \\
  \h^\bullet(\bar S,\bar\sG) \ar[r]
    & \h^\bullet(\bar S,\bar\sG)
}\text{,}
\]
où la seconde ligne est obtenue par extension des scalaires de $\dF_q$ à 
$\dF$ de l'endomorphisme $\dF_q$-linéaire $\Phi^f$ de 
$\h^\bullet(\bar S_0,\bar G_0)$. 

La théorie d'Artin-Scheier fournit une suite exacte longue 
\[\xymatrix{
  \cdots \ar[r]
    & \h^i(\bar S,j_! G) \ar[r]
    & \h^i(\bar S,\bar\sG) \ar[r]^-{\Phi-1}
    & \h^i(\bar S,\bar\sG) \ar[r]
    & \cdots
}\]
où $\Phi$ est l'extension $p$-linéaire à 
$\h^\bullet(\bar S,\bar \sG)= \h^\bullet(\bar S_0,\bar\sG_0)\otimes_{\dF_q}\dF$ 
de l'endomorphisme $p$-linéaire $\Phi$ de $\h^\bullet(\bar S_0,\bar\sG_0)$. 





\subsection{}\label{III:3-7}

Si $\bar S_0$ est propre sur $\dF_q$, les $\h^i(\bar S,\bar\sG)$ sont de 
dimension finie et $\Phi-1$ est surjectif (cf. \ref{III:3-6}b). On a donc 
\[
  \h_c^i(S,G) = \ker\left(\Phi-1 : \h^i(\bar S,\bar G) \to \h^i(\bar S,\bar\sG)\right) \text{,}
\]
et $F^*$ est induit par l'extension $\dF$-linéaire à 
$\h^i(\bar S,\bar\sG) = \h^i(\bar S_0,\bar\sG_0)\otimes_{\dF_q}\dF$ de 
l'endomorphisme $\dF_q$-linéaire $\Phi^f$ de $\h^i(\bar S_0m\bar\sG_0)$. 

Appliquant le lemme \ref{III:3-8} ci-dessous, on en déduit l'identité 
\begin{equation}\label{III:eq:3-7-1}
  \tr\left(F^*,\h_c^i(S,G)\right) = \tr\left(\Phi^f,\h^i(\bar S_0,\bar\sG_0)\right) \text{.}
\end{equation}





\begin{lemma_}\label{III:3-8}
Soit $\Phi$ un endomorphisme $p$-linéaire d'un espace vectoriel $V$ sur 
$\dF_q$: $\phi(\lambda x) = \lambda^p\phi(x)$. On pose $F=\Phi^f$ ($F$ est 
linéaire) et on note encore $\Phi$ et $F$ les extensions respectivement 
$p$-linéaire de $\Phi$ et $F$ à $V\otimes_{\dF_q}\dF$. Alors, $F$ stabilise 
$\ker(\Phi-1)\subset V\otimes_{\dF_q}\dF$ et 
\[
  \tr\left(F,\ker(\Phi-1)\right) = \tr(F,V) = \tr\left(F,V\otimes_{\dF_q}\dF\right) \text{.}
\]
\end{lemma_}

Sur $V\otimes_{\dF_q}\dF$, on a $F\Phi = \Phi F$: l'endomorphisme $p$-linéaire 
$F\Phi-\Phi F$ s'annule sur $V$, donc partout. Écrivons $V=V'+V''$, avec $F$ 
inversible sur $V'$ et nilpotent sur $V''$. La théorie des endomorphismes 
$p$-linéaires assure que 
\[
  \ker(\Phi-1)\otimes_{\dF_q}\dF \iso V'\otimes_{\dF_q}\dF \text{.}
\]
Puisque $\tr(F,V'') = 0$, on a donc 
\begin{align*}
  \tr\left(F,\ker(\Phi-1)\right) 
    &= \tr\left(F,\ker(\Phi-1)\otimes_{\dF_q}\dF\right) = \tr\left(F,V'\otimes_{\dF_q}\dF\right) \\
    &= \tr\left(F,V\otimes_{\dF_q}\dF\right) = \tr(F,V) \text{.}
\end{align*}










\section{Formule des traces modulo \texorpdfstring{$p$}{p}}\label{III:4}





\begin{theorem_}\label{III:4-1}
Soient $X_0$ un schéma séparé de type fini sur $\dF_q$, $A$ un anneau 
noethérien commutatif réduit et de caractéristique $p$, et 
$\sK_0\in\ob\D_\textnormal{ctf}(X_0,A)$. On a 
\begin{equation}\label{III:eq:4-1-1}
  \sum_{x\in X^F} \tr\left(F_x^*,\sK_x\right) = \tr\left(F^*,\R\Gamma_c(X,\sK)\right) \text{.}
\end{equation}
\end{theorem_}





\subsection{Réduction au cas où \texorpdfstring{$A$}{A} est un corps fini}\label{III:4-2}

%NOTE: in this paragraph and others, (3.1.1) seems to be referring to (4.1.1)
Un argument standard de passage à a limite nous ramène au cas où $A$ est 
de type fini sur $\dZ/p$. Pour chaque idéal maximal $\fm$ de $A$, l'image de 
chaque membre de \eqref{III:eq:4-1-1} dans $A/\fm$ est donée par la même 
formule, avec $\sK_0$ remplacé par 
$\sK_0\lotimes_A A/\fm\in\ob\D_\text{ctf}(X_0,A/\fm)$ (pour le second membre, 
cf. Rapport \ref{II:4-12}). Les $A/\fm$ étant finis, et l'intersection des 
idéaux maximaux de $A$ étant réduit à $0$, on obtient la réduction 
annoncée.





\subsection{Réduction au cas où \texorpdfstring{$A=\dZ/p$}{A=Z/p}}\label{III:4-3}

Supposons que $A$ soit un corps fini, et notons $T_A'(\sK_0)$ et $T_A''(\sK_0)$ 
les deux membres de \eqref{III:eq:4-1-1}. Si, par restriction des scalaires, on 
regarde $\sK_0$ comme un complexe de faisceaux de $\dZ/p$-modules, on a 
\[
  T_{\dZ/p}'(\sK_0) = \tr_{A/(\dZ/p)} T_A'(\sK_0)
\]
et de même pour $T''$. Si \eqref{III:eq:4-1-1} vaut pour $A=\dZ/p$, on a donc 
\begin{equation}\label{III:eq:4-3-1}
  \tr_{A/(\dZ/p)} T_A'(\sK_0) = \tr_{A/(\dZ/p)} T_A''(\sK_0) \text{.}
\end{equation}
J'affirme que, pour tout $\lambda\in A$, on a même 
\begin{equation}\label{III:eq:4-3-2}
  \tr_{A/(\dZ/p)}\left(\lambda\cdot T_A'(\sK_0)\right) = \tr_{A/(\dZ/p)}\left(\lambda\cdot T_A''(\sK_0)\right) \text{,}
\end{equation}
et donc $T_A'(\sK_0) = T_A''(\sK_0)$. C'est clair pour $\lambda=0$. Pour 
$\lambda$ inversible, notons $A(\lambda)$ l'image réciproque sur $X$ du 
faisceau de $A$-modules libre de rang un sur $\spec(\dF_q)$ pour lequel 
$F^*=\lambda$. La formule \eqref{III:eq:4-3-2} n'est autre que 
\eqref{III:eq:4-3-1} pour $A(\lambda)\otimes_A \sK_0$. 





% originally theorem 4.1bis (i.e. theorem 4.1 again)
\begin{theorem_}\label{III:4-1bis}
Soient $\bar S_0$ une courbe projective et lisse, 
$j:S_0\hookrightarrow \bar S_0$ un ouvert dense de $\bar S_0$, et $G_0$ un 
faisceau localement constant de $\dZ/p$-espaces vectoriels de rang fini sur 
$S_0$. On a 
\begin{equation}\label{III:eq:4-4-1}
  \sum_{x\in S^F} \tr\left(F_x^*,G_x\right) = \sum_i (-1)^i \tr\left(F^*,\h_c^i(S,G)\right) \text{.}
\end{equation}
\end{theorem_}
\begin{proof}
Appliquons \ref{III:3-3} à $(S_0,\bar S_0,G_0)$. Le prolongement cohérent 
$\bar\sG_0$ de $\sG_0$ dont \ref{III:3-3} garantit l'existence est localement 
libre car il est sans torsion (on  $\bar\sG_0\subset j_*\sG_0$) et que 
$\bar S_0$  est régulier de dimension un. L'endomorphisme $q$-linéaire 
$\Phi^f$ de $\bar\sG_0$ peut s'interpréter comme un morphisme 
\[
  u : F^*\bar\sG_0 \to \bar\sG_0 \text{.}
\]
Il se factorise par $I\bar G$, pour $I$ définissant le fermé 
$\bar S\setminus S$. Soit $v:j_!G_0 \to \bar\sG_0$ l'unique prolongement de 
l'application naturelle $G_0 \to \sG_0 = G_0\otimes_{\dZ/p}\sO$. Le diagramme 
\begin{equation}\label{III:eq:4-4-2}
\xymatrix{
  F^* j_! G_0 \ar[r]^-{F^*} \ar[d]^-{F^* v} 
    & j_! G_0 \ar[d]^-v \\
  F^*\bar \sG_0 \ar[r]^-u 
    & \bar\sG_0
}
\end{equation}
Enfin, notons $u F^*$ l'application composée 
\[\xymatrix{
  \h^\bullet(\bar S,\bar\sG) \ar[r]^-{F^*} 
    & \h^\bullet(\bar S,F^*\bar\sG) \ar[r]^-u 
    & \h^\bullet(\bar S,\bar\sG) \text{,}
}\]
on par \eqref{III:eq:3-7-1}
\begin{equation}\label{III:eq:4-4-3}
  \tr\left(F^*,\h_c^i(S,G)\right) = \tr\left(u F^*,\h^i(\bar S,\bar\sG)\right) \text{.}
\end{equation}

%NOTE: trace down reference for Woodshole seminar
Pour chaque point fermé $i_x:x\hookrightarrow \bar  S$ de $\bar S$, posons 
$\bar\sG_x = i_x^*\bar\sG$. La formule des traces de Lefschetz en cohomologie 
cohérente (``Woodshole fixed point formula''), appliqué à $F$ et $u$, dit 
que 
\begin{equation}\label{III:eq:4-4-4}
  \sum_i (-1)^i \tr\left(u^* F^*, \h^i(\bar S,\bar\sG)\right) = \sum_{x\in \bar S^F} \frac{\tr(u_x,\sG_x)}{\det(1-d F_x)} \text{.}
\end{equation}
Au second membre
\begin{enumerate}[\indent a)]
  \item puisque $d F=0$, les dénominateurs valent 1;
  \item pour $x\in \bar S^F\setminus S^F$, l'endomorphisme $u_x$ de $\bar\sG_x$ 
    est nul, donc aussi $\tr(u_x,\bar\sG_x)$;
  \item pour $x\in S^F$, \eqref{III:eq:4-4-4} fournit une contribution 
    \[
      \tr(u_x,\bar\sG_x) = \tr(F_x^*,G_x) \text{.}
    \]
    L'identité \eqref{III:eq:4-4-1} résulte donc de \eqref{III:eq:4-4-3} 
    et \eqref{III:eq:4-4-4}. 
\end{enumerate}
\end{proof}





\subsection{Un contre-exemple}\label{III:4-5}

On montre par un exemple qu'on ne peut pas, dans \ref{III:4-1}, omettre 
l'hypothèse que $A$ soit réduit.

Soit $Y_0$ la courbe elliptique sur $\dF_2$ pour laquelle les valeurs propres 
de Frobenius sont $\pm \sqrt{-2}$. Elle a $3$ points rationnels, et est 
supersingulière. L'involution $\sigma:x\mapsto -x$ n'a donc qu'un seul point 
fixe, le point $0$. Le quotient de $Y_0$ par $\sigma$ est un droite projective; 
celui de $Y_0\setminus 0$ par $\sigma$ est donc une droite affine $\dF_0^1$, et 
$Y_0\setminus 0$ est un revêtement double non ramifié de $\dF_0^1$:
\[
  \pi:Y_0\setminus 0 \to \dA_0^1 \qquad \text{, $\pi=\pi\sigma$.}
\]

Le faisceau $\pi_*\dZ/2$ est un module libre de rang un sur l'algèbre $A$ du 
groupe à deux éléments $\{1,\sigma\}$ sur $\dZ/2$. Cette algèbre est 
isomorphe à celle des nombres sur $\dZ/2$. 

Le faisceau de $A$-modules $\pi_*\dZ/2$ est notre exemple:
\begin{enumerate}[\indent a)]
  \item On a 
    $\h_c^\bullet(\dA^1,\pi_*\dZ/2) = \h_c^\bullet(Y\setminus 0,\dZ/2)$. 
    Puisque $Y_0$ est supersingulière, on a $\h^i(Y,\dZ/2)=0$ pour $i>0$, et 
    la suite exacte longue 
    \[
      \cdots \to \h_c^i(Y\setminus 0,\dZ/2) \to \h^i(Y,\dZ/2) \to \h^i(\{0\},\dZ/2) \to \cdots
    \]
    montre que $\h_c^\bullet(Y\setminus 0,\dZ/2)=0$. On a donc 
    $\h_c^\bullet(\dA^1,\pi_*\dZ/2)=0$ et 
    \[
      \tr_A\left(F^*,\R\Gamma_c(\dA^1,\pi_*\dZ/2)\right) = 0 \text{.}
    \]
  \item Puisque $Y_0\setminus 0$ a deux points rationnels, en l'un des points 
    rationnels de $\dA_1^0$, la substitution de Frobenius est l'identité, et 
    en l'autre, c'est $\sigma$. On a donc 
    \[
      \sum_{x\in {\dA^1}^F} \tr\left(F^*,\pi_*\dZ/2\right) = 1+\sigma \text{,}
    \]
    et $1+\sigma\ne 0$.
\end{enumerate}

L'équation du revêtement $\pi$ est $y^2-y=x^3$.
% !TEX root = sga4.5.tex

\chapter{La classe de cohomologie associée à un cycle}\label{IV}
\footnote{par A.\ Grothendieck, rédigé par P.\ Deligne}




Cet exposé est inspiré de notes de Grothendieck, qui formaient un état 0 
de \cite[IV]{sga5}. On y définit la classe de cohomologie d'un cycle $X$ dans 
un schéma séparé lisse de type fini sur un corps et on prouve que 
l'intersection correspond au cup-produit.

Le Chapitre 1 contient quelques sorites généraux. Au Chapitre 2, on 
définit la classe d'un cycle, dans plusieurs situations plus générales 
que celle dite plus haut. La principale compatibilité non considérée est 
celle entre image directe d'un cycle et morphisme trace en cohomologie. Au 
Chapitre 3, on déduit de ce formalisme la formule des traces de Lefschetz pour 
un endomorphisme à points fixes isolés d'un schéma propre et lisse sur $k$ 
algébriquement clos -- et pour un endomorphisme de Frobenius d'une courbe. 

Nous faisons les conditions suivantes:
\begin{enumerate}[\indent 1)]
  \item ``schéma'' signifie schéma noethérien séparé (ceci est 
    largement un hypothèse de commodité).
  \item Dans le Chapitre 2, on fixe un entier $n$, et $n$ est inversible sur 
    tous les schémas considérés.
  \item Dans le Chapitre 3, on fixe un nombre premier $\ell$, et $\ell$ est 
    inversible sur tout les schémas considérés. La cohomologie utilisée 
    est toujours la cohomologie $\ell$-adique:
    \[
      \h^\bullet(X) = \dQ_\ell\otimes_{\dZ_\ell} \varprojlim \h^\bullet(X,\dZ/\ell^n) \text{.}
    \]
\end{enumerate}
Classes de cohomologie de cycles, morphismes traces, \ldots sont définis par 
passage à limite à partir du cas de coefficients finis $\dZ/\ell^n$ (cf. 
\cite[VI]{sga5}).




















\section{Cohomologie à support et cup--produits}\label{IV:1}

Ce paragraphe contient des rappels de topologie générale, que le 
lecteur est invité à ne consulter qu'au fur et à mesure des besoins.










\subsection{\texorpdfstring{$H^1$}{H1} et torseurs}\label{IV:1-1}





\subsubsection{}\label{IV:1-1-1}

Soit $\sF$ un faisceau abélien sur un site $X$. On sait que $\h^1(X,\sF)$ 
classifie les $\sF$-torseurs sur $X$. Nous normaliserons ($=$choisirons le 
signe) de l'isomorphisme $($ensemble des classes d'isomorphisme de 
$F$-torseurs$)\to \h^1(X,\sF)$ de telle sorte que pour toute suite exacte 
$0\to \sF\xrightarrow\alpha\sG\xrightarrow\beta\sH\to 0$ et tout 
$h\in\h^0(X,\sH)$, le $\sF$-torseur $\beta^{-1}(h)\subset \sG$, sur lequel 
$F$ agit par $(f,x)\mapsto \alpha(f)+x$, soit de classe $\partial h$. 





\subsubsection{}\label{IV:1-1-2}

Soit $\sP$ un $\sF$-torseur. Si $(U_i)$ est un recouvrement ouvert de $X$, et 
$p_i$ un section de $\sP$ sur $U_i$, on associe à $\sP$ le cocycle de 
\v{C}ech
\[
  p_{i j} = p_j - p_i \qquad \text{($p_{i j}\in \h^0(U_i\times U_j,\sF)$).}
\]
Si, selon la règle usuelle, on définit l'application 
$\operatorname{\check H}^\bullet(X,\sF)\to \h^\bullet(X,\sF)$ de telle sorte 
que ce soit un morphisme de $\delta$-foncteurs, l'image de 
$(p_{i j})\in \operatorname{\check H}^1(X,\sF)$ dans $\h^1(X,\sF)$ est la 
classe de $\sP$, telle que normalisée par \ref{IV:1-1-1}. 





\subsubsection{}\label{IV:1-1-3}

La définition de $\h^\bullet(X,\sF)$ est la suivante: pour $\sF^\bullet$ une 
résolution à composantes acycliques de $\sF$, 
$\h^i(X,\sF) = \h^i\Gamma(X,\sF^\bullet)$. La structure de $\delta$-foncteur 
s'obtient en associant à une suite exacte courte de faisceaux une suite 
exacte courte de résolutions qui reste exacte après application du 
foncteur $\Gamma$. Si $\sF^\bullet$ est une résolution de $\sF$, 
l'homomorphisme de connection $\partial$ associé à 
\[\xymatrix{
  0 \ar[r]
    & \sF \ar[r]
    & \sF^0 \ar[r]^-d 
    & \ker(d) \ar[r] 
    & 0
}\]
induit l'opposé de l'isomorphisme de définition 
$\h^1(X,\sF) = \Gamma\left(X,\ker(d)\right)/d \Gamma(X,\sF^0)$. 





\subsubsection{}\label{IV:1-1-4}

Soit $U$ une partie ouverte de $X$ (un sous-faisceau du faisceau final) et 
soit $D$ le ``fermé complémentaire.'' On sait que $\h_D^1(X,\sF)$ 
classifie les $\sF$-torseurs sur $X$, trivialisés sur $U$. Pour toute suite 
exacte courte $0\to \sF\xrightarrow\alpha \sG\xrightarrow\beta\sH\to 0$, et 
toute section à support dans $D$ $h\in\h_D^0(X,\sH)$, le torseur 
$\beta^{-1}(h)$, trivialisé sur $U$ par la section $0$, a pour classe 
$\partial h$.

La suite exacte longue de cohomologie à support
\[\xymatrix{
  \cdots \ar[r]^-\partial 
    & \h_D^i(X,\sF) \ar[r] 
    & \h^i(X,\sF) \ar[r]
    & \h^i(U,\sF) \ar[r]^-\partial 
    & \cdots
}\]
est définie à partir de la suite de foncteurs 
\[\xymatrix{
  0 \ar[r] 
    & \Gamma_D \ar[r] 
    & \Gamma \ar[r] 
    & \Gamma(U,-) \ar[r] 
    & 0
}\]
(exacte sur les faisceaux injectifs). Pour toute section $f\in \h^0(U,\sF)$, 
$\partial f\in \h_D^1(X,\sF)$ est la classe du torseur trivial $\sF$, 
trivialisé sur $U$ par la section $f$. 





\subsubsection{}\label{IV:1-1-5}

Soit $j:U\hookrightarrow X$. Si $\sF$ s'injecte dans $j_* j^* \sF$, la suite 
exacte 
\[\xymatrix{
  0 \ar[r] 
  & \sF \ar[r]
  & j_* j^* \sF \ar[r]
  & j_* j^* \sF / \sF \ar[r]
  & 0
}\]
fournit 
$\partial:\h^0(j_* j^*\sF/\sF) = \h_D^0(j_* j^* \sF/\sF) \to \h_D^1(X,\sF)$. 
L'application composée 
\[\xymatrix{
  \h^0(U,\sF) = \h^0(X,j_* j^*\sF) \ar[r] 
    & \h^0(X,j_* j^*\sF / \sF) \ar[r]^-\partial 
    & \h_D^1(X,\sF)
}\]
est l'opposée de l'application considérée en \ref{IV:1-1-4}. 





\subsubsection{}\label{IV:1-1-6}

On rappelle que si $\sL$ est un faisceau inversible sur un schéma $X$, le 
$\dG_m$-torseur correspondant est le faisceau $\operatorname{Isom}(\sO,\sL)$, 
sur lequel $\dG_m$ agit par 
$(\lambda,f)\mapsto f\circ (\lambda\cdot) = \lambda f$. On rappelle aussi que 
si $D$ est un diviseur de Cartier sur $X$, et $j:U\hookrightarrow X$ 
l'inclusion d'ouvert complémentaire, le faisceau inversible $\sO(D)$ est le 
sous-faisceau de $j_*\sO_U$ formé des sections locales $s$ telles que $s f$ 
soit dans $\sO_X$, pour $f$ une équation locale de $D$.










\subsection{Cup-produits}\label{IV:1-2}

Dans ce numéro, nous développons quelques remarques sur les cup-produits en 
cohomologie à support qui nous reservirons, dans [\nameref{V}], pour relier 
dualité de Poincaré des courbes et autodualité de la jacobienne.

\subsubsection{}\label{IV:1-2-1}

Soient $X$ un site, $Y$ un partie fermée de $X$, et $\sF,\sG$ deux faisceaux 
abéliens sur $X$. Par exemple: $X$ un schéma, $Y$ un sous-schéma fermé 
et $\sF,\sG$ des faisceaux sur $\et X$. Définissons un produit 
\begin{equation*}\tag{1.2.1.1}\label{IV:eq:1-2-1-1}
  \Gamma_Y(X,\sF)\otimes \Gamma(Y,\sG) \to \Gamma_Y(X,\sF\otimes\sG) \text{.}
\end{equation*}
Le produit d'une section $s$, à support dans $Y$, de $\sF$ par une section 
$t$ de $\sG$ sur $Y$ s'obtient comme suit: localement sur $X$, $t$ est la 
restriction à $Y$ d'une section $t'$ de $\sG$ sur $X$, et on forme le produit 
$s\otimes t'$. Il est à support dans $Y$, et ne dépend pas du choix de 
$t'$, ce qui légitime et permet de globaliser la définition.

Soit $i:Y\hookrightarrow X$ le morphisme d'inclusion. L'analogue local de 
\eqref{IV:eq:1-2-1-1} est le produit.
\begin{equation*}\tag{1.2.1.2}\label{IV:eq:1-2-1-2}
  i^!\sF\otimes i^*\sG \to i^!(\sF\otimes\sG) \text{,}
\end{equation*}
dont \eqref{IV:eq:1-2-1-1} se déduit par application de $\Gamma(Y,-)$. 





\subsubsection{}\label{IV:1-2-2}

Dérivons ces flèches. Soient $\sK$, $\sL$ et $\sM$ dans la catégorie 
dérivée, et une application bilinéaire $\sK\lotimes\sL\to\sM$. Par 
dérivation de \eqref{IV:eq:1-2-1-1}, on en déduit 
\begin{equation*}\tag{1.2.2.1}\label{IV:eq:1-2-2-1}
  \R\Gamma_Y(X,\sK) \lotimes \R\Gamma(Y,\sL) \to \R\Gamma_Y(X,\sM) 
\end{equation*}
induisant 
\begin{equation*}\tag{1.2.2.2}\label{IV:eq:1-2-2-2}
  \h_Y^i(X,\sK)\otimes \h^j(Y,\sL) \to \h_Y^{i+j}(X,\sM)
\end{equation*}
(le cup-produit). La flèche locale \eqref{IV:eq:1-2-1-2} fournit 
\begin{equation*}\tag{1.2.2.3}\label{IV:eq:1-2-2-3}
  \R i^! \sK\lotimes i^*\sL \to \R i^!\sM \text{,}
\end{equation*}
dont \eqref{IV:eq:1-2-2-1} se déduit par application de $\R\Gamma(Y,-)$. 





\subsubsection{}\label{IV:1-2-3}

Ci-dessus, j'ai passé sous silence les conflits qu'entraine l'usage dans une 
même formule de dérivations à droite ($\R\Gamma$) et à gauche 
($\lotimes$). 
\begin{enumerate}[\indent a)]
  \item Au numéro suivant, les dérivés droits considérés seront tous 
    de dimension cohomologique finie. Ceci permet de travailler 
    systématiquement dans les catégories dérivées $\D^-$. Pour disposer 
    de complexes à la fois plats et flasques, on utilise les résolutions 
    flasques canoniques comme en \cite[XVII]{sga4}. 
  \item Pour une théorie plus générale, il cesse d'être d'interpréter 
    les applications bilinéaires de $\sK$ et $\sL$ dans $\sM$ comme des 
    morphismes de $\sK\lotimes\sL$ dans $\sM$. Par exemple, 
    $\R\Gamma_Y(X,\sK)\lotimes\R\Gamma(Y,\sL)$ n'est pas défini si des deux 
    facteurs sont dans $\D^+$. Une solution est de travailler dans $\D^+$, de 
    définir 
    \[
      \operatorname{Bil}(\sK,\sL;\sM) = \varinjlim \hom(\sK'\otimes\sL',\sM') \text{,}
    \]
    où la limite est prise sur les quasi-isomorphismes $\sK'\iso \sK$, 
    $\sL'\iso \sL$, $\sM\iso \sM'$ ($\sK'$, $\sL'$, $\sM'$ bornés 
    inférieurement) et où $\hom$ est pour ``morphisme de complexes, à 
    homotopie près,'' et d'utiliser systématiquement de telles applications 
    bilinéaires, sans jamais mentionner de $\otimes$.  
\end{enumerate}




\subsubsection{Deuxième thème}\label{IV:1-2-4}

Soit $U$ un ouvert de $X$ et $j:U\hookrightarrow X$ le morphisme d'inclusion. 
Pour $\sK$ dans la catégorie dérivée, sur $U$, on pose 
$\R\Gamma_!(U,\sK)=\R\Gamma(X,j_!\sK)$. Pour $\sK$, $\sL$ et $\sM$ sur $U$, et 
une application bilinéaire $\sK\lotimes\sL\to\sM$, on veut définir 
\begin{equation*}\tag{1.2.4.1}\label{IV:eq:1-2-4-1}
  \R\Gamma(U,\sK)\lotimes\R\Gamma_!(U,\sL)\to\R\Gamma_!(U,\sM) \text{.}
\end{equation*}
Au niveau des faisceaux, et de leurs sections globales, un tel produit se 
déduit de l'isomorphisme 
$j_*\sF\otimes j_!\sG\xleftarrow\sim j_!(\sF\otimes \sG)$, mais il faut prendre 
garde au fait que $\R\Gamma_!$ n'est en général pas le dérivé du 
foncteur $\Gamma(X,j_!-)$. 

On commence par définir 
\[\xymatrix{
  \R j_* \sK\lotimes j_! \sL 
    & \ar[l]_-\sim j_!(\sK\lotimes \sL) \ar[r] 
    & j_!\sM \text{.}
}\]
Appliquant $\R\Gamma(X,-)$, on trouve 
\[\xymatrix{
  \R\Gamma(U,\sK)\lotimes\R\Gamma_!(U,\sL) = \R\Gamma(X,\R j_*\sK)\lotimes \R\Gamma(X,j_!\sL) \ar[r] 
    & \R\Gamma(X,j_! \sM) \text{.}
}\]

Ici encore, le conflit entre la gauche et la droite se résoudra au numéro 
suivant en travaillant dans $\D^-$. 





\subsubsection{Coda}\label{IV:1-2-5}

Soient $j:U\hookrightarrow X$ une partie ouverte et $i:Y\hookrightarrow U$ une 
partie fermée de $U$. Soit $\bar Y$ une fermé de $X$ tel que 
$\bar Y\cap U=Y$, (par exemple, le complément de $U\setminus Y$). Soit $\sK$ 
sur $U$. On pose $\R\Gamma_{Y!}(U,\sK)=\R\Gamma_!(Y,\R i^!\sK)$ où 
$\R\Gamma_!$ est relatif à l'inclusion de $Y$ dans $\bar Y$. Pour tout ouvert 
$V$ de $U$, contenant $Y$, $\R\Gamma_{Y!}(U,\sK)$ s'envoie dans 
$\R\Gamma_!(V,\sK)$: si on note encore $i$ l'inclusion de $\bar Y$ dans $X$, 
$j$ celle de $Y$ dans $\bar Y$ et $k$ celle de $V$ dans $X$, on a 
$\R i^!\sK = j^*\R i^! k_!(k^*\sK)$, d'où un morphisme 
$j_!\R i^!\sK \to \R i^! k_!(k^*\sK)$. Lui appliquant 
$\R\Gamma(\bar Y,-)$, on trouve 
$\R\Gamma_{Y!}(U,\sK)\to \R\Gamma_{\bar Y}(k_! k^* \sK) \to \R\Gamma(k_! k^*\sK) = \R\Gamma_!(V,\sK)$. 

Pour $\sK,\sL,\sM$ sur $U$, et une application bilinéaire 
$\sK\lotimes\sL\to\sM$, on veut définir 
\begin{equation*}\tag{1.2.5.1}\label{IV:eq:1-2-5-1}
  \h^n(U,\sK)\otimes \h_!^m(Y,\sL) \to \h_{Y!}^{n+m}(U,\sM)
\end{equation*}
(ce dernier groupe lui-même s'envoyant dans $\h_!^{n+m}(V,\sM)$). Dans la 
catégorie dérivée, il s'agit de définir 
\begin{equation*}\tag{1.2.5.2}\label{IV:eq:1-2-5-2}
  \R\Gamma_Y(U,\sK) \lotimes \R\Gamma_!(R,\sL) \to \R\Gamma_{Y!}(U,\sM) \text{.}
\end{equation*}
On identifie $\R\Gamma_Y$ à $\R\Gamma(Y,\R i^!\sK)$. Le produit cherché 
est alors du type \eqref{IV:eq:1-2-4-1} relatif au produit locale 
\eqref{IV:eq:1-2-2-3} sur $Y$: $\R i^! \sK\lotimes\sL \to \R i^! \sM$. 





\subsubsection{}\label{IV:1-2-6}

Ci-dessus, on a déroulé le sorite absolu. On a un sorite relatif 
parallèle, avec $\Gamma$ remplacé par $f_*$ par $f$ un morphisme $X\to S$. 










\subsection{La règle de Koszul}\label{IV:1-3}

Soient $A$ un anneau commutatif, et $(V_i)_{i\in I}$ une famille finie de 
$A$-modules gradués (ou $\dZ/2$-gradués). Rappelons la définition du 
produit tensoriel gradué $\bigotimes_{i\in I} V_i$, au sens de la règle de 
Koszul (cf. \cite[XVII 1.1]{sga4}). Pour chaque ordre total $a$ sur $I$, on va 
définir un module $V(a)$. On va aussi définir un système transitif 
d'isomorphismes $\varphi_{a b}:V(b)\iso V(a)$ et le produit tensoriel 
gradué des $V_i$ sera la ``valeur commune'' $\varprojlim V(a)$ de ce 
système de modules. On prend 
\begin{enumerate}[\indent a)]
  \item $V(a) = \bigotimes_{i\in I} |V_i|$ (produit-tensoriel ordinaire des 
    modules non gradués sous-jacents aux $V_i$).
  \item si les $x_i\in V_i$ sont homogènes, on prend 
    $\varphi_{a b}(\otimes x_i) = (-1)^N \otimes x_i$, où $N$ est la somme 
    des $\deg(x_i)\deg(x_j)$ étendue aux couples $(i,j)$ tels que 
    $i<_a j$ et $i>_b j$.
\end{enumerate}





\subsubsection*{Exemple 1}

Prenons $I=\{1,2\}$ et soient $a$ l'ordre où $1<2$, $b$ l'ordre où $1>2$. 
Soient $v_i\in V_i$, homogènes, et notons $v_1\otimes v_2$ (resp. 
$v_2\otimes v_1$) l'image dans le produit tensoriel gradué du produit des 
$v_i$ dans $V(a)$ (resp. $V(b)$). On a 
\[
  v_1\otimes v_2 = (-1)^{\deg(v_1)\deg(v_2)} v_2\otimes v_1
\]
(règle de Koszul). 





\subsubsection*{Exemple 2}

Si les $V_i$ sont tous de degré $1$, on a , reliant le module sous-jacent au 
produit tensoriel gradué, et le produit tensoriel ordinaire des modules 
$|V_i|$ sous-jacents aux $V_i$, un isomorphisme canonique 
\[
  \left|\bigotimes V_i\right| \simeq \bigotimes |V_i|\otimes_{\dZ}\bigwedge^{|I|}\dZ^I \text{.}
\]





\subsubsection*{Exemple 3}

Pour $A$ un corps, et $X_i$ une famille finie d'espaces, la formule de 
K\"unneth s'écrit 
$\h^\bullet\left(\prod X_i,A\right) = \bigotimes \h^\bullet(X_i,A)$. 

Soit $V^\vee$ le dual gradué de $V$. L'application canonique 
\begin{equation*}\tag{1.3.1}\label{IV:eq:1-3-1}
  V^\vee\otimes V = V\otimes V^\vee \to A
\end{equation*}
est $v'\otimes v\to v'(v)$. 

On suppose maintenant que $A$ est un corps, et on ne considère que des 
espaces vectoriels de dimension finie. L'isomorphisme canonique 
\begin{equation*}\tag{1.3.2}\label{IV:eq:1-3-2}
  W\otimes V^\vee \to \hom(V,W) 
\end{equation*}
est $w\otimes v'\mapsto \left(v\mapsto w\cdot v'(v)\right)$. Via cet 
isomorphisme, la composition $\hom(Y,Z)\otimes \hom(X,Y)\to \hom(X,Z)$ 
s'identifie au morphisme induit par \eqref{IV:eq:1-3-1}: 
$Z\otimes Y^\vee\otimes Y\otimes X^\vee \to Z\otimes X^\vee$. 

La trace de $f:V\to V^\vee$ (nulle pour $f$ homogène de degré $\ne 0$) est 
l'image de $f$ par 
\begin{equation*}\tag{1.3.4}\label{IV:eq:1-3-4}
  \tr:\hom(V,V) {\xleftarrow\sim}_\text{\eqref{IV:eq:1-3-2}} V\otimes V^\vee = V^\vee \otimes V \to_\text{\eqref{IV:eq:1-3-1}} A .
\end{equation*}

On vérifie aisément que, pour $f$ de degré $0$, 
\begin{equation*}\tag{1.3.5}\label{IV:eq:1-3-5}
  \tr(f,V) = \sum (-1)^i \tr(f,V^i) \text{.}
\end{equation*}

Si on exprime que les deux morphismes composés de morphismes 
\eqref{IV:eq:1-3-1} $V^\vee\otimes V\otimes W^\vee\otimes W\to k$ commutent, on 
trouve que, pour $f:V\to W$ et $g:W\to V$ homogènes, on a 
\begin{equation*}\tag{1.3.6}\label{IV:eq:1-3-6}
  \tr(f g) = (-1)^{\deg(f)\deg(g)} \tr(g f) \text{.}
\end{equation*}




















\section{La classe de cohomologie associée à un cycle}\label{IV:2}










\subsection{La classe d'un diviseur}\label{IV:2-1}





\subsubsection{}\label{IV:2-1-1}

Soit $D$ un diviseur de Cartier dans un schéma $X$. Hors de $D$, le faisceau 
inversible $\sO(D)$ est trivialisé par la section $1$. La classe 
$\cl(D)$ de $D$, dans $\h_D^1(X,\dG_m)$, est la classe du 
$\dG_m$-torseur trivialisé sur $X\setminus D$ correspondant 
(\ref{IV:1-1-6} et \ref{IV:1-1-4}). 

Soit $\partial:\h^i(X\setminus D,\dG_m)\to \h_D^i(X,\dG_m)$ le morphisme 
\ref{IV:1-1-4}. Si $D$ admet une équation globale $f$, la multiplication par 
$f$ est un isomorphisme de $\sO(D)$, trivialisé par $1$ sur $X\setminus D$, 
avec $\sO$, trivialisé par $f$ sur $X\setminus D$. D'après \ref{IV:1-1-4}, 
on a donc 
\begin{equation*}\tag{2.1.1}\label{IV:eq:2-1-1}
  \cl(D) = \partial f \text{.}
\end{equation*}

Pour tout morphisme $u:X'\to X$ tel que $u^* D$ soit encore un diviseur de 
Cartier (i.e., $u^{-1} D$ disjoint de $\operatorname{Ass}(X')$), on a 
$\cl(u^* D) = u^*\cl(D)$. Si on voulait une telle 
fonctorialité pour tout morphisme $u$, il faudrait considérer non pas des 
diviseurs de Cartier, mais plus généralement des faisceaux inversibles 
munis d'une section.

Rappelons que l'entier $n$ est dorénavant supposé inversible sur les 
schémas considérés. Soit 
$\partial:\h_D^i(X,\dG_m) \to \h_D^{i+1}(X,\dmu_n)$ le cobord pour la suite 
exacte de Kummer $0 \to \dmu_n \to \dG_m \to \dG_m \to 0$.





\begin{definition}\label{IV:2-1-2}
La \emph{classe $\cl_n(D)$} de $D$ dans $\h_D^2(X,\dmu_n)$ est 
$\partial \cl(D)$. 
\end{definition}

Quand il n'y aura pas de risque de confusion, on omettra la mention de $n$. 





\subsubsection{}\label{IV:2-1-3}

Le diagramme 
\[\xymatrix{
  \h^0(X\setminus D,\dG_m) \ar[r]^-\partial \ar[d]^-\partial 
    & \h_D^1(X,\dG_m) \ar[d]^-\partial \\
  \h^1(X\setminus D,\dmu_n) \ar[r]^-\partial 
    & \h_D^2(X,\dmu_n)
}\]
est anticommutatif. Si $D$ admet une équation globale $f$, 
$\cl_n(D)$ est donc l'opposé de l'image par $\partial$ de la 
classe dans $\h^1(X\setminus D,\dmu_n)$ de $\dmu_n$-torseur des racines 
$n$-ièmes de $f$. 





\begin{proposition}\label{IV:2-1-4}
Soit $i$ l'inclusion de $D$ dans $X$. Si $D$ et $X$ sont réguliers, les 
faisceaux de cohomologie à support $\R^p i^! \dmu_n$ sont nuls pour $p=0,2$, 
et $\R^2 i^!\dmu_n = \underline{\dZ/n}$, engendré par 
$\cl_n(D)$. 
\end{proposition}

Il suffit de prouver que, pour $X$ strictement locale et $D$ défini par un 
paramètre régulier, on a $\h_D^p(X,d\mu_n) = 0$ pour $p=0,1$ et 
$\h^2(X,\dmu_n) = \dZ/n$ engendré par $\cl(D)$. Notant par 
$\sim$ la cohomologie réduite, on a 
$\widetilde\h^{p-1}(X\setminus D,\dmu_n)\iso \h_D^p(X,\dmu_n)$. L'assertion 
pour $p=0,1$ exprime que $D$ ne disconnecte par $X$, et pour $p=2$ résulte, 
via \ref{IV:2-1-3}, du lemme d'Abhyankar. 

Ceci est un analogue partiel du théorème relatif 
(\hyperref[I]{Cohomologie étale}, \ref{I:5-3-4}). Grothendieck conjecture 
que les $\R^p i^! \dmu_n$ sont nuls pour $p\ne 2$, du moins pour $X$ excellent 
(conjecture de pureté), mais ceci n'est connu qu'en caractéristique $0$ 
(\cite[XIX]{sga4}). 




\begin{theorem}[Compatibilité fondamentale]\label{IV:2-1-5}
Soient $X$ uen courbe lisse sur un corps algébriquement clos $k$, $P$ un 
point fermé de $X$ et $\tr$ le composé $\h_P^2(X,\dmu_n) \to \h_c^2(X,\dmu_n) \xrightarrow{\tr} \dZ/n$. On a 
\[
  \tr \cl(P) = 1 \text{.}
\]
\end{theorem}

Soit $\bar X$ la courbe projective et lisse complétant $X$. La formule 
exprime que le faisceau inversible $\sO(P)$ sur $\bar X$ est de degré $1$. 










\subsection{Méthode cohomologique}\label{IV:2-2}





\subsubsection{}\label{IV:2-2-1}

Soit $X$ un schéma (noethérien). Rappelons qu'un sous-schéma $Y$ de $X$ 
est dit d'intersection complète locale, de codimension $c$, si, localement 
(sur $Y$), il est défini par une suite régulière de $c$ équations dans 
$X$. Pour $X$ le spectre d'un anneau locale $A$ d'idéal maximal $\fm$ et 
$Y$ d'idéal $\fa$, cela signifie que $\exp^i(A/\fa,A) = 0$ pour 
$i<\dim(\fa/\fm\fa)=c$, et toute suite d'éléments de $\fa$, d'image dans 
$\fa/\fm\fa$ une base de $\fa/\fm\fa$, est une suite régulière 
d'équations pour $Y$. 





\subsubsection{}\label{IV:2-2-2}

Soit $i:Y\hookrightarrow X$ d'intersection complète locale, de codimension 
$c$. On se propose de définir une classe fondamentale locale 
$\cl(Y)$ qui soit une section globale du faisceau de cohomologie à support 
$\R^{2c}i^! \dZ/n(c)$ (rappelons que $\dZ/n(c)=\dmu_n^{\otimes c}$). 

Localement, $Y$ est l'intersection d'une suite de $c$ diviseurs $D_i$, et on 
définit $\cl(Y)$ comme le cup-produit des $\cl(D_i)$. Chaque $\cl(D_i)$ est 
à support dans $D_i$, leur produit est à support dans $Y$. Que, localement, 
ce produit ne dépende pas du choix des $D_i$, résulte de \ref{IV:2-2-3} 
ci-dessous et des propriétés d'invariance suivantes:
\begin{enumerate}[\indent a)]
  \item compatibilité à la localisation;
  \item indépendance de l'ordre des $D_i$ (les $\cl(D_i)$ sont de degré 
    $2$, pair, donc le cup-produit est commutatif);
  \item le produit ne dépend que du ``drapeau'' 
    $D_1\supset D_1\cap D_2\supset \cdots\supset Y$. 
\end{enumerate}
Pour prouver c), on note l'existence d'un produit (variante de \ref{IV:1-2-1}) 
\[
  \h_{D_1}^\bullet(X)\otimes \h_{D_1\cap D_2}^\bullet(D_1)\otimes \cdots \otimes \h_Y^\bullet(D_1\cap \cdots \cap D_{c-1}) \to \h_Y^\bullet(X) \text{;}
\]
le produit des $\cl(D_i)$ est encore le produit des ($\cl(D_i)$ restreint à 
$D_1\cap \cdots \cap D_{i-1})=(\cl(D_1\cap \cdots \cap D_i)$ dans 
$D_1\cap \cdots \cap D_{i-1})$. 





\begin{lemma}\label{IV:2-2-3}
Soient $A$ un anneau locale d'idéal maximal $\fm$ et $u=(u_1,\dots,u_c)$, 
$v=(v_1,\dots,v_c)$ deux suites régulières engendrant le même idéal 
$\fa$. Il existe alors une suite $w_i$ ($1\leqslant i\leqslant N$) de telles 
suites, les reliant, telle que $w_{i+1}$ se déduise de $w_i$ par une 
permutation, ou en ne changeant que le dernier élément.
\end{lemma}

Puisqu'on dispose des permutations, il reviendrait au même de se permettre de 
changer un seul élément, plutôt que le dernier. Par \ref{IV:2-2-1}, on se 
ramène alors à vérifier que, dans l'espace vectoriel $\fa/\fm\fa$, on 
peut passer d'une base à une autre par une suite de permutations et de 
changements de base ne modifiant qu'un seul vecteur. Le groupe linéaire est 
en effet engendré par les matrices diagonales, élémentaires et de 
permutation. 





\subsubsection{}\label{IV:2-2-4}

Les mêmes méthodes permettent de définir une classe fondamentale locale 
pour tout $Y\subset X$ localement définissable par $c$ équations (la où 
moins d'équations suffisent, la classe est nulle). 





\subsubsection{}\label{IV:2-2-5}

On peut passer d'une telle classe fondamentale locale, dans 
$\h^0\left(Y,\R^{2 c} i^!\dZ/n(c)\right)$, à une classe fondamentale globale, 
dans $\h_Y^{2 c}\left(X,\dZ/n(c)\right)$, lorsqu'on dispose de résultats de 
semi-pureté:





\begin{proposition}\label{IV:2-2-6}
\begin{enumerate}[(i)]
  \item Si $\R^p i^!\dZ/n=0$ pour $p<2 c$, alors 
    \[
      \h_Y^{2 c}\left(X,\dZ/n(c)\right) \iso \h^0\left(Y,\R^{2 c} i^! \dZ/n(c)\right) \text{.}
    \]
  \item Soit $Z$ une partie fermée de $Y$, de complément $V$ dans $Y$, et 
    $k$ l'inclusion de $Z$ dans $X$. Si $\R^p i^! \dZ/n = 0$ pour 
    $p\leqslant 2 c$, alors 
    \[
      \h_Y^{2 c}\left(X,\dZ/n(c)\right) \hookrightarrow \h_V^{2 c}\left(X,\dZ/n(c)\right) \text{.}
    \]
    Si $\R^p i^!\dZ/n=0$ pour $p\leqslant 2 c+1$, cette flèche est une 
    isomorphisme.
\end{enumerate}
\end{proposition}

L'hypothèse $\R^p i^!\dZ/n=0$ équivaut à $\R^p i^!\dZ/n(c)=0$. Ceci dit, 
(i) se lit sur la suite spectrale 
$\h^p(Y,\R^q i^!)\Rightarrow \h_Y^{p+q}(X,-)$. Si $k_1$ l'inclusion de $Z$ dans 
$Y$, on a $k=i k_1$, d'où $\R k^! \dZ/n=\R k_1^! \R i^!$, et la suite exacte 
longue de cohomologie pour $Z\subset Y$ fournit une suite exacte longue 
\[\xymatrix{
  \cdots \ar[r] 
    & \h^i\left(Z,\R k^! \dZ/n(c)\right) \ar[r] 
    & \h_Y^i\left(X,\dZ/n(c)\right) \ar[r] 
    & \h_V^i\left(X,\dZ/n(c)\right) \ar[r] 
    & \cdots
}\]
dont (ii) résulte. 





\subsubsection{Amplification}\label{IV:2-2-7}

La proposition \ref{IV:2-2-6} reste valable pour $i$ un quelconque morphisme 
séparé de type fini, et $2 c$ un entier (positif ou négatif) quelconque, 
pour autant qu'on y remplace $\h_Y^{2 c}\left(X,\dZ/n(c)\right)$ par 
$\h^{2 c}\left(Y,\R i^!\dZ/n(c)\right)$, et de même pour $\h_V$. 

Les résultats de semi-pureté suivants résultent de 
\cite[1.8, 1.10, 1.15]{sga2}. Nous rappellerons leur preuve. 





\begin{theorem}\label{IV:2-2-8}
Soit un morphisme de $S$-schémas de type fini 
\[\xymatrix{
  X \ar[r]^-i \ar[dr]_-f 
    & X \ar[d]^-g \\
  & S
}\]
On suppose $X$ lisse, purement de dimension relative $N$ et que $Y$ est fibre 
par fibre de dimension $\leqslant d$. Soit $c=N-d$. 
\begin{enumerate}[\indent (i)]
  \item On a $\R^p i^!\dZ/n=0$ pour $p<2 c$; de même pour $\dZ/n$ remplacé 
    par un faisceau $g^* \sF$.
  \item Si, au dessus d'un ouvert dense $U$ de $S$, $Y$ est fibre par fibre de 
    dimension $<d$, on a $\R^p i^!\dZ/n=0$ pour $p\leqslant 2 c$. Si de plus le 
    complément $Z$ de $U$ ne disconnecte pas localement $S$, on a 
    $\R^pi^!\dZ/n=0$ pour $p\leqslant 2 c+1$.
\end{enumerate}
\end{theorem}

Puisque $\R g^! \sF=g^*\sF(N)[2N]$, la formule de transitivité 
$\R i^!\R g^!=\R f^!$ montre que 
$\R^{2 c+q} i^!(g^*\sF)=0\Leftrightarrow \R^{-2 d+q}f^!(\sF)$. Ceci nous 
ramène à étudier $f$, i.e. à supposer que $X=S$. L'assertion (i) est 
alors \cite[XVIII 3.17]{sga4}. 

Soit $Y'$ l'image inverse de $Z$ dans $Y$:
\[\xymatrix{
  Y' \ar[r]^-v \ar[d]^-f 
    & Y \ar[d]^-f \\
  Z \ar[r]^-u 
    & S
}\]
D'après (i), les $\R^p f^! \dZ/n$ sont à support dans $Y'$ pour 
$p\leqslant -2 d+1$; et la suite spectrale 
$\R^a v^!\R^b f^! \Rightarrow \R^{a+b} (f v)^!$ montre qu'ils coïncident avec 
les $\R^p(f v)^!\dZ/n=\R^p(u f)^!\dZ/n$. Appliquant (i) à $Y'/Z$ et à la 
suite spectrale $\R^a f^!\R^b u^! \Rightarrow \R^{a+b} (u f)^!$, on trouve que 
(ii) résulte de la nullité de $\R^b u^! \dZ/n$ pour $b=0$, ou $b=0$ et $1$ 
selon le cas. 





\subsubsection{}\label{IV:2-2-9}

Grothendieck conjecture l'analogue absolu suivant de \ref{IV:2-2-8} (conjecture 
de semi-pureté -- une conséquence de la conjecture de pureté): pour $Y$ 
de codimension $\geqslant c$ dans $X$ régulier, on a $\h_Y^i(X,\dZ/n)=0$ pour 
$i<2 c$, tout au moins si $X$ est excellent. 





\subsubsection{}\label{IV:2-2-10}

On est maintenant à pied d'œuvre pour définir la classe d'un cycle $Y$ de 
codimension $c$ dans $X$ lisse sur un corps $k$. On écrit $Y=\sum d_i Y_i$, 
où les $Y_i$ sont réduits irréductibles. Un ouvert $U_i$ de $Y_i$, de 
complémentaire de codimension $>c$, est alors d'intersection complète 
locale dans $X$. Ceci permet de définir la classe fondamentale locale de 
$U_i$. D'après \ref{IV:2-2-6} et \ref{IV:2-2-8}, celle-ci provient d'une 
unique classe fondamentale $\cl(Y_i)\in \h_{Y_i}^{2 c}\left(X,\dZ/n(c)\right)$, 
et on pose 
\[
  \cl(Y) = \sum d_i \cl(Y_i) \in \h_{|Y|}^{2 c}\left(X,\dZ/n(c)\right) \text{.}
\]





\subsubsection{}\label{IV:2-2-11}

En géométrie analytique complexe, une construction de Baum, Fulton et 
Mac Pherson \cite{bfm75} permet de définir sans restriction la classe de 
cohomologie d'une intersection complète locale $Y\subset X$. Supposons $Y$ 
purement de codimension $c$, et soit $\sN$ le fibré vectoriel sur $Y$ fibré 
normal de $Y$ dans $X$. Ses sections locales sont celles de $(\sI/\sI^2)^\vee$, 
d'où $\sI$ est le faisceau d'idéaux de $Y$. Rappelons que le faisceau des 
fonctions $C^\infty$ sur $X$ est défini localement, en terme de plongements 
locaux de $X$ dans $\dC^n$, comme la restriction à $X$ du quotient du 
faisceaux des fonctions $C^\infty$ sur $\dC^n$ par l'idéal engendré par les 
parties réelles et imaginaires des équations qui définissent $X$. Le 
fibré $\sN$ se prolonge en un fibré vectoriel complexe $C^\infty N$ sur un 
voisinage $U$ de $Y$ dans $X$, et pour $U$ assez petit, il existe des sections 
$f$ de $N$, de lieu des zéros $Y$, et telles que, sur $Y$, $d f:\sN\to N$ 
soit l'identité. Deux choix de $N$ et $f$ sont homotopes sur $Y$ assez petit. 

La classe de cohomologie $\cl(U)$ de la section $0$ de $N$ (notée $z:U\to N$) 
est définie: $U$ est d'intersection complète locale dans $N$, et 
$\R^i z^!\dZ = 0$ pour $i\ne 2 c$, $\R^{2 c} z^!\dZ = \dZ$. On dispose de 
$f^*:\h_U^\bullet(N,\dZ) \to \h_Y^\bullet(X,\dZ)$ et on pose 
\[
  \cl(Y) = f^*\cl(U) \text{.}
\]

La même construction marche dès qu'on a sur un sous-espace analytique $Y$ 
de $X$ la structure normale suivante: un faisceau localement libre $\sC$ de 
rang $c$ sur $Y_\text{red}$, et un épimorphisme 
$\sC\to \sI/\sI\cdot \sI_\text{red}$. 










\subsection{Méthode homologique}\label{IV:2-3}





\subsubsection{}\label{IV:2-3-1}

Soit $f:Y\to X$ un morphisme plat de type fini, à fibres de dimension 
$\leqslant d$. Dans \cite[XVIII 2.9]{sga4}, nous avons défini un morphisme 
trace 
\begin{equation*}\tag{2.3.1.1}\label{IV:eq:2-3-1-1}
  \tr_f:\R^{2 d} f_! \dZ/n(d) \to \dZ/n \text{.}
\end{equation*}
On a $\R^i f_!\dZ/n(d)=0$ pour $i>2 d$, et $\R^i f^!\dZ/n=0$ pour $i<-2 d$. 
Ceci, et l'adjonction entre $\R f_!$ et $\R f^!$, fournissent des isomorphismes 
\begin{equation*}\tag{2.1.3.2}\label{IV:eq:2-1-3-2}
\begin{aligned}
  \hom\left(\R^{2 d}f_!\dZ/n(d),\dZ/n\right) 
    &= \hom\left(\R f_!\dZ/n(d),\dZ/n[-2 d]\right) \\
    &= \hom\left(\dZ/n,\R f^!\dZ/n(-d)[-2 d]\right) \\
    &= \h^0\left(Y,\R^{-2 d}f^!\dZ/n\right) \text{.}
\end{aligned}
\end{equation*}
Au moins sur $S$ un point, l'image de $\tr_f$ dans les deux derniers groupes 
mérite le nom de classe fondamentale de $Y$ (en homologie). 

On note encore $\tr_f$ l'image de $\tr_f$ dans le second groupe, et les 
morphismes qui s'en déduisent par fonctorialité. Par exemple, pour $Y/S$ 
propre, le morphisme 
\begin{equation*}\tag{2.3.1.3}\label{IV:eq:2-3-1-3}
\xymatrix{
  \h^i\left(Y,\dZ/n(d)\right) = \h^i\left(S,\R f_!\dZ/n(d)\right) \ar[r] 
    & \h^{i-2 d}(S,\dZ/n) \text{.}
}
\end{equation*}

Supposons $Y$ contenu dans $X$ lisse sur $S$, purement de dimension relative 
$N$, et posons $c=N-d$. 
\[\xymatrix{
  Y \ar@{^{(}->}[r]^-i \ar[dr]_-f 
    & X \ar[d] \\
  & S
}\]
On a $\R f^! = \R i^!\R g^!$, et $\R g^! \dZ/n=\dZ/n(-N)[-2 N]$. La classe 
fondamentale de $Y$ s'identifie à un élément de 
$\h^0\left(Y,\R i^!\dZ/n(c)[2 c]\right)=\h_Y^{2 c}\left(X,\dZ/n(c)\right)$, la 
classe $\cl(Y)$ de $Y$ dans $X$. Nous verrons plus loin qu'elle ne dépend que 
de $Y\subset X$, non de la projection de $X$ sur $S$, et que pour $Y$ 
d'intersection complète locale, elle induit la classe locale du numéro 
précédent. 

Explicitons l'isomorphisme 
\[
  \h_Y^{2 c}\left(X,\dZ/n(c)\right) \iso \hom\left(\R^{2 d}f_!\dZ/n(d),\dZ/n\right)
\]
qui transforme $\cl(Y)$ en $\tr_f$: via les isomorphismes 
\begin{align*}
  \h_Y^{2 c}\left(X,\dZ/n(c)\right) 
    &= \h_Y^{2 c}\left(\R i^!\dZ/n(c)\right) \\
    &= \hom_Y\left(\dZ/n(d)[2 d],\R i^!\dZ/n(N)[2 N]\right) \\
    &= \hom_X\left(\dZ/n(d)[2 d]_Y,\dZ/n(N)[2 n]\right) \text{,}
\end{align*}
c'est la ligne supérieure de 
\[\xymatrix{
  \h_Y^{2 c} \ar[r] \ar[dr]_-{(3)} 
    & \hom\left(\R f_!\dZ/n(d)[2 d],\R f_! \R i^!\dZ/n(N)[2 n]\right) \ar[r]^-{(1)} \ar[d]^-{\tr_i} \ar@{}[dr]|-{(2)}
    & \hom\left(\R f_!\dZ/n(d)[2d],\dZ/n\right) \ar@{=}[d] \\
  & \hom\left(\R g_!\dZ/n(d)[2 d]_Y,\R g_!\dZ/n(N)[2 N]\right) \ar[r]^-{\tr_g} 
    & \hom\left(\R g_!\dZ/n(d)[2 d]_Y,\dZ/n\right) \text{,}
}\]
où (1) est la flèche d'adjonction $\R f_!\R f^!\to \text{id}$. La 
commutativité (2) exprime que l'isomorphisme $\R f^!=\R i^!\R g^!$ est 
défini par adjonction. Enfin, la flèche déduite de (3)
\[
  \h_Y^{2c}\left(X,\dZ/n(c)\right) \otimes \R^{2 d} g_!\dZ/n(d) \to \R^{2 N} g_!\dZ/n(N)
\]
s'interprète comme un cup-produit (cf. \ref{IV:1-2}). 





\begin{definition}\label{IV:2-3-2}
La classe $\cl(Y)\in\h_Y^{2 c}\left(X,\dZ/n(c)\right)$ a pour propriété 
caractéristique que, pour tout section locale $u$ de $\R^{2 d}f_! \dZ/n(d)$, 
on a 
\[
  \tr_f(u) = \tr_g\left(\cl(Y)\smallsmile u\right) \text{.}
\]
\end{definition}

Du fait que $\cl(Y)$ ait déjà une image $\tr_f$ dans 
$\hom\left(\R f_!\dZ/n(d)[2 d],\dZ/n\right)$, le formule \ref{IV:2-3-2} vaut 
pour les flèches déduites de $\tr_f$ par fonctorialité. Par exemple, 
pour $X$ sur $S$ propre et $u\in \h^\bullet(Y,\dZ/n)$, la formule vaut dans 
$\h^\bullet\left(S,\dZ/n(-d)\right)$. Pour $v\in \h^\bullet(X,\dZ/n)$, elle 
donne 
\[
  \tr_f(i^* v) = \tr_g\left(\cl(Y)\smallsmile v\right) \text{,}
\]
où le cup-produit peut se calculer dans $\h^\bullet(X,\dZ/n)$. 





\subsubsection{}\label{IV:2-3-3}

Nous allons définir les morphisme trace, et donc la classe $\cl(Y)$, sous des 
hypothèses plus générales. Soit donc un diagramme 
\[\xymatrix{
  Y \ar@{^{(}->}[r] \ar[dr]_-f 
    & X \ar[d]^-g \\
  & S
}\]
avec $g$ lisse, purement de dimension relative $N$ et $Y$ fermé dans $X$, 
fibre par fibre de dimension $\leqslant d$. On munit de plus $Y$ d'un 
``poids'' $\sK$ du type suivant: $\sK\in\D_{\text{parf},X}^b$ est un 
complexe borné de faisceaux de $\sO$-modules sur $X$, de $\tor$-dimension 
finie sur $S$ (ou sur $X$, cela revient au même) et dont les faisceaux de 
cohomologie soient cohérents et à support dans $Y$. On se propose de 
définir un morphisme $\tr_{f,\sK}:\R^{2 d}f_! \dZ/n(d) \to \dZ/n$. Pour $Y$ 
plat sur $S$ et $\sK=\sO_Y$, ce sera le morphisme trace précédent. En 
général, il ne dépend que des longueurs de $\sK$ aux points 
génériques $y$ des composantes irréductibles de 
$Y(\operatorname{lg}_y(\sK) = \sum (-1)^i \operatorname{lg} \h^i(\sK)_y)$. 
% NOTE: notation is not clear, is it a calligraphic H?
Enfin, on notera $\cl(Y,\sK)$ la classe à support dans $Y$ telle que 
\begin{equation*}\tag{2.3.3.1}\label{IV:eq:2-3-3-1}
  \tr_g\left(\cl(Y,\sK)\smallsmile u\right) = \tr_{f,\sK}(u) \text{.}
\end{equation*}





\subsubsection{}\label{IV:2-3-4}

La construction de $\tr_{f,\sK}$ est parallèle à celle de 
\cite[XVIII.2]{sga4}; nour n'en indiquerons que les grandes lignes.

\paragraph{A}
$d=0$ ($f$ quasi-fini). Soit $x$ un point géométrique de $Y$. $s$ son image 
dans $S$, $\sK_{(x)}$ l'image réciproque de $\sK$ sur le localisé stricte 
de $X$ en $x$, $\sK_{(x)s} = \sK_{(x)}\lotimes_{\sO_{S,s}^h} k(s)$ son image 
réciproque sur la fibre géométrique en $s$, et 
$n(x)=\sum (-1)^i \dim_{k(s)} \h^i(\sK_{(x)s})$. La fonction $x\mapsto n(x)$ 
est une pondération de $f$, et on prend le morphisme trace correspondant 
\cite[XVII.6.2.5]{sga4}. La pondération $n(x)$, et $\tr_{f,\sK}$ sont de 
formation compatible à tout changement de base. 

\paragraph{B}
Cas général. Si $u:Z\to \dA_S^d$ est tel que $u i$ soit quasi-fini, on pose 
$\tr_{f,\sK} = \tr_{\dA_S^d}\circ \tr_{u i,\sK}$. Ce morphisme trace est 
clairement compatible à tout changement de base $S'/S$. Pour prouver qu'il ne 
dépend pas de $u$, on peut donc supposer que $S$ est le spectre d'un corps 
algébriquement clos $k$. Soient dans ce cas $Y_i$ les composantes 
irréductibles de $Y$, et $\operatorname{lg}_i$ la longueur de $\sK$ au 
point générique de $Y_i$. Si $\alpha_i$ est l'inclusion de 
$(Y_i)_\text{red}$ dans $Y$, on a 
\begin{equation*}\tag{2.3.4}\label{IV:eq:2-3-4}
  \tr_{f,\sK} = \sum \operatorname{lg}_i \tr_{Y_{i,\text{red}/S}} \alpha_i^\ast
\end{equation*}
(cette formule résulte de la formule analogue et facile pour $\tr_{u i}$). 

Ceci définit $\tr_{f,\sK}$ localement sur $Y$; on procède ensuite comme en 
\cite[XVIII.2.9]{sga4}. 





\begin{lemma}\label{IV:2-3-5}
Si $g=g'g'':X\xrightarrow{g''} S'\xrightarrow{g'} S$, avec $g'$ et $g''$ lisse 
et purement de dimension relative $N'$ et $N''$, et que $Y$ est encore, sur 
$S'$, fibre par fibre de codimension $\geqslant c$, alors la classe 
$\operatorname{cl}(Y,\sK)$ est la même, calculée en terme de $g$ ou de 
$g''$.
\end{lemma}

Soient $f'=g'' i$ et $d'=d-N'$. La formule $\tr_g=\tr_{g'}\tr_{g''}$ 
(\cite[XVIII.2.9]{sga4} Var 3) assure la commutativité de 
\[\xymatrix{
  \h_X^{2 c}\left(Z,\dZ/n(c)\right) \ar[d] \\
  \hom_{S'}\left(\R f_!\dZ/n(d')[2 d'],\R g_!''\dZ/n(N'')[2 N'']\right) \ar[r] \ar[d]^-{\R g_!'(N')} 
    & \hom_{S'}\left(\R^{2 d'} f_!' \dZ/n,\dZ/n\right) \ar[d]^-{\tr_{g'}\circ \R^{2 N'} g_!'} \\
  \hom_S\left(\R f_!\dZ/n(d)[2d],\R g_!'\dZ/n(N)[2 N]\right) \ar[r] 
    & \hom_S\left(\R^{2 d} f_!\dZ/n,\dZ/n\right)
}\]
tandis que la formule analogue et facile à vérifier 
$\tr_{f,\sK}=\tr_{g'}\tr_{f',\sK}$ assure que l'image de $\tr_{f',\sK}$ est 
$\tr_{f,\sK}$. L'assertion en résulte. 





\begin{lemma}\label{IV:2-3-6}
Si $Y$ est un diviseur dans $X$, la classe \ref{IV:2-3-2} coïncide avec la 
classe \ref{IV:2-1-2}. 
\end{lemma}

Par semi-pureté (\ref{IV:2-2-6}(i) et \ref{IV:2-2-8}(i)), le problème est 
locale sur $Y$; utilisant \ref{IV:2-3-5} pour remplacer $S$ par $S'$ 
convenable, ceci nous permet de supposer que $N=1$. Utilisant à nouveau la 
semi-pureté (\ref{IV:2-2-6}(ii) et \ref{IV:2-2-8}(ii)) il suffit de prouver 
\ref{IV:2-3-6} au-dessus des points génériques de $S$. Une localisation 
étale nous ramène alors à supposer $S$ spectre d'un corps 
algébriquement clos $k$. Les classes \ref{IV:2-3-2} et \ref{IV:2-1-2} étant 
chacune additive en $Y$, ceci nous ramène au cas où $Y$ est un point 
fermé sur un courbe lisse sur $k$. Que la propriété caractéristique 
\ref{IV:2-3-2} soit vérifiée est alors la compatibilité fondamentale 
\ref{IV:2-1-5}. 





\begin{lemma}\label{IV:2-3-7}
La formation de $\cl(Y,\sK)$ est compatible à tout changement de base 
$S'/S$.
\end{lemma}

Résulte de la même assertion pour les morphismes trace.





\begin{theorem}\label{IV:2-3-8}
(i) $\cl(Y,\sK)$ ne dépend que de $Y\subset X$ et des longueurs de $\sK$ aux 
points génériques de $Y$. Cette classe est additive en $\sK$. Pour 
$\sK=\sO_Y$ et $Y$ d'intersection complète locale, elle induit la classe 
locale \ref{IV:2-2-2}.

(ii) Soit un diagramme commutatif 
\[\xymatrix{
  X' \ar[r]^-u \ar[d] 
    & X \ar[d] \\
  S' \ar[r] 
    & S
}\]
avec $X'$ lisse sur $S'$. Si $u^{-1}(Y)$ est encore fibre par fibre de 
codimension $\geqslant c$, alors 
\[
  \cl\left(u^{-1}(Y),\eL u^\ast\sK\right) = u^\ast \cl(Y,\sK) \text{.}
\]

(iii) Soit $Y'\subset X$ fibre par fibre de codimension $\geqslant c'$, 
$\sK'$ un poids sur $Y'$ et supposons que $Y\cap Y'$ soit fibre par fibre de 
codimension $\geqslant c+c'$. Alors 
\[
\cl\left(Y\cap Y',\sK\lotimes \sK'\right) = \cl(Y,\sK) \smallsmile \cl(Y',\sK') \text{.}
\]
\end{theorem}


% The following paragraphs were originally numbered

% (2.3.8.1)
\paragraph{(A)}
Etant donné $Y\subset X/S$ comme dit, il résulte de la semi-pureté 
(\ref{IV:2-2-6}, \ref{IV:2-2-8}) que pour vérifier que deux classes dans 
$\h_Y^{2 c}\left(X,\dZ/n(c)\right)$ coïncident, il suffit de le vérifier 
localement aux points génériques de $Y$, voire même après tout 
changement de bas $s\to S$ ($s$ point géométrique générique de $S$), 
aux points génériques de $Y$. 

% (2.3.8.2)
\paragraph{(B) Preuve de (ii) pour $S=S'$ et $u$ un plongement fermé}
Par localisation (A) on peut supposer que $X'$ est l'image 
réciproque de la section $0$ par un morphisme lisse $v:X\to \dA_S^{N'}$ 
\[\xymatrix{
  X' \ar[r]^-u \ar[d] 
    & X \ar[d]^-v \\
  S \ar[r]^-0 
    & \dA_S^{N'} \ar[r]^-\pi 
    & S
}\]
On applique \ref{IV:2-3-5} à $g=\pi v$ et \ref{IV:2-3-7} au changement de 
base $S\to \dA_S^{N'}$. 

% (2.3.8.3)
\paragraph{(C) Dans une situation produit:}
$X=X'\times_S X''$, $Y=Y'\times_S Y''$, $\sK=\sK\lotimes_S \sK''$, on a 
$\cl(Y,\sK) = \cl(Y,\sK')\smallsmile \cl(Y'',\sK'')$ (cup-produit extérieur, 
i.e. $\operatorname{pr}_1^\ast \cl(Y',\sK')\smallsmile \operatorname{pr}_2^\ast \cl(Y'',\sK'')$). 

Des propriétés fonctorielles de $\tr_g$ \cite[XVIII.2.12]{sga4} résulte 
la commutativité de 
\small
\[\xymatrix@=0.5cm{
  \h_{X'}^{2c'}\otimes \h_{X''}^{2 c''}\ar[r] \ar[d] 
    & \hom_S\left(\R f_!' \dZ/n(d')[2 d'],\R g_!'\dZ/n(N')[2 N']\right)\otimes \cdots \ar[r] \ar[d] 
    & \hom_S\left(\R^{2 d'} f_!' \dZ/n,\dZ/n\right) \otimes \cdots \ar[d] \\
  \h_X^{2 c} \ar[r] 
    & \hom_S\left(\R f_!\dZ/n(d)[2 d],\R g_!\dZ/n(N)[2 N]\right) \ar[r] 
    & \hom_S\left(\R ^{2 d}f_!\dZ/n,\dZ/n\right)
}\]
\normalsize
et on vérifie que $\tr_f$ est l'image de $\tr_{f'}\otimes \tr_{f''}$ d'où 
l'assertion. 

\paragraph{Preuve de (ii)}
(\ref{IV:2-3-7}) et un changement de base préliminaire nous ramènent à 
supposer que $S=S'$. Factorisons $u$ par son graphe:
\[\xymatrix{
  X' \ar[r]^-{(\text{id},u)} 
    & X'\times X \ar[r]^-{\operatorname{pr}_2} 
    & X \text{.}
}\]
Ceci nous amène à traiter séparément $\operatorname{pr}_2$, justiciable 
de (C) pour $X'=X'$, $\sK'=\sO_{x'}$ (avec $\cl(Y')=1\in \h^0(X,\dZ/n)$), et 
$(\text{id},u)$, justiciable de (B). 

\paragraph{Preuve de (iii)}
Soit $\delta:X\to X\times_S X$ l'inclusion diagonale. On utilise la formule 
$\delta^\ast(Y_1\times_S Y_2) = Y_1\cap Y_2$ pour se ramener à 
(B) et (C). 

\paragraph{Preuve de (i)}
Le changement de base $S_\text{red}\to S$ remplace $X$ par $X_\text{red}$; ceci 
nous ramène à supposer $X$ et $S$ réduits. Une localisation au voisinage 
des points génériques de $Y$ nour ramène ensuite au cas où 
$Y_\text{red}$ est irréductibles et l'intersection complète de $c$ 
diviseurs $D_i$. La même localisation, et la définition de $\tr$ dans le 
cas quasi-fini. montre ensuite que 
$\cl(Y,\sK)=\operatorname{lg}\cl(Y_\text{red},\sO_{Y_\text{red}})$, où 
$\operatorname{lg}$ est la longueur de $\sK$ au point générique de $Y$. 
D'après (iv), on a donc 
\[
  \cl(Y,\sK) = \operatorname{lg} \prod_i \cl(D_i,\sO_{D_i})
\]
et on conclut par \ref{IV:2-3-6}. 





\subsubsection{Remarque}\label{IV:2-3-9}

Pour $S$ le spectre d'un corps, et la classe d'un cycle étant définie comme 
en \ref{IV:2-2-10}, l'assertion (iii) dit que, si deux cycles $Y'$ et $Y''$ se 
coupent avec la bonne dimension, alors 
\[
  \cl(Y'\cap Y'') = \cl(Y')\smallsmile \cl(Y'')\text{,}
\]
pourvu que les multiplicités des composantes de $Y'\cap Y''$ soient 
calculées comme sommes alternées de $\tor$. 





\subsubsection{Remarque}\label{IV:2-3-10}

Soit $X/\spec(k)$, $k$ algébriquement clos. Si deux cycles de codimension $c$ 
dans $X$ sont algébriquement équivalents, il résulte de l'existence de la 
théorie relative ci-dessus que les images dans $\h^{2 c}(X,\dZ/n(c))$ de 
leurs classes sont les fibres, en deux points de $S$ connexe convenable, d'une 
section sur $S$ du faisceau constant $\h^{2 c}(X,\dZ/n(c))$ (le faisceau 
$\R^{2 c} f_\ast \dZ/n(c)$ pour $f=\operatorname{pr}_2:X\times S\to S$): deux 
cycles algébriquement équivalentes ont même classe dans 
$\h^{2 c}(X,\dZ/n(c))$. 




















\section{Application: la formule des traces de Lefschetz dans le cas propre et lisse}\label{IV:3}





\subsection{}\label{IV:3-1}

Soient $X$ et $Y$ des variétés algébriques propres et lisses sur un corps 
algébriquement clos $k$, purement de dimensions $N$ et $M$. Via 
l'isomorphisme de K\"unneth 
$\h^\bullet(X\times Y)=\h^\bullet(X)\otimes \h^\bullet(Y)$, le morphisme trace 
$\h^\bullet(X\times Y)(M)\to \h^\bullet(X)$ n'est autre que 
$\h^\bullet(X)\otimes ($le morphisme trace $\h^\bullet(Y)(M)\to \dQ_\ell)$. Le 
morphisme étant homogène de degré pair, il n'y a pas de problème de 
signe. On le note $\int_Y$. 





\subsection{}\label{IV:3-2}

Une classe $\eta\in \h^{2 q}(X\times Y)(q)$ définit un morphisme de degré 
$2(q-N)$ $\eta_{XY}:\h^\bullet(Y)(N-q)\to \h^\bullet(X)$, par la formule 
\[
  \beta \mapsto \int_Y \eta\cdot \operatorname{pr}_2^\ast \beta \text{.}
\]





\begin{proposition_}\label{IV:3-3}
Soient $p$ et $q$ deux entiers, avec $p+q=N+M$, et 
$\varepsilon\in \h^{2 p}(X\times Y)(p)$, $\eta\in \h^{2 q}(X\times Y)(p)$. La 
trace de $\eta_{X Y}\varepsilon_{Y X}:\h^\bullet(X)\to \h^\bullet(X)$ étant 
entendue au sens \ref{IV:1-3}, on a alors 
\[
  \tr_{X\times Y}(\eta\cdot \varepsilon) = \tr\left(\eta_{X Y}\varepsilon_{Y X},\h^\bullet(X)\right) \text{.}
\]
\end{proposition_}

Pour exorciser les signes, il y a intérêt à ne retenir de la graduation 
de $\h^\bullet$ que la $\dZ/2$-graduation sous-jacente. Pour éviter de 
trainer avec soi les twist à la Tate, on fixe de plus un isomorphisme 
$\dQ_\ell(1)=\dQ_\ell$. Soit $\alpha_Y$ l'homomorphisme 
$\h^\bullet(Y)\to \h^\bullet(Y)^\vee$ (un isomorphisme) rendant commutatif le 
diagramme 
\[\xymatrix{
  \h^\bullet(Y)\otimes \h^\bullet(Y) \ar[r] \ar[d]^-{\alpha_Y\otimes 1} 
    & \h^\bullet(Y) \ar[r]^-{\tr} 
    & \dQ_\ell \ar@{=}[d] \\
  \h^\bullet(Y)^\vee \otimes \h^\bullet(Y) \ar[rr]^-{\text{(\ref{IV:eq:1-3-1})}} 
    & & \dQ_\ell \text{.}
}\]
L'application $\eta_{X Y}$ est l'image de $\eta$ par 
\[\xymatrix{
  \h^\bullet(X\times Y) = \h^\bullet(X)\otimes \h^\bullet(Y) \ar[r]^-{\alpha_Y} 
    & \h^\bullet(X) \otimes \h^\bullet(Y)^\vee \ar@{=}[r]^-{\text{(\ref{IV:eq:1-3-2})}} 
    & \hom\left(\h^\bullet(X),\h^\bullet(Y)\right) \text{,}
}\]
et de même pour $\varepsilon$. La définition \ref{IV:eq:1-3-4} de la trace 
ramène alors \ref{IV:3-3} à la formule 
$\tr_{X\times Y}=\tr_X\otimes \tr_Y$. 





\subsection{Remarque}\label{IV:3-4}

La preuve de \ref{IV:3-4} n'utilise pas que $\alpha_X$ et $\alpha_Y$ soient des 
isomorphismes. Elle vaut sans supposer $X$ et $Y$ lisses. Toutefois, en dehors 
du cas lisse, il est difficile de trouver des classes de \emph{cohomologie} 
auxquelles on veuille appliquer \ref{IV:3-3}. 





\subsection{Remarque}\label{IV:3-5}

Si $\varepsilon,\eta$ sont les classes de cohomologie de cycles algébriques 
$A$ et $B$ sur $X\times Y$, de codimension $p$ et $q$, et que $A\cap B$ est de 
dimension $0$, alors $\tr_{X\times Y}(\eta\varepsilon)$ est la multiplicité 
d'intersection $A\cdot B$, calculée par une somme alternée de $\tor$ 
(\ref{IV:2-3-9}). 





\begin{proposition_}\label{IV:3-6}
Pour $q=M$ et $\eta=\cl(B)$ la classe d'un sous-schéma $B\subset X\times Y$, 
fini sur $X$, l'application $\eta_{X Y}$ est la composée 
\[\xymatrix{
  \eta_{X Y} : \h^\bullet(Y) \ar[r]^-{\operatorname{pr}_2^\ast} 
    & \h^\bullet(B) \ar[r]^-{\tr_{B/X}} 
    & \h^\bullet(X) \text{.}
}\]
\end{proposition_}

D'après \ref{IV:3-2} et \eqref{IV:eq:2-3-3-1} pour $\sK=\sO_B$, on a en 
effet 
\[
  \eta_{X Y}(\beta) = \tr_{X\times Y/X}\left(\cl(B)\cdot \operatorname{pr}_2^\ast\beta\right) = \tr_{B/X}\left(\operatorname{pr}_2^\ast \beta\right) \text{.}
\]

Le cas le plus important est celui où $B$ est le graphe d'une application 
$f:X\to Y$. Dans ce cas, \ref{IV:3-6} montre que $\eta_{X Y}=f^\ast$. 





\begin{corollary_}\label{IV:3-7}
Soit $f$ un endomorphisme de $X$ propre et lisse sur un corps algébriquement 
clos $k$. On suppose que les points fixes de $f$ sont isolés. Alors, la 
trace $\sum (-1)^i \tr(f^\ast,\h^i(X))$ est le nombre de points fixes de $f$, 
chacun compté avec sa multiplicité.
\end{corollary_}

Dans \ref{IV:3-5}, on prend $A=$graphe de l'identité, $B=$graphe de $f$ et on 
applique \ref{IV:3-3}, \ref{IV:3-6}. 





\begin{corollary_}\label{IV:3-8}
Soient $X$ un courbe sur $k$ algébriquement clos, déduit par extension des 
scalaires de $X_0$ sur $\dF_q$, et $f$ le morphisme de Frobenius. Alors, 
$\sum (-1)^i \tr\left(f^\ast,\h_c^\bullet(X)\right)$ est le nombre de points 
fixes de $f$.
\end{corollary_}

On peut remplacer $X$ par $X_\text{red}$, donc supposer $X$ réduite. Soit $U$ 
l'ouvert de $X$ où $X$ est lisse de dimension $1$, et $\bar U$ le 
complétion projective et lisse de $U$. Les suites exactes longues de 
cohomologie pour $(X,U)$ et $(\bar U,U)$ donnent 
\[
  \tr\left(f^\ast,\h_c^\bullet(X)\right) = \tr\left(f^\ast,\h_c^\bullet(U)\right) + \tr\left(f^\ast,\h_c^\bullet(X\setminus U)\right) 
\]
et
\[
  \tr\left(f^\ast,\h_c^\bullet(\bar U)\right) = \tr\left(f^\ast,\h_c^\bullet(U)\right) + \tr\left(f^\ast,\h_c^\bullet(\bar U\setminus U)\right) \text{.}
\]
Le mêmes formules valent pour les nombres de points fixes. La formule 
\ref{IV:3-8} étant claire pour un schéma de dimension $0$, ceci ramène 
à prouver \ref{IV:3-8} pour $\bar U$. Ce cas est justiciable de \ref{IV:3-7}. 
Le fait que $d f=0$ garantit que les points fixes sont tous de 
multiplicité un. 





\subsection{Remarque}\label{IV:3-9}

Pour $X$ de dimension $\leqslant 1$ sur $k$ algébriquement clos, et 
$f:X\to X$ tel que $d f=0$, il n'est pas difficile de vérifier que 
\ref{IV:3-8} est encore valable. 



% !TEX root = sga4.5.tex

\chapter{Dualité}\label{V}




















On trouvera dans cet exposé quelques théorèmes et compatibilités, tous 
relatifs à la dualité de Poincaré. Au paragraphe \ref{V:1}, le théorème de 
bidualité locale en dimension $1$ \cite[I.5.1]{sga5} et quelques calculs de 
deux. Au paragraphe \ref{V:2}, une démonstration très économique de la dualité 
de Poincaré sur les courbes, que m'a apprise M.\ Artin. Au paragraphe 
\ref{V:3}, une compatibilité qui fait le lien entre deux définitions de 
l'accouplement qui donne lieu à la dualité de Poincaré pour les courbes: par 
cup-produit, ou par autodualité de la jacobienne. Au paragraphe \ref{V:4}, 
enfin, la preuve de la compatibilité du titre. 

Dans tout l'exposé, les schémas seront noethériens et séparés, et $n$ 
est un entier inversible sur tous les schémas considérés. 










\section{Bidualité locale, en dimension \texorpdfstring{$1$}{1}}\label{V:1}





\subsection{}\label{V:1-1}

Soit $S$ un schéma régulier purement de dimension $1$. Nous nous proposons 
de montrer que le complexe réduit à $\dZ/n$ en degré $0$ est 
dualisant, i.e. que pour $\sK\in \ob\D_c^b(S,\dZ/n)$ ($(-)_c$ pour 
constructible), si on pose $D\sK=\rHom(\sK,\dZ/n)$, alors $D\sK$ est encore 
constructible à cohomologie bornée, et que le morphisme canonique $\alpha$ 
de $\sK$ dans $D D\sK$ est un isomorphisme. 

Pour $\sK$ dans $\D^-$, et $\sL$ quelconque, on a 
\[
  \hom(\sK\lotimes \sL,\dZ/n) = \hom(\sL,\Hom(\sK,\dZ/n)) \text{.}
\]
Le morphisme de $\sK$ dans $DD\sK$ est défini en supposant $D\sK$ dans 
$\D^-$; si $\beta:D\sK\lotimes\sK\to \dZ/n$ est l'accouplement canonique, il 
est défini par l'accouplement 
$\sK\lotimes D\sK=D\sK\lotimes \sK \xrightarrow\beta \dZ/n$. 





\subsection{}\label{V:1-2}

Soit $f:X\to S$ un morphisme séparé de type fini. Posons 
$\sK_X=\R f^!\dZ/n$. Pour $\sK\in \ob\D^-(X,\dZ/n)$, on pose 
$D\sK = \rHom(\sK,\sK_X)$. L'adjonction entre $\R f_!$ et $\R f^!$ 
assure que 
\[\xymatrix{
  \R f_\ast \sK\lotimes \R f_\ast D\sK \ar[r] 
    & \R f_! (\sK\lotimes D\sK) \ar[r] 
    & \R f_! \R f^! \dZ/n \ar[r] 
    & \dZ/n \text{.}
}\]
La symétrie de cette description montre que, pour $f$ propre, le diagramme 
\begin{equation*}\tag{1.2.1}\label{V:eq:1-2-1}
\xymatrix{
  \R f_\ast \sK \ar[r] \ar[d] 
    & D D \R f_\ast \sK \ar[d]^-\wr \\
  \R f_\ast D D \sK \ar[r]^\sim 
    & D\R f_\ast D\sK
}
\end{equation*}
est commutatif. 

Si $f$ est l'inclusion d'un point fermé, on sait que 
$\R f^!\dZ/n=\dZ/n(-1)[-2]$ -- soit, à torsion et décalage près, $\dZ/n$. 
Sur le spectre d'un corps, ce complexe est dualisant (dualité de Pontrjagin 
pour les $\dZ/n$-modules). D'après \eqref{V:eq:1-2-1}, on a donc 
$\sK\iso D D\sK$ pour $\sK$ de la forme $\R f_!\sL$ -- et donc lorsque le 
support de $\underline\h^\bullet(\sK)$ est fini. 





\begin{theorem_}\label{V:1-3}
Soient $j:U\hookrightarrow S$ un ouvert dense de $S$ et $\sF$ un faisceau 
localement constant constructible de $\dZ/n$-modules sur $U$. On a 
$\D j_\ast \sF = j_\ast D \sF$, i.e. $\Hom(j_\ast \sF,\dZ/n) = j_\ast\Hom(\sF,\dZ/n)$ et $\Ext^i(j_\ast\sF,\dZ/n) = 0$ pour $i>0$.
\end{theorem_}

Sur $U$, $\Ext^i(j_\ast\sF,\dZ/n)=0$ pour $i>0$, car $\sF$ est localement 
constant (et $\dZ/n$ est un $\dZ/n$-module injectif), tandis que pour $i=0$ 
c'est le dual $\sF^\vee$ de $\sF$. On vérifie que 
$\hom(j_\ast\sF,\dZ/n)=j_\ast\sF^\vee$, et il reste à vérifier la nullité 
des $\Ext^i$ ($i>0$) en les points de $S\setminus U$. Le problème est local 
en ces points. Ceci nous ramène à supposer que $S$ est un trait strictement 
local et, que $U$ est réduit à son point générique $\eta$. Soit 
$I=\gal(\bar\eta/\eta)$. Le faisceau $\sF$ s'identifie au module galoisien 
$\sF_{\bar\eta}$, et la fibre spéciale de $j_\ast \sF$ à 
$\sF_{\bar\eta}^I$. 

Soit $i$ l'inclusion du point fermé $s$, et appliquons $D$ aux suites 
exactes 
\[\xymatrix{
  0 \ar[r] 
    & j_! \sF \ar[r] 
    & j_\ast \sF \ar[r] 
    & i_\ast \sF_{\bar\eta}^I \ar[r] 
    & 0 \\
  0 \ar[r] 
    & j_! \sF_{\bar\eta}^I \ar[r] \ar[u] 
    & \sF_{\bar\eta}^I \ar[r] \ar[u] 
    & i_\ast \sF_{\bar\eta}^I \ar[r] \ar[u]
    & 0 \text{.}
}\]

On obtient un morphisme de triangles 
\[\xymatrix{
  i_\ast\left(\sF_{\bar\eta}^I\right)^\vee (-1)[-2] \ar[r] \ar@{=}[d] 
    & D j_\ast\sF \ar[r] \ar[d] 
    & \R j_\ast \sF^\vee \ar[d] \\
  i_\ast\left(\sF_{\bar\eta}^I\right)^\vee(-1)[-2] \ar[r] 
    & \left(\sF_{\bar\eta}^I\right)^\vee \ar[r] 
    & \R j_\ast \sF_{\bar\eta}^I \text{.}
}\]
La suite exacte longue déduite de la première ligne fournit la nullité des 
$\Ext^i$ ($i>2$) et, prenant la fibre en $s$, on trouve 
\[\xymatrix@=0.7cm{
  0 \ar[r] 
    & \left(\underline\h^1(D j_\ast\sF\right)_s \ar[r] 
    & \h^1\left(I,\sF_{\bar\eta}^\vee\right) \ar[r]^-\partial \ar[d] 
    & \left(\sF_{\bar\eta}^I\right)^\vee(-1) \ar[r] \ar@{=}[d] 
    & \left(\underline\h^2 D j_\ast\sF\right)_s \ar[r] 
    & 0 \\
  & 0 \ar[r] 
    & \h^1\left(I,(\sF_{\bar\eta}^I)^\vee\right) \ar[r]^-\partial 
    & \left(\sF_{\bar\eta}^I\right)^\vee(-1) \ar[r] 
    & 0 \text{.}
}\]
Puisque 
$\left(\sF_{\bar\eta}^I\right)^\vee = \left(\sF_{\bar\eta}^\vee\right)_I$, 
posant $M=\sF_{\bar\eta}^\vee$, il faut finalement vérifier que 
\[\xymatrix{
  \h^1(I,M) \ar[r]^-\sim 
    & \h^1(I,M_I) \text{.}
}\]
Si $p$ est l'exposant caractéristique résiduel, $I$ est extension d'un 
groupe isomorphe à $\widehat\dZ_{p'} = \varprojlim_{(m,p)=1} \dZ/m$ par un 
$p$-groupe $P$. Puisque $P$ est premier à l'ordre de $M$, on a 
$\h^1(P,M)=0$ ($i>0$) et $(M_I)^P=(M^P)_I$. Ceci permet de remplacer $M$ par 
$M^P$ et $I$ par $I/P$. On a enfin un isomorphisme fonctoriel 
$\h^1(\widehat\dZ_{p'},M)\sim ($coinvariants de $\widehat\dZ_{p'}$, dans $M)$, 
d'où le théorème. 





\begin{theorem_}\label{V:1-4}
Pour $\sK\in \ob\D_c^b(S,\dZ/n)$, on a $\sK\iso D D \sK$.
\end{theorem_}

Par dévissage, on se ramène à supposer que $\sK$ est réduit à un 
faisceau constructible $\sF$ en degré $0$, et que $\sF$ est soit à support 
fini (\ref{V:1-2}), soit de la forme $j_\ast\sF_1$ comme en \ref{V:1-3}. Dans 
ce second cas, \ref{V:1-3} nous ramène à la bidualité locale pour $\sF$ 
localement constant sur $U$. 










\section{La dualité de Poincaré pour les courbes, d'après M.\ Artin}\label{V:2}





\subsection{}\label{V:2-1}

Soit $X$ une courbe projective et lisse sur $k$ algébriquement clos. On pose 
$\sK_X=\dZ/n(1)[2]$ et, pour $\sK\in\ob \D_c^b(X,\dZ/n)$, 
$D\sK=\rHom(\sK,\sK_X)$. Pour $M$ un $\dZ/n$-module, on pose aussi 
$D M = \hom(M,\dZ/n)$; de même pour les complexes de modules. Le morphisme 
trace $\h^0(X,\sK_X) \to \dZ/n$, ou $\R\Gamma(X,\sK_X)\to \dZ/n$, définit un 
accouplement $\R\Gamma(X,\sK)\lotimes\R\Gamma(X,D\sK) \to \dZ/n$. La dualité 
de Poincaré entre cohomologie et cohomologie à supports propres d'un ouvert 
$j:U\hookrightarrow X$ de $X$ dit que, pour $\sK=j_!\dZ/n$, cet accouplement 
identifie chaque facteur au dual de l'autre. 

J'expose ci-dessous une démonstration, qui m'a été communiquée par 
M.\ Artin, de ce que pour $\sK\in\ob\D_c^b(X,\dZ/n)$, cet accouplement est 
toujours parfait, i.e. définit un isomorphisme
\begin{equation*}\tag{2.1.1}\label{V:eq:2-1-1}
\xymatrix{
  \R\Gamma(X,\sK) \ar[r]^-\sim 
    & D \R\Gamma(X,D\sK) \text{.}
}
\end{equation*}

Pour tout faisceau constructible $\sF$, posons 
\[
  '\h^i(X,\sF) = \text{$\dZ/n$-dual de $\h^{-i}(X,D\sF)$.}
\]
Que \eqref{V:eq:2-1-1} soit un isomorphisme équivaut au 





\begin{theorem_}\label{V:2-2}
Pour $\sF$ un faisceau constructible de $\dZ/n$-modules, on a 
\begin{equation*}\tag{2.2.1}\label{V:eq:2-2-1}
\xymatrix{
  \h^i(X,\sF) \ar[r]^-\sim 
    & '\h^i(X,\sF) \text{.}
}
\end{equation*}
\end{theorem_}





\begin{lemma_}\label{V:2-3}
Soit $f:X\to Y$ un morphisme génériquement étale entre courbes lisses sur 
$k$. On a $\sK_X=f^\ast\sK_Y=\R f^!\sK_Y$, avec $\tr_f$ pour flèche 
d'adjonction $\R f_!\R f^!\sK_Y\to \sK_Y$.
\end{lemma_}

% NOTE: \rhom or \rHom ? originally \rhom
On se ramène à vérifier que pour $f$ fini, on a 
$f_\ast \R f^!\sK_Y = \rHom(f_\ast\dZ/n,\sK_Y)$ est $f_\ast\dZ/n$, avec $\tr_f$ 
pour flèche d'adjonction. La nullité des $\Ext^i$ ($i>0$) résulte de 
\ref{V:1-3}, et l'assertion en résulte. 

On peut dire, plus explicitement, que $\R f^!$ est le foncteur dérivé du 
foncteur $f^!:f^!\sF(U)=\hom(f_!\dZ_U,\sF)$ et \ref{V:1-3} implique la 
nullité des $\R^i f^!\dZ/n$ pour $i>0$.





\begin{corollary_}\label{V:2-4}
Soit $f:X'\to X$ un morphisme génériquement étale de courbes 
projectives et lisses sur $k$. On a $'\h^i(X,f_\ast \sF)='\h^i(X',\sF)$; cet 
isomorphisme est fonctoriel et le diagramme 
\[\xymatrix{
  \h^i(X,f_\ast\sF) \ar[r] \ar@{=}[d] 
    & '\h^i(X,f_\ast\sF) \ar@{=}[d] \\
  \h^i(X',\sF) \ar[r] 
    & '\h^i(X',\sF) 
}\]
est commutatif.
\end{corollary_}

L'isomorphisme est donné par \ref{V:1-2}: $D f_\ast\sF=f_\ast D\sF$. 





\subsection{}\label{V:2-5}

Prouvons \ref{V:2-2}. Si $\sF$ est réduit à $\dZ/n$ en un point, prolongé 
par $0$, on a $\h^\bullet(D\sF) = \ext^\bullet(\sF,\sK_X)=\dZ/n$ en degré 
$0$, et dans l'accouplement $\h^\bullet(\sF)\otimes\h^\bullet(D\sF) \to \dZ/n$, 
on a $1\otimes 1\mapsto 1$ (\hyperref[IV]{Cycle}, \ref{IV:2-1-5}): la dualité 
est parfaite. Le cas où $\sF$ est à support fini se traite de même. 

Pour $\sF$ constructible quelconque, $\sA$ le sous-faisceau de ses sections à 
support fini, et $\sG=\sF/\sA$, la suite exacte 
\begin{equation*}\label{V:eq:2-5-1}\tag{2.5.1}
\xymatrix{
  0 \ar[r] 
    & \sA \ar[r] 
    & \sF \ar[r] 
    & \sG \ar[r] 
    & 0 
}
\end{equation*}
fournit un diagramme 
\begin{equation*}\tag{2.5.2}\label{V:eq:2-5-2}
\xymatrix@=.5cm{
  & & 0 \ar[r] 
    & \h^0(X,\sA) \ar[r] \ar[d]^-\wr 
    & \h^0(X,\sF) \ar[r] \ar[d] 
    & \h^0(X,\sG) \ar[r] \ar[d] 
    & 0 \\
  0 \ar[r] 
  & '\h^{-1}(X,\sF) \ar[r] 
  & '\h^{-1}(X,\sF) \ar[r] 
  & '\h^0(X,\sA) \ar[r] 
  & '\h^0(X,\sF) \ar[r] 
  & '\h^0(X,\sF) \ar[r] 
  & 0 
}
\end{equation*}
et des isomorphismes compatibles $\h^i(X,\sF)\iso \h^i(X,\sG)$, 
$'\h^i(X,\sF) \iso '\h^i(X,\sG)$ ($i\ne 0,-1$). Si $j:U\to X$ est un ouvert sur 
lequel $\sF$ est localement constant, on a une suite exacte 
\begin{equation*}\tag{2.5.3}\label{V:eq:2-5-3}
\xymatrix{
  0 \ar[r] 
    & \sG \ar[r] 
    & j_\ast j^\ast \sF \ar[r] 
    & \sB \ar[r] 
    & 0
}
\end{equation*}
avec $\sB$ à support fini. Puisque 
$'\h^i(X,j_\ast j^\ast\sF) = D \h^{2-i} (X,j_\ast(j^\ast\sF)^\vee(1))$ 
d'après \ref{V:1-3}, cet $'\h^i=0$ pour $i<0$ et \eqref{V:eq:2-5-2} et la 
suite exacte longue définie par \eqref{V:eq:2-5-3} montrent que $'\h^i$ est 
toujours nul pour $i<0$. 

Si $\sF=\dZ/n$, le théorème est vrai pour $i=0$ et $2$. Ceci exprime que, 
pour $X$ connexe, le morphisme trace: $\h^2(X,\dmu_n) \iso \dZ/n$ est un 
isomorphisme. D'après \ref{V:2-4}, le théorème reste vrai pour $i=0$ et 
$\sF=f_\ast\dZ/n$. 

Pour $\sF$ à nouveau quelconque et $\sG$ comme plus haut, il existe 
$f:X'\to X$ comme en \ref{V:2-4} tel que $\sG$ se plonge dans $f_\ast\dZ/n$:
\[\xymatrix{
  0 \ar[r] 
    & \sG \ar[r] 
    & f_\ast\dZ/n \ar[r] 
    & \sQ \ar[r] 
    & 0 \text{.}
}\]
Changeant $X'$ en un $X''/X'$, on peut supposer que 
$\h^i(X,\sF) \to \h^i(X,f_\ast\dZ/n) = \h^i(X',\dZ/n)$ est nul pour $i>0$. 
Considérons alors 
\small
\[\xymatrix@=.3cm{
  0 \ar[r] 
    & \h^0(X,\sG) \ar[r] \ar[d] 
    & \h^0(S',\dZ/n) \ar[r] \ar[d]^-\wr 
    & \h^0(X,\sQ) \ar[r] \ar[d] 
    & \h^1(X,\sG) \ar[r]^-0 \ar[d] 
    & \h^1(X',\dZ/n) \ar[r] \ar[d] 
    & \h^1(X,\sQ) \ar[r] \ar[d] 
    & \cdots \\
  0 \ar[r] 
    & '\h^0(X,\sG) \ar[r] 
    & '\h^0(X',\dZ/n) \ar[r] 
    & '\h^0(X,\sQ) \ar[r] 
    & '\h^1(X,\sG) \ar[r] 
    & '\h^1(X',\dZ/n) \ar[r] 
    & '\h^1(X,\sQ) \ar[r] 
    & \cdots
}\]
\normalsize

L'isomorphisme en $\h^0(X')$ fournit $\h^0(X,\sG)\hookrightarrow '\h^0(X,\sG)$, 
et la même injectivité pour $\sF$ \eqref{V:eq:2-5-2}. Appliqué à $\sQ$, 
cela donne $\h^1(X,\sG)\hookrightarrow '\h^1(X,\sG)$. On applique cela à 
$\h^1(X',\dZ/n)$. Ce groupe ayant même ordre que 
$'\h^1(X',\dZ/n)=D(\h^1(X',\dZ/n)(1))$, on trouve un isomorphisme, et et 
$\h^1(X,\sG) \iso '\h^1(X,\sG)$. De même pour $\sF$. Pour $i=2$, on sait 
déjà que $\h^2(X',\dZ/n) \iso '\h^2(X',\dZ/n)$, et on procède de même, 
puis s'arrête, faute de combattants. 










\section{Dualité de Poincaré pour les courbes}\label{V:3}

Dans ce paragraphe, nous prouvons la compatibilité (\hyperref[I]{Arcata}  
\ref{I:eq:6-2-3-3}). 





\subsection{}\label{V:3-1}

Les notations sont celles de (\hyperref[I]{Arcata} \ref{I:6-2-3}): $\bar X$ est 
une courbe projective, lisse et connexe sur un corps algébriquement clos $k$, 
$D$ est un diviseur réduit sur $\bar X$, $0$ un point de 
$X=\bar X\setminus D$ 
\[\xymatrix{
  X \ar@{^{(}->}[r]^-j 
    & \bar X 
    & \ar[l]_-i D \text{,}
}\]
et $\pic_D(\bar X)=\h^1(\bar X,_D\dG_m)$ où $_D\dG_m$ est le faisceau des 
sections de $\dG_m$ congrues à $1\mod D$. On rappelle que 
$\h^1(X,\dmu_n) = \pic_D^0(\bar X)_n$, et que $x\to \sO(x)$ définit une 
application canonique $f:X\to \pic_D(\bar X)$. On pose 
$f_0(x)=f(x)-f(0):X\to \pic_D^0(\bar X)$.

Notons encore $j$ l'inclusion de $X\times X$ dans $X\times \bar X$. La 
diagonale $\Delta$ de $X\times X$ est fermée dans $X\times \bar X$; elle 
définit une classe dans 
$\h_\Delta^2(X\times X,\dmu_n)=\h_\Delta^2(X\times \bar X,j_!\dmu_n)$. On note 
$c$ son image dans $\h^2(X\times \bar X,j_!\dmu_n)$. 

La formule de K\"unneth assure que 
\[
  \h^\bullet(X\times \bar X,j_!\dmu_n) = \h^\bullet(X,\dZ/n)\otimes \h^\bullet(\bar X,j_!\dmu_n) = \h^\bullet(X,\dZ/n)\otimes \h_c^\bullet(X,\dmu_n) \text{.}
\]
Nous nous proposons de calculer la $(1,1)$-composante de $c$, 
\[
  c^{1,1} \in \h^1(X,\dZ/n)\otimes \h_c^1(X,\dmu_n) = \h^1(X,\h_c^1(X,\dmu_n)) = \h^1\left(X,\pic_D^0(\bar X)_n\right) \text{.}
\]

La suite exacte 
\[\xymatrix{
  0 \ar[r] 
    & \pic_D^0(\bar X)_n \ar[r] 
    & \pic_D^0(\bar X) \ar[r]^n 
    & \pic_D^0(\bar X) \ar[r] 
    & 0 
}\]
fait de $\pic_D^0(\bar X)$ un $\pic_D^0(\bar X)_n$-torseur sur 
$\pic_D^0(\bar X)$. Notons, $u$ la classe, dans 
$\h^1(X,\h_c^1(X,\dmu_n))=\h^1(X,\pic_D^0(\bar X)_n)$, de son image 
réciproque par $f_0$. 





\begin{proposition_}\label{V:3-2}
Avec les notations précédentes, $c^{1,1}=-u$.
\end{proposition_}

Soit $e$ la différence des images dans $\h^2(X\times \bar X,j_!\dmu_n)$ des 
classes de cohomologie de $\Delta$ et $X\times\{0\}$ étant de type de 
K\"unneth $(0,2)$, on a $c^{1,1}=e^{1,1}$. Soit la suite spectrale de Leray 
pour $\operatorname{pr}_1:X\times \bar X\to X$ et le faisceau $j_!\dmu_n$: 
\[
  E_2^{pq} = \h^p(X,\R^q {\operatorname{pr}_1}_\ast j_!\dmu_n) = \h^p(X,\h_c^q(X,\dmu_n)) \Rightarrow \h^{p+q}(X\times \bar X,j_!\dmu_n) \text{.}
\]
Elle dégénère: c'est le produit tensoriel de $\h^\bullet(X)$ par la 
suite spectrale de Leray (triviale) pour $\bar X\to \spec(k)$ et le faisceau 
$j_!\dmu_n$. Le diviseur $\Delta-X\times\{0\}$ est fibre par fibre de degré 
$0$, de sorte que l'image de $e$ dans $E^{02}$ est nulle. On peut donc parler 
de son image $\bar e$ dans $E^{1,1}$, et il nous faut montrer que 
$\bar e = -u$. 

Soit $\sG$ sur $X\times\bar X$ le sous-faisceau de $\dG_m$ formé des sections 
locales dont la restriction au sous-schéma $X\times D$ est $1$. On a encore 
une suite exacte 
\begin{equation*}\tag{3.2.1}\label{V:eq:3-2-1}
\xymatrix{
  0 \ar[r] 
    & j_!\dmu_n \ar[r] 
    & \sG \ar[r] 
    & \sG \ar[r] 
    & 0 
}
\end{equation*}
et $e$ est l'image par $\partial$ de la classe $e_1\in \h^1(X\times\bar X,\sG)$ 
du faisceau inversible $\sO(\Delta-X\times\{0\})$ trivialisé par $1$ sur 
$X\times D$. 

L'image directe par $\R\operatorname{pr}_{1\ast}$ du triangle distingué par 
\eqref{V:eq:2-2-1} est un triangle distingué 
\begin{equation*}\tag{3.2.2}\label{V:eq:3-2-2}
\xymatrix{
  & \ar[r]^-\partial 
    & \R \operatorname{pr}_{1\ast} j_!\dmu_n \ar[r] 
    & \R \operatorname{pr}_{2\ast}\sG \ar[r] 
    & \R \operatorname{pr}_{2\ast}\sG \ar[r]^-\partial 
    & 
}
\end{equation*}
et le diagramme 
\[\xymatrix{
  \h^1(X\times \bar X,\sG) \ar[r]^-\partial \ar@{=}[d]^-{(1)} 
    & \h^2(X\times \bar X,j_!\dmu_n) \ar@{=}[d] \\
  \h^1(X,\R\operatorname{pr}_{1\ast} \sG) \ar[r]^-\partial 
    & \h^2(X,\R\operatorname{pr}_{1\ast} j_!\dmu_n) 
}\]
est commutatif: notre problème devient celui d'identifier l'image dans 
$E^{11}=\h^1(X,\h_c^1(X,\dmu_n))$ (l'image dans $E^{02}$ étant nulle) de 
l'image par 
$\partial_{\text{\eqref{V:eq:3-2-2}}}:\h^1(X,\R\operatorname{pr}_{1\ast}\sG) \to \h^2(X,\R\operatorname{pr}_{1\ast} j_!\dmu_n)$ 
de la classe, encore notée $e_1$, qui correspond à $e_1$ par 
(1). 

Le faisceau $\R^1\operatorname{pr}_{1\ast}\sG$ est défini par le schéma en 
groupe sur $X$ image réciproque du groupe algébrique $\pic_D(\bar X)$ sur 
$\spec(k)$, et l'image de $e_1$ dans 
$\h^0(X,\R^1\operatorname{pr}_\ast\sG) = \hom(X,\pic_D(\bar X))$ est $f_0$. 

Représentons le triangle \eqref{V:eq:3-2-2} comme défini par une suite 
exacte courte de complexes de faisceaux.
\[\xymatrix{
  0 \ar[r] 
    & \sA \ar[r] 
    & \sB \ar[r] 
    & \sC \ar[r] 
    & 0 \text{.}
}\]
Soit $\sB'$ le sous-faisceau de 
\[
  \tau_{\leqslant 1} \sB : \cdots \sB^0 \to \ker(d) \to 0 \cdots
\]
obtenu en remplaçant ${\sB'}^1=\ker(d)$ par l'image réciproque, par 
$\ker(d)\to \underline\h^1(\sB) = \pic_D(\bar X)$ sur $X$, de 
$\pic_D^0(\bar X)$ sur $X$. Soient $\sA'=\sA\cap \sB'$ et $\sC'$ l'image de 
$\sB'$. On a $\sA'=\tau_{\leqslant 1}(\sA)$, et $\sC'$ se déduit de $\sC$ 
comme $\sB'$ de $\sB$, d'où un diagramme commutatif de suites exactes courtes 
\begin{equation*}\tag{3.2.3}\label{V:eq:3-2-3}
\xymatrix{
  0 \ar[r] 
    & \sA \ar[r] 
    & \sB \ar[r] 
    & \sC \ar[r] 
    & 0 \\
  0 \ar[r] 
    & \tau_{\leqslant 1} \sA \ar[r] \ar[u] \ar[d] 
    & \sB' \ar[r] \ar[u] \ar[d] 
    & \sC' \ar[r] \ar[u] \ar[d] 
    & 0 \\
  0 \ar[r] 
    & \h_c^1(X,\dmu_n)[1] \ar[r] 
    & \pic_D^0(\bar X)[1] \ar[r] 
    & \pic_D^0(\bar X)[1] \ar[r] 
    & 0
}
\end{equation*}
où la dernière ligne doit être vue comme une suite exactes de complexes 
de faisceaux réduits au degré $1$ sur $X$. 
\begin{enumerate}[a)]
  \item $e\in \hh^2(X,\sA)$ provient de 
    $\widetilde e\in \hh^2(X,\tau_{\leqslant 1}\sA)$ (car son image dans 
    $E^{02}$ est nulle). On cherche l'image $\bar e$ de $\widetilde e$ dans 
    $\hh^2(X,\h_c^1(X,\dmu_n)[-1]) = \h^1(X,\h_c^1(X,\dmu_n))$.
  \item $e_1\in \hh^1(X,\sC)$ provient de $\widetilde e_1\in \hh^1(X,\sC')$, 
    car son image $f_0$ dans $\h^0(X,\pic_D(\bar X))$ est dans 
    $\h^0(X,\pic_D^0(\bar X))$. On peut prendre 
    $\widetilde e=\partial \widetilde e_1$. 
  \item On a donc $\bar e = \partial f_0$, où $\partial $ est défini par 
    \eqref{V:eq:3-2-2}, et où on utilise les isomorphismes usuels 
    $\hh^2(X,\sF[-1]) = \h^1(X,\sF)$. Ces isomorphismes anticommutent à 
    $\partial$, de sorte que $\bar e=-\partial f_0$, pour $\partial$ défini 
    par le suite exacte 
    \[\xymatrix{
      0 \ar[r] 
        & \h_c^1(X,\dmu_n) \ar[r] 
        & \pic_D^0(\bar X) \ar[r] 
        & \pic_D^0(\bar X) \ar[r] 
        & 0 
    }\]
    On conclut par (\hyperref[IV]{Cycles}, \ref{IV:1-1-1}). 
\end{enumerate}





\subsection{}\label{V:3-3}

Pour tout homomorphisme $\varphi:\h_c^1(X,\dmu_n)\to \dZ/n$, soit 
$\varphi(u)\in \h^1(X,\dZ/n)$ l'image de $u$ par 
$\varphi:\h^1(X,\h_c^1(X,\dmu_n)) \to \h^1(X,\dZ/n)$. Si 
$x\in \h_c^1(X,\dmu_n)$, le cup-produit $\varphi(u)\smallsmile x$ est dans 
$\h_c^2(X,\dmu_n)$. On se propose de prouver que 





\begin{proposition_}\label{V:3-4}
$\tr(\varphi(u)\smallsmile x) = \varphi(x)$.
\end{proposition_}

Cette identité se déduit, par application de $\varphi$, de 
\begin{equation*}\tag{3.4.1}\label{V:eq:3-4-1}
  \tr(u\smallsmile x) = x 
\end{equation*}
où $\tr$ est cette fois l'application 
\[\xymatrix{
  \tr:\h_c^2(X,\h_c^1(X,\dmu_n)\otimes\dmu_n) \ar[r] 
    & \h_c^1(X,\dmu_n) \text{,}
}\]
de sorte que $x\mapsto \tr(u\smallsmile x)$ est l'application composée 
\small
\begin{equation*}\tag{3.4.2}\label{V:eq:3-4-2}
\begin{aligned}
&\xymatrix{
  \h_c^1(X,\dmu_n) \ar[r]^-{u\otimes x} 
    & \h^1(X,\h_c^1(X,\dmu_n)) \otimes \h_c^1(X,\dmu_n) 
    & \ar[l]_-\sim \h^1(X,\dZ/n)\otimes \h_c^1(X,\dmu_n) \otimes \h_c^1(X,\dmu_n)} \\ &\xymatrix{
    \ar[r]^-{(1)} 
      & \h_c^2(X,\dmu_n) \otimes \h_c^1(X,\dmu_n) \ar[r]^-{\tr\otimes\operatorname{id}} 
      & \h_c^1(X,\dmu_n)
}
\end{aligned}
\end{equation*}
\normalsize
où (1) est donné par 
$a\otimes b\otimes c\mapsto (a\smallsmile c)\otimes b$. 

Exprimons ce morphisme en terme du morphisme trace 
$\tr:\h_c^3(X\times X,\dmu_n^{\otimes 2}) \to \h_c^2(X,\dmu_n)$ relatif à la 
seconde projection. Pour $x\in \h_c^3$, $\tr(x)$ se calcule ainsi: de ses 
composantes de K\"unneth, on ne garde que celle de type $(2,1)$, et on applique 
$\tr:\h_c^2(X,\dmu_n) \to \dZ/n$. 

Notons $u'$ l'image de $u$ par l'application 
\begin{align*}
&\xymatrix{
  \h^1(X,\h_c^1(X,\dmu_n)) 
    & \ar[l]_-\sim \h^1(X,\dZ/n) \otimes \h_c^1(X,\dmu_n) = \h^1(X,\dZ/n)\otimes \h^1(\bar X,j_!\dmu_n)
} \\
&\xymatrix{
  \ar[r] 
    & \h^2(X\times \bar X,\dZ/n\boxtimes j_!\dmu_n) \text{.}
}
\end{align*}
Vu la relation entre K\"unneth et cup-produit, on a 
\[
  \tr(u\smallsmile x) = -\tr(u'\smallsmile \operatorname{pr}_1^\ast x) \text{.}
\]
Le second cup-produit est celui de (\hyperref[IV]{Cycle} \ref{IV:1-2-4}):
\[\xymatrix{
  \h^2(X\times \bar X,\dZ/n\boxtimes j_!\dmu_n) \otimes \h^1(\bar X\times \bar X,j_!\dmu_n\boxtimes \dZ/n) \ar[r] 
    & \h^3(\bar X\times \bar X,j_!\dmu_n \boxtimes j_!\dmu_n) \text{.}
}\]
Le signe -- provient de la permutation de deux symboles de degré $1$ dans 
(1). Puisque $c\smallsmile\operatorname{pr}_1^\ast x$ et 
$c^{1,1}\smallsmile \operatorname{pr}_1^\ast$ ont même $(2,1)$-composante de 
K\"unneth, et que $u'=c^{1,1}$, on a 
\[
  \tr(u\smallsmile x) = \tr(c\smallsmile \operatorname{pr}_1^\ast x) \text{.}
\]
Le cup-produit à droite peut cette fois s'interpréter comme le 
cup-produit (\hyperref[IV]{Cycle} \ref{IV:1-2-5}) de 
$\cl(\Delta)\in \h_\Delta^2(X\times X,\dmu_n)$ par 
$\operatorname{pr}_1^\ast x\in \h_c^1(\Delta,\dmu_n)$. Sur $\Delta$, on a 
$\operatorname{pr}_1=\operatorname{pr}_2$, d'ù 
\[
  \tr(c\smallsmile \operatorname{pr}_1^\ast x) = \tr(\cl(\Delta)\smallsmile \operatorname{pr}_1^\ast x) = \tr(\cl(\Delta)\smallsmile \operatorname{pr}_2^\ast x) = \tr(\cl(\Delta))\smallsmile x = x \text{.}
\]










\section{Compatibilité SGA 4 XVIII 3.1.10.3}\label{V:4}

Dans \cite[XVIII]{sga4}, en proie au découragement, je n'ai pas vérifié 
la compatibilité du titre. Bien qu'elle s'avère n'être que fumée, je me 
crois tenu du réparer ici cette lacune. Pour $f:X\to S$ séparé de type 
fini, il s'agissait de vérifier que le morphisme d'adjonction 
$\R f_!\R f^!\sL\to \sL$ est compatible à toute localisation étale 
$k:V\to S$. Désignant par $k$ et $f$ plusieurs flèches, comme ci-dessous, 
il s'agit de vérifier une commutativité:
\[\xymatrix{
  X_V \ar[r]^-k \ar[d]^-f 
    & X \ar[d]^-f \\
  V \ar[r]^-k 
    & S 
} \qquad 
\xymatrix{
  k^\ast \R f_! \R f^!\sL \ar[r]^-{k^\ast(\text{adj})} \ar[d]^-\sim 
    & k^\ast \sL \ar@{=}[d] \\
  \R f_! \R f^! k^\ast \sL \ar[r]^-{\text{adj}} 
    & k^\ast \sL \text{.}
}\]
Au niveau des complexes, la commutativité voulue s'écrit (loc. cit.) 
\[\xymatrix{
  k^\ast f_!^\bullet f^!_\bullet \sL \ar[r] \ar[d]^-{(1)} 
    & k^\ast \sL \ar@{=}[d] \\
  f_!^\bullet f^!_\bullet k^\ast \sL \ar[r] 
    & k^\ast \sL \text{.}
}\]
On la vérifie composante par composante, ce qui ramène à une 
compatibilité pour $\sL$ un faisceau et pour chacune des paires de foncteurs 
adjoints $(f_i^!,f_!^i)$, notés simplement $f^!$ et $f_!$. La flèche (1) 
est définie à partir de l'isomorphisme $(2):k^\ast f_!=f_! k^\ast$, et d'un 
morphisme $(3):k^\ast f^! \to f^! k^\ast$ qui s'en déduit. Dans loc. cit., on 
introduit d'abord un morphisme $(4):k_! f_!\to f_! k_!$, déduit de $(2)$ par 
adjonction de $k^\ast=k^!$ et de 
$k_!:k_! f_!\to k_! f_! k^\ast k_!\to k_! k^\ast f_! k_! \to f_! k_!$, et on 
déduit $(3)$ de $(4)$ par adjonction. Il revient au même de déduire $(3)$ 
de $(2)$ par adjonction de $f_!$ et 
$f^!:k^\ast \to f^! f_! k^\ast f^!\to f^! k^\ast f_! f^! \to f^! k^\ast$. 

Écrivant $k$ pour $k^\ast$ et omettant $\sL$, la commutativité est alors 
celle du bord extérieur de 
\[\xymatrix@=.5cm{
  k f_! f^! \ar[r] \ar@{=}[d]
    & f_! k f^! \ar[r] 
    & f_! f^! f_! k f^! \ar[r] 
    & f_! f^! k f_! f^! \ar@{=}[d] \\
  k f_! f^! \ar[d] 
    & & & \ar[lll] f_! f^! k f_! f^! \ar[d] \\
  k 
    & & & \ar[lll] f_! f^! k \text{.}
}\]
En première ligne, on trouve un homomorphisme 
$k _! f^! \to f_! f^! k f_! f^!$, défini parce que $k f_! f^!$ est de la 
forme $f_! X$ ($X=k f^!$); il admet pour rétraction le morphisme 
d'adjonction en seconde ligne, et la commutativité du carré inférieur la 
fonctorialité de ce morphisme d'adjonction. 





% !TEX root = sga4.5.tex

\chapter{Applications de la formule des traces aux sommes trigonométriques}\label{VI}
\chaptermark{Applications aux sommes trigonom\'etriques}










Dans cet expos\'e, j'explique comment la formule des traces permet de calculer 
ou d'\'etudier diverses sommes trigonom\'etriques et comment, jointe \`a la 
conjecture de Weil, elle peut permettre de les majorer.

Les deux premiers paragraphes donnent un ``mode d'emploi'' de ces outils. Le 
paragraphe 3 est un espos\'e, dans un langage cohomologique, des r\'esultats de 
Weil sur les sommes \`a $1$ variable. Les paragraphes $4$ \`a $6$ forment une 
\'etude d\'etaill\'ee des sommes de Gauss et de Jacobi -- y inclus les 
r\'esultats anciens et r\'ecents de Weil sur les caract\`eres de Hecke 
d\'efinis par des sommes de Jacobi. Au paragraphe $7$, nous \'etudions une 
g\'en\'eralisation \`a plusiers variables des sommes de Kloosterman. Enfin, au 
paragraphe $8$, on trouvera quelques indications sur d'autres usages qui ont 
\'et\'e faits on peuvent \^etre faits de ces m\'ethodes. 










\section*{Notations}\label{VI:0}





\subsection{}\label{VI:0-1}

On utilise les notations de \hyperref[II]{Rapport}, paragraphe \ref{II:1}. On 
aura souvent \`a consid\'erer une extension finie $\dF_{q^n}\subset \dF$ de 
$\dF_q$. On notera par un indice $0$ un objet sur $\dF_q$, et par un indice $1$ 
un objet sur $\dF_{q^n}$. Remplacer un indice $0$ par un indice $1$ (resp. 
supprimer l'incide) signifie qu'on \'etend les scalaires \`a $\dF_{q^n}$ (resp. 
\`a $\dF$).





\subsection{}\label{VI:0-2}

On d\'esigne par $\ell$ un nombre premier $\ne p$. Nous utilisons librement le 
language des $\dQ_\ell$-faisceaux, ainsii que celui des $E_\lambda$-faisceaux, 
pour $E_\lambda$ une extension finie de $\dQ_\ell$ (cf. \hyperref[II]{Rapport}, 
paragraphe \ref{II:2} et sp\'ecialement \ref{II:2-11}). 





\subsection{}\label{VI:0-3}

Soit $H^\bullet$ un espace vectoriel gradu\'e. Si $T$ est un endomorphisme de 
$H^\bullet$, on pose (cf. \hyperref[IV]{Cycle}, \ref{IV:eq:1-3-5}) 
\[
  \tr(T,H^\bullet) = \sum (-1)^i \tr(T,H^i) \text{.}
\]










\section{Principes}\label{VI:1}





\subsection{}\label{VI:1-1}

Soient $X_0$ un sch\'ema s\'epar\'e de type fini sur $\dF_q$, $E_\lambda$ une 
extension finie de $\dQ_\ell$ et $\sF_0$ un $E_\lambda$-faisceau sur $X-0$. La 
formule des traces dit que 
\begin{equation*}\tag{1.1.1}\label{VI:eq:1-1-1}
  \sum_{x\in X^F} \tr(F_x^\ast,\sF) = \sum (-1)^i \tr\left(F^\ast,\h_c^i(X,\sF)\right) \text{.}
\end{equation*}

Avec la notation \ref{VI:0-3}, le membre de droite s'\'ecrit simplement 
$\tr(F^\ast,\h_c^\bullet(X,\sF))$. 

Cette formule des traces pour les $E_\lambda$-faisceaux peut soit \^etre 
d\'eduite de \hyperref[II]{Rapport} \ref{II:4-10} par passage \`a limite (cf. 
\hyperref[II]{Rapport} \ref{II:4-11} \`a \ref{II:4-13}), soit \^etre d\'eduite 
de la formule des traces pour les $\dQ_\ell$-faisceaux (\hyperref[II]{Rapport} 
\ref{II:3-2}) par la m\'ethode de (\hyperref[III]{Functions $L$ mod. $\ell^n$}, 
\ref{III:4-3}). 

Nous allons interpr\'eter diverses sommes trigonom\'etriques comme le membre de 
gauche de \eqref{VI:eq:1-1-1}, pour $X_0$ et $\sF_0$ convenables. 





% NOTE: in this section, following the source (but not convention in the 
% LaTeX version, torsors are denoted in roman (not script) letters
\subsection{}\label{VI:1-2}

Soit $A$ un groupe fini. Un \emph{$A$-torseur} sur un sch\'ema $X$ est un 
faisceau $T$ sur $X$, muni d'une action \`a droite de $A$, qui, localement 
(pour la topologie \'etale) sur $X$, est isomorphe au faisceau constant $A$ sur 
lequel $A$ agit par translation \`a droite. 

Si $\tau:A\to B$ est un homomorphisme, et $T$ un $A$-torseur, il existe \`a 
isomorphisme unique pr\`es un et un seul $B$-torseur $\tau(T)$, muni d'un 
morphisme de faisceaux $\tau:T\to \tau(T)$ tel que 
\begin{equation*}\tag{1.2.1}\label{VI:eq:1-2-1}
  \tau(t a) = \tau(t) \tau(a)\text{.}
\end{equation*}

Si $T$ est un $A$-torseur sur $X$, et $\rho$ une repr\'esentation lin\'eaire 
de $A$: $\rho:A\to \gl(V)$ ($V$ un espace vectoriel de dimension finie sur 
$E_\lambda$), il existe \`a isomorphisme unique pr\`es un seul 
$E_\lambda$-faisceau $\sF$, lisse de rang $\dim(V)$, muni d'un morphisme de 
faisceaux 
\begin{equation*}\tag{1.2.2}\label{VI:eq:1-2-2}
  \rho:T\to \underline{\operatorname{Isom}}(V,\sF) \text{.}
\end{equation*}
tel que $\rho(t a) = \rho(t) \rho(a)$. On le note $\rho(T)$. Pour 
$\tau:A\to B$ et $\rho$ une repr\'esentation de $B$, on a un isomorphisme 
canonique $\rho(\tau(T)) = (\rho\tau)(T)$. 

Dans ce paragraphe (sauf l'appendice), on ne consid\`ere que le cas o\`u $A$ 
est commutatif et o\`u $V=E_\lambda$: $\rho$ est un caract\`ere 
$A\to E_\lambda^\times$, et $\rho(T)$ un $E_\lambda$-faisceau lisse de rang 
un. Le morphisme \eqref{VI:eq:1-2-2} s'interpr\`ete comme un morphisme partout 
non nul 
\begin{equation*}\tag{1.2.3}\label{VI:eq:1-2-3}
  \rho:T\to \rho(T) 
\end{equation*}
tel que $\rho(t a) = \rho(a)\rho(t)$. 





\subsection{}\label{VI:1-3}

Soient $S$ un sch\'ema, $G$ un sch\'ema en groupe commutatif sur $S$, et $G'$ 
une extension de sch\'emas en groupes commutatifs sur $S$
\[\xymatrix{
  0 \ar[r] 
    & A \ar[r] 
    & G' \ar[r]^-\pi 
    & G \ar[r] 
    & 0 \text{.}
}\]
Le faisceau $T$ des sections locales de $\pi$ est alors un $A$-torseur sur 
$G$. 

Si $X$ est un sch\'ema sur $S$, et $f,g$ deux $S$-morphismes de $X$ dans $G$, 
du fait que $T$ est d\'efini par une extension, on a un isomorphisme 
canonique 
\begin{equation*}\tag{1.3.1}\label{VI:eq:1-3-1}
  (f+g)^\ast T = f^\ast T + g^\ast T \text{.}
\end{equation*}
A gauche, $f+g$ est la somme de $f$ et $g$, au sens de la loi de groupe $G$: 
\`a droite, $+$ d\'esigne une somme de torseurs. 

Si $f:X\to G$ se factorise par $G'$, $f^\ast T$ est trivial (et un 
factorisation donne une trivialisation); \`a isomorphisme (non unique) pr\`es, 
le $A$-torseur $f^\ast T$ ne d\'epend que de l'image de $f$ dans 
$\hom_S(X,G)/\pi\hom_S(X,G')$: il est donn\'e par le morphisme $\partial$ dans 
la suite exacte 
\[\xymatrix{
  \hom(X,G') \ar[r] 
    & \hom(X,G) \ar[r]^-\partial 
    & \h^1(X,A) \text{.}
}\]

Nous appliquerons ces constructions aux suites exactes de Kummer 
$0 \to \dmu_n \to \dG_m \to \dG_m \to 0$, et aux torseurs de Lang. 





\subsection{Le torseur de Lang}\label{VI:1-4}

Soit $G_0$ un groupe alg\'ebrique commutatif connexe sur $\dF_q$ (pour le cas 
non commutatif, voir l'appendice). La loi de groupe est not\'ee 
\emph{multiplicativement}. L'isog\'enie de Lang 
\[
\fL:G_0 \to G_0: x\mapsto F x\cdot x^{-1}
\]
est \'etale; son image, un sous-groupe ouvert de $G_0$, ne peut \^etre que 
$G_0$ lui-m\^eme; son noyau est le groupe fini $G_0(\dF_q)$. Le \emph{torseur 
de Lang} $L$ est le $G_0(\dF_q)$-torseur sur $G_0$ d\'efini par la suite 
exacte 
\begin{equation*}\tag{1.4.1}\label{VI:eq:1-4-1}
\xymatrix{
  0 \ar[r] 
    & G_0(\dF_q) \ar[r] 
    & G_0 \ar[r]^-\fL 
    & G_0 \ar[r] 
    & 0 \text{.}
}
\end{equation*}


\paragraph{Notation}
On note encore $L$ et $\fL$ (ou $L_{(q)}$ et $\fL_{(q)}$) les objects d\'eduits 
de $L$ et $\fL$ par extension du corps de base. En cas de besoin, on 
pr\'ecisera par un indice $0$ ou $1$.


\paragraph{Exemples}
Pour $G_0=\dG_a$ ou $\dG_m$, les suites \eqref{VI:eq:1-4-1} s'\'ecrivent 
\begin{align*}\tag{1.4.2}\label{VI:eq:1-4-2}
&\xymatrix{
  0 \ar[r] 
    & \dF_q \ar[r] 
    & \dG_a \ar[r]^-{x^q-x} 
    & \dG_a \ar[r] 
    & 0 
} \\
\tag{1.4.3}\label{VI:eq:1-4-3}
&\xymatrix{
  0 \ar[r] 
    & \dmu_{q-1} \ar[r] 
    & \dG_m \ar[r]^-{x^{q-1}} 
    & \dG_m \ar[r] 
    & 0
}
\end{align*}





\subsection{}\label{VI:1-5}

Calculons l'endomorphisme $F^\ast$ de la fibre du torseur de Lang en 
$\gamma\in G^F=G_0(\dF_q)$. Si $g\in \fL^{-1}(\gamma)\subset G$, on a 
$F g = F g\cdot g^{-1}\cdot g = \gamma g$. D\`es lors (\hyperref[II]{Rapport}, 
\ref{II:1-2}) 
\begin{equation*}\tag{1.5.1}\label{VI:eq:1-5-1}
  \text{sur $L_0(G_0)_\gamma\simeq \fL^{-1}(\gamma)$, $F^\ast$ est $g\mapsto g\gamma^{-1}$.}
\end{equation*}





\subsection{}\label{VI:1-6}

Sur $\dF_{q^n}$, l'identit\'e $F_{(q^n)}=F_{(q)}^n$ entre endomorphismes de 
$G_1$ implique que $\fL_{(q^n)}=\fL_{(q)}\circ \prod_{i=0}^{n-1} F_{(q)}^i$. 
Sur $G_0(\dF_{q^n})$, $F_{(q)}^i$ agit comme l'\'el\'ement 
$x\mapsto x^{q^i}$ de $\gal(\dF_{q^n}/\dF_q)$. Sur $G_0(\dF_{q^n})$, 
$\prod F_{(q)}^i$ est donc le compos\'e de la norme 
$G_0(\dF_{q^n})\to G_0(\dF_q)$ et de l'inclusion 
$G_0(\dF_q)\subset G_0(\dF_{q^n})$: le diagramme 
\begin{equation*}\tag{1.6.1}\label{VI:eq:1-6-1}
\xymatrix{
  0 \ar[r] 
    & G_0(\dF_{q^n}) \ar[r] \ar[d]^-N 
    & G_1 \ar[r]^-{\fL_{(q^n)}} \ar[d]^-{\prod_{i=0}^{n-1} F_{(q)}^i} 
    & G_1 \ar[r] \ar@{=}[d] 
    & 0 \\
  0 \ar[r] 
    & G_0(\dF_q) \ar[r] 
    & G_q \ar[r]^-{\fL_{(q)}} 
    & G_1 \ar[r] 
    & 0
}
\end{equation*}
est commutatif. Dans le langage des torseurs, ceci fournit un isomorphisme 
canonique entre $G_0(\dF_q)$-torseurs sur $G_1$ 
\begin{equation*}\tag{1.6.2}\label{VI:eq:1-6-2}
  N L_{(q^n)} = L_{(q)} \text{.}
\end{equation*}





\begin{definition_}\label{VI:1-7}
Soient $f_0:X_0\to G_0$ un morphisme et 
$\chi:G_0(\dF_q)\to E_\lambda^\times$ un caract\`ere. On pose 
$\sF(\chi,f_0) = \chi^{-1}(f_0^\ast (L_0(G_0))) = f_0^\ast\chi^{-1}(L_0(G_0))$. 
\end{definition_}

On a les propri\'et\'es de fonctorialit\'e suivantes:
\begin{enumerate}[a)]
  \item $\sF(\chi,f_0)$ est bimultiplicatif en $\chi$ et $f_0$: 
    \begin{align*}\tag{1.7.1}\label{VI:eq:1-7-1}
      \sF(\chi,f_0'\cdot f_0'') &= \sF(\chi,f_0')\otimes \sF(\chi,f_0'') \text{,} \\
      \tag{1.7.2}\label{VI:eq:1-7-2}
      \sF(\chi'\chi'',\sF_0) &= \sF(\chi',f_0)\otimes \sF(\chi'',f_0) \text{.}
    \end{align*}
    Cela r\'esulte de la d\'efinition de $\chi(T)$, pour $T$ un torseur, joint 
    \`a \eqref{VI:eq:1-3-1} pour \eqref{VI:eq:1-7-1}. 
  \item Pour $g_0:Y_0\to X_0$ un morphisme, on a 
    \begin{equation*}\tag{1.7.3}\label{VI:eq:1-7-3}
      \sF(\chi,f_0\circ g_0) = g_0^\ast \sF(\chi,f_0) \text{.}
    \end{equation*}
    En particulier, $\sF(\chi,f_0) = f_0^\ast \sF(\chi)$, o\`u $\sF(\chi)$ 
    d\'esigne $\sF(\chi,\operatorname{id}_{G_0})$, 
  \item Pour $u_0:G_0\to H_0$ un morphisme, et 
    $\chi:H_0(\dF_q) \to E_\lambda^\times$, on a 
    \begin{equation*}\tag{1.7.4}\label{VI:eq:1-7-4}
      \sF(\chi,u_0 f_0) = \sF(\chi u_0,f_0) \text{.}
    \end{equation*}
  \item Pour $G_0=\prod_{i\in I} G_0^i$, $\chi$ de coordonn\'ees $\chi_i$ 
    ($i\in I$) et $f_0$ de coordonn\'ees $f_0^i$, il r\'esulte de 
    \eqref{VI:eq:1-7-1} \`a \eqref{VI:eq:1-7-4} que 
    \begin{equation*}\tag{1.7.5}\label{VI:eq:1-7-5}
      \sF(\chi,f_0) = \bigotimes_{i\in I} \sF(\chi_i,f_0^i) \text{.}
    \end{equation*}    
\end{enumerate}

Noter le $\chi^{-1}$ dans la d\'efinition \ref{VI:1-7}. Il assure que, pour 
$x\in X^F$, on ait sur $\sF(\chi,f_0)_x$ 
\begin{equation*}\tag{1.7.6}\label{VI:eq:1-7-6}
  F_x^\ast = \chi f_0(x)
\end{equation*}
(utiliser que la fibre en $x$ du morphisme \eqref{VI:eq:1-2-2}, pour 
$\rho=\chi^{-1}$, commute \`a $F_x^\ast$, et \eqref{VI:eq:1-5-1}). 

Si $\sF(\chi,f_0)_1$ est le faisceau d\'eduit de $\sF(\chi,f_0)$ par extension 
du corps de base \`a $\dF_{q^n}$, on d\'eduit de \eqref{VI:eq:1-6-2} que 
\begin{equation*}\tag{1.7.7}\label{VI:eq:1-7-7}
  \sF(\chi,f_0)_1 = \sF(\chi\circ N,f_1) \text{.}
\end{equation*}





\subsection{Abus de notations}\label{VI:1-8}

(i) Si $\Xi$ est une notation pour l'application compos\'ee 
$\chi f_0:X_0(\dF_q) \to E_\lambda^\times$, on \'ecrira parfois $\sF(\Xi)$ au 
lieu de $\sF(\chi,f_0)$. Grâce \`a \eqref{VI:eq:1-7-1} \`a \eqref{VI:eq:1-7-5}, 
on ne risque gu\`ere d'ambiguit\'e. Par exemple:
\begin{enumerate}[a)]
  \item on \'ecrit $\sF(\chi)$ pour $\sF(\chi,\operatorname{id}_{G_0})$ 
    (notations d\'ej\`a utilis\'ee en \ref{VI:1-7}b)); 
  \item en \'ecrit $\sF(\chi f_0)$ pour $\sF(\chi,f_0)$;
  \item avec les notations de \ref{VI:1-7}d), on \'ecrit 
    $\sF(\prod \chi_i f_0^i)$ pour $\sF(\chi,f))$. 
\end{enumerate}

Avec cette notation, \eqref{VI:eq:1-7-4} exprime que l'\'ecriture 
$\sF(\chi u_0 f_0)$ n'est pas ambiguë, \eqref{VI:eq:1-7-1} \eqref{VI:eq:1-7-2} 
\eqref{VI:eq:1-7-5} expriment une multiplicativit\'e de $\sF(\Xi)$ en $\Xi$, 
et \eqref{VI:eq:1-7-6} se r\'ecrit $F_x^\ast = \Xi(x)$ sur $\sF(\Xi)_x$.

(ii) Si $X$ est un sch\'ema sur une extension $k$ de $\dF_q$, et que $f$ est un 
morphisme de $X$ dans $G_0\otimes_{\dF_q} k$, on notera encore $\sF(\chi,f)$, 
$\sF(\chi f)$, ou $\sF(\chi)$ (pour $f$ une inclusion) l'image r\'eciproque par 
$f$ du $E_\lambda$-faisceau d\'eduit de $\chi^{-1}(L_0(G))$ par extension des 
scalaires de $\dF_q$ \`a $k$. Si $f_0$ est le compos\'e 
$X\to G_0\otimes_{\dF_q} k \to G_0$, c'est encore le $\sF(\chi,f_0)$ de 
\ref{VI:1-7}. Pour $k$ est une extension finie de $\dF_q$, on a 
$\sF(\chi f) = \chi(\chi N_{k/\dF_q} f)$. 

Appliquant \eqref{VI:eq:1-1-1} \`a \eqref{VI:eq:1-7-6} et \eqref{VI:eq:1-7-7}, 
on trouve: 





% NOTE: originally a 'scholium'
\begin{theorem_}\label{VI:1-9}
Soient $S_0$ un sch\'ema s\'epar\'e de type fini sur $\dF_q$, $G_0$ un groupe 
alg\'ebrique commutatif connexe sur $\dF_q$, $f_0:S_0\to G_0$ un morphisme et 
$\chi:G_0(\dF_q) \to E_\lambda^\times$. On a 
\begin{equation*}\tag{1.9.1}\label{VI:eq:1-9-1}
  \sum_{s\in S_0(\dF_q)} \chi f_0(s) = \tr(F^\ast,\h_c^\bullet(S,\sF(\chi,f_0))) 
\end{equation*}
et, pour tout entier $n\geqslant 1$
\begin{equation*}\tag{1.9.2}\label{VI:eq:1-9-2}
  \sum_{s\in S_0(\dF_{q^n})} \chi\circ N_{\dF_{q^n}/\dF_q} f_0(s) = \tr({F^\ast}^n, \h_c^\bullet\left(S,\sF(\chi,f_0))\right) \text{.}
\end{equation*}
\end{theorem_}





\subsubsection{Remarque}\label{VI:1-9-3}

Prenons pour $G_0$ un produit. Avec les notations de \ref{VI:1-7}d) et 
\ref{VI:1-8}c), la formule \eqref{VI:eq:1-9-2} devient 
\[
  \sum_{s\in S_0(\dF_{q^n})} \prod_i \chi_i\circ N_{\dF_{q^n}/\dF_q} (f_0^i(s)) = \tr\Big({F^\ast}^n, \h_c^\bullet\Big(S,\sF\Big(\prod_i \chi_i f_0^i\Big)\Big)\Big) \text{.}
\]





\subsubsection{Remarque}\label{VI:1-9-4}

Prenons $G_0=\{e\}$. La formule \eqref{VI:eq:1-9-1} devient 
\[
  \left|S_0(\dF_q)\right| = \sum_{s\in S_0(\dF_q)} 1 = \tr(F^\ast,\h_c^\bullet(S,\dQ_\ell)) \text{.}
\]

Il est rare qu'on puisse explicitement calculer le membre de droite de 
\eqref{VI:eq:1-9-1}. Voici un exemple amusant, avec $G_0=\{e\}$. 





\subsection{Exemple}\label{VI:1-10}

Une quadrique projective non singuli\`ere de dimension impaire sur \texorpdfstring{$\dF_q$}{Fq} a le m\^eme nombre de points rationnels que l'espace projectif sur \texorpdfstring{$\dF_q$}{Fq} de la m\^eme dimension.

Si, sur $\dC$, $X$ est une hypersurface quadrique non singuli\`ere dans 
l'espace projectif $\dP^{2 N}$, et $Y$ un hyperplan, on sait que pour 
$i\leqslant 2\dim(X)=2\dim(Y)$, les inclusions 
$X\hookrightarrow \dP^{2 N}\hookleftarrow Y$ induisent des isomorphismes (en 
cohomologie ordinaire) 
\[\xymatrix{
  \h^i(X,\dQ) 
    & \ar[l]_-\sim \h^i(\dP^{2 N},\dQ) \ar[r]^-\sim 
    & \h^i(Y,\dQ) \text{.}
}\]

Par sp\'ecialisation, il en r\'esulte que si $X_0'$ est une hypersurface 
quadrique non singuli\`ere dans l'espace projectif $\dP_0^{2 N}$ sur $\dF_q$, 
et $Y_0'$ un hyperplan, les inclusions 
$X_0'\hookrightarrow \dP_0^{2 N}\hookleftarrow Y_0'$ induisent des 
isomorphismes 
\[\xymatrix{
  \h^i(X',\dQ_\ell) 
    & \ar[l]_-\sim \h^i(\dP^{2 N},\dQ_\ell) \ar[r]^-\sim 
    & \h^i(Y',\dQ_\ell) \text{.}
}\]
Ces isomorphismes commutent \`a $F^\ast$, et on applique la formule des traces. 





\subsection{}\label{VI:1-11}

On peut aussi utiliser la formule des traces pour comprendre les formules 
classiques donnant le nombre de points rationnels des groupes lin\'eaires sur 
les corps finis, et ceux de certains espaces homog\`enes (voir le paragraphe 
8). 





\subsection{}\label{VI:1-12}

On dispose d'un dictionnaire permettant de traduire en termes cohomologiques 
ques divers types de manipulations classiques sur les sommes 
trigonom\'etriques. Ce dictionnaire sera donn\'e au paragraphe 2. L'\'enonc\'e 
cohomologique \'etant plus ``g\'eom\'etrique,'' il pourra parfois s'appliquer 
\`a des situations o\`u l'argument classique ne s'applique qu'apr\`es une 
extension du corps fini de base. Voici un exemple (un cas particulier d'un 
th\'eor\`eme de Kazhdan). 





\begin{theorem_}[Kazhdan]\label{VI:1-13}
Soit $X_0$ un sch\'ema sur $\dF_q$. Supposons qu'il existe une action $\rho$ de 
$\dG_a$ sur $X$, et un morphisme $f:X\to Y$ de sch\'emas sur $\dF$, faisant de 
$X$ un $\dG_a$-torseur sur $Y$. Alors, le nombre de points rationnels de $X_0$ 
est divisible par $q$.
\end{theorem_}

Supposons d'abord que $\rho$, $Y$ et $f$ soient d\'efinis sur $\dF_q$. 


\paragraph{Argument classique}
Les fibres de $f:X_0(\dF_q)\to Y_0(\dF_q)$ ont toutes $q$ \'el\'ements: 
$|X_0(\dF_q)| = q\cdot |Y_0(\dF_q)|$, d'o\`u la divisibilit\'e. 


\paragraph{Traduction cohomologique}
Les faisceaux images directes sup\'erieures $\R^i f_!\dQ_\ell$ sont 
\[
  \R^i f_!\dQ_\ell = \begin{cases}
                       \dQ_\ell(-1) & \text{si $i = 2$} \\
                       0            & \text{pour $i\ne 2$}
                     \end{cases}
\]
La suite spectrale de Leray de $f$ d\'eg\'en\`ere donc en un isomorphisme 
\[
  \h_c^i(X,\dQ_\ell) = \h_c^{i-2}(Y,\dQ_\ell)(-1) \text{.}
\]
Pour comprendre l'effet d'un twist \`a la Tate sur les valeurs propres de 
Frobenius, le plus commode est d'utiliser le point de vue galoisien 
(\hyperref[II]{Rapport} \ref{II:1-8}), et de noter que la substitution de 
Frobenius $\varphi$ agit sur $\dQ_\ell(1)$ par multiplication par $q$. Les 
valeurs propres de $F^\ast$ sur $\h_c^p(X,\dQ_\ell)$ sont donc les produits par 
$q$ des valeurs propres de $F^\ast$ agissant sur $\h_c^{i-2}(Y,\dQ_\ell)$. On 
sait que celles-ci sont des entiers alg\'ebriques \cite[XXI 5.2.2]{sga7}; d\`es 
lors 

\begin{equation*}\tag{$*$}\label{VI:eq:*}
  \substack{\displaystyle\text{les valeurs propres de $F^\ast$ agissant sur $\h_c^i(X,\dQ_\ell)$ sont} \\
  \displaystyle\text{des entiers alg\'ebriques divisibles par $q$.}}
\end{equation*}


\paragraph{Descente}
Revenons aux hypoth\`eses du th\'eor\`eme. Pour $n$ convenable, on peut 
supposer que $\rho$, $Y$ et $f$ sont d\'efinis sur $\dF_{q^n}$. Le morphisme de 
Frobenius relatif \`a $\dF_{q^n}$ \'etant la puissance $n$-i\`eme de celui 
relatif \`a $\dF_q$, l'\'enonc\'e \eqref{VI:eq:*} pour le sch\'ema sur 
$\dF_{q^n}$ d\'eduit de $X_0$ par extension des scalaires nous fournit: 
\begin{equation*}\tag{$**$}\label{VI:eq:**}
  \substack{\displaystyle\text{les puissances $n$-\`iemes des valeurs propres de $F^\ast$ agissant sur} \\ \displaystyle\text{$\h_c^i(X,\dQ_\ell)$ sont des entiers alg\'ebriques divisibles par $q^n$.}}
\end{equation*}

Cette assertion implique \eqref{VI:eq:*}. Le nombre de points rationnels de 
$X_0$, soit $\sum (-1)^i \tr(F^\ast,\h^i(X,\dQ_\ell))$, est donc entier 
alg\'ebrique divisible par $q$. Ceci implique qu'il soit divisible par $q$ en 
tant qu'entier rationnel. 





\subsection{}\label{VI:1-14}

La formule \ref{VI:1-9-3} contr\^ole la d\'ependance en $n$ de la somme 
trigonom\'etrique au membre de gauche. Elle implique des identit\'es entre une 
somme trigonom\'etrique et celles qui s'en d\'eduisennt par ``extension des 
scalaires.'' Le plus c\'el\`ebre de ces identit\'es est celle de 
Hasse-Davenport: soient $\chi:\dF_q^\times \to \dC^\times$ et 
$\psi:\dF_q\to \dC^\times$ des caract\`eres, $\psi$ non trivial, et 
d\'efinissons la somme de Gauss $\tau(\chi,\psi)$ par 
\begin{equation*}\tag{1.14.1}\label{VI:eq:1-14-1}
  \tau(\chi,\psi) = - \sum_{x\in \dF_q^\times} \psi(x) \chi^{-1}(x) \text{.}
\end{equation*}





\begin{theorem_}[Hasse-Davenport]\label{VI:1-15}
On a 
\begin{equation*}\tag{1.15.1}\label{VI:eq:1-15-1}
  \tau\left(\chi\circ N_{\dF_{q^n}/\dF_q},\psi\circ \tr_{\dF_{q^n}/\dF_q}\right) = \tau(\chi,\psi)^n \text{.}
\end{equation*}
\end{theorem_}

Soient $E\subset \dC$ un corps de nombres contenant les valeurs de $\chi$ et 
$\psi$, et $E_\lambda$ le compl\'et\'e de $E$ en une place de caract\'eristique 
$\ell\ne p$. L'identit\'e \eqref{VI:eq:1-15-1} est une identit\'e dans $E$. Il 
revient au m\^eme de la prouver dans $\dC$, ou dans $E_\lambda$. Nous la 
prouverons dans $E_\lambda$, en regardant $\chi$ et $\psi$ comme \`a valeurs 
dans $E_\lambda^\times$. 

On applique \ref{VI:1-9-3} pour $X_0=\dG_m$ sur $\dF_q$ et 
$G_0=\dG_m\times \dG_a$. Les $\h_c^i(\dG_m,\sF(\chi^{-1}\psi))$ sont nuls pour 
$i\ne 1$, et le $\h_c^1$ est de dimension $1$ (\ref{VI:4-2}). D'apr\`es 
\ref{VI:1-9-3}, 
$\tau(\chi\circ N_{\dF_{q^n}/\dF_q},\psi\circ \tr_{\dF_{q^n}/\dF_q})$ est 
l'unique valeur propre de $(F^\ast)^n$ sur $\h_c^1$, et \eqref{VI:eq:1-15-1} en 
r\'esulte. 

Des exemples analogues sont donn\'es dans Weil \cite[App V]{we74}. 





\subsection{}\label{VI:1-16}

Un $E_\lambda$-faisceau $\sF_0$ sur $X-0$ est dit \emph{ponctuellement de 
poids $n$} si pour tout point ferm\'e $x\in |X_0|$, les valeurs propres de 
$F_x^\ast$ (\hyperref[II]{Rapport} \ref{II:1-2}) sont des nombres alg\'ebriques 
dont tous les conjugu\'es complexes sont de valeur absolue 
$q_x^{n/2}$, o\`u $q_x$ est le nombre d'\'el\'ements de $k(x)$. Le th\'eor\`eme 
suivant sera d\'emontr\'e dans \cite{de80}. 





\begin{theorem_}\label{VI:1-17}
Si $\sF_0$ est ponctuellement de poids $n$, pour toute valeur propre $\alpha$ 
de l'endomorphisme $F^\ast$ de $\h_c^i(X,\sF)$, il existe un entier 
$m\leqslant n+i$ tel que les conjugu\'es complexes de $\alpha$ soient tous de 
valeurs absolue $q^{m/2}$.
\end{theorem_}

Les faisceaux consid\'er\'es en \ref{VI:1-9} sont de poids $0$. Le th\'eor\`eme 
fournit donc pour la somme \ref{VI:eq:1-9-1} (on plut\^ot, pour tous ses 
conjugu\'es complexes) la majoration 
\begin{equation*}\tag{1.17.1}\label{VI:eq:1-17-1}
  \left|\sum_{s\in S_0(\dF_q)} \prod_i \chi_i(f^i(s)) \right| \leqslant \sum_i \dim \h_c^i\left(S,\sF\left(\prod \chi_i f^i\right)\right)\cdot q^{i/2} \text{.}
\end{equation*}





\subsection{Remarques}\label{VI:1-18}

\begin{enumerate}[a)]
  \item Si $\dim S=n$, pour tout $E_\lambda$-faisceau $\sF$ sur $S$, on a 
    \[
      \h_c^i(S,\sF) = 0 \qquad \text{pour $i\notin [0,2n]$.}
    \]
  \item Le groupe $\h_c^0(X,\sF)$ est le groupe des sections globales de $\sF$ 
    sur $S$ dont le support est propre. Si $S$ est connexe, non complet et 
    $\sF$ lisse, on bien si $S$ est connexe, $\sF$ lisse de rang un, et non 
    constant, ce groupe est nul.
  \item Si $S$ est lisse purement de dimension $n$, et $\sF$ lisse, la 
    dualit\'e de Poincar\'e dit que les espaces vectoriels 
    $\h_c^i(X,\sF)$ et $\h^{2n-i}(X,\sF^\vee)(2n)$ sont duax l'un de l'autre 
    (pour l'effet d'un twist \`a la Tate sur les valeurs propres de Frobenius, 
    cf. la $2$\`eme partie de la preuve de \ref{VI:1-13}).
  \item Si $\dim S=n$, la dualit\'e de Poincar\'e permet de calculer comme 
    suite $\h_c^{2n}(S,\sF)$: on prend un ouvert $U$ de $S_\text{red}$, dont le 
    compl\'ementaire est de dimension $<n$, lisse purement de dimension $n$, et 
    sur lequel $\sF$ est lisse. On a alors 
    $\h_c^{2n}(U,\sF) \iso \h_c^{2n}(S,\sF)$, car dans la suite exacte longue 
    de cohomologie, $\h_c^i(S\setminus U,\sF) = 0$ pour $i=2n,2n-1$. Par 
    Poincar\'e, on a donc 
    \[
      \h_c^{2n}(S,\sF) \iso \left(\h^0(U,\sF^\vee)(2n)\right)^\vee \text{.}
    \]
    
    Supposons $U$ connexe, et soit $u$ un point g\'eom\'etrique de $U$. Le 
    ``syst\`eme local'' $\sF$ correspond \`a une repr\'esentation de 
    $\pi_1(U,u)$ sur $\sF_u$, et $\h^0(U,\sF^\vee)$ est 
    $(\sF_u^\vee)^{\pi_1(U,u)}=((\sF_u)_{\pi_1(U,u)})^\vee$ (dual des 
    coinvariants). On a donc 
    \[
      \h_c^{2n}(S,\sF) \simeq (\sF_u)_{\pi_1(U,u)}(-2n) \text{.}
    \]
  \item Ces remarques permettent souvent le calcul de $b_0$ et $b_{2n}$. 
    Calculer les autres $b_i$ peut \^etre difficile. Si $b_{2n}=0$ et que 
    $b_{2n-1}=0$, la majoration \eqref{VI:eq:1-17-1} est une majoration \`a la 
    Lang-Weil, avec, pour $q$ grand, un gain en $\sqrt q$ par rapport \`a la 
    majoration triviale en $O(q^n)$. On prendra toutefois garde que la 
    majoration \eqref{VI:eq:1-17-1} n'implique pas formellement la majoration 
    triviale 
    \[
      \left|\sum_{s\in S_0(\dF_q)} \prod_i \chi_i(f^i(s))\right| \leqslant \left| S_0(\dF_q)\right| \text{,}
    \]
    donc il peut \^etre utile de tenir compte (cf. la preuve de \ref{VI:3-8}). 
\end{enumerate}

J'explique ci-dessous une m\'ethode pour prouver que $b_i=0$ pour $i\ne n$. 
Quand elle s'applique, elle fournit une estimation en $O(q^{n/2}$. La constante 
implicate dans $0$ est $b_n=(-1)^n \chi(S,\sF(\prod \chi_i f^i))$ en peut 
\^etre difficile \`a calculer. 





\begin{proposition_}\label{VI:1-19}
Soient $\bar X$ un sch\'ema propre sur un corps alg\'briquement clos $k$, 
$j:X\hookrightarrow bar X$ un ouvert de $X$ et $\sF$ un $E_\lambda$-faisceaux 
sur $X$. On suppose que 
\begin{enumerate}[\indent a)]
  \item $(j_\ast\sF)_x = 0$ pour $x\in \bar X\setminus X$, i.e. 
    $j_!\sF\iso j_\ast\sF$; 
  \item $\R^i j_\ast \sF = 0$ pour $i>0$.
\end{enumerate}
Alors, les applications 
\[
  \h_c^i(X,\sF) \to \h^i(X,\sF)
\]
sont des isomorphismes. 
\end{proposition_}

Les hypoth\`eses a) b) signifiet que $j_! \sF \iso \R j_\ast \sF$, d'o\`u 
\[
  \h_c^i(X,\sF) = \h^i(\bar X,j_!\sF) \iso \hh^i(\bar X,\R j_\ast \sF) = \h^i(X,\sF) \text{.}
\]
Autrement dit, la suite spectrale de Leray pour $j$ 
\[
  E_2^{pq} = \h^p(\bar X,\R^q j_\ast \sF) \Rightarrow \h^{p+q}(X,\sF) 
\]
d\'eg\'en\`ere, par b) en des isomorphismes 
\[
  \h^i(\bar X,j_\ast\sF) \iso \h^i(X,\sF) \text{,}
\]
tandis que, par a) 
\[
  \h_c^i(X,\sF) = \h^i(\bar X,j_!\sF) \iso \h^i(\bar X,j_\ast\sF) \text{.}
\]





\subsubsection{Exemple}\label{VI:1-19-1}

Si $X$ est une courbe, ou si $\bar X$ est lisse, $X$ le compl\'ement d'un 
diviseur \`a croisements normaux et $\sF$ mod\'er\'ement ramifi\'e, 
l'hypoth\`ese a) de \ref{VI:1-19} (pas d'invariants sous la monodromie locale) 
implique l'hypoth\`ese b). 





\begin{proposition_}\label{VI:1-20}
Soient $X_0$ un sch\'ema affine lisse purement de dimension $n$ sur $\dF_q$ et 
$\sF_0$ un $E_\lambda$-faisceau lisse ponctuellement de poids $m$ sur $X_0$. On 
suppose que les applications $\h_c^i(X,\sF) \to \h^i(X,\sF)$ sont des 
isomorphismes (cf. \ref{VI:1-17}). Alors, $\h_c^i(X,\sF) = 0$ pour $i\ne n$, et 
les conjugu\'es complexes des valeurs propres de $F^\ast$ sur $\h_c^n(X,\sF)$ 
sont de valeur absolue $q^{(m+n)/2}$. 
\end{proposition_}

La dualit\'e de Poincar\'e met en dualit\'e $\h_c^i(X,\sF)$ (resp. 
$\h_c^i(X,\sF^\vee)$) avec $\h^{2n-i}(X,\sF^\vee(n))$ (resp. 
$\h^{2n-i}(X,\sF(n))$). D'apr\`es \cite[XIV 3.2]{sga4}, les $\h^i$ sans support 
sont nuls pour $i>n$. Par dualit\'e, les $\h_c^i$ sont nuls pour $i<n$; puisque 
$\h_c^i(X,\sF) \iso \h^i(X,\sF)$, ces groupes sont nuls pour $i\ne n$. 
Appliquons \ref{VI:1-15} \`a $\sF_0$ et \`a $\sF_0^\vee(n)$ (de poinds 
$-m-2n$). On trouve que les conjugu\'es complexes $\alpha$ des valeurs propres 
de $F^\ast$ sur $\h_c^n(X,\sF) = \h_c^n(X,\sF^\vee(n))^\vee$ v\'erifient 
\begin{align*}
  |\alpha| &\leqslant q^{(m+n)/2} && \text{et} \\
  |\alpha^{-1}| &\leqslant q^{((-m-2n)+n)/2} = q^{-(m+n)/2} \text{,}
\end{align*}
d'o\`u l'assertion. 





\subsection{Remarque}\label{VI:1-21}

Le dernier argument montre que si $X_0$ est un sch\'ema s\'epar\'e lisse sur 
$\dF_q$, que $\sF_0$ est un $E_\lambda$-faisceau lisse ponctuellement de poids 
$m$ sur $X_0$ et que $\h_c^i(X,\sF) \hookrightarrow \h^i(X,\sF)$, alors les 
conjugu\'es complexes des valeurs propres de $F^\ast$ sur $\h_c^i(X,\sF)$ sont 
de valeur absolue $q^{(m+i)/2}$. 





\subsection*{Appendice. L'isog\'enie de Lang dans le cas non commutatif}




\subsection{}\label{VI:1-22}

Soient $G-0$ un groupe alg\'ebrique connexe sur $\dF_q$ et $\fL$ le morphisme 
$x\mapsto F x\cdot x^{-1}$ de $G-0$ dans lui-m\^eme. C'est un orphisme de 
$G_0$-espaces homog\`enes, si \`a la source on fait agir $G-0$ par translation 
\`a gauche $x\mapsto (y\mapsto x y)$, au but par 
$x\mapsto (y\mapsto F x\cdot y\cdot x^{-1}$). On a $\fL(xy) = \fL(x)$ pour 
$y\in G_0(\dF_q)$, et $\fL$ induit un isomorphisme $G_0/G_0(\dF_q)\iso G_0$. 

Comme dans le cas commutatif, $\fL$ fait donc de $G_0$ un $G_0(\dF_q)$-torseur 
sur $G_0$, le \emph{torseur de Lang} $L$. Si 
$\rho:G_0(\dF_q) \to \gl(n,E_\lambda)$ est une repr\'esentation lin\'eaire de 
$G_0(\dF_q)$, nous noterons $\sF(\rho)$ le $E_\lambda$-faisceau $\rho(L)$ 
(\ref{VI:1-2}). 





\subsection{}\label{VI:1-23}

Soit $\gamma\in G^F=G_0(\dF_q)$, et calculons l'endomorphisme $F^\ast$ de 
$\sF(\rho)_\gamma$. Soit $g\in G$ tel que $\fL(g)=\gamma$, d'o\`u un rep\`ere 
$\rho(g)\in \operatorname{Isom}(E_\lambda^n,\sF(\rho)_\gamma)$. Pour 
$e\in E_\lambda^n$, on a 
\[
  F(\rho(g)(e)) = \rho(F g)(e) = \rho(\gamma g)(e) = \rho(g\cdot g^{-1}\gamma g)(e) \text{.}
\]
On verra que $g^{-1} \gamma g\in G_0(\dF_q)$, d'o\`u 
\[
  F(\rho(g)(e)) = \rho(g) \rho(g^{-1}\gamma g)(e)
\]
et 
\begin{equation*}\tag{1.23.1}\label{VI:eq:1-23-1}
  \rho(g)^{-1} (F_\gamma^\ast)^{-1} \rho(g) = \rho(g^{-1} \gamma g) \text{:}
\end{equation*}
$\rho(g)$ identifie l'inverse de $F^\ast$ en $\gamma$ \`a l'automorphisme 
$\rho(g^{-1} \gamma g)$ de $E_\lambda^n$. Reste \`a comprendre ce qu'est 
$g^{-1} \gamma g$. 





\begin{lemma_}\label{VI:1-24}
\begin{enumerate}[(i)]
  \item Si $F g\cdot g^{-1} \in G_0(\dF_q)$, on a 
    $g^{-1} (F g\cdot g^{-1}) g = g^{-1} F g\in G_0(\dF_q)$.
  \item Soient $\gamma\in G_0(\dF_q)$, et $g$ tel que 
    $F g\cdot g^{-1} = \gamma$. La classe de conjugaison de $\gamma'=g^{-1} Fg$ 
    dans $G_0(\dF_q)$ ne d\'epend que de celle de $\gamma$. 
  \item L'application induite par $\gamma\mapsto \gamma'$, de l'ensemble 
    $G_0(\dF_q)^\natural$ des classes de conjugaison de $G_0(\dF_q)$ dans 
    lui-m\^eme, est bijective. Son inverse est l'application pour le groupe 
    oppos\'e. 
\end{enumerate}
\end{lemma_}

(i) Si $\gamma= F g\cdot g^{-1}\in G_0(\dF_q)$, les identit\'es $F g=\gamma g$ 
et 
$F\gamma=\gamma$ donnent $F(g^{-1} F g) = (F g)^{-1} F F g = g^{-1} \gamma^{-1} F(\gamma g) = g^{-1} \gamma^{-1} F \gamma F g = g^{-1} F g$, 
de sorte que $g^{-1} F g \in G_0(\dF_q)$. 

(ii) Pour $\alpha,\gamma\in G_0(\dF_q)$, les $g$ tels que 
$F g\cdot g^{-1} = \gamma$ forment une classe \`a droite sous $G_0(\dF_q)$, et 
si $F g\cdot g^{-1} = \gamma$, on a 
$F(\alpha g \alpha^{-1})(\alpha g \alpha^{-1})^{-1} = \alpha \gamma\alpha^{-1}$. 
Pour $\gamma$ parcourant une classe de conjugaison dans $G_0(\dF_q)$, les $g$ 
tels que $F g\cdot g^{-1} = \gamma$ parcourent donc une double classe sous 
$G_0(\dF_q)$, et $\gamma'= g^{-1} F g$ parcourt une classe de conjugaison. 

(iii) Enfin, l'inverse de $F g\cdot g^{-1}\mapsto g^{-1} F g$ est 
$g^{-1} F g\mapsto F g\cdot g^{-1}$. 





\subsection{}\label{VI:1-25}

La formule \eqref{VI:eq:1-23-1} peut se lire, bri\`evement: $F^\ast$ en 
$\gamma$ est $\rho(\gamma')^{-1}$. Si $\gamma$ appartient \`a la composante 
neutre de son centralisateur, on peut prendre $g$ dans $Z(\gamma)$. On a alors 
$\gamma=\gamma'$. Si cette condition n'est pas remplie, $\gamma$ et $\gamma'$ 
peuvent ne pas \^etre conjugu\'es. 





\subsection{Un exemple}\label{VI:1-26}

Pour $G_0=\operatorname{SL}(2)$, $q$ impair, 
$\alpha\in F_q^\times\setminus {\dF_q^\times}^2$ et $\bar\gamma$ la classe de 
conjugaison de $-1(1+N)$, avec $N^2=0$, $\bar\gamma'$ est la classe de 
conjugaison de $-1\cdot (1+\alpha N)$, de sorte que $\bar\gamma'\ne \bar\gamma$ 
si $N\ne 0$. 










\section{Dictionnaire}\label{VI:2}

Une manipulation \'el\'ementaire sur les sommes trigonom\'etriques peut souvent 
\^etre vue comme le reflet, via la formule des traces, d'un \'enonc\'e 
cohomologique. Dans ce paragraphe, j\'enum\`ere de tels \'enonc\'es, et leur 
reflet; il portent le m\^eme num\'ero, augment\'e d'une $\ast$ pour 
l'\'enonc\'e cohomologique. 





\subsection{}\label{VI:2-1}

``Une somme d'entiers alg\'ebriques est un entier alg\'ebrique'' admet pour 
analogue cohomologique le th\'eor\`eme suivant, prouv\'e dans 
\cite[XXI 5.2.2]{sga7} (cf. l'usage de \ref{VI:2-1}$^\ast$ en 
\ref{VI:1-13}). 





\begin{theorem*}[2.1*]\label{VI:2-1*}
Soient $X_0$ un sch\'ema s\'epar\'e de type fini sur $\dF_q$ et $\sF_0$ un 
$E_\lambda$-faisceau sur $X_0$. Si, pour tout $x\in |X_0|$, les valeurs 
propres de $F_x^\ast$ sur $\sF_0$ sont des entiers alg\'ebriques, alors les 
valeurs propres de $F^\ast$ sur $\h_c^i(X,\sF)$ sont des entiers alg\'ebriques. 
\end{theorem*}





\subsection{}\label{VI:2-2}

D'autres r\'esultats d'int\'egralit\'es sont prouv\'es dans 
\cite[XXI, 5.2.2 et 5.4]{sga7}: si $\dim(x_0)\leqslant n$, et que $\alpha$ est 
une valeur propre de $F^\ast$ sur $\h_c^i(X,\sF)$, on a: 
\begin{enumerate}[a)]
  \item sous les hypoth\`eses de \ref{VI:2-1}$^\ast$, et si $i>n$, alors 
    $q^{i-n}$ divise $\alpha$;
  \item si les inverses des valeurs propres de $F_x^\ast$ sont des entiers 
    alg\'ebriques, alors $\alpha^{-1}$ est entier, sauf en $p$. Plus 
    pr\'ecisement, $q^{\inf(n,i)}\alpha^{-1}$ est un entier alg\'ebrique. 
\end{enumerate}

Pour les faisceaux consid\'er\'es en \ref{VI:1-9}, les valeurs propres des 
$F_x^\ast$ sont des racines de l'unit\'e, de sorte que les hypoth\`eses du 
th\'eor\`eme et celle de b) ci-dessus sont v\'erif\'ees. 





\subsection{}\label{VI:2-3}

Soient $f:A\to B$ une application d'un ensemble fini dans un autre, et 
$\varepsilon$ un function sur $A$. On a 
\begin{equation*}\tag{2.3.1}\label{VI:eq:2-3-1}
  \sum_{a\in A} \varepsilon(a) = \sum_{b\in B} \sum_{f(a)=b} \varepsilon(a) \text{.}
\end{equation*}





\subsection*{2.3*}\label{VI:2-3*}

Soient $f:X\to Y$ un morphisme de sch\'emas s\'epar\'es de type fini sur $k$ 
alg\'ebriquement clos, et $\sF$ un $E_\lambda$-faisceau sur $X$. La suite 
spectrale de Leray de $f$ en cohomologie \`a support propre s'\'ecrit 
\begin{align*}\tag{2.3.1*}\label{VI:eq:2-3-1*}
  E_2^{p q} = \h_c^p(Y,\R^q f_! \sF) \Rightarrow \h_c^{p+q}(X,\sF) \text{.}
\end{align*}

Soient $f_0:X_0 \to Y_0$ un morphisme de sch\'emas s\'epar\'es de type fini sur 
$\dF_q$ et $\sF_0$ un $E_\lambda$-faisceau sur $X_0$. On a entre 
\eqref{VI:eq:2-3-1} et \eqref{VI:eq:2-3-1*} la compatibilit\'e suivante. 

\begin{enumerate}[a)]
  \item Le faisceau $\R^q f_! \sF$ se d\'eduit par extension des scalaires de 
    $\dF_q$ \`a $\dF$ du faisceau $\R^q f_{0!} \sF_0$ sur $Y_0$. En tant que 
    tel, il est muni d'une correspondance de Frobenius 
    $F^\ast:F^\ast\R^q f_! \sF \to \R^q f_! \sF$. Elle se d\'eduit par 
    fonctorialit\'e de $\R f_!$ de la correspondance de Frobenius de $\sF$: 
    c'est le compos\'e 
    \[\xymatrix{
      F^\ast \R^q f_! \sF \ar[r]^-\sim 
        & \R^q f_! F^\ast \sF \ar[r]^-\sim 
        & \R^q f_! \sF \text{.}
    }\]
  \item La formule des traces pour $\sF_q$ s'\'ecrit 
    \begin{align*}\tag{1}\label{VI:eq:1}
      \sum_{x\in X_0(\dF_q)} \tr(F_x^\ast,\sF_0) = \tr(F^\ast,\h^\bullet(X,\sF)) \text{;}
    \end{align*}
    celle pour $\R^q f_! \sF$ s'\'ecrit 
    \begin{align*}\tag{2$_q$}\label{VI:eq:2_q} 
      \sum_{y\in Y_0(\dF_q)} \tr(F_y^\ast,\R^q f_! \sF) = \tr(F^\ast, \h^\bullet(Y,\R^q f_! \sF)) \text{.}
    \end{align*}
\end{enumerate}

On a, pour $y\in Y(\dF)$, $(\R^q f_! \sF)_y = \h_c^q(f^{-1}(y),\sF)$, et si 
$y\in Y_0(\dF_q)$, 
\[
  \tr(F_y^\ast,\R^q f_! \sF) = \tr(F^\ast,\h_c^q(f^{-1}(y),\sF)) \text{,}
\]
d'o\`u par la formule des traces sur la fibre 
\[
  \sum_{\substack{x\in X_0(\dF_q) \\ f(x) = y}} \tr(F_x^\ast,\sF_0) = \sum (-1)^q \tr(F_y^\ast,\R^q f_! \sF) \text{.}
\]
Prenant la somme altern\'ee des identit\'es \eqref{VI:eq:2_q}, on trouve donc 
\begin{align*}\tag{2}\label{VI:eq:2}
  \sum_{y\in Y_0(\dF_q)} \sum_{\substack{x \in X_0(\dF_q) \\ f(x) = y}} \tr(F_x^\ast,\sF_0) = \tr(F^\ast,\h^\bullet(Y,\R^\bullet f_! \sF)) \text{.}
\end{align*}

L'\'egalit\'e des membres de gauche de \eqref{VI:eq:1} et \eqref{VI:eq:2} 
r\'esulte de \eqref{VI:eq:2-3-1}, celle des membres de droite de 
\eqref{VI:eq:2-3-1*} (compte tenu du fait que $F^\ast$ est un endomorphisme de 
toute la suite spectrale). 





\subsection{}\label{VI:2-4}

Soient $(A_i)_{i\in I}$ une famille finie d'ensembles finis, 
$A=\prod_{i\in I} A_i$, $\varepsilon_i$ une fonction sur $A_i$, et 
$\varepsilon$ la fonction $a\mapsto \prod \varepsilon_i(a_i)$ sur $A$. On a 
\begin{align*}\tag{2.4.1}\label{VI:eq:2-4-1}
  \sum_{a\in A} \varepsilon(a) = \prod_{i\in I} \sum_{a_i\in A_i} \varepsilon_i(a_i) \text{.}
\end{align*}





\subsection*{2.4*}\label{VI:2-4*}

Soient $(X_i)_{i\in I}$ une famille finie de sch\'emas s\'epar\'es de type fini 
sur $k$ alg\'ebriquement clos, $X=\prod_{i\in I} X_i$, $\sF_i$ un 
$E_\lambda$-faisceau sur $X_i$ et $\sF$ le produit tensoriel externe 
$\bigboxtimes_{i\in I}\sF_i = \bigotimes_{i\in I} \operatorname{pr}_i^\ast(\sF_i)$ 
des $\sF_i$. On a la formule de K\"unneth 
\begin{align*}\tag{2.4.1*}\label{VI:eq:2-4-1*}
  \h_c^\bullet(X,\sF) = \bigotimes_{i\in I} \h_c^\bullet(X_i,\sF_i) \text{.}
\end{align*}

Si on veut un isomorphisme \eqref{VI:eq:2-4-1*} canonique, et ind\'ependant du 
choix d'un ordre total sur $I$, il faut prendre au membre droite le produit 
tensoriel gradu\'e au sens de la r\`egle de Koszul 
(\hyperref[IV]{Cycle}, \ref{V:1-3}). 





\subsection{}\label{VI:2-5}

Soient $A$ un ensemble fini, $B$ une partie de $A$ et $\varepsilon$ une 
fonction sur $A$. On a 
\begin{align*}\tag{2.5.1}\label{VI:2-5-1}
  \sum_{a\in A} \varepsilon(a) = \sum_{a\in A\setminus B}\varepsilon(a) + \sum_{a\in B} \varepsilon(a) \text{.}
\end{align*}





\subsection*{2.5*}\label{VI:2-5*}

Soient $x$ un sc\'ema s\'epar\'e de type fini sur $k$ alg\'ebriquement clos, 
$Y$ un sous-sch\'ema ferm\'e et $\sF$ un $E_\lambda$-faisceaux sur $X$. On a 
une suite exacte longue de cohomologie 
\begin{align*}\tag{2.5.1*}\label{VI:eq:2-5-1*}
\xymatrix{
  \cdots \ar[r]^-\partial 
    & \h_c^i(X\setminus Y,\sF) \ar[r] 
    & \h_c^i(X,\sF) \ar[r] 
    & \h_c^i(Y,\sF) \ar[r]^-\partial 
    & \cdots 
}
\end{align*}

Plus g\'en\'eralement, si on part d'une filtration finie de $X$ par des parties 
ferm\'ees $X_p$ ($p\in \dZ$; on suppose que $X_p\supset X_{p+1}$, que $X_p=X$ 
pour $p$ assez petit, et que $X_p=\varnothing$ for $p$ assez grand), on a une 
suite spectrale 
\begin{align*}\tag{2.5.2*}\label{VI:eq:2-5-2*}
  E_1^{p q} = \h_c^{p+1}(X_p\setminus X_{p+1},\sF) \Rightarrow \h_c^{p+q}(X,\sF) \text{.}
\end{align*}





\subsection{}\label{VI:2-6}

Soient $A$ un ensemble fini, $(A_i)_{i\in I}$ un recouvrement fini de $A$ et 
$\varepsilon$ une fonction sur $A$. Pour $J\subset I$, soit $A_J$ 
l'intersection des $A_j$ ($j\in J$). On a 
\begin{align*}\tag{2.6.1}\label{VI:eq:2-6-1}
  \sum_{J\subset I} (-1)^{|J|} \sum_{a\in A_J} \varepsilon(a) = 0 \text{.}
\end{align*}





\subsection*{2.6*}\label{VI:2-6*}

Soient $X$ un sch\'ema s\'epar\'e de type fini sur $k$ alg\'ebriquement clos, 
$(X_i)_{i\in I}$ un recouvrement ferm\'e (resp. ouvert) fini de $X$ par des 
sous-sch\'emas, et $\sF$ un $E_\lambda$-faisceau sur $X$. Pour $J\subset I$, 
soit $X_J$ l'intersection des $X_j$ ($j\in J$). On a des suites spectrales 
respectives (de Leray) 
\begin{align*}\tag{2.6.1*}\label{VI:eq:2-6-1*}
  E_1^{p q} &= \bigoplus_{|J|=p+1>0} \h_c^q(X_J,\sF) \otimes_\dZ \textstyle\bigwedge^{|J|} \dZ^J \Rightarrow \h_c^{p+q}(X,\sF) \\ 
  \tag{2.6.2*}\label{VI:eq:2-6-2*}
  E_1^{p q} &= \bigoplus_{|J|=1-p>0} \h_c^q(X_J,\sF) \otimes_\dZ \textstyle\bigwedge^{|J|} \dZ^J \Rightarrow \h_c^{p+q}(X,\sF)
\end{align*}
Si $J=\varnothing$ n'\'etait pas exclus, on aurait de m\^eme des suites 
spectrales convergeant vers $0$. Les facteurs $\bigwedge^{|J|} \dZ^J$ sont l\`a 
pour nous dispenser de choisir un ordre total sur $I$. 





\subsection{}\label{VI:2-7}

Soient $A$ un groupe commutatif fini, et $\chi:A \to E_\lambda^\times$ un 
caract\`ere non trivial. On a 
\begin{align*}\tag{2.7.1}\label{VI:eq:2-7-1}
  \sum_{a\in A} \chi(a) = 0 \text{.}
\end{align*}
On prouvera l'analogue cohomologique de \eqref{VI:eq:2-7-1} en en transposant 
la preuve suivante: si $x\in A$, $a\mapsto x a$ est une permutation de $A$, et 
\begin{align*}
  \sum_{a\in A}\chi(a) = \sum_{a\in A} \chi(x a) &= \chi(x) \sum_{a\in A} \chi(a) && \text{, d'o\`u} \\
  (\chi(x)-1)\sum_{a\in A} &= 0 \text{,}
\end{align*}
et il existe par hypoth\`ese $x$ tel que $\chi(x)-1\ne 0$. 





\begin{theorem*}[2.7*]\label{VI:2-7*}
Soient $G_0$ un groupe alg\'ebrique commutatif connexe sur $\dF_q$ et 
$\chi:G_0(\dF_q) \to E_\lambda^\times$ un caract\`ere non triviale. On a 
\begin{align*}\tag{2.7.1*}\label{VI:eq:2-7-1*}
  \h_c^\bullet(G,\sF(\chi)) = 0 \text{.}
\end{align*}
\end{theorem*}

Pour $x$ un point rationnel de $G$, notons $t_x$ la translation $t_x(g) = x g$ 
de $G$. La formule $\fL t_x = t_{\fL(x)} \fL$ exprime que $(t_x,t_{\fL(x)})$ 
est un automorphisme du diagrame $G\xrightarrow\fL G$ ($G$ muni du torseur de 
Lang). Soit $\rho(g)$ l'automorphisme de $(G,\sF(\chi))$ qui s'en d\'eduit. 
Pour $g\in G_0(\dF_q)$, i.e. pour $\fL(g) = e$, c'est l'identit\'e sur $G$, et 
la multiplication par $\chi(g)^{-1}$ sur $\sF(\chi)$. 

Notons $\rho_H(g)$ l'automorphisme de $\h_c^\bullet(G,\sF(\chi))$ d\'eduit de 
$\rho(g)$. Un argument d'homotopie (Lemme \ref{VI:2-8} ci-dessous) montre que 
$\rho_H(g) = \rho_H(e)$, donc est l'identit\'e. D'autre part, pour 
$g\in G_0(\dF_q)$, $\rho_H(g)$ est la multiplication par 
$(\chi^{-1}(g):\text{ sur }\h^\bullet(G,\sF(\chi))$, la multiplication par 
$(\chi^{-1}(g)-1)$ est nulle. Prenant $g$ tel que $\chi(g)\ne 1$, on obtient 
\ref{VI:2-7*}. 





\begin{lemma_}\label{VI:2-8}
Soient $X$ et $Y$ deux sch\'emas sur $k$ alg\'ebriquement clos, avec $X$ 
s\'epar\'e de type fini et $Y$ connexe. Soient $\sF$ un faisceau sur $X$ et 
$(\rho,\varepsilon)$ une famille d'endomorphismes de $(X,\sF)$ param\'etr\'ee 
par $Y$: 
\begin{align*} 
  \rho:Y\times_k X &\to Y\times_k X && \text{est un $Y$-morphisme, et} \\
  \varepsilon:\rho^\ast \operatorname{pr}_2^\ast \sF &\to \operatorname{pr}_2^\ast\sF && \text{un morphisme de faisceaux.} 
\end{align*}
On suppose que $\rho$ est propre. Pour $y\in Y(k)$, soit $\rho_H(y)^\ast$ 
l'endomorphisme de $\h_c^\bullet(X,\sF)$ induit par $\rho_y:X\to X$ et 
$\varepsilon_y:\rho_y^\ast\sF\to \sF$. Alors, $\rho_H(y)^\ast$ est 
ind\'ependant de $y$. 
\end{lemma_}

En effet, $\R^p \operatorname{pr}_{1!} \operatorname{pr}_2^\ast\sF$ est le 
faisceau constant sur $Y$ de valeur $\h^p(X,\sF)$, et $\rho_H(y)^\ast$ est la 
fibre en $y$ de l'endomorphisme 
\[\xymatrix{
  \R^p \operatorname{pr}_{1!} \operatorname{pr}_2^\ast \sF \ar[r]^-{\rho^\ast} 
    & \R^p \operatorname{pr}_{1!} \rho^\ast \operatorname{pr}_2^\ast \sF \ar[r]^-\varepsilon 
    & \R^p \operatorname{pr}_{1!} \operatorname{pr}_2^\ast \sF 
}\]
de ce faisceau. 





\subsection{Remarque}\label{VI:2-9}

Soient $G_0$ un groupe alg\'ebrique connexe sur $\dF_q$ et 
$\rho:G_0(\dF_q) \to \operatorname{GL}(V)$ un repr\'esentation lin\'eaire 
telle que $V^{G_0(\dF_q)}=0$. On a encore $\h_c^\bullet(G,\rho(L))=0$. 










\section{Sommes \`a une variable}\label{VI:3}





\subsection{}\label{VI:3-1}

Weil est le premier \`a avoir appliqu\'e des m\'ethodes ``cohomologiques'' \`a 
l'\'etude des sommes trigonom\'etriques \`a une variable; puisqu'il avait 
prouv\'e l'analogue de l'hypoth\`ese de Riemann pour les fonctions $L$ d'Artin 
sur les corps de fonctions, ces m\'ethodes lui fournissaient d'excellentes 
estimations (en $O(\text{racine carr\'ee du nombre de termes})$), et le 
comportement de ces sommes par ``extension du corps de base.'' 

L'essentiel de ce paragraphe est un expos\'e, dans un langage cohomologique, de 
ses r\'esultes. 





\subsection{}\label{VI:3-2}

Les m\'ethodes cohomologiques am\`enent ici \`a calculer des groupes 
$\h_c^i(X,\sF)$, pour $\sF_0$ un $E_\lambda$-faisceau sur un courbe $X-0$ sur 
$\dF_q$. Ces groupes sont nuls pour $i\ne 0,1,2$ et pour 
$i=0,2$, ils ont une interpr\'etation simple (\ref{VI:1-8} a,b,c). De plus, la 
caract\'eristique d'Euler-Poincar\'e 
$\chi_c(\sF) = \sum (-1)^i \dim \h_c^i(X,\sF)$ (par ailleurs \'egale \`a 
$\chi(\sF) = \sum (-1)^i \dim \h^i(X,\sF)$ peut \^etre calcul\'ee en terme de 
$x$, de rang de $\sF$ aux points g\'en\'eriques de $X$, et des propri\'et\'es 
de ramification de $\sF$. Le r\'esultat essentiel est le suivant. 

Soient $\bar X$ une courbe projective lisse et connexe, de genre $g$, sur un 
corps alg\'ebriquement clos $k$, $X$ un ouvert dense de $\bar X$, 
l'compl\'ement d'un ensemble fini $S$ de points et $\sF$ un 
$E_\lambda$-faisceau lisse sur $X$. Soient $\chi(X)=2-2 g-\# S$ la 
caract\'eristique d'Euler-Poincar\'e de $X$, et $\operatorname{rg}(\sF)$ le 
rang de $\sF$. Pour chaque point $s\in S$, on d\'efinit un entier 
$\swan_s(\sF)$, le conducteur de Swan, mesurant la ramification sauvage de 
$\sF$, et 
\begin{align*}\tag{3.2.1}\label{VI:eq:3-2-1}
  \chi_c(\sF) = \operatorname{rg}(\sF) \cdot \chi(X) - \sum_{s\in S} \swan_s(\sF) 
\end{align*}
\cite{ra65}. 





\subsection{}\label{VI:3-3}

Soient $\bar X$, $j:X\hookrightarrow\bar X$ et $\sF$ comme en \ref{VI:3-2}. Le 
th\'eor\`eme de dualit\'e de Poincar\'e admet la forme tr\`es maniable: 
$\h^i(\bar X,j_\ast \sF)$ et $\h^{2-i}(\bar X,j_\ast(\sF^\vee(1)))$ sont 
duaux l'un de l'autre (\hyperlink{V}{Dualit\'e}, \ref{V:1-3}). 





\subsection{}\label{VI:3-4}

Faisons $k=\dF$, et supposons que $X$, $\bar X$, et $\sF$ proviennent de 
$X_0$, $\bar X_0$ et $\sF_0$ sur $\dF_q$. Si $\sF_0$ devient trivial sur un 
rev\^etement fini de $X_0$, Weil a prouv\'e que les conjugu\'es complexes des 
valeurs propres de $F^\ast$ sur $\h^i(\bar X,j_\ast\sF)$ sont de valeur 
absolue $q^{i/2}$. De tels faisceaux $\sF_0$ correspondent aux fonctions $L$ 
d'Artin sur le corps de fonctions de $\bar X_0$. 





\subsection*{3.5}\label{VI:3-5_} % originally one of two 3.5s

Sous les m\^emes hypoth\`eses, ou plus g\'en\'eralement si $E_\lambda$ est le 
compl\'et\'e en une place $\lambda$ d'un corps de nombres $E$ et que $\sF_0$ 
appartient \`a un syst\`eme compatible infini de repr\'esentations $v$-adiques 
($v$ place de $E$), on peut calculer 
\[
  \prod_i \det(-F^\ast,\h^i(X,\sF))^{(-1)^i} 
\]
\`a partir d'informations \emph{locales} sur $\sF_0$. Ceci est explique dans 
\cite{de73}. Pour $\sF_0$ trivial sur un rev\^etement fini de $X_0$, c'est 
l'expression de la constante de l'\'equation fonctionnelle d'une fonction $L$ 
d'Artin comme produit de constantes locales. 





\subsection{Exemple}\label{VI:3-5}

Soit $\psi$ le caract\`ere 
$\exp\left(\frac{2\pi i}{p} \tr_{\dF_q/\dF_p}\right)$ de $\dF_q$; on a 
$\psi(a^p-a)=0$. Soient $X_0$ un courbe projective lisse absolument 
irr\'eductible de genre $g$ sur $\dF_q$, et $f$ une fonction rationnelle sur 
$X_0$, i.e. un morphisme $f:X_0 \to \dP^1$, non identiquement \'egale \`a 
$\infty$. On s'int\'erresse \`a la somme 
\[
  S_f' = \sum_{\substack{x\in X_0(\dF_q) \\ f(x)\ne \infty}} \psi(f(x)) 
  \text{.}
\]
Cette somme est nulle pour $f$ de la forme $g^p-g$. Ceci sugg\`ere de 
modifier la somme $S_f'$ comme suite: 
\begin{enumerate}[\indent a)]
  \item pour tout point ferm\'e $x$ de $X_0$, on pose $v_x(f) = $ ordre du 
    pole de $f$ en $x$ si $f(x)=\infty$, $v_x(f)=0$ sinon, et 
    $v_x^\ast(f) = \inf v_x(f+g^p-g)$ (borne inf\'erieure sur $g$). 
  \item Si $v_x^\ast(f)=0$, que $x\in X_0(\dF_q)$, et que $f+g^p-g$ est 
    r\'egulier en $x$, on pose $\psi(f(x)) = \psi((f+g^p-g)(x))$. On pose enfin 
    \begin{equation*}\tag{3.5.1}\label{VI:eq:3-5-1}
      S_f = \sum'_{x\in X_0(\dF_q)} \psi(f(x)) \text{,}
    \end{equation*}
    o\`u $(-)'$ indique que la somme est \'etendue aux $x$ tels que 
    $v_x^\ast(f)=0$. 
\end{enumerate}

On a $S_f=S_{f+g^p-g}$. On se propose de v\'erifier que si $f$ n'est pas la 
forme $g^p-g+c^{t e}$, alors 
\begin{equation*}\tag{3.5.2}\label{VI:eq:3-5-2}
  |S_f|\leqslant \left(2 g-2+\sum_{v_x^\ast(f)\ne 0} [k(x):\dF_q] (1+v_x^\ast(f))\right) q^{1/2} \text{.}
\end{equation*}

On commence par passer du complexe au $\ell$-adique, en remplaçant $\psi$ par 
un caract\`ere de la forme $\psi_0\circ \tr_{\dF_q/\dF_p}$, avec 
$\psi_0:\dF_p\hookrightarrow E_\lambda^\times$. Pour $j:U_0\hookrightarrow X_0$ 
l'inclusion de l'ouvert o\`u $f\ne\infty$, on preuve alors que 
\begin{equation*}\tag{3.5.3}\label{VI:eq:3-5-3}
  S_f = \sum (-1)^i \tr(F^\ast,\h^i(X,j_\ast \sF(\psi f))) \text{.} 
\end{equation*}

La formule des traes ram\`ene cet \'enonc\'e aux suivants (pour 
$x\in X_0(\dF_q)$): 
\begin{enumerate}[\indent a)]
  \item Si $v_x^\ast(f)=0$, alors $F_x^\ast$ sur $j_\ast \sF(\psi f)$ est 
    $\psi(f(x))$; 
  \item Si $v_x^\ast(f)\ne 0$, alors $j_\ast \sF(\psi f)$ se ramifie en $x$: 
    $(j_\ast \sF(\psi f))_{\bar x}=0$. 
\end{enumerate}

La formule a) r\'esulte de \eqref{VI:eq:1-7-6} si $v_x(f)=0$, et on se ram\`ene 
\`a ce cas en remplaçant $f$ par $f+g^p-g$: \`a isomorphisme pr\`es, ceci ne 
change pas $\sF(\psi f)_0$ sur l'ouvert o\`u $f$ et $g$ sont r\'eguliers 
(\ref{VI:1-3}). 

La formule b) r\'esulte de 
\begin{equation*}\tag{3.5.4}\label{VI:eq:3-5-4}
  \swan_x \sF(f,\psi) = v_x^\ast(f) \text{.} 
\end{equation*}
Apr\`es r\'eduction au cas o\`u $v_x(f)=v_x^\ast(f)$ (d'o\`u $p\nmid v_x(f)$) 
et extension des scalaires \`a $\dF$, cette formule est dans 
\cite[4.4]{se61}. 

Enfin, on v\'erifie que le faisceau $\sF(\psi f)$ sur $U$ est non constant si 
$f$ n'est pas de la forme $g^p-g+c^{t e}$: puisque $\psi_0$ est injectif, la 
trivialit\'e de $\sF(\psi f)=\sF(\psi_0 f)$ (\ref{VI:1-8}.ii) \'equivaut \`a 
celle du $\dF_q$-torseur d'\'equation serait l'image r\'eciproque d'un 
$\dF_q$-torseur sur $\spec(\dF_q)$, d'\'equation $T^p-T-\lambda=0$, le 
$\dF_p$-torseur d'\'equation $T^p-T-(f-\lambda)=0$ sur $U_0$ serait trivial 
et $f$ serait donc de la forme $g^p-g+\lambda$. Les 
$\h^i(\bar X,j_\ast \sF(\psi f))$ sont donc nuls pour $i\ne 1$, la formule 
\eqref{VI:eq:3-5-3} se r\'eduit \`a 
\begin{equation*}\tag{3.5.5}\label{VI:eq:3-5-5}
  S_f = -\tr(F^\ast,\h^1(\bar X,j_\ast \sF(\psi f))) 
\end{equation*}
et, d'apr\`es \eqref{VI:eq:3-2-1} et \eqref{VI:eq:3-5-4}, de $\h^1$ est de 
dimension $2 g-2+\sum_{v_x^\ast(f)>0} [k(x):\dF_q](1+v_x^\ast(f))$ et $-S_f$ 
est somme de ce nombre de valeurs propres de $F^\ast$, chacune de valeurs 
absolues complexes $q^{1/2}$. 





\subsection{}\label{VI:3-6}

Supposons que, pour un automorphisme $\sigma$ de $X_0$, on ait 
\[
  f(\sigma x) = - f(x) \text{.} 
\]
La somme $S_f$ est alors r\'eelle. Si $\sigma$ est involutif, la dualit\'e de 
Poincar\'e permet de dire un peu plus: si on pose 
$\sF_0=j_\ast \sF(\psi f)$, $\sG_0 = j_\ast \sF(\psi(-f))$, on a 
\begin{enumerate}[\indent a)]
  \item $\sF_0$ et $\sG_0$ sont en dualit\'e, 
  \item $\sigma^\ast \sF_0 \iso \sG_0$ et $\sigma^\ast \sG_0 \iso \sF_0$, pour 
    des isomorphismes naturels tel que le compos\'e 
    $\sF_0=(\sigma^2)^\ast \sF_0 \iso \sigma^\ast \sG_0 \iso \sF_0$ soit 
    l'identit\'e. 
  \item L'accouplement $\sF_0\otimes \sG_0 \to E_\lambda$ v\'erifie 
    $\sigma^\ast(f\cdot g) = \sigma^\ast(f) \cdot \sigma^\ast (g)$. 
\end{enumerate} 

Passant \`a cohomologie, on trouve que $\h^1(X,\sF)$ et $\h^1(X,\sF)$ sont en 
dualit\'e parfaite \`a valeur dans $E_\lambda(1)$, et que la forme bilin\'eaire 
$\alpha\cdot \sigma^\ast\beta$ sur $\h^1(X,\sF)$ est \emph{altern\'ee}: 
$\sigma^\ast$ agit trivialement sur $E_\lambda(1)$, et 
\[
  \alpha\cdot \sigma^\ast \beta = \sigma^\ast(\alpha\cdot \sigma^\ast \beta) = \sigma^\ast \alpha\cdot \beta = -\beta\cdot \sigma^\ast \alpha \text{.} 
\]
On en conclut que $\h^1(X,\sF)$ est de dimension paire (on v\'erifie d'ailleurs 
facilement que chaque $v_x^\ast(f)$ non nul est impair) et que les valeurs 
propres de $F^\ast$ sur $\h^1(X,\sF)$ sont group\'ees en pairs $\alpha$ et 
$q/\alpha$. 





\subsection{Exemple}\label{VI:3-7}

Ceci s'applique aux sommes de Kloosterman 
$\sum_{x\in \dF_q^\times} \psi\left(x+\frac a x\right)$ (faire 
$X_0=\dP^1$, $f=x+\frac a x$, $\sigma x=-x$; les p\^oles de $f$ sont $0$ et 
$\infty$, et en chacun d'eux $v_x^\ast(f)=1$; le $\h^1$ est de dimension 
$-2+2+2=2$). On a donc (pour $a\ne 0$) 
\[
  \sum_{\substack{x y=a \\ x,y\in \dF_{q^n}}} \exp\left(\frac{2\pi i}{p} \tr_{\dF_{q^n}/\dF_p}(x+y)\right) = (-\alpha^n+\alpha^{-n}) \qquad \text{, }\alpha\bar \alpha = q \text{.} 
\]
Ce r\'esultat est d\^u \`a L.\ Carlitz \cite{ca69}. 

Voici maintenant une application d'une m\'ethode de Lang-Weil. 





\begin{proposition_}\label{VI:3-8}
Soit $P\in \dF_q[X_1,\dots,X_n]$ un polyn\^ome \`a $n$ variable, de degr\'e 
$d$, dont on suppose qu'il n'est pas de la forme $Q^p-Q+C^{te}$. Posant encore 
$\psi(x) = \exp\left(\frac{2\pi i}{p} \tr_{\dF_q/\dF_p}(x)\right)$ on a 
\[
  \left|\sum_{x_i\in \dF_q} \psi P(x_1,\dots,x_n)\right| \leqslant (d+1) q^{n-\frac 1 2} \text{.}
\]
\end{proposition_}

On a pour le membre de gauche l'estimation triviale $q^n$. Il suffit donc de 
prouver \ref{VI:3-8} lorsque $d-1<q^{1/2}$; supposons seulement que 
$d<q+1$, et soit $P_d$ la partie homog\`ene de degr\'e $d$ de $P$. Rappelons le 





\begin{lemma_}\label{VI:3-9}
Une hypersurface de degr\'e $d$ dans $\dP^r$ ($r\geqslant i$) ne peut passer 
par tout les points rationnels sur $\dF_q$ qui se $d\geqslant q+1$. 
\end{lemma_}

On proc\`ede par r\'ecurrence: s'il existe un hyperplan rationnel non 
enti\`erement contenu dans l'hypersurface (tel n'est pas le cas pour $r=1$), on 
applique l'hypoth\`ese de r\'ecurrence \`a la trace de l'hypersurface sur cet 
hyperplan. Sinon, le degr\'e $d$ est $\geqslant$ le nombre d'hyperplans 
rationnel, $\geqslant q+1$. 

Distinguons maintenant deux cas. 


\paragraph{Cas 1: $p\nmid d$.}
Appliquant le lemma \`a $P_d$, on voit que, quitte \`a faire un changement 
lin\'eaire de variables, on peut supposer que $P_d(1,0,\ldots,0)\ne 0$, i.e. 
que le coefficient de $X_1^d$ dans $P$ est non nul. Un polyn\^ome \`a $1$ 
variable 
\[
  S(X) = \sum_{i=0}^d a_i x^i \text{,} 
\]
avec $a_d\ne 0$ ($p\nmid d$) n'est jamais de la forme $Q^p-Q+C^{t e}$, et 
l'estimation \eqref{VI:eq:3-5-2} se r\'eduit \`a 
\[
  \left| \sum \psi(S(x)) \right| \leqslant (d-1) q^{1/2} \text{.} 
\]
Appliquant cette estimation aux sommes partielles obtenues en ne faisant varier 
que $x_1$, on trouve 
\[
  \left| \sum \psi P(x_1,\dots,x_n) \right| = \left| \sum_{x_2,\dots,x_n} \sum_{x_1} \psi P(x_1,\dots,x_n)\right| \leqslant q^{n-1} (d-1) q^{1/2} 
\]
comme promis. 


\paragraph{Cas 2: $p\mid d$ ($d>0$).}
Si $P_d$ est une puissance $p$-i\`eme, remplacer $P$ par $P-(P_d-P_d^{1/p})$ ne 
change pas la somme consid\'er\'ee, et abaisse le degr\'e: on se d\'ebarasse de 
ce cas en proc\'edant par r\'ecurrence sur $d$. Sinon, la diff\'erentielle de 
$P_d$ n'est pas identiquement nulle; appliquant \ref{VI:3-9}, on peut supposer, 
quitte \`a faire un changement lin\'eaire de variables, qu'elle n'est pas nulle 
au point $(1,0,\dots,0)$. Si on \'ecrit 
\[
  P_d = \sum_{i=0}^d X_1^{d-i} S_i(X_2,\dots,X_n) \text{,} 
\]
cela signifie que la forme lin\'eaire $S_1$ n'est pas identiquement nulle. 

La forme $S_0$ est une constante, et remplaçant $P$ par 
$P-(S_0 X_1^d -S_0^{1/p} X_1^{d/p})$, on peut supposer qu'elle est nulle. Soit 
enfin $-\lambda$ le coefficient de $X_1^{d-1}$ dans $P$. Pour $x_2,\dots,x_n$ 
fixes, on a 
\[
  P(X_1,x_2,\dots,x_n) = (S_1(x_2,\dots,x_n)-\lambda) X_1^{d-1} + \text{termes de plus bas degr\'e en $X_1$} 
\]
et si $S_1(x_2,\dots,x_n)\ne \lambda$, on a donc 
\[
  \left| \sum_{x_1} \psi P(x_1,\dots,x_n)\right| \leqslant (d-2) q^{1/2} \text{.} 
\]
Au total, 
\begin{align*}
  \left|\sum \psi P(x_1,\dots,x_n)\right| 
    &\leqslant \left|\sum_{S=\lambda} \psi P(x_1,\dots,x_n)\right| + \sum_{S(X_2,\dots,X_n)\ne\lambda} \left| \sum_{x_1} \psi P(x_1,\dots,x_n)\right| \\
    &\leqslant q^{n-1} + (q^{n-1}-q^{n-2})(d-2) q^{1/2} \\
    &< (d-2) q^{n-1/2} + q^{n-1} \\
    &< (d-1) q^{n-1/2} \text{.}
\end{align*}

Un autre r\'esultat de cette nature est donn\'e par R.A.\ Smith 
\cite{sm70}. 










\section{Sommes de Gauss et sommes de Jacob}\label{VI:4}





\subsection{}\label{VI:4-1}

Soient $k$ un corps fini de caract\'eristique $p$, $\chi$ un caract\`ere de 
$k^\times$, et $\psi$ un caract\`ere non trivial du groupe additif de $k$. Nous 
prendrons pour d\'efinition des sommes de Gauss: 
\begin{equation*}\tag{4.1.1}\label{VI:eq:4-1-1}
  \tau(\chi,\psi) = -\sum_{x\in k^\times} \psi(x) \chi^{-1}(x) 
\end{equation*}
(noter le signe). Classiquement, $\chi$ et $\psi$ sont \`a valeurs complexes. 
Nous les prendrons \`a valeurs dans $E_\lambda^\times$ (cf. la preuve de 
\ref{VI:1-15}). Regardons $-\tau(\chi,\psi)$ comme une somme sur les points 
rationnels du sch\'ema $\dG_m$ sur $k$. D'apr\`es le paragraphe \ref{VI:1}, 
on a $-\tau(\chi,\psi) = \tr(F^\ast,\h_c^\ast(\dG_m,\sF(\psi \chi^{-1}))$. De 
plus 





\begin{proposition_}\label{VI:4-2}
La cohomologie de $\dG_m$ \`a coefficient dans $\sF(\psi\chi^{-1})$ v\'erifie 
\begin{enumerate}[(i)]
  \item $\h_c^i=0$ pour $i\ne 1$, et $\dim \h_c^1=1$. 
  \item $F^\ast$, agissant sur $\h_c^1$, est la multiplication par 
    $\tau(\chi,\psi)$. 
  \item Si $\chi$ est non trivial, on a $\h_c^\bullet \iso \h^\bullet$. 
\end{enumerate}
\end{proposition_}
\begin{proof}[Preuve]
Pour tout $n$ premier \`a $p$, notons $K_n$ le $\dmu_n$-torseur sur $\dG_m$ 
d\'efini par la suite exacte de Kummer 
$0 \to \dmu_n \to \dG_m \xrightarrow{x^n} \dG_m \to 0$. Si $k$ a $q$ 
\'el\'ements, on a sur $k$: $\dmu_{q-1}=k^\times$, et le torseur de Lang sur 
$\dG_m/k$ est $K_{q-1}$. D\`es lors, $\sF(\chi^{-1})=\chi(K_{q-1})$. 
L'assertion \ref{VI:4-2}(ii) r\'esulte de (i,iii), eux-m\^emes contenus dans 
l'\'enonc\'e g\'eom\'etrique suivant, o\`u $\sF(\psi)$ d\'esigne le faisceau 
sur $\dG_m/\dF$ d\'eduit par extension des scalaires de $k$ \`a $\dF$ de 
$\sF(\psi)$ sur $\dG_m/k$ (\ref{VI:1-8}.b). 
\end{proof}





\begin{proposition_}\label{VI:4-3}
Soit $\chi:\dmu_n \to E_\lambda^\times$. La cohomologie de $\dG_m$ \`a 
coefficient dans $\sF(\psi)\otimes \chi(K_n)$ v\'erifie 
\begin{enumerate}[\indent (i)]
  \item $\h_c^i=0$ pour $i\ne 1$, et $\dim \h_c^1{} = 1$. 
  \item Si $\chi$ est non triviale, on a $\h_c^\bullet{}\iso \h^\bullet{}$. 
\end{enumerate}
\end{proposition_}
\begin{proof}[Preuve]
Le faisceau $\sF(\psi)$ est la restriction \`a $\dG_m$ d'un faisceau localement 
constant sur $\dG_a$, sauvagement ramifi\'e \`a l'infini, de conducteur de 
Swan $1$. Le faisceau $\chi(K_n)$ est constant si $\chi=1$; si $\chi\ne 1$, il 
est ramifi\'e en $0$ et $\infty$, mod\'er\'ement. 

Le faisceau $\sF(\psi)\otimes\chi(K_n)$ est donc ramifi\'e \`a l'$\infty$; 
appliquant \ref{VI:1-18}(b,c), on trouve que 
$\h_c^i(\dG_m,\sF(\psi)\otimes \chi(K_n))=0$ pour $i\ne 1$. Si $\chi\ne 1$, il 
est ramifi\'e en $0$ et $\infty$ et (ii) r\'esulte de \ref{VI:1-19}.a. 

Les conducteurs de Swan sont $0$ en $0$ et $1$ en $\infty$. D'apr\`es 
\eqref{VI:eq:3-2-1}, la caract\'eristique d'Euler-Poincar\'e est donc $-1$ et 
ceci ach\`eve la d\'emonstration. 
\end{proof}





\subsection{Remarque}\label{VI:4-4}

Si $\chi$ est non trivial, \ref{VI:4-2}(iii) et la dualit\'e de Poincar\'e 
montrent que $\h_c^1(\dG_m,\sF(\psi\chi^{-1}))$ et 
$\h_c^1(\dG_m,\sF(\psi^{-1}\chi))$ sont en dualit\'e (dualit\'e \`a valeurs 
dans $E_\lambda(-1)$). On a donc 
$\tau(\chi,\psi)\cdot \tau(\chi^{-1},\psi^{-1}) = q$, i.e. 
$|\tau(\chi,\psi)|=1$. 





\subsection{}\label{VI:4-5}

Si $k$ est une extension de degr\'e $N$ de $\dF_q$, on peut aussi regarder 
\eqref{VI:eq:4-1-1} comme une somme \`a $N$ variables sur $\dF_q$. Soit plus 
g\'en\'eralement $k$ une alg\`ebre \'etale sur $\dF_q$, de degr\'e $N$ sur 
$\dF_q$. C'est un produits de corps $k_i$, de degr\'e $N_i$, et on pose 
\begin{equation*}\tag{4.5.1}\label{VI:eq:4-5-1}
  \varepsilon(k) = (-1)^{\sum (N_i+1)} \text{.} 
\end{equation*}
C'est la signature de la permutation de $S=\hom_{\dF_q}(k,\dF)$ induite par la 
substitution de Frobenius $\varphi\in \gal(\dF/\dF_q)$. 

Soient $\psi:\dF_q\to E_\lambda^\times$ un caract\`ere non trivial, et 
$\chi:k^\times \to E_\lambda^\times$. On pose 
\begin{equation*}\tag{4.5.2}\label{VI:eq:4-5-2}
  \tau_{\dF_q}(\chi,\psi) = (-1)^N \sum_{x\in k^\times} \psi \tr_{k/\dF_q}(x) \cdot \chi^{-1}(x) \text{.} 
\end{equation*}

Si $\chi$ a pour coordonn\'ees les $\chi_i:k_i^\times \to E_\lambda^\times$, on 
a l'identit\'e triviale 
\begin{equation*}\tag{4.5.3}\label{VI:eq:4-5-3}
  \tau_{\dF_q}(\chi,\psi) = \varepsilon(k) \prod \tau(\chi_i,\psi\circ \tr_{k_i/\dF_q}) \text{.}
\end{equation*}

Apr\`es quelques pr\'eliminaires, nous donnerons en \ref{VI:4-10} une 
interpr\'etation cohomologique des sommes \eqref{VI:eq:4-5-2}. En 
\ref{VI:4-12}, nous interpreterons l'identit\'e de Hasse-Davenport comme une 
forme tordue (cf. \ref{VI:1-12}) du cas particulier suivant de 
\eqref{VI:eq:4-5-3}: pour $k=\dF_q^N$, et $\chi$ un caract\`erre de 
$\dF_q^\times$, on a 
$\tau_{\dF_q}(\chi\circ N_{k/\dF_q},\psi)=\tau(\chi,\psi)^N$. 





\subsection{}\label{VI:4-6}

Rappelons que pour $M$ un module projectif de type fini sur un anneau $A$, le 
foncteur $\spec(B)\mapsto M\otimes_A B$, des sch\'emas affines sur $\spec(A)$ 
dans $\mathsf{Ens}$ est repr\'esent\'e par le sch\'ema affine $\dV(M^\vee)$ 
(notations des EGA), le spectre de $\operatorname{Sym}_A^\bullet(M^\vee)$. Si 
$M=A^n$, c'est l'espace affine type de dimension $n$. 

Supposons que $M$ soit une $A$-alg\`ebre \`a unit\'e, et posons 
$V=\dV(M^\vee)$. Par d\'efinition, pour toute extension $B$ de $A$, l'ensemble 
$V(B)$ de points de $V$ \`a coordonn\'ees dans $B$ est $M\otimes_A B$. Le 
foncteur $B\mapsto M\otimes_A B$ \'etant \`a valeurs dans les anneaux \`a 
unit\'e, $V$ est un sch\'ema en anneaux \`a unit\'es sur $\spec(A)$. Les 
morphismes norme et trace: $M\otimes_A B\to B$ sont fonctoriels en $B$; ils 
correspondent donc \`a des morphismes de sch\'ema $N$ et $T$ de $V$ dans 
$\dG_a$. Nous aurons \`a consid\'erer les sch\'emas d\'eduits de $V$ suivant: 
\begin{enumerate}[a)]
  \item $V^\ast$ est l'ouvert des \'el\'ements inversibles de $V$ (un sch\'ema 
    en groupes pour $\cdot$);
  \item $W$ est l'hyperplan d'\'equation $T=0$ et $W^\ast=W\cap V^\ast$; 
  \item $P$ est l'hyperplan \`a l'infini $V\setminus \{0\}/\dG_m$ de l'espace 
    affine $V$ sur $\spec(A)$. Si $Q$ est l'hyperplan \`a l'infini de $W$, 
    $W^\ast/\dG_m$ et $V^\ast/\dG_m$ sont des ouverts des espaces projectifs 
    $Q$ et $P$ sur $\spec(A)$. 
    \begin{equation*}\tag{4.6.1}\label{VI:eq:4-6-1}
    \xymatrix{
      W^\ast \ar@{^{(}->}[r] \ar[d]^-\pi 
        & V^\ast \ar[d]^-\pi \\
      W^\ast/\dG_m \ar@{^{(}->}[r] 
        & V^\ast/\dG_m 
    }\qquad\subset\qquad
    \xymatrix{
      W\setminus \{0\} \ar@{^{(}->}[r] \ar[d]^-\pi 
        & V\setminus \{0\} \ar[d]^-\pi \\
      Q \ar@{^{(}->}[r] 
        & P \text{.}
    }
    \end{equation*}
\end{enumerate}

Si $M=A^I$, $I$ un ensemble fini, alors $V\simeq \dG_a^I$, $V^\ast=\dG_m^I$, 
$V^\ast/\dG_m$ est le tore $\dG_m^I/(\dG_m\text{ diagonal})$, $N$ et $T$ 
s'\'ecrivant $\prod x_i$ et $\sum x_i$, et $Q$ est donc l'hyperplan 
projectif d'\'equation $\sum x_i=0$ de $P$. 

Le cas qui nous int\'eresse est celui o\`u $M$ est fini \'etale sur $A$. La 
description pr\'ec\'edente vaut alors localement sur $\spec(A)$ (pour la 
topologie \'etale) et on peut \'ecrire $V^\ast=\dG_m^I$, pour $I$ un faisceau 
localement constant d'ensembles finis sur $\spec(A)$. Si par exemple $A$ est un 
corps, de cl\^oture alg\'ebrique $\bar A$, et que $M$ est un $A$-alg\`ebre 
s\'eparable, on pose $I=\hom_A(M,\bar A)$, et, sur $\bar A$, on a 
$V\sim \dG_a^I$. Via cet isomorphisme, $N$ et $T$ s'\'ecrivent $\prod x_i$ et 
$\sum x_i$. 





\subsection{Torseur de Kummer}\label{VI:4-7}

Soient $S$ un sch\'ema, $n$ un entier inversible sur $S$ et $G$ un sch\'ema en 
groupes commutatifs \`a fibres connexes (not\'e multiplicativement) sur $S$. 
La suite 
\[\xymatrix{
  0 \ar[r] 
    & G_n \ar[r] 
    & G \ar[r]^-{x^n} 
    & G \ar[r] 
    & 0 
}\]
est exacte. Elle d\'efinit un $G_n$-torseur $K_n(G)$ (ou simplement $K_n$) sur 
$G$. Pour $\chi:G_n \to E_\lambda^\times$ un homomorphisme du $S$-faisceau 
\'etale $G_n$ dans le faisceau constant $E_\lambda^\times$, on note 
$\sK_n(\chi)$ le $E_\lambda$-faisceau $\chi^{-1}(K_n)$. 

Les torseurs $K_n$ forment un syst\`eme projectif de torseurs sous le 
syst\`eme projectif de groupes $G_n$ (morphismes de transition 
$x\mapsto x^d:G_{n d}\to G_n$). Ceci exprime la commutativit\'e des diagrammes 
\[\xymatrix{
  0 \ar[r] 
    & G_{n d} \ar[r] \ar[d]^-{x^d} 
    & G \ar[r]^-{x^{n d}} \ar[d]^-{x^d} 
    & G \ar[r] \ar@{=}[d] 
    & 0 \\
  0 \ar[r] 
    & G_n \ar[r] 
    & G \ar[r]^-{x^n} 
    & G \ar[r] 
    & 0 \text{.} 
}\]
On a donc $\sK_n(\chi) = \sK_{n d}(\chi\circ x^d)$. 





\subsection{}\label{VI:4-8}

Appliquons la construction \ref{VI:4-6} pour $A=\dF_q$, et $M=k$ un alg\`ebre 
\'etale sur $\dF_q$. Conform\'ement aux conventions g\'en\'erales, on notera 
avec un indice $0$ les $\dF_q$-sch\'emas not\'es $V,V^\ast,\ldots$ en 
\ref{VI:4-6}. Les m\^emes lettres sans indice d\'esignent les sch\'emas sur 
$\dF$ qui s'en d\'eduisent par extension des scalaires. 

On pose $I=\hom(k,\dF)$, d'o\`u $V\sim \dG_a^I$ et $T$ s'\'ecrit 
$(x_i)\mapsto \sum x_i$. 
\begin{equation*}\tag{4.8.1}\label{VI:eq:4-8-1}
  \sF(\psi\circ \tr_{k/\dF_q}) = T^\ast \sF(\psi) = \bigotimes_i \operatorname{pr}_i^\ast \sF(\psi) \text{.} 
\end{equation*}
Puisque $V^\ast\sim \dG+m^I$, un caract\`ere $\chi$ de $V_n^\ast\sim \dmu_n^I$, 
\`a valeurs dans $E_\lambda^\times$, s'\'ecrit comme une famille 
$(\chi_i)_{i\in I}$ de caract\`eres de $\dmu_n$ ind\'en\'ee par $I$. Elle est 
d\'efinie sur $\dF_q$ si, pour tout $\sigma\in \gal(\dF/\dF_q)$, on a 
$\chi_{\sigma_i} = \chi_i\circ \sigma^{-1}$. Elle d\'efinit alors un 
$E_\lambda$-faisceau $\sK_n((\chi_i)_{i\in I})$ sur $V_0^\ast$. Si le produit 
des $\chi_i$ est trivial, $\chi$ se factorise par un caract\`ere de 
$(V_0^\ast/\dG_m)_n$ et $\sK_n((\chi_i)_{i\in I})$ est l'image r\'eciproque 
d'un faisceau, not\'e de m\^eme, sur $V-0^\ast/\dG_m$. Cette construction se 
compare comme suit au torseur de Lang. 





\begin{lemma_}\label{VI:4-9}
\begin{enumerate}[(i)]
  \item Pour $n$ assez divisible, on a un diagramme commutatif 
    \[\xymatrix{
      0 \ar[r] 
        & (V_0^\ast)_n \ar[r] \ar[d]^-\tau 
        & V_0^\ast \ar[r]^-{x^n} \ar[d]^-\tau 
        & V_0^\ast \ar[r] \ar@{=}[d] 
        & 0 \\
      0 \ar[r] 
        & k^\times \ar[r] 
        & V_0^\ast \ar[r]^-\fL 
        & V_0^\ast \ar[r] 
        & 0 \text{;} 
    }\]
    si $\chi$ est un caract\`ere $\chi:k^\times \to E_\lambda^\times$, on a 
    $\sF(\chi) = \sK_n(\chi\circ\tau)$. 
  \item Pour $n$ assez divisible, $\tau$ identifie $k^\times$ aux coinvariants 
    de $\gal(\bar\dF/F)$ agissant sur $V_n^\ast\sim \dmu_n^I$. 
  \item Pour $k$ un corps, $N=[k:\dF_q]$, $\omega\in I$ un plongement de $k$ 
    dans $\dF$, et $n=q^N-1$, la composante d'indice $\omega$ de 
    $\chi\circ\tau$ est $\chi\circ \omega^{-1}$. 
  \item Si $\chi$ est non trivial sur chaque facteur de $k$, les 
    $(\chi\circ \tau)_i$ sont tous non triviaux. 
\end{enumerate}
\end{lemma_}

Il suffit de prouver le lemme lorsque $k$ est un corps, de degr\'e $N$ sur 
$\dF_q$. Dans ce cas, $n$ est ``assez divisible'' si $q^N-1\mid n$. 
Choisissons un plongement $\omega$ de $k$ dans $\dF_q$; $I$ s'identifie alors 
\`a $\dZ/N$: \`a $i\in \dZ/N$ correspond $\omega_i=\omega^{q^i}$. Via 
l'isomorphisme $V_0^\ast(\dF) = {F^\times}^I$, on a 
\begin{enumerate}[\indent a)]
  \item $x\in k^\times$ correspond \`a $(\omega_i(x))\in {\dF^\times}^I$; 
  \item $F((x_i)_{i\in \dZ/N})=(x_{i-1}^q)_{i\in \dZ/N}$. 
\end{enumerate}

Un caract\`ere $\chi=(\chi_i)$ de $V_n^\ast\sim \dmu_n^I$ sera d\'efini sur 
$\dF_q$ si $\chi_{-i} =\chi_0(x^{q^i})$ ($i\in \dZ$). Si $q^N-1\mid n$, il y a 
$q^N-1$ tels caract\`eres: $\chi_0$ se factorise par $\dmu_{q^N-1}$, et 
d\'etermine les $\chi_i$. Pour prouver (ii), il suffit donc de v\'erifier (i) 
et (iii) (ou son corollaire (iv)) qui assure que $\chi\mapsto \chi\circ \tau$ 
est injectif. 

Prouvons (i) et (iii), pour $n=q^N-1$. Notons additivement le groupe des 
endomorphismes de $V^\ast\sim \dG_m^I$; en particulier, notons $n$ 
l'op\'erateur $x\mapsto x^n$. Si $\alpha$ est l'op\'erateur de permutations 
circulaire $(x_i)\mapsto (x_{i-1})$, on a 
$q^N-1=(q\alpha)^N-1 = (q\alpha-1)((q\alpha)^{N-1}+\cdots + 1)$. Ceci 
d\'etermine $\tau$: on a $\tau((x_i)) = \prod_{0\leqslant j<N} x_{i-j}^{q^j}$, 
et l'application induite de $(V_0^\ast)_n$ dans $k^\times$ s'\'ecrit [a) 
ci-dessus] $(x_i)\mapsto \prod x_{-i}^{q^i}$; de l\`a r\'esulte (iii). 





\begin{proposition_}\label{VI:4-10}
La cohomologie de $V^\ast$ \`a coefficients dans 
$\sF(\chi^{-1}\cdot \psi\tr_{k/\dF_q})$ v\'erifie 
\begin{enumerate}[\indent (i)]
  \item $\h_c^i = 0$ pour $i\ne N$, et $\dim \h_c^N=1$. 
  \item Sur $\h_c^N$, $F^\ast$ est la multiplication par 
    $\tau_{\dF_q}(\chi,\psi)$. 
  \item Si $\chi$ est non trivial sur chaque facteur de $k$, on a 
    $\h_c^\bullet \iso \h^\bullet$. 
\end{enumerate}
\end{proposition_}

Appliquons \ref{VI:4-9}(i): sur $\dF$, si $\chi\circ \tau=(\chi_i)_{i\in I}$, 
on a $\sF(\chi) = \sK_n((\chi_i)_{i\in I})$. Les points (i) et (iii) 
r\'esultant donc de l'\'enonc\'e plus g\'eom\'etrique suivant (pour (iii), 
appliquer \ref{VI:4-9}(iv)) et (ii) en r\'esulte par la formule des traces: 





\begin{proposition_}\label{VI:4-11}
Soit $(\chi_i)_{i\in I}$ une famille de caract\`ere de $\dmu_n(\dF)$. La 
cohomologie de $V^\ast\simeq \dG_m^I$ \`a coefficient dans 
$\sK_n((\chi_i)_{i\in I})\otimes \sF(\psi\tr_{k/\dF_q})$ v\'erifie 
\begin{enumerate}[\indent (i)]
  \item $\h_c^i{}=0$ pour $i\ne N$, et $\dim \h_c^N{} = 1$. 
  \item Si les $\chi_i$ sont tous non triviaux, on a 
    $\h_c^\bullet{}\iso \h^\bullet$. 
\end{enumerate}
\end{proposition_}

On a $V^\ast\sim \dG_m^I$, 
$\sK((\chi_i)) = \bigotimes_i \operatorname{pr}_i^\ast \chi_i(K_n(\dG_m))$ et 
$\sF(\psi\circ \tr_{\dF_q/k})=T^\ast \sF(\psi) = \bigotimes_i \operatorname{pr}_i^\ast \sF(\psi)$. 
Ceci permet d'appliquer la formule de K\"unneth, et \ref{VI:4-11} r\'esulte de 
\ref{VI:4-3}. 





\subsection{}\label{VI:4-12}

De \ref{VI:2-4*}* et (\hyperlink{VI}{Cycle}, \ref{VI:1-3} exemple 2), on tire 
aussi que le groupe des permutations $\sigma$ de $I$ telles que 
$\chi_i = \chi_{\sigma i}$ ($i\in I$) agit sur $\h_c^N$ par multiplication par 
la signature $\varepsilon(\sigma)$. Nous allons en d\'eduire une seconde preuve 
de l'identit\'e de Hasse-Davenport. 

Si $\chi$ est un caract\`ere de $\dF_q^\times$, le fait que, sur $\dF$, $N$ 
s'\'ecrit $(x_i)\mapsto \prod x_i$ fournit: 
$\sF(\chi\circ N)=N^\ast\sF(\chi) = \bigotimes_i \operatorname{pr}_i^\ast\sF(\psi)$ 
et la formule de K\"unneth fournit: 
\[
  \h_c^\bullet(V^\ast,\sF(\chi^{-1}\circ N,\psi\circ \tr)) \sim \h_c^\bullet(\dG_m,\sF(\chi^{-1}\psi))^{\otimes I} \text{,} 
\]
o\`u, au membre de droite, le produit tensoriel est pris au sens 
\ref{VI:2-4*}*. Puirsque $\h_c^1(\dG_m,\sF(\psi\chi^{-1}))$ est de dimension 
$1$, sa puissance tensorielle $\otimes I$, au sens ordinaire, ne d\'epend que 
du cardinal $n$ de $I$. Appliquant (\hyperlink{IV}{Cycle}, \ref{IV:1-3} exemple 
2), on obtient un isomorphisme canonique 
\begin{equation*}\tag{4.12.1}\label{VI:eq:4-12-1}
  \h_c^N\left(V,\sF(\chi^{-1}\circ N)\otimes T^\ast \sF(\psi)\right) \sim \h_c^1(\dG_m,\sF(\psi\chi^{-1})) \otimes_\dZ \textstyle \bigwedge^N \dZ^I \text{.}
\end{equation*}
Pour calculer l'action de $F^\ast$, le plus commode est d'adopter le point de 
vue galoisien et de dire que l'isomorphisme \eqref{VI:eq:4-12-1} \'etant 
canonique, il est compatible \`a l'action par transport de structure de 
$\gal(\dF/\dF_q)$. Si $\varepsilon(k)$ est la signature de la permutation 
$\varphi$ de $I$, on trouve que 
\begin{align*} 
  \tau_{\dF_q}(\chi\circ N_{k/\dF_q},\psi) 
    &= \tr\left(\varphi^{-1},\h_c^N\left(V^\ast,N^\ast\sF(\chi^{-1})\otimes T^\ast \sF(\psi)\right)\right) \\ 
    &= \varepsilon(k) \tr\left(\varphi^{-1},\h_c^1(\dG_m,\sF(\chi^{-1}\psi))\right)^N \\
    &= \varepsilon(k) \tau(\chi,\psi)^N \text{,} 
\end{align*}
soit 
\begin{equation*}\tag{4.12.2}\label{VI:eq:4-12-2}
  \tau_{\dF_q}\left(\chi\circ N_{k/\dF_q},\psi\right) = \varepsilon(k) \tau(\chi,\psi)^N \text{.} 
\end{equation*}

Si $k$ est un corps, $\varphi$ est une permutation circulaire de $I$, 
$\varepsilon(k) = (-1)^{N+1}$, et on retrouve l'identit\'e de Hasse-Davenport: 
\[
  \tau\left(\chi\circ N_{k/\dF_q},\psi\circ \tr_{k/\dF_q}\right) = (-1)^{N+1} \tau_{\dF_q}\left(\chi\circ N_{k/\dF_q},\psi\right) = \tau(\chi,\psi)^N \text{.} 
\]





\begin{lemma_}\label{VI:4-13}
Pour $\chi\circ \tau=(\chi_i)_{i\in I}$ comme en \ref{VI:4-9}(i), les
conditions suivantes sont \'equivalentes 
\begin{enumerate}[\indent (i)]
  \item $\chi| \dF_q^\times$ est trivial 
  \item Le produit des $\chi_i$ est trivial 
  \item $\sF(\chi) = \sK_n((\chi_i))$ est image r\'eciproque d'un (unique) 
    faisceau sur $V^\ast/\dG_m$. 
  \item L'image r\'eciproque de $\sF(\chi)$ sur $\dG_m$ (envoy\'e dans 
    $V^\ast$ \`a partir du morphisme structural $\dF_q \to k$) est triviale. 
\end{enumerate}
\end{lemma_}

(i)$\Rightarrow$(ii). On regarde $\chi$ comme une caract\`ere de 
$V^\ast/\dG_m(\dF_q)=k^\times/\dF_q^\times$, et on prend $\sF(\chi)$ sur 
$V^\ast/\dG_m$. 

(ii)$\Rightarrow$(iii). De m\^eme, on regarde $(\chi_i)$ comme un caract\`ere 
de $(V^\ast/\dG_m)_n$. L'unicit\'e dans (iii) r\'esulte de ce que $V^\ast$ est 
un fibr\'e \`a fibres connexes sur $V^\ast/\dG_m$, et (iii)$\Rightarrow$(iv)  
est trivial. 

(iv)$\Rightarrow$(i), (ii). Cette image r\'eciproque est 
$\sF(\chi|\dF_q^\times)$ et $\sK_n(\prod \chi_i)$. 





\subsection{}\label{VI:4-14}

Soient $k$ une alg\`ebre \'etale sur $\dF_q$, de 
dimension $N+1$ et $\chi$ un caract\`ere non trivial de $k^\times$, trivial sur 
$\dF_q^\times$. La somme de Jacobi $J(\chi)$ est d\'efinie par 
\begin{equation*}\tag{4.14.1}\label{VI:eq:4-14-1}
  J(\chi) = (-1)^{N-1} \sum_{\substack{x\in k^\times/\dF_q^\times \\ \tr(x)=0}} \chi^{-1}(x) \text{.} 
\end{equation*}

On a entre sommes de Gauss et sommes de Jacobi l'identit\'e suivante. 





\begin{proposition_}\label{VI:4-15}
Pour $\chi$ comme ci-dessus et $\psi$ un caract\`ere additif non trivial de 
$\dF_q$, on a 
\[
  q J(\chi) = \tau_{\dF_q}(\chi,\psi) \text{.} 
\]
\end{proposition_}

Dans le cas particulier o\`u $k=\dF_q^n$, cette formule se r\'ecrit par 
\eqref{VI:eq:4-5-3} 
\begin{equation*}\tag{4.15.1}\label{VI:eq:4-15-1}
  q J(\chi) = \prod_i \tau(\chi_i,\psi) \text{,} 
\end{equation*}
plus \'ecrit sous la forme 
\begin{align*}
  \chi_0(-1) J(\chi) &= (-1)^{N-1} \sum_{\substack{x_1,\dots,x_N\in \dF_q^\times \\ \sum x_i = 1}} \prod_{i=1}^N \chi_i^{-1}(x_i) = \tau(\chi_0^{-1},\psi)^{-1} \prod_{i=1}^N \tau(\chi_i,\psi) \\
  \chi_0^{-1} &= \prod_{i=1}^N \chi_i 
\end{align*}





\begin{proof}[Preuve de \ref{VI:4-15}]
\[
  \tau_{\dF_q}(\chi,\psi) = (-1)^{N+1} \sum_{x\in k^\times} \chi(x)^{-1} \psi \tr(x) = (-1)^{N+1} \sum_{x\in k^\times/\dF_q^\times} \chi(x)^{-1} \sum_{\lambda\in \dF_q^\times} \psi(\lambda \tr x) \text{.}
\]
La somme $\sum_{\lambda\in \dF_q^\times} \psi(\lambda \tr x)$ vaut $q-1$ si 
$\tr(x)=0$, et $-1$ si $\tr(x)\ne 0$. D\`es lors, 
\[
  \tau_{\dF_q}(\chi,\psi) = (-1)^{N+1} \left(q\sum_{x\in k^\times/\dF_q^\times} \chi(x)^{-1} - \sum_{x\in k^\times/\dF_q^\times} \chi(x)^{-1}\right) \text{.}
\]
La second terme au second membre est nul, car somme des valeurs d'un 
caract\`ere, et \ref{VI:4-15} en r\'esulte. 
\end{proof}

La proposition \ref{VI:4-15} se transpose ainsi en cohomologie: 





\begin{proposition_}\label{VI:4-16}
La cohomologie de $W^\ast/\dG_m$ (\ref{VI:4-6}, \ref{VI:4-8}) \`a coefficient 
dans $\sF(\chi^{-1})$ v\'erifie 
\begin{enumerate}[\indent (i)]
  \item $\h_c^i=0$ pour $i\ne N-1$, et $\dim \h_c^{N-1}{}=1$. 
  \item Sur $\h_c^{N-1}$, $F^\ast$ est la multiplication par $J(\chi)$. 
  \item $\h_c^{N-1}$ est canoniquement isomorphe \`a 
    $\h_c^{N+1}\left(V^\ast,\sF(\chi^{-1},\psi\circ \tr_{k/\dF_q})\right)(1)$. 
\end{enumerate}
\end{proposition_}

L'assertion (ii) r\'esulte de (i) et de la formule des traces; (i) et (iii) 
r\'esultent de \ref{VI:4-11} et de l'\'enonc\'e plus g\'eom\'etrique suivant. 





\begin{proposition_}\label{VI:4-17}
Soit $(\chi_i)_{i\in I}$ une famille de caract\`eres non trous triviaux de 
$\dmu_n(\dF)$, de produit $1$. Alors, 
$\h_c^{i-1}\left(W^\ast/\dG_m,\sK_m((\chi_i)_{i\in I})\right)$ est 
canoniquement isomorphe \`a 
\newline
$\h_c^{i+1}\left(V^\ast,\sK_m((\chi_i)_{i\in I})\otimes \sF(\psi\tr_{k/\dF_q})\right)(1)$. 
\end{proposition_}

La premi\`ere ligne de \ref{VI:4-15} devient ($\pi$ comme en 
\eqref{VI:eq:4-6-1}). 
\begin{equation*}\tag{4.17.1}\label{VI:eq:4-17-1}
  \eR^i \pi_!\left(\sK_n((\chi_i)_{i\in I})\otimes \sF(\psi \tr_{k/\dF_q})\right) = \sK_n((\chi_i)_{i\in I}) \otimes \eR^i \pi_! \sF(\psi\tr_{k/\dF_q}) 
\end{equation*}
(sur $V^\ast/\dG_m$). Calculons les faisceaux 
$\eR^i\pi_! \sF(\psi\tr_{k/\dF_q}) = \eR^i \pi_! T^\ast \sF(\psi)$. Soit 
$v_0:\widetilde V_0 \to V_0$ l'\'eclat\'e de $V_0$ en $\{0\}$. Dans le 
diagramme 
\begin{equation*}\tag{4.17.2}\label{VI:eq:4-17-2}
\xymatrix{
  V_0\setminus \{0\} \ar@{^{(}->}[r] \ar[dr]_-\pi 
    & \widetilde V_0 \ar[r]^-{v_0} \ar[d]^-{\bar\pi} 
    & V_0 \\
  & P_0 
}
\end{equation*}
$\pi$ est une fibration de fibre des droites \'epoint\'ees, $\widetilde V_0$ 
est le fibr\'e en droites correspondant, et $Z_0=v_0^{-1}(0)$ sa section $0$. 
Le faisceau $T^\ast\sF(\psi)$ sur $V-0\setminus \{0\}$ se prolonge en 
$(T v_0)^\ast \sF(\psi)$ sur $\widetilde V_0$, et la restriction de ce faisceau 
\`a $Z_0$ est le faisceau constant $E_\lambda$, car $T_0 v_0$ est nul sur 
$Z_0$. La suite exacte longue de cohomologie \`a support propre s'\'ecrit 
donc 
\begin{equation*}\tag{4.17.3}\label{VI:eq:4-17-3}
\xymatrix{
  \ar[r] 
    & \eR^i \pi_! T^\ast \sF(\psi) \ar[r] 
    & \eR^i \bar\pi_! (T v)^\ast \sF(\psi) \ar[r] 
    & (E_\lambda\text{ pour $i=0$, $0$ sinon}) \ar[r] 
    & 
}
\end{equation*}

L'image r\'eciproque $\bar\pi^{-1}(Q_0)$ est l'\'eclat\'e $\widetilde W_0$ de 
$W_0$ en $0$; $T_0 V_0$ s'annule sur $\widetilde W_0$; on a donc 
$(T v)^\ast \sF(\psi)=E_\lambda$ sur $\pi^{-1}(Q_0)$ et, $\bar\pi$ \'etant un 
fibr\'e en droites 
\begin{equation*}\tag{4.17.4}\label{VI:eq:4-17-4}
  \eR^i \bar\pi_! (T v)^\ast \sF(\psi)|Q = 
    \begin{cases}
      0 & \text{pour $i\ne 2$} \\
      E_\lambda(-1) & \text{pour $i=2$.} 
    \end{cases} 
\end{equation*}

Si $x\in P$, $x\notin Q$, la droite $D=\pi^{-1}(x)$ est envoy\'e 
isomorphiquement sur $\dG_a$ par $T v$; on a donc 
\ref{VI:2-7}*) 
\begin{align} \notag 
  \h_c^\bullet(D,(T v)^\ast \sF(\psi)) &= \h^\bullet(\dG_a,\sF(\psi)) = 0 && \text{et} \\
  \tag{4.17.5}\label{VI:eq:4-17-5}
  \eR^i \bar\pi_! (T v)^\ast \sF(\psi) 
    &= \begin{cases} 
         0 & \text{pour $i\ne 2$} \\
         E_\lambda(-1)_Q & \text{pour $i=2$} 
       \end{cases} 
\end{align}

Conjuguant \eqref{VI:eq:4-17-3} et \eqref{VI:eq:4-17-5}, on trouve enfin 
\begin{equation*}\tag{4.17.6}\label{VI:eq:4-17-6}
  \eR^i \pi_! T^\ast\sF(\psi) = 
    \begin{cases}
      0 & \text{pour $i\ne 1,2$} \\
      E_\lambda & \text{pour $i=1$} \\
      E_\lambda(-1)_Q & \text{pour $i=2$.} 
    \end{cases}
\end{equation*}





\subsection{}\label{VI:4-18}

Calculons la cohomologie \`a support propre du faisceau 
$\sK_n((\chi_i))\otimes T^\ast \sF(\psi)$ sur $V^\ast$ \`a l'aide de la suite 
spectrale de Leray de $\pi:V^\ast\to V^\ast/\dG_m$. Appliquant 
\eqref{VI:eq:4-17-1} et \eqref{VI:eq:4-17-6}, on trouve comme termes initiaux 
les $E_2^{p,1} = \h_c^p(V^\ast/\dG_m,\sK_n(\chi_i)) = 0$ (car $(\chi_i)\ne 0$) 
et les $E_2^{p,2} = \h_c^p(W^\ast/\dG_m,\sK(\chi_i))(-1)$. 

La suite spectrale se r\'eduit \`a un isomorphisme, et \ref{VI:4-17} en 
r\'esulte. 





\subsection{}\label{VI:4-19}

Les faisceaux $\sK_n((\chi_i)_{i\in I})$ ont un sens sur n'importe quel corps, 
voire sur n'importe sch\'ema de base. Ceci va nous permettre de g\'en\'eraliser 
\ref{VI:4-16}(i). Avec les notations de \ref{VI:4-6}, supposons $M$ fini 
\'etale partout de rang $N$ sur $A$, et soit $n$ un entier inversible dans $A$. 
On note $a$ la projection de $W^\ast/\dG_m$ sur $\spec(A)$. Soit aussi $\chi$ 
un caract\`ere (morphisme de faisceaux) 
$\chi:(V^\ast/\dG_m)_n \to E_\lambda^\times$, et $\sK_n(\chi)$ le faisceau 
correspondant. Localement pour la topologie \'etale, on peut regarder $\chi$ 
comme une famille de caract\`eres $(\chi_i)_{i\in I}$ de $\dmu_n$, de produit 
trivial. 





\begin{proposition_}\label{VI:4-20}
\begin{enumerate}[(i)]
  \item Si $\chi$ est (en tout point) non trivial, on a 
    $\eR^i a_! (\sK_n(\chi)) = 0$ pour $i\ne N-1$, et 
    $\eR^{N-1} a_!(\sK_n(\chi))$ est lisse, de rang $1$. 
  \item Une automorphisme $\sigma$ de $M$ qui respecte $\chi$ agit sur ce 
    $\eR^{N-1} a_!$ par multiplication par la signature $\varepsilon(\sigma)$ 
    de $\sigma$, vu comme permutation de $I$. 
  \item Si les $\chi_i$ sont tous non triviaux, on a $\eR a_! \iso \eR a_\ast$. 
\end{enumerate}
\end{proposition_}
\begin{proof}[Preuve]
Le sch\'ema $W^\ast/\dG_m$ est le compl\'ement d'un diviseur \`a croisements 
normaux relatif dans le sch\'ema $Q$, propre et lisse sur $\spec(A)$, et 
$\sK_n(\chi)$ est localement constant sur $W^\ast/\dG_m$, \`a ramification 
mod\'er\'ee a l'infini. Il en r\'esulte que les $\eR^i a_!$ et 
$\eR^i a_\ast$ sont lisses, de formation compatible \`a tout changement de 
base. Un argument standard nous ram\`ene alors \`a supposer que $A$ est un 
corps fini, et (i) r\'esulte de \ref{VI:4-17} et \ref{VI:4-11}(i). Pour 
prouver (ii), on utilise que l'action de $\sigma$ est compatible \`a 
l'isomorphisme \ref{VI:4-17}, et \ref{VI:4-12}. Pour (iii), on note que si les 
$\chi_i$ sont tous non triviaux, alors $\sK_n(\chi)$ sur 
$W^\ast/\dG_m\subset Q$ est ramifi\'e le long de chaque diviseur \`a l'infini, 
et \ref{VI:1-19-1}. 
\end{proof}










\section{Caract\`eres de Hecke}\label{VI:5}





\subsection{}\label{VI:5-1}

Soient $F$ un corps de nombres (de degr\'e fini sur $\dQ$) et $k$ un corps de 
caract\'eristique $0$. Voici diverses façon \'equivalentes de dire ce qu'est un 
homomorphisme \emph{alg\'ebrique} $\alpha:F^\times \to k^\times$. 
\begin{enumerate}[(i)]
  \item Soit $\{e_i\}$ une base de $F$ sur $\dQ$. L'homomorphisme $\alpha$ est 
    alg\'ebrique s'il est donn\'e par une formule 
    $\alpha(\sum x^i e_i) = A(x^i)$, $A\in k(X^i)$. 
\end{enumerate}

Ceci signifie que $\alpha$ coïncide sur $F^\times$ avec une application 
rationnelle d\'efinie sur $k$ de $\res_{F/\dQ}(\dG_m)$ dans $\dG_m$. Par 
densit\'e de Zariski de $F^\times$, et le fait qu'un homomorphisme birationnel 
est partout d\'efini, une telle application est un homomorphisme de sch\'emas 
en groupes: 
\begin{enumerate}[(i)]
\setcounter{enumi}{1}
  \item $\alpha$ est induit par un homomorphisme de $k$-sch\'emas en groupes 
    \[
      \res_{F/\dQ}(\dG_m)\otimes_\dQ k \to \dG_m \text{.} 
    \] 
\end{enumerate}

Si $k$ est un corps de nombre, la propri\'et\'e d'adjonction de la restriction 
des scalaires $R$ montre que ceci \'equivaut \`a 
\begin{enumerate}[(i)]
\setcounter{enumi}{2}
  \item ($k$ un corps de nombres) $\alpha$ est induit par un homomorphisme de 
    $\dQ$-sch\'emas en groupes: $\res_{F/\dQ}(\dG_m) \to \res_{k/\dQ}(\dG_m)$. 
\end{enumerate}

Soient $\bar k$ un cl\^oture alg\'ebrique de $k$, et $I=\hom(F,\bar k)$. Sur 
$\bar k$, le groupe des caract\`eres de $\res_{F/\dQ}(\dG_m)$ est $\dZ^I$, de 
base les plongements de $F$ dans $\bar k$. Les caract\`eres d\'efinis sur $k$ 
sont ceux invariants par $\gal(\bar k/k)$; on peut les d\'ecrire soit comme 
ceux envoyant $F^\times$ dans $k^\times\subset \bar k$, soit en terme des 
orbites de $\gal(\bar k/k)$ dans $I$, correspondant elle-m\^eme aux facteurs de 
$F\otimes k$. 
\begin{enumerate}[(i)]
\setcounter{enumi}{3}
  \item $\alpha$ est de la forme 
    $\alpha=\prod_{\omega\in I} \omega^{n_\omega}$. Les familles d'exposants 
    $(n_\omega)$ permises sont celles telles que $n_\omega=n_{\omega'}$ pour 
    $\omega$ et $\omega'$ dans la m\^eme orbite de $\gal(\bar k/k)$. Ce sont 
    encore celles telle que $\prod_\omega (x)^{n_\omega}\in k$ pour tout 
    $x\in F$. 
  \item Posons $F\otimes k=\prod_{j\in J} F_j$, les $F_j$ \'etant des corps. 
    Alors, $\alpha$ s'\'ecrit $\alpha=\prod N_{F_j/k}^{m_j}$. 
\end{enumerate}

Dans le cas particulier o\`u $k$ contient une cl\^oture normale de $F$, ces 
expressions deviennent: $\alpha=\prod \omega^{n_\omega}$, o\`u $\omega$ 
parcourt les $[F:\dQ]$ plongements de $F$ dans $k$. 





\subsection{}\label{VI:5-2}

Supposons que $k$ soit un corps de nombres, et soit $\alpha$ un homomorphisme 
alg\'ebrique de $F^\times$ dans $k^\times$. Il existe alors un et un seul 
homomorphisme, encore not\'e $\alpha$, du groupe $I(F)$ des id\'eaux 
fractionnaires de $F$ dans celui de $k$, tel que $\alpha((x))=(\alpha(x))$. 
L'unicit\'e r\'esulte de ce que tout id\'eal $a$ une puissance qui est un 
id\'eal principal, et de ce que $I(k)$ est sans torsion. Pour prouver 
l'existence, on utilise par exemple \ref{VI:5-1}(v): on a 
$\alpha(a) = \prod N_{F_j/k}^{m_j}((a))$. 





\subsection{}\label{VI:5-3}

Rappelons la d\'efinition des caract\`eres de Hecke alg\'ebriques, appel\'es 
Weil caract\`eres de Hecke (ou: gr\"ossencharaktere) de type $A_0$. Soient $F$ 
un corps de nombres, $\fm$ un id\'eal de $F$ (i.e., de l'anneau des entiers de 
$F$), $I_\fm$ le groupe des id\'eaux fractionnaires de $F$ premiers \`a $\fm$, 
et $k$ un corps de caract\'eristique $0$. Un homomorphisme 
$\chi:I_\fm \to k^\times$ est un \emph{caract\`ere de Hecke alg\'ebrique} (de 
conducteur $\leqslant \fm$) s'il existe un homomorphisme alg\'ebrique 
$\chi_\text{alg}:F^\times \to k^\times$ v\'erifiant 
\begin{equation*}\tag{$*$}\label{VI:eq:5-3-1}
\text{Pour $x\in F^\times$, premier \`a $\fm$, totalement positif et 
$\equiv 1\pmod\fm$, on a $\chi((x))=\chi_\text{alg}(x)$.}
\end{equation*}

Par densit\'e de l'ensemble des $x$ de \eqref{VI:eq:5-3-1}, $\chi_\text{alg}$ 
est enti\`erement d\'etermin\'e par $\chi$. C'est la \emph{partie alg\'ebrique} 
de $\chi$. Si $\chi((x))=\chi_\text{alg}(x)$ pour $x$ totalement positif et 
$\equiv 1\pmod{\fm'}$, $\chi$ se prolonge en un caract\`ere de Hecke de 
conducteur $\leqslant \inf(\fm,\fm')$: $I_{\fm+\fm'}\to k^\times$. On 
identifiera les caract\`eres de Hecke qui coïncident sur leur domaine commun de 
d\'efinition, et on appelle \emph{conducteur} de $\chi$ le plus petit $\fm'$ 
tel que $\chi$ soit de conducteur $\leqslant \fm'$. 





\subsection{Remarque}\label{VI:5-4}

Si un caract\`ere de Hecke alg\'ebrique $\chi$ prend ses valeurs dans un 
sous-corps $k'$ de $k$, c'est d\'ej\`a un caract\`ere de Hecke \`a valeurs dans 
$k'$: il suffit de voire que $\chi_\text{alg}:\res_{F/\dQ}(\dG_m) \to \dG_m$ 
est d\'ej\`a d\'efini sur $k'$, et ceci r\'esulte de la densit\'e de Zariski 
dans $\res_{F/\dQ}(\dG_m)$ de l'ensemble des $x$ totalement positif 
$\equiv 1\pmod\fm$. On pour toujours prendre pour $k'$ un sous-corps de $k$ de 
degr\'e fini sur $\dQ$. 





\subsection{}\label{VI:5-5}

Si $\varepsilon$ est une unit\'e totalement positive $\equiv 1\pmod\fm$, on a 
$\chi_\text{alg}(\varepsilon) = \chi((\varepsilon)) = 1$. L'homomorphisme 
$\chi_\text{alg}$ se factorise donc par le quotient $T_\fm$ de 
$\res_{F/\dQ}(\dG_m)$ par l'adh\'erence de Zariski du groupe 
$E_\fm\subset F^\times$ des unit\'es totalement positifs $\equiv 1\pmod\fm$. 

D'apr\`es Serre \cite[II.3]{se68}, si $k$ est une cl\^oture alg\'ebrique de 
$\dQ$, et que $\fm$ est assez grand, les caract\`eres $\prod \omega^{n_\omega}$ 
de $\res_{F/\dQ}(\dG_m)$ de la forme $\chi_\text{alg}$ sont caract\'eris\'es 
comme suit: il doit exister un entier $N$, le \emph{poids} de $\chi$ (ou de 
$\chi_\text{alg}$), tel que pour tout \'el\'ement $\sigma$ de $\gal(k/\dQ)$ 
conju\'e \`a la conjugaison complexe, on ait $n_\omega+n_{\sigma \omega}=N$. Si 
$F_1$ est le plus grand sous-corps de $F$ qui soit une extension quadratique 
totalement imaginaire d'un corps totalement r\'eel $F_1'$, cela revient \`a 
dire que $\chi_\text{alg}$ est de la forme $\chi_1\circ N_{F/F_1}$, et que 
$\chi_1|F_1=(N_{F_1/\dQ})^N$. 

Si $\chi$ est de poids $N$, pour tout id\'eal $\fa$ de $F$, $\chi(\fa)$ est un 
nombre alg\'ebrique dont tous les conjugu\'es sont de valeur absolue 
$N(\fa)^{N/2}$ \cite[II.3 prop.2]{se68}. 





\subsection{}\label{VI:5-6}

Pour $k$ un corps de nombres fix\'e, des conditions suppl\'ementaires sont 
impos\'ees \`a $\chi_\text{alg}$. Pour tout id\'eal $\fa$ de $F$ premier au 
conducteur, on a en effet 
\begin{equation*}\tag{5.6.1}\label{VI:eq:5-6-1}
  \chi_\text{alg}(\fa) = (\chi(\fa)) \text{,} 
\end{equation*}
de sorte que $\chi_\text{alg}(\fa)$ est principal. Puisque le groupe des 
id\'eaux fractionnaire de $k$ est sous torsion, il suffit de prouver la 
puissance $n$-i\`eme de \eqref{VI:eq:5-6-1}, $n\ne 0$ convenable; ceci permet 
de remplacer $\fa$ par $\fa^n$, donc de supposer $\fa=(x)$, avec $x$ totalement 
positif $\equiv 1\pmod\fm$. Il ne reste qu'\`a utiliser les d\'efinitions. 

Ceci, joint \`a \ref{VI:5-5}, montre que $\chi_\text{alg}$ d\'etermine la norme 
des $\chi(\fa)$ en toutes les places de $k$. 





\subsection{}\label{VI:5-7}

Le groupe $S_\fm$ de Serre pourrait \^etre caract\'eris\'e comme \'etant le 
groupe de type multiplicatif dont le groupe des caract\`eres (sur n'importe 
quel corps) est le groupe des caract\`eres de Hecke alg\'ebriques de conducteur 
$\leqslant \fm$. (cf. \cite[II 2.1 et 2.2]{se68}). Sa relation avec les 
repr\'esentations $\ell$-adiques est expliqu\'ee en \cite[II 2.3]{se68}. 





\begin{theorem_}[{\cite{se68}}]\label{VI:5-8}
Soit $\chi$ un caract\`ere de Hecke alg\'ebrique de $F$ dans $E_\lambda$, pour 
$E_\lambda$ une extension finie de $\dQ_\ell$. Il existe alors un (et un seul) 
homomorphisme $\chi_\lambda:\gal(\bar F/F)^\textnormal{ab} \to E_\lambda$, tel 
que 
\begin{enumerate}[\indent (i)]
  \item $\chi_\lambda$ est non ramifi\'e en dehors du conducteur $\ff$ de 
    $\chi$ et de $\ell$. 
  \item Pour $\fp$ un id\'eal premier de $F$ premier \`a $\ff$ et \`a $\ell$, 
    et $F_\fp\in \gal(\bar F/F)^\textnormal{ab}$ le Frobenius g\'eom\'etrique 
    en $\fp$, on a 
    \[
      \chi_\lambda(F_\fp) = \chi(\fp) \text{.}
    \] 
\end{enumerate}
\end{theorem_}











\chapter{Théorèmes de finitude en cohomologie \texorpdfstring{$\ell$}{l}-adique}\label{VII}

\section{Énoncé des théorèmes}\label{VII:1}

\begin{theorem_}\label{VII:1-1}
foo
\end{theorem_}





% !TEX root = sga4.5.tex

\chapter{Catégories dérivées}\label{VIII}










\section{Catégories triangulées: définitions et exemples}\label{VIII:1}





\subsection{Définition des catégories triangulées}\label{VIII:1-1}


\addtocounter{subsubsection}{-1}
\subsubsection{}\label{VIII:1-1-0}

Soient $\cA$ un catégorie additive, $T$ un automorphisme additif de $\cA$. 
$X$ et $Y$ étant deux objets de $\cA$, nous poserons 
\[
  \hom^i(X,Y) = \hom_\cA(X,T^i(Y)) \qquad i\in \dZ 
\]
Soient $X$, $Y$, $Z$ trois objets de $\cA$, $\alpha\in \hom^i(X,Y)$, 
$\beta\in \hom^j(X,Y)$. Nous définirons le composé $\beta\circ\alpha$ 
appartenant à $\hom^{i+j}(X,Z)$ par: 
\[
  \beta\circ \alpha = \beta\circ T^i(\alpha) \text{.}
\]
On vérifie qu'on obtient ainsi une nouvelle catégorie dont les groupes de 
morphismes entre les objets sont gradués. Cette nouvelle catégorie sera 
appelée, par abus de langage: la catégorie $\cA$, graduée par le foncteur 
de translation $T$. 

Soit $\cA$ une catégorie graduée par un foncteur de translation $T$. 
Lorsqu'on nommera ou notera un morphisme sans spécifier le degré, il 
s'agira toujours d'un morphisme de degré zéro. 

Soient $\cA$ et $\cA'$ deux catégories additives graduées par les foncteurs
$T$ et $T'$, $F$ un foncteur additif de $\cA$ dans $\cA'$. On dira que $F$ 
\emph{est gradué} si l'on s'est donné un isomorphisme de 
foncteurs\footnote{Texte modifié en 1976: le texte original demandait une 
égalité $F\circ T=T'\circ F$. Le cas des foncteurs à plusieurs variables 
est discuté dans \cite[XVIII 0.2]{sga4}. Il pose un problème de signes.} 
\[
  F\circ T = T'\circ F 
\]

La définition des morphismes de foncteurs gradués est donnée au 
\ref{VIII:4-1-1}. 

$\cA$ étant une categorie additive graduée par le foncteur $T$, on 
appellers triangle, un ensemble de trois objets $X$, $Y$, $Z$ et de trois 
morphismes $u:X\to Y$, $v:Y\to Z$, $w:Z\to T(X)$ ($\deg w=1$). Pour désigner 
les triangles on utilisera la notation: $(X,Y,Z,u,v,w)$ on bien le diagramme: 
\[\xymatrix{
  & Z \ar[dl]_-w \\
  X \ar[rr]^-u 
    & & Y \ar[lu]_-v 
}\qquad \deg(w)=1
\]
Soient $(X,Y,Z,u,v,w)$, $(X',Y',Z',u',v',w')$ deux triangles de $\cA$; un 
morphisme du premier triangle dans le second est un ensemble de trois 
morphismes $f:X\to X'$, $g:Y\to Y'$, $h:Z\to Z'$, tels que le diagramme 
suivant soit commutatif: 
\[\xymatrix{
  X \ar[r]^-u \ar[d]^-f 
    & Y \ar[r]^-v \ar[d]^-g 
    & Z \ar[r]^-w \ar[d]^-h 
    & T(X) \ar[d]^-{T(f)} \\
  X' \ar[r]^-{u'} 
    & Y' \ar[r]^-{v'} 
    & Z' \ar[r]^-{w'} 
    & T(X') 
}\]





\subsubsection{}\label{VIII:1-1-1}

On appelle \emph{catégorie triangulée} une catégorie additive graduée 
par un foncteur de translation, munie d'une famille de triangles qu'on appelle 
famille des triangles distingués. Cette famille doit vérifier de plus les 
axiomes: 


\paragraph{TR1}
\phantomsection\hypertarget{VIII:TR1}{}

Tout triangle isomorphe à triangle distingué est distingué. Tout 
morphisme $u:X\to Y$ est contenu dans un triangle distingué 
$(X,Y,Z,u,v,w)$. Le triangle $(X,X,0,\operatorname{id}_{X'},0,0)$ est 
distingué. 


\paragraph{TR2}
\phantomsection\hypertarget{VIII:TR2}{}

Pour que $(X,Y,Z,u,v,w)$ soit distingué, il faut et il suffit que 
$(Y,Z,T(X),v,w,-T(u))$ soit distingué. 


\paragraph{TR3}
\phantomsection\hypertarget{VIII:TR3}{}

$(X,Y,Z,u,v,w)$, $(X',Y',Z',u',v',w')$ étant distingués, pour tout 
morphisme $(f,g):u\to u'$, il existe un morphisme $h:Z\to Z'$ tel que $(f,g,h)$ 
soit un morphisme de triangles. 


\paragraph{TR4}
\phantomsection\hypertarget{VIII:TR4}{}

\[\xymatrix{
  & Z' \ar[dl]|{\deg=1} \\
  X \ar[rr]^-u 
    & & Y \ar[ul]_-i 
}\quad
\xymatrix{
  & X' \ar[dl]|{\deg(j)=1} \\
  Y \ar[rr]^-v 
    & & Z \ar[ul] 
}\quad
\xymatrix{
  & Y' \ar[dl]|{\deg=1} \\
  X \ar[rr]^-w 
    & & Z \ar[ul] 
}\]
étant trois triangles distingués tels que $w=v\circ u$, il existe deux 
morphismes $f:Z'\to Y'$, $g:Y' \to X'$ tels que 
\begin{enumerate}
  \item $(\operatorname{id}_X, v, f)$ soit un morphisme de triangle 
  \item $(u,\operatorname{id}_Z,g)$ soit un morphisme de triangle 
  \item $(Z',Y',X',f,g,T(i) j)$ soit un triangle distingué
\end{enumerate}

$\cC$ et $\cC'$ étant deux catégories triangulées, on appelle 
\emph{foncteur exact} de $\cC$ dans $\cC'$, tout foncteur additif, gradué, 
transformant les triangles distingués en triangles distingués. Les 
morphismes de foncteurs exacts sont les morphismes de foncteurs gradués. 

On déduit immédiatement dex axiomes \hyperlink{VIII:TR1}{TR1}, 
\hyperlink{VIII:TR2}{TR2}, \hyperlink{VIII:TR3}{TR3} la propriété suivante: 





\begin{proposition}\label{VIII:1-1-2}
Soient $(X,Y,Z,u,v,w)$ un triangle distingué et $M$ un objet d'une 
catégorie triangulée. On a alors les suites exactes: 
\[\xymatrix{
  \cdots \ar[r] 
    & \hom^i(M,X) \ar[r]^-{\hom(M,u)} 
    & \hom^i(M,Y) \ar[r]^-{\hom(M,v)} 
    & \hom^i(M,Z) \ar[r]^-{\hom(M,w)} 
    & \hom^{i+1}(M,X) \ar[r] 
    & \cdots
}\]
\[\xymatrix{
  \cdots \ar[r] 
    & \hom^i(Z,M) \ar[r]^-{\hom(v,M)} 
    & \hom^i(Y,M) \ar[r]^-{\hom(u,M)} 
    & \hom^i(X,M) \ar[r]^-{\hom(w,M)} 
    & \hom^{i+1}(Z,M) \ar[r] 
    & \cdots
}\]
\end{proposition}

On déduit alors de cette proposition des énoncés du genre: 
\begin{itemize}
  \item $(f,g,h)$ étant un morphisme de triangle distingué et $f,g$ des 
    isomorphismes. 
  \item Les triangles distingués construits sur un morphisme 
    (\hyperlink{VIII:TR1}{TR1}), sont tour isomorphismes (\emph{mais les 
    isomorphismes ne sont pas, en général, uniquoment déterminés}). 
  \item Lorsque dans un triangle distingué, un des morphismes admet un noyau 
    ou un conoyau, celui-ci se scinde i.e. il est de la forme: 
    \[\xymatrix{
      & Z+T(X) \ar[dl] \\
      X+Y \ar[rr] 
        & & Y+Z \ar[ul]
    }\]
  \item La somme de deux triangles distingués est distinguée. (On utilise 
    l'axiome \hyperlink{VIII:TR4}{TR4}). 
\end{itemize}










\subsection{Exemples}\label{VIII:1-2}





\subsubsection{}\label{VIII:1-2-1}

Nous allons d'abord fixer quelques notations. Soit $\cA$ une catégorie 
additive $\eC(\cA)$ désignera la catégorie suivante: 
\begin{itemize}
  \item Les objets de $\eC(\cA)$ seront les complexes de $\cA$, sans limitation 
    de degré, à différentielle de degré $+1$. 
  \item Les morphismes de $\eC(\cA)$ seront les morphismes de complexes (qui 
    commutent avec la différentielle) conservant le degré. 
\end{itemize}

Soit $T$ le foncteur suivant: pour tout objet $X^\bullet$ de $\eC(\cA)$ 
\begin{align*}
  T(X^\bullet)_i &= (X^\bullet)_{i+1} \\
  d_i(T(X^\bullet)) &= -d_{i+1}(X^\bullet)  \text{.}
\end{align*}
Sur les morphismes le foncteur $T$ agit de la manière suivante: pour tout 
morphisme $f:X^\bullet \to Y^\bullet$, le morphisme 
$T(f):T(X^\bullet) \to T(Y^\bullet)$ est défini par 
\[
  T(f)_i = f_{i+1} \text{.}
\]
On vérifie immédiatement que $T$ est un automorphisme additif de 
$\eC(\cA)$. Les morphismes de degré $n$, définis à l'aide de ce foncteur 
de translation, sont alors les morphismes qui augmentent le degré de $n$, et 
qui commutent ou anticommutent avec la différentielle suivant la parité de 
$n$. On retrouve ainsi la définition généralement adoptée. 

$\sC$ désignera la famille de triangles suivante: un triangle 
$(X^\bullet,Y^\bullet,Z^\bullet,u,v,w)$ est un élément de la famille 
si\footnote{Texte modifié en 1976, pour assurer la validité de 
\ref{VIII:1-2-4}.}
\begin{itemize}
  \item $Z^\bullet$ est le complexe simple associé au complexe double 
    \[\xymatrix{
      \cdots \ar[r] 
        & 0 \ar[r] 
        & X^\bullet \ar[r]^-u 
        & Y^\bullet \ar[r] 
        & 0 \ar[r] 
        & \cdots 
    }\]
    o'ù les objets de $Y^\bullet$ sont les objets de premier degré zéro, 
    et les objets de $X^\bullet$, les objets de premier degré $-1$. 
  \item $v$ est l'image par le foncteur: complexe double $\mapsto$ complexe 
    simple, du morphisme de doubles complexes: 
    \[\xymatrix{
      \cdots \ar[r] 
        & 0 \ar[r] \ar[d] 
        & 0^\bullet \ar[r] \ar[d] 
        & Y^\bullet \ar[r] \ar[d] 
        & 0 \ar[r] \ar[d] 
        & \cdots \\
      \cdots \ar[r] 
        & 0 \ar[r] 
        & X^\bullet \ar[r] 
        & Y^\bullet \ar[r] 
        & 0 \ar[r] 
        & \cdots 
    }\]
  \item $w$ est l'opposé de l'image, par le même foncteur, du morphisme de 
    doubles complexes: 
    \[\xymatrix{
      \cdots \ar[r] 
        & 0 \ar[r] 
        & X^\bullet \ar[r] 
        & 0 \ar[r] 
        & 0 \ar[r] 
        & \cdots \\
      \cdots  \ar[r] 
        & 0 \ar[r] \ar[u] 
        & X^\bullet \ar[r] \ar[u] 
        & Y^\bullet \ar[r] \ar[u] 
        & 0 \ar[r] \ar[u] 
        & \cdots 
    }\]
\end{itemize}





\subsubsection{}\label{VIII:1-2-2}

L'ensemble des morphismes homotopes à zéro est un idéal bilatère (si 
$f$ et $g$ sont composables et si $f$ ou $g$ homotope à zéro, la composé 
est homotope à zéro. Si $f$ et $g$ sont homotopes à zéro, le morphisme 
$f\oplus g$ est homotope à zéro). Pour tout couple d'objet $X^\bullet$, 
$Y^\bullet$ on désigne par $\operatorname{Htp}(X^\bullet,Y^\bullet)$ le 
sous-groupe de $\hom_{\eC(\cA)}(X^\bullet,Y^\bullet)$ des morphismes homotopes 
à zéro. $\eK(\cA)$ sera alors la catégorie dont les objets sont les 
objets de $\eC(\cA)$ et les morphismes de source $X^\bullet$ et de but 
$Y^\bullet$, le groupe: 
\[
  \hom_{\eC(\cA)}(X^\bullet,Y^\bullet) / \operatorname{Htp}(X^\bullet,Y^\bullet) \text{.}
\]
La loi de composition sur les morphismes de $\eK(\cA)$ se définit à partir 
de la loi de composition sur les morphismes de $\eC(\cA)$ en passant au 
quotient. $\eK(\cA)$ est une catégorie additive. Le foncteur de translation 
$T$ sur $\eC(\cA)$ passe au quotient (le translaté d'un morphisme homotope 
à zéro, est homotope à zéro). Le foncteur obtenu est un automorphisme 
additif de $\eK(\cA)$ que nous noterons encore $T$. Par abus de notation 
$\eK(\cA)$ désignera par la suite la catégorie $\eK(\cA)$ graduée par le 
foncteur $T$. 

On appellera triangle distingué dans $\eK(\cA)$, tout triangle 
\emph{isomorphe} à un triangle provenant de $\sC$. La famille des triangles 
distingués vérifie les axioms \hyperlink{VIII:TR1}{TR1}, 
\hyperlink{VIII:TR2}{TR2}, \hyperlink{VIII:TR3}{TR3} et 
\hyperlink{VIII:TR4}{TR4}. Par abus de notation $\eK(\cA)$ désignera la 
catégorie $\eK(\cA)$ triangulée par la famille de triangles définie 
ci-dessus. 





\subsubsection{}\label{VIII:1-2-3}

On définit de plus tout un arsenal de sous-catégories: 
\begin{itemize}
  \item Un complexe est dit borné inférieurement si tous ses objets de 
    degré négatif, sauf au plus un nombre fini, sont nuls. On définit de 
    même les complexes bornés supérieurement, les complexes bornés. On 
    désignera par $\eK^+(\cA)$, $\eK^-(\cA)$, $\eK^b(\cA)$ les 
    sous-catégories pleines de $\eK(\cA)$ engendrées par ses familles 
    d'objets. Ces sous-catégories sont des ``\emph{sous-catégories 
    triangulées}'' de $\eK(\cA)$: tout triangle distingué dont deux des 
    objets sont des objets de la sous-catégorie, est isomorphe à un 
    triangle dont les trois objets sont des objets de la sous-catégorie. 
  \item Supposons que $\cA$ soit une catégorie abélienne. Un complexe est 
    dit cohomologiquement borné inférieurement si tous ses objets de 
    cohomologie de degré négatif sont nuls sauf un nombre fini d'entre 
    eux. On définit de manière analogue les complexes cohomologiquement 
    bornés supérieurement, les complexes cohomologiquement bornés, les 
    complexes acycliques. $\eK^{\infty,+}(\cA)$, $\eK^{\infty,-}(\cA)$, 
    $\eK^{\infty,b}(\cA)$, $\eK^{\infty,\varnothing}(\cA)$ désigneront les 
    sous-catégories pleines engendrées par ces familles d'objets. Ce sont 
    des sous-catégories triangulées. 
    
    On paiera aussi des sous-catégories triangulées $\eK^{+,b}(\cA)$, 
    $\eK^{+,\varnothing}(\cA)$, \ldots etc \ldots (le premier signe en 
    exposant donne des renseignements sur les objets des complexes, le 
    deuxième signe sur les objets de cohomologie). 
  \item $\cA$ n'étant plus nécessairement abélienne, soit $O$ un 
    ensemble d'objets de $\cA$ stable par isomorphisme et par somme directe. 
    $O(\cA)$ désignera la sous-catégorie pleine de $\eC(\cA)$ engendrée 
    par les complexes dont les objets en tout degré sont des objets de $O$. 
    On notera $\underline O(\cA)$ la sous-catégorie triangulée 
    correspondante dans $\eK(\cA)$. $\cA$ étant abélienne, on pourra 
    prendre pour ensemble $O$, l'ensemble $I$ des injectifs ou l'ensemble $P$ 
    des projectifs. 
\end{itemize}





\subsubsection{}\label{VIII:1-2-4}

$\eK(\cA)$ est uniquement déterminée par la donnée de $\eC(\cA)$ et de la 
famille $\sC$. De manière précise: tout foncteur additif gradué de 
$\eC(\cA)$ dans une catégorie triangulée, transformant tout triangle de 
$\sC$ en un triangle distingué, se factorise de manière unique par 
$\eK(\cA)$, et le foncteur obtenu est exact. 

Pour tout complexe $X^\bullet$ de $\eC(\cA)$, $\bar X^\bullet$ désignera le 
complexe obtenu à partir de $X^\bullet$ en annulant la différentielle. Une 
suite à trois termes: 
\[\xymatrix{
  0 \ar[r] 
    & X^\bullet \ar[r] 
    & Y^\bullet \ar[r] 
    & Z^\bullet \ar[r] 
    & 0 
}\]
sera dite suite semi-scindée si la suite 
\[\xymatrix{
  0 \ar[r] 
    & \bar X^\bullet \ar[r] 
    & \bar Y^\bullet \ar[r] 
    & \bar Z^\bullet \ar[r] 
    & 0 
}\]
est scindée; i.e. si elle est isomorphe à la suite: 
\[\xymatrix{
  0 \ar[r] 
    & \bar X^\bullet \ar[r] 
    & \bar X^\bullet + \bar Z^\bullet \ar[r] 
    & \bar Z^\bullet \ar[r] 
    & 0 \text{.}
}\]

Soit $0 \to X^\bullet \to Y^\bullet \to Z^\bullet \to 0$ une suite 
semi-scindée. Choisissons un scindage i.e. pour tout $n$ un isomorphisme 
\[
  (Y^\bullet)^n \simeq (X^\bullet)^n + (Z^\bullet)^n \text{.}
\]
La matrice de la différentielle de $Y^\bullet$, dans la décomposition en 
somme directe si-dessus est de la forme: 
\[
  \begin{pmatrix}
    d_{X^\bullet} & \varphi_n \\
    0 & d_{Y^\bullet} 
  \end{pmatrix}
\]
où $\varphi_n$ est un morphisme de $(Z^\bullet)^n$ dans 
$(X^\bullet)^{n+1}$. Les $\varphi_n$ déterminent un morphisme de complexes 
\[
  \varphi:Z^\bullet \to T(X^\bullet) \text{.}
\]
Lorsqu'on change le scindage, la classe modulo homotopie de $\varphi$ ne 
change pas. De plus, en notant $\underline u$, $\underline v$, 
$\underline\varphi$ les images dans $\eK(\cA)$ des morphismes $u$, $v$, 
$\varphi$, le triangle 
$(X^\bullet,Y^\bullet,Z^\bullet,\underline u,\underline v,\underline\varphi)$ 
est distingué. Désignons par $\mathsf{SSS}(\cA)$ la catégorie des 
suites semi-scindées de $\eC(\cA)$, et par $\mathsf{TrK}(\cA)$ la 
catégorie des triangles distingués de $\eK(\cA)$. Ce qui précède 
permet de définir un foncteur: 
\[
  \rho:\mathsf{SSS}(\cA) \to \mathsf{TrK}(\cA) \text{.}
\]
Ce foncteur est essentiellement surjectif. 










\subsection{Exemples de foncteurs cohomologiques et de foncteurs exacts}\label{VIII:1-3}





\begin{definition}
Soient $\cA$ une catégorie triangulée, $\cB$ une catégorie abélienne. 
Un foncteur additif $F:\cA\to \cB$ est die \emph{foncteur cohomologique} si 
pour tout triangle distingué $(X,Y,Z,u,v,w)$ la suite 
\[\xymatrix{
  F(X) \ar[r]^-{F(u)} 
    & F(Y) \ar[r]^-{F(v)} 
    & F(Z) 
}\]
est exacte. 
\end{definition}

Le foncteur $F\circ T^i$ sera souvent noté $F^i$. En vertu de l'axiome 
\hyperlink{VIII:TR2}{TR2} des catégories triangulées, on a la suite 
exacte illimitée: 
\[\xymatrix{
  \cdots \ar[r] 
    & F^i(X) \ar[r] 
    & F^i(Y) \ar[r] 
    & F^i(Z) \ar[r] 
    & F^{i+-}(X) \ar[r] 
    & \cdots 
}\]





\subsubsection{Exemples}\label{VIII:1-3-2}

\begin{enumerate}
  \item Soit $\cA$ une catégorie triangulée. Pour tout objet $X$ de $\cA$ 
    le foncteur $\hom_\cA(X,-)$ est un foncteur cohomologique. Le foncteur 
    $\hom_\cA(-,X)$ est un foncteur cohomologique $\cA^\circ\to\mathsf{Ab}$. 
  \item Soient $\cA$ une catégorie abélienne, $\eC(\cA)$ la catégorie des 
    complexes de $\cA$. On désigne par $\h^0:\eC(\cA)\to\cA$ le foncteur 
    suivant: Soit $X^\bullet$ un objet de $\eC(\cA)$. On pose 
    \[
      \h^0(X^\bullet) = \ker d^0 / \im{d^{-1}} \text{.}
    \]
    $\h^0$ annule les morphismes homotopes à zéro, donc se factorise de 
    manière unique par $\eK(\cA)$. On désignera encore par 
    $\h^0:\eK(\cA) \to \cA$ le foncteur obtenu. 
\end{enumerate}





\subsubsection{Exemple de bi-foncteur exact}\label{VIII:1-3-3}

Soient $\cA$, $\cA'$, $\cA''$ trois catégories additives, 
\[
  P:\cA\times \cA'\to \cA''
\]
un foncteur bilinéaire (i.e. additif par rapport à chacun des arguments 
séparément). On en déduit alors le foncteur bilinéaire: 
\[
  P^\ast:\eC(\cA)\times \eC(\cA') \to \eC(\cA'') 
\]
de la manière suivante: 

Soient $X^\bullet$ un objet de $\eC(\cA)$ et $Y^\bullet$ un objet de 
$\eC(\cA')$. $P(X^\bullet,Y^\bullet)$ est un complexe double de $\cA''$. On 
pose alors: $P^\ast(X^\bullet,Y^\bullet) = $ complexe simple associé à 
$P(X^\bullet,Y^\bullet)$. 

Soient $f$ un morphisme de $\eC(\cA)$ (resp. $\eC(\cA')$) homotope à zéro 
et $Z^\bullet$ un objet de $\eC(\cA')$ (resp. $\eC(\cA)$). Le morphisme 
$P^\ast(f,Z^\bullet)$ (resp. $P^\bullet(Z^\bullet,f)$) est alors homotope à 
zéro. On en déduit que $P^\ast$ définit d'une manière unique un 
foncteur: 
\[
  P^\bullet:\eK(\cA) \times \eK(\cA') \to \eK(\cA'') \text{.}
\]
$P^\bullet$ est un bi-foncteur exact. 

En particulier, soit $\cA$ un catégorie additive. On peut prendre pour $P$ le 
foncteur: 
\begin{align*}
  \cA^\circ\times \cA &\to \mathsf{Ab} \\
  (X,Y) &\mapsto \hom(X,Y) 
\end{align*}

On obtient alors par la construction précédente un foncteur 
\[
  \hom^\bullet:\eK(\cA)^\circ \times \eK(\cA) \to \eK(\mathsf{Ab}) 
\]
qui, composé avec le foncteur $\h^0:\eK(\mathsf{Ab})\to \mathsf{Ab}$, 
redonne évidemment le foncteur $\hom_{\eK(\cA)}$. 




















\section{Les catégories quotients}\label{VIII:2}










\subsection{Sous-catégories épaisses. Systèmes multiplicatifs de morphismes}\label{VIII:2-1}

Nous commencerons par donner deux définitions. 





\begin{definition}\label{VIII:2-1-1}
Une sous-catégorie $\cB$ d'une catégorie triangulée $\cA$ est dite 
\emph{épaisse} si $\cB$ est une sous-catégorie triangulée pleine de $\cA$ 
et si de plus $\cB$ possède la propriété suivante: 

Pour tout morphisme $f:X\to Y$, se factorisant par un objet de $\cB$ et contenu 
dans un triangle distingué $(X,Y,Z,f,g,h)$, où $Z$ est un objet de $\cB$, 
la source de $f$ et le but de $f$ sont des objets de $\cB$. 
\end{definition}

L'ensemble des sous-catégories épaisses de $\cA$ sera noté $\sN$. 

L'intersection d'une famille quelconque de sous-catégories épaisses est 
une sous-catégorie épaisse. 





\begin{definition}
Soit $\cA$ un catégorie triangulée. Un ensemble $S$ de morphismes de $\cA$ 
est appelé \emph{système multiplicatif de morphismes} si'il posède les 
cinq propriétés suivantes: 

\paragraph{FR1}
\phantomsection\hypertarget{VIII:FR1}{}
Si $f,g\in S$ et si $f$ et $g$ sont composables, $f\circ g\in S$. Pout tout 
objet $X$ de $\cA$, le morphisme identique de $X$ est un élément de $S$. 

\paragraph{FR2}
\phantomsection\hypertarget{VIII:FR2}{}
Dans la catégorie $\cA$, tout diagramme 
\[\xymatrix{
  & Y \ar[d]^-{s\in S} \\
  Z \ar[r]^-f 
    & X 
}\]
se complète en un diagramme commutatif 
\[\xymatrix{
  P \ar[r]^-g \ar[d]^-{t\in S} 
    & Y \ar[d]^-{s\in S} \\
  Z \ar[r]^-f 
    & X 
}\]
De plus la propriété symétrique est vraie. 

\paragraph{FR3}
\phantomsection\hypertarget{VIII:FR3}{}
Si $f$ et $g$ sont des morphismes, les propriétés suivantes sont 
équivalentes: 
\begin{enumerate}[\indent i)]
  \item il existe un $s\in S$ tel que $s\circ f=s\circ g$ 
  \item il existe un $t\in S$ tel que $f\circ t=g\circ t$
\end{enumerate}

\paragraph{FR4}
\phantomsection\hypertarget{VIII:FR4}{}
Soit $T$ le foncteur de translation de $\cA$. Par tout élément $s$ de $S$, 
$T(s)\in S$. 

\paragraph{FR5}
\phantomsection\hypertarget{VIII:FR5}{}
$(X,Y,Z,u,v,w)$, $(X',Y',Z',u',v',w')$ étant deux triangles distinguées, 
$f$ et $g$ deux éléments de $S$ tels que le couple $(f,g)$ soit un 
morphisme de $u$ dans $u'$, il existe un morphisme $h\in S$, tel que 
$(f,g,h)$ soit un morphisme de triangle. 
\end{definition}

Un système multiplicatif de morphismes est dit \emph{saturé} s'il possède 
la propriété suivante: 

Un morphisme $f$ appartient à $S$ si et seulement s'il existe deux 
morphismes $g$ et $g'$ tels que $g\circ f\in S$ et $f\circ g'\in S$. 

L'intersection d'une famille quelconque de systèmes multiplicatifs saturés 
un système multiplicatif saturé. 

L'ensemble des systèmes multiplicatifs saturés sera noté $\sS$. 

Soit $\cB$ une sous-catégorie épaisse; on désigne par $\varphi(\cB)$ 
l'ensemble des morphismes $f$ qui sont contenus dans un triangle distingué 
$(X,Y,Z,f,g,h)$ où $Z$ est un objet de $\cB$. $\varphi(\cB)$ est un 
système multiplicatif saturé. 

Soit $S$ un système multiplicatif saturé; on désigne par $\psi(S)$ la 
sous-catégorie pleine engendrée par les objets $Z$ contenus dans un 
triangle distingué $(X,Y,Z,f,g,h)$ où $f$ est un élément de $S$. La 
sous-catégorie $\psi(S)$ est une sous-catégorie épaisse. 

Le résultat principal de ce numéro est alors le suivant: $\varphi$ est un 
isomorphisme (conservant la relation d'ordre définie par l'inclusion) de 
$\sN$ sur $\sS$. L'isomorphisme inverse n'est autre que $\psi$. 










\subsection{Problèmes universels}\label{VIII:2-2}

Soient $\cA$ et $\cA'$ deux catégories triangulées et $F$ un foncteur 
exacte de $\cA$ dans $\cA'$. Soient $S(F)$ l'ensemble des morphismes de $\cA$ 
qui sont transformés par $F$ en isomorphismes et $\cB(F)$ la sous-catégorie 
pleine engendrée par les objets de $\cA$ qui sont transformés par $F$ en 
objets nuls de $\cA'$. $S(F)$ est un système multiplicatif saturé et 
$\cB(F)$ est une sous-catégorie épaisse. De plus $S(F) = \varphi(\cB(F))$. 
Soient alors $\cA$ une catégorie triangulée, $\cB$ une sous-catégorie 
épaisse et $S=\varphi(\cB)$ le système multiplicatif saturé 
correspondant. Les deux problèmes universels ci-dessous sont équivalents. 


\paragraph{Problème 1}
\phantomsection\hypertarget{VIII:prob1}{}
Trouver une catégorie triangulée $\cA/\cB$ et un foncteur exact 
$Q:\cA\to \cA/\cB$ tels que tout foncteur exact de $\cA$ dans une catégorie 
triangulée $\cA'$, transformant les objets de $\cB$ en objets nuls de 
$\cA'$, admette, de manière unique, une factorisation de la forme $G\circ Q$ 
où $G$ est un foncteur exact. 


\paragraph{Problème 2}
\phantomsection\hypertarget{VIII:prob2}{}
Trouver une catégorie triangulée $\cA_S$ et un foncteur exact 
$Q:\cA\to \cA_S$ tels que tout foncteur exact de $\cA$ dans une catégorie 
triangulée $\cA'$, transformant les morphismes de $S$ en isomorphismes de 
$\cA'$, admette, de manière unique, une factorisation de la forme 
$G\circ Q$ où $G$ est un foncteur exact. 

(Le rédacteur prie le lecteur de bien vouloir l'excuser pour l'emploi de ce 
langage désuet et rétrograde.) Nous allons montrer, dans le numéro 
suivant, que le \hyperlink{VIII:prob2}{problème 2} admet une solution. Donc 
le \hyperlink{VIII:prob1}{problème 1} correspondant admet la même solution. 










\subsection{Calcul de fraction}\label{VIII:2-3}





\subsubsection{}\label{VIII:2-3-1}

Soient $\cA$ un catégorie triangulée, $S$ un système multiplicatif 
saturé. Désignons par $\cA[S^{-1}]$ la catégorie de fraction de $\cA$ 
pour l'ensemble de morphismes $S$ et $Q$ le foncteur canonique de $\cA$ dans 
$\cA[S^{-1}]$. Le couple $(\cA[S^{-1}],Q)$ résout le problème suivant: Tout 
foncteur de $\cA$ dans une catégorie quelconque (non nécessairement 
additive) transformant morphisme de $S$ en isomorphisme, se factorise, d'une 
manière unique, par $(\cA[S^{-1}],Q)$. Un tel couple existe toujours, sans 
hypothèse sur l'ensemble de morphismes $S$ \cite[1.1]{gz67}\footnote{The 
original text references [C.G.G] here. Most likely, this is the lost work 
\emph{Catégories et foncteurs} by Chevalley, Gabriel and Grothendieck, 
referenced in\cite{sc72}. }. Cependant l'ensemble 
de morphismes $S$, étant un syst\`me multiplicatif saturé, possède les 
propriétés \hyperlink{VIII:FR1}{FR1}, \hyperlink{VIII:FR2}{FR2}, 
\hyperlink{VIII:FR3}{FR3}. Par suite, d'après \cite{gz67}, la catégorie 
$\cA[S^{-1}]$ peut s'obtenir par un calcul de fractions, à droite ou à 
gauche. Nous allons rappeler comment on obtient $\cA[S^{-1}]$ par un calcul de 
fractions à droite. (Le calcul de fractions à gauche s'obtient par le 
procédé du renversement des flèches.) 





\subsubsection{}\label{VIII:2-3-2}

Les objets de $\cA[S^{-1}]$ sont le objets de $\cA$. 
\begin{itemize}
  \item Pour tout objet $X$ de $\cA$, désignons par $S_X$ la catégorie des 
    flèches appartenant à $S$, ayant pour but $X$. A tout objet $Y$ de 
    $\cA$, on associe, fonctoriellement en $Y$, le foncteur suivant sur 
    $S_X^\circ$ à valeur dans la catégorie des ensembles: 
    ($S_X^\circ$ désigne la catégorie opposée) 
    \[
      H_Y:s\mapsto \hom(\operatorname{source}(s),Y) \text{.}
    \]
    On pose alors: 
    \[
      \hom_{\cA[S^{-1}]}(X,Y) = \varinjlim_{S_X^\circ} H_Y \text{.}
    \]
    Pour tout ce qui concerne les limites inductives, on se référera à 
    \cite{ar62}. 
    
    On constatera alors, avec plaisir en utilisant \hyperlink{VIII:FR1}{FR1}, 
    \hyperlink{VIII:FR2}{FR2}, \hyperlink{VIII:FR3}{FR3} que la catégorie 
    $S_X^\circ$ possède les propriétés L1, L2, L3. Les limites inductives 
    possèdent donc toutes les bonnes propriétés des limites inductives 
    sur les ensembles ordonnés filtrants. 
  \item Soient $X$, $Y$, $Z$, trois objets de $\cA$, $a$ un élément de 
    $\hom_{\cA[S^{-1}]}(X,Y)$ et $b$ un élément de 
    $\hom_{\cA[S^{-1}]}(Y,Z)$. Soient $s$ un objet de $S_X$, $\underline a$ un 
    élément de $\hom(\operatorname{source}(s),Y)$ dont l'image est $a$ et 
    un objet de $S_y$, $\underline b$ un élément de 
    $\hom(\operatorname{source}(t),Z)$ dont l'image est $b$. On a alors un 
    diagramme: 
    \begin{equation*}\tag{1}\label{VIII:eq:2-3-2-1}
    \xymatrix{
      & \bullet \ar[dl]_-s \ar[dr]^-{\underline a} 
        & 
        & \bullet \ar[dl]_-t \ar[dr]^-{\underline b} \\
      X 
        & 
        & Y 
        & 
        & Z 
    }
    \end{equation*}
    que, d'après \hyperlink{VIII:FR2}{FR2}, on peut compléter en un 
    diagramme: 
    \[\xymatrix{
      & & \bullet \ar[dl]_-{t'} \ar[dr]^-{\underline c} \\
      & \bullet \ar[dl]_-s \ar[dr]^-{\underline a} 
        & & \bullet \ar[dl]_-t \ar[dr]^-{\underline b} \\
      X 
        & & Y 
        & & Z 
    }\]
    Notons $b\circ a$ l'image dans $\hom_{\cA[S^{-1}]}(X,Z)$ du morphisme 
    $\underline{b\circ c}$. On vérifie que gr\^ace aux propriétés 
    \hyperlink{VIII:FR1}{FR1}, \hyperlink{VIII:FR2}{FR2}, 
    \hyperlink{VIII:FR3}{FR3}, $b\circ a$ ne dépend pas des représentants 
    $\underline a$ et $\underline b$ choisis et qu'il ne dépend pas non plus 
    de la manière de compléter le diagramme \eqref{VIII:eq:2-3-2-1}. On 
    vérifie de plus qu'on a ainsi défini une catégorie $\cA[S^{-1}]$. 
\end{itemize}

Nous noterons $Q$ le foncteur évident: $\cA\to \cA[S^{-1}]$. 

$\cA$ étant une catégorie additive, $\cA[S^{-1}]$ est une catégorie 
additive. De plus, on démontre en utilisant \hyperlink{VIII:FR4}{FR4} qu'il 
existe un et un seul foncteur de translation sur $\cA[S^{-1}]$, que nous 
noterons $T$, vérifiant la relation: 
\[
  Q\circ T = T\circ Q \text{.}
\]
Enfin, à l'aide de \hyperlink{VIII:FR5}{FR5}, on démontre qu'il existe une 
et une seule structure triangulée sur $\cA[S^{-1}]$ telle que le foncteur $Q$ 
soit exact. Les triangles distingués de $\cA[S^{-1}]$ ne sont qutre que les 
triangles isomorphes aux images, par $Q$, des triangles distingués de $\cA$. 
En désignant par $\cA_S$, la catégorie $\cA[S^{-1}]$ muni de cette 
structure triangulée, on démontre sans difficulté que le couple 
$(\cA_S,Q)$ est une solution au \hyperlink{VIII:prob2}{problème 2}. 





\begin{definition}\label{VIII:2-3-3}
Soient $\cA$ un catégorie triangulée, $\cB$ une sous-catégorie épaisse 
de $\cA$, $(Q,\cA/\cB)$ la solution du \hyperlink{VIII:prob1}{problème 1} 
(\S \ref{VIII:2-2}). $\cA/\cB$ sera appelée la \emph{catégorie quotient} de 
$\cA$ par $\cB$. $Q$ sera appelé le foncteur canonique de passage au 
quotient. Plus généralement soit $\cN$ une sous-catégorie triangulée de 
$\cA$ et soit $\underline\cN$ la plus petit sous-catégorie épaisse 
contenant $\cN$. La catégorie quotient de $\cA$ par $\cN$: $\cA/\cN$ sera par 
définition $\cA/\underline\cN$. 
\end{definition}










\subsection{Propriétés des catégories quotients}\label{VIII:2-4}





\subsubsection{}\label{VIII:2-4-1}

Soient $\cA$ une catégorie triangulée et $H$ un foncteur cohomologique à 
valeur dans une catégorie abélienne $\cG$. Soient $\cB_H$ la 
sous-catégorie pleine engendrée par les objets dont tous les translatés 
sont transformés par $H$ en objets nuls de $\cG$ et $S_H$ l'ensemble des 
morphismes dont tous les translatés sont transformés par $H$ en 
isomorphismes de $\cG$. $\cB_H$ est une sous-catégorie épaisse de $\cA$; 
$S_H$ est un système multiplicatif saturé. De plus $S_H=\varphi(\cB_H)$ 
(\S \ref{VIII:2-1}). Le foncteur $H$ se factorise d'un manière unique par 
$(Q,\cA/\cB_H)$. 





\begin{theorem}\label{VIII:2-4-2}
Soient $\cA$ une catégorie triangulée, $\cB$ un sous-catégorie 
triangulée, $\cN$ une sous-catégorie épaisse de $\cA$, $S=\varphi(\cN)$ 
le système multiplicatif saturé correspondant. 
\begin{enumerate}[(a)]
  \item La catégorie $\cN\cap\cB$ est une sous-catégorie épaisse de 
    $\cB$. Le système multiplicatif correspondant est $S\cap \cB$. 
  \item Les deux propriétés suivantes sont équivalentes: 
    \begin{enumerate}[(i)]
      \item Pour tout objet $X$ de $\cB$ et tout morphisme $s:R\to X$ où $s$ 
        est un élément de $S$, il existe un morphisme $s':R'\to R$ où 
        $s\circ'$ est un élément de $S\cap \cB$; et 
        $R'\in\operatorname{Ob}\cB$. 
      \item Tout morphisme d'un objet $X$ de $\cB$ dans un objet $Y$ de $\cN$ 
        se factorise par un objet de $\cN\cap \cB$. 
    \end{enumerate}
  \item Les propriétés (i)' et (ii)' obtenues à partir de (i) et (ii) en 
    renversant les flèches sont équivalents. 
  \item Si les proprietés (i) et (ii) sont vérifiées ou bien si les 
    propriétés (i)' et (ii)' sont vérifiées, le foncteur canonique 
    \[\xymatrix{
      \cB/\cN\cap \cB \ar[r] & \cA/\cN
    }\]
    est fidèle. 
    
    Comme ce foncteur est injectif sur les objets, il réalise 
    $\cB/\cN\cap\cB$ comme sous-catégorie de $\cA/\cN$. 
  \item Si de plus $\cB$ est une sous-catégorie pleine de $\cA$, 
    $\cB/\cN\cap \cB$ est une sous-catégorie pleine de $\cA/\cN$. 
\end{enumerate}
\end{theorem}





\begin{corollary}\label{VIII:2-4-3}
Soient $\cA\subset \cB\subset \cC$ trois catégories triangulées, $\cA$ sous 
catégorie épaisse de $\cB$, $\cB$ sous-catégorie épaisse de $\cC$. 
Alors $\cA$ est une sous-catégorie épaisse de $\cC$, $\cB/\cA$ est une 
sous-catégorie épaisse de $\cC/\cA$. Le foncteur canonique 
$\cC/\cB \to (\cC/\cA)/(\cB/\cA)$ est un isomorphisme. 
\end{corollary}










\subsection{Propriétés du foncteur de passage au quotient}\label{VIII:2-5}

Soient $\cA$ un catégorie triangulée, $\cB$ un sous-catégorie épaisse, 
$S$ le système multiplicatif correspondant, $Q:\cA\to \cA/\cB$ le foncteur 
canonique de passage au quotient. 





\begin{proposition}\label{VIII:2-5-1}
$X$ et $Y$ étant des objets de $\cA$, les assertions suivantes sont 
équivalentes: 
\begin{enumerate}[(a)]
  \item Tout morphisme $f:X\to Y$, tel qu'il existe un morphisme $s:Z\to X$, de 
    $S$, vérifiant $f\circ s=0$, est un morphisme nul.
  \item Tout morphisme $f:X\to Y$, tel qu'il existe un morphisme $s:Y\to Z$, 
    $s\in S$, vérifiant $s\circ f=0$, est un morphisme nul. 
  \item Tout morphisme $f:X\to Y$ qui se factorise par un objet de $\cB$ est 
    nul. 
  \item L'application canonique: 
    \[
      \hom_\cA(X,Y) \to \hom_{\cA/\cB}(Q(X),Q(Y))
    \]
    est injective. 
\end{enumerate}
\end{proposition}





\begin{proposition}\label{VIII:2-5-2}
Soient $X$ et $Y$ deux objets de $\cA$. Les assertions suivantes sont 
équivalentes: 
\begin{enumerate}[(a)]
  \item Tout diagramme ($s\in S$): 
    \[\xymatrix{
      & Z \ar[dl]_-s \ar[dr]^-f \\
      X 
        & & Y 
    }\]
    se compl`ete en un diagramme: $(s,t\in S$): 
    \[\xymatrix{
      & Z' \ar[d]_-t \\
      & Z \ar[dl]_-s \ar[dr]^-f \\
      X 
        & & Y 
    }\]
    où $f\circ t$ se factorise par $(X,s\circ t)$. 
  \item Assertion obtenue en renversant le sens de flèches dans (a) et en 
    permutant $X$ et $Y$. 
  \item Tout diagramme: 
    \[\xymatrix{
      X \ar[r]^-f 
        & N \ar[r]^-g 
        & Y 
    }\]
    où $g\circ f=0$ et où $N$ est un objet de $\cB$, se complète en un 
    diagramme: 
    \[\xymatrix{
      & N' \ar[d]^-i \\
      X \ar[ur]^-h \ar[r]^-f 
        & N \ar[r]^-g 
        & Y 
    }\]
    où $f=i\circ h$ et où $g\circ i=0$. 
  \item Assertion obtenue en changeant le sens des flèches dans (c) et en 
    permutant $X$ et $Y$. 
  \item L'application canonique: 
    \[\xymatrix{
      \hom_\cA(X,Y) \ar[r]^-Q 
        & \hom_{\cA/\cB}(Q(X),Q(Y)) 
    }\]
    est surjective. 
\end{enumerate}
\end{proposition}





\begin{proposition}\label{VIII:2-5-3}
Soit $X$ un objet de $\cA$. Les assertions suivantes sont équivalentes: 
\begin{enumerate}[(a)]
  \item Pour tout objet $Y$ de $\cA$, l'application canonique 
    \[\xymatrix{
      \hom_\cA(X,Y) \ar[r]^-Q 
        & \hom_{\cA/\cB}(Q(X),Q(Y)) 
    }\]
    est un isomorphisme. 
  \item Tout morphisme de $X$ dans un objet de $\cB$ est nul. 
\end{enumerate}
De même, les deux assertions suivantes sont équivalentes: 
\begin{enumerate}[(a)']
  \item Pour tout objet $Y$ de $\cA$, l'application canonique: 
    \[\xymatrix{
      \hom_\cA(X,Y) \ar[r]^-Q 
        & \hom_{\cA/\cB}(Q(X),Q(Y)) 
    }\]
    est un isomorphisme. 
  \item Tout morphisme d'un objet de $\cB$ dans $X$ est nul. 
\end{enumerate}
\end{proposition}





\begin{definition}\label{VIII:2-5-4}
Tout objet possédant les propriétés équivalentes (a) et (b) de la 
proposition précédente, sera appelé: \emph{objet libre à gauche sur 
$\cA/\cB$} ou bien encore objet \emph{$Q$-libre à gauche}. De même, tout 
objet possédant les propriétés équivalentes (a)' et (b)' sera appelé 
\emph{objet libre à droite sur $\cA/\cB$} ou encore objet 
\emph{$Q$-libre à droite}. Il est défini à isomorphisme près par la 
connaissance de $Q(X)\in\operatorname{Ob}{\cA/\cB}$. 
\end{definition}





\subsection{Foncteurs adjoints au foncteur de passage au quotient}\label{VIII:2-6}





\begin{definition}\label{VIII:2-6-1}
Deux sous-catégories triangulées $\cN$ et $\cN'$ d'une catégorie 
triangulée $\cA$ sont dites \emph{orthagonales} si pour tout objet $X$ de 
$\cN$ et tout objet $Y$ de $\cN'$, on a: 
\[
  \hom_\cA(X,Y) = 0 
\]
$\cN'$ sera dite alors orthogonale à droite à $\cN$ et $\cN$ orthogonale 
à gauche à $\cN'$. 
\end{definition}





\begin{proposition}\label{VIII:2-6-2}
Soit $\cN$ une sous-catégorie triangulée d'une catégorie triangulée 
$\cA$. La catégorie pleine $\cN^\bot$ (resp. $^\bot\cN$) engendrée par les 
objets $X$ de $\cA$ tels que pour tout objet $Y$ de $\cN$ on ait 
$\hom_\cA(Y,X)=0$ (resp. $\hom_\cA(X,Y)=0$), est une sous-catégorie épaisse 
de $\cA$. La catégorie $\cN^\bot$ est appelée par abus de langage, 
l'orthogonale à droite de $\cN$. 
\end{proposition}

Cette proposition nous permet de traduire la proposition \ref{VIII:2-5-3}. 





\begin{proposition}\label{VIII:2-6-3}
Soit $\cB\subset\cA$, une sous-catégorie épaisse d'une catégorie 
triangulée $\cA$. La catégorie pleine engendrée par les objets libre à 
droite sur $\cA/\cB$ (resp. à gauche), n'est autre que l'orthogonale à 
droite (resp. à gauche) de $\cB$. 
\end{proposition}





\begin{proposition}\label{VIII:2-6-4}
Soient $\cA$ une catégorie triangulée, $\cB$ une sous-catégorie 
épaisse de $\cA$, $\cB^\bot$ l'orthogonale à droite de $\cB$, $S(\cB)$, 
$s(\cB^\bot)$ les systèmes multiplicatifs correspondants, $Q$ et $Q^\bot$ les 
foncteurs canoniques de passage au quotient. Considérons pour un objet $x$ de 
$\cA$, les propriétés suivantes: 
\begin{enumerate}[(i)]
  \item $X$ s'envoie par un morphisme de $S(\cB)$ dans un objet de $\cB^\bot$. 
  \item $X$ reçoit par un morphisme de $S(\cB^\bot)$ un objet de $\cB$. 
  \item La catégorie $S_X(\cB)$ des flèches de $S(\cB)$ de source $X$, 
    admet un objet final. 
  \item La catégorie $S^X(\cB^\bot)$ des flèches de $S(\cB^\bot)$ de but 
    $X$, admet un objet initial. 
  \item La catégorie $\cB/X$ des objets de $\cB$ au-dessus de $X$, admet un 
    objet final. 
  \item La catégorie $\cB^\bot/X$ des objets de $\cB^\bot$ au-dessous de $X$ 
    admet un objet initial. 
\end{enumerate}
On a alors 
\[\xymatrix@=0.5cm{
  & & & & (iv) \\
  (i) \ar@{<=>}[r] 
    & (ii) \ar@{<=>}[r] 
    & (iii) \ar@{<=>}[r] 
    & (v) \ar@{=>}[ur] \ar@{=>}[dr] \\
  & & & & (vi) 
}\]

Si de plus $^\bot(\cB^\bot)=\cB$ alors toutes les propriétés sont 
équivalentes. 
\end{proposition}

Soient $\cB$ une sous-catégorie épaisse d'une catégorie triangulée 
$\cA$, le foncteur d'injection $i$ de $\cB$ dans $\cA$ et $Q$ le foncteur de 
passage au quotient de $\cA$ dans $\cA/\cB$, $\cB^\bot$ l'orthogonale à 
droite de $\cB$, les foncteurs correspondant $i^\bot$ et $Q^\bot$. On déduit 
immédiatement de la proposition \ref{VIII:2-6-4} les propositions suivantes: 





\begin{proposition}\label{VIII:2-6-5}
Les deux propriétés suivantes sont équivalentes: 
\begin{enumerate}[(i)]
  \item Le foncteur $i$ admet un adjoint à droite. 
  \item Le foncteur $Q$ admet un adjoint à droite. 
\end{enumerate}
\end{proposition}

La proposition est encore vraie lorsqu'on remplace le mot droite par le omt 
gauche. 





\begin{proposition}\label{VIII:2-6-6}
Les deux propriétés suivantes sont équivalentes: 
\begin{enumerate}[(i)]
  \item Le foncteur $i$ admet un adjoint à droite. 
  \item Le foncteur $i^\bot$ admet un adjoint à gauche. L'orthogonale à 
    gauche de la catégorie $\cB^\bot$ est égale à $\cB$. 
\end{enumerate}
\end{proposition}





\begin{proposition}\label{VIII:2-6-7}
Supposons vérifiées les propriétés des propositions \ref{VIII:2-6-5} et 
\ref{VIII:2-6-6}. Soient $i^\ast$ et $Q^\ast$ les adjoints à droite de $i$ et 
$Q$. De même soient $^\ast i^\bot$ et $^\ast Q^\bot$ les adjoints à gauche 
de $i^\bot$ et de $Q^\bot$. Tous ces foncteurs sont exactes. De plus: 

Il existe un isomorphisme fonctoriel 
$Q^\ast\circ Q\iso i^\bot\circ ^\ast i^\bot$ tel que le diagramme ci-après 
soit commutatif: 
\[\xymatrix{
  & \operatorname{id} \ar[dl] \ar[dr] \\
  Q^\ast \circ Q \ar[rr]^\sim 
    & & i^\bot\circ ^\ast i^\bot 
}\]
(les flèches obliques sont définies par les propriétés d'adjonction).

Il existe de même un isomorphisme 
$^\ast Q^\bot \circ Q^\bot \iso i\circ i^\ast$ tel que le diagramme suivant soit 
commutatif: 
\[\xymatrix{
  ^\ast Q^\bot\circ Q^\bot \ar[rr]^-\sim \ar[dr] 
    & & i\circ i^\ast \ar[dl] \\
  & \operatorname{id}{}
}\]

Enfin, il existe un morphisme fonctoriel de degré $1$: 
$\partial:i^\bot\circ ^\ast i^\bot \to i\circ i^\ast$ tel que le le triangle 
suivant soit distingué $(\deg\partial=1$): 
\[\xymatrix{
  i^\bot \circ ^\ast i^\bot \ar[rr]^-\partial 
    & & i\circ i^\ast \ar[dl] \\
  & \operatorname{id}{} \ar[ul] 
}\]

Un tel morphisme $\delta$ est unique et vérifie 
$\delta(T X_0) = - T \delta(X)$ ($X$ objet $q\circ q$, $T$ translation). 
\end{proposition}















\section{Les catégories dérivées d'une catégorie abélienne}\label{VIII:3}





\subsection{Les catégories dérivées}\label{VIII:3-1}

On utilise les notations du \ref{VIII:1-2-3}. On utilisera de plus les 
notations suivantes: 

\subsubsection{Notations}\label{VIII:3-1-1}

Soit $\cA$ une catégorie abélienne. On pose: 
\begin{align*}
  \D(\cA) &= \eK^{\infty,\infty}(\cA)/\eK^{\infty,\varnothing}(\cA) \\
  \D^b(\cA) &= \eK^{b,b}(\cA)/\eK^{b,\varnothing}(\cA) \\
  \D^+(\cA) &= \eK^{+,+}(\cA) / \eK^{+,\varnothing}(\cA) \\
  \D^-(\cA) &= \eK^{-,-}(\cA) / \eK^{-,\varnothing}(\cA) \text{.}
\end{align*}





\begin{proposition}\label{VIII:3-1-2}
Les foncteurs naturels qui figurent dans le diagramme suivant: 
\[\xymatrix@=0.5cm{
  & \D(\cA) \\
  \D^+(\cA) \ar[ur] 
    & & \D^-(\cA) \ar[ul] \\
  & \D^b(\cA) \ar[ul] \ar[ur] 
}\]
sont pleinement fidèles. 

Les foncteurs naturels: 
\begin{align*}
  \D^b(\cA) &= \eK^{b,b}(\cA)/\eK^{b,\varnothing}(\cA) \to \eK^{+,b}(\cA)/\eK^{+,\varnothing}(\cA) \\
  \D^b(\cA) &= \eK^{b,b}(\cA)/\eK^{b,\varnothing}(\cA) \to \eK^{-,b}(\cA) / \eK^{-,\varnothing}(\cA) 
\end{align*}
sont des équivalences de catégories. 

Les foncteurs du diagramme: 
\[\xymatrix@=0.5cm{
  & \D(\cA) \\
  \D^+(\cA) \ar[ur] 
    & & \D^-(\cA) \ar[ul] \\
  & \D^b(\cA) \ar[ul] \ar[ur]
}\]
étant injectifs sur les objets, réalisent les catégories $\D^b(\cA)$, 
$\D^+(\cA)$, $\D^-(\cA)$ comme sous-catégories pleines de $\D(\cA)$. 
\end{proposition}

La preuve de cette proposition se trouve au théorème \ref{VIII:2-4-2}. 





\begin{definition}\label{VIII:3-1-3}
La catégorie $\D(\cA)$ sera appelée la \emph{catégorie dérivée} de la 
catégorie $\cA$. Les objets de $\D(\cA)$ isomorphes aux objets de $\D^b(\cA)$ 
seront appelés les objets \emph{bornés} de $\D(\cA)$. Les objets de 
$\D(\cA)$ isomorphes aux objets de $\D^+(\cA)$ seront appelés les objets 
\emph{limités à gauche}, les objets de $\D(\cA)$ isomorphes aux objets de 
$\D^-(\cA)$, seront appelés les objets \emph{limités à droite}. 
\end{definition}

Nous désignerons par $D$ le foncteur canonique: 
\[
  D : \eC(\cA) \to \D(\cA) \text{.}
\]
La ``restriction'' de $D$ à la sous-catégorie pleine $\cA$ de $\eC(\cA)$ 
sera encore noté $D$. Le foncteur $D$ restrient à $\cA$ est pleinement 
fidèle. Le foncteur $D$ se factorise d'une manière unique par $\eK(\cA)$. 
Soient $X$ et $Y$ deux objets de $\cA$. 

On vérifie sans difficulté que pour tout $m\geqslant 0$, les groupes: 
\[
  \hom_{\D(\cA)}(D(X),T^{-m} D(Y)) \qquad \text{($T$ est le foncteur translation)} 
\]
sont nuls. La dimension cohomologique de $\cA$ sera le plus petit des entiers 
$n$ tels que pour tout $m>n$ on ait: 
\[
  \hom_{\D(\cA)}(D(X),T^m D(Y)) = 0 \qquad\text{pout tout couple $X,Y$ d'objets de $\cA$.} 
\]
S'il n'y a pas de tels entiers, on dira que la dimension cohomologique de $\cA$ 
est infinie. 

Désignons par $\eI^+(\cA)$ (resp. $\eI(\cA)$) la sous-catégorie triangulée 
pleine de $\eK^+(\cA)$ (resp. de $\eK(\cA)$) définie par les complexes dont 
les objets sont injectifs en tout degré. 





\begin{proposition}\label{VIII:3-1-4}
Supposons que la catégorie $\cA$ possède suffisamment d'injectifs. Le 
foncteur naturel: 
\[
  Q^+:\eK^+(\cA) \to \eD^+(\cA) 
\]
induit une équivalence de catégorie: 
\[
  \eI^+(\cA) \to \eD^+(\cA)  \text{.}
\]
Le foncteur quasi-inverse est un adjoint à droite de $Q^+$. 
\end{proposition}

Si de plus $\cA$ est de dimension cohomologique finie, l'assertion 
précédente est encore vraie lorsqu'on y supprime les exposants $+$. 

On a énoncé analogue concernant les projectifs. 

La démonstration de la proposition \ref{VIII:3-1-4} s'appuie sur 
\ref{VIII:2-6-4}. 

La structure triangulée de $\eD(\cA)$, est précisée par la proposition 
suivante. 

Soit $\eE(\cA)$ la catégorie des suites exactes à trois termes de 
$\eC(\cA)$. La catégorie $\mathsf{SSS}(\cA)$ des suites exactes 
semi-scindées (\S\ref{VIII:1-2-4}) est une sous-catégorie pleine de 
$\eE(\cA)$. On a défini (Loc. cit.) un foncteur: 
\[
  \rho:\mathsf{SSS}(\cA) \to \mathsf{TrK}(\cA) 
\]
où $\mathsf{TrK}(\cA)$ désigne la catégorie des triangles distingués 
de $\eK(\cA)$. On en déduit un foncteur 
\[
  \sigma : \mathsf{SSS}(\cA) \to \mathsf{TrD}(\cA) 
\]
où $\mathsf{TrD}(\cA)$ désigne la catégorie des triangles distingués 
de $\eD(\cA)$. 





\begin{proposition}\label{VIII:3-1-5}
Il existe un et un seul foncteur: 
\[
  \Pi:\eE(\cA) \to \mathsf{TrD}(\cA) 
\]
de la forme ($\deg \delta(S)=1$): 
\[
(S=0 \to X^\bullet \to Y^\bullet \to Z^\bullet \to 0) 
\mapsto 
\xymatrix{
  & D(Z^\bullet) \ar[dl]_-{\delta(S)} \\
  D(X^\bullet) \ar[rr]^-{D(u)} 
    & & D(Y^\bullet) \ar[ul]_-{D(v)} 
}
\]
dont la restriction à $\mathsf{SSS}(\cA)$ soit $\sigma$. \emph{Ce foncteur 
est essentiellement surjectif.}
\end{proposition}





\begin{proposition}\label{VIII:3-1-6}
Le foncteur cohomologique canonique: 
\[
  \h^0:\eK(\cA) \to \cA 
\]
se factorise d'une manière unique par un foncteur que nous noterns encore 
$\h^0$ 
\[
  \h^0:\eD(\cA) \to \cA \text{.}
\]
Sur $\eD(\cA)$ le foncteur $\h^0$ et ses translatés $\h^i$ forment un 
système conservatif, i.e. un morphisme $f:X^\bullet \to Y^\bullet$ et un 
isomorphisme si et seulement si pour tout entier $i$ 
\[
  \h^i(f):\h^i(X) \to \h^i(Y) 
\]
est un isomorphisme. 
\end{proposition}





\begin{definition}\label{VIII:3-1-7}
Un morphisme $f:X^\bullet \to Y^\bullet$ de $\eC(\cA)$ (resp. de $\eK(\cA)$) 
est appelé un \emph{quasi-isomorphisme} lorsque pour tout entier $i$ 
\[
  \h^i(f):\h^i(X) \to \h^i(Y) 
\]
est un isomorphisme. 
\end{definition}

D'après la proposition précédente les quasi-isomorphismes sont les 
morphismes qui deviennent des isomorphismes dans $\eD(\cA)$. 





\begin{definition}\label{VIII:3-1-8}
Soient $\sigma$ un ensemble d'objets de $\cA$, $X^\bullet$ un objet de 
$\eC(\cA)$ (resp. $\eK(\cA)$). Une re\'solution droite (reps. gauche) de type 
$\sigma$ de $X^\bullet$, est un quasi-isomorphisme $X^\bullet \to V^\bullet$ 
(resp. $V^\bullet \to X^\bullet$) où $V^\bullet$ est un complexe dont tous 
les objets appartiennent à $\sigma$. On dira \emph{résolution injective} 
(resp. projective) au lieu de résolution droite (resp. gauche) de type 
injectif (resp. projectif). 
\end{definition}

De manière générale, on se permettra de supprimer les mots ``droite'' ou 
``gauche'' lorsqu'aucune ambiguité n'en résultera. 





\begin{proposition}\label{VIII:3-1-9}
\begin{enumerate}
  \item Supposons que tout objet de $\cA$ s'injecte dans un objet de $\sigma$ 
    (resp. soit quotient d'un objet de $\sigma$). Alors tout objet de 
    $\eC^+(\cA)$ (resp. de $\eC^-(\cA)$) admet une résolution droite 
    (resp. gauche) de type $\sigma$ dans $\eC^+(\cA)$ (resp. dans 
    $\eC^-(\cA)$). 
  \item Supposons de plus qu'il existe un entier $n$ tel que pour toute suite 
    exacte: 
    \begin{align*}
      Y^0 &\to Y^1 \to \cdots \to Y^{n-1} \to Y^n \to 0 \\
      \text{(resp. } 0 &\to Y^n \to Y^{n-1} \to \cdots \to Y^1 \to Y^0 \text{)} 
    \end{align*}
    d'objets de $\cA$ ou pout tout $i$, $0\leqslant i\leqslant n-1$, $Y^i$ est 
    un objet de $\sigma$, l'objet $Y^n$ soit un objet de $\sigma$. Alors tout 
    objet de $\eC(\cA)$ admet une résolution droite (resp. gauche) de type 
    $\sigma$. 
\end{enumerate}
\end{proposition}










\subsection{Etude des $\ext$}\label{VIII:3-2}

Soient $X^\bullet$ et $Y^\bullet$ deux objets de $\eC(\cA)$ (resp. $\eK(\cA)$) 
et $D:\eC(\cA) \to \eD(\cA)$ (resp. $Q:\eK(\cA) \to \eD(\cA)$) le foncteur 
canonique. 





\begin{definition}\label{VIII:3-2-1}
On appelle \emph{$i$-ème hyper-ext} et on note: 
\[
  (X^\bullet,Y^\bullet)\mapsto \ext^i(X^\bullet,Y^\bullet) 
\]
le foncteur: $\hom_{\eD(\cA)}(D(X^\bullet),T^i D(Y^\bullet))$ (resp. 
$\hom_{\eD(\cA)}(Q(X^\bullet),T^i Q(Y^\bullet))$). 
\end{definition}





\begin{proposition}\label{VIII:3-2-2}
Soit $0 \to X^\bullet \to Y^\bullet \to Z^\bullet \to 0$ une suite exacte de 
$\eC(\cA)$. Soit $V^\bullet$ un objet de $\eC(\cA)$. On a les suites exactes 
illimitées: 
\[\xymatrix@=0.5cm{
  \cdots \ar[r] 
    & \ext^i(V^\bullet,X^\bullet) \ar[r] 
    & \ext^i(V^\bullet,Y^\bullet) \ar[r] 
    & \ext^i(V^\bullet,Z^\bullet) \ar[r]^-\delta 
    & \ext^{i+1}(V^\bullet,X^\bullet) \ar[r] 
    & \cdots \\
  \cdots \ar[r] 
    & \ext^i(Z^\bullet,V^\bullet) \ar[r] 
    & \ext^i(Y^\bullet,V^\bullet) \ar[r] 
    & \ext^i(X^\bullet,V^\bullet) \ar[r]^-\delta 
    & \ext^{i+1}(Z^\bullet,V^\bullet) \ar[r] 
    & \cdots
}\]
\end{proposition}

Soient $X^\bullet$ et $Y^\bullet$ deux objets de $\eK(\cA)$. On désigne par 
$\mathsf{Qis}/X^\bullet$ (resp. $\mathsf{Qis}^+/X^\bullet$, 
$\mathsf{Qis}^-/X^\bullet$, $\mathsf{Qis}^b/X^\bullet$) la catégorie des 
quasi-isomorphismes de but $X^\bullet$ et de source dans $\eK(\cA)$ (resp. 
$\eK^+(\cA)$, $\eK^-(\cA)$, $\eK^b(\cA)$). De même on définit les 
catégories $Y^\bullet/\mathsf{Qis}$, $Y^\bullet/\mathsf{Qis}^+$,\ldots 
(quasi-isomorphismes de source $Y^\bullet)$. 

On se propose de résumer dans la proposition suivante les différentes 
définitions équivalentes des $\ext^i$. 





\begin{proposition}\label{VIII:3-2-3}
1. Il existe des isomorphismes de foncteurs: 
\[\xymatrix{
  \ext^0(X^\bullet,Y^\bullet) = \hom_{\eD(\cA)}(Q(X^\bullet),Q(Y^\bullet)) \ar[r]^-\sim 
    & \varinjlim_{(\mathsf{Qis}/X^\bullet)^\circ} \hom_{\eK(\cA)}(-,Y^\bullet) \ar[r]^-\sim 
    & \\
  \varinjlim_{Y^\bullet/\mathsf{Qis}} \hom_{\eK(\cA)}(X^\bullet,-) \ar[r]^-\sim 
    & \varinjlim_{(\mathsf{Qis}/X^\bullet)^\circ \times Y^\bullet/\mathsf{Qis}} \hom_{\eK(\cA)}(-,-) \text{.}
}\]
Si $X^\bullet$ est un objet de $\eK^-(\cA)$: 
\[
  \ext^0(X^\bullet,Y^\bullet) \iso \varinjlim_{(\mathsf{Qis}^-/X^\bullet)^\circ} \hom_{\eK(\cA)}(-,Y^\bullet) 
\]
Si $Y^\bullet$ est un objet de $\eK^+(\cA)$: 
\[
  \ext^0(X^\bullet,Y^\bullet) \iso \varinjlim_{Y^\bullet/\mathsf{Qis}} \hom_{\eK(\cA)}(X^\bullet,-) 
\]
Si $X^\bullet$ est un objet de $\eK^-(\cA)$ et si $Y^\bullet$ est un 
objet de $\eK^b(\cA)$: 
\[
  \ext^0(X^\bullet,Y^\bullet) \iso \varinjlim_{Y^\bullet/\mathsf{Qis}^b} \hom_{\eK(\cA)}(X^\bullet,-) 
\]
Si $X^\bullet$ est un objet de $\eK^b(\cA)$ et $Y^\bullet$ un objet de 
$\eK^+(\cA)$: 
\[
  \ext^0(X^\bullet,Y^\bullet) \iso \varinjlim_{(\mathsf{Qis}^b/X^\bullet)^\circ} \hom_{\eK(\cA)}(-,Y^\bullet) 
\]
Si la catégorie $\cA$ possède suffisamment d'injectif et si 
$Y^\bullet$ est un objet de $\eK^+(\cA)$, $Y^\bullet$ admet une 
résolution injective et une seule (à isomorphisme près dans 
$\eK^+(\cA)$): 
\[
  Y^\bullet \to I(Y^\bullet) 
\]
et on a: 
\[
  \ext^0(X^\bullet,Y^\bullet) \iso \hom_{\eK(\cA)}(X^\bullet,I(Y^\bullet)) \text{.}
\]
On a de même un énoncé analogue pour les projectifs. 
Si la catégorie $\cA$ admet suffisamment d'injectifs et si $\cA$ est 
de \emph{dimension cohomologique finie}, tout objet $Y^\bullet$ de 
$\eK(\cA)$ admet une résolution injective et une seule: 
\[
  Y^\bullet \to I(Y^\bullet)
\]
et on a: 
\[
  \ext^0(X^\bullet,Y^\bullet) \iso \hom_{\eK(\cA)}(X^\bullet,I(Y^\bullet)) 
\]
Enoncé analogue pour les projectifs. 
\end{proposition}

\paragraph{Remarque:}
En explicitant l'isomorphisme de la proposition \ref{VIII:3-2-3}(3), on 
retrouve la définition de Yoneda. 










\section{Les foncteurs dérivés}\label{VIII:4}










\subsection{Définition des foncteurs dérivés}\label{VIII:4-1}





\begin{definition}\label{VIII:4-1-1}
Soient $\cC$ et $\cC'$ deux catégories graduées (on désigne par $T$ le 
foncteur de translation de $\cC$ et de $\cC'$), $F$ et $G$ deux foncteurs 
gradués de $\cC$ dans $\cC'$. Un \emph{morphisme de foncteurs gradués} et 
un morphisme de foncteurs: 
\[
  u:F\to G 
\]
qui possède la propriété suivante: 

Pour tout objet $X$ de $\cC$ le diagramme suivant est commutatif: 
\[\xymatrix{
  F(T X) \ar[r]^-{u(T X)} 
    & G(T X) \\
  T F(X) \ar[r]^-{T u(X)} \ar[u]_-\wr 
    & T G(X) \ar[u]_-\wr 
}\]
\end{definition}

Soient $\cC$ et $\cC'$ deux catégories triangulées. On désigne par 
$\mathsf{Fex}(\cC,\cC')$ la catégorie des foncteurs exacts de $\cC$ dans 
$\cC'$, le morphismes entre deux foncteurs étant les morphismes de foncteurs 
gradués. 

Soient $\cA$ et $\cB$ deux catégories abéliennes de 
$\Phi:\eK^\ast(\cA)\to \eK^{\ast'}(\cB)$ un foncteur exact ($\ast$ et $\ast'$ 
désignent l'un des signes $+$, $-$, $b$, ou $v$ ``vide''). Le foncteur 
canonique $Q:\eK^\ast(\cA)\to \eD^\ast(\cA)$ nous donne, par composition, un 
foncteur 
\[\xymatrix{
  \mathsf{Fex}(\eD^\ast(\cA),\eD^{\ast'}(\cB)) \ar[r] 
    & \mathsf{Fex}(\eK^\ast(\cA),\eD^{\ast'}(\cB)) 
}\]
d'où (en désignant aussi par $Q'$ le foncteur canonique 
$\eK^{\ast'}(\cB) \to \eD^{\ast'}(\cB)$) un foncteur: 
$\%$ (resp. $\%'$): 
$\mathsf{Fex}(\eD^\ast(\cA),\eD^{\ast'}(\cB)) \to \mathsf{Ab}$: 
\begin{align*}
  \Psi  &\mapsto \hom(Q'\circ \Phi,\Psi\circ Q) \\
  (\text{resp. }\Psi &\mapsto \hom(\Psi\circ Q,Q'\circ \Phi)\text{ ).} 
\end{align*}





\begin{definition}\label{VIII:4-1-2}
On dira que $\Phi$ admet un \emph{functeur dérivé total} à droite (resp. 
à gauche) si le foncteur $\%$ (resp. $\%'$) est représentable. Un objet 
représentant le foncteur $\%$ (resp. $\%'$) ser appelé foncteur dérivé 
total à droite (resp. à gauche) de $\Phi$ et sera noté 
$\eR \Phi$ (resp. $\eL \Phi$).\footnote{Une définition plus maniable (et en 
practique équivalente) est donnée dans \cite[XVII 1.2]{sga4}.}
\end{definition}





\subsubsection{Notations}\label{VIII:4-1-3}

Soient $\cA$ et $\cB$ deux catégories abéliennes et $f:\cA\to \cB$ un 
foncteur additif. Le \emph{foncteur dérivé total à droite} du foncteur 
$\eK^+(f):\eK^+(\cA) \to \eK^+(\cB)$ sera noté $\eR^+ f$. On définit de 
même $\eR^- f$, $\eR f$, $\eL^+ f$, $\eL^- f$, $\eL f$. 

Soit de même $F:\eK^\ast(\cA) \to \cB$, un foncteur cohomologique. Le 
foncteur canonique de passage au quotient $Q:\eK^\ast(\cA) \to \eD^\ast(\cA)$ 
définit par composition un foncteur: 
\[\xymatrix{
  \mathsf{Foco}(\eD^\ast(\cA),\cB) \ar[r] 
    & \mathsf{Foco}(\eK^\ast(\cA),\cB) 
}\]
($\mathsf{Foco}(-,-)$ désigne la catégorie des foncteurs cohomologiques) 
d'où un foncteur $\%$ (resp. $\%'$) 
$\mathsf{Foco}(\eD^\ast(\cA),\cB) \to \mathsf{Ab}$, $G\mapsto \hom(F,G\circ Q)$ 
(resp. $G\mapsto \hom(G\circ Q,F)$). 





\begin{definition}\label{VIII:4-1-4}
On dira que le foncteur $F$ admet un \emph{foncteur dérivé cohomologique} 
à droite (resp. à gauche) si le foncteur $\%$ (resp. $\%'$) est 
représentable. Un objet représentant le foncteur $\%$ (resp. $\%'$) sera 
appelé foncteur dérivé cohomologique à droite (resp. à gauche) et 
sera noté $\eR F$ (resp. $\eL F$). 
\end{definition}





\subsubsection{Notations}\label{VIII:4-1-5}

Soient $\cA$ et $\cB$ deux catégories abéliennes et $f:\cA\to \cB$ un 
foncteur additif. Le foncteur dérivé cohomologique à droite du foncteur: 
\[
  \h^0{}\circ \eK^+ f:\eK^+(\cA) \to \cB 
\]
sera noté $\eR^+ f$. Lorsqu'aucune confusion n'en résultera le foncteur 
$\eR^+ f\circ T^i$ ($T$ est le foncteur de translation) sera noté: 
$\eR^i f$. Le foncteur dérivé cohomologique à droite du foncteur 
$\h{}\circ \eK f$ sera noté $\eR f$. Lorsqu'aucune confusion n'en résultera 
le foncteur $\eR f\circ T^i$ sera noté $\eR^i f$. 

On définit de même les foncteurs $\eL^- f$, $\eL f$, $\eL^i f$. 





\subsubsection{Remarques}\label{VIII:4-1-6}

Ces définitions contiennent, ainsi que nous le verrons dans le numéro 
suivant, la définition classique des foncteurs dérivés lorsque les 
catégories ont suffisamment d'injectifs ou de projectifs et que les complexes 
sont convenablement limités. Elles sont cependant plus générales. Mais 
sans autres hypothèses elles sont peu maniables. Par exemple, soit 
$f:\cA\to \cB$ un foncteur additif entre deux catégories abéliennes. 
Supposons que les foncteurs $\eR^+ f$ et $\eR f$ existent. Je ne sais pas 
d\'montrer que $\eR f$ induit sur $\eD^+(\cA)$ le foncteur $\eR^+ f$. Supposons 
de plus que le foncteur $\eR^+ f$ (dérivé cohomologique) existe. Je ne sais 
pas démontrer que le foncteur $\h^0{}\circ \eR^+ f$ est isomorphe au foncteur 
$\eR^+ f$. Cependant dans la pratique ces propriétés seront vérifées. 










\subsection{Existence des foncteurs dérivés}\label{VIII:4-2}





\begin{proposition}\label{VIII:4-2-1}
Les hypothèses sont celles de la définition \ref{VIII:4-1-4}. Supposons 
en outre que $\cA$ soit un $\sU$-catégorie, où $\sU$ est un universe tel 
que l'ensemble $\dZ$ des entiers rationnel soit un élément de $\sU$, telle 
que l'ensemble des objets de $\cA$ soit un élément de $\sU$. Supposons 
de plus que la catégorie $\cB$ soit la catégorie des $\sU$-groupes 
abéliens ou plus générelement la catégorie des faisceaux de 
$\sU$-groupes abéliens sur un $\sU$-site quelconque. (Un $\sU$-site est une 
$\sU$-catégorie, dont l'ensemble des onjets est un élément de $\sU$, 
munie d'une topologie.) Le foncteur $F$ admet un foncteur dérivé 
cohomologique à droite. Ce foncteur se calcule de la manière suivante: 

Soient $X$ un objet de $\eD^\ast(\cA)$, $\underline X$ l'objet de 
$\eK^\ast(\cA)$ au-dessus de $X$. On a alors un isomorphisme fonctoriel: 
\[
  \eR F(X) = \varinjlim_{\underline X/\mathsf{Qis}^\ast} F(-) \text{.}
\]
\end{proposition}

En particulier soit $f:\cA\to \cB$ un foncteur additif. Les foncteurs 
$\eR f$ et $\eR^+ f$ existent et se calculent de la manière indiquée 
ci-dessus. Le foncteur $\eR f$ induit sur $\eD^+(\cA)$ le foncteur 
$\eR^+ f$. Comme aucune confusion ne peut alors en résulter, nous 
emploierons la notation $\eR^i f$. 





\subsubsection{Remarques}\label{VIII:4-2-2}

On peut exprimer la proposition \ref{VIII:4-2-1} sous une forme plus 
générale dans le cadre des catégories triangulées. Les hypothèses 
à faire sur la catégorie $\cB$, pour assurer la validité de la 
proposition \ref{VIII:4-2-1} sont des hypothèses d'existence et d'exactitude 
des limites inductives suivant les $\sU$-catégories pseudo-filtrantes, dont 
les ensembles d'objet sont éléments de $\sU$. La proposition 
\ref{VIII:4-2-1} introduit une dissymétrie entre les foncteurs dérivés 
droits et les foncteurs dérivés gauches tout au moins dans la pratique. 





\begin{theorem}\label{VIII:4-2-3} % labelled 4.2.2 in original
Les données sont celles de la définition \ref{VIII:4-1-2}. On se donne de 
plus une sous-catégorie triangulée pleine de $\eK^\ast(\cA)$: $\cC$, et on 
d\'signe par $\cN$ la sous-catégorie triangulée pleine des objets de $\cC$ 
qui sont acycliques. Supposons que 
\begin{enumerate}[(1)]
  \item Tout objet de $\cN$ soit transformé par $\Phi$ en objet acyclique 
    de $\eK^{\ast'}(\cB)$. 
  \item Tout objet $X$ de $\eK^\ast(\cA)$ soit source (resp. but) d'un 
    quasi-isomorphisme dont le but (resp. la source) soit un objet de $\cC$. 
\end{enumerate}
\begin{enumerate}[a)]
  \item Alors $\Phi$ admet un foncteur dérivé total à droite (resp. 
    à gauche). 
  \item Le fonteur $\eR \Phi$ (resp. $\eL \Phi$) s'obtient de la manière 
    suivante: 
    
    La restriction du foncteur 
    $Q'\circ \Phi:\eK^\ast(\cA) \to \eD^{\ast'}(\cB)$ à la catégorie $\cC$ 
    s'annule sur les objets de $\cN$, donc se factorise par $\cC/\cN$. On 
    obtient ainsi un foncteur $\Phi':\cC/\cN\to \eD^{\ast'}(\cB)$. Mais le 
    foncteur naturel: $\cC/\cN\to \eD^\ast(\cA)$ est une équivalence de 
    catégories (th\'or\`me \ref{VIII:2-4-2}). D'où en composant avec le 
    foncteur quasi-inverse le foncteur $\eR\Phi$ (resp. $\eL \Phi$). 
  \item Soit $X$ un objet de $\eK^\ast(\cA)$. Soient $Y$ un objet de $\cC$ et 
    $X\to Y$ (resp. $Y\to X$) un quasi-isomorphisme. L'objet 
    $\eR\Phi\circ Q(X)$ (resp. $\eL \Phi\circ Q(X)$) est isomorphe à 
    $Q\circ \Phi(Y)$. 
  \item Le foncteur $X\mapsto \eR\Phi\circ Q(X)$ (resp. 
    $X\mapsto \eL \Phi\circ Q(X)$) est isomorphe au foncteur: 
    $X\mapsto \varinjlim_{X/\mathsf{Qis}^\ast} Q\circ\Phi(-)$ (resp. 
    $X\mapsto \varinjlim_{\mathsf{Qis}^\ast/X} Q\circ \Phi(-)$). 
  \item Le foncteur $\h^0{}\circ \Phi:\eK^\ast(\cA)\to \cB$, admet un 
    foncteur dérivé cohomologique à droite (resp. à gauche) qui n'est 
    autre que le foncteur $\h^0{}\circ\eR\Phi$ (resp. $\h^0{}\circ \eL\Phi$). 
\end{enumerate}
\end{theorem}





\begin{corollary}\label{VIII:4-2-3-1} % originally corollary 1
Supposons que $\ast=+$ (resp. $\ast=-$) dans les hypothèses du théorème 
précédent, et que $\cA$ possède suffisamment d'injectifs (resp. de 
projectifs). Alors le théorème \ref{VIII:4-2-3} s'applique en prenant pour 
$\cC$ la catégorie des complexes injectifs (resp. projectifs). Supposons de 
plus que $\cA$ soit de dimension cohomologique finie, alors le théorème 
s'applique, sans restriction sur $\ast$, en prenant sur $\cC$ la catégorie 
des complexes injectifs (resp. projectifs). 
\end{corollary}





\begin{definition}\label{VIII:4-2-5} % originally 2.4
Soient $\cA$ et $\cB$ deux catégories abéliennes et $f:\cA\to \cB$ un 
foncteur additif. 

On dira que $f$ est de \emph{dimension cohomologique finie} à droite (resp. 
à gauche) si $\eR^+ f$ (resp. $\eL^- f$) existe et s'il existe un entier 
$m\geqslant 0$ tel que pour tout entier $n>m$, $\eR^n f$ (resp. 
$\eL^{-n} f$) soit nul sur les objets provenant de $\cA$ par le foncteur $D$. 
La \emph{dimension cohomologique} à droite (resp. à gauche) de $f$ est 
alors le plus petit des entiers $m$ possédant la propriété ci-dessus. Si 
$\eR^+ f$ (resp. $\eL^- f$) existe et si $f$ n'est pas de dimension 
cohomologique finie, on dira que $f$ est de dimension cohomologique infinie. 

Un objet $X$ de $\cA$ sera dit \emph{$f$-acyclique} à droite (resp. à 
gauche) si pour tout entier $n>0$, $\eR^n f(D(X)) = 0$ (resp. 
$\eL^{-n} f(D(X))=0$). 
\end{definition}





\begin{corollary}\label{VIII:4-2-3-2} % originally corollary 2
Les données sont celles de la définition \ref{VIII:4-2-5}. Supposons qu'il 
existe un ensemble $\sigma$ d'objets de $\cA$ stable par somme directe 
possédant les propriétés suivantes: 
\begin{enumerate}
  \item Tout complexe acyclique $\in \operatorname{Ob}{\eK^+(\cA)}$ (resp. 
    $\in\operatorname{Ob}{\eK^-(\cA)}$) d'objets de $\sigma$, est transformé 
    par $f$ en complexe acyclique. 
  \item Tout objet $X$ de $\cA$ s'injecte dans (resp. est quotient de) un objet 
    de $\sigma$. Les hypothèses du théorème \ref{VIII:2-4-3} sont 
    vérifiées en prenant $\ast=+$ (resp. $\ast=-$) et en prenant pour 
    catégorie $\cC$, la catégorie des complexes dont les objets et tout 
    degré sont des éléments de $\sigma$. Les éléments de $\sigma$ 
    sont des objets $f$-acycliques à droite (resp. à gauche). 
\end{enumerate}

Si de plus $f$ est de dimension cohomologique finie à droite (resp. à 
gauche), l'ensemble des objets $f$-acycliques à droite (resp. à gauche) 
[pssède en plus des propriétés (1) et (2) ci-dessus, la propriété 
(2) de proposition \ref{VIII:3-1-9}. Le Théorème \ref{VIII:4-2-3} 
s'applique alors au foncteur $\eK f$ en prenant pour catégorie $\cC$ la 
catégorie des complexes dont les objets sont $f$-acycliques à droite 
(resp. à gauche) en tout degré. Le foncteur $\eR f$ (resp. $\eL f$) existe 
et induit sur $\eD^+(\cA)$ le foncteur $\eR^+ f$ (resp. $\eL^+ f$) et sur 
$\eD^-(\cA)$ le foncteur $\eR^- f$ (resp. $\eL^- f$). 
\end{corollary}










\subsection{Produit. Composition}\label{VIII:4-3}

Soient $\cA$ une catégorie abélienne, $F:\eD(\cA) \to \mathsf{Ab}$ un 
foncteur cohomologique. Soient $X^\bullet$ et $Y^\bullet$ deux objets de 
$\eC(\cA)$. Le foncteur $F$ définit une application: 
\[
  \ext^0(X^\bullet,Y^\bullet) \to \hom_{\mathsf{Ab}}(F(D(X^\bullet)),F(D(Y^\bullet)))
\]
d'où un accouplement: 
$F(D(X^\bullet))\times \ext^0(X^\bullet,Y^\bullet) \to F(D(Y^\bullet))$. 

Cet accouplement donne, en appliquant des translations, des accouplements: 
\[
  F^i(D(X^\bullet))\times \ext^j(X^\bullet,Y^\bullet) \to F^{i+j}(D(Y^\bullet)) \text{.}
\]
Ces accouplements, appliqués aux foncteurs dérivés cohomologiques, ne 
sont autres, par définition, que les produits de Yoneda. Les propriétés 
de ces produits se déduisent immédiatement de cette définition. Nous ne 
les développerons pas ici. 

Soient $\cA$, $\cA'$, $\cA''$ trois catégories abéliennes et 
$f:\cA\to \cA'$, $g:\cA'\to \cA''$ deux foncteurs additifs. Supposons que les 
foncteurs $\eR^+ f$ et $\eR^+ g$ existent. Je ne sais pas en déduire que le 
foncteur $\eR^+ g\circ f$ existe et même si ce dernier foncteur existe, il 
n'est probablement pas vraie en général que l'on ait la formule: 
\[\xymatrix{
  \eR^+ g\circ f\ar[r]^-\sim 
    & \eR^+ g \circ \eR^+ f 
}\]
On a cependant la proposition suivante: 





\begin{proposition}\label{VIII:4-3-1}
Supposons que la catégorie $\cA'$ possède suffisamment d'objets 
$g$-acycliques à droite et que la catégorie $\cA$ possède suffisamment 
d'objets $f$-acycliques à droite transformés par $f$ en objets 
$g$-acycliques à droite. Alors le foncteur $\eR^+ g\circ f$ existe et on a un 
isomorphisme: 
\[\xymatrix{
  \eR^+ g\circ f \ar[r]^-\sim 
    & \eR^+ g \circ \eR^+ f \text{.}
}\]
\end{proposition}

Enoncé analogues pour les foncteurs dérivés gauches. 















\section{Exemples}\label{VIII:6} % originally chapter 2, section 3










\subsection{Le foncteur \texorpdfstring{$\mathsf{Rhom}^\bullet$}{Hom*}}\label{VIII:6-1}

Soit $\cA$ une catégorie abélienne, possédant suffisamment d'injectifs. 

Le foncteur: $Y\mapsto \hom^\bullet(X^\bullet,Y^\bullet)$ (\ref{VIII:1-3}) 
où $X^\bullet$ est un objet de $\eK(\cA)$ et $Y^\bullet$ un objet de 
$\eK^+(\cA)$, admet un foncteur dérivé total à droite 
(Corollaire \ref{VIII:4-2-3-1}) qui sera noté $\rhom^\bullet(X^\bullet,-)$.  
Soit $Y$ un objet de $\eD^+(\cA)$ le foncteur 
\begin{align*}
  \eK(\cA) &\to \eD(\mathsf{Ab}) \\
  X^\bullet &\mapsto \rhom^\bullet(X^\bullet,Y) 
\end{align*}
est exact et s'annule sur les complexes acycliques, d'où un bi-foncteur 
exact 
\[\xymatrix{
  \eD(\cA)^\circ \times \eD^+(\cA) \ar[r] 
    & \eD(\mathsf{Ab}) 
}\]
que nous désignerons par $\mathsf{Rhom}^\bullet$. 

Lorsque $\cA$ est de dimension cohomologique finie, le foncteur 
$\mathsf{Rhom}^\bullet$ se prolonge à $\eD(\cA)^\circ\times \eD(\cA)$ 
(loc. cit.). 

Si en outre $\cA$ possède suffisamment de projectifs on peut dériver le 
foncteur $\hom$ par rapport au premier argument. Les deux définitions 
co\"incident dans leur domaine commun de validité. 










\subsection{\texorpdfstring{$\Tor$}{Tor} de faisceaux}\label{VIII:6-2}

Soient $E$ un site annelé (en particulier un espace topologique annelé) 
$\sA$ le faisceau d'anneaux et $\mathsf{Mod}(E)$ la catégorie des 
$\sA$-modules. Un objet $X$ de $\mathsf{Mod}(E)$ est dit 
\emph{$\sA$-plat} si pout toute suite exacte: 
\[\xymatrix{
  0 \ar[r] 
    & Y' \ar[r] 
    & Y \ar[r] 
    & Y'' \ar[r] 
    & 0 
}\]
la suite: $0 \to Y'\otimes_\sA X \to Y\otimes_\sA X \to Y''\otimes_\sA X\to 0$ 
est exacte. Il existe suffisamment d'objets $\sA$-plats: les sommes directes 
d'objets $\sA_U$, où $U$ est un objet de $E$ et $\sA_U$ le faisceau 
$\sA$ ``prolongué par zéro en dehors de $U$.'' Le théorème 
\ref{VIII:4-2-3} s'applique donc et on peut définir le foncteur 
\begin{align*}
  \eD^-(E,\sA) \times \eD^-(E,\sA) &\to \eD^-(E,\sA) \\
  (X,Y) &\mapsto X\lotimes Y 
\end{align*}
produit tensoriel total, dérivé gauche du foncteur produit tensoriel. 
On définit ainsi en passant à la cohomologie les $\Tor_i^\sA(X,Y)$: 
hyper-tor locaux. 

La catégorie $\mathsf{Mod}(E)$ ayant suffisamment d'injectifs, on peut 
dériver le foncteur: $\hom^\bullet(X,Y)$: complexe des homomorphismes locaux,  
d'où $\mathsf{Rhom}(X,Y)$ ($X\in \eD$, $Y\in \eD^+$). 

Soient $X$ et $Y$ deux objets de $\eD^-(E,\sA)$, $Z$ un objet de 
$\eD^+(E,\sA)$. Il existe un isomorphisme fonctoriel: 
\[\xymatrix{
  \hom_{\eD(E,\sA)}(X\lotimes Y,Z) \ar[r]^-\sim 
    &\hom_{\eD(E,\sA)}(X,\mathsf{Rhom}(Y,Z)) \text{.}
}\]










\subsection{Foncteur dérivé gauche de l'image réciproque}\label{VIII:6-3}

Soit $f:E\to E'$ une application d'espaces topologiques annelés. Le 
foncteur image réciproque: 
\[
  f^\ast:\mathsf{Mod}(E') \to \mathsf{Mod}(E) 
\]
n'est pas nécessairement exact. Mais on sait définir des objets 
$f^\ast$-plats et il existe suffisamment de tels objets. On sait donc 
définir le foncteur $\eL^- f^\ast$. On a de même une formule de dualité: 

Soient $X$ un objet de $\eD^-(E,\sA)$ et $Y$ un objet de 
$\eD^+(E',\sA')$; il existe un isomorphisme fonctoriel: 
\[\xymatrix{
  \hom_{\eD(E',\sA')}(\eL^- f^\ast(X),Y) \ar[r]^-\sim 
    & \hom_{\eD(E,\sA)}(X,\eR^+ f_\ast(Y)) \text{.}
}\]






















\appendix
\chapter{Erratum pour SGA 4, tome 3}

\paragraph{XIV p.18 1.14}
(XIX 6) au lieu de (XX 6). 

\paragraph{XVI 2.2}
Il faut supposer $F$ localement constant!

\paragraph{XVI 5.2}
La démonstration donnée est incomplète. Arpès l'énoncé de 
l'hypothèse de récurrence, il faut d'abord se ramener au cas o\`u $F$ est 
constant ($F$ devient constant sur un rev\^etement galoisien étale 
$\pi:X'\to X$ de $X$, de groupe de Galois $G$, et on invoque la suite spectrale 
de Hochschild-Serre $E_2^{p q}=\h^p(G,\h^q(X',\pi^\ast F))\Rightarrow \h^{p+q}(X,F)$). 
Les arguments qui suivent sont alors corrects. 

\paragraph{XVII 1.1.8}
Le signe (1.1.8.1) est erroné lorsque $F$ est contravariant en certaines 
variables. Il faut lire: 
\begin{equation*}\tag{1.1.8.1}
  \rho^{\underline k} = (-1)^{A(\underline k)}: \text{ automorphisme de }
F\circ (G_j)\left(K_i^{k_i\varepsilon(i)\varepsilon(\psi(i))}\right) 
\end{equation*}
avec 
\[
  A(\underline k) = \sum_{\varepsilon(i) = \varepsilon(\psi(i))=-} k_i + \sum_{\varepsilon(j)=-} \sum_{\substack{\psi(a)=\psi(b)=j \\ a<b}} k_a k_b 
\]

\paragraph{XVII 2.1.3}
La démonstration contient des erreurs flagrantes. Il faut supprimer la 
3-ème ligne (p.34 1.9) et remplacer les 6-ème et 7-ème (p.34 1.12, 13) 
par: 

Les flèches du diagramme (2.1.3.2) induisent des applications 
\[
  \hom_y(F,F) \to \hom_{f g'}({g'}^\ast F,f_\ast F) = \hom_{x'}(f'_\ast {g'}^\ast F,g^\ast f_\ast F) \text{,}
\]

\paragraph{XVIII 2.14.4}
Lire XVII 6.2.7.2 au lieu de XVII 6.2.4.

\paragraph{XVIII p.99 1-1}
Lire $u$ au lieu de $U$ et 3.1.16.1 au lieu de 3.1.11.1.






% to use BibTeX
\bibliographystyle{amsplain}
\bibliography{sga-sources}






\end{document}
