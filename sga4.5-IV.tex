\chapter{La classe de cohomologie associée à un cycle\texorpdfstring{\footnote{par A. Grothendieck, r\'edig\'e par P. Deligne}}{}}\label{IV}




Cet expos\'e est inspir\'e de notes de Grothendieck, qui formaient un \'etat 0 
de \cite[IV]{SGA5}. On y d\'efinit la classe de cohomologie d'un cycle $X$ dans 
un sch\'ema s\'epar\'e lisse de type fini sur un corps et on prouve que 
l'intersection correspond au cup-produit.

Le Chapitre 1 contient quelques sorites g\'en\'eraux. Au Chapitre 2, on 
d\'efinit la classe d'un cycle, dans plusieurs situations plus g\'en\'erales 
que celle dite plus haut. La princnipale compatibilit\'e non consid\'er\'ee est 
celle entre image directe d'un cycle et morphisme trace en cohomologie. Au 
Chapitr 3, on d\'eduit de ce formalisme la formule des traces de Lefschetz pour 
un endomorphisme \`a points fixes isol\'es d'un sch\'ema propre et lisse sur $k$ 
alg\'ebriquement clos -- et pour un endomorphisme de Frobenius d'une courbe. 

Nous faisons les conditions suivantes:
\begin{enumerate}[\indent 1)]
  \item ``sch\'ema'' signifie sch\'ema noeth\'erien s\'epar\'e (ceci est 
    largement un hypoth\`ese de commodit\'e).
  \item Dans le Chapitre 2, on fixe un entier $n$, et $n$ est inversible sur 
    tous les sch\'emas consid\'er\'es.
  \item Dans le Chapitre 3, on fixe un nombre premier $\ell$, et $\ell$ est 
    inversible sur tout les sch\'emas consid\'er\'es. La cohomologie utilis\'ee 
    est toujours la cohomologie $\ell$-adique:
    \[
      \h^\bullet(X) = \dQ_\ell\otimes_{\dZ_\ell} \varprojlim \h^\bullet(X,\dZ/\ell^n) \text{.}
    \]
\end{enumerate}
Classes de cohomologie de cycles, morphismes traces, \ldots sont d\'efinis par 
passage \`a limite \`a partir du cas de coefficients finis $\dZ/\ell^n$ (cf. 
\cite[VI]{SGA5}).




















\section{Cohomologie \`a support et cup--produits}\label{IV:1}

Ce paragraphe contient des rappels de topologie g\'en\'erale, que le 
lecteur est invit\'e \`a ne consulter qu'au fur et \`a mesure des besoins.










\subsection{\texorpdfstring{$H^1$}{H1} et torseurs}\label{IV:1-1}





\subsubsection{}\label{IV:1-1-1}

Soit $\cF$ un faisceau ab\'elien sur un site $X$. On sait que $\h^1(X,\cF)$ 
classifie les $\cF$-torseurs sur $X$. Nous normaliserons ($=$choisirons le 
signe) de l'isomorphisme $($ensemble des classes d'isomorphisme de 
$F$-torseurs$)\to \h^1(X,\cF)$ de telle sorte que pour toute suite exacte 
$0\to \cF\xrightarrow\alpha\cG\xrightarrow\beta\cH\to 0$ et tout 
$h\in\h^0(X,\cH)$, le $\cF$-torseur $\beta^{-1}(h)\subset \cG$, sur lequel 
$F$ agit par $(f,x)\mapsto \alpha(f)+x$, soit de classe $\partial h$. 





\section{Application: la formule des traces de Lefschetz dans le cas propre et lisse}\label{IV:3}

\begin{corollary_}\label{IV:3-7}
foo
\end{corollary_}