% !TEX root = sga4.5.tex

\chapter{La classe de cohomologie associée à un cycle}\label{IV}
\blfootnote{par A.\ Grothendieck, r\'edig\'e par P.\ Deligne}




Cet expos\'e est inspir\'e de notes de Grothendieck, qui formaient un \'etat 0 
de \cite[IV]{sga5}. On y d\'efinit la classe de cohomologie d'un cycle $X$ dans 
un sch\'ema s\'epar\'e lisse de type fini sur un corps et on prouve que 
l'intersection correspond au cup-produit.

Le Chapitre 1 contient quelques sorites g\'en\'eraux. Au Chapitre 2, on 
d\'efinit la classe d'un cycle, dans plusieurs situations plus g\'en\'erales 
que celle dite plus haut. La princnipale compatibilit\'e non consid\'er\'ee est 
celle entre image directe d'un cycle et morphisme trace en cohomologie. Au 
Chapitr 3, on d\'eduit de ce formalisme la formule des traces de Lefschetz pour 
un endomorphisme \`a points fixes isol\'es d'un sch\'ema propre et lisse sur $k$ 
alg\'ebriquement clos -- et pour un endomorphisme de Frobenius d'une courbe. 

Nous faisons les conditions suivantes:
\begin{enumerate}[\indent 1)]
  \item ``sch\'ema'' signifie sch\'ema noeth\'erien s\'epar\'e (ceci est 
    largement un hypoth\`ese de commodit\'e).
  \item Dans le Chapitre 2, on fixe un entier $n$, et $n$ est inversible sur 
    tous les sch\'emas consid\'er\'es.
  \item Dans le Chapitre 3, on fixe un nombre premier $\ell$, et $\ell$ est 
    inversible sur tout les sch\'emas consid\'er\'es. La cohomologie utilis\'ee 
    est toujours la cohomologie $\ell$-adique:
    \[
      \h^\bullet(X) = \dQ_\ell\otimes_{\dZ_\ell} \varprojlim \h^\bullet(X,\dZ/\ell^n) \text{.}
    \]
\end{enumerate}
Classes de cohomologie de cycles, morphismes traces, \ldots sont d\'efinis par 
passage \`a limite \`a partir du cas de coefficients finis $\dZ/\ell^n$ (cf. 
\cite[VI]{sga5}).




















\section{Cohomologie \`a support et cup--produits}\label{IV:1}

Ce paragraphe contient des rappels de topologie g\'en\'erale, que le 
lecteur est invit\'e \`a ne consulter qu'au fur et \`a mesure des besoins.










\subsection{\texorpdfstring{$H^1$}{H1} et torseurs}\label{IV:1-1}





\subsubsection{}\label{IV:1-1-1}

Soit $\cF$ un faisceau ab\'elien sur un site $X$. On sait que $\h^1(X,\cF)$ 
classifie les $\cF$-torseurs sur $X$. Nous normaliserons ($=$choisirons le 
signe) de l'isomorphisme $($ensemble des classes d'isomorphisme de 
$F$-torseurs$)\to \h^1(X,\cF)$ de telle sorte que pour toute suite exacte 
$0\to \cF\xrightarrow\alpha\cG\xrightarrow\beta\cH\to 0$ et tout 
$h\in\h^0(X,\cH)$, le $\cF$-torseur $\beta^{-1}(h)\subset \cG$, sur lequel 
$F$ agit par $(f,x)\mapsto \alpha(f)+x$, soit de classe $\partial h$. 





\subsubsection{}\label{IV:1-1-2}

Soit $\cP$ un $\cF$-torseur. Si $(U_i)$ est un recouvrement ouvert de $X$, et 
$p_i$ un section de $\cP$ sur $U_i$, on associe \`a $\cP$ le cocycle de 
\v{C}ech
\[
  p_{i j} = p_j - p_i \qquad \text{($p_{i j}\in \h^0(U_i\times U_j,\cF)$).}
\]
Si, selon la r\`egle usuelle, on d\'efinit l'application 
$\operatorname{\check H}^\bullet(X,\cF)\to \h^\bullet(X,\cF)$ de telle sorte 
que ce soit un morphisme de $\delta$-foncteurs, l'image de 
$(p_{i j})\in \operatorname{\check H}^1(X,\cF)$ dans $\h^1(X,\cF)$ est la 
classe de $\cP$, telle que normalis\'ee par \ref{IV:1-1-1}. 





\subsubsection{}\label{IV:1-1-3}

La d\'efinition de $\h^\bullet(X,\cF)$ est la suivante: pour $\cF^\bullet$ une 
r\'esolution \`a composantes acycliques de $\cF$, 
$\h^i(X,\cF) = \h^i\Gamma(X,\cF^\bullet)$. La structure de $\delta$-functor 
s'obtient en associant \`a une suite exacte courte de faisceaux une suite 
exacte courte de r\'esolutions qui reste exacte apr\`es application du 
foncteur $\Gamma$. Si $\cF^\bullet$ est une r\'esolution de $\cF$, 
l'homomorphisme de connection $\partial$ associ\'e \`a 
\[\xymatrix{
  0 \ar[r]
    & \cF \ar[r]
    & \cF^0 \ar[r]^-d 
    & \ker(d) \ar[r] 
    & 0
}\]
induit l'oppos\'e de l'isomorphisme de d\'efinition 
$\h^1(X,\cF) = \Gamma\left(X,\ker(d)\right)/d \Gamma(X,\cF^0)$. 





\subsubsection{}\label{IV:1-1-4}

Soit $U$ une partie ouverte de $X$ (un sous-faisceau du faisceau final) et 
soit $D$ le ``ferm\'e compl\'ementaire.'' On sait que $\h_D^1(X,\cF)$ 
classifie les $\cF$-torseurs sur $X$, trivialis\'es sur $U$. Pour toute suite 
exacte courte $0\to \cF\xrightarrow\alpha \cG\xrightarrow\beta\cH\to 0$, et 
toute section \`a support dans $D$ $h\in\h_D^0(X,\cH)$, le torseur 
$\beta^{-1}(h)$, trivialis\'e sur $U$ par la section $0$, a pour classe 
$\partial h$.

La suite exacte longue de cohomologie \`a support
\[\xymatrix{
  \cdots \ar[r]^-\partial 
    & \h_D^i(X,\cF) \ar[r] 
    & \h^i(X,\cF) \ar[r]
    & \h^i(U,\cF) \ar[r]^-\partial 
    & \cdots
}\]
est d\'efinie \`a partir de la suite de foncteurs 
\[\xymatrix{
  0 \ar[r] 
    & \Gamma_D \ar[r] 
    & \Gamma \ar[r] 
    & \Gamma(U,-) \ar[r] 
    & 0
}\]
(exacte sur les faisceaux injectifs). Pour toute section $f\in \h^0(U,\cF)$, 
$\partial f\in \h_D^1(X,\cF)$ est la classe du torseur trivial $\cF$, 
trivialis\'e sur $U$ par la section $f$. 





\subsubsection{}\label{IV:1-1-5}

Soit $j:U\hookrightarrow X$. Si $\cF$ s'injecte dans $j_* j^* \cF$, la suite 
exacte 
\[\xymatrix{
  0 \ar[r] 
  & \cF \ar[r]
  & j_* j^* \cF \ar[r]
  & j_* j^* \cF / \cF \ar[r]
  & 0
}\]
fournit 
$\partial:\h^0(j_* j^*\cF/\cF) = \h_D^0(j_* j^* \cF/\cF) \to \h_D^1(X,\cF)$. 
L'application compos\'ee 
\[\xymatrix{
  \h^0(U,\cF) = \h^0(X,j_* j^*\cF) \ar[r] 
    & \h^0(X,j_* j^*\cF / \cF) \ar[r]^-\partial 
    & \h_D^1(X,\cF)
}\]
est l'oppos\'ee de l'application consid\'er\'ee en \ref{IV:1-1-4}. 





\subsubsection{}\label{IV:1-1-6}

On rappelle que si $\cL$ est un faisceau inversible sur un sch\'ema $X$, le 
$\dG_m$-torseur correspondant est le faisceau $\operatorname{Isom}(\cO,\cL)$, 
sur lequel $\dG_m$ agit par 
$(\lambda,f)\mapsto f\circ (\lambda\cdot) = \lambda f$. On rappelle aussi que 
si $D$ est un diviseur de Cartier sur $X$, et $j:U\hookrightarrow X$ 
l'inclusion d'ouvert compl\'ementaire, le faisceau inversible $\cO(D)$ est le 
sous-faisceau de $j_*\cO_U$ form\'e des sections locales $s$ telles que $s f$ 
soit dans $\cO_X$, pour $f$ une \'equation locale de $D$.










\subsection{Cup-produits}\label{IV:1-2}

Dans ce num\'ero, nous d\'eveloppons quelques remarques sur les cup-produits en 
cohomologie \`a support qui nous reservirons, dans [\nameref{V}], pour relier 
dualit\'e de Poincar\'e des courbes et autodualit\'e de la jacobienne.

\subsubsection{}\label{IV:1-2-1}

Soient $X$ un site, $Y$ un partie ferm\'ee de $X$, et $\cF,\cG$ deux faisceaux 
ab\'eliens sur $X$. Par exemple: $X$ un sch\'ema, $Y$ un sous-sch\'ema ferm\'e 
et $\cF,\cG$ des faisceaux sur $\et X$. D\'efinissons un produit 
\begin{equation}\label{IV:eq:1-2-1-1}
  \Gamma_Y(X,\cF)\otimes \Gamma(Y,\cG) \to \Gamma_Y(X,\cF\otimes\cG) \text{.}
\end{equation}
Le produit d'une section $s$, \`a support dans $Y$, de $\cF$ par une section 
$t$ de $\cG$ sur $Y$ s'obtient comme suit: localement sur $X$, $t$ est la 
restriction \`a $Y$ d'une section $t'$ de $\cG$ sur $X$, et on forme le produit 
$s\otimes t'$. Il est \`a support dans $Y$, et ne d\'epend pas du choix de 
$t'$, ce qui l\'egitime et permet de globaliser la d\'efinition.

Soit $i:Y\hookrightarrow X$ le morphisme d'inclusion. L'analogue local de 
\eqref{IV:eq:1-2-1-1} est le produit.
\begin{equation}\label{IV:eq:1-2-1-2}
  i^!\cF\otimes i^*\cG \to i^!(\cF\otimes\cG) \text{,}
\end{equation}
dont \eqref{IV:eq:1-2-1-1} se d\'eduit par application de $\Gamma(Y,-)$. 





\subsubsection{}\label{IV:1-2-2}

D\'erivons ces fl\`eches. Soient $\cK$, $\cL$ et $\cM$ dans la cat\'egorie 
d\'eriv\'ee, et une application bilin\'eaire $\cK\lotimes\cL\to\cM$. Par 
d\'erivation de \eqref{IV:eq:1-2-1-1}, on en d\'eduit 
\begin{equation}\label{IV:eq:1-2-2-1}
  \R\Gamma_Y(X,\cK) \lotimes \R\Gamma(Y,\cL) \to \R\Gamma_Y(X,\cM) 
\end{equation}
induisant 
\begin{equation}\label{IV:eq:1-2-2-2}
  \h_Y^i(X,\cK)\otimes \h^j(Y,\cL) \to \h_Y^{i+j}(X,\cM)
\end{equation}
(le cup-produit). La fl\`eche locale \eqref{IV:eq:1-2-1-2} fournit 
\begin{equation}\label{IV:eq:1-2-2-3}
  \R i^! \cK\lotimes i^*\cL \to \R i^!\cM \text{,}
\end{equation}
dont \eqref{IV:eq:1-2-2-1} se d\'eduit par application de $\R\Gamma(Y,-)$. 





\subsubsection{}\label{IV:1-2-3}

Ci-dessus, j'ai pass\'e sous silence les conflits qu'entraine l'usage dans une 
m\^eme formule de d\'erivations \`a droite ($\R\Gamma$) et \`a gauche 
($\lotimes$). 
\begin{enumerate}[\indent a)]
  \item Au num\'ero suivant, les d\'eriv\'es droits consid\'er\'es seront tous 
    de dimension cohomologique finie. Ceci permet de travailler 
    syst\'ematiquement dans les cat\'egories d\'eriv\'ees $\D^-$. Pour disposer 
    de complexes \`a la fois plats et flasques, on utilise les r\'esolutions 
    flasques canoniques comme en \cite[XVII]{sga4}. 
  \item Pour une th\'eorie plus g\'en\'erale, il cesse d'être d'interpr\'eter 
    les applications bbilin\'eaires de $\cK$ et $\cL$ dans $\cM$ comme des 
    morphismes de $\cK\lotimes\cL$ dans $\cM$. Par exemple, 
    $\R\Gamma_Y(X,\cK)\lotimes\R\Gamma(Y,\cL)$ n'est pas d\'efini si des deux 
    facteurs sont dans $\D^+$. Une solution est de travailler dans $\D^+$, de 
    d\'efinir 
    \[
      \operatorname{Bil}(\cK,\cL;\cM) = \varinjlim \hom(\cK'\otimes\cL',\cM') \text{,}
    \]
    o\`u la limite est prise sur les quasi-isomorphismes $\cK'\iso \cK$, 
    $\cL'\iso \cL$, $\cM\iso \cM'$ ($\cK'$, $\cL'$, $\cM'$ born\'es 
    inf\'erieurement) et o\`u $\hom$ est pour ``morphisme de complexes, \`a 
    homotopie pr\`es,'' et d'utiliser syst\'ematiquement de telles applications 
    bilin\'eaires, sans jamais mentionner de $\otimes$.  
\end{enumerate}




\subsubsection{Deuxi\`eme thème}\label{IV:1-2-4}

Soit $U$ un ouvert de $X$ et $j:U\hookrightarrow X$ le morphisme d'inclusion. 
Pour $\cK$ dans la cat\'egorie d\'eriv\'ee, sur $U$, on pose 
$\R\Gamma_!(U,\cK)=\R\Gamma(X,j_!\cK)$. Pour $\cK$, $\cL$ et $\cM$ sur $U$, et 
une application bilin\'eaire $\cK\lotimes\cL\to\cM$, on veut d\'efinir 
\begin{equation}\label{IV:eq:1-2-4-1}
  \R\Gamma(U,\cK)\lotimes\R\Gamma_!(U,\cL)\to\R\Gamma_!(U,\cM) \text{.}
\end{equation}
Au niveau des faisceaux, et de leurs sections globales, un tel produit se 
d\'eduit de l'isomorphisme 
$j_*\cF\otimes j_!\cG\xleftarrow\sim j_!(\cF\otimes \cG)$, mais il faut prendre 
garde au fait que $\R\Gamma_!$ n'est en g\'en\'eral pas le d\'eriv\'e du 
foncteur $\Gamma(X,j_!-)$. 

On commence par d\'efinir 
\[\xymatrix{
  \R j_* \cK\lotimes j_! \cL 
    & \ar[l]_-\sim j_!(\cK\lotimes \cL) \ar[r] 
    & j_!\cM \text{.}
}\]
Appliquant $\R\Gamma(X,-)$, on trouve 
\[\xymatrix{
  \R\Gamma(U,\cK)\lotimes\R\Gamma_!(U,\cL) = \R\Gamma(X,\R j_*\cK)\lotimes \R\Gamma(X,j_!\cL) \ar[r] 
    & \R\Gamma(X,j_! \cM) \text{.}
}\]

Ici encore, le conflit entre la gauche et la droite se r\'esoudra au nom\'ero 
suivant en travaillant dans $\D^-$. 





\subsubsection{Coda}\label{IV:1-2-5}

Soient $j:U\hookrightarrow X$ une partie ouverte et $i:Y\hookrightarrow U$ une 
partie ferm\'ee de $U$. Soit $\bar Y$ une ferm\'e de $X$ tel que 
$\bar Y\cap U=Y$, (par exemple, le compl\'ement de $U\setminus Y$). Soit $\cK$ 
sur $U$. On pose $\R\Gamma_{Y!}(U,\cK)=\R\Gamma_!(Y,\R i^!\cK)$ o\`u 
$\R\Gamma_!$ est relatif \`a l'inclusion de $Y$ dans $\bar Y$. Pour tout ouvert 
$V$ de $U$, contenant $Y$, $\R\Gamma_{Y!}(U,\cK)$ s'envoie dans 
$\R\Gamma_!(V,\cK)$: si on note encore $i$ l'inclusion de $\bar Y$ dans $X$, 
$j$ celle de $Y$ dans $\bar Y$ et $k$ celle de $V$ dans $X$, on a 
$\R i^!\cK = j^*\R i^! k_!(k^*\cK)$, d'o\`u un morphisme 
$j_!\R i^!\cK \to \R i^! k_!(k^*\cK)$. Lui appliquant 
$\R\Gamma(\bar Y,-)$, on trouve 
$\R\Gamma_{Y!}(U,\cK)\to \R\Gamma_{\bar Y}(k_! k^* \cK) \to \R\Gamma(k_! k^*\cK) = \R\Gamma_!(V,\cK)$. 

Pour $\cK,\cL,\cM$ sur $U$, et une applicatioin bilin\'eaire 
$\cK\lotimes\cL\to\cM$, on veut d\'efinir 
\begin{equation}\label{IV:eq:1-2-5-1-1}
  \h^n(U,\cK)\otimes \h_!^m(Y,\cL) \to \h_{Y!}^{n+m}(U,\cM)
\end{equation}
(ce dernier groupe lui-m\^eme s'envoyant dans $\h_!^{n+m}(V,\cM)$). Dans la 
cat\'egorie d\'eriv\'ee, il s'agit de d\'efinir 
\begin{equation}\label{IV:eq:1-2-5-2}
  \R\Gamma_Y(U,\cK) \lotimes \R\Gamma_!(R,\cL) \to \R\Gamma_{Y!}(U,\cM) \text{.}
\end{equation}
On identifie $\R\Gamma_Y$ \`a $\R\Gamma(Y,\R i^!\cK)$. Le produit cherch\'e 
est alors du type \eqref{IV:eq:1-2-4-1} relatif au produit locale 
\eqref{IV:eq:1-2-2-3} sur $Y$: $\R i^! \cK\lotimes\cL \to \R i^! \cM$. 





\subsubsection{}\label{IV:1-2-6}

Ci-dessus, on a d\'eroul\'e le sorite absolu. On a un sorite relatif 
parall\`ele, avec $\Gamma$ remplac\'e par $f_*$ par $f$ un morphisme $X\to S$. 










\subsection{La r\`egle de Koszul}\label{IV:1-3}

Soient $A$ un anneau commutatif, et $(V_i)_{i\in I}$ une famille finie de 
$A$-modules gradu\'es (ou $\dZ/2$-gradu\'es). Rappelons la d\'efinition du 
produit tensoriel gradu\'e $\bigotimes_{i\in I} V_i$, au sens de la r\`egle de 
Koszul (cf. \cite[XVII 1.1]{sga4}). Pour chaque ordre total $a$ sur $I$, on va 
d\'efinir un module $V(a)$. On va aussi d\'efinir un syst\`eme transitif 
d'isomorphismes $\varphi_{a b}:V(b)\iso V(a)$ et le produit tensoriel 
gradu\'e des $V_i$ sera la ``valeur commune'' $\varprojlim V(a)$ de ce 
syst\`eme de modules. On prend 
\begin{enumerate}[\indent a)]
  \item $V(a) = \bigotimes_{i\in I} |V_i|$ (produit-tensoriel ordinaire des 
    modules non gradu\'es sous-jacents aux $V_i$).
  \item si les $x_i\in V_i$ sont homog\`enes, on prend 
    $\varphi_{a b}(\otimes x_i) = (-1)^N \otimes x_i$, o\`u $N$ est la somme 
    des $\deg(x_i)\deg(x_j)$ \'etendue aux couples $(i,j)$ tels que 
    $i<_a j$ et $i>_b j$.
\end{enumerate}





\subsubsection*{Exemple 1}

Prenons $I=\{1,2\}$ et soient $a$ l'ordre o\`u $1<2$, $b$ l'ordre o\`u $1>2$. 
Soient $v_i\in V_i$, homog\`enes, et notons $v_1\otimes v_2$ (resp. 
$v_2\otimes v_1$) l'image dans le produit tensoriel gradu\'e du produit des 
$v_i$ dans $V(a)$ (resp. $V(b)$). On a 
\[
  v_1\otimes v_2 = (-1)^{\deg(v_1)\deg(v_2)} v_2\otimes v_1
\]
(r\`egle de Koszul). 





\subsubsection*{Exemple 2}

Si les $V_i$ sont tous de degr\'e $1$, on a , reliant le module sous-jacent au 
produit tensoriel gradu\'e, et le produit tensoriel ordinaire des modules 
$|V_i|$ sous-jacents aux $V_i$, un isomorphisme canonique 
\[
  \left|\bigotimes V_i\right| \simeq \bigotimes |V_i|\otimes_{\dZ}\bigwedge^{|I|}\dZ^I \text{.}
\]





\subsubsection{Exemple 3}

Pour $A$ un corps, et $X_i$ une famille finie d'espaces, la formule de 
K\"unneth s'\'ecrit 
$\h^\bullet\left(\prod X_i,A\right) = \bigotimes \h^\bullet(X_i,A)$. 

Soit $V^\vee$ le dual gradu\'e de $V$. L'application canonique 
\begin{equation}\label{IV:eq:1-3-1}
  V^\vee\otimes V = V\otimes V^\vee \to A
\end{equation}
est $v'\otimes v\to v'(v)$. 

On suppose maintenant que $A$ est un corps, et on ne consid\`ere que des 
espaces vectoriels de dimension finie. L'isomorphisme canonique 
\begin{equation}\label{IV:eq:1-3-2}
  W\otimes V^\vee \to \hom(V,W) 
\end{equation}
est $w\otimes v'\mapsto \left(v\mapsto w\cdot v'(v)\right)$. Via cet 
isomorphisme, la composition $\hom(Y,Z)\otimes \hom(X,Y)\to \hom(X,Z)$ 
s'identifie au morphisme induit par \eqref{IV:eq:1-3-1}: 
$Z\otimes Y^\vee\otimes Y\otimes X^\vee \to Z\otimes X^\vee$. 

La trace de $f:V\to V^\vee$ (nulle pour $f$ homog\`ene de degr\'e $\ne 0$) est 
l'image de $f$ par 
\begin{equation}\label{IV:eq:1-3-4}
  \tr:\hom(V,V) {\xleftarrow\sim}_\text{\eqref{IV:eq:1-3-2}} V\otimes V^\vee = V^\vee \otimes V \to_\text{\eqref{IV:eq:1-3-1}} A .
\end{equation}

On v\'erifie ais\'ement que, pour $f$ de degr\'e $0$, 
\begin{equation}\label{IV:eq:1-3-5}
  \tr(f,V) = \sum (-1)^i \tr(f,V^i) \text{.}
\end{equation}

Si on exprime que les deux morphismes compos\'es de morphismes 
\eqref{IV:eq:1-3-1} $V^\vee\otimes V\otimes W^\vee\otimes W\to k$ commutent, on 
trouve que, pour $f:V\to W$ et $g:W\to V$ homog\`enes, on a 
\begin{equation}\label{IV:eq:1-3-6}
  \tr(f g) = (-1)^{\deg(f)\deg(g)} \tr(g f) \text{.}
\end{equation}




















\section{La classe de cohomologie associ\'ee \`a un cycle}\label{IV:2}










\subsection{La classe d'un diviseur}\label{IV:2-1}





\subsubsection{}\label{IV:2-1-1}

Soit $D$ un diviseur de Cartier dans un sch\'ema $X$. Hors de $D$, le faisceau 
inversible $\cO(D)$ est trivialis\'e par la section $1$. La classe 
$\cl(D)$ de $D$, dans $\h_D^1(X,\dG_m)$, est la classe du 
$\dG_m$-torseur trivialis\'e sur $X\setminus D$ correspondant 
(\ref{IV:1-1-6} et \ref{IV:1-1-4}). 

Soit $\partial:\h^i(X\setminus D,\dG_m)\to \h_D^i(X,\dG_m)$ le morphisme 
\ref{IV:1-1-4}. Si $D$ admet une équation globale $f$, la multiplication par 
$f$ est un isomorphisme de $\cO(D)$, trivialis\'e par $1$ sur $X\setminus D$, 
avec $\cO$, trivialis\'e par $f$ sur $X\setminus D$. D'apr\`es \ref{IV:1-1-4}, 
on a donc 
\begin{equation}\label{IV:2-1-1-1}
  \cl(D) = \partial f \text{.}
\end{equation}

Pour tout morphisme $u:X'\to X$ tel que $u^* D$ soit encore un diviseur de 
Cartier (i.e., $u^{-1} D$ disjoint de $\operatorname{Ass}(X')$), on a 
$\cl(u^* D) = u^*\cl(D)$. Si on voulait une telle 
fonctoralit\'e pour tout morphisme $u$, il faudrait consid\'erer non pas des 
diviseurs de Cartier, mais plus g\'en\'eralement des faisceaux inversibles 
munis d'une section.

Rappelons que l'entier $n$ est dor\'enavant suppos\'e inversible sur les 
sch\'emas consid\'er\'es. Soit 
$\partial:\h_D^i(X,\dG_m) \to \h_D^{i+1}(X,\dmu_n)$ le cobord pour la suite 
exacte de Kummer $0 \to \dmu_n \to \dG_m \to \dG_m \to 0$.





\begin{definition}\label{IV:2-1-2}
La \emph{classe $\cl_n(D)$} de $D$ dans $\h_D^2(X,\dmu_n)$ est 
$\partial \cl(D)$. 
\end{definition}

Quand il n'y aura pas de risque de confusion, on omettra la mention de $n$. 





\subsubsection{}\label{IV:2-1-3}

Le diagramme 
\[\xymatrix{
  \h^0(X\setminus D,\dG_m \ar[r]^-\partial \ar[d]^-\partial 
    & \h_D^1(X,\dG_m) \ar[d]^-\partial \\
  \h^1(X\setminus D,\dmu_n) \ar[r]^-\partial 
    & \h_D^2(X,\dmu_n)
}\]
est anticommutatif. Si $D$ admet une \'equation globale $f$, 
$\cl_n(D)$ est donc l'oppos\'e de l'image par $\partial$ de la 
classe dans $\h^1(X\setminus D,\dmu_n)$ de $\dmu_n$-torseur des racines 
$n$-i\`emes de $f$. 





\begin{proposition}\label{IV:2-1-4}
Soit $i$ l'inclusion de $D$ dans $X$. Si $D$ et $X$ sont r\'eguliers, les 
faisceaux de cohomologie \`a support $\R^p i^! \dmu_n$ sont nuls pour $p=0,2$, 
et $\R^2 i^!\dmu_n = \underline{\dZ/n}$, engendr\'e par 
$\cl_n(D)$. 
\end{proposition}

Il suffit de prouver que, pour $X$ strictement locale et $D$ d\'efini par un 
param\`etre r\'egulier, on a $\h_D^p(X,d\mu_n) = 0$ pour $p=0,1$ et 
$\h^2(X,\dmu_n) = \dZ/n$ engendr\'e par $\cl(D)$. Notant par 
$\sim$ la cohomologie r\'eduite, on a 
$\widetilde\h^{p-1}(X\setminus D,\dmu_n)\iso \h_D^p(X,\dmu_n)$. L'assertion 
pour $p=0,1$ exprime que $D$ ne disconnecte par $X$, et pour $p=2$ r\'esulte, 
via \ref{IV:2-1-3}, du lemme d'Abhyankar. 

Ceci est un analogue partiel du th\'eor\`eme relatif 
(\hyperref[I]{Cohomologie \'etale}, \ref{I:5-3-4}). Grothendieck conjecture 
que les $\R^p i^! \dmu_n$ sont nuls pour $p\ne 2$, du moins pour $X$ excellent 
(conjecture de puret\'e), mais ceci n'est connu qu'en caract\'eristique $0$ 
(\cite[XIX]{sga4}). 




\begin{theorem}[Compatibilit\'e fondamentale]\label{IV:2-1-5}
Soient $X$ uen courbe lisse sur un corps alg\'ebriquement clos $k$, $P$ un 
point ferm\'e de $X$ et $\tr$ le compos\'e $\h_P^2(X,\dmu_n) \to \h_c^2(X,\dmu_n) \xrightarrow{\tr} \dZ/n$. On a 
\[
  \tr \cl(P) = 1 \text{.}
\]
\end{theorem}

Soit $\bar X$ la courbe projective et lisse compl\'etant $X$. La formule 
exprime que le faisceau inversible $\cO(P)$ sur $\bar X$ est de degr\'e $1$. 










\subsection{M\'ethode cohomologique}\label{IV:2-2}





\subsubsection{}\label{IV:2-2-1}

Soit $X$ un sch\'ema (noeth\'erien). Rappelons qu'un sous-sch\'ema $Y$ de $X$ 
est dit d'intersection compl\`ete locale, de codimension $c$, si, localement 
(sur $Y$), il est d\'efini par une suite r\'eguli\`ere de $c$ \'equations dans 
$X$. Pour $X$ le spectre d'un anneau locale $A$ d'id\'eal maximal $\fm$ et 
$Y$ d'id\'eal $\fa$, cela signifie que $\exp^i(A/\fa,A) = 0$ pour 
$i<\dim(\fa/\fm\fa)=c$, et toute suite d'\'el\'ements de $\fa$, d'image dans 
$\fa/\fm\fa$ une base de $\fa/\fm\fa$, est une suite r\'eguli\`ere 
d'\'equations pour $Y$. 





\subsubsection{}\label{IV:2-2-2}

Soit $i:Y\hookrightarrow X$ d'intersection compl\`ete locale, de codimension 
$c$. On se propose de d\'efinir une classe fondamentale locale 
$\cl(Y)$ qui soit une section globale du faisceau de cohomologie \`a support 
$\R^{2c}i^! \dZ/n(c)$ (rappelons que $\dZ/n(c)=\dmu_n^{\otimes c}$). 

Localement, $Y$ est l'intersection d'une suite de $c$ diviseurs $D_i$, et on 
d\'efinit $\cl(Y)$ comme le cup-produit des $\cl(D_i)$. Chaque $\cl(D_i)$ est 
\`a support dans $D_i$, leur produit est \`a support dans $Y$. Que, localement, 
ce produit ne d\'epende pas du choix des $D_i$, r\'esulte de \ref{IV:2-2-3} 
ci-dessous et des propri\'et\'es d'invariance suivantes:
\begin{enumerate}[\indent a)]
  \item compatibilit\'e \`a la localisation;
  \item ind\'ependance de l'order des $D_i$ (les $\cl(D_i)$ sont de degr\'e 
    $2$, pair, donc le cup-produit est commutatif);
  \item le produit ne d\'epend que du ``drapeau'' 
    $D_1\supset D_1\cap D_2\supset \cdots\supset Y$. 
\end{enumerate}
Pour prouver c), on note l'existence d'un produit (variante de \ref{IV:1-2-1}) 
\[
  \h_{D_1}^\bullet(X)\otimes \h_{D_1\cap D_2}^\bullet(D_1)\otimes \cdots \otimes \h_Y^\bullet(D_1\cap \cdots \cap D_{c-1}) \to \h_Y^\bullet(X) \text{;}
\]
le produit des $\cl(D_i)$ est encore le produit des ($\cl(D_i)$ restreint \`a 
$D_1\cap \cdots \cap D_{i-1})=(\cl(D_1\cap \cdots \cap D_i)$ dans 
$D_1\cap \cdots \cap D_{i-1})$. 





\begin{lemma}\label{IV:2-2-3}
Soient $A$ un anneau locale d'id\'eal maximal $\fm$ et $u=(u_1,\dots,u_c)$, 
$v=(v_1,\dots,v_c)$ deux suites r\'eguli\`eres engendrant le m\^eme id\'eal 
$\fa$. Il existe alors une suite $w_i$ ($1\leqslant i\leqslant N$) de telles 
suites, les reliant, telle que $w_{i+1}$ se d\'eduise de $w_i$ par une 
permutation, ou en ne changeant que le dernier \'el\'ement.
\end{lemma}

Puisqu'on dispose des permutations, il reviendrait au m\^eme de se permettre de 
changer un seul \'el\'ement, plut\^ot que le dernier. Par \ref{IV:2-2-1}, on se 
ram\`ene alors \`a v\'erifier que, dans l'espace vectoriel $\fa/\fm\fa$, on 
peut passer d'une base \`a une autre par une suite de permutations et de 
changements de base ne modifiant qu'un seul vecteur. Le groupe lin\'eaire est 
en effet engendr\'e par les matrices diagonales, \'el\'ementaires et de 
permutation. 





\subsubsection{}\label{IV:2-2-4}

Les m\^emes m\'ethodes permettent de d\'efinir une classe fondamentale locale 
pour tout $Y\subset X$ localement d\'efinissable par $c$ \'equations (la o\`u 
moins d\'equations suffisent, la classe est nulle). 





\subsubsection{}\label{IV:2-2-5}

On peut passer d'une telle classe fondamentale locale, dans 
$\h^0\left(Y,\R^{2 c} i^!\dZ/n(c)\right)$, \`a une classe fondamentale globale, 
dans $\h_Y^{2 c}\left(X,\dZ/n(c)\right)$, lorsqu'on dispose de r\'esultats de 
semi-puret\'e:





\begin{proposition}\label{IV:2-2-6}
\begin{enumerate}[(i)]
  \item Si $\R^p i^!\dZ/n=0$ pour $p<2 c$, alors 
    \[
      \h_Y^{2 c}\left(X,\dZ/n(c)\right) \iso \h^0\left(Y,\R^{2 c} i^! \dZ/n(c)\right) \text{.}
    \]
  \item Soit $Z$ une partie ferm\'ee de $Y$, de compl\'ement $V$ dans $Y$, et 
    $k$ l'inclusion de $Z$ dans $X$. Si $\R^p i^! \dZ/n = 0$ pour 
    $p\leqslant 2 c$, alors 
    \[
      \h_Y^{2 c}\left(X,\dZ/n(c)\right) \hookrightarrow \h_V^{2 c}\left(X,\dZ/n(c)\right) \text{.}
    \]
    Si $\R^p i^!\dZ/n=0$ pour $p\leqslant 2 c+1$, cette fl\`eche est une 
    isomorphisme.
\end{enumerate}
\end{proposition}

L'hypoth\`ese $\R^p i^!\dZ/n=0$ \'equivaut \`a $\R^p i^!\dZ/n(c)=0$. Ceci dit, 
(i) se lit sur la suite spectrale 
$\h^p(Y,\R^q i^!)\Rightarrow \h_Y^{p+q}(X,-)$. Si $k_1$ l'inclusion de $Z$ dans 
$Y$, on a $k=i k_1$, d'o\`u $\R k^! \dZ/n=\R k_1^! \R i^!$, et la suite exacte 
longue de cohomologie pour $Z\subset Y$ fournit une suite exacte longue 
\[\xymatrix{
  \cdots \ar[r] 
    & \h^i\left(Z,\R k^! \dZ/n(c)\right) \ar[r] 
    & \h_Y^i\left(X,\dZ/n(c)\right) \ar[r] 
    & \h_V^i\left(X,\dZ/n(c)\right) \ar[r] 
    & \cdots
}\]
dont (ii) r\'esulte. 





\subsubsection{Amplification}{\label{IV:2-2-7}

La proposition \ref{IV:2-2-6} reste valable pour $i$ un quelconque morphisme 
s\'epar\'e de type fini, et $2 c$ un entier (positif ou n\'egatif) quelconque, 
pour autant qu'on y remplace $\h_Y^{2 c}\left(X,\dZ/n(c)\right)$ par 
$\h^{2 c}\left(Y,\R i^!\dZ/n(c)\right)$, et de m\^eme pour $\h_V$. 

Les r\'esultats de semi-puret\'e suivants r\'esultent de 
\cite[1.8, 1.10, 1.15]{sga2}. Nous rappellerons leur preuve. 





\begin{theorem}\label{IV:2-2-8}
Soit un morphisme de $S$-sch\'emas de type fini 
\[\xymatrix{
  X \ar[r]^-i \ar[dr]_-f 
    & X \ar[d]^-g \\
  & S
}\]
On suppose $X$ lisse, purement de dimension relative $N$ et que $Y$ est fibre 
par fibre de dimension $\leqslant d$. Soit $c=N-d$. 
\begin{enumerate}[\indent (i)]
  \item On a $\R^p i^!\dZ/n=0$ pour $p<2 c$; de m\^eme pour $\dZ/n$ remplac\'e 
    par un faisceau $g^* \cF$.
  \item Si, au dessus d'un ouvert dense $U$ de $S$, $Y$ est fibre par fibre de 
    dimension $<d$, on a $\R^p i^!\dZ/n=0$ pour $p\leqslant 2 c$. Si de plus le 
    compl\'ement $Z$ de $U$ ne disconnecte pas localement $S$, on a 
    $\R^pi^!\dZ/n=0$ pour $p\leqslant 2 c+1$.
\end{enumerate}
\end{theorem}

Puisque $\R g^! \cF=g^*\cF(N)[2N]$, la formule de transitivit\'e 
$\R i^!\R g^!=\R f^!$ montre que 
$\R^{2 c+q} i^!(g^*\cF)=0\Leftrightarrow \R^{-2 d+q}f^!(\cF)$. Ceci nous 
ram\`ene \`a \'etudier $f$, i.e. \`a supposer que $X=S$. L'assertion (i) est 
alors \cite[XVIII 3.17]{sga4}. 

Soit $Y'$ l'image inverse de $Z$ dans $Y$:
\[\xymatrix{
  Y' \ar[r]^-v \ar[d]^-f 
    & Y \ar[d]^-f \\
  Z \ar[r]^-u 
    & S
}\]
D'apr\`es (i), les $\R^p f^! \dZ/n$ sont \`a support dans $Y'$ pour 
$p\leqslant -2 d+1$; et la suite spectrale 
$\R^a v^!\R^b f^! \Rightarrow \R^{a+b} (f v)^!$ montre qu'ils coinc\"ident avec 
les $\R^p(f v)^!\dZ/n=\R^p(u f)^!\dZ/n$. Appliquant (i) \`a $Y'/Z$ et \`a la 
suite spectrale $\R^a f^!\R^b u^! \Rightarrow \R^{a+b} (u f)^!$, on trouve que 
(ii) r\'esulte de la nullit\'e de $\R^b u^! \dZ/n$ pour $b=0$, ou $b=0$ et $1$ 
selon le cas. 





\subsubsection{}\label{IV:2-2-9}

Grothendieck conjecture l'analogue absolu suivant de \ref{IV:2-2-8} (conjecture 
de semi-puret\'e -- une cons\'equence de la conjecture de puret\'e): pour $Y$ 
de codimension $\geqslant c$ dans $X$ r\'egulier, on a $\h_Y^i(X,\dZ/n)=0$ pour 
$i<2 c$, tout au moins si $X$ est excellent. 





\subsubsection{}\label{IV:2-2-10}

On est maintenant \`a pied d'oeuvre pour d\'efinir la classe d'un cycle $Y$ de 
codimension $c$ dans $x$ lisse sur un corps $k$. On \'ecrit $Y=\sum d_i Y_i$, 
o\`u les $Y_i$ sont r\'eduits irr\'eductibles. Un ouvert $U_i$ de $Y_i$, de 
compl\'ementaire de codimension $>c$, est alors d'intersection compl\`ete 
locale dans $X$. Ceci permet de d\'efinir la classe fondamentale locale de 
$U_i$. D'apr\`es \ref{IV:2-2-6} et \ref{IV:2-2-8}, celle-ci provient d'une 
unique classe fondamentale $\cl(Y_i)\in \h_{Y_i}^{2 c}\left(X,\dZ/n(c)\right)$, 
et on pose 
\[
  \cl(Y) = \sum d_i \cl(Y_i) \in \h_{|Y|}^{2 c}\left(X,\dZ/n(c)\right) \text{.}
\]





\subsubsection{}\label{IV:2-2-11}

En g\'eom\'etrie analytique complexe, une construction de Baum, Fulton et 
Mac Pherson \cite{bfm75} permet de d\'efinir sans restriction la classe de 
cohomologie d'une intersection compl\`ete locale $Y\subset X$. Supposons $Y$ 
purement de codimension $c$, et soit $\cN$ le fibr\'e vectoriel sur $Y$ fibr\'e 
normal de $Y$ dans $X$. Ses sections locales sont celles de $(\cI/\cI^2)^\vee$, 
d'o\`u $\cI$ est le faisceau d'id\'eaux de $Y$. Rappelons que le faisceau des 
fonctions $C^\infty$ sur $X$ est d\'efini localement, en terme de plongements 
locaux de $X$ dans $\dC^n$, comme la restriction \`a $X$ du quotient du 
faiceaux des fonctions $C^\infty$ sur $\dC^n$ par l'id\'eal engendr\'e par les 
parties r\'eelles et imaginaires des \'equations qui d\'efinissent $X$. Le 
fibr\'e $\cN$ se prolonge en un fibr\'e vectoriel complexe $C^\infty N$ sur un 
voisinage $U$ de $Y$ dans $X$, et pour $U$ assez petit, il existe des sections 
$f$ de $N$, de lieu des z\'eros $Y$, et telles que, sur $Y$, $d f:\cN\to N$ 
soit l'identit\'e. Deux choix de $N$ et $f$ sont homotopes sur $Y$ assez petit. 

La classe de cohomologie $\cl(U)$ de la section $0$ de $N$ (not\'ee $z:U\to N$) 
est d\'efinie: $U$ est d'intersection compl\`ete locale dans $N$, et 
$\R^i z^!\dZ = 0$ pour $i\ne 2 c$, $\R^{2 c} z^!\dZ = \dZ$. On dispose de 
$f^*:\h_U^\bullet(N,\dZ) \to \h_Y^\bullet(X,\dZ)$ et on pose 
\[
  \cl(Y) = f^*\cl(U) \text{.}
\]

La m\^eme construction marche d\`es qu'on a sur un sous-espace analytique $Y$ 
de $X$ la structure normale suivante: un faisceau localement libre $\cC$ de 
rang $c$ sur $Y_\text{red}$, et un \'epimorphisme 
$\cC\to \cI/\cI\cdot \cI_\text{red}$. 










\subsection{M\'ethode homologique}\label{IV:2-3}





\subsubsection{}\label{IV:2-3-1}

Soit $f:Y\to X$ un morphisme plat de type fini, \`a fibres de dimension 
$\leqslant d$. Dans \cite[XVIII 2.9]{sga4}, nous avons d\'efini un morphisme 
trace 
\begin{equation}\label{IV:eq:2-3-1-1}
  \tr_f:\R^{2 d} f_! \dZ/n(d) \to \dZ/n \text{.}
\end{equation}
On a $\R^i f_!\dZ/n(d)=0$ pour $i>2 d$, et $\R^i f^!\dZ/n=0$ pour $i<-2 d$. 
Ceci, et l'adjonction entre $\R f_!$ et $\R f^!$, fournissent des isomorphismes 
\begin{equation}\label{IV:eq:2-1-3-2}
\begin{aligned}
  \hom\left(\R^{2 d}f_!\dZ/n(d),\dZ/n\right) 
    &= \hom\left(\R f_!\dZ/n(d),\dZ/n[-2 d]\right) \\
    &= \hom\left(\dZ/n,\R f^!\dZ/n(-d)[-2 d]\right) \\
    &= \h^0\left(Y,\R^{-2 d}f^!\dZ/n\right) \text{.}
\end{aligned}
\end{equation}
Au moins sur $S$ un point, l'image de $\tr_f$ dans les deux derniers groupes 
m\'erite le nom de classe fondamentale de $Y$ (en homologie). 

On note encore $\tr_f$ l'image de $\tr_f$ dans le second groupe, et les 
morphismes qui s'en d\'eduisent par fonctorialit\'e. Par exemple, pour $Y/S$ 
propre, le morphisme 
\begin{equation}\label{IV:eq:2-3-1-3}
\xymatrix{
  \h^i\left(Y,\dZ/n(d)\right) = \h^i\left(S,\R f_!\dZ/n(d)\right) \ar[r] 
    & \h^{i-2 d}(S,\dZ/n) \text{.}
}
\end{equation}

Supposons $Y$ contenu dans $X$ lisse sur $S$, purement de dimension relative 
$N$, et posons $c=N-d$. 
\[\xymatrix{
  Y \ar@{^{(}->}[r]^-i \ar[dr]_-f 
    & X \ar[d] \\
  & S
}\]
On a $\R f^! = \R i^!\R g^!$, et $\R g^! \dZ/n=\dZ/n(-N)[-2 N]$. La classe 
fondamentale de $Y$ s'identifie \`a un \'el\'emente de 
$\h^0\left(Y,\R i^!\dZ/n(c)[2 c]\right)=\h_Y^{2 c}\left(X,\dZ/n(c)\right)$, la 
classe $\cl(Y)$ de $Y$ dans $X$. Nous verrons plus loin qu'elle ne d\'epend que 
de $Y\subset X$, non de la projection de $X$ sur $S$, et que pour $Y$ 
d'intersection compl\`ete locale, elle induit la classe locale du nom\'ero 
pr\'ec\'edent. 

Explicitons l'isomorphisme 
\[
  \h_Y^{2 c}\left(X,\dZ/n(c)\right) \iso \hom\left(\R^{2 d}f_!\dZ/n(d),\dZ/n\right)
\]
qui transforme $\cl(Y)$ en $\tr_f$: via les isomorphismes 
\begin{align*}
  \h_Y^{2 c}\left(X,\dZ/n(c)\right) 
    &= \h_Y^{2 c}\left(\R i^!\dZ/n(c)\right) \\
    &= \hom_Y\left(\dZ/n(d)[2 d],\R i^!\dZ/n(N)[2 N]\right) \\
    &= \hom_X\left(\dZ/n(d)[2 d]_Y,\dZ/n(N)[2 n]\right) \text{,}
\end{align*}
c'est la ligne sup\'erieure de 
\[\xymatrix{
  \h_Y^{2 c} \ar[r] \ar[dr]_-{(3)} 
    & \hom\left(\R f_!\dZ/n(d)[2 d],\R f_! \R i^!\dZ/n(N)[2 n]\right) \ar[r]^-{(1)} \ar[d]^-{\tr_i} \ar@{}[dr]|-{(2)}
    & \hom\left(\R f_!\dZ/n(d)[2d],\dZ/n\right) \ar@{=}[d] \\
  & \hom\left(\R g_!\dZ/n(d)[2 d]_Y,\R g_!\dZ/n(N)[2 N]\right) \ar[r]^-{\tr_g} 
    & \hom\left(\R g_!\dZ/n(d)[2 d]_Y,\dZ/n\right) \text{,}
}\]
o\`u (1) est la fl\`eche d'adjonction $\R f_!\R f^!\to \text{id}$. La 
commutativit\'e (2) exprime que l'isomorphisme $\R f^!=\R i^!\R g^!$ est 
d\'efini par adjonction. Enfin, la fl\`eche d\'eduite de (3)
\[
  \h_Y^{2c}\left(X,\dZ/n(c)\right) \otimes \R^{2 d} g_!\dZ/n(d) \to \R^{2 N} g_!\dZ/n(N)
\]
s'interpr\`ete comme un cup-produit (cf. \ref{IV:1-2}). 








\section{Application: la formule des traces de Lefschetz dans le cas propre et lisse}\label{IV:3}

\begin{corollary_}\label{IV:3-7}
foo
\end{corollary_}
