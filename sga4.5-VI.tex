% !TEX root = sga4.5.tex

\chapter{Applications de la formule des traces aux sommes trigonométriques}\label{VI}










Dans cet expos\'e, j'explique comment la formule des traces permet de calculer 
ou d'\'etudier diverses sommes trigonom\'etriques et comment, jointe \`a la 
conjecture de Weil, elle peut permettre de les majorer.

Les deux premiers paragraphes donnent un ``mode d'emploi'' de ces outils. Le 
paragraphe 3 est un espos\'e, dans un langage cohomologique, des r\'esultats de 
Weil sur les sommes \`a $1$ variable. Les paragraphes $4$ \`a $6$ forment une 
\'etude d\'etaill\'ee des sommes de Gauss et de Jacobi -- y inclus les 
r\'esultats anciens et r\'ecents de Weil sur les caract\`eres de Hecke 
d\'efinis par des sommes de Jacobi. Au paragraphe $7$, nous \'etudions une 
g\'en\'eralisation \`a plusiers variables des sommes de Kloosterman. Enfin, au 
paragraphe $8$, on trouvera quelques indications sur d'autres usages qui ont 
\'et\'e faits on peuvent \^etre faits de ces m\'ethodes. 