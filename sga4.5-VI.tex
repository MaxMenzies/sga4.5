% !TEX root = sga4.5.tex

\chapter{Applications de la formule des traces aux sommes trigonométriques}\label{VI}
\chaptermark{Applications aux sommes trigonom\'etriques}










Dans cet expos\'e, j'explique comment la formule des traces permet de calculer 
ou d'\'etudier diverses sommes trigonom\'etriques et comment, jointe \`a la 
conjecture de Weil, elle peut permettre de les majorer.

Les deux premiers paragraphes donnent un ``mode d'emploi'' de ces outils. Le 
paragraphe 3 est un espos\'e, dans un langage cohomologique, des r\'esultats de 
Weil sur les sommes \`a $1$ variable. Les paragraphes $4$ \`a $6$ forment une 
\'etude d\'etaill\'ee des sommes de Gauss et de Jacobi -- y inclus les 
r\'esultats anciens et r\'ecents de Weil sur les caract\`eres de Hecke 
d\'efinis par des sommes de Jacobi. Au paragraphe $7$, nous \'etudions une 
g\'en\'eralisation \`a plusiers variables des sommes de Kloosterman. Enfin, au 
paragraphe $8$, on trouvera quelques indications sur d'autres usages qui ont 
\'et\'e faits on peuvent \^etre faits de ces m\'ethodes. 










\section*{Notations}\label{VI:0}





\subsection{}\label{VI:0-1}

On utilise les notations de \hyperref[II]{Rapport}, paragraphe \ref{II:1}. On 
aura souvent \`a consid\'erer une extension finie $\dF_{q^n}\subset \dF$ de 
$\dF_q$. On notera par un indice $0$ un objet sur $\dF_q$, et par un indice $1$ 
un objet sur $\dF_{q^n}$. Remplacer un indice $0$ par un indice $1$ (resp. 
supprimer l'incide) signifie qu'on \'etend les scalaires \`a $\dF_{q^n}$ (resp. 
\`a $\dF$).





\subsection{}\label{VI:0-2}

On d\'esigne par $\ell$ un nombre premier $\ne p$. Nous utilisons librement le 
language des $\dQ_\ell$-faisceaux, ainsii que celui des $E_\lambda$-faisceaux, 
pour $E_\lambda$ une extension finie de $\dQ_\ell$ (cf. \hyperref[II]{Rapport}, 
paragraphe \ref{II:2} et sp\'ecialement \ref{II:2-11}). 





\subsection{}\label{VI:0-3}

Soit $H^\bullet$ un espace vectoriel gradu\'e. Si $T$ est un endomorphisme de 
$H^\bullet$, on pose (cf. \hyperref[IV]{Cycle}, \ref{IV:eq:1-3-5}) 
\[
  \tr(T,H^\bullet) = \sum (-1)^i \tr(T,H^i) \text{.}
\]










\section{Principes}\label{VI:1}





\subsection{}\label{VI:1-1}

Soient $X_0$ un sch\'ema s\'epar\'e de type fini sur $\dF_q$, $E_\lambda$ une 
extension finie de $\dQ_\ell$ et $\sF_0$ un $E_\lambda$-faisceau sur $X-0$. La 
formule des traces dit que 
\begin{equation*}\tag{1.1.1}\label{VI:eq:1-1-1}
  \sum_{x\in X^F} \tr(F_x^\ast,\sF) = \sum (-1)^i \tr\left(F^\ast,\h_c^i(X,\sF)\right) \text{.}
\end{equation*}

Avec la notation \ref{VI:0-3}, le membre de droite s'\'ecrit simplement 
$\tr(F^\ast,\h_c^\bullet(X,\sF))$. 

Cette formule des traces pour les $E_\lambda$-faisceaux peut soit \^etre 
d\'eduite de \hyperref[II]{Rapport} \ref{II:4-10} par passage \`a limite (cf. 
\hyperref[II]{Rapport} \ref{II:4-11} \`a \ref{II:4-13}), soit \^etre d\'eduite 
de la formule des traces pour les $\dQ_\ell$-faisceaux (\hyperref[II]{Rapport} 
\ref{II:3-2}) par la m\'ethode de (\hyperref[III]{Functions $L$ mod. $\ell^n$}, 
\ref{III:4-3}). 

Nous allons interpr\'eter diverses sommes trigonom\'etriques comme le membre de 
gauche de \eqref{VI:eq:1-1-1}, pour $X_0$ et $\sF_0$ convenables. 





% NOTE: in this section, following the source (but not convention in the 
% LaTeX version, torsors are denoted in roman (not script) letters
\subsection{}\label{VI:1-2}

Soit $A$ un groupe fini. Un \emph{$A$-torseur} sur un sch\'ema $X$ est un 
faisceau $T$ sur $X$, muni d'une action \`a droite de $A$, qui, localement 
(pour la topologie \'etale) sur $X$, est isomorphe au faisceau constant $A$ sur 
lequel $A$ agit par translation \`a droite. 

Si $\tau:A\to B$ est un homomorphisme, et $T$ un $A$-torseur, il existe \`a 
isomorphisme unique pr\`es un et un seul $B$-torseur $\tau(T)$, muni d'un 
morphisme de faisceaux $\tau:T\to \tau(T)$ tel que 
\begin{equation*}\tag{1.2.1}\label{VI:eq:1-2-1}
  \tau(t a) = \tau(t) \tau(a)\text{.}
\end{equation*}

Si $T$ est un $A$-torseur sur $X$, et $\rho$ une repr\'esentation lin\'eaire 
de $A$: $\rho:A\to \gl(V)$ ($V$ un espace vectoriel de dimension finie sur 
$E_\lambda$), il existe \`a isomorphisme unique pr\`es un seul 
$E_\lambda$-faisceau $\sF$, lisse de rang $\dim(V)$, muni d'un morphisme de 
faisceaux 
\begin{equation*}\tag{1.2.2}\label{VI:eq:1-2-2}
  \rho:T\to \underline{\operatorname{Isom}}(V,\sF) \text{.}
\end{equation*}
tel que $\rho(t a) = \rho(t) \rho(a)$. On le note $\rho(T)$. Pour 
$\tau:A\to B$ et $\rho$ une repr\'esentation de $B$, on a un isomorphisme 
canonique $\rho(\tau(T)) = (\rho\tau)(T)$. 

Dans ce paragraphe (sauf l'appendice), on ne consid\`ere que le cas o\`u $A$ 
est commutatif et o\`u $V=E_\lambda$: $\rho$ est un caract\`ere 
$A\to E_\lambda^\times$, et $\rho(T)$ un $E_\lambda$-faisceau lisse de rang 
un. Le morphisme \eqref{VI:eq:1-2-2} s'interpr\`ete comme un morphisme partout 
non nul 
\begin{equation*}\tag{1.2.3}\label{VI:eq:1-2-3}
  \rho:T\to \rho(T) 
\end{equation*}
tel que $\rho(t a) = \rho(a)\rho(t)$. 





\subsection{}\label{VI:1-3}

Soient $S$ un sch\'ema, $G$ un sch\'ema en groupe commutatif sur $S$, et $G'$ 
une extension de sch\'emas en groupes commutatifs sur $S$
\[\xymatrix{
  0 \ar[r] 
    & A \ar[r] 
    & G' \ar[r]^-\pi 
    & G \ar[r] 
    & 0 \text{.}
}\]
Le faisceau $T$ des sections locales de $\pi$ est alors un $A$-torseur sur 
$G$. 

Si $X$ est un sch\'ema sur $S$, et $f,g$ deux $S$-morphismes de $X$ dans $G$, 
du fait que $T$ est d\'efini par une extension, on a un isomorphisme 
canonique 
\begin{equation*}\tag{1.3.1}\label{VI:eq:1-3-1}
  (f+g)^\ast T = f^\ast T + g^\ast T \text{.}
\end{equation*}
A gauche, $f+g$ est la somme de $f$ et $g$, au sens de la loi de groupe $G$: 
\`a droite, $+$ d\'esigne une somme de torseurs. 

Si $f:X\to G$ se factorise par $G'$, $f^\ast T$ est trivial (et un 
factorisation donne une trivialisation); \`a isomorphisme (non unique) pr\`es, 
le $A$-torseur $f^\ast T$ ne d\'epend que de l'image de $f$ dans 
$\hom_S(X,G)/\pi\hom_S(X,G')$: il est donn\'e par le morphisme $\partial$ dans 
la suite exacte 
\[\xymatrix{
  \hom(X,G') \ar[r] 
    & \hom(X,G) \ar[r]^-\partial 
    & \h^1(X,A) \text{.}
}\]

Nous appliquerons ces constructions aux suites exactes de Kummer 
$0 \to \dmu_n \to \dG_m \to \dG_m \to 0$, et aux torseurs de Lang. 





\subsection{Le torseur de Lang}\label{VI:1-4}

Soit $G_0$ un groupe alg\'ebrique commutatif connexe sur $\dF_q$ (pour le cas 
non commutatif, voir l'appendice). La loi de groupe est not\'ee 
\emph{multiplicativement}. L'isog\'enie de Lang 
\[
\fL:G_0 \to G_0: x\mapsto F x\cdot x^{-1}
\]
est \'etale; son image, un sous-groupe ouvert de $G_0$, ne peut \^etre que 
$G_0$ lui-m\^eme; son noyau est le groupe fini $G_0(\dF_q)$. Le \emph{torseur 
de Lang} $L$ est le $G_0(\dF_q)$-torseur sur $G_0$ d\'efini par la suite 
exacte 
\begin{equation*}\tag{1.4.1}\label{VI:eq:1-4-1}
\xymatrix{
  0 \ar[r] 
    & G_0(\dF_q) \ar[r] 
    & G_0 \ar[r]^-\fL 
    & G_0 \ar[r] 
    & 0 \text{.}
}
\end{equation*}


\paragraph{Notation}
On note encore $L$ et $\fL$ (ou $L_{(q)}$ et $\fL_{(q)}$) les objects d\'eduits 
de $L$ et $\fL$ par extension du corps de base. En cas de besoin, on 
pr\'ecisera par un indice $0$ ou $1$.


\paragraph{Exemples}
Pour $G_0=\dG_a$ ou $\dG_m$, les suites \eqref{VI:eq:1-4-1} s'\'ecrivent 
\begin{align*}\tag{1.4.2}\label{VI:eq:1-4-2}
&\xymatrix{
  0 \ar[r] 
    & \dF_q \ar[r] 
    & \dG_a \ar[r]^-{x^q-x} 
    & \dG_a \ar[r] 
    & 0 
} \\
\tag{1.4.3}\label{VI:eq:1-4-3}
&\xymatrix{
  0 \ar[r] 
    & \dmu_{q-1} \ar[r] 
    & \dG_m \ar[r]^-{x^{q-1}} 
    & \dG_m \ar[r] 
    & 0
}
\end{align*}





\subsection{}\label{VI:1-5}

Calculons l'endomorphisme $F^\ast$ de la fibre du torseur de Lang en 
$\gamma\in G^F=G_0(\dF_q)$. Si $g\in \fL^{-1}(\gamma)\subset G$, on a 
$F g = F g\cdot g^{-1}\cdot g = \gamma g$. D\`es lors (\hyperref[II]{Rapport}, 
\ref{II:1-2}) 
\begin{equation*}\tag{1.5.1}\label{VI:eq:1-5-1}
  \text{sur $L_0(G_0)_\gamma\simeq \fL^{-1}(\gamma)$, $F^\ast$ est $g\mapsto g\gamma^{-1}$.}
\end{equation*}





\subsection{}\label{VI:1-6}

Sur $\dF_{q^n}$, l'identit\'e $F_{(q^n)}=F_{(q)}^n$ entre endomorphismes de 
$G_1$ implique que $\fL_{(q^n)}=\fL_{(q)}\circ \prod_{i=0}^{n-1} F_{(q)}^i$. 
Sur $G_0(\dF_{q^n})$, $F_{(q)}^i$ agit comme l'\'el\'ement 
$x\mapsto x^{q^i}$ de $\gal(\dF_{q^n}/\dF_q)$. Sur $G_0(\dF_{q^n})$, 
$\prod F_{(q)}^i$ est donc le compos\'e de la norme 
$G_0(\dF_{q^n})\to G_0(\dF_q)$ et de l'inclusion 
$G_0(\dF_q)\subset G_0(\dF_{q^n})$: le diagramme 
\begin{equation*}\tag{1.6.1}\label{VI:eq:1-6-1}
\xymatrix{
  0 \ar[r] 
    & G_0(\dF_{q^n}) \ar[r] \ar[d]^-N 
    & G_1 \ar[r]^-{\fL_{(q^n)}} \ar[d]^-{\prod_{i=0}^{n-1} F_{(q)}^i} 
    & G_1 \ar[r] \ar@{=}[d] 
    & 0 \\
  0 \ar[r] 
    & G_0(\dF_q) \ar[r] 
    & G_q \ar[r]^-{\fL_{(q)}} 
    & G_1 \ar[r] 
    & 0
}
\end{equation*}
est commutatif. Dans le langage des torseurs, ceci fournit un isomorphisme 
canonique entre $G_0(\dF_q)$-torseurs sur $G_1$ 
\begin{equation*}\tag{1.6.2}\label{VI:eq:1-6-2}
  N L_{(q^n)} = L_{(q)} \text{.}
\end{equation*}





\begin{definition_}\label{VI:1-7}
Soient $f_0:X_0\to G_0$ un morphisme et 
$\chi:G_0(\dF_q)\to E_\lambda^\times$ un caract\`ere. On pose 
$\sF(\chi,f_0) = \chi^{-1}(f_0^\ast (L_0(G_0))) = f_0^\ast\chi^{-1}(L_0(G_0))$. 
\end{definition_}

On a les propri\'et\'es de fonctorialit\'e suivantes:
\begin{enumerate}[a)]
  \item $\sF(\chi,f_0)$ est bimultiplicatif en $\chi$ et $f_0$: 
    \begin{align*}\tag{1.7.1}\label{VI:eq:1-7-1}
      \sF(\chi,f_0'\cdot f_0'') &= \sF(\chi,f_0')\otimes \sF(\chi,f_0'') \text{,} \\
      \tag{1.7.2}\label{VI:eq:1-7-2}
      \sF(\chi'\chi'',\sF_0) &= \sF(\chi',f_0)\otimes \sF(\chi'',f_0) \text{.}
    \end{align*}
    Cela r\'esulte de la d\'efinition de $\chi(T)$, pour $T$ un torseur, joint 
    \`a \eqref{VI:eq:1-3-1} pour \eqref{VI:eq:1-7-1}. 
  \item Pour $g_0:Y_0\to X_0$ un morphisme, on a 
    \begin{equation*}\tag{1.7.3}\label{VI:eq:1-7-3}
      \sF(\chi,f_0\circ g_0) = g_0^\ast \sF(\chi,f_0) \text{.}
    \end{equation*}
    En particulier, $\sF(\chi,f_0) = f_0^\ast \sF(\chi)$, o\`u $\sF(\chi)$ 
    d\'esigne $\sF(\chi,\operatorname{id}_{G_0})$, 
  \item Pour $u_0:G_0\to H_0$ un morphisme, et 
    $\chi:H_0(\dF_q) \to E_\lambda^\times$, on a 
    \begin{equation*}\tag{1.7.4}\label{VI:eq:1-7-4}
      \sF(\chi,u_0 f_0) = \sF(\chi u_0,f_0) \text{.}
    \end{equation*}
  \item Pour $G_0=\prod_{i\in I} G_0^i$, $\chi$ de coordonn\'ees $\chi_i$ 
    ($i\in I$) et $f_0$ de coordonn\'ees $f_0^i$, il r\'esulte de 
    \eqref{VI:eq:1-7-1} \`a \eqref{VI:eq:1-7-4} que 
    \begin{equation*}\tag{1.7.5}\label{VI:eq:1-7-5}
      \sF(\chi,f_0) = \bigotimes_{i\in I} \sF(\chi_i,f_0^i) \text{.}
    \end{equation*}    
\end{enumerate}

Noter le $\chi^{-1}$ dans la d\'efinition \ref{VI:1-7}. Il assure que, pour 
$x\in X^F$, on ait sur $\sF(\chi,f_0)_x$ 
\begin{equation*}\tag{1.7.6}\label{VI:eq:1-7-6}
  F_x^\ast = \chi f_0(x)
\end{equation*}
(utiliser que la fibre en $x$ du morphisme \eqref{VI:eq:1-2-2}, pour 
$\rho=\chi^{-1}$, commute \`a $F_x^\ast$, et \eqref{VI:eq:1-5-1}). 

Si $\sF(\chi,f_0)_1$ est le faisceau d\'eduit de $\sF(\chi,f_0)$ par extension 
du corps de base \`a $\dF_{q^n}$, on d\'eduit de \eqref{VI:eq:1-6-2} que 
\begin{equation*}\tag{1.7.7}\label{VI:eq:1-7-7}
  \sF(\chi,f_0)_1 = \sF(\chi\circ N,f_1) \text{.}
\end{equation*}





\subsection{Abus de notations}\label{VI:1-8}

(i) Si $\Xi$ est une notation pour l'application compos\'ee 
$\chi f_0:X_0(\dF_q) \to E_\lambda^\times$, on \'ecrira parfois $\sF(\Xi)$ au 
lieu de $\sF(\chi,f_0)$. Grâce \`a \eqref{VI:eq:1-7-1} \`a \eqref{VI:eq:1-7-5}, 
on ne risque gu\`ere d'ambiguit\'e. Par exemple:
\begin{enumerate}[a)]
  \item on \'ecrit $\sF(\chi)$ pour $\sF(\chi,\operatorname{id}_{G_0})$ 
    (notations d\'ej\`a utilis\'ee en \ref{VI:1-7}b)); 
  \item en \'ecrit $\sF(\chi f_0)$ pour $\sF(\chi,f_0)$;
  \item avec les notations de \ref{VI:1-7}d), on \'ecrit 
    $\sF(\prod \chi_i f_0^i)$ pour $\sF(\chi,f))$. 
\end{enumerate}

Avec cette notation, \eqref{VI:eq:1-7-4} exprime que l'\'ecriture 
$\sF(\chi u_0 f_0)$ n'est pas ambiguë, \eqref{VI:eq:1-7-1} \eqref{VI:eq:1-7-2} 
\eqref{VI:eq:1-7-5} expriment une multiplicativit\'e de $\sF(\Xi)$ en $\Xi$, 
et \eqref{VI:eq:1-7-6} se r\'ecrit $F_x^\ast = \Xi(x)$ sur $\sF(\Xi)_x$.

(ii) Si $X$ est un sch\'ema sur une extension $k$ de $\dF_q$, et que $f$ est un 
morphisme de $X$ dans $G_0\otimes_{\dF_q} k$, on notera encore $\sF(\chi,f)$, 
$\sF(\chi f)$, ou $\sF(\chi)$ (pour $f$ une inclusion) l'image r\'eciproque par 
$f$ du $E_\lambda$-faisceau d\'eduit de $\chi^{-1}(L_0(G))$ par extension des 
scalaires de $\dF_q$ \`a $k$. Si $f_0$ est le compos\'e 
$X\to G_0\otimes_{\dF_q} k \to G_0$, c'est encore le $\sF(\chi,f_0)$ de 
\ref{VI:1-7}. Pour $k$ est une extension finie de $\dF_q$, on a 
$\sF(\chi f) = \chi(\chi N_{k/\dF_q} f)$. 

Appliquant \eqref{VI:eq:1-1-1} \`a \eqref{VI:eq:1-7-6} et \eqref{VI:eq:1-7-7}, 
on trouve: 





% NOTE: originally a 'scholium'
\begin{theorem_}\label{VI:1-9}
Soient $S_0$ un sch\'ema s\'epar\'e de type fini sur $\dF_q$, $G_0$ un groupe 
alg\'ebrique commutatif connexe sur $\dF_q$, $f_0:S_0\to G_0$ un morphisme et 
$\chi:G_0(\dF_q) \to E_\lambda^\times$. On a 
\begin{equation*}\tag{1.9.1}\label{VI:eq:1-9-1}
  \sum_{s\in S_0(\dF_q)} \chi f_0(s) = \tr(F^\ast,\h_c^\bullet(S,\sF(\chi,f_0))) 
\end{equation*}
et, pour tout entier $n\geqslant 1$
\begin{equation*}\tag{1.9.2}\label{VI:eq:1-9-2}
  \sum_{s\in S_0(\dF_{q^n})} \chi\circ N_{\dF_{q^n}/\dF_q} f_0(s) = \tr({F^\ast}^n, \h_c^\bullet\left(S,\sF(\chi,f_0))\right) \text{.}
\end{equation*}
\end{theorem_}





\subsubsection{Remarque}\label{VI:1-9-3}

Prenons pour $G_0$ un produit. Avec les notations de \ref{VI:1-7}d) et 
\ref{VI:1-8}c), la formule \eqref{VI:eq:1-9-2} devient 
\[
  \sum_{s\in S_0(\dF_{q^n})} \prod_i \chi_i\circ N_{\dF_{q^n}/\dF_q} (f_0^i(s)) = \tr\Big({F^\ast}^n, \h_c^\bullet\Big(S,\sF\Big(\prod_i \chi_i f_0^i\Big)\Big)\Big) \text{.}
\]





\subsubsection{Remarque}\label{VI:1-9-4}

Prenons $G_0=\{e\}$. La formule \eqref{VI:eq:1-9-1} devient 
\[
  \left|S_0(\dF_q)\right| = \sum_{s\in S_0(\dF_q)} 1 = \tr(F^\ast,\h_c^\bullet(S,\dQ_\ell)) \text{.}
\]

Il est rare qu'on puisse explicitement calculer le membre de droite de 
\eqref{VI:eq:1-9-1}. Voici un exemple amusant, avec $G_0=\{e\}$. 





\subsection{Exemple: Une quadrique projective non singuli\`ere de dimension impaire sur \texorpdfstring{$\dF_q$}{Fq} a le m\^eme nombre de points rationnels que l'espace projectif sur \texorpdfstring{$\dF_q$}{Fq} de la m\^eme dimension.}

Si, sur $\dC$, $X$ est une hypersurface quadrique non singuli\`ere dans 
l'espace projectif $\dP^{2 N}$, et $Y$ un hyperplan, on sait que pour 
$i\leqslant 2\dim(X)=2\dim(Y)$, les inclusions 
$X\hookrightarrow \dP^{2 N}\hookleftarrow Y$ induisent des isomorphismes (en 
cohomologie ordinaire) 
\[\xymatrix{
  \h^i(X,\dQ) 
    & \ar[l]_-\sim \h^i(\dP^{2 N},\dQ) \ar[r]^-\sim 
    & \h^i(Y,\dQ) \text{.}
}\]

Par sp\'ecialisation, il en r\'esulte que si $X_0'$ est une hypersurface 
quadrique non singuli\`ere dans l'espace projectif $\dP_0^{2 N}$ sur $\dF_q$, 
et $Y_0'$ un hyperplan, les inclusions 
$X_0'\hookrightarrow \dP_0^{2 N}\hookleftarrow Y_0'$ induisent des 
isomorphismes 
\[\xymatrix{
  \h^i(X',\dQ_\ell) 
    & \ar[l]_-\sim \h^i(\dP^{2 N},\dQ_\ell) \ar[r]^-\sim 
    & \h^i(Y',\dQ_\ell) \text{.}
}\]
Ces isomorphismes commutent \`a $F^\ast$, et on applique la formule des traces. 





\subsection{}\label{VI:1-11}

On peut aussi utiliser la formule des traces pour comprendre les formules 
classiques donnant le nombre de points rationnels des groupes lin\'eaires sur 
les corps finis, et ceux de certains espaces homog\`enes (voir le paragraphe 
8). 





\subsection{}\label{VI:1-12}

On dispose d'un dictionnaire permettant de traduire en termes cohomologiques 
ques divers types de manipulations classiques sur les sommes 
trigonom\'etriques. Ce dictionnaire sera donn\'e au paragraphe 2. L'\'enonc\'e 
cohomologique \'etant plus ``g\'eom\'etrique,'' il pourra parfois s'appliquer 
\`a des situations o\`u l'argument classique ne s'applique qu'apr\`es une 
extension du corps fini de base. Voici un exemple (un cas particulier d'un 
th\'eor\`eme de Kazhdan). 





\begin{theorem_}[Kazhdan]\label{VI:1-13}
Soit $X_0$ un sch\'ema sur $\dF_q$. Supposons qu'il existe une action $\rho$ de 
$\dG_a$ sur $X$, et un morphisme $f:X\to Y$ de sch\'emas sur $\dF$, faisant de 
$X$ un $\dG_a$-torseur sur $Y$. Alors, le nombre de points rationnels de $X_0$ 
est divisible par $q$.
\end{theorem_}

Supposons d'abord que $\rho$, $Y$ et $f$ soient d\'efinis sur $\dF_q$. 


\paragraph{Argument classique}
Les fibres de $f:X_0(\dF_q)\to Y_0(\dF_q)$ ont toutes $q$ \'el\'ements: 
$|X_0(\dF_q)| = q\cdot |Y_0(\dF_q)|$, d'o\`u la divisibilit\'e. 


\paragraph{Traduction cohomologique}
Les faisceaux images directes sup\'erieures $\R^i f_!\dQ_\ell$ sont 
\[
  \R^i f_!\dQ_\ell = \begin{cases}
                       \dQ_\ell(-1) & \text{si $i = 2$} \\
                       0            & \text{pour $i\ne 2$}
                     \end{cases}
\]
La suite spectrale de Leray de $f$ d\'eg\'en\`ere donc en un isomorphisme 
\[
  \h_c^i(X,\dQ_\ell) = \h_c^{i-2}(Y,\dQ_\ell)(-1) \text{.}
\]
Pour comprendre l'effet d'un twist \`a la Tate sur les valeurs propres de 
Frobenius, le plus commode est d'utiliser le point de vue galoisien 
(\hyperref[II]{Rapport} \ref{II:1-8}), et de noter que la substitution de 
Frobenius $\varphi$ agit sur $\dQ_\ell(1)$ par multiplication par $q$. Les 
valeurs propres de $F^\ast$ sur $\h_c^p(X,\dQ_\ell)$ sont donc les produits par 
$q$ des valeurs propres de $F^\ast$ agissant sur $\h_c^{i-2}(Y,\dQ_\ell)$. On 
sait que celles-ci sont des entiers alg\'ebriques \cite[XXI 5.2.2]{sga7}; d\`es 
lors 

\begin{equation*}\tag{$*$}\label{VI:eq:*}
  \substack{\displaystyle\text{les valeurs propres de $F^\ast$ agissant sur $\h_c^i(X,\dQ_\ell)$ sont} \\
  \displaystyle\text{des entiers alg\'ebriques divisibles par $q$.}}
\end{equation*}


\paragraph{Descente}
Revenons aux hypoth\`eses du th\'eor\`eme. Pour $n$ convenable, on peut 
supposer que $\rho$, $Y$ et $f$ sont d\'efinis sur $\dF_{q^n}$. Le morphisme de 
Frobenius relatif \`a $\dF_{q^n}$ \'etant la puissance $n$-i\`eme de celui 
relatif \`a $\dF_q$, l'\'enonc\'e \eqref{VI:eq:*} pour le sch\'ema sur 
$\dF_{q^n}$ d\'eduit de $X_0$ par extension des scalaires nous fournit: 
\begin{equation*}\tag{$**$}\label{VI:eq:**}
  \substack{\displaystyle\text{les puissances $n$-\`iemes des valeurs propres de $F^\ast$ agissant sur} \\ \displaystyle\text{$\h_c^i(X,\dQ_\ell)$ sont des entiers alg\'ebriques divisibles par $q^n$.}}
\end{equation*}

Cette assertion implique \eqref{VI:eq:*}. Le nombre de points rationnels de 
$X_0$, soit $\sum (-1)^i \tr(F^\ast,\h^i(X,\dQ_\ell))$, est donc entier 
alg\'ebrique divisible par $q$. Ceci implique qu'il soit divisible par $q$ en 
tant qu'entier rationnel. 





\subsection{}\label{VI:1-14}

La formule \ref{VI:1-9-3} contr\^ole la d\'ependance en $n$ de la somme 
trigonom\'etrique au membre de gauche. Elle implique des identit\'es entre une 
somme trigonom\'etrique et celles qui s'en d\'eduisennt par ``extension des 
scalaires.'' Le plus c\'el\`ebre de ces identit\'es est celle de 
Hasse-Davenport: soient $\chi:\dF_q^\times \to \dC^\times$ et 
$\psi:\dF_q\to \dC^\times$ des caract\`eres, $\psi$ non trivial, et 
d\'efinissons la somme de Gauss $\tau(\chi,\psi)$ par 
\begin{equation*}\tag{1.14.1}\label{VI:eq:1-14-1}
  \tau(\chi,\psi) = - \sum_{x\in \dF_q^\times} \psi(x) \chi^{-1}(x) \text{.}
\end{equation*}





\begin{theorem_}[Hasse-Davenport]\label{VI:1-15}
On a 
\begin{equation*}\tag{1.15.1}\label{VI:eq:1-15-1}
  \tau\left(\chi\circ N_{\dF_{q^n}/\dF_q},\psi\circ \tr_{\dF_{q^n}/\dF_q}\right) = \tau(\chi,\psi)^n \text{.}
\end{equation*}
\end{theorem_}

Soient $E\subset \dC$ un corps de nombres contenant les valeurs de $\chi$ et 
$\psi$, et $E_\lambda$ le compl\'et\'e de $E$ en une place de caract\'eristique 
$\ell\ne p$. L'identit\'e \eqref{VI:eq:1-15-1} est une identit\'e dans $E$. Il 
revient au m\^eme de la prouver dans $\dC$, ou dans $E_\lambda$. Nous la 
prouverons dans $E_\lambda$, en regardant $\chi$ et $\psi$ comme \`a valeurs 
dans $E_\lambda^\times$. 

On applique \ref{VI:1-9-3} pour $X_0=\dG_m$ sur $\dF_q$ et 
$G_0=\dG_m\times \dG_a$. Les $\h_c^i(\dG_m,\sF(\chi^{-1}\psi))$ sont nuls pour 
$i\ne 1$, et le $\h_c^1$ est de dimension $1$ (\ref{VI:4-2}). D'apr\`es 
\ref{VI:1-9-3}, 
$\tau(\chi\circ N_{\dF_{q^n}/\dF_q},\psi\circ \tr_{\dF_{q^n}/\dF_q})$ est 
l'unique valeur propre de $(F^\ast)^n$ sur $\h_c^1$, et \eqref{VI:eq:1-15-1} en 
r\'esulte. 

Des exemples analogues sont donn\'es dans Weil \cite[App V]{we74}. 





\subsection{}\label{VI:1-16}

Un $E_\lambda$-faisceau $\sF_0$ sur $X-0$ est dit \emph{ponctuellement de 
poids $n$} si pour tout point ferm\'e $x\in |X_0|$, les valeurs propres de 
$F_x^\ast$ (\hyperref[II]{Rapport} \ref{II:1-2}) sont des nombres alg\'ebriques 
dont tous les conjugu\'es complexes sont de valeur absolue 
$q_x^{n/2}$, o\`u $q_x$ est le nombre d'\'el\'ements de $k(x)$. Le th\'eor\`eme 
suivant sera d\'emontr\'e dans \cite{de80}. 





\begin{theorem_}\label{VI:1-17}
Si $\sF_0$ est ponctuellement de poids $n$, pour toute valeur propre $\alpha$ 
de l'endomorphisme $F^\ast$ de $\h_c^i(X,\sF)$, il existe un entier 
$m\leqslant n+i$ tel que les conjugu\'es complexes de $\alpha$ soient tous de 
valeurs absolue $q^{m/2}$.
\end{theorem_}

Les faisceaux consid\'er\'es en \ref{VI:1-9} sont de poids $0$. Le th\'eor\`eme 
fournit donc pour la somme \ref{VI:eq:1-9-1} (on plut\^ot, pour tous ses 
conjugu\'es complexes) la majoration 
\begin{equation*}\tag{1.17.1}\label{VI:eq:1-17-1}
  \left|\sum_{s\in S_0(\dF_q)} \prod_i \chi_i(f^i(s)) \right| \leqslant \sum_i \dim \h_c^i\left(S,\sF\left(\prod \chi_i f^i\right)\right)\cdot q^{i/2} \text{.}
\end{equation*}





\subsection{Remarques}\label{VI:1-18}

\begin{enumerate}[a)]
  \item Si $\dim S=n$, pour tout $E_\lambda$-faisceau $\sF$ sur $S$, on a 
    \[
      \h_c^i(S,\sF) = 0 \qquad \text{pour $i\notin [0,2n]$.}
    \]
  \item Le groupe $\h_c^0(X,\sF)$ est le groupe des sections globales de $\sF$ 
    sur $S$ dont le support est propre. Si $S$ est connexe, non complet et 
    $\sF$ lisse, on bien si $S$ est connexe, $\sF$ lisse de rang un, et non 
    constant, ce groupe est nul.
  \item Si $S$ est lisse purement de dimension $n$, et $\sF$ lisse, la 
    dualit\'e de Poincar\'e dit que les espaces vectoriels 
    $\h_c^i(X,\sF)$ et $\h^{2n-i}(X,\sF^\vee)(2n)$ sont duax l'un de l'autre 
    (pour l'effet d'un twist \`a la Tate sur les valeurs propres de Frobenius, 
    cf. la $2$\`eme partie de la preuve de \ref{VI:1-13}).
  \item Si $\dim S=n$, la dualit\'e de Poincar\'e permet de calculer comme 
    suite $\h_c^{2n}(S,\sF)$: on prend un ouvert $U$ de $S_\text{red}$, dont le 
    compl\'ementaire est de dimension $<n$, lisse purement de dimension $n$, et 
    sur lequel $\sF$ est lisse. On a alors 
    $\h_c^{2n}(U,\sF) \iso \h_c^{2n}(S,\sF)$, car dans la suite exacte longue 
    de cohomologie, $\h_c^i(S\setminus U,\sF) = 0$ pour $i=2n,2n-1$. Par 
    Poincar\'e, on a donc 
    \[
      \h_c^{2n}(S,\sF) \iso \left(\h^0(U,\sF^\vee)(2n)\right)^\vee \text{.}
    \]
    
    Supposons $U$ connexe, et soit $u$ un point g\'eom\'etrique de $U$. Le 
    ``syst\`eme local'' $\sF$ correspond \`a une repr\'esentation de 
    $\pi_1(U,u)$ sur $\sF_u$, et $\h^0(U,\sF^\vee)$ est 
    $(\sF_u^\vee)^{\pi_1(U,u)}=((\sF_u)_{\pi_1(U,u)})^\vee$ (dual des 
    coinvariants). On a donc 
    \[
      \h_c^{2n}(S,\sF) \simeq (\sF_u)_{\pi_1(U,u)}(-2n) \text{.}
    \]
  \item Ces remarques permettent souvent le calcul de $b_0$ et $b_{2n}$. 
    Calculer les autres $b_i$ peut \^etre difficile. Si $b_{2n}=0$ et que 
    $b_{2n-1}=0$, la majoration \eqref{VI:eq:1-17-1} est une majoration \`a la 
    Lang-Weil, avec, pour $q$ grand, un gain en $\sqrt q$ par rapport \`a la 
    majoration triviale en $O(q^n)$. On prendra toutefois garde que la 
    majoration \eqref{VI:eq:1-17-1} n'implique pas formellement la majoration 
    triviale 
    \[
      \left|\sum_{s\in S_0(\dF_q)} \prod_i \chi_i(f^i(s))\right| \leqslant \left| S_0(\dF_q)\right| \text{,}
    \]
    donc il peut \^etre utile de tenir compte (cf. la preuve de \ref{VI:3-8}). 
\end{enumerate}

J'explique ci-dessous une m\'ethode pour prouver que $b_i=0$ pour $i\ne n$. 
Quand elle s'applique, elle fournit une estimation en $O(q^{n/2}$. La constante 
implicate dans $0$ est $b_n=(-1)^n \chi(S,\sF(\prod \chi_i f^i))$ en peut 
\^etre difficile \`a calculer. 





\begin{proposition_}\label{VI:1-19}
Soient $\bar X$ un sch\'ema propre sur un corps alg\'briquement clos $k$, 
$j:X\hookrightarrow bar X$ un ouvert de $X$ et $\sF$ un $E_\lambda$-faisceaux 
sur $X$. On suppose que 
\begin{enumerate}[\indent a)]
  \item $(j_\ast\sF)_x = 0$ pour $x\in \bar X\setminus X$, i.e. 
    $j_!\sF\iso j_\ast\sF$; 
  \item $\R^i j_\ast \sF = 0$ pour $i>0$.
\end{enumerate}
Alors, les applications 
\[
  \h_c^i(X,\sF) \to \h^i(X,\sF)
\]
sont des isomorphismes. 
\end{proposition_}

Les hypoth\`eses a) b) signifiet que $j_! \sF \iso \R j_\ast \sF$, d'o\`u 
\[
  \h_c^i(X,\sF) = \h^i(\bar X,j_!\sF) \iso \hh^i(\bar X,\R j_\ast \sF) = \h^i(X,\sF) \text{.}
\]
Autrement dit, la suite spectrale de Leray pour $j$ 
\[
  E_2^{pq} = \h^p(\bar X,\R^q j_\ast \sF) \Rightarrow \h^{p+q}(X,\sF) 
\]
d\'eg\'en\`ere, par b) en des isomorphismes 
\[
  \h^i(\bar X,j_\ast\sF) \iso \h^i(X,\sF) \text{,}
\]
tandis que, par a) 
\[
  \h_c^i(X,\sF) = \h^i(\bar X,j_!\sF) \iso \h^i(\bar X,j_\ast\sF) \text{.}
\]





\subsubsection{Exemple}\label{VI:1-9-1}

Si $X$ est une courbe, ou si $\bar X$ est lisse, $X$ le compl\'ement d'un 
diviseur \`a croisements normaux et $\sF$ mod\'er\'ement ramifi\'e, 
l'hypoth\`ese a) de \ref{VI:1-19} (pas d'invariants sous la monodromie locale) 
implique l'hypoth\`ese b). 





\begin{proposition_}\label{VI:1-20}
Soient $X_0$ un sch\'ema affine lisse purement de dimension $n$ sur $\dF_q$ et 
$\sF_0$ un $E_\lambda$-faisceau lisse ponctuellement de poids $m$ sur $X_0$. On 
suppose que les applications $\h_c^i(X,\sF) \to \h^i(X,\sF)$ sont des 
isomorphismes (cf. \ref{VI:1-17}). Alors, $\h_c^i(X,\sF) = 0$ pour $i\ne n$, et 
les conjugu\'es complexes des valeurs propres de $F^\ast$ sur $\h_c^n(X,\sF)$ 
sont de valeur absolue $q^{(m+n)/2}$. 
\end{proposition_}

La dualit\'e de Poincar\'e met en dualit\'e $\h_c^i(X,\sF)$ (resp. 
$\h_c^i(X,\sF^\vee)$) avec $\h^{2n-i}(X,\sF^\vee(n))$ (resp. 
$\h^{2n-i}(X,\sF(n))$). D'apr\`es \cite[XIV 3.2]{sga4}, les $\h^i$ sans support 
sont nuls pour $i>n$. Par dualit\'e, les $\h_c^i$ sont nuls pour $i<n$; puisque 
$\h_c^i(X,\sF) \iso \h^i(X,\sF)$, ces groupes sont nuls pour $i\ne n$. 
Appliquons \ref{VI:1-15} \`a $\sF_0$ et \`a $\sF_0^\vee(n)$ (de poinds 
$-m-2n$). On trouve que les conjugu\'es complexes $\alpha$ des valeurs propres 
de $F^\ast$ sur $\h_c^n(X,\sF) = \h_c^n(X,\sF^\vee(n))^\vee$ v\'erifient 
\begin{align*}
  |\alpha| &\leqslant q^{(m+n)/2} && \text{et} \\
  |\alpha^{-1}| &\leqslant q^{((-m-2n)+n)/2} = q^{-(m+n)/2} \text{,}
\end{align*}
d'o\`u l'assertion. 





\subsection{Remarque}\label{VI:1-21}

Le dernier argument montre que si $X_0$ est un sch\'ema s\'epar\'e lisse sur 
$\dF_q$, que $\sF_0$ est un $E_\lambda$-faisceau lisse ponctuellement de poids 
$m$ sur $X_0$ et que $\h_c^i(X,\sF) \hookrightarrow \h^i(X,\sF)$, alors les 
conjugu\'es complexes des valeurs propres de $F^\ast$ sur $\h_c^i(X,\sF)$ sont 
de valeur absolue $q^{(m+i)/2}$. 





\subsection*{Appendice. L'isog\'enie de Lang dans le cas non commutatif}




\subsection{}\label{VI:1-22}

Soient $G-0$ un groupe alg\'ebrique connexe sur $\dF_q$ et $\fL$ le morphisme 
$x\mapsto F x\cdot x^{-1}$ de $G-0$ dans lui-m\^eme. C'est un orphisme de 
$G_0$-espaces homog\`enes, si \`a la source on fait agir $G-0$ par translation 
\`a gauche $x\mapsto (y\mapsto x y)$, au but par 
$x\mapsto (y\mapsto F x\cdot y\cdot x^{-1}$). On a $\fL(xy) = \fL(x)$ pour 
$y\in G_0(\dF_q)$, et $\fL$ induit un isomorphisme $G_0/G_0(\dF_q)\iso G_0$. 

Comme dans le cas commutatif, $\fL$ fait donc de $G_0$ un $G_0(\dF_q)$-torseur 
sur $G_0$, le \emph{torseur de Lang} $L$. Si 
$\rho:G_0(\dF_q) \to \gl(n,E_\lambda)$ est une repr\'esentation lin\'eaire de 
$G_0(\dF_q)$, nous noterons $\sF(\rho)$ le $E_\lambda$-faisceau $\rho(L)$ 
(\ref{VI:1-2}). 





\subsection{}\label{VI:1-23}

Soit $\gamma\in G^F=G_0(\dF_q)$, et calculons l'endomorphisme $F^\ast$ de 
$\sF(\rho)_\gamma$. Soit $g\in G$ tel que $\fL(g)=\gamma$, d'o\`u un rep\`ere 
$\rho(g)\in \operatorname{Isom}(E_\lambda^n,\sF(\rho)_\gamma)$. Pour 
$e\in E_\lambda^n$, on a 
\[
  F(\rho(g)(e)) = \rho(F g)(e) = \rho(\gamma g)(e) = \rho(g\cdot g^{-1}\gamma g)(e) \text{.}
\]
On verra que $g^{-1} \gamma g\in G_0(\dF_q)$, d'o\`u 
\[
  F(\rho(g)(e)) = \rho(g) \rho(g^{-1}\gamma g)(e)
\]
et 
\begin{equation*}\tag{1.23.1}\label{VI:eq:1-23-1}
  \rho(g)^{-1} (F_\gamma^\ast)^{-1} \rho(g) = \rho(g^{-1} \gamma g) \text{:}
\end{equation*}
$\rho(g)$ identifie l'inverse de $F^\ast$ en $\gamma$ \`a l'automorphisme 
$\rho(g^{-1} \gamma g)$ de $E_\lambda^n$. Reste \`a comprendre ce qu'est 
$g^{-1} \gamma g$. 





\begin{lemma_}\label{VI:1-24}
\begin{enumerate}[(i)]
  \item Si $F g\cdot g^{-1} \in G_0(\dF_q)$, on a 
    $g^{-1} (F g\cdot g^{-1}) g = g^{-1} F g\in G_0(\dF_q)$.
  \item Soient $\gamma\in G_0(\dF_q)$, et $g$ tel que 
    $F g\cdot g^{-1} = \gamma$. La classe de conjugaison de $\gamma'=g^{-1} Fg$ 
    dans $G_0(\dF_q)$ ne d\'epend que de celle de $\gamma$. 
  \item L'application induite par $\gamma\mapsto \gamma'$, de l'ensemble 
    $G_0(\dF_q)^\natural$ des classes de conjugaison de $G_0(\dF_q)$ dans 
    lui-m\^eme, est bijective. Son inverse est l'application pour le groupe 
    oppos\'e. 
\end{enumerate}
\end{lemma_}

(i) Si $\gamma= F g\cdot g^{-1}\in G_0(\dF_q)$, les identit\'es $F g=\gamma g$ 
et 
$F\gamma=\gamma$ donnent $F(g^{-1} F g) = (F g)^{-1} F F g = g^{-1} \gamma^{-1} F(\gamma g) = g^{-1} \gamma^{-1} F \gamma F g = g^{-1} F g$, 
de sorte que $g^{-1} F g \in G_0(\dF_q)$. 

(ii) Pour $\alpha,\gamma\in G_0(\dF_q)$, les $g$ tels que 
$F g\cdot g^{-1} = \gamma$ forment une classe \`a droite sous $G_0(\dF_q)$, et 
si $F g\cdot g^{-1} = \gamma$, on a 
$F(\alpha g \alpha^{-1})(\alpha g \alpha^{-1})^{-1} = \alpha \gamma\alpha^{-1}$. 
Pour $\gamma$ parcourant une classe de conjugaison dans $G_0(\dF_q)$, les $g$ 
tels que $F g\cdot g^{-1} = \gamma$ parcourent donc une double classe sous 
$G_0(\dF_q)$, et $\gamma'= g^{-1} F g$ parcourt une classe de conjugaison. 

(iii) Enfin, l'inverse de $F g\cdot g^{-1}\mapsto g^{-1} F g$ est 
$g^{-1} F g\mapsto F g\cdot g^{-1}$. 





\subsection{}\label{VI:1-25}

La formule \eqref{VI:eq:1-23-1} peut se lire, bri\`evement: $F^\ast$ en 
$\gamma$ est $\rho(\gamma')^{-1}$. Si $\gamma$ appartient \`a la composante 
neutre de son centralisateur, on peut prendre $g$ dans $Z(\gamma)$. On a alors 
$\gamma=\gamma'$. Si cette condition n'est pas remplie, $\gamma$ et $\gamma'$ 
peuvent ne pas \^etre conjugu\'es. 





\subsection{Un exemple}\label{VI:1-26}

Pour $G_0=\operatorname{SL}(2)$, $q$ impair, 
$\alpha\in F_q^\times\setminus {\dF_q^\times}^2$ et $\bar\gamma$ la classe de 
conjugaison de $-1(1+N)$, avec $N^2=0$, $\bar\gamma'$ est la classe de 
conjugaison de $-1\cdot (1+\alpha N)$, de sorte que $\bar\gamma'\ne \bar\gamma$ 
si $N\ne 0$. 










\section{Dictionnaire}\label{VI:2}

Une manipulation \'el\'ementaire sur les sommes trigonom\'etriques peut souvent 
\^etre vue comme le reflet, via la formule des traces, d'un \'enonc\'e 
cohomologique. Dans ce paragraphe, j\'enum\`ere de tels \'enonc\'es, et leur 
reflet; il portent le m\^eme num\'ero, augment\'e d'une $\ast$ pour 
l'\'enonc\'e cohomologique. 





\subsection{}\label{VI:2-1}

``Une somme d'entiers alg\'ebriques est un entier alg\'ebrique'' admet pour 
analogue cohomologique le th\'eor\`eme suivant, prouv\'e dans 
\cite[XXI 5.2.2]{sga7} (cf. l'usage de \ref{VI:2-1}$^\ast$ en 
\ref{VI:1-13}). 




\addtocounter{subsection}{-1}
\begin{theorem_}[$\ast$]\label{VI:2-1*}
Soient $X_0$ un sch\'ema s\'epar\'e de type fini sur $\dF_q$ et $\sF_0$ un 
$E_\lambda$-faisceau sur $X_0$. Si, pour tout $x\in |X_0|$, les valeurs 
propres de $F_x^\ast$ sur $\sF_0$ sont des entiers alg\'ebriques, alors les 
valeurs propres de $F^\ast$ sur $\h_c^i(X,\sF)$ sont des entiers alg\'ebriques. 
\end{theorem_}





\subsection{}\label{VI:2-2}

D'autres r\'esultats d'int\'egralit\'es sont prouv\'es dans 
\cite[XXI, 5.2.2 et 5.4]{sga7}: si $\dim(x_0)\leqslant n$, et que $\alpha$ est 
une valeur propre de $F^\ast$ sur $\h_c^i(X,\sF)$, on a: 
\begin{enumerate}[a)]
  \item sous les hypoth\`eses de \ref{VI:2-1}$^\ast$, et si $i>n$, alors 
    $q^{i-n}$ divise $\alpha$;
  \item si les inverses des valeurs propres de $F_x^\ast$ sont des entiers 
    alg\'ebriques, alors $\alpha^{-1}$ est entier, sauf en $p$. Plus 
    pr\'ecisement, $q^{\inf(n,i)}\alpha^{-1}$ est un entier alg\'ebrique. 
\end{enumerate}

Pour les faisceaux consid\'er\'es en \ref{VI:1-9}, les valeurs propres des 
$F_x^\ast$ sont des racines de l'unit\'e, de sorte que les hypoth\`eses du 
th\'eor\`eme et celle de b) ci-dessus sont v\'erif\'ees. 





\subsection{}\label{VI:2-3}

Soient $f:A\to B$ une application d'un ensemble fini dans un autre, et 
$\varepsilon$ un function sur $A$. On a 
\begin{equation*}\tag{2.3.1}\label{VI:eq:2-3-1}
  \sum_{a\in A} \varepsilon(a) = \sum_{b\in B} \sum_{f(a)=b} \varepsilon(a) \text{.}
\end{equation*}





\addtocounter{subsection}{-1}
\subsection{*}\label{VI:2-3*}

Soient $f:X\to Y$ un morphisme de sch\'emas s\'epar\'es de type fini sur $k$ 
alg\'ebriquement clos, et $\sF$ un $E_\lambda$-faisceau sur $X$. La suite 
spectrale de Leray de $f$ en cohomologie \`a support propre s'\'ecrit 
\begin{align*}\tag{2.3.1*}\label{VI:eq:2-3-1*}
  E_2^{p q} = \h_c^p(Y,\R^q f_! \sF) \Rightarrow \h_c^{p+q}(X,\sF) \text{.}
\end{align*}

Soient $f_0:X_0 \to Y_0$ un morphisme de sch\'emas s\'epar\'es de type fini sur 
$\dF_q$ et $\sF_0$ un $E_\lambda$-faisceau sur $X_0$. On a entre 
\eqref{VI:eq:2-3-1} et \eqref{VI:eq:2-3-1*} la compatibilit\'e suivante. 

\begin{enumerate}[a)]
  \item Le faisceau $\R^q f_! \sF$ se d\'eduit par extension des scalaires de 
    $\dF_q$ \`a $\dF$ du faisceau $\R^q f_{0!} \sF_0$ sur $Y_0$. En tant que 
    tel, il est muni d'une correspondance de Frobenius 
    $F^\ast:F^\ast\R^q f_! \sF \to \R^q f_! \sF$. Elle se d\'eduit par 
    fonctorialit\'e de $\R f_!$ de la correspondance de Frobenius de $\sF$: 
    c'est le compos\'e 
    \[\xymatrix{
      F^\ast \R^q f_! \sF \ar[r]^-\sim 
        & \R^q f_! F^\ast \sF \ar[r]^-\sim 
        & \R^q f_! \sF \text{.}
    }\]
  \item La formule des traces pour $\sF_q$ s'\'ecrit 
    \begin{align*}\tag{1}\label{VI:eq:1}
      \sum_{x\in X_0(\dF_q)} \tr(F_x^\ast,\sF_0) = \tr(F^\ast,\h^\bullet(X,\sF)) \text{;}
    \end{align*}
    celle pour $\R^q f_! \sF$ s'\'ecrit 
    \begin{align*}\tag{2$_q$}\label{VI:eq:2_q} 
      \sum_{y\in Y_0(\dF_q)} \tr(F_y^\ast,\R^q f_! \sF) = \tr(F^\ast, \h^\bullet(Y,\R^q f_! \sF)) \text{.}
    \end{align*}
\end{enumerate}

On a, pour $y\in Y(\dF)$, $(\R^q f_! \sF)_y = \h_c^q(f^{-1}(y),\sF)$, et si 
$y\in Y_0(\dF_q)$, 
\[
  \tr(F_y^\ast,\R^q f_! \sF) = \tr(F^\ast,\h_c^q(f^{-1}(y),\sF)) \text{,}
\]
d'o\`u par la formule des traces sur la fibre 
\[
  \sum_{\substack{x\in X_0(\dF_q) \\ f(x) = y}} \tr(F_x^\ast,\sF_0) = \sum (-1)^q \tr(F_y^\ast,\R^q f_! \sF) \text{.}
\]
Prenant la somme altern\'ee des identit\'es \eqref{VI:eq:2_q}, on trouve donc 
\begin{align*}\tag{2}\label{VI:eq:2}
  \sum_{y\in Y_0(\dF_q)} \sum_{\substack{x \in X_0(\dF_q) \\ f(x) = y}} \tr(F_x^\ast,\sF_0) = \tr(F^\ast,\h^\bullet(Y,\R^\bullet f_! \sF)) \text{.}
\end{align*}

L'\'egalit\'e des membres de gauche de \eqref{VI:eq:1} et \eqref{VI:eq:2} 
r\'esulte de \eqref{VI:eq:2-3-1}, celle des membres de droite de 
\eqref{VI:eq:2-3-1*} (compte tenu du fait que $F^\ast$ est un endomorphisme de 
toute la suite spectrale). 





\subsection{}\label{VI:2-4}

Soient $(A_i)_{i\in I}$ une famille finie d'ensembles finis, 
$A=\prod_{i\in I} A_i$, $\varepsilon_i$ une fonction sur $A_i$, et 
$\varepsilon$ la fonction $a\mapsto \prod \varepsilon_i(a_i)$ sur $A$. On a 
\begin{align*}\tag{2.4.1}\label{VI:eq:2-4-1}
  \sum_{a\in A} \varepsilon(a) = \prod_{i\in I} \sum_{a_i\in A_i} \varepsilon_i(a_i) \text{.}
\end{align*}





\addtocounter{subsection}{-1}
\subsection{*}\label{VI:2-4*}

Soient $(X_i)_{i\in I}$ une famille finie de sch\'emas s\'epar\'es de type fini 
sur $k$ alg\'ebriquement clos, $X=\prod_{i\in I} X_i$, $\sF_i$ un 
$E_\lambda$-faisceau sur $X_i$ et $\sF$ le produit tensoriel externe 
$\bigboxtimes_{i\in I}\sF_i = \bigotimes_{i\in I} \operatorname{pr}_i^\ast(\sF_i)$ 
des $\sF_i$. On a la formule de K\"unneth 
\begin{align*}\tag{2.4.1*}\label{VI:eq:2-4-1*}
  \h_c^\bullet(X,\sF) = \bigotimes_{i\in I} \h_c^\bullet(X_i,\sF_i) \text{.}
\end{align*}

Si on veut un isomorphisme \eqref{VI:eq:2-4-1*} canonique, et ind\'ependant du 
choix d'un ordre total sur $I$, il faut prendre au membre droite le produit 
tensoriel gradu\'e au sens de la r\`egle de Koszul 
(\hyperref[IV]{Cycle}, \ref{V:1-3}). 





\subsection{}\label{VI:2-5}

Soient $A$ un ensemble fini, $B$ une partie de $A$ et $\varepsilon$ une 
fonction sur $A$. On a 
\begin{align*}\tag{2.5.1}\label{VI:2-5-1}
  \sum_{a\in A} \varepsilon(a) = \sum_{a\in A\setminus B}\varepsilon(a) + \sum_{a\in B} \varepsilon(a) \text{.}
\end{align*}





\addtocounter{subsection}{-1}
\subsection{*}\label{VI:2-5*}

Soient $x$ un sc\'ema s\'epar\'e de type fini sur $k$ alg\'ebriquement clos, 
$Y$ un sous-sch\'ema ferm\'e et $\sF$ un $E_\lambda$-faisceaux sur $X$. On a 
une suite exacte longue de cohomologie 
\begin{align*}\tag{2.5.1*}\label{VI:eq:2-5-1*}
\xymatrix{
  \cdots \ar[r]^-\partial 
    & \h_c^i(X\setminus Y,\sF) \ar[r] 
    & \h_c^i(X,\sF) \ar[r] 
    & \h_c^i(Y,\sF) \ar[r]^-\partial 
    & \cdots 
}
\end{align*}

Plus g\'en\'eralement, si on part d'une filtration finie de $X$ par des parties 
ferm\'ees $X_p$ ($p\in \dZ$; on suppose que $X_p\supset X_{p+1}$, que $X_p=X$ 
pour $p$ assez petit, et que $X_p=\varnothing$ for $p$ assez grand), on a une 
suite spectrale 
\begin{align*}\tag{2.5.2*}\label{VI:eq:2-5-2*}
  E_1^{p q} = \h_c^{p+1}(X_p\setminus X_{p+1},\sF) \Rightarrow \h_c^{p+q}(X,\sF) \text{.}
\end{align*}





\subsection{}\label{VI:2-6}

Soient $A$ un ensemble fini, $(A_i)_{i\in I}$ un recouvrement fini de $A$ et 
$\varepsilon$ une fonction sur $A$. Pour $J\subset I$, soit $A_J$ 
l'intersection des $A_j$ ($j\in J$). On a 
\begin{align*}\tag{2.6.1}\label{VI:eq:2-6-1}
  \sum_{J\subset I} (-1)^{|J|} \sum_{a\in A_J} \varepsilon(a) = 0 \text{.}
\end{align*}





\addtocounter{subsection}{-1}
\subsection{}\label{VI:2-6*}

Soient $X$ un sch\'ema s\'epar\'e de type fini sur $k$ alg\'ebriquement clos, 
$(X_i)_{i\in I}$ un recouvrement ferm\'e (resp. ouvert) fini de $X$ par des 
sous-sch\'emas, et $\sF$ un $E_\lambda$-faisceau sur $X$. Pour $J\subset I$, 
soit $X_J$ l'intersection des $X_j$ ($j\in J$). On a des suites spectrales 
respectives (de Leray) 
\begin{align*}\tag{2.6.1*}\label{VI:eq:2-6-1}
  E_1^{p q} &= \bigoplus_{|J|=p+1>0} \h_c^q(X_J,\sF) \otimes_\dZ \textstyle\bigwedge^{|J|} \dZ^J \Rightarrow \h_c^{p+q}(X,\sF) \\ 
  \tag{2.6.2*}\label{VI:eq:2-6-2*}
  E_1^{p q} &= \bigoplus_{|J|=1-p>0} \h_c^q(X_J,\sF) \otimes_\dZ \textstyle\bigwedge^{|J|} \dZ^J \Rightarrow \h_c^{p+q}(X,\sF)
\end{align*}
Si $J=\varnothing$ n'\'etait pas exclus, on aurait de m\^eme des suites 
spectrales convergeant vers $0$. Les facteurs $\bigwedge^{|J|} \dZ^J$ sont l\`a 
pour nous dispenser de choisir un ordre total sur $I$. 





\subsection{}\label{VI:2-7}

Soient $A$ un groupe commutatif fini, et $\chi:A \to E_\lambda^\times$ un 
caract\`ere non trivial. On a 
\begin{align*}\tag{2.7.1}\label{VI:eq:2-7-1}
  \sum_{a\in A} \chi(a) = 0 \text{.}
\end{align*}
On prouvera l'analogue cohomologique de \eqref{VI:eq:2-7-1} en en transposant 
la preuve suivante: si $x\in A$, $a\mapsto x a$ est une permutation de $A$, et 
\begin{align*}
  \sum_{a\in A}\chi(a) = \sum_{a\in A} \chi(x a) &= \chi(x) \sum_{a\in A} \chi(a) && \text{, d'o\`u} \\
  (\chi(x)-1)\sum_{a\in A} &= 0 \text{,}
\end{align*}
et il existe par hypoth\`ese $x$ tel que $\chi(x)-1\ne 0$. 





\addtocounter{subsection}{-1}
\begin{theorem_}[*]\label{VI:2-7*}
Soient $G_0$ un groupe alg\'ebrique commutatif connexe sur $\dF_q$ et 
$\chi:G_0(\dF_q) \to E_\lambda^\times$ un caract\`ere non triviale. On a 
\begin{align*}\tag{2.7.1*}\label{VI:eq:2-7-1*}
  \h_c^\bullet(G,\sF(\chi)) = 0 \text{.}
\end{align*}
\end{theorem_}

Pour $x$ un point rationnel de $G$, notons $t_x$ la translation $t_x(g) = x g$ 
de $G$. La formule $\fL t_x = t_{\fL(x)} \fL$ exprime que $(t_x,t_{\fL(x)})$ 
est un automorphisme du diagrame $G\xrightarrow\fL G$ ($G$ muni du torseur de 
Lang). Soit $\rho(g)$ l'automorphisme de $(G,\sF(\chi))$ qui s'en d\'eduit. 
Pour $g\in G_0(\dF_q)$, i.e. pour $\fL(g) = e$, c'est l'identit\'e sur $G$, et 
la multiplication par $\chi(g)^{-1}$ sur $\sF(\chi)$. 

Notons $\rho_H(g)$ l'automorphisme de $\h_c^\bullet(G,\sF(\chi))$ d\'eduit de 
$\rho(g)$. Un argument d'homotopie (Lemme \ref{VI:2-8} ci-dessous) montre que 
$\rho_H(g) = \rho_H(e)$, donc est l'identit\'e. D'autre part, pour 
$g\in G_0(\dF_q)$, $\rho_H(g)$ est la multiplication par 
$(\chi^{-1}(g):\text{ sur }\h^\bullet(G,\sF(\chi))$, la multiplication par 
$(\chi^{-1}(g)-1)$ est nulle. Prenant $g$ tel que $\chi(g)\ne 1$, on obtient 
\ref{VI:2-7*}. 





\begin{lemma_}\label{VI:2-8}
Soient $X$ et $Y$ deux sch\'emas sur $k$ alg\'ebriquement clos, avec $X$ 
s\'epar\'e de type fini et $Y$ connexe. Soient $\sF$ un faisceau sur $X$ et 
$(\rho,\varepsilon)$ une famille d'endomorphismes de $(X,\sF)$ param\'etr\'ee 
par $Y$: 
\begin{align*} 
  \rho:Y\times_k X &\to Y\times_k X && \text{est un $Y$-morphisme, et} \\
  \varepsilon:\rho^\ast \operatorname{pr}_2^\ast \sF &\to \operatorname{pr}_2^\ast\sF && \text{un morphisme de faisceaux.} 
\end{align*}
On suppose que $\rho$ est propre. Pour $y\in Y(k)$, soit $\rho_H(y)^\ast$ 
l'endomorphisme de $\h_c^\bullet(X,\sF)$ induit par $\rho_y:X\to X$ et 
$\varepsilon_y:\rho_y^\ast\sF\to \sF$. Alors, $\rho_H(y)^\ast$ est 
ind\'ependant de $y$. 
\end{lemma_}

En effet, $\R^p \operatorname{pr}_{1!} \operatorname{pr}_2^\ast\sF$ est le 
faisceau constant sur $Y$ de valeur $\h^p(X,\sF)$, et $\rho_H(y)^\ast$ est la 
fibre en $y$ de l'endomorphisme 
\[\xymatrix{
  \R^p \operatorname{pr}_{1!} \operatorname{pr}_2^\ast \sF \ar[r]^-{\rho^\ast} 
    & \R^p \operatorname{pr}_{1!} \rho^\ast \operatorname{pr}_2^\ast \sF \ar[r]^-\varepsilon 
    & \R^p \operatorname{pr}_{1!} \operatorname{pr}_2^\ast \sF 
}\]
de ce faisceau. 





\subsection{Remarque}\label{VI:2-9}

Soient $G_0$ un groupe alg\'ebrique connexe sur $\dF_q$ et 
$\rho:G_0(\dF_q) \to \operatorname{GL}(V)$ un repr\'esentation lin\'eaire 
telle que $V^{G_0(\dF_q)}=0$. On a encore $\h_c^\bullet(G,\rho(L))=0$. 










\section{Sommes \`a une variable}\label{VI:3}





\subsection{}\label{VI:3-1}

Weil est le premier \`a avoir appliqu\'e des m\'ethodes ``cohomologiques'' \`a 
l'\'etude des sommes trigonom\'etriques \`a une variable; puisqu'il avait 
prouv\'e l'analogue de l'hypoth\`ese de Riemann pour les fonctions $L$ d'Artin 
sur les corps de fonctions, ces m\'ethodes lui fournissaient d'excellentes 
estimations (en $O(\text{racine carr\'ee du nombre de termes})$), et le 
comportement de ces sommes par ``extension du corps de base.'' 

L'essentiel de ce paragraphe est un expos\'e, dans un langage cohomologique, de 
ses r\'esultes. 





\subsection{}\label{VI:3-2}

Les m\'ethodes cohomologiques am\`enent ici \`a calculer des groupes 
$\h_c^i(X,\sF)$, pour $\sF_0$ un $E_\lambda$-faisceau sur un courbe $X-0$ sur 
$\dF_q$. Ces groupes sont nuls pour $i\ne 0,1,2$ et pour 
$i=0,2$, ils ont une interpr\'etation simple (\ref{VI:1-8} a,b,c). De plus, la 
caract\'eristique d'Euler-Poincar\'e 
$\chi_c(\sF) = \sum (-1)^i \dim \h_c^i(X,\sF)$ (par ailleurs \'egale \`a 
$\chi(\sF) = \sum (-1)^i \dim \h^i(X,\sF)$ peut \^etre calcul\'ee en terme de 
$x$, de rang de $\sF$ aux points g\'en\'eriques de $X$, et des propri\'et\'es 
de ramification de $\sF$. Le r\'esultat essentiel est le suivant. 

Soient $\bar X$ une courbe projective lisse et connexe, de genre $g$, sur un 
corps alg\'ebriquement clos $k$, $X$ un ouvert dense de $\bar X$, 
l'compl\'ement d'un ensemble fini $S$ de points et $\sF$ un 
$E_\lambda$-faisceau lisse sur $X$. Soient $\chi(X)=2-2 g-\# S$ la 
caract\'eristique d'Euler-Poincar\'e de $X$, et $\operatorname{rg}(\sF)$ le 
rang de $\sF$. Pour chaque point $s\in S$, on d\'efinit un entier 
$\swan_s(\sF)$, le conducteur de Swan, mesurant la ramification sauvage de 
$\sF$, et 
\begin{align*}\tag{3.2.1}\label{VI:eq:2-3-1}
  \chi_c(\sF) = \operatorname{rg}(\sF) \cdot \chi(X) - \sum_{s\in S} \swan_s(\sF) 
\end{align*}
\cite{ra65}. 





\begin{proposition_}\label{VI:3-8}
Soit $P\in \dF_q[X_1,\dots,X_n]$ un polyn\^ome \`a $n$ variable, de degr\'e 
$d$, dont on suppose qu'il n'est pas de la forme $Q^P-Q+C^{te}$. Posant encore 
$\psi(x) = \exp\left(\frac{2\pi i}{p} \tr_{\dF_q/\dF_p}(x)\right)$ on a 
\[
  \left|\sum_{x_i\in \dF_q} \psi P(x_1,\dots,x_n)\right| \leqslant (d+1) q^{n-\frac 1 2} \text{.}
\]
\end{proposition_}










\section{Sommes de Gauss et sommes de Jacob}\label{VI:4}

\begin{proposition_}\label{VI:4-2}
La cohomologie de $\dG_m$ \`a coefficient dans $\sF(\psi\chi^{-1})$ v\'erifie 
\begin{enumerate}[(i)]
  \item $\h_c^i=0$ pour $i\ne 1$, et $\dim \h_c^1=1$. 
  \item $F^\ast$, agissant sur $\h_c^1$, est la multiplication par 
    $\tau(\chi,\psi)$. 
  \item Si $\chi$ est non trivial, on a $\h_c^\bullet \iso \h^\bullet$. 
\end{enumerate}
\end{proposition_}




