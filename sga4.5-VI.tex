% !TEX root = sga4.5.tex

\chapter{Applications de la formule des traces aux sommes trigonométriques}\label{VI}
\chaptermark{Applications aux sommes trigonom\'etriques}










Dans cet expos\'e, j'explique comment la formule des traces permet de calculer 
ou d'\'etudier diverses sommes trigonom\'etriques et comment, jointe \`a la 
conjecture de Weil, elle peut permettre de les majorer.

Les deux premiers paragraphes donnent un ``mode d'emploi'' de ces outils. Le 
paragraphe 3 est un espos\'e, dans un langage cohomologique, des r\'esultats de 
Weil sur les sommes \`a $1$ variable. Les paragraphes $4$ \`a $6$ forment une 
\'etude d\'etaill\'ee des sommes de Gauss et de Jacobi -- y inclus les 
r\'esultats anciens et r\'ecents de Weil sur les caract\`eres de Hecke 
d\'efinis par des sommes de Jacobi. Au paragraphe $7$, nous \'etudions une 
g\'en\'eralisation \`a plusiers variables des sommes de Kloosterman. Enfin, au 
paragraphe $8$, on trouvera quelques indications sur d'autres usages qui ont 
\'et\'e faits on peuvent \^etre faits de ces m\'ethodes. 










\section*{Notations}\label{VI:0}





\subsection{}\label{VI:0-1}

On utilise les notations de \hyperref[II]{Rapport}, paragraphe \ref{II:1}. On 
aura souvent \`a consid\'erer une extension finie $\dF_{q^n}\subset \dF$ de 
$\dF_q$. On notera par un indice $0$ un objet sur $\dF_q$, et par un indice $1$ 
un objet sur $\dF_{q^n}$. Remplacer un indice $0$ par un indice $1$ (resp. 
supprimer l'incide) signifie qu'on \'etend les scalaires \`a $\dF_{q^n}$ (resp. 
\`a $\dF$).





\subsection{}\label{VI:0-2}

On d\'esigne par $\ell$ un nombre premier $\ne p$. Nous utilisons librement le 
language des $\dQ_\ell$-faisceaux, ainsii que celui des $E_\lambda$-faisceaux, 
pour $E_\lambda$ une extension finie de $\dQ_\ell$ (cf. \hyperref[II]{Rapport}, 
paragraphe \ref{II:2} et sp\'ecialement \ref{II:2-11}). 





\subsection{}\label{VI:0-3}

Soit $H^\bullet$ un espace vectoriel gradu\'e. Si $T$ est un endomorphisme de 
$H^\bullet$, on pose (cf. \hyperref[IV]{Cycle}, \ref{IV:eq:1-3-5}) 
\[
  \tr(T,H^\bullet) = \sum (-1)^i \tr(T,H^i) \text{.}
\]










\section{Principes}\label{VI:1}





\subsection{}\label{VI:1-1}

Soient $X_0$ un sch\'ema s\'epar\'e de type fini sur $\dF_q$, $E_\lambda$ une 
extension finie de $\dQ_\ell$ et $\sF_0$ un $E_\lambda$-faisceau sur $X-0$. La 
formule des traces dit que 
\begin{equation*}\tag{1.1.1}\label{VI:eq:1-1-1}
  \sum_{x\in X^F} \tr(F_x^\ast,\sF) = \sum (-1)^i \tr\left(F^\ast,\h_c^i(X,\sF)\right) \text{.}
\end{equation*}

Avec la notation \ref{VI:0-3}, le membre de droite s'\'ecrit simplement 
$\tr(F^\ast,\h_c^\bullet(X,\sF))$. 

Cette formule des traces pour les $E_\lambda$-faisceaux peut soit \^etre 
d\'eduite de \hyperref[II]{Rapport} \ref{II:4-10} par passage \`a limite (cf. 
\hyperref[II]{Rapport} \ref{II:4-11} \`a \ref{II:4-13}), soit \^etre d\'eduite 
de la formule des traces pour les $\dQ_\ell$-faisceaux (\hyperref[II]{Rapport} 
\ref{II:3-2}) par la m\'ethode de (\hyperref[III]{Functions $L$ mod. $\ell^n$}, 
\ref{III:4-3}). 

Nous allons interpr\'eter diverses sommes trigonom\'etriques comme le membre de 
gauche de \eqref{VI:eq:1-1-1}, pour $X_0$ et $\sF_0$ convenables. 





% NOTE: in this section, following the source (but not convention in the 
% LaTeX version, torsors are denoted in roman (not script) letters
\subsection{}\label{VE:1-2}

Soit $A$ un groupe fini. Un \emph{$A$-torseur} sur un sch\'ema $X$ est un 
faisceau $T$ sur $X$, muni d'une action \`a droite de $A$, qui, localement 
(pour la topologie \'etale) sur $X$, est isomorphe au faisceau constant $A$ sur 
lequel $A$ agit par translation \`a droite. 

Si $\tau:A\to B$ est un homomorphisme, et $T$ un $A$-torseur, il existe \`a 
isomorphisme unique pr\`es un et un seul $B$-torseur $\tau(T)$, muni d'un 
morphisme de faisceaux $\tau:T\to \tau(T)$ tel que 
\begin{equation*}\tag{1.2.1}\label{VI:eq:1-2-1}
  \tau(t a) = \tau(t) \tau(a)\text{.}
\end{equation*}

Si $T$ est un $A$-torseur sur $X$, et $\rho$ une repr\'esentation lin\'eaire 
de $A$: $\rho:A\to \gl(V)$ ($V$ un espace vectoriel de dimension finie sur 
$E_\lambda$), il existe \`a isomorphisme unique pr\`es un seul 
$E_\lambda$-faisceau $\sF$, lisse de rang $\dim(V)$, muni d'un morphisme de 
faisceaux 
\begin{equation*}\tag{1.2.2}\label{VI:eq:1-2-2}
  \rho:T\to \underline{\operatorname{Isom}}(V,\sF) \text{.}
\end{equation*}
tel que $\rho(t a) = \rho(t) \rho(a)$. On le note $\rho(T)$. Pour 
$\tau:A\to B$ et $\rho$ une repr\'esentation de $B$, on a un isomorphisme 
canonique $\rho(\tau(T)) = (\rho\tau)(T)$. 

Dans ce paragraphe (sauf l'appendice), on ne consid\`ere que le cas o\`u $A$ 
est commutatif et o\`u $V=E_\lambda$: $\rho$ est un caract\`ere 
$A\to E_\lambda^\times$, et $\rho(T)$ un $E_\lambda$-faisceau lisse de rang 
un. Le morphisme \eqref{VI:eq:1-2-2} s'interpr\`ete comme un morphisme partout 
non nul 
\begin{equation*}\tag{1.2.3}\label{VI:eq:1-2-3}
  \rho:T\to \rho(T) 
\end{equation*}
tel que $\rho(t a) = \rho(a)\rho(t)$. 





\subsection{}\label{VI:1-3}

Soient $S$ un sch\'ema, $G$ un sch\'ema en groupe commutatif sur $S$, et $G'$ 
une extension de sch\'emas en groupes commutatifs sur $S$
\[\xymatrix{
  0 \ar[r] 
    & A \ar[r] 
    & G' \ar[r]^-\pi 
    & G \ar[r] 
    & 0 \text{.}
}\]
Le faisceau $T$ des sections locales de $\pi$ est alors un $A$-torseur sur 
$G$. 

Si $X$ est un sch\'ema sur $S$, et $f,g$ deux $S$-morphismes de $X$ dans $G$, 
du fait que $T$ est d\'efini par une extension, on a un isomorphisme 
canonique 
\begin{equation*}\tag{1.3.1}\label{VI:eq:1-3-1}
  (f+g)^\ast T = f^\ast T + g^\ast T \text{.}
\end{equation*}
A gauche, $f+g$ est la somme de $f$ et $g$, au sens de la loi de groupe $G$: 
\`a droite, $+$ d\'esigne une somme de torseurs. 

Si $f:X\to G$ se factorise par $G'$, $f^\ast T$ est trivial (et un 
factorisation donne une trivialisation); \`a isomorphisme (non unique) pr\`es, 
le $A$-torseur $f^\ast T$ ne d\'epend que de l'image de $f$ dans 
$\hom_S(X,G)/\pi\hom_S(X,G')$: il est donn\'e par le morphisme $\partial$ dans 
la suite exacte 
\[\xymatrix{
  \hom(X,G') \ar[r] 
    & \hom(X,G) \ar[r]^-\partial 
    & \h^1(X,A) \text{.}
}\]

Nous appliquerons ces constructions aux suites exactes de Kummer 
$0 \to \dmu_n \to \dG_m \to \dG_m \to 0$, et aux torseurs de Lang. 





\subsection{Le torseur de Lang}\label{VI:1-4}

Soit $G_0$ un groupe alg\'ebrique commutatif connexe sur $\dF_q$ (pour le cas 
non commutatif, voir l'appendice). La loi de groupe est not\'ee 
\emph{multiplicativement}. L'isog\'enie de Lang 
\[
\fL:G_0 \to G_0: x\mapsto F x\cdot x^{-1}
\]
est \'etale; son image, un sous-groupe ouvert de $G_0$, ne peut \^etre que 
$G_0$ lui-m\^eme; son noyau est le groupe fini $G_0(\dF_q)$. Le \emph{torseur 
de Lang} $L$ est le $G_0(\dF_q)$-torseur sur $G_0$ d\'efini par la suite 
exacte 
\begin{equation*}\tag{1.4.1}\label{VI:eq:1-4-1}
\xymatrix{
  0 \ar[r] 
    & G_0(\dF_q) \ar[r] 
    & G_0 \ar[r]^-\fL 
    & G_0 \ar[r] 
    & 0 \text{.}
}
\end{equation*}


\paragraph{Notation}
On note encore $L$ et $\fL$ (ou $L_{(q)}$ et $\fL_{(q)}$) les objects d\'eduits 
de $L$ et $\fL$ par extension du corps de base. En cas de besoin, on 
pr\'ecisera par un indice $0$ ou $1$.


\paragraph{Exemples}
Pour $G_0=\dG_a$ ou $\dG_m$, les suites \eqref{VI:eq:1-4-1} s'\'ecrivent 
\begin{align*}\tag{1.4.2}\label{VI:eq:1-4-2}
&\xymatrix{
  0 \ar[r] 
    & \dF_q \ar[r] 
    & \dG_a \ar[r]^-{x^q-x} 
    & \dG_a \ar[r] 
    & 0 
} \\
\tag{1.4.3}\label{VI:eq:1-4-3}
&\xymatrix{
  0 \ar[r] 
    & \dmu_{q-1} \ar[r] 
    & \dG_m \ar[r]^-{x^{q-1}} 
    & \dG_m \ar[r] 
    & 0
}
\end{align*}





\subsection{}\label{VI:1-6}

Calculons l'endomorphisme $F^\ast$ de la fibre du torseur de Lang en 
$\gamma\in G^F=G_0(\dF_q)$. Si $g\in \fL^{-1}(\gamma)\subset G$, on a 
$F g = F g\cdot g^{-1}\cdot g = \gamma g$. D\`es lors (\hyperref[II]{Rapport}, 
\ref{II:1-2}) 
\begin{equation*}\tag{1.5.1}\label{VI:eq:1-5-1}
  \text{sur $L_0(G_0)_\gamma\simeq \fL^{-1}(\gamma)$, $F^\ast$ est $g\mapsto g\gamma^{-1}$.}
\end{equation*}





\subsection{}\label{VI:1-6}

Sur $\dF_{q^n}$, l'identit\'e $F_{(q^n)}=F_{(q)}^n$ entre endomorphismes de 
$G_1$ implique que $\fL_{(q^n)}=\fL_{(q)}\circ \prod_{i=0}^{n-1} F_{(q)}^i$. 
Sur $G_0(\dF_{q^n})$, $F_{(q)}^i$ agit comme l'\'el\'ement 
$x\mapsto x^{q^i}$ de $\gal(\dF_{q^n}/\dF_q)$. Sur $G_0(\dF_{q^n})$, 
$\prod F_{(q)}^i$ est donc le compos\'e de la norme 
$G_0(\dF_{q^n})\to G_0(\dF_q)$ et de l'inclusion 
$G_0(\dF_q)\subset G_0(\dF_{q^n})$: le diagramme 
\begin{equation*}\tag{1.6.1}\label{VI:eq:1-6-1}
\xymatrix{
  0 \ar[r] 
    & G_0(\dF_{q^n}) \ar[r] \ar[d]^-N 
    & G_1 \ar[r]^-{\fL_{(q^n)}} \ar[d]^-{\prod_{i=0}^{n-1} F_{(q)}^i} 
    & G_1 \ar[r] \ar@{=}[d] 
    & 0 \\
  0 \ar[r] 
    & G_0(\dF_q) \ar[r] 
    & G_q \ar[r]^-{\fL_{(q)}} 
    & G_1 \ar[r] 
    & 0
}
\end{equation*}
est commutatif. Dans le langage des torseurs, ceci fournit un isomorphisme 
canonique entre $G_0(\dF_q)$-torseurs sur $G_1$ 
\begin{equation*}\tag{1.6.2}\label{VI:eq:1-6-2}
  N L_{(q^n)} = L_{(q)} \text{.}
\end{equation*}





\begin{definition_}\label{VI:1-7}
Soient $f_0:X_0\to G_0$ un morphisme et 
$\chi:G_0(\dF_q)\to E_\lambda^\times$ un caract\`ere. On pose 
$\sF(\chi,f_0) = \chi^{-1}(f_0^\ast (L_0(G_0))) = f_0^\ast\chi^{-1}(L_0(G_0))$. 
\end{definition_}

On a les propri\'et\'es de fonctorialit\'e suivantes:
\begin{enumerate}[a)]
  \item $\sF(\chi,f_0)$ est bimultiplicatif en $\chi$ et $f_0$: 
    \begin{align*}\tag{1.7.1}\label{VI:eq:1-7-1}
      \sF(\chi,f_0'\cdot f_0'') &= \sF(\chi,f_0')\otimes \sF(\chi,f_0'') \text{,} \\
      \tag{1.7.2}\label{VI:eq:1-7-2}
      \sF(\chi'\chi'',\sF_0) &= \sF(\chi',f_0)\otimes \sF(\chi'',f_0) \text{.}
    \end{align*}
    Cela r\'esulte de la d\'efinition de $\chi(T)$, pour $T$ un torseur, joint 
    \`a \eqref{VI:eq:1-3-1} pour \eqref{VI:eq:1-7-1}. 
  \item Pour $g_0:Y_0\to X_0$ un morphisme, on a 
    \begin{equation*}\tag{1.7.3}\label{VI:eq:1-7-3}
      \sF(\chi,f_0\circ g_0) = g_0^\ast \sF(\chi,f_0) \text{.}
    \end{equation*}
    En particulier, $\sF(\chi,f_0) = f_0^\ast \sF(\chi)$, o\`u $\sF(\chi)$ 
    d\'esigne $\sF(\chi,\operatorname{id}_{G_0})$, 
  \item Pour $u_0:G_0\to H_0$ un morphisme, et 
    $\chi:H_0(\dF_q) \to E_\lambda^\times$, on a 
    \begin{equation*}\tag{1.7.4}\label{VI:eq:1-7-4}
      \sF(\chi,u_0 f_0) = \sF(\chi u_0,f_0) \text{.}
    \end{equation*}
  \item Pour $G_0=\prod_{i\in I} G_0^i$, $\chi$ de coordonn\'ees $\chi_i$ 
    ($i\in I$) et $f_0$ de coordonn\'ees $f_0^i$, il r\'esulte de 
    \eqref{VI:eq:1-7-1} \`a \eqref{VI:eq:1-7-4} que 
    \begin{equation*}\tag{1.7.5}\label{VI:eq:1-7-5}
      \sF(\chi,f_0) = \bigotimes_{i\in I} \sF(\chi_i,f_0^i) \text{.}
    \end{equation*}    
\end{enumerate}

Noter le $\chi^{-1}$ dans la d\'efinition \ref{VI:1-7}. Il assure que, pour 
$x\in X^F$, on ait sur $\sF(\chi,f_0)_x$ 
\begin{equation*}\tag{1.7.6}\label{VI:eq:1-7-6}
  F_x^\ast = \chi f_0(x)
\end{equation*}
(utiliser que la fibre en $x$ du morphisme \eqref{VI:eq:1-2-2}, pour 
$\rho=\chi^{-1}$, commute \`a $F_x^\ast$, et \eqref{VI:eq:1-5-1}). 

Si $\sF(\chi,f_0)_1$ est le faisceau d\'eduit de $\sF(\chi,f_0)$ par extension 
du corps de base \`a $\dF_{q^n}$, on d\'eduit de \eqref{VI:eq:1-6-2} que 
\begin{equation*}\tag{1.7.7}\label{VI:eq:1-7-7}
  \sF(\chi,f_0)_1 = \sF(\chi\circ N,f_1) \text{.}
\end{equation*}





\subsection{Abus de notations}\label{VI:1-8}

(i) Si $\Xi$ est une notation pour l'application compos\'ee 
$\chi f_0:X_0(\dF_q) \to E_\lambda^\times$, on \'ecrira parfois $\sF(\Xi)$ au 
lieu de $\sF(\chi,f_0)$. Grâce \`a \eqref{VI:eq:1-7-1} \`a \eqref{VI:eq:1-7-5}, 
on ne risque gu\`ere d'ambiguit\'e. Par exemple:
\begin{enumerate}[a)]
  \item on \'ecrit $\sF(\chi)$ pour $\sF(\chi,\operatorname{id}_{G_0})$ 
    (notations d\'ej\`a utilis\'ee en \ref{VI:1-7}b)); 
  \item en \'ecrit $\sF(\chi f_0)$ pour $\sF(\chi,f_0)$;
  \item avec les notations de \ref{VI:1-7}d), on \'ecrit 
    $\sF(\prod \chi_i f_0^i)$ pour $\sF(\chi,f))$. 
\end{enumerate}

Avec cette notation, \eqref{VI:eq:1-7-4} exprime que l'\'ecriture 
$\sF(\chi u_0 f_0)$ n'est pas ambiguë, \eqref{VI:eq:1-7-1} \eqref{VI:eq:1-7-2} 
\eqref{VI:eq:1-7-5} expriment une multiplicativit\'e de $\sF(\Xi)$ en $\Xi$, 
et \eqref{VI:eq:1-7-6} se r\'ecrit $F_x^\ast = \Xi(x)$ sur $\sF(\Xi)_x$.

(ii) Si $X$ est un sch\'ema sur une extension $k$ de $\dF_q$, et que $f$ est un 
morphisme de $X$ dans $G_0\otimes_{\dF_q} k$, on notera encore $\sF(\chi,f)$, 
$\sF(\chi f)$, ou $\sF(\chi)$ (pour $f$ une inclusion) l'image r\'eciproque par 
$f$ du $E_\lambda$-faisceau d\'eduit de $\chi^{-1}(L_0(G))$ par extension des 
scalaires de $\dF_q$ \`a $k$. Si $f_0$ est le compos\'e 
$X\to G_0\otimes_{\dF_q} k \to G_0$, c'est encore le $\sF(\chi,f_0)$ de 
\ref{VI:1-7}. Pour $k$ est une extension finie de $\dF_q$, on a 
$\sF(\chi f) = \chi(\chi N_{k/\dF_q} f)$. 

Appliquant \eqref{VI:eq:1-1-1} \`a \eqref{VI:eq:1-7-6} et \eqref{VI:eq:1-7-7}, 
on trouve: 





% NOTE: originally a 'scholium'
\begin{theorem_}\label{VI:1-9}
Soient $S_0$ un sch\'ema s\'epar\'e de type fini sur $\dF_q$, $G_0$ un groupe 
alg\'ebrique commutatif connexe sur $\dF_q$, $f_0:S_0\to G_0$ un morphisme et 
$\chi:G_0(\dF_q) \to E_\lambda^\times$. On a 
\begin{equation*}\tag{1.9.1}\label{VI:eq:1-9-1}
  \sum_{s\in S_0(\dF_q)} \chi f_0(s) = \tr(F^\ast,\h_c^\bullet(S,\sF(\chi,f_0))) 
\end{equation*}
et, pour tout entier $n\geqslant 1$
\begin{equation*}\tag{1.9.2}\label{VI:eq:1-9-2}
  \sum_{s\in S_0(\dF_{q^n})} \chi\circ N_{\dF_{q^n}/\dF_q} f_0(s) = \tr({F^\ast}^n, \h_c^\bullet\left(S,\sF(\chi,f_0))\right) \text{.}
\end{equation*}
\end{theorem_}





\subsubsection{Remarque}\label{VI:1-9-3}

Prenons pour $G_0$ un produit. Avec les notations de \ref{VI:1-7}d) et 
\ref{VI:1-8}c), la formule \eqref{VI:eq:1-9-2} devient 
\[
  \sum_{s\in S_0(\dF_{q^n})} \prod_i \chi_i\circ N_{\dF_{q^n}/\dF_q} (f_0^i(s)) = \tr\Big({F^\ast}^n, \h_c^\bullet\Big(S,\sF\Big(\prod_i \chi_i f_0^i\Big)\Big)\Big) \text{.}
\]





\subsubsection{Remarque}\label{VI:1-9-4}

Prenons $G_0=\{e\}$. La formule \eqref{VI:eq:1-9-1} devient 
\[
  \left|S_0(\dF_q)\right| = \sum_{s\in S_0(\dF_q)} 1 = \tr(F^\ast,\h_c^\bullet(S,\dQ_\ell)) \text{.}
\]

Il est rare qu'on puisse explicitement calculer le membre de droite de 
\eqref{VI:eq:1-9-1}. Voici un exemple amusant, avec $G_0=\{e\}$. 





\subsection{Exemple: Une quadrique projective non singuli\`ere de dimension impaire sur \texorpdfstring{$\dF_q$}{Fq} a le m\^eme nombre de points rationnels que l'espace projectif sur \texorpdfstring{$\dF_q$}{Fq} de la m\^eme dimension.}

Si, sur $\dC$, $X$ est une hypersurface quadrique non singuli\`ere dans 
l'espace projectif $\dP^{2 N}$, et $Y$ un hyperplan, on sait que pour 
$i\leqslant 2\dim(X)=2\dim(Y)$, les inclusions 
$X\hookrightarrow \dP^{2 N}\hookleftarrow Y$ induisent des isomorphismes (en 
cohomologie ordinaire) 
\[\xymatrix{
  \h^i(X,\dQ) 
    & \ar[l]_-\sim \h^i(\dP^{2 N},\dQ) \ar[r]^-\sim 
    & \h^i(Y,\dQ) \text{.}
}\]

Par sp\'ecialisation, il en r\'esulte que si $X_0'$ est une hypersurface 
quadrique non singuli\`ere dans l'espace projectif $\dP_0^{2 N}$ sur $\dF_q$, 
et $Y_0'$ un hyperplan, les inclusions 
$X_0'\hookrightarrow \dP_0^{2 N}\hookleftarrow Y_0'$ induisent des 
isomorphismes 
\[\xymatrix{
  \h^i(X',\dQ_\ell) 
    & \ar[l]_-\sim \h^i(\dP^{2 N},\dQ_\ell) \ar[r]^-\sim 
    & \h^i(Y',\dQ_\ell) \text{.}
}\]
Ces isomorphismes commutent \`a $F^\ast$, et on applique la formule des traces. 





\subsection{}\label{VI:1-11}

On peut aussi utiliser la formule des traces pour comprendre les formules 
classiques donnant le nombre de points rationnels des groupes lin\'eaires sur 
les corps finis, et ceux de certains espaces homog\`enes (voir le paragraphe 
8). 





\subsection{}\label{VI:1-12}

On dispose d'un dictionnaire permettant de traduire en termes cohomologiques 
ques divers types de manipulations classiques sur les sommes 
trigonom\'etriques. Ce dictionnaire sera donn\'e au paragraphe 2. L'\'enonc\'e 
cohomologique \'etant plus ``g\'eom\'etrique,'' il pourra parfois s'appliquer 
\`a des situations o\`u l'argument classique ne s'applique qu'apr\`es une 
extension du corps fini de base. Voici un exemple (un cas particulier d'un 
th\'eor\`eme de Kazhdan). 





\begin{theorem_}[Kazhdan]\label{VI:1-13}
Soit $X_0$ un sch\'ema sur $\dF_q$. Supposons qu'il existe une action $\rho$ de 
$\dG_a$ sur $X$, et un morphisme $f:X\to Y$ de sch\'emas sur $\dF$, faisant de 
$X$ un $\dG_a$-torseur sur $Y$. Alors, le nombre de points rationnels de $X_0$ 
est divisible par $q$.
\end{theorem_}

Supposons d'abord que $\rho$, $Y$ et $f$ soient d\'efinis sur $\dF_q$. 


\paragraph{Argument classique}
Les fibres de $f:X_0(\dF_q)\to Y_0(\dF_q)$ ont toutes $q$ \'el\'ements: 
$|X_0(\dF_q)| = q\cdot |Y_0(\dF_q)|$, d'o\`u la divisibilit\'e. 


\paragraph{Traduction cohomologique}
Les faisceaux images directes sup\'erieures $\R^i f_!\dQ_\ell$ sont 
\[
  \R^i f_!\dQ_\ell = \begin{cases}
                       \dQ_\ell(-1) & \text{si $i = 2$} \\
                       0            & \text{pour $i\ne 2$}
                     \end{cases}
\]
La suite spectrale de Leray de $f$ d\'eg\'en\`ere donc en un isomorphisme 
\[
  \h_c^i(X,\dQ_\ell) = \h_c^{i-2}(Y,\dQ_\ell)(-1) \text{.}
\]
Pour comprendre l'effet d'un twist \`a la Tate sur les valeurs propres de 
Frobenius, le plus commode est d'utiliser le point de vue galoisien 
(\hyperref[II]{Rapport} \ref{II:1-8}), et de noter que la substitution de 
Frobenius $\varphi$ agit sur $\dQ_\ell(1)$ par multiplication par $q$. Les 
valeurs propres de $F^\ast$ sur $\h_c^p(X,\dQ_\ell)$ sont donc les produits par 
$q$ des valeurs propres de $F^\ast$ agissant sur $\h_c^{i-2}(Y,\dQ_\ell)$. On 
sait que celles-ci sont des entiers alg\'ebriques \cite[XXI 5.2.2]{sga7}; d\`es 
lors 

\begin{equation*}\tag{$*$}\label{VI:eq:*}
  \substack{\displaystyle\text{les valeurs propres de $F^\ast$ agissant sur $\h_c^i(X,\dQ_\ell)$ sont} \\
  \displaystyle\text{des entiers alg\'ebriques divisibles par $q$.}}
\end{equation*}


\paragraph{Descente}
Revenons aux hypoth\`eses du th\'eor\`eme. Pour $n$ convenable, on peut 
supposer que $\rho$, $Y$ et $f$ sont d\'efinis sur $\dF_{q^n}$. Le morphisme de 
Frobenius relatif \`a $\dF_{q^n}$ \'etant la puissance $n$-i\`eme de celui 
relatif \`a $\dF_q$, l'\'enonc\'e \eqref{VI:eq:*} pour le sch\'ema sur 
$\dF_{q^n}$ d\'eduit de $X_0$ par extension des scalaires nous fournit: 
\begin{equation*}\tag{$**$}\label{VI:eq:**}
  \substack{\displaystyle\text{les puissances $n$-\`iemes des valeurs propres de $F^\ast$ agissant sur} \\ \displaystyle\text{$\h_c^i(X,\dQ_\ell)$ sont des entiers alg\'ebriques divisibles par $q^n$.}}
\end{equation*}

Cette assertion implique \eqref{VI:eq:*}. Le nombre de points rationnels de 
$X_0$, soit $\sum (-1)^i \tr(F^\ast,\h^i(X,\dQ_\ell))$, est donc entier 
alg\'ebrique divisible par $q$. Ceci implique qu'il soit divisible par $q$ en 
tant qu'entier rationnel. 




