% !TEX root = sga4.5.tex

\chapter{Applications de la formule des traces aux sommes trigonométriques}\label{VI}
\chaptermark{Applications aux sommes trigonométriques}










Dans cet exposé, j'explique comment la formule des traces permet de calculer 
ou d'étudier diverses sommes trigonométriques et comment, jointe à la 
conjecture de Weil, elle peut permettre de les majorer.

Les deux premiers paragraphes donnent un ``mode d'emploi'' de ces outils. Le 
paragraphe 3 est un exposé, dans un langage cohomologique, des résultats de 
Weil sur les sommes à $1$ variable. Les paragraphes $4$ à $6$ forment une 
étude détaillée des sommes de Gauss et de Jacobi -- y inclus les 
résultats anciens et récents de Weil sur les caractères de Hecke 
définis par des sommes de Jacobi. Au paragraphe $7$, nous étudions une 
généralisation à plusieurs variables des sommes de Kloosterman. Enfin, au 
paragraphe $8$, on trouvera quelques indications sur d'autres usages qui ont 
été faits on peuvent être faits de ces méthodes. 










\section*{Notations}\label{VI:0}





\subsection{}\label{VI:0-1}

On utilise les notations de \hyperref[II]{Rapport}, paragraphe \ref{II:1}. On 
aura souvent à considérer une extension finie $\dF_{q^n}\subset \dF$ de 
$\dF_q$. On notera par un indice $0$ un objet sur $\dF_q$, et par un indice $1$ 
un objet sur $\dF_{q^n}$. Remplacer un indice $0$ par un indice $1$ (resp. 
supprimer l'incide) signifie qu'on étend les scalaires à $\dF_{q^n}$ (resp. 
à $\dF$).





\subsection{}\label{VI:0-2}

On désigne par $\ell$ un nombre premier $\ne p$. Nous utilisons librement le 
langage des $\dQ_\ell$-faisceaux, ainsi que celui des $E_\lambda$-faisceaux, 
pour $E_\lambda$ une extension finie de $\dQ_\ell$ (cf. \hyperref[II]{Rapport}, 
paragraphe \ref{II:2} et spécialement \ref{II:2-11}). 





\subsection{}\label{VI:0-3}

Soit $H^\bullet$ un espace vectoriel gradué. Si $T$ est un endomorphisme de 
$H^\bullet$, on pose (cf. \hyperref[IV]{Cycle}, \ref{IV:eq:1-3-5}) 
\[
  \tr(T,H^\bullet) = \sum (-1)^i \tr(T,H^i) \text{.}
\]










\section{Principes}\label{VI:1}





\subsection{}\label{VI:1-1}

Soient $X_0$ un schéma séparé de type fini sur $\dF_q$, $E_\lambda$ une 
extension finie de $\dQ_\ell$ et $\sF_0$ un $E_\lambda$-faisceau sur $X-0$. La 
formule des traces dit que 
\begin{equation*}\tag{1.1.1}\label{VI:eq:1-1-1}
  \sum_{x\in X^F} \tr(F_x^\ast,\sF) = \sum (-1)^i \tr\left(F^\ast,\h_c^i(X,\sF)\right) \text{.}
\end{equation*}

Avec la notation \ref{VI:0-3}, le membre de droite s'écrit simplement 
$\tr(F^\ast,\h_c^\bullet(X,\sF))$. 

Cette formule des traces pour les $E_\lambda$-faisceaux peut soit être 
déduite de \hyperref[II]{Rapport} \ref{II:4-10} par passage à limite (cf. 
\hyperref[II]{Rapport} \ref{II:4-11} à \ref{II:4-13}), soit être déduite 
de la formule des traces pour les $\dQ_\ell$-faisceaux (\hyperref[II]{Rapport} 
\ref{II:3-2}) par la méthode de (\hyperref[III]{Fonctions $L$ mod. $\ell^n$}, 
\ref{III:4-3}). 

Nous allons interpréter diverses sommes trigonométriques comme le membre de 
gauche de \eqref{VI:eq:1-1-1}, pour $X_0$ et $\sF_0$ convenables. 





% NOTE: in this section, following the source (but not convention in the 
% LaTeX version, torsors are denoted in roman (not script) letters
\subsection{}\label{VI:1-2}

Soit $A$ un groupe fini. Un \emph{$A$-torseur} sur un schéma $X$ est un 
faisceau $T$ sur $X$, muni d'une action à droite de $A$, qui, localement 
(pour la topologie étale) sur $X$, est isomorphe au faisceau constant $A$ sur 
lequel $A$ agit par translation à droite. 

Si $\tau:A\to B$ est un homomorphisme, et $T$ un $A$-torseur, il existe à 
isomorphisme unique près un et un seul $B$-torseur $\tau(T)$, muni d'un 
morphisme de faisceaux $\tau:T\to \tau(T)$ tel que 
\begin{equation*}\tag{1.2.1}\label{VI:eq:1-2-1}
  \tau(t a) = \tau(t) \tau(a)\text{.}
\end{equation*}

Si $T$ est un $A$-torseur sur $X$, et $\rho$ une représentation linéaire 
de $A$: $\rho:A\to \gl(V)$ ($V$ un espace vectoriel de dimension finie sur 
$E_\lambda$), il existe à isomorphisme unique près un seul 
$E_\lambda$-faisceau $\sF$, lisse de rang $\dim(V)$, muni d'un morphisme de 
faisceaux 
\begin{equation*}\tag{1.2.2}\label{VI:eq:1-2-2}
  \rho:T\to \underline{\operatorname{Isom}}(V,\sF) \text{.}
\end{equation*}
tel que $\rho(t a) = \rho(t) \rho(a)$. On le note $\rho(T)$. Pour 
$\tau:A\to B$ et $\rho$ une représentation de $B$, on a un isomorphisme 
canonique $\rho(\tau(T)) = (\rho\tau)(T)$. 

Dans ce paragraphe (sauf l'appendice), on ne considère que le cas où $A$ 
est commutatif et où $V=E_\lambda$: $\rho$ est un caractère 
$A\to E_\lambda^\times$, et $\rho(T)$ un $E_\lambda$-faisceau lisse de rang 
un. Le morphisme \eqref{VI:eq:1-2-2} s'interprète comme un morphisme partout 
non nul 
\begin{equation*}\tag{1.2.3}\label{VI:eq:1-2-3}
  \rho:T\to \rho(T) 
\end{equation*}
tel que $\rho(t a) = \rho(a)\rho(t)$. 





\subsection{}\label{VI:1-3}

Soient $S$ un schéma, $G$ un schéma en groupe commutatif sur $S$, et $G'$ 
une extension de schémas en groupes commutatifs sur $S$
\[\xymatrix{
  0 \ar[r] 
    & A \ar[r] 
    & G' \ar[r]^-\pi 
    & G \ar[r] 
    & 0 \text{.}
}\]
Le faisceau $T$ des sections locales de $\pi$ est alors un $A$-torseur sur 
$G$. 

Si $X$ est un schéma sur $S$, et $f,g$ deux $S$-morphismes de $X$ dans $G$, 
du fait que $T$ est défini par une extension, on a un isomorphisme 
canonique 
\begin{equation*}\tag{1.3.1}\label{VI:eq:1-3-1}
  (f+g)^\ast T = f^\ast T + g^\ast T \text{.}
\end{equation*}
A gauche, $f+g$ est la somme de $f$ et $g$, au sens de la loi de groupe $G$: 
à droite, $+$ désigne une somme de torseurs. 

Si $f:X\to G$ se factorise par $G'$, $f^\ast T$ est trivial (et un 
factorisation donne une trivialisation); à isomorphisme (non unique) près, 
le $A$-torseur $f^\ast T$ ne dépend que de l'image de $f$ dans 
$\hom_S(X,G)/\pi\hom_S(X,G')$: il est donné par le morphisme $\partial$ dans 
la suite exacte 
\[\xymatrix{
  \hom(X,G') \ar[r] 
    & \hom(X,G) \ar[r]^-\partial 
    & \h^1(X,A) \text{.}
}\]

Nous appliquerons ces constructions aux suites exactes de Kummer 
$0 \to \dmu_n \to \dG_m \to \dG_m \to 0$, et aux torseurs de Lang. 





\subsection{Le torseur de Lang}\label{VI:1-4}

Soit $G_0$ un groupe algébrique commutatif connexe sur $\dF_q$ (pour le cas 
non commutatif, voir l'appendice). La loi de groupe est notée 
\emph{multiplicativement}. L'isogénie de Lang 
\[
\fL:G_0 \to G_0: x\mapsto F x\cdot x^{-1}
\]
est étale; son image, un sous-groupe ouvert de $G_0$, ne peut être que 
$G_0$ lui-même; son noyau est le groupe fini $G_0(\dF_q)$. Le \emph{torseur 
de Lang} $L$ est le $G_0(\dF_q)$-torseur sur $G_0$ défini par la suite 
exacte 
\begin{equation*}\tag{1.4.1}\label{VI:eq:1-4-1}
\xymatrix{
  0 \ar[r] 
    & G_0(\dF_q) \ar[r] 
    & G_0 \ar[r]^-\fL 
    & G_0 \ar[r] 
    & 0 \text{.}
}
\end{equation*}


\paragraph{Notation}
On note encore $L$ et $\fL$ (ou $L_{(q)}$ et $\fL_{(q)}$) les objects déduits 
de $L$ et $\fL$ par extension du corps de base. En cas de besoin, on 
précisera par un indice $0$ ou $1$.


\paragraph{Exemples}
Pour $G_0=\dG_a$ ou $\dG_m$, les suites \eqref{VI:eq:1-4-1} s'écrivent 
\begin{align*}\tag{1.4.2}\label{VI:eq:1-4-2}
&\xymatrix{
  0 \ar[r] 
    & \dF_q \ar[r] 
    & \dG_a \ar[r]^-{x^q-x} 
    & \dG_a \ar[r] 
    & 0 
} \\
\tag{1.4.3}\label{VI:eq:1-4-3}
&\xymatrix{
  0 \ar[r] 
    & \dmu_{q-1} \ar[r] 
    & \dG_m \ar[r]^-{x^{q-1}} 
    & \dG_m \ar[r] 
    & 0
}
\end{align*}





\subsection{}\label{VI:1-5}

Calculons l'endomorphisme $F^\ast$ de la fibre du torseur de Lang en 
$\gamma\in G^F=G_0(\dF_q)$. Si $g\in \fL^{-1}(\gamma)\subset G$, on a 
$F g = F g\cdot g^{-1}\cdot g = \gamma g$. Dès lors (\hyperref[II]{Rapport}, 
\ref{II:1-2}) 
\begin{equation*}\tag{1.5.1}\label{VI:eq:1-5-1}
  \text{sur $L_0(G_0)_\gamma\simeq \fL^{-1}(\gamma)$, $F^\ast$ est $g\mapsto g\gamma^{-1}$.}
\end{equation*}





\subsection{}\label{VI:1-6}

Sur $\dF_{q^n}$, l'identité $F_{(q^n)}=F_{(q)}^n$ entre endomorphismes de 
$G_1$ implique que $\fL_{(q^n)}=\fL_{(q)}\circ \prod_{i=0}^{n-1} F_{(q)}^i$. 
Sur $G_0(\dF_{q^n})$, $F_{(q)}^i$ agit comme l'élément 
$x\mapsto x^{q^i}$ de $\gal(\dF_{q^n}/\dF_q)$. Sur $G_0(\dF_{q^n})$, 
$\prod F_{(q)}^i$ est donc le composé de la norme 
$G_0(\dF_{q^n})\to G_0(\dF_q)$ et de l'inclusion 
$G_0(\dF_q)\subset G_0(\dF_{q^n})$: le diagramme 
\begin{equation*}\tag{1.6.1}\label{VI:eq:1-6-1}
\xymatrix{
  0 \ar[r] 
    & G_0(\dF_{q^n}) \ar[r] \ar[d]^-N 
    & G_1 \ar[r]^-{\fL_{(q^n)}} \ar[d]^-{\prod_{i=0}^{n-1} F_{(q)}^i} 
    & G_1 \ar[r] \ar@{=}[d] 
    & 0 \\
  0 \ar[r] 
    & G_0(\dF_q) \ar[r] 
    & G_q \ar[r]^-{\fL_{(q)}} 
    & G_1 \ar[r] 
    & 0
}
\end{equation*}
est commutatif. Dans le langage des torseurs, ceci fournit un isomorphisme 
canonique entre $G_0(\dF_q)$-torseurs sur $G_1$ 
\begin{equation*}\tag{1.6.2}\label{VI:eq:1-6-2}
  N L_{(q^n)} = L_{(q)} \text{.}
\end{equation*}





\begin{definition_}\label{VI:1-7}
Soient $f_0:X_0\to G_0$ un morphisme et 
$\chi:G_0(\dF_q)\to E_\lambda^\times$ un caractère. On pose 
$\sF(\chi,f_0) = \chi^{-1}(f_0^\ast (L_0(G_0))) = f_0^\ast\chi^{-1}(L_0(G_0))$. 
\end{definition_}

On a les propriétés de fonctorialité suivantes:
\begin{enumerate}[a)]
  \item $\sF(\chi,f_0)$ est bimultiplicatif en $\chi$ et $f_0$: 
    \begin{align*}\tag{1.7.1}\label{VI:eq:1-7-1}
      \sF(\chi,f_0'\cdot f_0'') &= \sF(\chi,f_0')\otimes \sF(\chi,f_0'') \text{,} \\
      \tag{1.7.2}\label{VI:eq:1-7-2}
      \sF(\chi'\chi'',\sF_0) &= \sF(\chi',f_0)\otimes \sF(\chi'',f_0) \text{.}
    \end{align*}
    Cela résulte de la définition de $\chi(T)$, pour $T$ un torseur, joint 
    à \eqref{VI:eq:1-3-1} pour \eqref{VI:eq:1-7-1}. 
  \item Pour $g_0:Y_0\to X_0$ un morphisme, on a 
    \begin{equation*}\tag{1.7.3}\label{VI:eq:1-7-3}
      \sF(\chi,f_0\circ g_0) = g_0^\ast \sF(\chi,f_0) \text{.}
    \end{equation*}
    En particulier, $\sF(\chi,f_0) = f_0^\ast \sF(\chi)$, où $\sF(\chi)$ 
    désigne $\sF(\chi,\operatorname{id}_{G_0})$, 
  \item Pour $u_0:G_0\to H_0$ un morphisme, et 
    $\chi:H_0(\dF_q) \to E_\lambda^\times$, on a 
    \begin{equation*}\tag{1.7.4}\label{VI:eq:1-7-4}
      \sF(\chi,u_0 f_0) = \sF(\chi u_0,f_0) \text{.}
    \end{equation*}
  \item Pour $G_0=\prod_{i\in I} G_0^i$, $\chi$ de coordonnées $\chi_i$ 
    ($i\in I$) et $f_0$ de coordonnées $f_0^i$, il résulte de 
    \eqref{VI:eq:1-7-1} à \eqref{VI:eq:1-7-4} que 
    \begin{equation*}\tag{1.7.5}\label{VI:eq:1-7-5}
      \sF(\chi,f_0) = \bigotimes_{i\in I} \sF(\chi_i,f_0^i) \text{.}
    \end{equation*}    
\end{enumerate}

Noter le $\chi^{-1}$ dans la définition \ref{VI:1-7}. Il assure que, pour 
$x\in X^F$, on ait sur $\sF(\chi,f_0)_x$ 
\begin{equation*}\tag{1.7.6}\label{VI:eq:1-7-6}
  F_x^\ast = \chi f_0(x)
\end{equation*}
(utiliser que la fibre en $x$ du morphisme \eqref{VI:eq:1-2-2}, pour 
$\rho=\chi^{-1}$, commute à $F_x^\ast$, et \eqref{VI:eq:1-5-1}). 

Si $\sF(\chi,f_0)_1$ est le faisceau déduit de $\sF(\chi,f_0)$ par extension 
du corps de base à $\dF_{q^n}$, on déduit de \eqref{VI:eq:1-6-2} que 
\begin{equation*}\tag{1.7.7}\label{VI:eq:1-7-7}
  \sF(\chi,f_0)_1 = \sF(\chi\circ N,f_1) \text{.}
\end{equation*}





\subsection{Abus de notations}\label{VI:1-8}

(i) Si $\Xi$ est une notation pour l'application composée 
$\chi f_0:X_0(\dF_q) \to E_\lambda^\times$, on écrira parfois $\sF(\Xi)$ au 
lieu de $\sF(\chi,f_0)$. Grâce à \eqref{VI:eq:1-7-1} à \eqref{VI:eq:1-7-5}, 
on ne risque guère d'ambiguité. Par exemple:
\begin{enumerate}[a)]
  \item on écrit $\sF(\chi)$ pour $\sF(\chi,\operatorname{id}_{G_0})$ 
    (notations déjà utilisée en \ref{VI:1-7}b)); 
  \item en écrit $\sF(\chi f_0)$ pour $\sF(\chi,f_0)$;
  \item avec les notations de \ref{VI:1-7}d), on écrit 
    $\sF(\prod \chi_i f_0^i)$ pour $\sF(\chi,f))$. 
\end{enumerate}

Avec cette notation, \eqref{VI:eq:1-7-4} exprime que l'écriture 
$\sF(\chi u_0 f_0)$ n'est pas ambiguë, \eqref{VI:eq:1-7-1} \eqref{VI:eq:1-7-2} 
\eqref{VI:eq:1-7-5} expriment une multiplicativité de $\sF(\Xi)$ en $\Xi$, 
et \eqref{VI:eq:1-7-6} se récrit $F_x^\ast = \Xi(x)$ sur $\sF(\Xi)_x$.

(ii) Si $X$ est un schéma sur une extension $k$ de $\dF_q$, et que $f$ est un 
morphisme de $X$ dans $G_0\otimes_{\dF_q} k$, on notera encore $\sF(\chi,f)$, 
$\sF(\chi f)$, ou $\sF(\chi)$ (pour $f$ une inclusion) l'image réciproque par 
$f$ du $E_\lambda$-faisceau déduit de $\chi^{-1}(L_0(G))$ par extension des 
scalaires de $\dF_q$ à $k$. Si $f_0$ est le composé 
$X\to G_0\otimes_{\dF_q} k \to G_0$, c'est encore le $\sF(\chi,f_0)$ de 
\ref{VI:1-7}. Pour $k$ est une extension finie de $\dF_q$, on a 
$\sF(\chi f) = \chi(\chi N_{k/\dF_q} f)$. 

Appliquant \eqref{VI:eq:1-1-1} à \eqref{VI:eq:1-7-6} et \eqref{VI:eq:1-7-7}, 
on trouve: 





% NOTE: originally a 'scholium'
\begin{theorem_}\label{VI:1-9}
Soient $S_0$ un schéma séparé de type fini sur $\dF_q$, $G_0$ un groupe 
algébrique commutatif connexe sur $\dF_q$, $f_0:S_0\to G_0$ un morphisme et 
$\chi:G_0(\dF_q) \to E_\lambda^\times$. On a 
\begin{equation*}\tag{1.9.1}\label{VI:eq:1-9-1}
  \sum_{s\in S_0(\dF_q)} \chi f_0(s) = \tr(F^\ast,\h_c^\bullet(S,\sF(\chi,f_0))) 
\end{equation*}
et, pour tout entier $n\geqslant 1$
\begin{equation*}\tag{1.9.2}\label{VI:eq:1-9-2}
  \sum_{s\in S_0(\dF_{q^n})} \chi\circ N_{\dF_{q^n}/\dF_q} f_0(s) = \tr({F^\ast}^n, \h_c^\bullet\left(S,\sF(\chi,f_0))\right) \text{.}
\end{equation*}
\end{theorem_}





\addtocounter{subsubsection}{2}
\subsubsection{Remarque}\label{VI:1-9-3}

Prenons pour $G_0$ un produit. Avec les notations de \ref{VI:1-7}d) et 
\ref{VI:1-8}c), la formule \eqref{VI:eq:1-9-2} devient 
\[
  \sum_{s\in S_0(\dF_{q^n})} \prod_i \chi_i\circ N_{\dF_{q^n}/\dF_q} (f_0^i(s)) = \tr\Big({F^\ast}^n, \h_c^\bullet\Big(S,\sF\Big(\prod_i \chi_i f_0^i\Big)\Big)\Big) \text{.}
\]





\subsubsection{Remarque}\label{VI:1-9-4}

Prenons $G_0=\{e\}$. La formule \eqref{VI:eq:1-9-1} devient 
\[
  \left|S_0(\dF_q)\right| = \sum_{s\in S_0(\dF_q)} 1 = \tr(F^\ast,\h_c^\bullet(S,\dQ_\ell)) \text{.}
\]

Il est rare qu'on puisse explicitement calculer le membre de droite de 
\eqref{VI:eq:1-9-1}. Voici un exemple amusant, avec $G_0=\{e\}$. 





\subsection{Exemple}\label{VI:1-10}

Une quadrique projective non singulière de dimension impaire sur \texorpdfstring{$\dF_q$}{Fq} a le même nombre de points rationnels que l'espace projectif sur \texorpdfstring{$\dF_q$}{Fq} de la même dimension.

Si, sur $\dC$, $X$ est une hypersurface quadrique non singulière dans 
l'espace projectif $\dP^{2 N}$, et $Y$ un hyperplan, on sait que pour 
$i\leqslant 2\dim(X)=2\dim(Y)$, les inclusions 
$X\hookrightarrow \dP^{2 N}\hookleftarrow Y$ induisent des isomorphismes (en 
cohomologie ordinaire) 
\[\xymatrix{
  \h^i(X,\dQ) 
    & \ar[l]_-\sim \h^i(\dP^{2 N},\dQ) \ar[r]^-\sim 
    & \h^i(Y,\dQ) \text{.}
}\]

Par spécialisation, il en résulte que si $X_0'$ est une hypersurface 
quadrique non singulière dans l'espace projectif $\dP_0^{2 N}$ sur $\dF_q$, 
et $Y_0'$ un hyperplan, les inclusions 
$X_0'\hookrightarrow \dP_0^{2 N}\hookleftarrow Y_0'$ induisent des 
isomorphismes 
\[\xymatrix{
  \h^i(X',\dQ_\ell) 
    & \ar[l]_-\sim \h^i(\dP^{2 N},\dQ_\ell) \ar[r]^-\sim 
    & \h^i(Y',\dQ_\ell) \text{.}
}\]
Ces isomorphismes commutent à $F^\ast$, et on applique la formule des traces. 





\subsection{}\label{VI:1-11}

On peut aussi utiliser la formule des traces pour comprendre les formules 
classiques donnant le nombre de points rationnels des groupes linéaires sur 
les corps finis, et ceux de certains espaces homogènes (voir le paragraphe 
8). 





\subsection{}\label{VI:1-12}

On dispose d'un dictionnaire permettant de traduire en termes cohomologiques 
ques divers types de manipulations classiques sur les sommes 
trigonométriques. Ce dictionnaire sera donné au paragraphe 2. L'énoncé 
cohomologique étant plus ``géométrique,'' il pourra parfois s'appliquer 
à des situations où l'argument classique ne s'applique qu'après une 
extension du corps fini de base. Voici un exemple (un cas particulier d'un 
théorème de Kazhdan). 





\begin{theorem_}[Kazhdan]\label{VI:1-13}
Soit $X_0$ un schéma sur $\dF_q$. Supposons qu'il existe une action $\rho$ de 
$\dG_a$ sur $X$, et un morphisme $f:X\to Y$ de schémas sur $\dF$, faisant de 
$X$ un $\dG_a$-torseur sur $Y$. Alors, le nombre de points rationnels de $X_0$ 
est divisible par $q$.
\end{theorem_}

Supposons d'abord que $\rho$, $Y$ et $f$ soient définis sur $\dF_q$. 


\paragraph{Argument classique}
Les fibres de $f:X_0(\dF_q)\to Y_0(\dF_q)$ ont toutes $q$ éléments: 
$|X_0(\dF_q)| = q\cdot |Y_0(\dF_q)|$, d'où la divisibilité. 


\paragraph{Traduction cohomologique}
Les faisceaux images directes supérieures $\eR^i f_!\dQ_\ell$ sont 
\[
  \eR^i f_!\dQ_\ell = \begin{cases}
                       \dQ_\ell(-1) & \text{si $i = 2$} \\
                       0            & \text{pour $i\ne 2$}
                     \end{cases}
\]
La suite spectrale de Leray de $f$ dégénère donc en un isomorphisme 
\[
  \h_c^i(X,\dQ_\ell) = \h_c^{i-2}(Y,\dQ_\ell)(-1) \text{.}
\]
Pour comprendre l'effet d'un twist à la Tate sur les valeurs propres de 
Frobenius, le plus commode est d'utiliser le point de vue galoisien 
(\hyperref[II]{Rapport} \ref{II:1-8}), et de noter que la substitution de 
Frobenius $\varphi$ agit sur $\dQ_\ell(1)$ par multiplication par $q$. Les 
valeurs propres de $F^\ast$ sur $\h_c^p(X,\dQ_\ell)$ sont donc les produits par 
$q$ des valeurs propres de $F^\ast$ agissant sur $\h_c^{i-2}(Y,\dQ_\ell)$. On 
sait que celles-ci sont des entiers algébriques \cite[XXI 5.2.2]{sga7}; dès 
lors 

\begin{equation*}\tag{$*$}\label{VI:eq:*}
  \substack{\displaystyle\text{les valeurs propres de $F^\ast$ agissant sur $\h_c^i(X,\dQ_\ell)$ sont} \\
  \displaystyle\text{des entiers algébriques divisibles par $q$.}}
\end{equation*}


\paragraph{Descente}
Revenons aux hypothèses du théorème. Pour $n$ convenable, on peut 
supposer que $\rho$, $Y$ et $f$ sont définis sur $\dF_{q^n}$. Le morphisme de 
Frobenius relatif à $\dF_{q^n}$ étant la puissance $n$-ième de celui 
relatif à $\dF_q$, l'énoncé \eqref{VI:eq:*} pour le schéma sur 
$\dF_{q^n}$ déduit de $X_0$ par extension des scalaires nous fournit: 
\begin{equation*}\tag{$**$}\label{VI:eq:**}
  \substack{\displaystyle\text{les puissances $n$-\`iemes des valeurs propres de $F^\ast$ agissant sur} \\ \displaystyle\text{$\h_c^i(X,\dQ_\ell)$ sont des entiers algébriques divisibles par $q^n$.}}
\end{equation*}
Cette assertion implique \eqref{VI:eq:*}. Le nombre de points rationnels de 
$X_0$, soit $\tr(F^\ast,\h^\bullet(X,\dQ_\ell))$, est donc entier 
algébrique divisible par $q$. Ceci implique qu'il soit divisible par $q$ en 
tant qu'entier rationnel. 





\subsection{}\label{VI:1-14}

La formule \ref{VI:1-9-3} contrôle la dépendance en $n$ de la somme 
trigonométrique au membre de gauche. Elle implique des identités entre une 
somme trigonométrique et celles qui s'en déduisent par ``extension des 
scalaires.'' Le plus célèbre de ces identités est celle de 
Hasse-Davenport: soient $\chi:\dF_q^\times \to \dC^\times$ et 
$\psi:\dF_q\to \dC^\times$ des caractères, $\psi$ non trivial, et 
définissons la somme de Gauss $\tau(\chi,\psi)$ par 
\begin{equation*}\tag{1.14.1}\label{VI:eq:1-14-1}
  \tau(\chi,\psi) = - \sum_{x\in \dF_q^\times} \psi(x) \chi^{-1}(x) \text{.}
\end{equation*}





\begin{theorem_}[Hasse-Davenport]\label{VI:1-15}
On a 
\begin{equation*}\tag{1.15.1}\label{VI:eq:1-15-1}
  \tau\left(\chi\circ N_{\dF_{q^n}/\dF_q},\psi\circ \tr_{\dF_{q^n}/\dF_q}\right) = \tau(\chi,\psi)^n \text{.}
\end{equation*}
\end{theorem_}

Soient $E\subset \dC$ un corps de nombres contenant les valeurs de $\chi$ et 
$\psi$, et $E_\lambda$ le complété de $E$ en une place de caractéristique 
$\ell\ne p$. L'identité \eqref{VI:eq:1-15-1} est une identité dans $E$. Il 
revient au même de la prouver dans $\dC$, ou dans $E_\lambda$. Nous la 
prouverons dans $E_\lambda$, en regardant $\chi$ et $\psi$ comme à valeurs 
dans $E_\lambda^\times$. 

On applique \ref{VI:1-9-3} pour $X_0=\dG_m$ sur $\dF_q$ et 
$G_0=\dG_m\times \dG_a$. Les $\h_c^i(\dG_m,\sF(\chi^{-1}\psi))$ sont nuls pour 
$i\ne 1$, et le $\h_c^1$ est de dimension $1$ (\ref{VI:4-2}). D'après 
\ref{VI:1-9-3}, 
$\tau(\chi\circ N_{\dF_{q^n}/\dF_q},\psi\circ \tr_{\dF_{q^n}/\dF_q})$ est 
l'unique valeur propre de $(F^\ast)^n$ sur $\h_c^1$, et \eqref{VI:eq:1-15-1} en 
résulte. 

Des exemples analogues sont donnés dans Weil \cite[App V]{we74}. 





\subsection{}\label{VI:1-16}

Un $E_\lambda$-faisceau $\sF_0$ sur $X-0$ est dit \emph{ponctuellement de 
poids $n$} si pour tout point fermé $x\in |X_0|$, les valeurs propres de 
$F_x^\ast$ (\hyperref[II]{Rapport} \ref{II:1-2}) sont des nombres algébriques 
dont tous les conjugués complexes sont de valeur absolue 
$q_x^{n/2}$, où $q_x$ est le nombre d'éléments de $k(x)$. Le théorème 
suivant sera démontré dans \cite{de80}. 





\begin{theorem_}\label{VI:1-17}
Si $\sF_0$ est ponctuellement de poids $n$, pour toute valeur propre $\alpha$ 
de l'endomorphisme $F^\ast$ de $\h_c^i(X,\sF)$, il existe un entier 
$m\leqslant n+i$ tel que les conjugués complexes de $\alpha$ soient tous de 
valeurs absolue $q^{m/2}$.
\end{theorem_}

Les faisceaux considérés en \ref{VI:1-9} sont de poids $0$. Le théorème 
fournit donc pour la somme \ref{VI:eq:1-9-1} (on plutôt, pour tous ses 
conjugués complexes) la majoration 
\begin{equation*}\tag{1.17.1}\label{VI:eq:1-17-1}
  \left|\sum_{s\in S_0(\dF_q)} \prod_i \chi_i(f^i(s)) \right| \leqslant \sum_i \dim \h_c^i\left(S,\sF\left(\prod \chi_i f^i\right)\right)\cdot q^{i/2} \text{.}
\end{equation*}





\subsection{Remarques}\label{VI:1-18}

\begin{enumerate}[a)]
  \item Si $\dim S=n$, pour tout $E_\lambda$-faisceau $\sF$ sur $S$, on a 
    \[
      \h_c^i(S,\sF) = 0 \qquad \text{pour $i\notin [0,2n]$.}
    \]
  \item Le groupe $\h_c^0(X,\sF)$ est le groupe des sections globales de $\sF$ 
    sur $S$ dont le support est propre. Si $S$ est connexe, non complet et 
    $\sF$ lisse, on bien si $S$ est connexe, $\sF$ lisse de rang un, et non 
    constant, ce groupe est nul.
  \item Si $S$ est lisse purement de dimension $n$, et $\sF$ lisse, la 
    dualité de Poincaré dit que les espaces vectoriels 
    $\h_c^i(X,\sF)$ et $\h^{2n-i}(X,\sF^\vee)(2n)$ sont duaux l'un de l'autre 
    (pour l'effet d'un twist à la Tate sur les valeurs propres de Frobenius, 
    cf. la $2$ème partie de la preuve de \ref{VI:1-13}).
  \item Si $\dim S=n$, la dualité de Poincaré permet de calculer comme 
    suite $\h_c^{2n}(S,\sF)$: on prend un ouvert $U$ de $S_\text{red}$, dont le 
    complémentaire est de dimension $<n$, lisse purement de dimension $n$, et 
    sur lequel $\sF$ est lisse. On a alors 
    $\h_c^{2n}(U,\sF) \iso \h_c^{2n}(S,\sF)$, car dans la suite exacte longue 
    de cohomologie, $\h_c^i(S\setminus U,\sF) = 0$ pour $i=2n,2n-1$. Par 
    Poincaré, on a donc 
    \[
      \h_c^{2n}(S,\sF) \iso \left(\h^0(U,\sF^\vee)(2n)\right)^\vee \text{.}
    \]
    
    Supposons $U$ connexe, et soit $u$ un point géométrique de $U$. Le 
    ``système local'' $\sF$ correspond à une représentation de 
    $\pi_1(U,u)$ sur $\sF_u$, et $\h^0(U,\sF^\vee)$ est 
    $(\sF_u^\vee)^{\pi_1(U,u)}=((\sF_u)_{\pi_1(U,u)})^\vee$ (dual des 
    coinvariants). On a donc 
    \[
      \h_c^{2n}(S,\sF) \simeq (\sF_u)_{\pi_1(U,u)}(-2n) \text{.}
    \]
  \item Ces remarques permettent souvent le calcul de $b_0$ et $b_{2n}$. 
    Calculer les autres $b_i$ peut être difficile. Si $b_{2n}=0$ et que 
    $b_{2n-1}=0$, la majoration \eqref{VI:eq:1-17-1} est une majoration à la 
    Lang-Weil, avec, pour $q$ grand, un gain en $\sqrt q$ par rapport à la 
    majoration triviale en $O(q^n)$. On prendra toutefois garde que la 
    majoration \eqref{VI:eq:1-17-1} n'implique pas formellement la majoration 
    triviale 
    \[
      \left|\sum_{s\in S_0(\dF_q)} \prod_i \chi_i(f^i(s))\right| \leqslant \left| S_0(\dF_q)\right| \text{,}
    \]
    donc il peut être utile de tenir compte (cf. la preuve de \ref{VI:3-8}). 
\end{enumerate}

J'explique ci-dessous une méthode pour prouver que $b_i=0$ pour $i\ne n$. 
Quand elle s'applique, elle fournit une estimation en $O(q^{n/2}$. La constante 
implicite dans $0$ est $b_n=(-1)^n \chi(S,\sF(\prod \chi_i f^i))$ en peut 
être difficile à calculer. 





\begin{proposition_}\label{VI:1-19}
Soient $\bar X$ un schéma propre sur un corps alg\'briquement clos $k$, 
$j:X\hookrightarrow bar X$ un ouvert de $X$ et $\sF$ un $E_\lambda$-faisceaux 
sur $X$. On suppose que 
\begin{enumerate}[\indent a)]
  \item $(j_\ast\sF)_x = 0$ pour $x\in \bar X\setminus X$, i.e. 
    $j_!\sF\iso j_\ast\sF$; 
  \item $\eR^i j_\ast \sF = 0$ pour $i>0$.
\end{enumerate}
Alors, les applications 
\[
  \h_c^i(X,\sF) \to \h^i(X,\sF)
\]
sont des isomorphismes. 
\end{proposition_}

Les hypothèses a) b) signifient que $j_! \sF \iso \eR j_\ast \sF$, d'où 
\[
  \h_c^i(X,\sF) = \h^i(\bar X,j_!\sF) \iso \hh^i(\bar X,\eR j_\ast \sF) = \h^i(X,\sF) \text{.}
\]
Autrement dit, la suite spectrale de Leray pour $j$ 
\[
  E_2^{pq} = \h^p(\bar X,\eR^q j_\ast \sF) \Rightarrow \h^{p+q}(X,\sF) 
\]
dégénère, par b) en des isomorphismes 
\[
  \h^i(\bar X,j_\ast\sF) \iso \h^i(X,\sF) \text{,}
\]
tandis que, par a) 
\[
  \h_c^i(X,\sF) = \h^i(\bar X,j_!\sF) \iso \h^i(\bar X,j_\ast\sF) \text{.}
\]





\subsubsection{Exemple}\label{VI:1-19-1}

Si $X$ est une courbe, ou si $\bar X$ est lisse, $X$ le complément d'un 
diviseur à croisements normaux et $\sF$ modérément ramifié, 
l'hypothèse a) de \ref{VI:1-19} (pas d'invariants sous la monodromie locale) 
implique l'hypothèse b). 





\begin{proposition_}\label{VI:1-20}
Soient $X_0$ un schéma affine lisse purement de dimension $n$ sur $\dF_q$ et 
$\sF_0$ un $E_\lambda$-faisceau lisse ponctuellement de poids $m$ sur $X_0$. On 
suppose que les applications $\h_c^i(X,\sF) \to \h^i(X,\sF)$ sont des 
isomorphismes (cf. \ref{VI:1-17}). Alors, $\h_c^i(X,\sF) = 0$ pour $i\ne n$, et 
les conjugués complexes des valeurs propres de $F^\ast$ sur $\h_c^n(X,\sF)$ 
sont de valeur absolue $q^{(m+n)/2}$. 
\end{proposition_}

La dualité de Poincaré met en dualité $\h_c^i(X,\sF)$ (resp. 
$\h_c^i(X,\sF^\vee)$) avec $\h^{2n-i}(X,\sF^\vee(n))$ (resp. 
$\h^{2n-i}(X,\sF(n))$). D'après \cite[XIV 3.2]{sga4}, les $\h^i$ sans support 
sont nuls pour $i>n$. Par dualité, les $\h_c^i$ sont nuls pour $i<n$; puisque 
$\h_c^i(X,\sF) \iso \h^i(X,\sF)$, ces groupes sont nuls pour $i\ne n$. 
Appliquons \ref{VI:1-15} à $\sF_0$ et à $\sF_0^\vee(n)$ (de poinds 
$-m-2n$). On trouve que les conjugués complexes $\alpha$ des valeurs propres 
de $F^\ast$ sur $\h_c^n(X,\sF) = \h_c^n(X,\sF^\vee(n))^\vee$ vérifient 
\begin{align*}
  |\alpha| &\leqslant q^{(m+n)/2} && \text{et} \\
  |\alpha^{-1}| &\leqslant q^{((-m-2n)+n)/2} = q^{-(m+n)/2} \text{,}
\end{align*}
d'où l'assertion. 





\subsection{Remarque}\label{VI:1-21}

Le dernier argument montre que si $X_0$ est un schéma séparé lisse sur 
$\dF_q$, que $\sF_0$ est un $E_\lambda$-faisceau lisse ponctuellement de poids 
$m$ sur $X_0$ et que $\h_c^i(X,\sF) \hookrightarrow \h^i(X,\sF)$, alors les 
conjugués complexes des valeurs propres de $F^\ast$ sur $\h_c^i(X,\sF)$ sont 
de valeur absolue $q^{(m+i)/2}$. 





\subsection*{Appendice. L'isogénie de Lang dans le cas non commutatif}




\subsection{}\label{VI:1-22}

Soient $G-0$ un groupe algébrique connexe sur $\dF_q$ et $\fL$ le morphisme 
$x\mapsto F x\cdot x^{-1}$ de $G-0$ dans lui-même. C'est un orphisme de 
$G_0$-espaces homogènes, si à la source on fait agir $G-0$ par translation 
à gauche $x\mapsto (y\mapsto x y)$, au but par 
$x\mapsto (y\mapsto F x\cdot y\cdot x^{-1}$). On a $\fL(xy) = \fL(x)$ pour 
$y\in G_0(\dF_q)$, et $\fL$ induit un isomorphisme $G_0/G_0(\dF_q)\iso G_0$. 

Comme dans le cas commutatif, $\fL$ fait donc de $G_0$ un $G_0(\dF_q)$-torseur 
sur $G_0$, le \emph{torseur de Lang} $L$. Si 
$\rho:G_0(\dF_q) \to \gl(n,E_\lambda)$ est une représentation linéaire de 
$G_0(\dF_q)$, nous noterons $\sF(\rho)$ le $E_\lambda$-faisceau $\rho(L)$ 
(\ref{VI:1-2}). 





\subsection{}\label{VI:1-23}

Soit $\gamma\in G^F=G_0(\dF_q)$, et calculons l'endomorphisme $F^\ast$ de 
$\sF(\rho)_\gamma$. Soit $g\in G$ tel que $\fL(g)=\gamma$, d'où un repère 
$\rho(g)\in \operatorname{Isom}(E_\lambda^n,\sF(\rho)_\gamma)$. Pour 
$e\in E_\lambda^n$, on a 
\[
  F(\rho(g)(e)) = \rho(F g)(e) = \rho(\gamma g)(e) = \rho(g\cdot g^{-1}\gamma g)(e) \text{.}
\]
On verra que $g^{-1} \gamma g\in G_0(\dF_q)$, d'où 
\[
  F(\rho(g)(e)) = \rho(g) \rho(g^{-1}\gamma g)(e)
\]
et 
\begin{equation*}\tag{1.23.1}\label{VI:eq:1-23-1}
  \rho(g)^{-1} (F_\gamma^\ast)^{-1} \rho(g) = \rho(g^{-1} \gamma g) \text{:}
\end{equation*}
$\rho(g)$ identifie l'inverse de $F^\ast$ en $\gamma$ à l'automorphisme 
$\rho(g^{-1} \gamma g)$ de $E_\lambda^n$. Reste à comprendre ce qu'est 
$g^{-1} \gamma g$. 





\begin{lemma_}\label{VI:1-24}
\begin{enumerate}[(i)]
  \item Si $F g\cdot g^{-1} \in G_0(\dF_q)$, on a 
    $g^{-1} (F g\cdot g^{-1}) g = g^{-1} F g\in G_0(\dF_q)$.
  \item Soient $\gamma\in G_0(\dF_q)$, et $g$ tel que 
    $F g\cdot g^{-1} = \gamma$. La classe de conjugaison de $\gamma'=g^{-1} Fg$ 
    dans $G_0(\dF_q)$ ne dépend que de celle de $\gamma$. 
  \item L'application induite par $\gamma\mapsto \gamma'$, de l'ensemble 
    $G_0(\dF_q)^\natural$ des classes de conjugaison de $G_0(\dF_q)$ dans 
    lui-même, est bijective. Son inverse est l'application pour le groupe 
    opposé. 
\end{enumerate}
\end{lemma_}

(i) Si $\gamma= F g\cdot g^{-1}\in G_0(\dF_q)$, les identités $F g=\gamma g$ 
et 
$F\gamma=\gamma$ donnent $F(g^{-1} F g) = (F g)^{-1} F F g = g^{-1} \gamma^{-1} F(\gamma g) = g^{-1} \gamma^{-1} F \gamma F g = g^{-1} F g$, 
de sorte que $g^{-1} F g \in G_0(\dF_q)$. 

(ii) Pour $\alpha,\gamma\in G_0(\dF_q)$, les $g$ tels que 
$F g\cdot g^{-1} = \gamma$ forment une classe à droite sous $G_0(\dF_q)$, et 
si $F g\cdot g^{-1} = \gamma$, on a 
$F(\alpha g \alpha^{-1})(\alpha g \alpha^{-1})^{-1} = \alpha \gamma\alpha^{-1}$. 
Pour $\gamma$ parcourant une classe de conjugaison dans $G_0(\dF_q)$, les $g$ 
tels que $F g\cdot g^{-1} = \gamma$ parcourent donc une double classe sous 
$G_0(\dF_q)$, et $\gamma'= g^{-1} F g$ parcourt une classe de conjugaison. 

(iii) Enfin, l'inverse de $F g\cdot g^{-1}\mapsto g^{-1} F g$ est 
$g^{-1} F g\mapsto F g\cdot g^{-1}$. 





\subsection{}\label{VI:1-25}

La formule \eqref{VI:eq:1-23-1} peut se lire, brièvement: $F^\ast$ en 
$\gamma$ est $\rho(\gamma')^{-1}$. Si $\gamma$ appartient à la composante 
neutre de son centralisateur, on peut prendre $g$ dans $Z(\gamma)$. On a alors 
$\gamma=\gamma'$. Si cette condition n'est pas remplie, $\gamma$ et $\gamma'$ 
peuvent ne pas être conjugués. 





\subsection{Un exemple}\label{VI:1-26}

Pour $G_0=\operatorname{SL}(2)$, $q$ impair, 
$\alpha\in F_q^\times\setminus {\dF_q^\times}^2$ et $\bar\gamma$ la classe de 
conjugaison de $-1(1+N)$, avec $N^2=0$, $\bar\gamma'$ est la classe de 
conjugaison de $-1\cdot (1+\alpha N)$, de sorte que $\bar\gamma'\ne \bar\gamma$ 
si $N\ne 0$. 










\section{Dictionnaire}\label{VI:2}

Une manipulation élémentaire sur les sommes trigonométriques peut souvent 
être vue comme le reflet, via la formule des traces, d'un énoncé 
cohomologique. Dans ce paragraphe, j'énumère de tels énoncés, et leur 
reflet; il portent le même numéro, augmenté d'une $\ast$ pour 
l'énoncé cohomologique. 





\subsection{}\label{VI:2-1}

``Une somme d'entiers algébriques est un entier algébrique'' admet pour 
analogue cohomologique le théorème suivant, prouvé dans 
\cite[XXI 5.2.2]{sga7} (cf. l'usage de \ref{VI:2-1}$^\ast$ en 
\ref{VI:1-13}). 





\begin{theorem*}[2.1*]\label{VI:2-1*}
Soient $X_0$ un schéma séparé de type fini sur $\dF_q$ et $\sF_0$ un 
$E_\lambda$-faisceau sur $X_0$. Si, pour tout $x\in |X_0|$, les valeurs 
propres de $F_x^\ast$ sur $\sF_0$ sont des entiers algébriques, alors les 
valeurs propres de $F^\ast$ sur $\h_c^i(X,\sF)$ sont des entiers algébriques. 
\end{theorem*}





\subsection{}\label{VI:2-2}

D'autres résultats d'intégralités sont prouvés dans 
\cite[XXI, 5.2.2 et 5.4]{sga7}: si $\dim(x_0)\leqslant n$, et que $\alpha$ est 
une valeur propre de $F^\ast$ sur $\h_c^i(X,\sF)$, on a: 
\begin{enumerate}[a)]
  \item sous les hypothèses de \ref{VI:2-1}$^\ast$, et si $i>n$, alors 
    $q^{i-n}$ divise $\alpha$;
  \item si les inverses des valeurs propres de $F_x^\ast$ sont des entiers 
    algébriques, alors $\alpha^{-1}$ est entier, sauf en $p$. Plus 
    précisément, $q^{\inf(n,i)}\alpha^{-1}$ est un entier algébrique. 
\end{enumerate}

Pour les faisceaux considérés en \ref{VI:1-9}, les valeurs propres des 
$F_x^\ast$ sont des racines de l'unité, de sorte que les hypothèses du 
théorème et celle de b) ci-dessus sont vérifiées. 





\subsection{}\label{VI:2-3}

Soient $f:A\to B$ une application d'un ensemble fini dans un autre, et 
$\varepsilon$ un fonction sur $A$. On a 
\begin{equation*}\tag{2.3.1}\label{VI:eq:2-3-1}
  \sum_{a\in A} \varepsilon(a) = \sum_{b\in B} \sum_{f(a)=b} \varepsilon(a) \text{.}
\end{equation*}





\subsection*{2.3*}\label{VI:2-3*}

Soient $f:X\to Y$ un morphisme de schémas séparés de type fini sur $k$ 
algébriquement clos, et $\sF$ un $E_\lambda$-faisceau sur $X$. La suite 
spectrale de Leray de $f$ en cohomologie à support propre s'écrit 
\begin{align*}\tag{2.3.1*}\label{VI:eq:2-3-1*}
  E_2^{p q} = \h_c^p(Y,\eR^q f_! \sF) \Rightarrow \h_c^{p+q}(X,\sF) \text{.}
\end{align*}

Soient $f_0:X_0 \to Y_0$ un morphisme de schémas séparés de type fini sur 
$\dF_q$ et $\sF_0$ un $E_\lambda$-faisceau sur $X_0$. On a entre 
\eqref{VI:eq:2-3-1} et \eqref{VI:eq:2-3-1*} la compatibilité suivante. 

\begin{enumerate}[a)]
  \item Le faisceau $\eR^q f_! \sF$ se déduit par extension des scalaires de 
    $\dF_q$ à $\dF$ du faisceau $\eR^q f_{0!} \sF_0$ sur $Y_0$. En tant que 
    tel, il est muni d'une correspondance de Frobenius 
    $F^\ast:F^\ast\eR^q f_! \sF \to \eR^q f_! \sF$. Elle se déduit par 
    fonctorialité de $\eR f_!$ de la correspondance de Frobenius de $\sF$: 
    c'est le composé 
    \[\xymatrix{
      F^\ast \eR^q f_! \sF \ar[r]^-\sim 
        & \eR^q f_! F^\ast \sF \ar[r]^-\sim 
        & \eR^q f_! \sF \text{.}
    }\]
  \item La formule des traces pour $\sF_q$ s'écrit 
    \begin{align*}\tag{1}\label{VI:eq:1}
      \sum_{x\in X_0(\dF_q)} \tr(F_x^\ast,\sF_0) = \tr(F^\ast,\h^\bullet(X,\sF)) \text{;}
    \end{align*}
    celle pour $\eR^q f_! \sF$ s'écrit 
    \begin{align*}\tag{2$_q$}\label{VI:eq:2_q} 
      \sum_{y\in Y_0(\dF_q)} \tr(F_y^\ast,\eR^q f_! \sF) = \tr(F^\ast, \h^\bullet(Y,\eR^q f_! \sF)) \text{.}
    \end{align*}
\end{enumerate}

On a, pour $y\in Y(\dF)$, $(\eR^q f_! \sF)_y = \h_c^q(f^{-1}(y),\sF)$, et si 
$y\in Y_0(\dF_q)$, 
\[
  \tr(F_y^\ast,\eR^q f_! \sF) = \tr(F^\ast,\h_c^q(f^{-1}(y),\sF)) \text{,}
\]
d'où par la formule des traces sur la fibre 
\[
  \sum_{\substack{x\in X_0(\dF_q) \\ f(x) = y}} \tr(F_x^\ast,\sF_0) = \sum (-1)^q \tr(F_y^\ast,\eR^q f_! \sF) \text{.}
\]
Prenant la somme alternée des identités \eqref{VI:eq:2_q}, on trouve donc 
\begin{align*}\tag{2}\label{VI:eq:2}
  \sum_{y\in Y_0(\dF_q)} \sum_{\substack{x \in X_0(\dF_q) \\ f(x) = y}} \tr(F_x^\ast,\sF_0) = \tr(F^\ast,\h^\bullet(Y,\eR^\bullet f_! \sF)) \text{.}
\end{align*}

L'égalité des membres de gauche de \eqref{VI:eq:1} et \eqref{VI:eq:2} 
résulte de \eqref{VI:eq:2-3-1}, celle des membres de droite de 
\eqref{VI:eq:2-3-1*} (compte tenu du fait que $F^\ast$ est un endomorphisme de 
toute la suite spectrale). 





\subsection{}\label{VI:2-4}

Soient $(A_i)_{i\in I}$ une famille finie d'ensembles finis, 
$A=\prod_{i\in I} A_i$, $\varepsilon_i$ une fonction sur $A_i$, et 
$\varepsilon$ la fonction $a\mapsto \prod \varepsilon_i(a_i)$ sur $A$. On a 
\begin{align*}\tag{2.4.1}\label{VI:eq:2-4-1}
  \sum_{a\in A} \varepsilon(a) = \prod_{i\in I} \sum_{a_i\in A_i} \varepsilon_i(a_i) \text{.}
\end{align*}





\subsection*{2.4*}\label{VI:2-4*}

Soient $(X_i)_{i\in I}$ une famille finie de schémas séparés de type fini 
sur $k$ algébriquement clos, $X=\prod_{i\in I} X_i$, $\sF_i$ un 
$E_\lambda$-faisceau sur $X_i$ et $\sF$ le produit tensoriel externe 
$\bigboxtimes_{i\in I}\sF_i = \bigotimes_{i\in I} \pr_i^\ast(\sF_i)$ 
des $\sF_i$. On a la formule de K\"unneth 
\begin{align*}\tag{2.4.1*}\label{VI:eq:2-4-1*}
  \h_c^\bullet(X,\sF) = \bigotimes_{i\in I} \h_c^\bullet(X_i,\sF_i) \text{.}
\end{align*}

Si on veut un isomorphisme \eqref{VI:eq:2-4-1*} canonique, et indépendant du 
choix d'un ordre total sur $I$, il faut prendre au membre droite le produit 
tensoriel gradué au sens de la règle de Koszul 
(\hyperref[IV]{Cycle}, \ref{V:1-3}). 





\subsection{}\label{VI:2-5}

Soient $A$ un ensemble fini, $B$ une partie de $A$ et $\varepsilon$ une 
fonction sur $A$. On a 
\begin{align*}\tag{2.5.1}\label{VI:2-5-1}
  \sum_{a\in A} \varepsilon(a) = \sum_{a\in A\setminus B}\varepsilon(a) + \sum_{a\in B} \varepsilon(a) \text{.}
\end{align*}





\subsection*{2.5*}\label{VI:2-5*}

Soient $x$ un schéma séparé de type fini sur $k$ algébriquement clos, 
$Y$ un sous-schéma fermé et $\sF$ un $E_\lambda$-faisceaux sur $X$. On a 
une suite exacte longue de cohomologie 
\begin{align*}\tag{2.5.1*}\label{VI:eq:2-5-1*}
\xymatrix{
  \cdots \ar[r]^-\partial 
    & \h_c^i(X\setminus Y,\sF) \ar[r] 
    & \h_c^i(X,\sF) \ar[r] 
    & \h_c^i(Y,\sF) \ar[r]^-\partial 
    & \cdots 
}
\end{align*}

Plus généralement, si on part d'une filtration finie de $X$ par des parties 
fermées $X_p$ ($p\in \dZ$; on suppose que $X_p\supset X_{p+1}$, que $X_p=X$ 
pour $p$ assez petit, et que $X_p=\varnothing$ for $p$ assez grand), on a une 
suite spectrale 
\begin{align*}\tag{2.5.2*}\label{VI:eq:2-5-2*}
  E_1^{p q} = \h_c^{p+1}(X_p\setminus X_{p+1},\sF) \Rightarrow \h_c^{p+q}(X,\sF) \text{.}
\end{align*}





\subsection{}\label{VI:2-6}

Soient $A$ un ensemble fini, $(A_i)_{i\in I}$ un recouvrement fini de $A$ et 
$\varepsilon$ une fonction sur $A$. Pour $J\subset I$, soit $A_J$ 
l'intersection des $A_j$ ($j\in J$). On a 
\begin{align*}\tag{2.6.1}\label{VI:eq:2-6-1}
  \sum_{J\subset I} (-1)^{|J|} \sum_{a\in A_J} \varepsilon(a) = 0 \text{.}
\end{align*}





\subsection*{2.6*}\label{VI:2-6*}

Soient $X$ un schéma séparé de type fini sur $k$ algébriquement clos, 
$(X_i)_{i\in I}$ un recouvrement fermé (resp. ouvert) fini de $X$ par des 
sous-schémas, et $\sF$ un $E_\lambda$-faisceau sur $X$. Pour $J\subset I$, 
soit $X_J$ l'intersection des $X_j$ ($j\in J$). On a des suites spectrales 
respectives (de Leray) 
\begin{align*}\tag{2.6.1*}\label{VI:eq:2-6-1*}
  E_1^{p q} &= \bigoplus_{|J|=p+1>0} \h_c^q(X_J,\sF) \otimes_\dZ \textstyle\bigwedge^{|J|} \dZ^J \Rightarrow \h_c^{p+q}(X,\sF) \\ 
  \tag{2.6.2*}\label{VI:eq:2-6-2*}
  E_1^{p q} &= \bigoplus_{|J|=1-p>0} \h_c^q(X_J,\sF) \otimes_\dZ \textstyle\bigwedge^{|J|} \dZ^J \Rightarrow \h_c^{p+q}(X,\sF)
\end{align*}
Si $J=\varnothing$ n'était pas exclus, on aurait de même des suites 
spectrales convergeant vers $0$. Les facteurs $\bigwedge^{|J|} \dZ^J$ sont là 
pour nous dispenser de choisir un ordre total sur $I$. 





\subsection{}\label{VI:2-7}

Soient $A$ un groupe commutatif fini, et $\chi:A \to E_\lambda^\times$ un 
caractère non trivial. On a 
\begin{align*}\tag{2.7.1}\label{VI:eq:2-7-1}
  \sum_{a\in A} \chi(a) = 0 \text{.}
\end{align*}
On prouvera l'analogue cohomologique de \eqref{VI:eq:2-7-1} en en transposant 
la preuve suivante: si $x\in A$, $a\mapsto x a$ est une permutation de $A$, et 
\begin{align*}
  \sum_{a\in A}\chi(a) = \sum_{a\in A} \chi(x a) &= \chi(x) \sum_{a\in A} \chi(a) && \text{, d'où} \\
  (\chi(x)-1)\sum_{a\in A} &= 0 \text{,}
\end{align*}
et il existe par hypothèse $x$ tel que $\chi(x)-1\ne 0$. 





\begin{theorem*}[2.7*]\label{VI:2-7*}
Soient $G_0$ un groupe algébrique commutatif connexe sur $\dF_q$ et 
$\chi:G_0(\dF_q) \to E_\lambda^\times$ un caractère non triviale. On a 
\begin{align*}\tag{2.7.1*}\label{VI:eq:2-7-1*}
  \h_c^\bullet(G,\sF(\chi)) = 0 \text{.}
\end{align*}
\end{theorem*}

Pour $x$ un point rationnel de $G$, notons $t_x$ la translation $t_x(g) = x g$ 
de $G$. La formule $\fL t_x = t_{\fL(x)} \fL$ exprime que $(t_x,t_{\fL(x)})$ 
est un automorphisme du diagramme $G\xrightarrow\fL G$ ($G$ muni du torseur de 
Lang). Soit $\rho(g)$ l'automorphisme de $(G,\sF(\chi))$ qui s'en déduit. 
Pour $g\in G_0(\dF_q)$, i.e. pour $\fL(g) = e$, c'est l'identité sur $G$, et 
la multiplication par $\chi(g)^{-1}$ sur $\sF(\chi)$. 

Notons $\rho_H(g)$ l'automorphisme de $\h_c^\bullet(G,\sF(\chi))$ déduit de 
$\rho(g)$. Un argument d'homotopie (Lemme \ref{VI:2-8} ci-dessous) montre que 
$\rho_H(g) = \rho_H(e)$, donc est l'identité. D'autre part, pour 
$g\in G_0(\dF_q)$, $\rho_H(g)$ est la multiplication par 
$(\chi^{-1}(g):\text{ sur }\h^\bullet(G,\sF(\chi))$, la multiplication par 
$(\chi^{-1}(g)-1)$ est nulle. Prenant $g$ tel que $\chi(g)\ne 1$, on obtient 
\ref{VI:2-7*}. 





\begin{lemma_}\label{VI:2-8}
Soient $X$ et $Y$ deux schémas sur $k$ algébriquement clos, avec $X$ 
séparé de type fini et $Y$ connexe. Soient $\sF$ un faisceau sur $X$ et 
$(\rho,\varepsilon)$ une famille d'endomorphismes de $(X,\sF)$ paramétrée 
par $Y$: 
\begin{align*} 
  \rho:Y\times_k X &\to Y\times_k X && \text{est un $Y$-morphisme, et} \\
  \varepsilon:\rho^\ast \pr_2^\ast \sF &\to \pr_2^\ast\sF && \text{un morphisme de faisceaux.} 
\end{align*}
On suppose que $\rho$ est propre. Pour $y\in Y(k)$, soit $\rho_H(y)^\ast$ 
l'endomorphisme de $\h_c^\bullet(X,\sF)$ induit par $\rho_y:X\to X$ et 
$\varepsilon_y:\rho_y^\ast\sF\to \sF$. Alors, $\rho_H(y)^\ast$ est 
indépendant de $y$. 
\end{lemma_}

En effet, $\eR^p \pr_{1!} \pr_2^\ast\sF$ est le 
faisceau constant sur $Y$ de valeur $\h^p(X,\sF)$, et $\rho_H(y)^\ast$ est la 
fibre en $y$ de l'endomorphisme 
\[\xymatrix{
  \eR^p \pr_{1!} \pr_2^\ast \sF \ar[r]^-{\rho^\ast} 
    & \eR^p \pr_{1!} \rho^\ast \pr_2^\ast \sF \ar[r]^-\varepsilon 
    & \eR^p \pr_{1!} \pr_2^\ast \sF 
}\]
de ce faisceau. 





\subsection{Remarque}\label{VI:2-9}

Soient $G_0$ un groupe algébrique connexe sur $\dF_q$ et 
$\rho:G_0(\dF_q) \to \operatorname{GL}(V)$ un représentation linéaire 
telle que $V^{G_0(\dF_q)}=0$. On a encore $\h_c^\bullet(G,\rho(L))=0$. 










\section{Sommes à une variable}\label{VI:3}





\subsection{}\label{VI:3-1}

Weil est le premier à avoir appliqué des méthodes ``cohomologiques'' à 
l'étude des sommes trigonométriques à une variable; puisqu'il avait 
prouvé l'analogue de l'hypothèse de Riemann pour les fonctions $L$ d'Artin 
sur les corps de fonctions, ces méthodes lui fournissaient d'excellentes 
estimations (en $O(\text{racine carrée du nombre de termes})$), et le 
comportement de ces sommes par ``extension du corps de base.'' 

L'essentiel de ce paragraphe est un exposé, dans un langage cohomologique, de 
ses résultats. 





\subsection{}\label{VI:3-2}

Les méthodes cohomologiques amènent ici à calculer des groupes 
$\h_c^i(X,\sF)$, pour $\sF_0$ un $E_\lambda$-faisceau sur un courbe $X-0$ sur 
$\dF_q$. Ces groupes sont nuls pour $i\ne 0,1,2$ et pour 
$i=0,2$, ils ont une interprétation simple (\ref{VI:1-8} a,b,c). De plus, la 
caractéristique d'Euler-Poincaré 
$\chi_c(\sF) = \sum (-1)^i \dim \h_c^i(X,\sF)$ (par ailleurs égale à 
$\chi(\sF) = \sum (-1)^i \dim \h^i(X,\sF)$ peut être calculée en terme de 
$x$, de rang de $\sF$ aux points génériques de $X$, et des propriétés 
de ramification de $\sF$. Le résultat essentiel est le suivant. 

Soient $\bar X$ une courbe projective lisse et connexe, de genre $g$, sur un 
corps algébriquement clos $k$, $X$ un ouvert dense de $\bar X$, 
le complément d'un ensemble fini $S$ de points et $\sF$ un 
$E_\lambda$-faisceau lisse sur $X$. Soient $\chi(X)=2-2 g-\# S$ la 
caractéristique d'Euler-Poincaré de $X$, et $\operatorname{rg}(\sF)$ le 
rang de $\sF$. Pour chaque point $s\in S$, on définit un entier 
$\swan_s(\sF)$, le conducteur de Swan, mesurant la ramification sauvage de 
$\sF$, et 
\begin{align*}\tag{3.2.1}\label{VI:eq:3-2-1}
  \chi_c(\sF) = \operatorname{rg}(\sF) \cdot \chi(X) - \sum_{s\in S} \swan_s(\sF) 
\end{align*}
\cite{ra65}. 





\subsection{}\label{VI:3-3}

Soient $\bar X$, $j:X\hookrightarrow\bar X$ et $\sF$ comme en \ref{VI:3-2}. Le 
théorème de dualité de Poincaré admet la forme très maniable: 
$\h^i(\bar X,j_\ast \sF)$ et $\h^{2-i}(\bar X,j_\ast(\sF^\vee(1)))$ sont 
duaux l'un de l'autre (\hyperref[V]{Dualité}, \ref{V:1-3}). 





\subsection{}\label{VI:3-4}

Faisons $k=\dF$, et supposons que $X$, $\bar X$, et $\sF$ proviennent de 
$X_0$, $\bar X_0$ et $\sF_0$ sur $\dF_q$. Si $\sF_0$ devient trivial sur un 
revêtement fini de $X_0$, Weil a prouvé que les conjugués complexes des 
valeurs propres de $F^\ast$ sur $\h^i(\bar X,j_\ast\sF)$ sont de valeur 
absolue $q^{i/2}$. De tels faisceaux $\sF_0$ correspondent aux fonctions $L$ 
d'Artin sur le corps de fonctions de $\bar X_0$. 





\subsection*{3.5}\label{VI:3-5_} % originally one of two 3.5s

Sous les mêmes hypothèses, ou plus généralement si $E_\lambda$ est le 
complété en une place $\lambda$ d'un corps de nombres $E$ et que $\sF_0$ 
appartient à un système compatible infini de représentations $v$-adiques 
($v$ place de $E$), on peut calculer 
\[
  \prod_i \det(-F^\ast,\h^i(X,\sF))^{(-1)^i} 
\]
à partir d'informations \emph{locales} sur $\sF_0$. Ceci est explique dans 
\cite{de73}. Pour $\sF_0$ trivial sur un revêtement fini de $X_0$, c'est 
l'expression de la constante de l'équation fonctionnelle d'une fonction $L$ 
d'Artin comme produit de constantes locales. 





\subsection{Exemple}\label{VI:3-5}

Soit $\psi$ le caractère 
$\exp\left(\frac{2\pi i}{p} \tr_{\dF_q/\dF_p}\right)$ de $\dF_q$; on a 
$\psi(a^p-a)=0$. Soient $X_0$ un courbe projective lisse absolument 
irréductible de genre $g$ sur $\dF_q$, et $f$ une fonction rationnelle sur 
$X_0$, i.e. un morphisme $f:X_0 \to \dP^1$, non identiquement égale à 
$\infty$. On s'intéresse à la somme 
\[
  S_f' = \sum_{\substack{x\in X_0(\dF_q) \\ f(x)\ne \infty}} \psi(f(x)) 
  \text{.}
\]
Cette somme est nulle pour $f$ de la forme $g^p-g$. Ceci suggère de 
modifier la somme $S_f'$ comme suite: 
\begin{enumerate}[\indent a)]
  \item pour tout point fermé $x$ de $X_0$, on pose $v_x(f) = $ ordre du 
    pole de $f$ en $x$ si $f(x)=\infty$, $v_x(f)=0$ sinon, et 
    $v_x^\ast(f) = \inf v_x(f+g^p-g)$ (borne inférieure sur $g$). 
  \item Si $v_x^\ast(f)=0$, que $x\in X_0(\dF_q)$, et que $f+g^p-g$ est 
    régulier en $x$, on pose $\psi(f(x)) = \psi((f+g^p-g)(x))$. On pose enfin 
    \begin{equation*}\tag{3.5.1}\label{VI:eq:3-5-1}
      S_f = \sum'_{x\in X_0(\dF_q)} \psi(f(x)) \text{,}
    \end{equation*}
    où $(-)'$ indique que la somme est étendue aux $x$ tels que 
    $v_x^\ast(f)=0$. 
\end{enumerate}

On a $S_f=S_{f+g^p-g}$. On se propose de vérifier que si $f$ n'est pas la 
forme $g^p-g+c^{t e}$, alors 
\begin{equation*}\tag{3.5.2}\label{VI:eq:3-5-2}
  |S_f|\leqslant \left(2 g-2+\sum_{v_x^\ast(f)\ne 0} [k(x):\dF_q] (1+v_x^\ast(f))\right) q^{1/2} \text{.}
\end{equation*}

On commence par passer du complexe au $\ell$-adique, en remplaçant $\psi$ par 
un caractère de la forme $\psi_0\circ \tr_{\dF_q/\dF_p}$, avec 
$\psi_0:\dF_p\hookrightarrow E_\lambda^\times$. Pour $j:U_0\hookrightarrow X_0$ 
l'inclusion de l'ouvert où $f\ne\infty$, on preuve alors que 
\begin{equation*}\tag{3.5.3}\label{VI:eq:3-5-3}
  S_f = \sum (-1)^i \tr(F^\ast,\h^i(X,j_\ast \sF(\psi f))) \text{.} 
\end{equation*}

La formule des traes ramène cet énoncé aux suivants (pour 
$x\in X_0(\dF_q)$): 
\begin{enumerate}[\indent a)]
  \item Si $v_x^\ast(f)=0$, alors $F_x^\ast$ sur $j_\ast \sF(\psi f)$ est 
    $\psi(f(x))$; 
  \item Si $v_x^\ast(f)\ne 0$, alors $j_\ast \sF(\psi f)$ se ramifie en $x$: 
    $(j_\ast \sF(\psi f))_{\bar x}=0$. 
\end{enumerate}

La formule a) résulte de \eqref{VI:eq:1-7-6} si $v_x(f)=0$, et on se ramène 
à ce cas en remplaçant $f$ par $f+g^p-g$: à isomorphisme près, ceci ne 
change pas $\sF(\psi f)_0$ sur l'ouvert où $f$ et $g$ sont réguliers 
(\ref{VI:1-3}). 

La formule b) résulte de 
\begin{equation*}\tag{3.5.4}\label{VI:eq:3-5-4}
  \swan_x \sF(f,\psi) = v_x^\ast(f) \text{.} 
\end{equation*}
Après réduction au cas où $v_x(f)=v_x^\ast(f)$ (d'où $p\nmid v_x(f)$) 
et extension des scalaires à $\dF$, cette formule est dans 
\cite[4.4]{se61}. 

Enfin, on vérifie que le faisceau $\sF(\psi f)$ sur $U$ est non constant si 
$f$ n'est pas de la forme $g^p-g+c^{t e}$: puisque $\psi_0$ est injectif, la 
trivialité de $\sF(\psi f)=\sF(\psi_0 f)$ (\ref{VI:1-8}.ii) équivaut à 
celle du $\dF_q$-torseur d'équation serait l'image réciproque d'un 
$\dF_q$-torseur sur $\spec(\dF_q)$, d'équation $T^p-T-\lambda=0$, le 
$\dF_p$-torseur d'équation $T^p-T-(f-\lambda)=0$ sur $U_0$ serait trivial 
et $f$ serait donc de la forme $g^p-g+\lambda$. Les 
$\h^i(\bar X,j_\ast \sF(\psi f))$ sont donc nuls pour $i\ne 1$, la formule 
\eqref{VI:eq:3-5-3} se réduit à 
\begin{equation*}\tag{3.5.5}\label{VI:eq:3-5-5}
  S_f = -\tr(F^\ast,\h^1(\bar X,j_\ast \sF(\psi f))) 
\end{equation*}
et, d'après \eqref{VI:eq:3-2-1} et \eqref{VI:eq:3-5-4}, de $\h^1$ est de 
dimension $2 g-2+\sum_{v_x^\ast(f)>0} [k(x):\dF_q](1+v_x^\ast(f))$ et $-S_f$ 
est somme de ce nombre de valeurs propres de $F^\ast$, chacune de valeurs 
absolues complexes $q^{1/2}$. 





\subsection{}\label{VI:3-6}

Supposons que, pour un automorphisme $\sigma$ de $X_0$, on ait 
\[
  f(\sigma x) = - f(x) \text{.} 
\]
La somme $S_f$ est alors réelle. Si $\sigma$ est involutif, la dualité de 
Poincaré permet de dire un peu plus: si on pose 
$\sF_0=j_\ast \sF(\psi f)$, $\sG_0 = j_\ast \sF(\psi(-f))$, on a 
\begin{enumerate}[\indent a)]
  \item $\sF_0$ et $\sG_0$ sont en dualité, 
  \item $\sigma^\ast \sF_0 \iso \sG_0$ et $\sigma^\ast \sG_0 \iso \sF_0$, pour 
    des isomorphismes naturels tel que le composé 
    $\sF_0=(\sigma^2)^\ast \sF_0 \iso \sigma^\ast \sG_0 \iso \sF_0$ soit 
    l'identité. 
  \item L'accouplement $\sF_0\otimes \sG_0 \to E_\lambda$ vérifie 
    $\sigma^\ast(f\cdot g) = \sigma^\ast(f) \cdot \sigma^\ast (g)$. 
\end{enumerate} 

Passant à cohomologie, on trouve que $\h^1(X,\sF)$ et $\h^1(X,\sF)$ sont en 
dualité parfaite à valeur dans $E_\lambda(1)$, et que la forme bilinéaire 
$\alpha\cdot \sigma^\ast\beta$ sur $\h^1(X,\sF)$ est \emph{alternée}: 
$\sigma^\ast$ agit trivialement sur $E_\lambda(1)$, et 
\[
  \alpha\cdot \sigma^\ast \beta = \sigma^\ast(\alpha\cdot \sigma^\ast \beta) = \sigma^\ast \alpha\cdot \beta = -\beta\cdot \sigma^\ast \alpha \text{.} 
\]
On en conclut que $\h^1(X,\sF)$ est de dimension paire (on vérifie d'ailleurs 
facilement que chaque $v_x^\ast(f)$ non nul est impair) et que les valeurs 
propres de $F^\ast$ sur $\h^1(X,\sF)$ sont groupées en pairs $\alpha$ et 
$q/\alpha$. 





\subsection{Exemple}\label{VI:3-7}

Ceci s'applique aux sommes de Kloosterman 
$\sum_{x\in \dF_q^\times} \psi\left(x+\frac a x\right)$ (faire 
$X_0=\dP^1$, $f=x+\frac a x$, $\sigma x=-x$; les pôles de $f$ sont $0$ et 
$\infty$, et en chacun d'eux $v_x^\ast(f)=1$; le $\h^1$ est de dimension 
$-2+2+2=2$). On a donc (pour $a\ne 0$) 
\[
  \sum_{\substack{x y=a \\ x,y\in \dF_{q^n}}} \exp\left(\frac{2\pi i}{p} \tr_{\dF_{q^n}/\dF_p}(x+y)\right) = (-\alpha^n+\alpha^{-n}) \qquad \text{, }\alpha\bar \alpha = q \text{.} 
\]
Ce résultat est dû à L.\ Carlitz \cite{ca69}. 

Voici maintenant une application d'une méthode de Lang-Weil. 





\begin{proposition_}\label{VI:3-8}
Soit $P\in \dF_q[X_1,\dots,X_n]$ un polynôme à $n$ variable, de degré 
$d$, dont on suppose qu'il n'est pas de la forme $Q^p-Q+C^{te}$. Posant encore 
$\psi(x) = \exp\left(\frac{2\pi i}{p} \tr_{\dF_q/\dF_p}(x)\right)$ on a 
\[
  \left|\sum_{x_i\in \dF_q} \psi P(x_1,\dots,x_n)\right| \leqslant (d+1) q^{n-\frac 1 2} \text{.}
\]
\end{proposition_}

On a pour le membre de gauche l'estimation triviale $q^n$. Il suffit donc de 
prouver \ref{VI:3-8} lorsque $d-1<q^{1/2}$; supposons seulement que 
$d<q+1$, et soit $P_d$ la partie homogène de degré $d$ de $P$. Rappelons le 





\begin{lemma_}\label{VI:3-9}
Une hypersurface de degré $d$ dans $\dP^r$ ($r\geqslant i$) ne peut passer 
par tout les points rationnels sur $\dF_q$ qui se $d\geqslant q+1$. 
\end{lemma_}

On procède par récurrence: s'il existe un hyperplan rationnel non 
entièrement contenu dans l'hypersurface (tel n'est pas le cas pour $r=1$), on 
applique l'hypothèse de récurrence à la trace de l'hypersurface sur cet 
hyperplan. Sinon, le degré $d$ est $\geqslant$ le nombre d'hyperplans 
rationnel, $\geqslant q+1$. 

Distinguons maintenant deux cas. 


\paragraph{Cas 1: $p\nmid d$.}
Appliquant le lemma à $P_d$, on voit que, quitte à faire un changement 
linéaire de variables, on peut supposer que $P_d(1,0,\ldots,0)\ne 0$, i.e. 
que le coefficient de $X_1^d$ dans $P$ est non nul. Un polynôme à $1$ 
variable 
\[
  S(X) = \sum_{i=0}^d a_i x^i \text{,} 
\]
avec $a_d\ne 0$ ($p\nmid d$) n'est jamais de la forme $Q^p-Q+C^{t e}$, et 
l'estimation \eqref{VI:eq:3-5-2} se réduit à 
\[
  \left| \sum \psi(S(x)) \right| \leqslant (d-1) q^{1/2} \text{.} 
\]
Appliquant cette estimation aux sommes partielles obtenues en ne faisant varier 
que $x_1$, on trouve 
\[
  \left| \sum \psi P(x_1,\dots,x_n) \right| = \left| \sum_{x_2,\dots,x_n} \sum_{x_1} \psi P(x_1,\dots,x_n)\right| \leqslant q^{n-1} (d-1) q^{1/2} 
\]
comme promis. 


\paragraph{Cas 2: $p\mid d$ ($d>0$).}
Si $P_d$ est une puissance $p$-ième, remplacer $P$ par $P-(P_d-P_d^{1/p})$ ne 
change pas la somme considérée, et abaisse le degré: on se débarrasse de 
ce cas en procédant par récurrence sur $d$. Sinon, la différentielle de 
$P_d$ n'est pas identiquement nulle; appliquant \ref{VI:3-9}, on peut supposer, 
quitte à faire un changement linéaire de variables, qu'elle n'est pas nulle 
au point $(1,0,\dots,0)$. Si on écrit 
\[
  P_d = \sum_{i=0}^d X_1^{d-i} S_i(X_2,\dots,X_n) \text{,} 
\]
cela signifie que la forme linéaire $S_1$ n'est pas identiquement nulle. 

La forme $S_0$ est une constante, et remplaçant $P$ par 
$P-(S_0 X_1^d -S_0^{1/p} X_1^{d/p})$, on peut supposer qu'elle est nulle. Soit 
enfin $-\lambda$ le coefficient de $X_1^{d-1}$ dans $P$. Pour $x_2,\dots,x_n$ 
fixes, on a 
\[
  P(X_1,x_2,\dots,x_n) = (S_1(x_2,\dots,x_n)-\lambda) X_1^{d-1} + \text{termes de plus bas degré en $X_1$} 
\]
et si $S_1(x_2,\dots,x_n)\ne \lambda$, on a donc 
\[
  \left| \sum_{x_1} \psi P(x_1,\dots,x_n)\right| \leqslant (d-2) q^{1/2} \text{.} 
\]
Au total, 
\begin{align*}
  \left|\sum \psi P(x_1,\dots,x_n)\right| 
    &\leqslant \left|\sum_{S=\lambda} \psi P(x_1,\dots,x_n)\right| + \sum_{S(X_2,\dots,X_n)\ne\lambda} \left| \sum_{x_1} \psi P(x_1,\dots,x_n)\right| \\
    &\leqslant q^{n-1} + (q^{n-1}-q^{n-2})(d-2) q^{1/2} \\
    &< (d-2) q^{n-1/2} + q^{n-1} \\
    &< (d-1) q^{n-1/2} \text{.}
\end{align*}

Un autre résultat de cette nature est donné par R.A.\ Smith 
\cite{sm70}. 










\section{Sommes de Gauss et sommes de Jacob}\label{VI:4}





\subsection{}\label{VI:4-1}

Soient $k$ un corps fini de caractéristique $p$, $\chi$ un caractère de 
$k^\times$, et $\psi$ un caractère non trivial du groupe additif de $k$. Nous 
prendrons pour définition des sommes de Gauss: 
\begin{equation*}\tag{4.1.1}\label{VI:eq:4-1-1}
  \tau(\chi,\psi) = -\sum_{x\in k^\times} \psi(x) \chi^{-1}(x) 
\end{equation*}
(noter le signe). Classiquement, $\chi$ et $\psi$ sont à valeurs complexes. 
Nous les prendrons à valeurs dans $E_\lambda^\times$ (cf. la preuve de 
\ref{VI:1-15}). Regardons $-\tau(\chi,\psi)$ comme une somme sur les points 
rationnels du schéma $\dG_m$ sur $k$. D'après le paragraphe \ref{VI:1}, 
on a $-\tau(\chi,\psi) = \tr(F^\ast,\h_c^\ast(\dG_m,\sF(\psi \chi^{-1}))$. De 
plus 





\begin{proposition_}\label{VI:4-2}
La cohomologie de $\dG_m$ à coefficient dans $\sF(\psi\chi^{-1})$ vérifie 
\begin{enumerate}[(i)]
  \item $\h_c^i=0$ pour $i\ne 1$, et $\dim \h_c^1=1$. 
  \item $F^\ast$, agissant sur $\h_c^1$, est la multiplication par 
    $\tau(\chi,\psi)$. 
  \item Si $\chi$ est non trivial, on a $\h_c^\bullet \iso \h^\bullet$. 
\end{enumerate}
\end{proposition_}
\begin{proof}[Preuve]
Pour tout $n$ premier à $p$, notons $K_n$ le $\dmu_n$-torseur sur $\dG_m$ 
défini par la suite exacte de K\"ummer 
$0 \to \dmu_n \to \dG_m \xrightarrow{x^n} \dG_m \to 0$. Si $k$ a $q$ 
éléments, on a sur $k$: $\dmu_{q-1}=k^\times$, et le torseur de Lang sur 
$\dG_m/k$ est $K_{q-1}$. Dès lors, $\sF(\chi^{-1})=\chi(K_{q-1})$. 
L'assertion \ref{VI:4-2}(ii) résulte de (i,iii), eux-mêmes contenus dans 
l'énoncé géométrique suivant, où $\sF(\psi)$ désigne le faisceau 
sur $\dG_m/\dF$ déduit par extension des scalaires de $k$ à $\dF$ de 
$\sF(\psi)$ sur $\dG_m/k$ (\ref{VI:1-8}.b). 
\end{proof}





\begin{proposition_}\label{VI:4-3}
Soit $\chi:\dmu_n \to E_\lambda^\times$. La cohomologie de $\dG_m$ à 
coefficient dans $\sF(\psi)\otimes \chi(K_n)$ vérifie 
\begin{enumerate}[\indent (i)]
  \item $\h_c^i=0$ pour $i\ne 1$, et $\dim \h_c^1{} = 1$. 
  \item Si $\chi$ est non triviale, on a $\h_c^\bullet{}\iso \h^\bullet{}$. 
\end{enumerate}
\end{proposition_}
\begin{proof}[Preuve]
Le faisceau $\sF(\psi)$ est la restriction à $\dG_m$ d'un faisceau localement 
constant sur $\dG_a$, sauvagement ramifié à l'infini, de conducteur de 
Swan $1$. Le faisceau $\chi(K_n)$ est constant si $\chi=1$; si $\chi\ne 1$, il 
est ramifié en $0$ et $\infty$, modérément. 

Le faisceau $\sF(\psi)\otimes\chi(K_n)$ est donc ramifié à l'$\infty$; 
appliquant \ref{VI:1-18}(b,c), on trouve que 
$\h_c^i(\dG_m,\sF(\psi)\otimes \chi(K_n))=0$ pour $i\ne 1$. Si $\chi\ne 1$, il 
est ramifié en $0$ et $\infty$ et (ii) résulte de \ref{VI:1-19}.a. 

Les conducteurs de Swan sont $0$ en $0$ et $1$ en $\infty$. D'après 
\eqref{VI:eq:3-2-1}, la caractéristique d'Euler-Poincaré est donc $-1$ et 
ceci achève la démonstration. 
\end{proof}





\subsection{Remarque}\label{VI:4-4}

Si $\chi$ est non trivial, \ref{VI:4-2}(iii) et la dualité de Poincaré 
montrent que $\h_c^1(\dG_m,\sF(\psi\chi^{-1}))$ et 
$\h_c^1(\dG_m,\sF(\psi^{-1}\chi))$ sont en dualité (dualité à valeurs 
dans $E_\lambda(-1)$). On a donc 
$\tau(\chi,\psi)\cdot \tau(\chi^{-1},\psi^{-1}) = q$, i.e. 
$|\tau(\chi,\psi)|=1$. 





\subsection{}\label{VI:4-5}

Si $k$ est une extension de degré $N$ de $\dF_q$, on peut aussi regarder 
\eqref{VI:eq:4-1-1} comme une somme à $N$ variables sur $\dF_q$. Soit plus 
généralement $k$ une algèbre étale sur $\dF_q$, de degré $N$ sur 
$\dF_q$. C'est un produits de corps $k_i$, de degré $N_i$, et on pose 
\begin{equation*}\tag{4.5.1}\label{VI:eq:4-5-1}
  \varepsilon(k) = (-1)^{\sum (N_i+1)} \text{.} 
\end{equation*}
C'est la signature de la permutation de $S=\hom_{\dF_q}(k,\dF)$ induite par la 
substitution de Frobenius $\varphi\in \gal(\dF/\dF_q)$. 

Soient $\psi:\dF_q\to E_\lambda^\times$ un caractère non trivial, et 
$\chi:k^\times \to E_\lambda^\times$. On pose 
\begin{equation*}\tag{4.5.2}\label{VI:eq:4-5-2}
  \tau_{\dF_q}(\chi,\psi) = (-1)^N \sum_{x\in k^\times} \psi \tr_{k/\dF_q}(x) \cdot \chi^{-1}(x) \text{.} 
\end{equation*}

Si $\chi$ a pour coordonnées les $\chi_i:k_i^\times \to E_\lambda^\times$, on 
a l'identité triviale 
\begin{equation*}\tag{4.5.3}\label{VI:eq:4-5-3}
  \tau_{\dF_q}(\chi,\psi) = \varepsilon(k) \prod \tau(\chi_i,\psi\circ \tr_{k_i/\dF_q}) \text{.}
\end{equation*}

Après quelques préliminaires, nous donnerons en \ref{VI:4-10} une 
interprétation cohomologique des sommes \eqref{VI:eq:4-5-2}. En 
\ref{VI:4-12}, nous interpréterons l'identité de Hasse-Davenport comme une 
forme tordue (cf. \ref{VI:1-12}) du cas particulier suivant de 
\eqref{VI:eq:4-5-3}: pour $k=\dF_q^N$, et $\chi$ un caractère de 
$\dF_q^\times$, on a 
$\tau_{\dF_q}(\chi\circ N_{k/\dF_q},\psi)=\tau(\chi,\psi)^N$. 





\subsection{}\label{VI:4-6}

Rappelons que pour $M$ un module projectif de type fini sur un anneau $A$, le 
foncteur $\spec(B)\mapsto M\otimes_A B$, des schémas affines sur $\spec(A)$ 
dans $\mathsf{Ens}$ est représenté par le schéma affine $\dV(M^\vee)$ 
(notations des EGA), le spectre de $\operatorname{Sym}_A^\bullet(M^\vee)$. Si 
$M=A^n$, c'est l'espace affine type de dimension $n$. 

Supposons que $M$ soit une $A$-algèbre à unité, et posons 
$V=\dV(M^\vee)$. Par définition, pour toute extension $B$ de $A$, l'ensemble 
$V(B)$ de points de $V$ à coordonnées dans $B$ est $M\otimes_A B$. Le 
foncteur $B\mapsto M\otimes_A B$ étant à valeurs dans les anneaux à 
unité, $V$ est un schéma en anneaux à unités sur $\spec(A)$. Les 
morphismes norme et trace: $M\otimes_A B\to B$ sont fonctoriels en $B$; ils 
correspondent donc à des morphismes de schéma $N$ et $T$ de $V$ dans 
$\dG_a$. Nous aurons à considérer les schémas déduits de $V$ suivant: 
\begin{enumerate}[a)]
  \item $V^\ast$ est l'ouvert des éléments inversibles de $V$ (un schéma 
    en groupes pour $\cdot$);
  \item $W$ est l'hyperplan d'équation $T=0$ et $W^\ast=W\cap V^\ast$; 
  \item $P$ est l'hyperplan à l'infini $V\setminus \{0\}/\dG_m$ de l'espace 
    affine $V$ sur $\spec(A)$. Si $Q$ est l'hyperplan à l'infini de $W$, 
    $W^\ast/\dG_m$ et $V^\ast/\dG_m$ sont des ouverts des espaces projectifs 
    $Q$ et $P$ sur $\spec(A)$. 
    \begin{equation*}\tag{4.6.1}\label{VI:eq:4-6-1}
    \xymatrix{
      W^\ast \ar@{^{(}->}[r] \ar[d]^-\pi 
        & V^\ast \ar[d]^-\pi \\
      W^\ast/\dG_m \ar@{^{(}->}[r] 
        & V^\ast/\dG_m 
    }\qquad\subset\qquad
    \xymatrix{
      W\setminus \{0\} \ar@{^{(}->}[r] \ar[d]^-\pi 
        & V\setminus \{0\} \ar[d]^-\pi \\
      Q \ar@{^{(}->}[r] 
        & P \text{.}
    }
    \end{equation*}
\end{enumerate}

Si $M=A^I$, $I$ un ensemble fini, alors $V\simeq \dG_a^I$, $V^\ast=\dG_m^I$, 
$V^\ast/\dG_m$ est le tore $\dG_m^I/(\dG_m\text{ diagonal})$, $N$ et $T$ 
s'écrivant $\prod x_i$ et $\sum x_i$, et $Q$ est donc l'hyperplan 
projectif d'équation $\sum x_i=0$ de $P$. 

Le cas qui nous intéresse est celui où $M$ est fini étale sur $A$. La 
description précédente vaut alors localement sur $\spec(A)$ (pour la 
topologie étale) et on peut écrire $V^\ast=\dG_m^I$, pour $I$ un faisceau 
localement constant d'ensembles finis sur $\spec(A)$. Si par exemple $A$ est un 
corps, de clôture algébrique $\bar A$, et que $M$ est un $A$-algèbre 
séparable, on pose $I=\hom_A(M,\bar A)$, et, sur $\bar A$, on a 
$V\sim \dG_a^I$. Via cet isomorphisme, $N$ et $T$ s'écrivent $\prod x_i$ et 
$\sum x_i$. 





\subsection{Torseur de Kummer}\label{VI:4-7}

Soient $S$ un schéma, $n$ un entier inversible sur $S$ et $G$ un schéma en 
groupes commutatifs à fibres connexes (noté multiplicativement) sur $S$. 
La suite 
\[\xymatrix{
  0 \ar[r] 
    & G_n \ar[r] 
    & G \ar[r]^-{x^n} 
    & G \ar[r] 
    & 0 
}\]
est exacte. Elle définit un $G_n$-torseur $K_n(G)$ (ou simplement $K_n$) sur 
$G$. Pour $\chi:G_n \to E_\lambda^\times$ un homomorphisme du $S$-faisceau 
étale $G_n$ dans le faisceau constant $E_\lambda^\times$, on note 
$\sK_n(\chi)$ le $E_\lambda$-faisceau $\chi^{-1}(K_n)$. 

Les torseurs $K_n$ forment un système projectif de torseurs sous le 
système projectif de groupes $G_n$ (morphismes de transition 
$x\mapsto x^d:G_{n d}\to G_n$). Ceci exprime la commutativité des diagrammes 
\[\xymatrix{
  0 \ar[r] 
    & G_{n d} \ar[r] \ar[d]^-{x^d} 
    & G \ar[r]^-{x^{n d}} \ar[d]^-{x^d} 
    & G \ar[r] \ar@{=}[d] 
    & 0 \\
  0 \ar[r] 
    & G_n \ar[r] 
    & G \ar[r]^-{x^n} 
    & G \ar[r] 
    & 0 \text{.} 
}\]
On a donc $\sK_n(\chi) = \sK_{n d}(\chi\circ x^d)$. 





\subsection{}\label{VI:4-8}

Appliquons la construction \ref{VI:4-6} pour $A=\dF_q$, et $M=k$ un algèbre 
étale sur $\dF_q$. Conformément aux conventions générales, on notera 
avec un indice $0$ les $\dF_q$-schémas notés $V,V^\ast,\ldots$ en 
\ref{VI:4-6}. Les mêmes lettres sans indice désignent les schémas sur 
$\dF$ qui s'en déduisent par extension des scalaires. 

On pose $I=\hom(k,\dF)$, d'où $V\sim \dG_a^I$ et $T$ s'écrit 
$(x_i)\mapsto \sum x_i$. 
\begin{equation*}\tag{4.8.1}\label{VI:eq:4-8-1}
  \sF(\psi\circ \tr_{k/\dF_q}) = T^\ast \sF(\psi) = \bigotimes_i \pr_i^\ast \sF(\psi) \text{.} 
\end{equation*}
Puisque $V^\ast\sim \dG+m^I$, un caractère $\chi$ de $V_n^\ast\sim \dmu_n^I$, 
à valeurs dans $E_\lambda^\times$, s'écrit comme une famille 
$(\chi_i)_{i\in I}$ de caractères de $\dmu_n$ indénée par $I$. Elle est 
définie sur $\dF_q$ si, pour tout $\sigma\in \gal(\dF/\dF_q)$, on a 
$\chi_{\sigma_i} = \chi_i\circ \sigma^{-1}$. Elle définit alors un 
$E_\lambda$-faisceau $\sK_n((\chi_i)_{i\in I})$ sur $V_0^\ast$. Si le produit 
des $\chi_i$ est trivial, $\chi$ se factorise par un caractère de 
$(V_0^\ast/\dG_m)_n$ et $\sK_n((\chi_i)_{i\in I})$ est l'image réciproque 
d'un faisceau, noté de même, sur $V-0^\ast/\dG_m$. Cette construction se 
compare comme suit au torseur de Lang. 





\begin{lemma_}\label{VI:4-9}
\begin{enumerate}[(i)]
  \item Pour $n$ assez divisible, on a un diagramme commutatif 
    \[\xymatrix{
      0 \ar[r] 
        & (V_0^\ast)_n \ar[r] \ar[d]^-\tau 
        & V_0^\ast \ar[r]^-{x^n} \ar[d]^-\tau 
        & V_0^\ast \ar[r] \ar@{=}[d] 
        & 0 \\
      0 \ar[r] 
        & k^\times \ar[r] 
        & V_0^\ast \ar[r]^-\fL 
        & V_0^\ast \ar[r] 
        & 0 \text{;} 
    }\]
    si $\chi$ est un caractère $\chi:k^\times \to E_\lambda^\times$, on a 
    $\sF(\chi) = \sK_n(\chi\circ\tau)$. 
  \item Pour $n$ assez divisible, $\tau$ identifie $k^\times$ aux coinvariants 
    de $\gal(\bar\dF/F)$ agissant sur $V_n^\ast\sim \dmu_n^I$. 
  \item Pour $k$ un corps, $N=[k:\dF_q]$, $\omega\in I$ un plongement de $k$ 
    dans $\dF$, et $n=q^N-1$, la composante d'indice $\omega$ de 
    $\chi\circ\tau$ est $\chi\circ \omega^{-1}$. 
  \item Si $\chi$ est non trivial sur chaque facteur de $k$, les 
    $(\chi\circ \tau)_i$ sont tous non triviaux. 
\end{enumerate}
\end{lemma_}

Il suffit de prouver le lemme lorsque $k$ est un corps, de degré $N$ sur 
$\dF_q$. Dans ce cas, $n$ est ``assez divisible'' si $q^N-1\mid n$. 
Choisissons un plongement $\omega$ de $k$ dans $\dF_q$; $I$ s'identifie alors 
à $\dZ/N$: à $i\in \dZ/N$ correspond $\omega_i=\omega^{q^i}$. Via 
l'isomorphisme $V_0^\ast(\dF) = {F^\times}^I$, on a 
\begin{enumerate}[\indent a)]
  \item $x\in k^\times$ correspond à $(\omega_i(x))\in {\dF^\times}^I$; 
  \item $F((x_i)_{i\in \dZ/N})=(x_{i-1}^q)_{i\in \dZ/N}$. 
\end{enumerate}

Un caractère $\chi=(\chi_i)$ de $V_n^\ast\sim \dmu_n^I$ sera défini sur 
$\dF_q$ si $\chi_{-i} =\chi_0(x^{q^i})$ ($i\in \dZ$). Si $q^N-1\mid n$, il y a 
$q^N-1$ tels caractères: $\chi_0$ se factorise par $\dmu_{q^N-1}$, et 
détermine les $\chi_i$. Pour prouver (ii), il suffit donc de vérifier (i) 
et (iii) (ou son corollaire (iv)) qui assure que $\chi\mapsto \chi\circ \tau$ 
est injectif. 

Prouvons (i) et (iii), pour $n=q^N-1$. Notons additivement le groupe des 
endomorphismes de $V^\ast\sim \dG_m^I$; en particulier, notons $n$ 
l'opérateur $x\mapsto x^n$. Si $\alpha$ est l'opérateur de permutations 
circulaire $(x_i)\mapsto (x_{i-1})$, on a 
$q^N-1=(q\alpha)^N-1 = (q\alpha-1)((q\alpha)^{N-1}+\cdots + 1)$. Ceci 
détermine $\tau$: on a $\tau((x_i)) = \prod_{0\leqslant j<N} x_{i-j}^{q^j}$, 
et l'application induite de $(V_0^\ast)_n$ dans $k^\times$ s'écrit [a) 
ci-dessus] $(x_i)\mapsto \prod x_{-i}^{q^i}$; de là résulte (iii). 





\begin{proposition_}\label{VI:4-10}
La cohomologie de $V^\ast$ à coefficients dans 
$\sF(\chi^{-1}\cdot \psi\tr_{k/\dF_q})$ vérifie 
\begin{enumerate}[\indent (i)]
  \item $\h_c^i = 0$ pour $i\ne N$, et $\dim \h_c^N=1$. 
  \item Sur $\h_c^N$, $F^\ast$ est la multiplication par 
    $\tau_{\dF_q}(\chi,\psi)$. 
  \item Si $\chi$ est non trivial sur chaque facteur de $k$, on a 
    $\h_c^\bullet \iso \h^\bullet$. 
\end{enumerate}
\end{proposition_}

Appliquons \ref{VI:4-9}(i): sur $\dF$, si $\chi\circ \tau=(\chi_i)_{i\in I}$, 
on a $\sF(\chi) = \sK_n((\chi_i)_{i\in I})$. Les points (i) et (iii) 
résultant donc de l'énoncé plus géométrique suivant (pour (iii), 
appliquer \ref{VI:4-9}(iv)) et (ii) en résulte par la formule des traces: 





\begin{proposition_}\label{VI:4-11}
Soit $(\chi_i)_{i\in I}$ une famille de caractère de $\dmu_n(\dF)$. La 
cohomologie de $V^\ast\simeq \dG_m^I$ à coefficient dans 
$\sK_n((\chi_i)_{i\in I})\otimes \sF(\psi\tr_{k/\dF_q})$ vérifie 
\begin{enumerate}[\indent (i)]
  \item $\h_c^i{}=0$ pour $i\ne N$, et $\dim \h_c^N{} = 1$. 
  \item Si les $\chi_i$ sont tous non triviaux, on a 
    $\h_c^\bullet{}\iso \h^\bullet$. 
\end{enumerate}
\end{proposition_}

On a $V^\ast\sim \dG_m^I$, 
$\sK((\chi_i)) = \bigotimes_i \pr_i^\ast \chi_i(K_n(\dG_m))$ et 
$\sF(\psi\circ \tr_{\dF_q/k})=T^\ast \sF(\psi) = \bigotimes_i \pr_i^\ast \sF(\psi)$. 
Ceci permet d'appliquer la formule de K\"unneth, et \ref{VI:4-11} résulte de 
\ref{VI:4-3}. 





\subsection{}\label{VI:4-12}

De \ref{VI:2-4*}* et (\hyperref[VI]{Cycle}, \ref{VI:1-3} exemple 2), on tire 
aussi que le groupe des permutations $\sigma$ de $I$ telles que 
$\chi_i = \chi_{\sigma i}$ ($i\in I$) agit sur $\h_c^N$ par multiplication par 
la signature $\varepsilon(\sigma)$. Nous allons en déduire une seconde preuve 
de l'identité de Hasse-Davenport. 

Si $\chi$ est un caractère de $\dF_q^\times$, le fait que, sur $\dF$, $N$ 
s'écrit $(x_i)\mapsto \prod x_i$ fournit: 
$\sF(\chi\circ N)=N^\ast\sF(\chi) = \bigotimes_i \pr_i^\ast\sF(\psi)$ 
et la formule de K\"unneth fournit: 
\[
  \h_c^\bullet(V^\ast,\sF(\chi^{-1}\circ N,\psi\circ \tr)) \sim \h_c^\bullet(\dG_m,\sF(\chi^{-1}\psi))^{\otimes I} \text{,} 
\]
où, au membre de droite, le produit tensoriel est pris au sens 
\ref{VI:2-4*}*. Puisque $\h_c^1(\dG_m,\sF(\psi\chi^{-1}))$ est de dimension 
$1$, sa puissance tensorielle $\otimes I$, au sens ordinaire, ne dépend que 
du cardinal $n$ de $I$. Appliquant (\hyperref[IV]{Cycle}, \ref{IV:1-3} exemple 
2), on obtient un isomorphisme canonique 
\begin{equation*}\tag{4.12.1}\label{VI:eq:4-12-1}
  \h_c^N\left(V,\sF(\chi^{-1}\circ N)\otimes T^\ast \sF(\psi)\right) \sim \h_c^1(\dG_m,\sF(\psi\chi^{-1})) \otimes_\dZ \textstyle \bigwedge^N \dZ^I \text{.}
\end{equation*}
Pour calculer l'action de $F^\ast$, le plus commode est d'adopter le point de 
vue galoisien et de dire que l'isomorphisme \eqref{VI:eq:4-12-1} étant 
canonique, il est compatible à l'action par transport de structure de 
$\gal(\dF/\dF_q)$. Si $\varepsilon(k)$ est la signature de la permutation 
$\varphi$ de $I$, on trouve que 
\begin{align*} 
  \tau_{\dF_q}(\chi\circ N_{k/\dF_q},\psi) 
    &= \tr\left(\varphi^{-1},\h_c^N\left(V^\ast,N^\ast\sF(\chi^{-1})\otimes T^\ast \sF(\psi)\right)\right) \\ 
    &= \varepsilon(k) \tr\left(\varphi^{-1},\h_c^1(\dG_m,\sF(\chi^{-1}\psi))\right)^N \\
    &= \varepsilon(k) \tau(\chi,\psi)^N \text{,} 
\end{align*}
soit 
\begin{equation*}\tag{4.12.2}\label{VI:eq:4-12-2}
  \tau_{\dF_q}\left(\chi\circ N_{k/\dF_q},\psi\right) = \varepsilon(k) \tau(\chi,\psi)^N \text{.} 
\end{equation*}

Si $k$ est un corps, $\varphi$ est une permutation circulaire de $I$, 
$\varepsilon(k) = (-1)^{N+1}$, et on retrouve l'identité de Hasse-Davenport: 
\[
  \tau\left(\chi\circ N_{k/\dF_q},\psi\circ \tr_{k/\dF_q}\right) = (-1)^{N+1} \tau_{\dF_q}\left(\chi\circ N_{k/\dF_q},\psi\right) = \tau(\chi,\psi)^N \text{.} 
\]





\begin{lemma_}\label{VI:4-13}
Pour $\chi\circ \tau=(\chi_i)_{i\in I}$ comme en \ref{VI:4-9}(i), les
conditions suivantes sont équivalentes 
\begin{enumerate}[\indent (i)]
  \item $\chi| \dF_q^\times$ est trivial 
  \item Le produit des $\chi_i$ est trivial 
  \item $\sF(\chi) = \sK_n((\chi_i))$ est image réciproque d'un (unique) 
    faisceau sur $V^\ast/\dG_m$. 
  \item L'image réciproque de $\sF(\chi)$ sur $\dG_m$ (envoyé dans 
    $V^\ast$ à partir du morphisme structural $\dF_q \to k$) est triviale. 
\end{enumerate}
\end{lemma_}

(i)$\Rightarrow$(ii). On regarde $\chi$ comme une caractère de 
$V^\ast/\dG_m(\dF_q)=k^\times/\dF_q^\times$, et on prend $\sF(\chi)$ sur 
$V^\ast/\dG_m$. 

(ii)$\Rightarrow$(iii). De même, on regarde $(\chi_i)$ comme un caractère 
de $(V^\ast/\dG_m)_n$. L'unicité dans (iii) résulte de ce que $V^\ast$ est 
un fibré à fibres connexes sur $V^\ast/\dG_m$, et (iii)$\Rightarrow$(iv)  
est trivial. 

(iv)$\Rightarrow$(i), (ii). Cette image réciproque est 
$\sF(\chi|\dF_q^\times)$ et $\sK_n(\prod \chi_i)$. 





\subsection{}\label{VI:4-14}

Soient $k$ une algèbre étale sur $\dF_q$, de 
dimension $N+1$ et $\chi$ un caractère non trivial de $k^\times$, trivial sur 
$\dF_q^\times$. La somme de Jacobi $J(\chi)$ est définie par 
\begin{equation*}\tag{4.14.1}\label{VI:eq:4-14-1}
  J(\chi) = (-1)^{N-1} \sum_{\substack{x\in k^\times/\dF_q^\times \\ \tr(x)=0}} \chi^{-1}(x) \text{.} 
\end{equation*}

On a entre sommes de Gauss et sommes de Jacobi l'identité suivante. 





\begin{proposition_}\label{VI:4-15}
Pour $\chi$ comme ci-dessus et $\psi$ un caractère additif non trivial de 
$\dF_q$, on a 
\[
  q J(\chi) = \tau_{\dF_q}(\chi,\psi) \text{.} 
\]
\end{proposition_}

Dans le cas particulier où $k=\dF_q^n$, cette formule se récrit par 
\eqref{VI:eq:4-5-3} 
\begin{equation*}\tag{4.15.1}\label{VI:eq:4-15-1}
  q J(\chi) = \prod_i \tau(\chi_i,\psi) \text{,} 
\end{equation*}
plus écrit sous la forme 
\begin{align*}
  \chi_0(-1) J(\chi) &= (-1)^{N-1} \sum_{\substack{x_1,\dots,x_N\in \dF_q^\times \\ \sum x_i = 1}} \prod_{i=1}^N \chi_i^{-1}(x_i) = \tau(\chi_0^{-1},\psi)^{-1} \prod_{i=1}^N \tau(\chi_i,\psi) \\
  \chi_0^{-1} &= \prod_{i=1}^N \chi_i 
\end{align*}





\begin{proof}[Preuve de \ref{VI:4-15}]
\[
  \tau_{\dF_q}(\chi,\psi) = (-1)^{N+1} \sum_{x\in k^\times} \chi(x)^{-1} \psi \tr(x) = (-1)^{N+1} \sum_{x\in k^\times/\dF_q^\times} \chi(x)^{-1} \sum_{\lambda\in \dF_q^\times} \psi(\lambda \tr x) \text{.}
\]
La somme $\sum_{\lambda\in \dF_q^\times} \psi(\lambda \tr x)$ vaut $q-1$ si 
$\tr(x)=0$, et $-1$ si $\tr(x)\ne 0$. Dès lors, 
\[
  \tau_{\dF_q}(\chi,\psi) = (-1)^{N+1} \left(q\sum_{x\in k^\times/\dF_q^\times} \chi(x)^{-1} - \sum_{x\in k^\times/\dF_q^\times} \chi(x)^{-1}\right) \text{.}
\]
La second terme au second membre est nul, car somme des valeurs d'un 
caractère, et \ref{VI:4-15} en résulte. 
\end{proof}

La proposition \ref{VI:4-15} se transpose ainsi en cohomologie: 





\begin{proposition_}\label{VI:4-16}
La cohomologie de $W^\ast/\dG_m$ (\ref{VI:4-6}, \ref{VI:4-8}) à coefficient 
dans $\sF(\chi^{-1})$ vérifie 
\begin{enumerate}[\indent (i)]
  \item $\h_c^i=0$ pour $i\ne N-1$, et $\dim \h_c^{N-1}{}=1$. 
  \item Sur $\h_c^{N-1}$, $F^\ast$ est la multiplication par $J(\chi)$. 
  \item $\h_c^{N-1}$ est canoniquement isomorphe à 
    $\h_c^{N+1}\left(V^\ast,\sF(\chi^{-1},\psi\circ \tr_{k/\dF_q})\right)(1)$. 
\end{enumerate}
\end{proposition_}

L'assertion (ii) résulte de (i) et de la formule des traces; (i) et (iii) 
résultent de \ref{VI:4-11} et de l'énoncé plus géométrique suivant. 





\begin{proposition_}\label{VI:4-17}
Soit $(\chi_i)_{i\in I}$ une famille de caractères non trous triviaux de 
$\dmu_n(\dF)$, de produit $1$. Alors, 
$\h_c^{i-1}\left(W^\ast/\dG_m,\sK_m((\chi_i)_{i\in I})\right)$ est 
canoniquement isomorphe à 
\newline
$\h_c^{i+1}\left(V^\ast,\sK_m((\chi_i)_{i\in I})\otimes \sF(\psi\tr_{k/\dF_q})\right)(1)$. 
\end{proposition_}

La première ligne de \ref{VI:4-15} devient ($\pi$ comme en 
\eqref{VI:eq:4-6-1}). 
\begin{equation*}\tag{4.17.1}\label{VI:eq:4-17-1}
  \eR^i \pi_!\left(\sK_n((\chi_i)_{i\in I})\otimes \sF(\psi \tr_{k/\dF_q})\right) = \sK_n((\chi_i)_{i\in I}) \otimes \eR^i \pi_! \sF(\psi\tr_{k/\dF_q}) 
\end{equation*}
(sur $V^\ast/\dG_m$). Calculons les faisceaux 
$\eR^i\pi_! \sF(\psi\tr_{k/\dF_q}) = \eR^i \pi_! T^\ast \sF(\psi)$. Soit 
$v_0:\widetilde V_0 \to V_0$ l'éclaté de $V_0$ en $\{0\}$. Dans le 
diagramme 
\begin{equation*}\tag{4.17.2}\label{VI:eq:4-17-2}
\xymatrix{
  V_0\setminus \{0\} \ar@{^{(}->}[r] \ar[dr]_-\pi 
    & \widetilde V_0 \ar[r]^-{v_0} \ar[d]^-{\bar\pi} 
    & V_0 \\
  & P_0 
}
\end{equation*}
$\pi$ est une fibration de fibre des droites épointées, $\widetilde V_0$ 
est le fibré en droites correspondant, et $Z_0=v_0^{-1}(0)$ sa section $0$. 
Le faisceau $T^\ast\sF(\psi)$ sur $V-0\setminus \{0\}$ se prolonge en 
$(T v_0)^\ast \sF(\psi)$ sur $\widetilde V_0$, et la restriction de ce faisceau 
à $Z_0$ est le faisceau constant $E_\lambda$, car $T_0 v_0$ est nul sur 
$Z_0$. La suite exacte longue de cohomologie à support propre s'écrit 
donc 
\begin{equation*}\tag{4.17.3}\label{VI:eq:4-17-3}
\xymatrix{
  \ar[r] 
    & \eR^i \pi_! T^\ast \sF(\psi) \ar[r] 
    & \eR^i \bar\pi_! (T v)^\ast \sF(\psi) \ar[r] 
    & (E_\lambda\text{ pour $i=0$, $0$ sinon}) \ar[r] 
    & 
}
\end{equation*}

L'image réciproque $\bar\pi^{-1}(Q_0)$ est l'éclaté $\widetilde W_0$ de 
$W_0$ en $0$; $T_0 V_0$ s'annule sur $\widetilde W_0$; on a donc 
$(T v)^\ast \sF(\psi)=E_\lambda$ sur $\pi^{-1}(Q_0)$ et, $\bar\pi$ étant un 
fibré en droites 
\begin{equation*}\tag{4.17.4}\label{VI:eq:4-17-4}
  \eR^i \bar\pi_! (T v)^\ast \sF(\psi)|Q = 
    \begin{cases}
      0 & \text{pour $i\ne 2$} \\
      E_\lambda(-1) & \text{pour $i=2$.} 
    \end{cases} 
\end{equation*}

Si $x\in P$, $x\notin Q$, la droite $D=\pi^{-1}(x)$ est envoyé 
isomorphiquement sur $\dG_a$ par $T v$; on a donc 
\ref{VI:2-7}*) 
\begin{align} \notag 
  \h_c^\bullet(D,(T v)^\ast \sF(\psi)) &= \h^\bullet(\dG_a,\sF(\psi)) = 0 && \text{et} \\
  \tag{4.17.5}\label{VI:eq:4-17-5}
  \eR^i \bar\pi_! (T v)^\ast \sF(\psi) 
    &= \begin{cases} 
         0 & \text{pour $i\ne 2$} \\
         E_\lambda(-1)_Q & \text{pour $i=2$} 
       \end{cases} 
\end{align}

Conjuguant \eqref{VI:eq:4-17-3} et \eqref{VI:eq:4-17-5}, on trouve enfin 
\begin{equation*}\tag{4.17.6}\label{VI:eq:4-17-6}
  \eR^i \pi_! T^\ast\sF(\psi) = 
    \begin{cases}
      0 & \text{pour $i\ne 1,2$} \\
      E_\lambda & \text{pour $i=1$} \\
      E_\lambda(-1)_Q & \text{pour $i=2$.} 
    \end{cases}
\end{equation*}





\subsection{}\label{VI:4-18}

Calculons la cohomologie à support propre du faisceau 
$\sK_n((\chi_i))\otimes T^\ast \sF(\psi)$ sur $V^\ast$ à l'aide de la suite 
spectrale de Leray de $\pi:V^\ast\to V^\ast/\dG_m$. Appliquant 
\eqref{VI:eq:4-17-1} et \eqref{VI:eq:4-17-6}, on trouve comme termes initiaux 
les $E_2^{p,1} = \h_c^p(V^\ast/\dG_m,\sK_n(\chi_i)) = 0$ (car $(\chi_i)\ne 0$) 
et les $E_2^{p,2} = \h_c^p(W^\ast/\dG_m,\sK(\chi_i))(-1)$. 

La suite spectrale se réduit à un isomorphisme, et \ref{VI:4-17} en 
résulte. 





\subsection{}\label{VI:4-19}

Les faisceaux $\sK_n((\chi_i)_{i\in I})$ ont un sens sur n'importe quel corps, 
voire sur n'importe schéma de base. Ceci va nous permettre de généraliser 
\ref{VI:4-16}(i). Avec les notations de \ref{VI:4-6}, supposons $M$ fini 
étale partout de rang $N$ sur $A$, et soit $n$ un entier inversible dans $A$. 
On note $a$ la projection de $W^\ast/\dG_m$ sur $\spec(A)$. Soit aussi $\chi$ 
un caractère (morphisme de faisceaux) 
$\chi:(V^\ast/\dG_m)_n \to E_\lambda^\times$, et $\sK_n(\chi)$ le faisceau 
correspondant. Localement pour la topologie étale, on peut regarder $\chi$ 
comme une famille de caractères $(\chi_i)_{i\in I}$ de $\dmu_n$, de produit 
trivial. 





\begin{proposition_}\label{VI:4-20}
\begin{enumerate}[(i)]
  \item Si $\chi$ est (en tout point) non trivial, on a 
    $\eR^i a_! (\sK_n(\chi)) = 0$ pour $i\ne N-1$, et 
    $\eR^{N-1} a_!(\sK_n(\chi))$ est lisse, de rang $1$. 
  \item Une automorphisme $\sigma$ de $M$ qui respecte $\chi$ agit sur ce 
    $\eR^{N-1} a_!$ par multiplication par la signature $\varepsilon(\sigma)$ 
    de $\sigma$, vu comme permutation de $I$. 
  \item Si les $\chi_i$ sont tous non triviaux, on a $\eR a_! \iso \eR a_\ast$. 
\end{enumerate}
\end{proposition_}
\begin{proof}[Preuve]
Le schéma $W^\ast/\dG_m$ est le complément d'un diviseur à croisements 
normaux relatif dans le schéma $Q$, propre et lisse sur $\spec(A)$, et 
$\sK_n(\chi)$ est localement constant sur $W^\ast/\dG_m$, à ramification 
modérée a l'infini. Il en résulte que les $\eR^i a_!$ et 
$\eR^i a_\ast$ sont lisses, de formation compatible à tout changement de 
base. Un argument standard nous ramène alors à supposer que $A$ est un 
corps fini, et (i) résulte de \ref{VI:4-17} et \ref{VI:4-11}(i). Pour 
prouver (ii), on utilise que l'action de $\sigma$ est compatible à 
l'isomorphisme \ref{VI:4-17}, et \ref{VI:4-12}. Pour (iii), on note que si les 
$\chi_i$ sont tous non triviaux, alors $\sK_n(\chi)$ sur 
$W^\ast/\dG_m\subset Q$ est ramifié le long de chaque diviseur à l'infini, 
et \ref{VI:1-19-1}. 
\end{proof}










\section{Caractères de Hecke}\label{VI:5}





\subsection{}\label{VI:5-1}

Soient $F$ un corps de nombres (de degré fini sur $\dQ$) et $k$ un corps de 
caractéristique $0$. Voici diverses façon équivalentes de dire ce qu'est un 
homomorphisme \emph{algébrique} $\alpha:F^\times \to k^\times$. 
\begin{enumerate}[(i)]
  \item Soit $\{e_i\}$ une base de $F$ sur $\dQ$. L'homomorphisme $\alpha$ est 
    algébrique s'il est donné par une formule 
    $\alpha(\sum x^i e_i) = A(x^i)$, $A\in k(X^i)$. 
\end{enumerate}

Ceci signifie que $\alpha$ coïncide sur $F^\times$ avec une application 
rationnelle définie sur $k$ de $\res_{F/\dQ}(\dG_m)$ dans $\dG_m$. Par 
densité de Zariski de $F^\times$, et le fait qu'un homomorphisme birationnel 
est partout défini, une telle application est un homomorphisme de schémas 
en groupes: 
\begin{enumerate}[(i)]
\setcounter{enumi}{1}
  \item $\alpha$ est induit par un homomorphisme de $k$-schémas en groupes 
    \[
      \res_{F/\dQ}(\dG_m)\otimes_\dQ k \to \dG_m \text{.} 
    \] 
\end{enumerate}

Si $k$ est un corps de nombre, la propriété d'adjonction de la restriction 
des scalaires $R$ montre que ceci équivaut à 
\begin{enumerate}[(i)]
\setcounter{enumi}{2}
  \item ($k$ un corps de nombres) $\alpha$ est induit par un homomorphisme de 
    $\dQ$-schémas en groupes: $\res_{F/\dQ}(\dG_m) \to \res_{k/\dQ}(\dG_m)$. 
\end{enumerate}

Soient $\bar k$ un clôture algébrique de $k$, et $I=\hom(F,\bar k)$. Sur 
$\bar k$, le groupe des caractères de $\res_{F/\dQ}(\dG_m)$ est $\dZ^I$, de 
base les plongements de $F$ dans $\bar k$. Les caractères définis sur $k$ 
sont ceux invariants par $\gal(\bar k/k)$; on peut les décrire soit comme 
ceux envoyant $F^\times$ dans $k^\times\subset \bar k$, soit en terme des 
orbites de $\gal(\bar k/k)$ dans $I$, correspondant elle-même aux facteurs de 
$F\otimes k$. 
\begin{enumerate}[(i)]
\setcounter{enumi}{3}
  \item $\alpha$ est de la forme 
    $\alpha=\prod_{\omega\in I} \omega^{n_\omega}$. Les familles d'exposants 
    $(n_\omega)$ permises sont celles telles que $n_\omega=n_{\omega'}$ pour 
    $\omega$ et $\omega'$ dans la même orbite de $\gal(\bar k/k)$. Ce sont 
    encore celles telle que $\prod_\omega (x)^{n_\omega}\in k$ pour tout 
    $x\in F$. 
  \item Posons $F\otimes k=\prod_{j\in J} F_j$, les $F_j$ étant des corps. 
    Alors, $\alpha$ s'écrit $\alpha=\prod N_{F_j/k}^{m_j}$. 
\end{enumerate}

Dans le cas particulier où $k$ contient une clôture normale de $F$, ces 
expressions deviennent: $\alpha=\prod \omega^{n_\omega}$, où $\omega$ 
parcourt les $[F:\dQ]$ plongements de $F$ dans $k$. 





\subsection{}\label{VI:5-2}

Supposons que $k$ soit un corps de nombres, et soit $\alpha$ un homomorphisme 
algébrique de $F^\times$ dans $k^\times$. Il existe alors un et un seul 
homomorphisme, encore noté $\alpha$, du groupe $I(F)$ des idéaux 
fractionnaires de $F$ dans celui de $k$, tel que $\alpha((x))=(\alpha(x))$. 
L'unicité résulte de ce que tout idéal $a$ une puissance qui est un 
idéal principal, et de ce que $I(k)$ est sans torsion. Pour prouver 
l'existence, on utilise par exemple \ref{VI:5-1}(v): on a 
$\alpha(a) = \prod N_{F_j/k}^{m_j}((a))$. 





\subsection{}\label{VI:5-3}

Rappelons la définition des caractères de Hecke algébriques, appelés 
Weil caractères de Hecke (ou: gr\"ossencharaktere) de type $A_0$. Soient $F$ 
un corps de nombres, $\fm$ un idéal de $F$ (i.e., de l'anneau des entiers de 
$F$), $I_\fm$ le groupe des idéaux fractionnaires de $F$ premiers à $\fm$, 
et $k$ un corps de caractéristique $0$. Un homomorphisme 
$\chi:I_\fm \to k^\times$ est un \emph{caractère de Hecke algébrique} (de 
conducteur $\leqslant \fm$) s'il existe un homomorphisme algébrique 
$\chi_\text{alg}:F^\times \to k^\times$ vérifiant 
\begin{equation*}\tag{$*$}\label{VI:eq:5-3-1}
\text{Pour $x\in F^\times$, premier à $\fm$, totalement positif et 
$\equiv 1\pmod\fm$, on a $\chi((x))=\chi_\text{alg}(x)$.}
\end{equation*}

Par densité de l'ensemble des $x$ de \eqref{VI:eq:5-3-1}, $\chi_\text{alg}$ 
est entièrement déterminé par $\chi$. C'est la \emph{partie algébrique} 
de $\chi$. Si $\chi((x))=\chi_\text{alg}(x)$ pour $x$ totalement positif et 
$\equiv 1\pmod{\fm'}$, $\chi$ se prolonge en un caractère de Hecke de 
conducteur $\leqslant \inf(\fm,\fm')$: $I_{\fm+\fm'}\to k^\times$. On 
identifiera les caractères de Hecke qui coïncident sur leur domaine commun de 
définition, et on appelle \emph{conducteur} de $\chi$ le plus petit $\fm'$ 
tel que $\chi$ soit de conducteur $\leqslant \fm'$. 





\subsection{Remarque}\label{VI:5-4}

Si un caractère de Hecke algébrique $\chi$ prend ses valeurs dans un 
sous-corps $k'$ de $k$, c'est déjà un caractère de Hecke à valeurs dans 
$k'$: il suffit de voire que $\chi_\text{alg}:\res_{F/\dQ}(\dG_m) \to \dG_m$ 
est déjà défini sur $k'$, et ceci résulte de la densité de Zariski 
dans $\res_{F/\dQ}(\dG_m)$ de l'ensemble des $x$ totalement positif 
$\equiv 1\pmod\fm$. On pour toujours prendre pour $k'$ un sous-corps de $k$ de 
degré fini sur $\dQ$. 





\subsection{}\label{VI:5-5}

Si $\varepsilon$ est une unité totalement positive $\equiv 1\pmod\fm$, on a 
$\chi_\text{alg}(\varepsilon) = \chi((\varepsilon)) = 1$. L'homomorphisme 
$\chi_\text{alg}$ se factorise donc par le quotient $T_\fm$ de 
$\res_{F/\dQ}(\dG_m)$ par l'adhérence de Zariski du groupe 
$E_\fm\subset F^\times$ des unités totalement positifs $\equiv 1\pmod\fm$. 

D'après Serre \cite[II.3]{se68}, si $k$ est une clôture algébrique de 
$\dQ$, et que $\fm$ est assez grand, les caractères $\prod \omega^{n_\omega}$ 
de $\res_{F/\dQ}(\dG_m)$ de la forme $\chi_\text{alg}$ sont caractérisés 
comme suit: il doit exister un entier $N$, le \emph{poids} de $\chi$ (ou de 
$\chi_\text{alg}$), tel que pour tout élément $\sigma$ de $\gal(k/\dQ)$ 
conjugué à la conjugaison complexe, on ait $n_\omega+n_{\sigma \omega}=N$. Si 
$F_1$ est le plus grand sous-corps de $F$ qui soit une extension quadratique 
totalement imaginaire d'un corps totalement réel $F_1'$, cela revient à 
dire que $\chi_\text{alg}$ est de la forme $\chi_1\circ N_{F/F_1}$, et que 
$\chi_1|F_1=(N_{F_1/\dQ})^N$. 

Si $\chi$ est de poids $N$, pour tout idéal $\fa$ de $F$, $\chi(\fa)$ est un 
nombre algébrique dont tous les conjugués sont de valeur absolue 
$N(\fa)^{N/2}$ \cite[II.3 prop.2]{se68}. 





\subsection{}\label{VI:5-6}

Pour $k$ un corps de nombres fixé, des conditions supplémentaires sont 
imposées à $\chi_\text{alg}$. Pour tout idéal $\fa$ de $F$ premier au 
conducteur, on a en effet 
\begin{equation*}\tag{5.6.1}\label{VI:eq:5-6-1}
  \chi_\text{alg}(\fa) = (\chi(\fa)) \text{,} 
\end{equation*}
de sorte que $\chi_\text{alg}(\fa)$ est principal. Puisque le groupe des 
idéaux fractionnaire de $k$ est sous torsion, il suffit de prouver la 
puissance $n$-ième de \eqref{VI:eq:5-6-1}, $n\ne 0$ convenable; ceci permet 
de remplacer $\fa$ par $\fa^n$, donc de supposer $\fa=(x)$, avec $x$ totalement 
positif $\equiv 1\pmod\fm$. Il ne reste qu'à utiliser les définitions. 

Ceci, joint à \ref{VI:5-5}, montre que $\chi_\text{alg}$ détermine la norme 
des $\chi(\fa)$ en toutes les places de $k$. 





\subsection{}\label{VI:5-7}

Le groupe $S_\fm$ de Serre pourrait être caractérisé comme étant le 
groupe de type multiplicatif dont le groupe des caractères (sur n'importe 
quel corps) est le groupe des caractères de Hecke algébriques de conducteur 
$\leqslant \fm$. (cf. \cite[II 2.1 et 2.2]{se68}). Sa relation avec les 
représentations $\ell$-adiques est expliquée en \cite[II 2.3]{se68}. 





\begin{theorem_}[{\cite{se68}}]\label{VI:5-8}
Soit $\chi$ un caractère de Hecke algébrique de $F$ dans $E_\lambda$, pour 
$E_\lambda$ une extension finie de $\dQ_\ell$. Il existe alors un (et un seul) 
homomorphisme $\chi_\lambda:\gal(\bar F/F)^\textnormal{ab} \to E_\lambda$, tel 
que 
\begin{enumerate}[\indent (i)]
  \item $\chi_\lambda$ est non ramifié en dehors du conducteur $\ff$ de 
    $\chi$ et de $\ell$. 
  \item Pour $\fp$ un idéal premier de $F$ premier à $\ff$ et à $\ell$, 
    et $F_\fp\in \gal(\bar F/F)^\textnormal{ab}$ le Frobenius géométrique 
    en $\fp$, on a 
    \[
      \chi_\lambda(F_\fp) = \chi(\fp) \text{.}
    \] 
\end{enumerate}
\end{theorem_}





\subsection{}\label{VI:5-9}

Dans la fin de ce paragraphe, nous donnerons un critère pour qu'une 
représentation $\lambda$-adique provienne ainsi d'un caractère de Hecke, 
et, au paragraphe suivant, nous l'appliquerons aux sommes de Jacobi. Nous 
retrouverons ainsi, un peu généralisés des résultats de Weil 
\cite{we52,we74-2}, avec la seconde partie de la démonstration de Weil 
remplacée par un argument de cohomologie $\ell$-adique. 





\begin{theorem_}\label{VI:5-10}
Soient $F$ de $k$ deux corps de nombres, $\lambda$ une place de $k$ de 
caractéristique $\ell$, et 
$\chi_\lambda:\gal(\bar F/F)^\textnormal{ab}\to k_\lambda^\times$ un 
homomorphisme, non ramifié en dehors de $\ell$ et d'un ensemble fini $S$ de 
places. On suppose qu'il existe un ensemble $T$ de places (contenant $S$ et les 
places au-dessus de $\ell$), de densité $0$, et un homomorphisme algébrique 
$\chi_0:F^\times \to k^\times$ tels que 
\begin{enumerate}[(i)]
  \item Pour $\bar k$ une clôture algébrique de $k$, 
    $\chi_0:F^\times \to k^\times \hookrightarrow \bar k^\times$ est du type 
    considéré en \ref{VI:5-5}, de poids $N$; 
  \item Pour $\fp$ premier $\notin T$, $\chi_\lambda(F_\fp)$ est dans 
    $k^\times$, $(\chi_\lambda(F_\fp))=\chi_0(\fp)$, et tous les conjugués 
    complexes de $\chi_\lambda(F_\fp)$ sont de valeur absolue $(N \fp)^{N/2}$. 
\end{enumerate}

Alors, $\chi_\lambda$ est défini (\ref{VI:5-8}) par un caractère de Hecke 
algébrique de $F$ à valeurs dans $k$, de partie algébrique $\chi_0$. 
\end{theorem_}

Soit $\chi'$ un caractère de Hecke algébrique à valeurs dans une 
extension galoisienne finie $k'$ de $k$, de partie algébrique $\chi_0$ 
(\ref{VI:5-5}). Quitte à agrandir $T$, on peut supposer que le conducteur de 
$\chi'$ est à support dans $T$. Soient $\lambda'$ une place de $k'$ au-dessus 
de $\lambda$, $\chi_{\lambda'}$ le composé 
$\gal(\bar F/F)^\text{ab}\to k_\lambda^\times \to {k'}_{\lambda'}^\times$, et 
$\chi_{\lambda'}':\gal(\bar F/F)^\text{ab}\to {k'}_{\lambda'}^\times$ défini 
par $\chi'$ (\ref{VI:5-8}). Posons 
$\varepsilon = \chi_{\lambda'} \cdot {\chi'}_{\lambda'}^{-1}$. 

L'hypothèse (ii), et \ref{VI:5-5}, \ref{VI:5-6}, \ref{VI:5-8} assurent que 
pour $\fp\notin T$, on a $\varepsilon(F_\fp)\in {k'}^\times$, et que, dans tous 
les complétés de $k'$, ce nombre est de norme $1$. Il en résulte que 
$\varepsilon(F_\fp)$ appartient au groupe (fini) $\dmu$ racines de l'unité 
de $k'$. D'après le théorème de densité de Cebotarev, et l'hypothèse 
de densité faite sur $T$, les $F_\fp$ ($\fp\notin T$) sont denses dans 
$\gal(\bar F/F)^\text{ab}$, de sorte que $\varepsilon(\sigma)\in \mu$ pour tout 
$\sigma\in \gal(\bar F/F)^\text{ab}$: le caractère $\varepsilon$ est un 
caractère d'ordre fini $\gal(\bar F/F)^\text{ab}\to \mu$. D'après la 
théorie du corps de classes, il correspond à un caractère d'ordre fini 
du groupe des classes d'idèles de $F$, i.e. à un caractère de Hecke 
algébrique de partie algébrique triviale. Corrigeant $\chi'$ par ce 
caractère, on se ramène à supposer que 
$\chi_{\lambda'}=\chi_{\lambda'}'$,  et il reste à montrer que $\chi'$, à 
priori à valeurs dans $k'$, est en fait à valeurs dans $k$. Si 
$\sigma\in \gal(k'/k)$, $\chi'$ et ${\chi'}^\sigma$ coïncident sur tout idéal 
premier $\fp\notin T$, puisque 
$\chi'(\fp) = \chi'_{\lambda'}(F_\fp) = \chi_{\lambda'}(F_\fp) \in k^\times$. 
Les caractères $\chi_{\lambda'}'$, et ${\chi_{\lambda'}'}^\sigma$, coïncident 
donc sur une partie dense de $\gal(\bar F/F)^\text{ab}$, donc sont égaux, de 
sorte que $\chi'={\chi'}^\sigma$: $\chi'$ est à valeurs dans $k$. 










\section{Les caractères de Hecke définis par les sommes de Jacobi}\label{VI:6}

L'idée de ce paragraphe est la suivante. Soit une somme de Gauss 
$\tau_{\dF_q}(\chi,\psi)$. Elle-même, et le faisceau correspondant 
$\sF(\chi^{-1}\psi)$ sont deux fois liés à la caractéristique $p$, et au 
corps de base $\dF_q$: 
\begin{enumerate}[\indent a)]
  \item car il y appara\^it le caractère additif $\psi$, et le faisceau 
    $\sf(\psi)$; 
  \item car $\sF(\chi)$ est défni à partir de la suite exacte de Lang 
    \[\xymatrix{
      0 \ar[r] 
        & k^\times \ar[r] 
        & V^\ast \ar[r] 
        & V^\ast \ar[r] 
        & 0 \text{.} 
    }\]
\end{enumerate}
Deux difficultés si on veut relever sur un corps de nombres $F$ les 
constructions cohomologiques correspondantes, et trouver des 
représentations $\ell$-adiques de $\gal(\bar F/F)$ où les valeurs propres 
de Frobenius soient des sommes de Gauss. 

Si $\chi$ est trivial sur $\dF_q^\times$, la somme $\tau_{\dF_q}(\chi,\psi)$ 
est indépendante de $\psi$, et $q$ fois une somme de Jacobi. Celle-ci est 
liée à la cohomologie de $W^\ast/\dG_m$, à valeurs dans un faisceau 
déduit de $\chi$: $\psi$, et la difficulté a), ont disparu. Pour 
résoudre b), il suffit de décrire les faisceaux utilisés à l'aide de 
suites exactes de K\"ummer (qui ont un sens en toute caractéristique). C'est 
possible gr\^ace à \ref{VI:4-9}. 

Une fois trouvées les représentations $\ell$-adiques, on utilise 
\ref{VI:5-10} et le théorème de Stickelberger pour montrer qu'elles 
proviennent de caractères de Hecke algébriques. 





\subsection{}\label{VI:6-1}

Si $E$ est un corps, de clôture algébrique $\bar E$, et que $\lambda$ est 
un caractère d'ordre divisant $n$ de $\mu_d(\bar E)$, on note $_n\lambda$ le 
caractère de $\mu_n(\bar E)$ tel que $\lambda$ et $_n \lambda$ induisent le 
même caractère de $\widehat\dZ(1)_{\bar E} = \varinjlim \mu_n(\bar E)$. De 
même pour $\lambda$ un caractère de $\widehat\dZ(1)_{\bar E}$. 

Soient $F$ et $k$ deux corps de nombres, $\bar F$ un clôture algébrique de 
$F$, $I$ un ensemble fini muni d'une action de $\gal(\bar F/F)$ et 
$\lambda=(\lambda_i)_{i\in I}$ une famille de caractères 
$\lambda_i:\widehat\dZ(1)_{\bar F} \to k^\times$. On suppose que 
$\lambda_i\ne 1$ ($i\in I$), que le produit des $\lambda_i$ est trivial et que 
la famille $\lambda$ est défini sur $F$, i.e. que pour tout 
$\sigma\in \gal(\bar F/F)$, on a 
$\lambda_{\sigma(i)}=\lambda_i\circ \sigma^{-1}$. 

L'ensemble galoisien $I$ est l'ensemble des homomorphismes dans $\bar F$ d'une 
alg\`bre $E$ séparable sur $F$. Si $C$ est l'ensemble des orbites de 
$\gal(\bar F/F)$ dans $I$, on a une décomposition de $E$ en produit de corps 
$E=\prod_{\alpha\in C} E_\alpha$, où $\alpha$ s'identifie à l'ensemble des 
plongements de $E_\alpha$ dans $\bar F$. Si les caractères $\lambda_i$, pour 
$i\in \alpha$, sont d'ordre $d_\alpha$, le corps $E_\alpha$ contient les 
racines $d_\alpha$-ièmes de $1$ et il existe 
$\lambda_\alpha:\mu_{d_\alpha}(E_\alpha) \hookrightarrow k^\times$  tel que, si 
$i\in \alpha$ correspond au plongement $\sigma_i$ de $E_\alpha$ dans $\bar F$, 
on ait $_{d_\alpha} \lambda_i = \lambda_\alpha \sigma_i^{-1}$. 





\subsection{}\label{VI:6-2}

Soient $n>0$ un multiple des $d_\alpha$, $\Sigma$ l'ensemble des places de $F$ 
où $E/F$ se ramifie ou qui divisent $n$, $\mu$ une place de $k$ de 
caractéristique résiduelle $\ell$, et $A$ l'anneau des éléments de $F$ 
entiers en dehors de $\Sigma$ et $\ell$. La clôture intégrale $M$ de $A$ 
est finie étale de rang $N=|I|$ sur $A$. Avec les notations de \ref{VI:4-6}, 
\ref{VI:4-8}, les $_n \lambda_i$ définissent 
\[
  _n\lambda:(V^\ast/\dG_m)_n \to k^\times \subset k_\mu^\times \text{.} 
\]

Soit $a$ la projection de $W^\ast/\dG_m$ sur $\spec(A)$. D'après 
\ref{VI:4-20}, le faisceau $\eR^{N-2} a_! \sK_n(_n \lambda)$ est lisse de rang 
un; il définit une représentation $\ell$-adique 
\[
  j_\mu[\lambda]:\gal(\bar F/F)^\text{ab} \to k_\mu^\times 
\]
(ou simplement $j_\mu$) nono ramifiée en dehors de $\sigma$ et $\ell$. 





\subsection{}\label{VI:6-3}

Calculons $j_\mu(F_\fp)$. D'après \ref{VI:4-20}, il suffit de le faire 
après réduction modulo $\fp$. Soient donc $f=A/\fp$, $e=M/\fp  M$, et 
$\bar f$ la clôture algébrique de $f$ définie par une place de $\bar F$ 
au-dessus de $\fp$. On a encore $I=\hom_f(e,\bar f)$. Après réduction, 
$_n\lambda$ admet encore une description \ref{VI:6-1}: 
\begin{enumerate}[a)]
  \item Soit $D$ l'ensemble des orbites de $\gal(\bar f/f)$ dans $I$. La 
    décomposition de $e$ en produit de corps s'écrit 
    $e=\prod_{\beta\in D} e_\beta$, où les $\sigma_i$ ($i\in \beta$) 
    s'identifient aux plongements de $e_\beta$ dans $F$. Pour $\beta\in D$, on 
    note $\alpha(\beta)$ l'élément de $C$ contenant $\beta$ et 
    $d_\beta=d_{\alpha(\beta)}$. Soit 
    \[\xymatrix{
      \lambda_\beta : \mu_{d_\beta}(e_\beta) 
        & \ar[l]_-\sim \mu_{d_\beta}(E_{\alpha(\beta)}) \ar[r]^-{\lambda_{d(\beta)}} 
        & k^\times \text{.} 
    }\]
  \item Si $i\in I$ correspond au plongement $\sigma_i$ de $e_\beta$ dans 
    $\bar f$, la réduction 
    $_{d\beta} \bar\lambda_i:\mu_{d\beta}(\bar f) \to k^\times$ de 
    $_{d\beta} \lambda_i$ de $_{d\beta} \lambda_i$ est 
    $\lambda_\beta \circ \sigma_i^{-1}$. 
\end{enumerate}

Soient $q_\beta$ le nombre d'éléments de $e_\beta$, 
$\chi_\beta=_{(q_{\beta^{-1}})} \lambda_\beta$, et 
$\chi:e^\times \to k^\times$ de coordonnées les $\chi_\beta$. Sur l'espace 
affine sur $A/\fp$ défini par $e$ (\ref{VI:4-6}), ou plutôt sur le tore 
$e_\beta^\times$, on a des isomorphismes 
$\sK_n((_n\bar\lambda_i)_{i\in\beta}) = \sK_{d_\beta}((\lambda_\beta \circ \sigma_i^{-1})_{i\in\beta}) =\sK_{q_{\beta^{-1}}}((\chi_\beta\circ \sigma_i^{-1})_{i\in I}) = \sF(\chi_\beta)$ (\ref{VI:4-9}). 
Sur $e^\times$, on a donc $\sK_n((_n\bar\lambda_i)_{i\in \beta}) = \sF(\chi)$ 
et cet isomorphismes se descend à la réduction mod $\fp$ de $V^\ast/\dG_m$. 
Dès lors





\begin{proposition_}\label{VI:6-4}
Avec les notations de \ref{VI:6-2} et \ref{VI:6-3}, $j_\mu(F_\fp)$ est la 
somme de Jacobi $J(\chi)$. 
\end{proposition_}

Récrivons les formules liant $(\lambda_i)_{i\in I}$ à 
$(\chi_\beta)_{\beta\in B}$: pour $i\in \beta\subset \alpha$, 
\begin{align*}
  _{d_\alpha} \lambda_i &= \lambda_\alpha\circ \sigma_i^{-1} && \lambda_\alpha:\mu_{d_\alpha}(E_\alpha) \to k^\times \\
  \lambda_\beta &= \text{réduction de $\lambda_\alpha$} && :\mu_{d_\alpha}(e_\beta) \to k^\times \\
  \chi_\beta &= _{q_{\beta^{-1}}}\lambda_\beta: x\mapsto \lambda_\beta(x^{(q_\beta-1)/d_\alpha}) && :e_\beta^\times \to k^\times 
\end{align*}





\begin{theorem_}\label{VI:6-5}
La représentation $j_\mu$ est définie par un caractère de Hecke 
algébrique de $F$ à valeurs dans $k^\times$. 
\end{theorem_}

Nous le vérifierons à l'aide de \ref{VI:5-10}. On rejoint ici la 
démonstration de Weil \cite{we52,we74-2}, par l'usage du théorème de 
Stickelberger, et je ne donnerai que quelques indications. 





\subsection{}\label{VI:6-6}

Etant donné $j:\gal(\bar F/F) \to k_\lambda^\times$, et une extension finie 
$F'/F$, on vérifie à l'aide de \ref{VI:5-10} que $j$ est défini par un 
caractère de Hecke si et seulement si $j|\gal(\bar F/F')$ l'est. Cette 
remarque permet dans \ref{VI:6-5} de se réduire au cas où Galois agit 
trivialement sur $I$, et où $F$ contient les racines $n$-ièmes de 
l'unité. On peut ensuite se réduire au cas où $F=\dQ(\sqrt[n]{1})$, puis 
prendre $k=F$. Dans ce cas, chaque $_n \lambda_i$ est défini par 
$a_i\in \dZ/n$: c'est $x\mapsto x^{a_i}:\mu_n \to k^\times$. On a 
$\sum a_i=0$, et $a_i\ne 0$. 





\subsection{}\label{VI:6-7}

Dans un corps de nombres, les places de degré $1$ forment toujours un 
ensemble de densité $0$. En appliquant \ref{VI:5-10}, on peut donc négliger 
les autres, et n'utiliser Stickelberger que pour un corps premier. Soient donc 
$p$ premier, $0<a<p-1$, $\widetilde x:\dF_p^\times \to \dQ_p^\times$ le 
caractère ``relèvement multiplicatif'' et 
$g=-\sum_{x\in \dF_p^\times} \widetilde x^{-a} \zeta^z\in \dQ_p(\zeta)$, pour 
$\zeta$ une racine $p$-ième de $1$. Notons encore $x$ l'entier dans $[0,p-1]$ 
relevant $x$, et développons $\zeta^x=(1+\pi)x$ par la formule du binôme. 
On trouve $g=-\sum_{0=0}^{p-1} A_i\pi^i$ avec $A_i\in \dZ_p$, de réduction 
$\mod p$ la somme $\sum_{x\in \dF_p^\times} x^{-a} \binom x i$. Pour $b$ non 
divisible par $p-1$, on a $\sum_{x\in \dF_p^\times} x^b=0$. Si $i<a$, 
$\binom x i$ est un polynôme en $x$ de degré $i<a$, d'où 
$A_i\equiv 0\pmod p$. De même, $A_a\equiv \frac{p-1}{a!} \pmod p$, et 
$g\sim \frac{\pi^a}{a!}$: la valuation de $g$ est donnée par 
$v(g)/v(p) = \frac{a}{p-1}$. 





\subsection{}\label{VI:6-8}

Soit $\sigma_i$ ($i\in (\dZ/n)^\times$) l'élément de 
$\gal(\dQ(\sqrt[n]{1})/\dQ)$ tel que $\sigma_i(\zeta)=\zeta^i$ pour $\zeta^n=1$, 
et notons additivement le groupe des homomorphismes algébriques 
$\dQ(\sqrt[n]{1})^\times \to \dQ(\sqrt[n]{1})^\times$. Pour $\lambda$ comme en 
\ref{VI:6-6}, défini par une famille $(a_i)_{i\in I}$ d'entiers $\mod n$, non 
nuls et de somme nulle \ref{VI:5-10}, \eqref{VI:eq:4-15-1} et \ref{VI:6-7} 
montrent que $j_\mu[\lambda]$ st défini par un caractère de Hecke 
algébrique $j[a]$ et que, si $N$ est la norme, sa partie algébrique est 
donnée par 
\begin{equation*}\tag{6.8.1}\label{VI:eq:6-8-1}
  (N j[a])_\text{alg} = \sum_{i\in (\dZ/n)^\times} \sum_{j\in I} \left \lfloor \frac{i a_j}{n} \right\rfloor \sigma_i^{-1} 
\end{equation*}
où $\lfloor \cdot \rfloor$ est la partie fractionnaire. 





\subsection{}\label{VI:6-9}

Le cas le plus intéressant est celui où $F\subset \dQ(\sqrt[n]{1})$, et 
où la famille des $_n \lambda_i$ est construite comme suite: 
\begin{enumerate}[a)]
  \item On prend une famille finie $(Aj)_{j\in J}$ d'orbites de 
    $\gal(\sqrt[n]{1}/F)=H\subset (\dZ/n)^\times$ dans $\dZ/n$. On suppose que 
    $A_j\ne\{0\}$ et que $\sum_{j\in J} \sum_{a\in A_j} a=0$. 
  \item On prend $I=\sqcup_{j\in J} A_j$; Galois agit sur chaque $A_j$. 
  \item On prend $\bar F\supset \dQ(\sqrt[n]{1})$, $k=\dQ(\sqrt[n]{1})$ et, si 
    $i\in I$ correspond à $a\in A_j\subset (\dZ/n)^\times$, 
    $_n\lambda_i(x)=x^a$. 
\end{enumerate}

On vérifie à l'aide de \ref{VI:4-20} qu'un $j_\mu$ général (relatif à 
$F_1$) se déduit d'un $j_\mu$ comme ci-dessus (pour 
$F\subset \dQ(\sqrt[n]{1})$, $F_1$ extension de $F$) par restriction à 
$\gal(\bar F/F_1)$ et multiplication par un caractère d'ordre deux. 

Pour $a\in \gal(k/\dQ)=(\dZ/n)^\times$, $a$ transformé la famille des $A_j$ 
en celle des $a A_j$. Si $a\in H$, la famille des $A_j$ est donc préservée, 
et les $j_\mu(F_\fp)$ sont invariant par $H$: 





\begin{proposition_}\label{VI:6-10}
Sous les hypothèses de \ref{VI:6-9}, $j_\mu$ est défini par un caractère 
de Hecke algébrique de $F$ à valeurs dans $F$. 
\end{proposition_}

Faisons à nouveau $F=\dQ(\sqrt[n]{1})$, et considérons les caractères de 
Hecke $j[a]$ définis par une famille $(a_i)_{i\in I}$, $i\in \dZ/n$, 
$a_i\ne 0$, $\sum a_i=0$. Le théorème suivant a été obtenu 
indépendamment par Vishik. 





\begin{theorem_}\label{VI:6-11}
Tout caractère de Hecke algébrique de $\dQ(\sqrt[n]{1})$, $a$ un puissance 
dans le groupe engendré par les $j[a]$ et la norme. 
\end{theorem_}

Puisqu'on travaille ``à torsion près,'' seule compte la partie algébrique 
des caractères de Hecke considérés. Appliquant \eqref{VI:eq:6-8-1} aux 
$j[a]$, avec $a=$ un élément de $\dZ/n$ répété $n$ fois, on se 
ramène au lemme suivant.





\begin{lemma_}\label{VI:6-12}
Notons $\sum x_j \sigma_j$ les éléments de l'algèbre de groupe 
$\dQ[(\dZ/n)^\times]$, et soit $x$ le sous-groupe de ceux tels que 
$x_j+x_{-j}=C^{\underline{t e}}$. Alors, $X$ est engendré sur $\dQ$ par 
$N=\sum \sigma_j$ et par les 
$g_a=\sum \left\lfloor \frac{i a}{n}\right\rfloor \sigma)i^{-1}$ 
($a\in \dZ/n$, $a\ne 0$). 
\end{lemma_}

Tant $X$ que $Y=\langle N,(g_a)\rangle$ sont des idéaux de l'algèbre du 
groupe, et il suffit de vérifier qu'ils sont annulés par les mêmes 
caractères complexes $\chi$ de $(\dZ/n)^\times$. Ceci revient à dire qu'un 
caractère impair $(\chi(j)=-\chi(-j)$) n'annule jamais tous les $g_a$. Si 
$\chi$ est primitif, on a 
\[
  \chi(g_1) = \sum \frac i n \chi^{-1}(i) = -L(\chi,0)\ne 0 \text{.} 
\]
Dans le cas général, si $\chi$ provient d'un caractère primitif de 
$(\dZ/n)^\times$, $\chi(g_{n/d})$ se ramène à une somme analogue sur 
$(\dZ/n)^\times$, d'où le lemme. 










\section{Sommes de Kloosterman généralisées}\label{VI:7}





\subsection{}\label{VI:7-1}

Soient $\dF_q$ un corps fini à $q$ éléments, et $\psi$ un caractère 
additif non trivial. Dans ce paragraphe, nous faisons une étude cohomologique 
des sommes de Kloosterman généralisées 
\begin{equation*}\tag{7.1.1}\label{VI;eq:7-1-1}
  \kloos_{n,a} = \sum_{x_1\dotsm x_n = a} \psi(x_1+\cdots + x_n) \qquad (n\geqslant 1) \text{.} 
\end{equation*}
Pour $a=0$, cette somme est élémentaire (cf. \ref{VI:7-7}): 
\begin{equation*}\tag{7.1.2}\label{VI:eq:7-1-2}
  \kloos_{n,0} = (-1)^{n-1} \text{.} 
\end{equation*}
La cas intéressant est celui où $a\ne 0$. Les $x_i$ sont alors dans 
$\dF_q^\times$. Nous prouverons dans ce cas une majoration 
\begin{equation*}\tag{7.1.3}\label{VI:eq:7-1-3}
  \left|K_{n,a}\right| \leqslant n q^{\frac{n-1}{2}} \text{.} 
\end{equation*}

Nous outils essentiels seront les énoncés cohomologiques de reflet les 
identités suivantes 
\begin{align*}\tag{7.1.4}\label{VI:eq:7-1-4}
  \kloos_{n,a} &= \sum_{x_1} \psi(x_i) \sum_{x_2\dotsm x_n = a/x_1} \psi(x_2+\cdots + x_n) \\ \notag
    &= \sum_{x\in \dF_q^\times} \psi(x) \kloos_{n-1,a/x} \qquad (a\ne 0,n\geqslant 2) \\ \tag{7.1.5}\label{VI:eq:7-1-5}
  \sum_a \kloos_{n,a} &= \sum_a \sum_{x_1 \dotsm x_n = a} \psi\left(\sum x_i\right) = \sum_{x_1,\dots,x_n} \psi\left(\sum x_i\right) = 0 
\end{align*}
Pour $\chi$ un caractère non trivial de $\dF_q^\times$, 
\begin{equation*}\tag{7.1.6}\label{VI:eq:7-1-6}
  \sum \chi(a) \kloos_{n,a} = \sum_{x_1,\dots,x_n} \chi\left(\prod x_i\right) \cdot \psi\left(\sum x_i\right) = (-\tau(\chi^{-1},\psi))^n 
\end{equation*}
(somme de Gauss). 





\subsection{}\label{VI:7-2}

Soient $k$ un algèbre étale de degré $n$ sur $\dF_q$, et $\varepsilon(k)$ 
comme en \eqref{VI:eq:4-5-1}. Posons 
\[
  \kloos_{k,a} = \sum_{N_{k/\dF_q}(x)=a} \psi \tr(x) \text{.} 
\]
On a encore 
\begin{align*}\tag{7.2.1}\label{VI:eq:7-2-1}
  \kloos_{k,0} &= (-1)^{n-1} \cdot \varepsilon(k) \\ \tag{7.2.2}\label{VI:eq:7-2-2}
  \sum_a \kloos_{k,a} &= 0 \\ \tag{7.2.3}\label{VI:eq:7-2-3}
  \sum_a \chi(a) \kloos_{k,a} &= (-1)^n \tau_{\dF_q}\left(\chi^{-1}\circ N_{k/\dF_q},\psi\right) \text{.} 
\end{align*}

Par inversion de Fourier sur $\dF_q^\times$, on en déduit une expression de 
$\kloos_{k,a}$ en terme de sommes de Gauss: 
\begin{align*} 
  \kloos_{k,a} &= \frac{1}{q-1} \sum_\chi \chi(a) \sum_{b\in \dF_q^\times} \chi^{-1}(b) \kloos_{k,b} \\ \tag{7.2.4}\label{VI:eq:7-2-4}
  (-1)^n \kloos_{k,a} &= \frac{1}{q-1} \left(\varepsilon(k) + \sum_{\chi\ne 1} \chi(a) \tau_{\dF_q}\left(\chi\circ N_{k/\dF_q},\psi\right) \right) \text{.} 
\end{align*}

L'identité de Hasse-Davenport permet maintenant de comparer $\kloos_{k,a}$ 
à $\kloos_{n,a} = \kloos_{\dF_{q^n},a}$: 
\begin{equation*}\tag{7.2.5}\label{VI:eq:7-2-5}
  \kloos_{k,a} = \varepsilon(k) \kloos_{n,a} \text{.} 
\end{equation*}

Je ne connais pas d'interprétation cohomologique de \eqref{VI:eq:7-2-4}, mais 
j'en donnerai une de \eqref{VI:eq:7-2-5}. Revenons aux sommes $\kloos_{n,a}$. 





\subsection{}\label{VI:7-3}

Soit $\dF$ un clôture algébrique de $\dF_q$ et, pour $a\in \dF$, soit 
$V_a^{n-1}$, ou simplement $V_a$, l'hypersurface de $\dA^n$ d'équation 
$x_1 \dotsm x_n=a$. Comme d'habitude, nous regarderons $\psi$ comme étant à 
valeurs $\ell$-adiques plutôt que complexes. Avec la notation 
\ref{VI:1-8}(ii), \eqref{VI:eq:7-1-3} se déduit du 





\begin{theorem_}\label{VI:7-4}
La cohomologie de $V_a^{n-1}$ à valeurs dans $\sF(\psi(\sum x_i))$ vérifie 
\begin{enumerate}[\indent (i)]
  \item $\h_c^i=0$ pour $i\ne n-1$; 
  \item $\h_c^\bullet{} \iso \h^\bullet{}$; 
  \item pour $a\ne 0$, $\dim \h_c^{n-1}{}=n$; 
  \item pour $a=0$, $\h_c^{n-1}{}$ est canoniquement isomorphe à $E_\lambda$. 
\end{enumerate}
\end{theorem_}





\subsection{}\label{VI:7-5}

Comme d'habitude, ce théorème contrôle aussi la dépendance de 
$\kloos_{n,a}$ en $q$: pour chaque $a\in \dF_q^\times$, il existe $n$ valeurs 
propres de Frobenius $\alpha_1,\dots,\alpha_n$, de valeur absolue complexe 
$q^{\frac{n-1}{2}}$, telles que 
\[
  \sum_{\substack{x_1\dotsm x_n = a \\ x_i\in \dF_{q^m}}} \psi\circ \tr\left(\sum x_i\right) = (-1)^{n-1} \left(\alpha_1^m + \cdots + \alpha_n^m \right) \text{.} 
\]

Pour $n$ pair, $x\mapsto -x$ est une involution de $V_a$, qui transforme 
$\sF(\psi)$ en son dual. Raisonnant comme en \ref{VI:3-6}, on en déduit une 
forme alternée non dégénérée canonique sur $\h_c^{n-1}(V_a,\sF(\psi))$, 
à valeurs dans $\dQ_\ell(-(n-1))$ et le 





\begin{corollary_}\label{VI:7-6}
Avec les notations de \ref{VI:7-5}, si $n$ est pair, les $\alpha_i$ se groupent 
en $\frac n 2$ paires de racines $\alpha,\bar\alpha$ de produit égal à 
$q^{n-1}$. 
\end{corollary_}





\subsection{}\label{VI:7-7}

Vérifions le cas $a=0$ de \ref{VI:7-5}. L'hypersurface $V_0$ est la réunion 
des hyperplans de coordonnée $x_i=0$. Calculons la suite spectrale de Leray 
\eqref{VI:eq:2-6-1*} de ce recouvrement de $V_0$, pour la cohomologie avec ou 
sans support à coefficient dans $\sF(\psi)$. D'après \ref{VI:2-7*}*, tout 
les termes initiaux sont nuls, sauf la cohomologie de l'intersection $\{0\}$ de 
tous les hyperplans de coordonnée. Là subsiste un isomorphisme 
$\h_c^0{}\iso \h^0{}=E_\lambda$ qui justifie \ref{VI:7-4}. 

Nous prouverons \ref{VI:7-4} et \ref{VI:7-8} ci-dessous par une récurrence 
simultanée sur $n$, et nous appuyant sur la suite spectrale dont 
\eqref{VI:eq:7-1-4} est le reflet. Ceci exige un contrôle de la dépendance 
en $a$ de $\kloos_{n,a}$. 

Soit $\pi:\dA^n \to \dA$ l'application ``produit des coordonnées,'' et 
$\sigma$ la somme des coordonnées. Les hypersurfaces $V_a$ sont les fibres de 
$\pi$. 





\begin{theorem_}\label{VI:7-8}
\begin{enumerate}[(i)]
  \item Le faisceau $\eR^{n-1}\pi_! \sF(\psi\sigma)$ est lisse de rang $n$ sur 
    $\dA^1\setminus \{0\}$. 
  \item son prolongement par $0$ sur $\dP^1$ est ll'image directe de sa 
    restriction à $\dA^1\setminus \{0\}$. 
  \item en $0$, la monodromie est unipotente, avec un seul bloc de Jordan
  \item en $\infty$, l'inertie sauvage agit sans point fixe $\ne 0$, et le 
    conducteur de Swan vaut $1$, 
  \item on a $\eR^i \pi_! \sF(\psi\sigma) \iso \eR^i \pi_\ast \sF(\psi\sigma)$ 
    (nul pour $i\ne n-1$).
\end{enumerate}
\end{theorem_}
\begin{proof}[Preuve]
de ce que \ref{VI:7-4} (pour une valeur donnée de $n$) implique \ref{VI:7-8} 
(pour la même valeur de $n$). 

La fibre de $\eR^i \pi_! \sF(\psi\sigma)$ en $a$ est 
$\h_c^i(V_a,\sF(\psi\sigma))$, et sur l'ouvert dense $\dA^1$ où les 
$\eR^i \pi_!\sF(\psi^{-1}\sigma)$ sont localement constants celle de 
$\eR^i \pi_\ast \sF(\psi\sigma)$ est $\h^i(V_a,\sF(\psi\sigma))$ 
(\hyperref[VII]{Th.\ Finitude} \ref{VII:2-1}). L'hypothèse \ref{VI:7-4}($n$) 
nous fournit donc le 





\begin{lemma_}\label{VI:7-9}
\begin{enumerate}[(i)]
  \item $\eR^i \pi_! \sF(\psi\sigma)=0$ pour $i\ne n-1$. 
  \item Pour $i=n-1$, la fibre de ce faisceau en tout point $a\ne 0$ est de rang 
    constant $n$. En $0$, elle est de rang $1$. 
  \item Sur un ouvert dense $U$, on a 
    $\eR^i \pi_! \sF(\psi) \iso \eR^i \pi_\ast \sF(\psi)$. 
\end{enumerate}
\end{lemma_}





\subsection{}\label{VI:7-10}

Nous allons maintenant utiliser la suite spectrale de Leray de $\pi$, pour la 
cohomologie avec ou sans support, à coefficients dans $\sF(\psi\sigma)$. 
L'aboutissement est connu: 
$\h_c^\bullet(\dA^n,\sF(\psi\sigma))=\h^\bullet(\dA^n,\sF(\psi\sigma)) = 0$ 
(\ref{VI:2-7*}*). En cohomologie à support propre, \ref{VI:7-9}(i) assure que 
$E_2^{pq}=0$ pour $q\ne n-1$. Les termes $E_2$ sont donc tous nuls: 
\begin{equation*}\tag{7.10.1}\label{VI:eq:7-10-1}
  \h_c^p(\dA^1,\eR^{n-1} \pi_!\sF(\psi\sigma)) = 0 \text{.} 
\end{equation*}

Pour $p=0$, ceci signifie que $\eR^{n-1}\pi+! \sF(\psi\sigma)$ est sans section 
à support ponctuel. Vu la constance de rang \ref{VI:7-9}(ii), on en déduit 
\begin{equation*}\tag{7.10.2}\label{VI:eq:7-10-2}
  \begin{array}{c}
    \text{Le faisceau $\eR^{n-1} \pi_! \sF(\psi\sigma)$ sur $\dA^1$ est lisse sur $\dA^1\setminus \{0\}$, et est } \\ 
    \text{un sous-faisceau de l'image directe $\sG$ de sa restriction à $\dA^1\setminus \{0\}$.}
  \end{array}
\end{equation*}

La suite exacte longue de cohomologie de 
\[\xymatrix{
  0 \ar[r] 
    & \eR^{n-1} \pi_! \sF(\psi\sigma) \ar[r] 
    & \sG \ar[r] 
    & \sQ \ar[r] 
    & 0 \text{.} 
}\]
où le support de $\sQ$ est concentré en $0$, donne 
\[\xymatrix{
  \h_c^0(\dA^1,\sG) \ar[r] 
    & \sQ_0 \ar[r] 
    & \h_c^1(\dA^1,\eR^{n-1} \pi_! \sF(\psi\sigma)) \text{.} 
}\]
Les termes extrêmes étant nuls, $\sQ=0$ et 
\begin{equation*}\tag{7.10.3}\label{VI:eq:7-10-3}
  \text{$\eR^{n-1} \pi_! \sF(\psi\sigma)$ est l'image directe de sa restriction à $\dA^1\setminus \{0\}$.}
\end{equation*}

Soit $j$ l'inclusion de $\dA^1$ dans $\dP^1$. La version ``catégorie 
dérivée'' des suites spectrales de Leray est 
\begin{align*}
  \hh^\bullet(\dP^1,j_! \eR\pi_! \sF(\psi\sigma)) &= \hh_c^\bullet(\dA^1,\eR \pi_! \sF(\psi\sigma)) = \h_c^\bullet(\dA^n,\sF(\psi\sigma)) = 0 \\
  \hh^\bullet(\dP^1,\eR j_\ast \eR \pi_\ast \sF(\psi\sigma)) &= \hh^\bullet(\dA^1,\eR \pi_\ast \sF(\psi\sigma)) = \h^\bullet(\dA^n,\sF(\psi\sigma)) = 0 \text{.}
\end{align*}
Soit $\Delta$ le mapping cylinder de 
$j_! \eR\pi_! \sF(\psi\sigma) \to \eR j_\ast \eR \pi_\ast \sF(\psi\sigma)$. 
D'après \ref{VI:7-9}(iii), les faisceaux de cohomologie de ce complexe de 
faisceaux sont à support fini. D'autre part, la suite exacte longue de 
cohomologie du triangle 
$(j_!\eR \pi_! \sF(\psi\sigma),\eR j_\ast \eR \pi_\ast \sF(\psi\sigma),\Delta)$ 
montre que $\hh^\bullet(\dP^1,\Delta)=0$. On a 
$\hh^\bullet(\dP^1,\Delta) = \h^0(\dP^1,\sH^\bullet(\Delta))$, et finalement 
$\Delta=0$: 
\[\xymatrix{
  j_! \eR\pi_! \sF(\psi\sigma) \ar[r]^-\sim 
    & \eR j_\ast \eR \pi_\ast \sF(\psi\sigma) \text{.} 
}\]

En particulier, $\eR\pi_! \sF(\psi\sigma)\iso \eR\pi_\ast \sF(\psi\sigma)$ et, 
ces complexes de faisceaux n'ayant qu'un faisceau de cohomologie non nul, 
\begin{equation*}\tag{7.10.4}\label{VI:eq:7-10-4}
\xymatrix{
  j_! \eR \pi_! \sF(\psi\sigma) \ar[r]^-\sim 
    & j_\ast \eR \pi_! \sF(\psi\sigma) \text{.} 
}
\end{equation*}
Ceci complète la preuve de (i) (ii) (v). 





\subsection{}\label{VI:7-11}

Soient $\swan_0$ et $\swan_\infty$ les conducteurs de Swan en $0$ et en 
$\infty$ de $\eR^{n-1} \pi_! \sF(\psi\sigma)$. On a 
$\chi_c(\eR^{n-1} \pi_! \sF(\psi\sigma))=0$ \eqref{VI:eq:7-10-1} et la formule 
d'Euler-Poincaré \eqref{VI:eq:3-2-1} se réduit à 
\begin{equation*}\tag{7.11.1}\label{VI:eq:7-11-1}
  \swan_0 + \swan_\infty =1 \text{:} 
\end{equation*}
un de cces conducteurs vaut $0$ (ramification modérée), l'autre vaut $1$. 





\subsection{}\label{VI:7-12}

Soit $\sG$ un faisceau de rang $1$ sur $\dG_m$, de type kummérien 
\ref{VI:4-7} ($\sG=\sK_n(\chi)$) et non constant ($\chi\ne 1$). Le faisceau 
$\sF(\psi\sigma)\otimes \pi^\ast \sG$ sur 
$V^\ast=\pi^{-1}(\dG_m) = \dA^n\setminus \pi^{-1}(0)$ est alors du type 
rencontré dans l'étude des sommes de Gauss, et \ref{VI:4-11}(ii) nous donne 
\begin{equation*}\tag{7.12.1}\label{VI:eq:7-12-1}
\xymatrix{
  \h_c^\bullet(V^\ast,\pi^\ast\sG\otimes \sF(\psi\sigma)) \ar[r]^-\sim 
    & \h^\bullet(V^\ast,\pi^\ast \sG\otimes \sF(\psi\sigma)) \text{.} 
}
\end{equation*}
La suite spectrale de Leray de $\pi$ transforme cet énoncé en 
\begin{equation*}\tag{7.12.2}\label{VI:eq:7-12-2}
\xymatrix{
  \h_c^\bullet(\dG_m,\sG\otimes \eR^{n-1} \pi_! \sF(\psi\sigma)) \ar[r]^-\sim 
    & \h^\bullet(\dG_m,\sG\otimes \eR^{n-1} \pi_\ast \sF(\psi\sigma)) \text{.} 
}
\end{equation*}

Soit $i$ l'inclusion de $\dG_m$ dans $\dP^1$, et $\Delta$ le mapping cylinder 
de 
$\delta:i_!(\sG\otimes \eR^{n-1}\pi_! \sF(\psi))\to \eR i_\ast (\sG\otimes \eR^{n-1} \pi_! \sF(\psi\sigma))$
(où $\eR^{n-1} \pi_! = \eR^{n-1}\pi_\ast$). Raisonnant comme en 
\ref{VI:7-10}, on déduit de \eqref{VI:eq:7-12-2} que $\Delta=0$. En 
particulier 
\begin{equation*}\tag{7.12.3}\label{VI:eq:7-12-3}
  \text{En $0$ et en $\infty$, l'inertie agit sans point fixe $\ne 0$ sur $\sG\otimes \eR^{n-1}\pi_! \sF(\psi\sigma)$.}
\end{equation*}

En $\infty$, ceci vaut même si $\sG$ est trivial \eqref{VI:eq:7-10-4}, de 
sorte que l'inertie sauvage agit sans point fixe. En particulier, la 
ramification est sauvage: d'après \ref{VI:7-11}, on a $\swan_\infty=1$, et la 
ramification en $0$ es modérée. 

En $0$, \eqref{VI:eq:7-12-3} impose à la ramification -- modérée -- de 
$\eR^{n-1} \pi_! \sF(\psi\sigma)$ être unipotente. D'après 
\eqref{VI:eq:7-10-3} et \ref{VI:7-9}(ii), on a seul bloc de Jordan. Ceci 
achève la preuve de \ref{VI:7-8}($n$). 





\subsection{}\label{VI:7-13}

Pour $n=1$, \ref{VI:7-4} est trivial. Pour achever la preuve de \ref{VI:7-4} et 
\ref{VI:7-8}, il nous reste à prouver que \ref{VI:7-8}, pour une valeur 
donnée de $n$, implique \ref{VI:7-4}, pour $n+1$. Le cas $a=0$ ayant été 
traité (\ref{VI:7-7}), on suppose que $a=0$. On écrira $x_0,\dots,x_n$ pour 
les coordonnées de $\dA^{n+1}$. Soient $g=x_0:V_a\to \dG_m$, $\tau$ 
l'involution $x\mapsto a x^{-1}$ de $\dG_m$, et soit comme plus haut $\pi$ le 
produit des coordonnées: $\dA^{n+1} \to \dA^1$, et $\sigma$ leur somme. On 
notera $\sF(\psi\sigma)$ le faisceau $\sF(\psi(\sum x_i))$ tant sur $\dA^{n+1}$ 
que sur $\dA^n$. 





\subsection{}\label{VI:7-14}

Écrivons la suite spectrale de Leray de $g$. Le faisceau $\sF(\psi\sigma)$ sur 
$\dA^{n+1}$ étant produit tensoriel externe des faisceaux $\sF(\psi)$ sur 
$\dA^1$ et $\sF(\psi\sigma)$ sur $\dA^n$, on trouve (cf. \eqref{VI:eq:7-1-4}). 
\begin{align*}
  'E_2^{p q} &= \h_c^p(\dG_m,\sF(\psi)\otimes \tau^\ast \eR^q \pi_! \sF(\psi\sigma)) \Rightarrow \h_c^{p+q}(V_a,\sF(\psi\sigma)) \\
  ''E_2^{p q} &= \h^p(\dG_m,\sF(\psi)\otimes \tau^\ast \eR^q \pi_\ast \sF(\psi\sigma)) \Rightarrow \h^{p+q}(V_a,\sF(\psi\sigma)) \text{.} 
\end{align*}
Par hypothèse, 
$\eR^q \pi_! \sF(\psi\sigma) \iso \eR^q \pi_\ast \sF(\psi\sigma)$, et ce 
faisceau est nul pour $q\ne n-1$. Le faisceau $\sF(\psi)$ sur $\dG_m$ est 
sauvagement ramifié en $\infty$ et non ramifié en $0$, tandis que 
$\tau^\ast \eR^q \pi_! \sF(\psi\sigma)$ est sauvagement ramifié en $0$ (sans 
invariant sous l'inertie sauvage, de conducteur de Swan $1$) et modérément 
ramifié en $\infty$. Leur produit tensoriel est donc 
\begin{enumerate}[\indent a)]
  \item sauvagement ramifié en $0$ et $\infty$, sans invariant sous l'inertie 
    sauvage; 
  \item de conducteur de Swan $1$ en $0$ et $n$ (le rang de 
    $\eR^q \pi_! \sF(\psi))$ en $\infty$. 
\end{enumerate}
On a $'E_2^{p q} \iso ''E_2^{p q}$, et $'E_2^{p q}=0$ sauf pour $p=1$ et 
$q=n-1$. La formule d'Euler-Poincaré donne enfin 
\[
  \dim{'E_2^{p q}} = 1+n \text{;} 
\]
ceci achève la démonstration. 
\end{proof}





\subsection{Remarque}\label{VI:7-15}

L'isomorphisme obtenu en \ref{VI:7-14} 
\begin{equation*}\tag{7.15.1}\label{VI:eq:7-15-1}
  \h_c^n(V_a^n,\sF(\psi\sigma)) = \h_c^1(\dG_m,\sF(\psi)\otimes \tau^\ast \eR^{n-1} \pi_! \sF(\psi\sigma)) 
\end{equation*}
permet, avec les notations de \ref{VI:7-5}, de calculer le produit $\det(F)$ 
des valeurs propres $\alpha_i$ de Frobenius. Le formalisme de \cite{de73} 
s'applique, puisque $\sF(\psi)$ et $\eR^{n-1} \pi_! \sF(\psi\sigma)$ 
appartiennent à des systèmes compatibles infinis de représentations 
$\ell$-adiques. Ceci nous ramène à des problèmes locaux, qu'on peut 
simplifier en notant que (cf. \cite[9.5]{de73}) 
\begin{enumerate}[\indent a)]
  \item les constantes globales pour 
    $\sF(\psi)$ et $\tau^\ast \eR^{n-1} \pi_! \sF(\psi)$ sont aisées à 
    calculer, la cohomologie de ces faisceaux étant de nature triviale. 
  \item en tout point, pour l'un ou l'autre de ces faisceaux, la 
    représentation $\ell$-adique correspondante du groupe de décomposition 
    a une semi-simplifiée non ramifiée. 
\end{enumerate}

La résultat, pour la somme à $n$ variables, est que 
\begin{equation*}\tag{7.15.2}\label{VI:eq:7-15-2}
  \det\left(F,\h_c^{n-1}(V_a^{n-1},\sF(\psi))\right) = q^{\frac{n(n-1)}{2}} \text{.} 
\end{equation*}





\subsection{Remarque}\label{VI:7-16}

Pour $p=2$, le faisceau $\sf(\psi)$ est orthogonal. Pour $a\ne 0$ et $n$ 
impair, on en déduit par \ref{VI:7-4}(ii) et dualité de Poincaré une 
forme bilinéaire symétrique sur $\h_c^{n-1}(V_a^{n-1},\sF(\psi))$, à 
valeurs dans $\bar\dQ_\ell(1-n)$. Relativement à cette forme, $F$ est une 
similitude orthogonale de multiplicateur $q^{n-1}$, et $q^{\frac{1-n}{2}} F$ 
appartient au groupe spécial orthogonal, d'après \eqref{VI:eq:7-15-2}. Une 
des valeurs propres de $F$ est donc $q^{\frac{1-n}{2}}$, et les autres se 
rangent par paires de racines $\alpha$, $\bar\alpha$ de produit $q^{n-1}$. 

Pour $n=3$, L.\ Carlitz \cite{ca69-2} a obtenu un résultat plus précis, 
équivalent à la proposition suivante. 





\begin{proposition_}\label{VI:7-17}
Pour $p=2$, et $\psi(x)=(-1)^{\tr_{\dF_q/\dF_q}(x)}$, on a un isomorphisme 
\[
  \h_c^2(V_a^2,\sF(\psi\sigma)) = \operatorname{Sym}^2 \h_c^1(V_a^1,\sF(\psi\sigma)) \text{.} 
\]
\end{proposition_}

La puissance tensorielle seconde de $\h_c^1(V_a^1,\sF(\psi\sigma))$ est 
$\h_c^1(V_a^1\times V_a^1,\sF(\psi\sigma)\boxtimes \sF(\psi\sigma))$ 
(K\"unneth), la symétrie $x\otimes y\mapsto y\otimes x$ se représentant 
géométriquement comme $-\tau^\ast$, où $\tau$ est l'automorphisme 
$(x,y)\mapsto (y,x)$ de $V_a\times V_a$. Si $X'$ est le quotient de 
$V_a^1\times V_a^1$ par $\{\operatorname{id}{},\tau\}$ et $\pi$ est la 
projection de $V_a^1\times V_a^1$ sur $X'$, le groupe 
$\operatorname{Sym}^2 \h_c^1(V_a^1,\sF(\psi\sigma))$ s'identifie donc à la 
cohomologie de $X'$ à coefficient dans la partie antiinvariante par $\tau$ de 
$\pi_\ast(\sF(\psi\sigma)\boxtimes \sF(\psi\sigma))$. Notons $s$ la fonction 
sur $X'$ telle que $s\pi =\sigma\pr_1+\sigma\pr_2$. On a 
$\sF(\psi\sigma)\boxtimes \sF(\psi\sigma) = \sF(\psi(\sigma\pr_1+\sigma\pr_2)) = \sF(\psi s \pi) = \pi^\ast \sF(\psi s)$ 
(isomorphisme compatible à $\tau$). Sur l'image $\pi(\Delta)$ de la 
diagonale, le faisceau d'antiinvariants cherchés est donc nul, tandis que sur 
le complément $X=X'\setminus \pi(\Delta)$, c'est le produit tensoriel de 
$\sF(\psi s)$ par $\varepsilon(P)$, pour $\varepsilon$ l'unique caractère 
d'ordre $2$ de $\{1,\tau\}$, et $P$ le $\{1,\tau\}$-torseur sur $X$ qu'est le 
revêtement double induit par $V_a^1\times V_a^1$. On a 
\begin{equation*}\tag{7.17.1}\label{VI:eq:7-17-1}
  \operatorname{Sym}^2 \h_c^1(V_a^1,\sF(\psi\sigma)) = \h_c^2(X,\sF(\psi\sigma)\otimes \varepsilon(P)) \text{.}
\end{equation*}

Calculons $X$, $s$ et $\varepsilon(P)$. Sur $V_a^1\times V_a^1$, posons 
$x_1=x_1\circ\pr_1$, $x_2=x_2\circ\pr_2$, $x_1'=x_1\circ \pr_2$, 
$x_2'=x_2\circ \pr_2$. On a $x_1 x_2=x_1'x_2'=a$. Identifions fonctions sur 
$X'$ et fonctions sur $X$ invariante par $\tau$. Soient $s_1=x_1+x_1'$, 
$s_2=x_2+x_2'$, $p_1=x_1 x_1'$, $p_2=x_2 x_2'$. On a $s_1 p_2 = a s_2$, 
$s_2 p_1 = a s_1$ et $X$ est défini dans $X'$ par $s_1\ne 0$, $s_2\ne 0$. Sur 
$X$, on a donc $p_1 = a s_1 s_2^{-1}$, et $(s_1,s_2)$ identifie $x$ à 
$\dG_m\times \dG_m$. Sur $X$, le revêtement $P$ admet pour équation 
$T^2-s_1 T+a s_1 s_2^{-1}$ (prendre pour $T$ la coordonnée $x_1$). Si 
$T=s_1 t$, cela se récrit $t^2-t+a s_1^{-1} s_2^{-1}$. On a donc 
$\varepsilon(P) = \sF(\psi(a s_1^{-1} s_2^{-1}))$. Puisque $s=s_1+s-2$, on a 
$\sF(\psi s)\otimes \varepsilon(P) = \sF(\psi(s_1+s_2+ a s_1^{-1} s_2^{-1}))$ 
et le second membre de \eqref{VI:eq:7-17-1} s'identifie au premier membre de 
\ref{VI:7-17}. 


\paragraph{Remarque}

$\h_c^1(V_a^1,\sF(\psi\sigma))$ est aussi le premier groupe de cohomologie de 
la courbe elliptique complétée de la courbe affine d'équation 
\begin{align*}
  y^2-y &= x_1+x_2 \\
  x_1 x_2 &= a \text{;} 
\end{align*}
son invariant modulaire est $j=a^{-2}$. 





\subsection{Remarque}\label{VI:7-18}

Pour $\chi$ un caractère multiplicatif de $\dF_q^\times$, et $a\ne 0$, la 
méthode de \ref{VI:7-14} s'applique à l'étude de la somme 
\[
  s = \sum_{x_0\dotsm x_n = a} \chi(x_0) \psi(x_0+\cdots + x_n) 
\]
(remplacer le faisceau $\sF(\psi)$ sur $\dG_m$ par $\sF(\chi\cdot \psi)$). On 
trouve encore une cohomologie de dimension $n+1$, et 
$|S|\leqslant (n+1)q^{n/2}$. De telles sommes ont été considérées par 
Salie et Mordell \cite{mo73}.

On peut espérer que nos méthodes permettent d'étudier les sommes 
\[
  S_{k,\chi,a} = \sum_{N(x)=a} \chi(x) \psi \tr(x) 
\]
($k$ algèbre étale sur $\dF_q$, $a\in \dF_q^\times$, $\chi$ caractère de 
$k^\times$). L'analogue de \ref{VI:7-4} (pour $a\ne 0$) devrait être vrai. 





\subsection{}\label{VI:7-19}

Le groupe symétrique $S_n$ agit sur $\dA^n$ en respectant $V_a$ et 
l'application $\sigma$, donc aussi le faisceau $\sf(\psi\sigma)$, image 
réciproque par $\sigma$ du faisceau $\sf(\psi)$ sur $\dG_a$. Il s'agit donc 
sur $\h_c^{n-1}(V_a,\sF(\psi\sigma))$. La formule \eqref{VI:eq:7-2-5} admet 
l'interprétation cohomologique suivante





\begin{proposition_}\label{VI:7-20}
L'action de $S_n$ sur $\h_c^{n-1}(V_a,\sF(\psi\sigma))$ est la multiplication 
par le caractère signe. 
\end{proposition_}

Ceci revient à montrer qu'une transposition $\tau$ agit par multiplication 
par $-1$. Puisque $\tau^2=1$, $\tau$ ne peut avoir pour valeur propres que 
$\pm 1$. Soit $V_a/\tau$ le quotient de $V_a$ pour 
$\{\operatorname{id},\tau\}$. L'application $\sigma$, donc le faisceau 
$\sf(\psi\sigma)$, passent au quotient, et le sous-espace de 
$\h_c^{n-1}(V_a,\sF(\psi))$ fixe par 
$\tau$ est $\h_c^{n-1}(V_a/\tau,\sF(\psi\sigma))$. Il nous faut prouver que 
cette cohomologie est nulle. 

Supposons que $a\ne 0$. Pour $\tau=(1,2)$ le quotient $V_a/\tau$ s'identifie 
à $\dG_a\times \dG_m^{n-2}$, avec pour application de passage au quotient 
$(x_1,x_2,x_3,\dots,x_n)\mapsto (x_1+x_2,x_3,\dots,x_n)$. En effet, $x_1$ et 
$x_2$ sont déterminés à l'ordre près par $x_1+x_2$ et 
$x_1x_2 = a/x_3\dotsm x_n$. En ces coordonnées $(t,x_3,\dots,x_n)$, le 
faisceau s'écrit 
$\sF(\psi(t+x_3+\cdots + x_n)) = \sF(\psi(t))\otimes \sF(\psi(x_3+\cdots + x_n))$. 
La formule de K\"unneth, et $\h_0^\bullet(\dG_a,\sF(\psi)) = 0$, fournissent 
donc la nullité demandée. 

Pour $a=0$, on peut déduire \ref{VI:7-20} de calcul \ref{VI:7-7}. 





\subsection{}\label{VI:7-21}

Les sommes $\kloos_{n,a}$ de ce paragraphe ont été étudiées par 
S.\ Sperber, dans sa thèse, et dans \cite{sp77} par des méthodes 
$p$-adiques inspirées de Dwork et de l'article de Bombieri \cite{bo66}. 
Pour $p\ne 2$, ces méthodes lui donnent la dépendance en $q$ \ref{VI:7-5}, 
l'équation fonctionnelle $\alpha_i = q^{n-1} \beta_i^{-1}$ entre les 
valeurs propres de Frobenius pour les sommes $\kloos_{n,a}$ et 
$\kloos_{n,b}$ si $b=(-1)^n a$ (cf. \ref{VI:7-6}), et la valeur \ref{VI:7-15} 
du produit des racines. Pour $p\geqslant n+3$, il détermine de plus les 
valuations $p$-adiques des racines. Le résultat est très beau: si on range 
les racines $\alpha_i$ dans un ordre convenable, avec 
$0\leqslant i\leqslant n$, les $\alpha_i q^{-i}$ sont des unités. Les 
restrictions sur $p$ ne sont sans doute pas essentielles. 










\section{Autres applications}\label{VI:8}





\subsection{}\label{VI:8-1}

L'identité \ref{VI:1-9-4} peut parfois être utilisée pour calculer 
le nombre de points rationnels d'une variété algébrique sur $\dF_q$. Dans 
la plupart des cas où un résultat exact a été obtenu, la cohomologie 
s'exprime en terme de cycles algébriques; tel est le cas pour les surfaces 
rationnelles, les quadriques, les intersections lisses de deux quadriques de 
dimension impaire, les variétés de drapeaux\ldots. Ce n'est toutefois par 
toujours le cas; dans le très bel article de Lusztig: \cite{lu76}, les cycles 
algébriques n'apparaissent pas. 





\subsection{}\label{VI:8-2}

Pour les groupes réductifs, la cohomologie peut se calculer en se relevant en 
caractéristique $0$: si $B$ est un sous-groupe de Borel de $G$, $G$ est un 
$B$-torseur sur $G/B$; la cohomologie de la variété projective et lisse 
$G/B$ est invariant pour spécialisation, et on utilise la suite spectrale de 
Leray de $G\to G/B$ pour prouver le même résultat pour $G$. Elle s'exprime 
commodément en terme \emph{du} tore maximal $T$ de $G$ et de l'action 
\emph{du} groupe de Weil $W$ sur $T$ (pour la définition \emph{du} tore 
maximal, cf. \cite[VIII \S 2 Rem.2]{bo05} ou \cite[p.105]{dl76}). C'est 
l'algèbre extérieure de sa partie primitive, qui coïncide à un décalage 
près avec le quotient $I(\h^\bullet(B G,\dQ_\ell))$ de la cohomologie de 
$B G$ formé des ``éléments indécomposables,'' et 
$\h^\bullet(B G,\dQ_\ell)=\h^\bullet(B T,\dQ_\ell)^W$. Si $X$ est le groupe des 
caractères de $T$, on a au total 
\[
  \h^\bullet(G,\dQ_\ell) = \bigwedge \left(I\left(\operatorname{Sym}^\bullet(X\otimes \dQ_\ell(-1))^W\right)[-1]\right) \text{.}
\]
Si $G$ est défini sur un corps fini $\dF_q$, $\gal(\dF/\dF_q)$ agit sur $X$ 
et cet isomorphisme est compatible à Galois. Passant de là à la 
cohomologie à supports propres, on trouve les formules classiques pour la 
nombre de points rationnels de $G$. Si $F^\ast:X\to X$ est défini par le 
morphisme de Frobenius $F:T\to T$, et que $\deg_G(F)$ est le degré 
$q^{\dim G}$ de $F:G\to G$, on a 
\begin{equation*}\tag{8.2.1}\label{VI:eq:8-2-1}
  \left|G_0(\dF_q)\right| = \deg_G(F) \cdot \det\left(1-{F^\ast}^{-1},I(\operatorname{Sym}^\bullet(X\otimes \dQ)^W)\right) \text{.}
\end{equation*}

La même formule vaut pour les groupes de Ree et de Suzuki; ces groupes sont 
de la forme $G^F$ ($G$ réductif, $F^2$ un Frobenius), et il suffit de 
vérifier que la formule des traces de Lefschetz est vraie pour $F$ agissant 
sur $G$. Si $B$ est un sous-groupe de Borel stable par $F$, on le vérifie 
directement pour $F$ agissant sur $B$, et pour l'action sur $G/B$, propre et 
lisse, on peut invoquer les théorèmes généraux. 





\subsection{}\label{VI:8-3}

La formule des traces a été utilisée par Deligne-Lusztig \cite{dl76}, 
Kazhdan \cite{ka77} et Springer \cite{sp76} dans l'étude des 
représentations complexes des groupes finis $G_0(\dF_q)$, pour $G_0$ 
réductif sur $\dF_q$. Les travaux de Kazhdan et Springer contiennent des 
exemples admirables de comment la formule des traces permet le ``prolongement 
analytique'' d'une situation déployée à une situation non déployée 
(cf. \ref{VI:1-13}). 






