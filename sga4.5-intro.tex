\chapter*{Introduction}

Ce volume a pour but de faciliter au non-expert l'usage de la 
cohomologie $\ell$-adique. J'espère qu'il lui permettra souvent d'éviter le 
recours aux exposés touffus de SGA 4 et SGA 5. Il contient aussi quelques 
résultats nouveaux. 

Le premier exposé, édigé par J.F. Boutot, survole SGA 4. Il donne les 
principaux résultats -- avec une généralité minimale, souvent 
insuffisante pour les applications -- et une idée de leur démonstration. Pour 
des résultats complets, ou des démonstrations détaillées, SGA 4 reste 
indispensable. 

Le ``\nameref{II}'' contient une démonstration complété de 
la formule des traces pour l'endomorphisme de Frobenius. La démonstration est 
celle donnée par Grothendieck dans SGA 5, élaguée de tout détail inutile. Ce 
Rapport devrait permettre à utilisateur d'oublier SGA 5, qu'on pourra 
considérer comme une série de digression, certaines très intéressantes. Son 
existence permettra de publier prochainement SGA 5 tel quel. Il est complété 
par l'exposé ``\nameref{VI}'' qui explique comment la formule des traces permet 
l'étude de sommes trigonométriques, et donne des exemples. 

Le public visé par les autres exposés est plus limité, et leur style s'en 
ressent. L'exposé ``\nameref{III}'' est une 
généralisation ``modulaire'' du Rappoport, basée sur l'étude SGA 4 XVII 5.5 des 
puissances symétriques. L'exposé ``\nameref{IV}'' 
définit cette classes dans divers contextes, et donne la compatibilité 
entre intersections et cup-produits. Dans ``\nameref{V}'' sont rassemblés 
quelques résultats connus, pour lesquels manquait une référence, et quelques 
compatibilités. L'exposé ``\nameref{VII}'' 
est nouveau. Il donne notamment, en cohomologie sans supports, des théorèmes de 
finitude analogues à ceux connus en cohomologie à supports compacts. 

Pour plus de détails sur les exposés, je renvoie à leur introduction 
respective. 

Je remercie enfin J.L. Verdier de m'avoir permis de reproduire ici ses notes 
``Catégories dérivées (Etat $0$).'' Elles restent je crois très utiles, et 
étaien: devenues introuvables. 

Dans les références internes à ce volume, les exposés sont cités par 
un titre abrégé, indiqué entre [ ] dans la table des matières. 

\vspace{20pt}
Bures-sur-Yvette, le 20 Septembre 1976

\vspace{5pt}
Pierre Deligne