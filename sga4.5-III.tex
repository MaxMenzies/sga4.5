% !TEX root = sga4.5.tex

\chapter{Fonctions \texorpdfstring{$L$}{L} modulo \texorpdfstring{$\ell^n$}{l n} et modulo \texorpdfstring{$p$}{p}}\label{III}

Dans cet exposé, je prouve une formule analogue au théorème du rapport sur 
la formule des traces (ce volume; cité ``Rapport'' par la suite) pour un 
anneau de coefficients $A$ qui soit de torsion premier à $p$, ou un corps de 
caractéristique $p$. Un usage essentiel sera fait de \cite[XVII 5.5]{sga4}. 










\section{Tenseurs symétriques}\label{III:1}





\subsection{}\label{III:1-1}

Soit $A$ un anneau commutatif. Pour une liste de propriétés des foncteurs 
$\Gamma^n:(\text{$A$-modules})\to (\text{$A$-modules})$, je renvoie à 
\cite[XVII 5.5.1 et 2]{sga4}. Rappelons seulement que pour $M$ plat (en fait, seul 
le cas où $M$ est projectif de type fini nous intéresse), on a 
\[
  \Gamma^n M = \left(M^{\otimes n}\right)^{S_n} \qquad \text{(tenseurs symétriques de degré $n$).}
\]





\subsection{}\label{III:1-2}

Soient $X$ un $S$-schéma, $x^n/S$ la puissance fibrée $n$-ième de $X$ et 
$\pi$ la projection de $X^n/S$ sur $(X^n/S)/S_n=\operatorname{Sym}_S^n(X)$ 
(supposé exister). Si $\sG$ est un faisceau de $A$-modules sur $X$, le 
produit tensoriel externe de $n$ copies de $\sG$; $\sG^{\boxtimes n}$, est un 
faisceau $S_n$-équivariant sur $X^n/S$. Nous aurons à faire usage du 
foncteur $\Gamma^n$ \emph{externe} de \cite[XVII 5.5.7 à 9]{sga4}. Rappelons 
seulement que pour $\sG$ un faisceau de $A$-modules plats (en fait, seul le cas 
constructible et plat nous intéresse), on a 
\[
  \Gamma_\text{ext}^n(\sG) = \left(\pi_*\sG^{\boxtimes n}\right)^{S_n} \text{.}
\]





\begin{lemma_}\label{III:1-3}
Pour $S=\spec(k)$, avec $k$ algébriquement clos, $X$ une somme finie de 
copies de $S$, et $G$ un faisceau plat de $A$-modules sur $X$, on a 
\[
  \Gamma\left(\operatorname{Sym}_S^n(X),\Gamma_\text{ext}(\sG)\right) = \Gamma^n \Gamma(X,\sG)
\]
\end{lemma_}

En effet, 
\begin{align*}
  \Gamma\left(\operatorname{Sym}_S^n(X),\Gamma_\text{ext}^n(\sG)\right) 
    &= \Gamma\left(\operatorname{Sym}_S^n(X),\pi_*\sG^{\boxtimes n}\right)^{S_n} \\
    &= \Gamma\left(X^n/S,\sG^{\boxtimes n}\right)^{S_n} 
    = \left(\Gamma(X,\sG)^{\otimes n}\right)^{S_n} 
    = \Gamma^n \Gamma(X,\sG) \text{.}
\end{align*}





\subsection{}\label{III:1-4}

Nous aurons à utiliser les dérivés des foncteurs non additifs $\Gamma^n$ 
et $\Gamma_\text{ext}^n$. 

La théorie de tela foncteurs dérivés est due à Dold et Puppe 
\cite{do61}. Pour des rappels plus complets que ceux donnés ci-dessous, je 
renvoie à \cite[XVII 5.5.3]{sga4}. 

Soit $\cA$ une catégorie abélienne (voire une catégorie additive où 
tout endomorphisme idempotent soit la projection sur un facteur direct). Notons 
$N$ le foncteur de normalisation: $N:($complexes cosimpliciaux d'objets de 
$\cA)\to($complexes différentiables d'objets de $\cA$, avec $K^i$ pour 
$i<0)$. C'est une équivalence de catégories. Ainsi que son inverse, elle 
transforme homotopes en morphismes homotopes. Si $\Gamma$ est un foncteur 
$\cA\to\cB$, et si $K$ est un complexe d'objets de $\cA$, à degrés 
positifs ($K^i=0$ pour $i<0$), on pose $T K=N T N^{-1} K$. 





\subsection{Exemple}\label{III:1-5}

Si $K$ est réduit à $M$ en degré $0$, $T K$ st réduit à $T M$ en 
degré $0$. 





\subsection{}\label{III:1-6}

Si $K$ est un complexe borné, à degrés positifs, de $A$-module projectifs 
de type fini, le complexe de $A$-modules $\Gamma^n K$ a les mêmes 
propriétés. 

Notons $\D_\text{parf}^{\geqslant 0}(A)$ la sous-catégorie de 
$\D_\text{parf}(A)$ (Rapport, \ref{II:4-3}) formée des complexes de 
tor-dimension $\leqslant 0$. On voit que $\Gamma^n$ induit un foncteur 
\[
  \eL\Gamma^n : \D_\text{parf}^{\geqslant 0}(A) \to \D_\text{parf}^{\geqslant 0}(A) \text{.}
\]

Pour la définition analogue, mais plus compliquée, de 
$\eL\Gamma_\text{ext}^n$, je renvoie à \cite[XVII 5.5.14]{sga4}. 





\subsection{}\label{III:1-7}

Soient $K$ un complexe borné de $A$-modules projectifs de type fini, et $u$ 
un endomorphisme de $K$. On note $\det(1-u t,K)$ la série formelle de terme 
constant $1$ 
\[
  \det(1-u t,K) = \prod_i \det(1-u t,K^i)^{(-1)^i} \text{.}
\]

On laisse au lecteur la vérification des propriétés suivantes. 


\subsubsection{}\label{III:1-7-1}

Soit $F$ une filtration finie stable par $u$, telle que les $\gr_F^p(K^q)$ 
soient projectifs de type fini. Alors 
\[
  \det(1-u t,K) = \prod_p \det(1-u t,\gr_F^p(K)) \text{.}
\]


\subsubsection{}\label{III:1-7-2}

Si on translate $K$ de $q$ crans vers la gauche, on a 
\[
  \det(1-u t,K[q]) = \det(1-u t,K)^{(-1)^q} \text{.}
\]





\begin{proposition_}\label{III:1-8}
Soit $u$ un endomorphisme d'un complexe borné, à degrés positifs, de 
$A$-modules projectifs de type fini. On a \
\begin{equation}\label{III:eq:1-8-1}
  \det(1-u t,K)^{-1} = \sum_n \tr(u,\Gamma^n K) t^n \text{.}
\end{equation}
\end{proposition_}

Notons $D(u,K)$ le second membre de \eqref{III:eq:1-8-1}. 





\begin{lemma_}\label{III:1-9}
Soit $u$ un endomorphisme d'une suite exacte courte de complexes bornés à 
degrés $\geqslant 0$ de $A$-modules projectifs de type fini 
\[
  0 \to K' \to K \to K'' \to 0 \text{.}
\]
On a $D(u,K)=D(u,K')\cdot D(u,K'')$.
\end{lemma_}

Si $0\to M'\to M\to M''\to 0$ est une suite exacte courte scindable de 
$A$-modules, $\Gamma^n M$ a une filtration naturelle, de quotients successifs 
les $\Gamma^i M'\otimes \Gamma^j M''$ ($i+j=n$). Dès lors, $\Gamma^n K$ admet 
une filtration naturelle, de quotients successifs les complexes 
$N(N^{-1} \Gamma^i K'\otimes N^{-1}\Gamma^j K'')$ ($i+j=n$), canoniquement 
homotopes aux complexes simples $s(\Gamma^i K'\otimes \Gamma^i K'')$ déduit 
des complexes doubles $\Gamma^i K'\otimes \Gamma^i K''$, et 
\[
  \tr(u,\Gamma^n K) = \sum_{i+j=n} \tr(u,\Gamma^i K')\cdot \tr(u,\Gamma^j K'') \text{,}
\]
d'où le lemme. De \ref{III:1-9}, on tire par récurrence. 





\begin{lemma_}\label{III:1-10}
Soient $F$ une filtration finie stable par $u$ de $K$, telle que les 
$\gr_F^p(K^q)$ soient projectifs de type fini. Alors, 
\[
  D(u,K) = \prod_p D\left(u,\gr_F^p(K)\right) \text{.}
\]
\end{lemma_}





\begin{lemma_}\label{III:1-11}
Soient $u$ un endomorphisme d'un $A$-module projectif de type fini $M$, et 
$M[-q]$ le complexe réduit à $M$ en degré $q$. On a 
\[
  D\left(u,M[-q]\right) = \left(\sum \tr(u,\Gamma^n M) t^n\right)^{(-1)^q} \qquad \text{($q\geqslant 0$.)}
\]
\end{lemma_}





D'après \ref{III:1-5}, c'est vrai pour $q=0$. Précédons par récurrence, il 
reste à montrer que 
\[
  D\left(u,M[-q]\right) \cdot D\left(u,M[-(q+1)]\right) = 1 \text{.}
\]
Cette formule résulte de \ref{III:1-9}, appliqué ou complexe $K$ réduit a 
$M$ en degrés $q$ et $q+1$ (avec $d=\operatorname{id}$), et à sa 
filtration bête:
\[
  (\cdots 0 \to M \cdots) \to (\cdots M \to M) \to (\cdots M \to 0 \to 0 \cdots) \text{.}
\]
En effet, $K$ est homotope à $0$. Dans la catégorie dérivée, on a donc 
$\Gamma^n(K)=\Gamma^n(0)$; $0$ pour $n>0$ et $A$ pour $n=0$, et $D(u,K)=1$. 





\subsection{}\label{III:1-12}

Prouvons \ref{III:1-8}. D'après \ref{III:1-10} appliqué à la filtration 
``bête'' de $K$ par les sous-complexes 
\[
  0 \to \cdots \to 0 \to K^q \to K^{q+1} \to \cdots \text{,}
\]
on a 
\[
  D(u,K) = \prod_q D\left(u,K^q[-q]\right) =_{\text{(\ref{III:1-10})}} \prod_q \left(\sum \tr\left(u,\Gamma^n(K^q)\right) t^n \right)^{(-1)^q} \text{.}
\]
Il reste à prouver que pour $u$ un endomorphisme d'un module projectif de type 
fini $M$, on a 
\begin{equation}\label{III:eq:1-12-1}
  \det\left(1-u t,M\right)^{-1} = \sum \tr(u,\Gamma^n M) t^n \text{.}
\end{equation}
Si $u=0$, les deux membres valent $1$. Ajoutant à $M$ un module $N$ sur lequel 
$u=0$, et appliquent \ref{III:1-9}, on se ramène à supposer que $M$ est libre. 
Prenons-en un base. Il s'agit alors de prouver des identités algébriques 
entres les coordonnées de $u$. Le principe de prolongement des identités 
algébriques nous ramène à supposer que $A$ est un corps algébriquement 
clos. Le module $M$ admet alors une filtration stable par $u$ à quotients 
successifs de rang $1$, et \ref{III:1-10} nous ramène au case où $M$ est de 
rang $1$. La formule \eqref{III:eq:1-12-1} se ramène alors à l'identité 
\[
  \frac{1}{1-a t} = \sum a^n t^n \text{.}
\]





\begin{corollary_}\label{III:1-13}
Soient $u$ et $u'$ des endomorphismes de complexes bornés $K$ et $K'$ de 
$A$-modules projectifs de type fini. Si, dans la catégorie dérivée, $K$, 
muni de $u$, est isomorphe à $K'$, muni de $u'$, alors 
\[
  \det(1-u t,K) = \det(1-u' t,K') \text{.}
\]
\end{corollary_}

Appliquent \ref{III:1-7-2}, on se ramène à supposer que $K$ et $K'$ sont à 
degrés positifs. On observe alors que le second membre de \ref{III:1-8} 
possède la propriété d'invariance voulue. 





\subsection{}\label{III:1-14}

Ce corollaire permet de définir $\det(1-u t,K)$ pour $u$ un endomorphisme de 
$K\in \ob\D_\text{parf}(A)$. Cette construction est ad hoc; en fait, pour $v$ un 
automorphisme de $L\in\ob\D_\text{parf}(\Lambda)$, on peut définir 
$\det(v)\in\Lambda^\times$, et $\det(1-u t,K)$ s'obtient pour $v=1-u t$, 
$\Lambda=A\llbracket t\rrbracket$ et $L=K\otimes_A\Lambda$. 










\section{Le Théorème}\label{III:2}





\subsection{}\label{III:2-1}

Reprenons les notations de (Rapport, \S\ref{II:1}). Pour $A$ un anneau 
noethérien commutatif de torsion, $X_0$ un schéma séparé de type fini sur 
$\dF_q$ et $\sK_0\in\ob\D_\text{ctf}(X_0;A)$, on pose 
\[
  L(X_0,\sK_0) = \prod_{x\in |X_0|} \det\left(1-F_x^* t^{\deg(x)}, \sK\right)^{-1}
\]
(\ref{III:1-7}, \ref{III:1-13} et Rapport, \ref{II:1-5}, \ref{II:1-6}). 





\begin{theorem_}\label{III:2-2}
Dans chacun des deux cas suivants
\begin{enumerate}[\indent a)]
  \item $A$ est de torsion première à $p$;
  \item $A$ est réduit et de caractéristique $p$
\end{enumerate}
on a 
\begin{equation}\label{III:eq:2-2-1}
  L(X_0,\sK_0) = \det\left(1-F^* t^f, \R\Gamma_c(X,\sK)\right)^{-1} \text{.}
\end{equation}
\end{theorem_}





\subsection{Remarque}\label{III:2-3}

Dans le cas a), la méthode de Rapport, \S\ref{II:3} permet de déduire de 
la formule des traces (Rapport \ref{II:4-10}) que les dérivées logarithmiques 
des deux membres de \eqref{III:eq:2-2-1} sont égales. L'anneau $A$ étant de 
torsion, cela ne suffit pas pour prouver \ref{III:2-2}. 





\subsection{Remarque}\label{III:2-4}

Si $A$ est un corps, on a 
\[
  L(X_0,\sK_0) = \prod_i L\left(X_0,\underline\h^i(\sK_0)\right)^{(-1)^i}
\]
et la suite spectrale 
$E_2^{p q} = \h_c^p(X,\underline\h^q(\sK))\Rightarrow \h_c^{p+q}(X,\sK)$ montre 
que 
\begin{align*}
  \det\left(1-F^* t, \R\Gamma_c(X,\sK)\right) 
    &= \prod_n \det\left(1-F^* t^f,\h_c^n(X,\sK)\right)^{(-1)^n} \\
    &= \prod_{p,q} \det\left(1-F^* t,\h_c^p(X,\underline\h^q(\sK))\right)^{(-1)^{p+1}} \text{.}
\end{align*}
Ceci ramène \ref{III:2-2} (pour $A$ un corps) au cas particulier suivant. 


\subsubsection{}\label{III:2-4-1}

Si $\sG_0$ est un faisceau constructible de $A$-espaces vectoriels, on a 
\[
  L(X_0,\sG_0) = \prod_n \det\left(1-F^* t^f,\h_c^n(X,\sG)\right)^{(-1)^{n+1}} \text{.}
\]





\subsection{Remarque}\label{III:2-5}

Il est facile de réduire le cas b) à celui où $A$ est un corps (voire 
même un corps fini) de caractéristique $p$, cf. \ref{III:4-2}. 





\subsection{Remarque}\label{III:2-6}

Le cas particulier de b) où $A=\dZ/p$, et où $X$ est réduit su faisceau 
constant $\dZ/p$ est degré $0$;
\[
  Z(X_0,t) = \prod_{x\in |X_0|} \left(1-t^{\deg(x)}\right)^{-1}
      \equiv \prod_n \det\left(1-F^* t^f,\h_c^n(X,\dZ/p)\right)^{(-1)^{n+1}} \pmod p
\]
avait été prouvé par N. Katz \cite[XXII 3.1]{sga7}. Sa méthode, toute 
différente de celle suivie ici, part de la théorie $p$-adique de Dwork de 
$Z(X_0,t)$. 





\subsection{Remarque}\label{III:2-7}


Contrairement à ce que j'avais imprudemment affirmé à des amis, on ne 
peut pas dans b) omettre l'hypothèse ``$A$ réduit.'' Pour un contre 
exemple, cf. \ref{III:4-5}. 





\subsection{Plan de la preuve de \ref{III:2-2}}\label{III:2-8}

Une translation sur les degrés \ref{III:1-7-2} nous ramène à supposer 
que $\sK_0$ est un complexe borné, à degré positifs, de faisceaux de 
$A$-modules constructibles et plats. On a alors 
$\R\Gamma_c(X,\sK)\in \ob\D_\text{parf}^{\geqslant 0}(A)$ (Rapport, 
preuve \ref{II:4-9}, b). Nous allons utiliser \ref{III:1-8} pour développer 
les deux membres de \eqref{III:eq:2-2-1} en série de puissances de $T=t^f$. 
L'égalité des coefficients de $T^n$ de déduira de 
\cite[XVII 5.5.21]{sga4} et d'une formule des traces sur 
$\operatorname{Sym}^n(X)$; dans le cas a), le formule (Rapport, 
\ref{II:4-10}), et dans le cas b) une formule qui sera prouvée dans le 
chapitre suivant. Il y a lieu de se ramener au préalable au cas où les 
puissances symétriques de $X_0$ existent (par exemple au cas où $X$ est 
affine) à l'aide des propriétés de multiplicativité en $X_0$ des deux 
membres de \eqref{III:eq:2-2-1}: pour $U_0$ un ouvert de $X_0$, de fermé 
complémentaire $Y_0$, on a $L(X_0,\sK_0)=L(U_0,\sK_0)\cdot L(Y_0,\sK_0)$ et 
la formule analogue pour le membre de droite résulte de \ref{III:1-7-1} 
par la méthode de (Rapport \S\ref{II:6}A). 





\subsection{Le cas où $X_0$ est fini}\label{III:2-9}

La multiplicativité en $X_0$ des deux membres de \eqref{III:eq:2-2-1} 
ramène ce cas à celui où $X_0=\spec(\dF_{q^k})$. Revenant aux 
définitions, on voit que \eqref{III:eq:2-2-1} équivaut alors à 
Rapport \ref{II:3-4}. 





\subsection{Le premier membre}\label{III:2-10}

D'après \ref{III:1-8}, on a 
\[
  L(X_0,\sK_0) = \prod_{x\in |X_0|} \sum_n \tr\left(F_x^*,\Gamma^n\sK\right) t^{m\deg(x)} \text{.}
\]
Le coefficient de $T^n$ est donc la somme, étendue à celles des 
combinaisons linéaires formelles $\sum_{x\in |X_0|} n_x\cdot x$ 
($n_x\geqslant 0$, les $n_x$ presque tous nuls) d'éléments de $|X_0|$ 
telles que $\sum n_x [k(x):\dF_q] = n$, des produits 
\begin{equation}\label{III:eq:2-10-1}
  \prod_x \tr\left(F_x^*,\Gamma^{n_x} \sK\right) \text{.}
\end{equation}

Pour $x\in |X_0|$, notons $(x)$ le $0$-cycle des $[k(x):\dF_q]$ points de $X$ 
au-dessus de $x$ (une orbite de $F$). L'application 
$\sum n_x\mapsto \sum n_x\cdot (x)$ identifie les combinaisons linéaires 
formelles comme ci-dessus, avec $\sum n_x [k(x):\dF_q]=n$, aux $0$-cycles 
effectifs de degré $n$ de $X$ fixes sous $F$, soit encore aux points fixes de 
$F$ dans $\operatorname{Sym}^n(X)$. Le coefficient de $T^n$ dans $L(X_0,\sK_0)$ 
apparait ainsi comme une somme de produits \eqref{III:eq:2-10-1}, indenée 
par $\operatorname{Sym}^n(X)^F$. Plus précisément, 





\begin{lemma_}\label{III:2-11}
 $L(X_0,\sK_0) = \sum_n \sum_{x\in\operatorname{Sym}^n(X)^F} \tr\left(F_x^*,\eL\Gamma_\textnormal{ext}^n \sK\right) T^n$.
\end{lemma_}

Pour $Y_0$ un sous-schéma fini de plus en plus grand de $X_0$, on a 
\[
  L(X_0,\sK_0) = \lim L(Y_0,\sK_0) \text{,}
\]
i.e., pour tout $m$, il existe un sous-schéma fini $Z_0\subset X_0$ tel que 
si $Y_0\supset Z_0$, $L(X_0,\sK_0)$ et $L(Y_0,\sK_0|Y_0)$ soient congrus 
$\mod T^m$. 
Il suffit que $Z_0$ contienne les points fermés de degré $\leqslant m$. 
Le même fait valant pour le membre de droite, il suffit de prouver 
\ref{III:2-11} pour $X_0$ fini. On a alors 
\begin{align*}
  L(X_0,\sK_0) 
    &=_\text{\ref{III:2-9}} \det\left(1-F^* T,\Gamma(X,\sK)\right)^{-1} \\
    &=_\text{\ref{III:1-8}} \sum_n \tr\left(F^*,\Gamma^n \Gamma(X,\sK)\right) T^n \\
    &=_\text{\ref{III:1-3}} \sum_n \tr\left(F^*,\Gamma(\operatorname{Sym}^n(X),\Gamma_\text{ext}^n(\sK)\right) T^n \\
    &= \sum_n \sum_{x\in \operatorname{Sym}^n(X)^F} \tr\left(F_x^*,\Gamma_\text{ext}^n(\sK)\right) T^n
\end{align*}
(par la formule des traces dans le cas trivial d'un ensemble fini). 





\subsection{Le second membre}\label{III:2-12}

D'après \ref{III:1-8} et \cite[XVII 5.5.12]{sga4}, on a 
\begin{align*}
  \det\left(1-F^* T,\R\Gamma_c(X,\sK)\right)^{-1} 
    &= \sum_n \tr\left(F^*,\eL\Gamma^n\R\Gamma_c(X,\sK)\right) T^n \\
    &= \sum_n \tr\left(F^*,\R\Gamma_c(\operatorname{Sym}^n(X),\Gamma_\text{ext}^n\sK)\right) T^n \text{.}
\end{align*}





\subsection{}\label{III:2-13}

Le théorème équivaut donc aux formules des traces 
\[
  \tr\left(F^*,\R\Gamma_c(\operatorname{Sym}^n(X,\Gamma_\text{ext}^n \sK)\right) = \sum_{x\in\operatorname{Sym}^n(X)^F} \tr\left(F_x^*,\Gamma_\text{ext}^n\sK\right) \text{.}
\]
Dans le cas a), elles résultent de Rapport \ref{II:4-10}. La formule 
analogue dans le cas b) sera prouvée au \S\ref{III:4}. 










\section{Théorie d'Artin-Scheier}\label{III:3}





\subsection{}\label{III:3-1}

Dans ce qui suit, un faisceau cohérent sur un schéma $S$ sera toujours 
regardé comme un faisceau sur $S_\text{ét}$. Soit $S$ un schéma de 
caractéristique $p>0$. Si $G$ un faisceau localement constant de 
$\dZ/p$-vectoriels de rang fini, $\sG=G\otimes_{\dZ/p}\sO$ est un faisceau 
cohérent localement libre sur $S$. On notera $\Phi$ tant l'endomorphisme 
$f\mapsto f^p$ de $\sO$ que son produit tensoriel avec l'identité de 
$G$: $\phi:\sG\to\sG$. La théorie d'Artin-Scheier \cite[IX 3.5]{sga4} 
affirme que la suite 
\[\xymatrix{
  0 \ar[r] 
    & \dZ/p \ar[r] 
    & \sO \ar[r]^-{\Phi-1} 
    & \sO \ar[r] 
    & 0
}\]
est exacte. Par tensorisation avec $G$, on trouve un suite exacte 
\[\xymatrix{
  0 \ar[r] 
    & G \ar[r] 
    & \sG \ar[r]^-{\Phi-1} 
    & \sG \ar[r] 
    & 0
}\]





\subsection{}\label{III:3-2}

Supposons que $S$ soit un ouvert dans un schéma noethérien $\bar S$. 
Soient $j:S\hookrightarrow \bar S$ le morphisme d'inclusion, et $I$ un 
faisceau d'idéaux qui définit le fermé complémentaire 
$\bar S\setminus S$. Notons encore $\Phi$ le prolongement par fonctorialité 
de $\Phi$ à $j_*\sG$. 





\begin{lemma_}\label{III:3-3}
Il existe des prolongements cohérents $\bar\sG\subset j_*\sG$ de $\sG$ tels 
que 
\begin{equation}\label{III:eq:3-3-1}
  \Phi(\bar\sG)\subset I\cdot \bar\sG\text{.}
\end{equation}
Si $\bar\sG$ est un tel prolongement, la suite 
\begin{equation}\label{III:eq:3-3-2}
\xymatrix{
  0 \ar[r] 
    & j_! G \ar[r] 
    & \bar\sG \ar[r]^-{\Phi-1} 
    & \bar\sG \ar[r] 
    & 0
}
\end{equation}
ext exacte.
\end{lemma_}
\begin{enumerate}[a)]
  \item \emph{Existence}: Soit $\sG'\subset j_*\sG$ un prolongement cohérent 
    de $\sG$ à $\bar S$. Pour tout $n$, on a 
    $\Phi(I^n \sG')\subset I^{p^n}\Phi(\sG')$. Soit $n$ assez grand pour 
    que $I^{n(p-1)-1}\Phi(\sG')\subset \sG'$, et posons $\bar\sG=I^n\sG'$. 
    On a $\Phi(\bar\sG)\subset I^{p n}\Phi(\sG') 
    =I\cdot I^n\cdot I^{n(p-1)-1}\Phi(\sG')\subset I\cdot I^n\sG' 
    = I\bar\sG$. 
  \item \emph{Noyau de $\Phi-1$}: Sur $S$, l'assertion résulte de 
    \ref{III:3-1}; il reste à vérifier qu'une section de $x$ de 
    $\ker(\Phi-1)$, sur $U$ étale sur $\bar S$, s'annule dans un voisinage de 
    l'image réciproque de $\bar S\setminus S$. Puisque $x=\Phi(x)$, on a en 
    effet par \eqref{III:eq:3-3-1} $x\in I\bar\sG$; de plus, 
    $x\in I^n\bar\sG\Rightarrow x=\Phi(x)\in\Phi(I^n\bar\sG)\subset I^{n p}\bar\sG$, et $x$ est dans tous les $I^n\bar\sG$, donc nul au voisinage de 
    $\bar S\setminus S$.
  \item \emph{Surjectivité de $\Phi-1$}: On peut supposer $\bar S$ affine, en 
    écrire $\sG$ comme quotient de $\sO^n$; $\Phi$ se relève alors en une 
    application $p$-linéaire $\widetilde\Phi:\sO^n\to\sO^n$, et il suffit de 
    vérifier la surjectivité de $\widetilde\Phi-1$:
    \[\xymatrix{
      \sO^n \ar[r]^-{\widetilde \Phi-1} \ar[d] 
        & \sO^n \ar[d] \\
      \sG \ar[r]^-{\Phi-1} 
        & \sG 
    }\]
    Le faisceau $\sO^n$ est le faisceau des sections locales du schéma en 
    groupes $\dG_a^n$, et $\widetilde\Phi-1$ est induit par un homomorphisme 
    $\dG_a^n$ dans $\dG_a^n$. Cet homomorphisme est étale. Pour vérifier 
    qu'il est surjectif, il suffit de le voir après tout changement de base 
    $\spec(k)\to\bar S$ ($k$ algébriquement clos) et sur $k$ algébriquement 
    clos, tout homomorphisme étale entre groupes connexes de même dimension 
    est surjectif. Enfin, un morphisme étale et surjectif de 
    $\bar S$-schémas induit un épimorphisme entre les faisceaux 
    correspondant sur $\et{\bar S}$.
\end{enumerate}





\subsection{Frobenius et Frobenius}\label{III:3-4}

Le morphisme de Frobenius absolu $F_\text{abs}:S\to S$ est l'identité sur 
l'espace topologique sous-jacent à $S$, et $f\mapsto f^p$ sur le faisceau 
structural. Pour $U/S$ étale, on a un isomorphisme canonique 
$F_\text{abs}^* U=U$: le diagramme 
\[\xymatrix{
  U \ar[r]^-{F_\text{abs}} \ar[d] 
    & S \ar[d] \\
  S \ar[r]^-{F_\text{abs}} 
    & S
}\]
est cartésien. L'endomorphisme de site annelé 
$(\et{\bar S},\sO)\to(\et{\bar S},\sO)$ défini par $F_\text{abs}$ peut donc 
être décrit comme suit:
\begin{center}
  $F_\text{abs}^*$ est l'identité sur $\et S$; l'homomorphisme 
  $F_\text{abs}^*\sO=\sO\to\sO$ est $\Phi$.
\end{center}
Avec cette description, pour tout faisceau étale $\sG$ sur $S$, la 
correspondance de Frobenius $F_\text{abs}^*\sG\to\sG$ est l'identité. 

Se donner un morphisme $p$-linéaire de faisceaux quasi-cohérents 
$u:\sM\to\sN$ sur $S$ revient à se donner un morphisme linéaire 
$u':F_\text{abs}^*\sM\to\sN$. En particulier, avec les notations de 
\ref{III:3-5}, l'homomorphisme $\Phi:\sG\to\sG$ correspond à 
$\Phi':F_\text{abs}^*\sG\to\sG$. 





\begin{lemma_}\label{III:3-5}
Avec les notations de \ref{III:3-4}, le diagramme 
\[\xymatrix{
  F_\textnormal{abs}^* G \ar[r] \ar[d] 
    & G \ar[d] \\
  F_\textnormal{abs}^*\sG \ar[r]^-{\Phi'} 
    & \sG 
}\]
est commutatif.
\end{lemma_}

Il suffit de le vérifier localement. On peut donc supposer $G$ constant, 
auquel cas le lemme est évident. 

La même théorie vaut pour les puissances de $F_\text{abs}$.





\subsection{}\label{III:3-6}

Supposons maintenant que $\bar S$ soit un schéma sur $\dF_q$ ($q=p^f$) et 
conformément aux conventions de Rapport \S\ref{II:1}, écrivons 
$S_0,\bar S_0,G_0,\sG_0,\bar\sG_0$ au lieu de $S,\dots$. Le lemme 
\ref{III:3-5}, pour $F_\text{abs}^F:S_0\to S_0$, donne un diagramme commutatif 
\[\xymatrix{
  F^* G_0 \ar[r] \ar[d] 
    & G_0 \ar[d] \\
  F^*\sG_0 \ar[r]^-{(\Phi^f)'} \ar[r] 
    & \sG_0 
}\qquad\text{, puis}\qquad 
\xymatrix{
  F^*(j_! G_0) \ar[r] \ar[d] 
    & j_!G_0 \ar[d] \\
  F^*\bar\sG_0 \ar[r]^-{(\Phi^f)'}
    & \sG_0
}\text{,}
\]
puis, par extension des scalaires de $\dF_q$ à $\dF$ et passage à la 
cohomologie 
\[\xymatrix{
  F^*(j_! G) \ar[r]^-{F_*} \ar[d] 
    & j_!G \ar[d] \\
  F^*\bar\sG \ar[r]
    & \bar\sG
}\qquad\text{et}\qquad 
\xymatrix{
  \h^\bullet(\bar S,j_! G) \ar[r]^-{F^*} \ar[d] 
    & \h^\bullet(\bar S,j_!G) \ar[d] \\
  \h^\bullet(\bar S,\bar\sG) \ar[r]
    & \h^\bullet(\bar S,\bar\sG)
}\text{,}
\]
où la seconde ligne est obtenue par extension des scalaires de $\dF_q$ à 
$\dF$ de l'endomorphisme $\dF_q$-linéaire $\Phi^f$ de 
$\h^\bullet(\bar S_0,\bar G_0)$. 

La théorie d'Artin-Scheier fournit une suite exacte longue 
\[\xymatrix{
  \cdots \ar[r]
    & \h^i(\bar S,j_! G) \ar[r]
    & \h^i(\bar S,\bar\sG) \ar[r]^-{\Phi-1}
    & \h^i(\bar S,\bar\sG) \ar[r]
    & \cdots
}\]
où $\Phi$ est l'extension $p$-linéaire à 
$\h^\bullet(\bar S,\bar \sG)= \h^\bullet(\bar S_0,\bar\sG_0)\otimes_{\dF_q}\dF$ 
de l'endomorphisme $p$-linéaire $\Phi$ de $\h^\bullet(\bar S_0,\bar\sG_0)$. 





\subsection{}\label{III:3-7}

Si $\bar S_0$ est propre sur $\dF_q$, les $\h^i(\bar S,\bar\sG)$ sont de 
dimension finie et $\Phi-1$ est surjectif (cf. \ref{III:3-6}b). On a donc 
\[
  \h_c^i(S,G) = \ker\left(\Phi-1 : \h^i(\bar S,\bar G) \to \h^i(\bar S,\bar\sG)\right) \text{,}
\]
et $F^*$ est induit par l'extension $\dF$-linéaire à 
$\h^i(\bar S,\bar\sG) = \h^i(\bar S_0,\bar\sG_0)\otimes_{\dF_q}\dF$ de 
l'endomorphisme $\dF_q$-linéaire $\Phi^f$ de $\h^i(\bar S_0m\bar\sG_0)$. 

Appliquant le lemme \ref{III:3-8} ci-dessous, on en déduit l'identité 
\begin{equation}\label{III:eq:3-7-1}
  \tr\left(F^*,\h_c^i(S,G)\right) = \tr\left(\Phi^f,\h^i(\bar S_0,\bar\sG_0)\right) \text{.}
\end{equation}





\begin{lemma_}\label{III:3-8}
Soit $\Phi$ un endomorphisme $p$-linéaire d'un espace vectoriel $V$ sur 
$\dF_q$: $\phi(\lambda x) = \lambda^p\phi(x)$. On pose $F=\Phi^f$ ($F$ est 
linéaire) et on note encore $\Phi$ et $F$ les extensions respectivement 
$p$-linéaire de $\Phi$ et $F$ à $V\otimes_{\dF_q}\dF$. Alors, $F$ stabilise 
$\ker(\Phi-1)\subset V\otimes_{\dF_q}\dF$ et 
\[
  \tr\left(F,\ker(\Phi-1)\right) = \tr(F,V) = \tr\left(F,V\otimes_{\dF_q}\dF\right) \text{.}
\]
\end{lemma_}

Sur $V\otimes_{\dF_q}\dF$, on a $F\Phi = \Phi F$: l'endomorphisme $p$-linéaire 
$F\Phi-\Phi F$ s'annule sur $V$, donc partout. Écrivons $V=V'+V''$, avec $F$ 
inversible sur $V'$ et nilpotent sur $V''$. La théorie des endomorphismes 
$p$-linéaires assure que 
\[
  \ker(\Phi-1)\otimes_{\dF_q}\dF \iso V'\otimes_{\dF_q}\dF \text{.}
\]
Puisque $\tr(F,V'') = 0$, on a donc 
\begin{align*}
  \tr\left(F,\ker(\Phi-1)\right) 
    &= \tr\left(F,\ker(\Phi-1)\otimes_{\dF_q}\dF\right) = \tr\left(F,V'\otimes_{\dF_q}\dF\right) \\
    &= \tr\left(F,V\otimes_{\dF_q}\dF\right) = \tr(F,V) \text{.}
\end{align*}










\section{Formule des traces modulo \texorpdfstring{$p$}{p}}\label{III:4}





\begin{theorem_}\label{III:4-1}
Soient $X_0$ un schéma séparé de type fini sur $\dF_q$, $A$ un anneau 
noethérien commutatif réduit et de caractéristique $p$, et 
$\sK_0\in\ob\D_\textnormal{ctf}(X_0,A)$. On a 
\begin{equation}\label{III:eq:4-1-1}
  \sum_{x\in X^F} \tr\left(F_x^*,\sK_x\right) = \tr\left(F^*,\R\Gamma_c(X,\sK)\right) \text{.}
\end{equation}
\end{theorem_}





\subsection{Réduction au cas où \texorpdfstring{$A$}{A} est un corps fini}\label{III:4-2}

%NOTE: in this paragraph and others, (3.1.1) seems to be referring to (4.1.1)
Un argument standard de passage à a limite nous ramène au cas où $A$ est 
de type fini sur $\dZ/p$. Pour chaque idéal maximal $\fm$ de $A$, l'image de 
chaque membre de \eqref{III:eq:4-1-1} dans $A/\fm$ est donée par la même 
formule, avec $\sK_0$ remplacé par 
$\sK_0\lotimes_A A/\fm\in\ob\D_\text{ctf}(X_0,A/\fm)$ (pour le second membre, 
cf. Rapport \ref{II:4-12}). Les $A/\fm$ étant finis, et l'intersection des 
idéaux maximaux de $A$ étant réduit à $0$, on obtient la réduction 
annoncée.





\subsection{Réduction au cas où \texorpdfstring{$A=\dZ/p$}{A=Z/p}}\label{III:4-3}

Supposons que $A$ soit un corps fini, et notons $T_A'(\sK_0)$ et $T_A''(\sK_0)$ 
les deux membres de \eqref{III:eq:4-1-1}. Si, par restriction des scalaires, on 
regarde $\sK_0$ comme un complexe de faisceaux de $\dZ/p$-modules, on a 
\[
  T_{\dZ/p}'(\sK_0) = \tr_{A/(\dZ/p)} T_A'(\sK_0)
\]
et de même pour $T''$. Si \eqref{III:eq:4-1-1} vaut pour $A=\dZ/p$, on a donc 
\begin{equation}\label{III:eq:4-3-1}
  \tr_{A/(\dZ/p)} T_A'(\sK_0) = \tr_{A/(\dZ/p)} T_A''(\sK_0) \text{.}
\end{equation}
J'affirme que, pour tout $\lambda\in A$, on a même 
\begin{equation}\label{III:eq:4-3-2}
  \tr_{A/(\dZ/p)}\left(\lambda\cdot T_A'(\sK_0)\right) = \tr_{A/(\dZ/p)}\left(\lambda\cdot T_A''(\sK_0)\right) \text{,}
\end{equation}
et donc $T_A'(\sK_0) = T_A''(\sK_0)$. C'est clair pour $\lambda=0$. Pour 
$\lambda$ inversible, notons $A(\lambda)$ l'image réciproque sur $X$ du 
faisceau de $A$-modules libre de rang un sur $\spec(\dF_q)$ pour lequel 
$F^*=\lambda$. La formule \eqref{III:eq:4-3-2} n'est autre que 
\eqref{III:eq:4-3-1} pour $A(\lambda)\otimes_A \sK_0$. 





% originally theorem 4.1bis (i.e. theorem 4.1 again)
\begin{theorem_}\label{III:4-1bis}
Soient $\bar S_0$ une courbe projective et lisse, 
$j:S_0\hookrightarrow \bar S_0$ un ouvert dense de $\bar S_0$, et $G_0$ un 
faisceau localement constant de $\dZ/p$-espaces vectoriels de rang fini sur 
$S_0$. On a 
\begin{equation}\label{III:eq:4-4-1}
  \sum_{x\in S^F} \tr\left(F_x^*,G_x\right) = \sum_i (-1)^i \tr\left(F^*,\h_c^i(S,G)\right) \text{.}
\end{equation}
\end{theorem_}
\begin{proof}
Appliquons \ref{III:3-3} à $(S_0,\bar S_0,G_0)$. Le prolongement cohérent 
$\bar\sG_0$ de $\sG_0$ dont \ref{III:3-3} garantit l'existence est localement 
libre car il est sans torsion (on  $\bar\sG_0\subset j_*\sG_0$) et que 
$\bar S_0$  est régulier de dimension un. L'endomorphisme $q$-linéaire 
$\Phi^f$ de $\bar\sG_0$ peut s'interpréter comme un morphisme 
\[
  u : F^*\bar\sG_0 \to \bar\sG_0 \text{.}
\]
Il se factorise par $I\bar G$, pour $I$ définissant le fermé 
$\bar S\setminus S$. Soit $v:j_!G_0 \to \bar\sG_0$ l'unique prolongement de 
l'application naturelle $G_0 \to \sG_0 = G_0\otimes_{\dZ/p}\sO$. Le diagramme 
\begin{equation}\label{III:eq:4-4-2}
\xymatrix{
  F^* j_! G_0 \ar[r]^-{F^*} \ar[d]^-{F^* v} 
    & j_! G_0 \ar[d]^-v \\
  F^*\bar \sG_0 \ar[r]^-u 
    & \bar\sG_0
}
\end{equation}
Enfin, notons $u F^*$ l'application composée 
\[\xymatrix{
  \h^\bullet(\bar S,\bar\sG) \ar[r]^-{F^*} 
    & \h^\bullet(\bar S,F^*\bar\sG) \ar[r]^-u 
    & \h^\bullet(\bar S,\bar\sG) \text{,}
}\]
on par \eqref{III:eq:3-7-1}
\begin{equation}\label{III:eq:4-4-3}
  \tr\left(F^*,\h_c^i(S,G)\right) = \tr\left(u F^*,\h^i(\bar S,\bar\sG)\right) \text{.}
\end{equation}

%NOTE: trace down reference for Woodshole seminar
Pour chaque point fermé $i_x:x\hookrightarrow \bar  S$ de $\bar S$, posons 
$\bar\sG_x = i_x^*\bar\sG$. La formule des traces de Lefschetz en cohomologie 
cohérente (``Woodshole fixed point formula''), appliqué à $F$ et $u$, dit 
que 
\begin{equation}\label{III:eq:4-4-4}
  \sum_i (-1)^i \tr\left(u^* F^*, \h^i(\bar S,\bar\sG)\right) = \sum_{x\in \bar S^F} \frac{\tr(u_x,\sG_x)}{\det(1-d F_x)} \text{.}
\end{equation}
Au second membre
\begin{enumerate}[\indent a)]
  \item puisque $d F=0$, les dénominateurs valent 1;
  \item pour $x\in \bar S^F\setminus S^F$, l'endomorphisme $u_x$ de $\bar\sG_x$ 
    est nul, donc aussi $\tr(u_x,\bar\sG_x)$;
  \item pour $x\in S^F$, \eqref{III:eq:4-4-4} fournit une contribution 
    \[
      \tr(u_x,\bar\sG_x) = \tr(F_x^*,G_x) \text{.}
    \]
    L'identité \eqref{III:eq:4-4-1} résulte donc de \eqref{III:eq:4-4-3} 
    et \eqref{III:eq:4-4-4}. 
\end{enumerate}
\end{proof}





\subsection{Un contre-exemple}\label{III:4-5}

On montre par un exemple qu'on ne peut pas, dans \ref{III:4-1}, omettre 
l'hypothèse que $A$ soit réduit.

Soit $Y_0$ la courbe elliptique sur $\dF_2$ pour laquelle les valeurs propres 
de Frobenius sont $\pm \sqrt{-2}$. Elle a $3$ points rationnels, et est 
supersingulière. L'involution $\sigma:x\mapsto -x$ n'a donc qu'un seul point 
fixe, le point $0$. Le quotient de $Y_0$ par $\sigma$ est un droite projective; 
celui de $Y_0\setminus 0$ par $\sigma$ est donc une droite affine $\dF_0^1$, et 
$Y_0\setminus 0$ est un revêtement double non ramifié de $\dF_0^1$:
\[
  \pi:Y_0\setminus 0 \to \dA_0^1 \qquad \text{, $\pi=\pi\sigma$.}
\]

Le faisceau $\pi_*\dZ/2$ est un module libre de rang un sur l'algèbre $A$ du 
groupe à deux éléments $\{1,\sigma\}$ sur $\dZ/2$. Cette algèbre est 
isomorphe à celle des nombres sur $\dZ/2$. 

Le faisceau de $A$-modules $\pi_*\dZ/2$ est notre exemple:
\begin{enumerate}[\indent a)]
  \item On a 
    $\h_c^\bullet(\dA^1,\pi_*\dZ/2) = \h_c^\bullet(Y\setminus 0,\dZ/2)$. 
    Puisque $Y_0$ est supersingulière, on a $\h^i(Y,\dZ/2)=0$ pour $i>0$, et 
    la suite exacte longue 
    \[
      \cdots \to \h_c^i(Y\setminus 0,\dZ/2) \to \h^i(Y,\dZ/2) \to \h^i(\{0\},\dZ/2) \to \cdots
    \]
    montre que $\h_c^\bullet(Y\setminus 0,\dZ/2)=0$. On a donc 
    $\h_c^\bullet(\dA^1,\pi_*\dZ/2)=0$ et 
    \[
      \tr_A\left(F^*,\R\Gamma_c(\dA^1,\pi_*\dZ/2)\right) = 0 \text{.}
    \]
  \item Puisque $Y_0\setminus 0$ a deux points rationnels, en l'un des points 
    rationnels de $\dA_1^0$, la substitution de Frobenius est l'identité, et 
    en l'autre, c'est $\sigma$. On a donc 
    \[
      \sum_{x\in {\dA^1}^F} \tr\left(F^*,\pi_*\dZ/2\right) = 1+\sigma \text{,}
    \]
    et $1+\sigma\ne 0$.
\end{enumerate}

L'équation du revêtement $\pi$ est $y^2-y=x^3$.